\chapter{On the description of the battle between \=Adi and Baka after the discourse on \'Sunah\'sepha}

1-6. Indra said :-- ``O Prince! The King Hari\'schandra promised before to Varu\d{n}a that he would celebrate for his propitiation the great Naramedha sacrifice when he would offer his own son as a victim to be immolated. O Prince! You are very intelligent; can't you grasp this idea that your father has become merciless due to his suffering in this illness and no sooner you go there than he will make you the victim and tie you

to the sacrificial post when you will be slaughtered.'' The indomitable Indra thus prohibiting the son, he began to stay there deluded by the M\=ay\=a of the great Mah\=a M\=ay\=a. O King! Thus whenever the prince heard of his father's severe illness, he wanted to go to his father, Indra repeatedly used to go to him and prevent him from doing so. On the other hand, the King Hari\'schandra became very much afflicted, and, seeing his family Guru all-knowing well-wisher Va\'sistha close by, asked him, ``O Bhagav\=an! what am I to do now? I am now very impatient with the agonies of this disease and am very weak; besides I am very much afraid of it. Kindly give me a good advice and save me.''

7-9. Va\'sistha said :-- ``O King! There is a good remedy for the cure of your disease. It is stated in the \'S\=astras that the sons are of thirteen kinds; Aurasa, K\d{s}attraja, Datrima, Kr\={\i}trima; etc. Therefore pay the reasonable price and buy one good Br\=ahmi\d{n} boy and perform your sacrifice with that boy. O King! Thus Varu\d{n}a will be pleased and you will be cured of your disease.''

10-24. Vy\=asa said :-- O King! Hearing thus the words of Va\'sistha, the King Hari\'schandra addressed to his minister :-- ``O Minister-in-Chief! You are very sharp and intelligent, therefore you better try your best and seek in my kingdom a Br\=ahmi\d{n} boy. In case a poor Br\=ahmi\d{n} be willing, out of his love for money, to give over his son, then give him any amount he wants and bring his son. O Minister! By all means, bring a Br\=ahmi\d{n} boy for this sacrifice; in other words, do not be miserly or act lazily to perform my business. You should pray to any Br\=ahmin thus :-- Take this money and give your son, who will be sacrificed in a sacrificial ceremony as a victim.'' Thus ordered, the minister sought for a Br\=ahmi\d{n} boy in towns after towns, villages after villages, and houses after houses. Till, at last, he came to know that in his kingdom there was a poor distressed Br\=ahmi\d{n} named Aj\={\i}garta, who had three sons. Then the minister gave to the Br\=ahmi\d{n} that he wanted and purchased his second son named \'Sunah\'sepha and brought him before the King. And handed him over to the King, saying that this Br\=ahmi\d{n} boy is fit for the sacrificial victim. The King then gladly brought the best Br\=ahmi\d{n}s versed in the Vedas for the performance of the sacrifice, and collected all the articles requisite for the purpose. When the sacrifice was commenced, the great Muni Vi\'sv\=amitra, seeing \'Sunah\'sepha tied, prohibited the King and said :-- ``O King! Do not be so bold as to sacrifice this boy; let this boy be free. O long-lived One! I am asking this thing from you today and if you obey it, certainly it will do good to you. O King! This boy \'Sunah\'sepha is crying; his cries are paining my heart and I am feeling pity for him. Hear my word and free this

boy out of mercy. See! The purehearted K\d{s}attriyas, in ancient days, used to sacrifice their own bodies and thus preserve other\'s bodies, so that they might attain the Heavens. And now you are killing this Br\=ahmi\d{n} boy forcibly so that you may preserve your own body; judge how vicious is this your act! Be merciful to this boy. O King! Everyone likes his own body to the same extent; you are feeling this yourself; therefore if you take my word, then quit this boy.''

25-36. Vy\=asa said :-- O King! The King Hari\'schandra was ailing very much; hence be did not pay any heed to the Muni's words and did not quit the boy. Thereupon the very fiery spirited Vi\'sv\=amitra became very angry with the King. Then Vi\'sv\=amitra, the son of Ku\'sika, the foremost of the knowers of the Vedas, shewed mercy on \'Sunah\'sepha and gave him the ``Varu\d{n}a Mantram.'' \'Sunah\'sepha, very much afraid to lose his life, earnestly repeatedly remembered Varu\d{n}a and uttered that mantram in pluta tone (lengthened or prolonged). Varu\d{n}a, too, the ocean of mercy, knowing that the Br\=ahmi\d{n} boy was praising him with hymns came up to that spot and freed \'Sunah\'sepha from his bondage and freed the King also from his disease and went back to his own abode. Thus the Mahar\d{s}i Vi\'sv\=amitra became very glad to rescue the Muni's son from the jaws of death. The King Hari\'schandra did not observe the words of Vi\'sv\=amitra; hence the son of G\=adhi harboured within his heart anger towards the King. One day while the King Hari\'schandra was riding in a forest and there, at midday, on the banks of the river Kau\'sik, when he desired to kill a boar, Vi\'sv\=amitra in the garb of an old Br\=ahmi\d{n} asked from him everything that he had, including his dominion and thus cunningly took away everything from the King. The Mahar\d{s}i Va\'sistha, seeing his Yajam\=ana Hari\'schandra suffering much, became wounded and felt pain in his mind. One day when he met casually with Vi\'sv\=amitra in a forest, he said :-- ``O wicked K\d{s}attriya! A disgrace to your family! You have in vain put on the garb of a Br\=ahmi\d{n}; your religion is like a crane; you ate full of vanity; you boast for nothing. The best of kings, Hari\'schandra is my client; be is faultless; still, O Fool! Why are you giving him so much trouble. As you are religious as a crane is religious, so take your birth as a crane.'' Vi\'sv\=amitra, thus cursed by Va\'sistha, cursed Va\'sistha in return, and said :-- ``O Va\'sistha! As long as I will remain a crane, so long you also remain as \'Sar\=ali or \=Adi bird.''

37-42. Vy\=asa said :-- O King! The two angry Munis thus cursed each other and the two were born as Crane and \'Sar\=ali or \=Adi bird. The crane Vi\'sv\=amitra built its nest on the top of a tree on the M\=anasarovara lake and began to live there. Va\'sistha, too, assumed the form of an \=Adi bird,

and built his nest on the top of another tree and lived there. Thus the two \d{R}i\d{s}is spent their days in full enmity towards each other. These two birds used to shriek so terribly loud that they became a nuisance to all; they fought daily with each other. They used to strike each other with beaks and wings and nails and thus they were covered all over their bodies with cuts and wounds and they were smeared with blood. They began to look like Kim\'suka trees. Thus the two \d{R}i\d{s}is, in the shape of birds, in their states of bondage, due to each other's curse, passed many years there.

43. Janamejaya said :-- ``O Br\=ahma\d{n}a! Kindly tell me how Va\'sistha and Kau\'sika, the two \d{R}i\d{s}is, became free from their curses; I am very curious to hear this.''

44-54. Vy\=asa said :-- Brahm\=a, the Grandsire of his subjects, came there with all the Devas, filled with mercy, on seeing those two \d{R}i\d{s}is at war against each other. Brahm\=a, the Lotus-seated, made them desist from such a fight, consoled them and freed both of them from each other's curses. Then the Devas went back to their own abodes and the illustrious lotus-seated Brahm\=a went to the Satyaloka, seated on his Swan. Mahar\d{s}i Va\'sistha and Vi\'sv\=amitra became then friends and were tied with bonds of affection at the advice of Brahm\=a; they went back to their own \=A\'sramas. O King! Now see, that the Mahar\d{s}i Va\'sistha, the son of Mitr\=a-Varu\d{n}a, fought for nothing with Vi\'sv\=amitra, so painful to both the parties. Who, then, amongst the human beings, the D\=anavas or the Devas can conquer his Ahamk\=ara (egoism) and be always happy? Therefore the Chitta-\'Suddhi, the purity of the heart (that purity which imparts to man the blessedness of God-vision) is very difficult even for the high-souled persons; with the greatest caution and utmost effort one has to practise for that. To those persons, that are void of this Chitta \'Suddhi, it is all vain to go to places of pilgrimage, to make charities, to practise tapasy\=a, to be truthful; in fact, anything, which is the means to attain Dharma, becomes useless. O King! \'Sraddh\=a (Faith) is of three kinds :-- (1) S\=attvik\={\i}, (2) R\=ajasik\={\i} and (3) T\=amasik\={\i} to all persons in all their religious matters. The S\=attvik faith is the only one of the three that yields entire results; and it is very rare in this world. The R\=ajasik faith, done according to due rules, yields half the results thereof and the T\=amasik faith is fruitless and inglorious; the T\=amasik faith arises with those persons that are overwhelmed with lust, anger, greed, etc. Therefore, O King! Keep to the company of the good and hear the \'S\=astras, Ved\=anta, etc., and free the heart of worldly desires and then concentrate it to the worship of the Dev\={\i} and live in a sacred place of pilgrimage. Men afraid and troubled with the defects of the K\=al\={\i}yuga, should always

take the name of the Dev\={\i}, sing praises, and meditate on Her lotus feet. Thus the J\={\i}vas will not have any fear of K\=al\={\i} and the fallen vicious persons will easily be able to cross this ocean of the world and be free. There is no doubt in this.

Here ends the Thirteenth Chapter of the Sixth Book on the description of the battle between \=Adi and Baka after the discourse on \'Sunah\'sepha in \'Sr\={\i} Mad Dev\={\i} Bh\=agavatam, the Mah\=apur\=a\d{n}am of 18,000 verses by Mahar\d{s}i Veda Vy\=asa.



