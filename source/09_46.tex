\chapter{On the anecdote of Sasth\={\i} Dev\={\i}}

1. N\=arada said :-- ``O Thou, the foremost of the Knowers of the Vedas! I have heard from you the anecdotes of many Dev\={\i}s. Now I want to hear the lives of other Dev\={\i}s also. Kindly describe.''

2. N\=ar\=aya\d{n}a said :-- ``O Best of the Br\=ahma\d{n}as! The lives and glorious deeds of all the Dev\={\i}s are described separately. Now say, which lives you want to hear.''

3. N\=arada said :--``O Lord! Sasth\={\i}, Mangal\=a Chand\={\i}, and Manas\=a, are the parts of Prakriti. Now I want to hear the lives of them.''

4-22. N\=ar\=aya\d{n}a said :-- O Child! The sixth part of Prakriti is named as Sasth\={\i}. The Dev\={\i} Sasth\={\i} is the Presiding Deity of infants and children; She is the M\=ay\=a of Vi\d{s}\d{n}u and She bestows sons to all. She is one of the sixteen M\=atrik\=as. She is known by the name of Devasen\=a. She performs Vratas (vows); She is the chaste and dearest wife of Skanda. She decides on the longevity of children and is always engaged in their preservation. So much so, that this Siddha Yogin\={\i} always keeps the children on her side.

O Best of Br\=ahma\d{n}as! I will now talk about the method of worshipping this Dev\={\i} and the history about Her bestowing children that I heard from Dharma Deva. Hear. Sv\=ayambhuva Manu had one son Priyavrata. He was a great Yog\={\i}ndra and remained in practising austerities. So he was not inclined to have any wife. At last by the effort and request of Brahm\=a, he got himself married. But many days passed, and he could not see the face of a son. Then Mahar\d{s}i Ka\'syapa became his priest in the Putresti Sacrifice (to get a son); and when the sacrifice was over, he gave the sacrificial offering called charu to his wife M\=alin\={\i}. On eating the charu, the queen M\=alin\={\i} become pregnant. For twelve Deva years she held the womb. After twelve years she delivered a full developed son, of a golden colour; but the son was lifeless and his two eyeballs were upset. At this, the friend\'s wives became very sorry and began to weep. The mother of the child became so very sorrowful that she became senseless. O Muni! The King himself took the son on his breast and went to the burning ground. There with his child on his breast he began to cry aloud. Rather he got himself ready to quit his own

life than leave the son from away his breast. At this time he saw in the space overhead an aerial car, white as crystal, made of excellent jewels, coming towards him. The car was shining with its own lustre, encircled with woven silken cloth, which added to its beauty. Innumerable garlands of variegated colours gave it a very nice and charming appearance. On that car was seated a Siddha Yogin\={\i}, very beautiful, of a lovely appearance of a colour like that of white champakas, always youthful, smiling, adorned with jewel ornaments, ever gracious to show favour to the devotees. On seeing Her, the King Priyavrata placed the child from his breast on the ground and began to worship Her and chant hymns to Her with great love and devotion. And he then asked that peaceful lady, the wife of Skanda, Who was shining like a summer sun :-- ``O Beautiful! Who art Thou? Whose wife art Thou and whose daughter art Thou? From Thy appearance it seems that Thou art fortunate and respected amongst the female sex.''

23-24. O N\=arada! In ancient times, when the Daityas dispossessed the Devas of their positions, this Lady was elected as a general and got victory for the Devas; hence She was named Devasen\=a. Hearing the words of the King Priyavrata, Devasen\=a, who fought for the Devas and was all good to the whole world, said :--

25-42. O King! I am the mind-born daughter of Brahm\=a. My name is Devasen\=a. The Creator before created Me out of His mind and made Me over to the hands of Skanda. Amongst the M\=atrik\=as, I am known as Sasth\={\i}, the wife of Skanda. I am the sixth part of Prakriti; hence I am named Sasth\={\i}. I give sons to those who have no sons, wives to those who have no wives, wealth to the poor, and I give works to those who are workers (karm\={\i}s.)

Happiness, pain, fear, sorrow, joy, good, and wealth and adversity all are the fruits of Karmas. As the result of one's Karmas, people get lots of sons and it is due to the result of one's Karma again that people are denuded of all the issues of their family. As the result of Karma, the people get dead sons; and as the result of Karma the people get long lives. All enjoy the fruits of their Karmas, whether they be well qualified, or deformed or whether they have many wives, or whether they have no wife, whether they be beautiful, religious, diseased, it is all through Karmas, Karmas. Disease comes out of their Karmas. Again they get cured by their Karmas. So, O King! It is stated in the Vedas that Karma is the most powerful of all. Thus saying, Devasen\=a took the dead child on her lap; and, by the wisdom, early made the child alive. The King saw the child, of a golden colour got back his life and

began to smile. Thus bidding goodbye to the King, Devasen\=a took the child and became ready to depart. At this the King's palate and throat got dry and he began again to chant hymns to Her. The Dev\={\i} Sasth\={\i} became pleased at the stotra made by the King. The Dev\={\i} then addressed the King and said :-- ``O King! All that is stated in the Vedas, is made up of Karmas. You are the son of Sv\=ayambhuva Manu, and the Lord of the three worlds. You better promulgate My worship in the three worlds and you yourself worship Me. Then I will give you your beautiful son, the lotus of your family. Your son, born in part of N\=ar\=aya\d{n}a, will be famous by the name of Suvrata. He will be well-qualified, a great literary man, able to remember his conditions in his former lives, the best of Yogis, performer of one hundred Yaj\~nas, the best of all, bowed down by the K\d{s}attriyas, strong as one lakh powerful elephants, wealthy, fortunate, pure, favourite of literary persons, learned and bestower of the fruits of the ascetics, renowned and bestower of wealth and prosperity to the three worlds.'' Thus saying Devasen\=a gave the the child to the king. When the king promised that he would promulgate Her worship, the Dev\={\i} granted him boons and went up to the Heavens.

43-57. The king, too, becoming very glad and surrounded by his ministers, returned to his own abode and informed all about the son. The ladies of the house, become highly delighted when they heard everything. On the occasion of the son's getting back his life, the king performed everywhere auspicious ceremonies. The worship of Sasth\={\i} Dev\={\i} commenced. Wealth was bestowed to the Br\=ahmi\d{n}s. Since then, on every sixth day of the bright fortnight in every month, great festivals in honour of Sasth\={\i} Dev\={\i} began to be celebrated. Since then, throughout the kingdom, on every sixth day after the birth of a child in the lying-in-chamber, Sasth\={\i} Dev\={\i} began to be worshipped. On the twenty-first day, the auspicious moment, at the ceremony of giving rice to a child for the first time, when sixth months old, and on all other auspicious ceremonies of the children, Sasth\={\i} Dev\={\i}'s worship was made extant and the king himself performed those worships with great care and according to due rules. Now I will tell you about the Dhy\=anam and method of worship and stotra as I heard from Dharma Deva, and as stated in Kauthuma \'S\=akh\=a. Hear. He has said :-- In a \'S\=alagr\=ama stone, in a jar, on the root of a Bata tree, or drawing the figure on the floor of the rooms, or making an image of Sasth\={\i} Dev\={\i}, the sixth part of Prakriti and installing it, one should worship the Dev\={\i}. The Dhy\=anam is this :-- ``O Devasen\=a Thou art the bestower of good sons, the giver of good luck; Thou art mercy and kindness and the progenitor of the world; Thy colour is

bright like that of the white Champaka flowers. Thou art decked with jewel ornaments. Thou art pure, and the highest and best Dev\={\i}. Obeisance to Thee! I meditate on Thee.'' Thus meditating, the worshipper should place flower on his own head. Then again meditating and uttering the principal mantra one is to offer the P\=adya (water for washing feet), Arghya, \=Achaman\={\i}ya, scents, flowers, dh\=up, lights, offerings of food and best roots and fruits and one should worship thus with various things Sasth\={\i} Dev\={\i}. ``Om Hr\={\i}m Sasth\={\i} Devyai Svaha'' is the principal Mantra of Sasth\={\i} Dev\={\i}. This great Mantra of eight letters a man should repeat as his strength allows. After the Japam, the worshipper should chant hymns with devotion and undivided attention and then bow down. The Stotra (hymn) of Sasth\={\i} Dev\={\i} as per S\=ama Veda is very beautiful and son-bestowing.The lotus-born Brahm\=a has said :-- If one repeats (makes Japam) this eight lettered mantra one lakh of times, one gets certainly a good son. O Best of Munis! Now I am going to say the auspicious stotra of Sasth\={\i} Dev\={\i} as composed by Priyavrata. Hear.

58-73. One's desires are fulfilled when one reads this very secret stotra. Thus the King Priyavrata said :-- ``O Dev\={\i}, Devasen\=a! I bow down to Thee. O Great Dev\={\i}! Obeisance to Thee! Thou art the bestower of Siddhis; Thou art peaceful; obeisance to Thee! Thou art the bestower of good; Thou art Devasen\=a; Thou art Sasth\={\i} Dev\={\i}, I bow down to Thee! Thou grantest boons to persons; Thou bestowest sons and wealth to men. So obeisance to Thee! Thou givest happiness and mok\d{s}a; Thou art Sasth\={\i} Dev\={\i}; I bow down to Thee. Thou thyself art Siddha; so I bow down to Thee. O Sasth\={\i} Dev\={\i}! Thou art the sixth part of this creation; Thou art Siddha Yogin\={\i}, so I bow down to Thee. Thou art the essence, Thou art S\=arad\=a; Thou art the Highest Dev\={\i}. So I bow down again and again to Thee. Thou art the Presiding Deity Sasth\={\i} Dev\={\i} of the children; I bow down to Thee. Thou grantest good; Thou Thyself art good and Thou bestowest the fruits of all Karmas. O Thou O Sasth\={\i} Dev\={\i}! Thou shewest thy form to thy devotees; I bow down to Thee! Thou art \'Suddha Sattva and respected by all the persons in all their actions. Thou art the wife of Skanda. All worship Thee. O Sasth\={\i} Dev\={\i}! Thou hadst saved the Devas. So obeisance to Thee O Sasth\={\i} Dev\={\i}! Thou hast no envy, no anger; so obeisance to Thee. O Sure\'svar\={\i}! Give me wealth, give me dear things, give me sons. Give me respect from all persons; give me victory; slay my enemies. O Mahe\d{s}var\={\i}! Give me Dharma; give me name and fame; I bow down again again to Sasth\={\i} Dev\={\i}. O Sasth\={\i} Dev\={\i}! worshipped reverentially by all! Give me lands, give me subjects, give me learning; have welfare for me; I bow down again and

again to Sasth\={\i} Dev\={\i}.'' O N\=arada! Thus praising the Dev\={\i}, Priyavrata got a son, renowned and ruling over a great kingdom through the favour of Sasth\={\i} Dev\={\i}. If any man that has no son, hears this stotra of Sasth\={\i} Dev\={\i} for one year with undivided attention, he gets easily an excellent son, having a long life. If one worships for one year with devotion this Devasen\=a and hears this stotra, even the most barren woman becomes freed from all her sins and gets a son. Through the grace of Sasth\={\i} Dev\={\i}, that son becomes a hero, well qualified, literate, renowned and long-lived. If any woman who bears only a single child or delivers dead children hears with devotion for one year this stotra, she gets easily, through the Dev\={\i}'s grace, a good son. If the father and mother both hear with devotion, this story during the period of their child's illness, then the child becomes cured by the Grace of the Dev\={\i}.

Here ends the Forty-sixth chapter of the Ninth Book on the anecdote of Sasth\={\i} Dev\={\i} in the Mah\=a Pur\=a\d{n}am \'Sr\={\i} Mad Dev\={\i} Bh\=agavatam of 18,000 verses by Mahar\d{s}i Veda Vy\=asa.



