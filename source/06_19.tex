\chapter{On the origin of Haihayas from a mare}

1-3. Vy\=asa said :-- O King! Thus granting the boon to the Goddess Lak\d{s}m\={\i}, \'Sambhu quickly returned to the lovely Kail\=a\'sa, adorned with Apsar\=as (celestial nymphs) and frequented and served by the Gods. He then despatched his expert attendant Chitrar\=upa to Vaikuntha to bring the purpose of Lak\d{s}m\={\i} to a successful issue. He said to him thus :-- ``O Chitrar\=upa! Go to Hari and speak to him on my behalf that

He would go and remove the sorrows of His distressed and bereaved wife and thus make Her comfortable.''

4-9. Thus ordered, Chitrar\=upa started immediately and reached at once Vaikuntha, the highest place, covered all over by the Vai\d{s}\d{n}avas. The place was diversified with lots of various trees, with hundreds of lovely lakes, and echoed with sweet lovely sounds of swans, K\=arandavas, peacocks, parrots, cuckoos and various other birds and adorned with beautiful places, decked with flags and banners. It was filled with charming dancings, music and other artistic things. There were the lovely Bakula, A\'soka, Tilaka, Champaka and other trees; and the beautiful tree Mand\=ara looked beautiful and shed all around the sweet fragrance of its sweet flowers for a long distance. Thus seeing the lovely nice palace of Vi\d{s}\d{n}u and the two doorkeepers Jaya and Vijaya standing with canes in their hands, Chitraratha bowed down to them and said :-- Well! You go quickly and inform the Supreme Soul Hari that a messenger has come under the orders of the Bhagav\=an \'S\=ulap\=a\d{n}\={\i} and is now waiting at His doors.

10-18. Hearing his words, the intelligent Jaya went to Hari and, with folded hands, said :-- ``O Thou Ocean of Mercy! O Ke\'sava! O Lord of Ram\=a! O Deva of the Devas! A messenger has come from the Lord of Bhav\=an\={\i} and is waiting at the doors. I do not not know on what important business he has come. Please order whether I will bring him before You or not. On hearing the Jaya's words, Hari, aware of the inner feelings, knew at once the cause and said :-- O Jaya! Bring before me the messenger come from Rudra. Thus hearing, Jaya called the \'Siva's servant, of a graceful form, and brought him to the presence of Jan\=ardana. Chitrar\=upa, of variegated appearance, prostrated himself flat before Him in the form of a stick and stood up and remained with folded hands. The Bhagav\=an N\=ar\=aya\d{n}a, Whose carrier is Garuda, saw that servant of \'Siva, of variegated appearance and full of all humility, and became very much astonished. The Lord of Kamal\=a then smiled and asked Chitrar\=upa :-- ``O Pure One! Is it all well with Mah\=adeva, the Lord of the Devas and his other families and attendants? On what business has He sent you here? What does He want me to do? Or tell me if I have to do any other business of the gods.''

19-34. The messenger said :-- ``O Thou, the Knower of all that is within one's heart! There is nothing in this world hidden from Your knowledge; when is that which I will say unknown to you! O Thou, the Knower of present, past and future! I am now saying to you what \'Sambhu has told me to inform You. He has said :-- O Lord! The Goddess Lak\d{s}m\={\i} is Your dear consort. She, the daughter of the Ocean, and the Bestower of all

success, though an object fit to be meditated by Yak\d{s}as, Kinnaras, Naras and Immortals, is now undergoing severe penance at the confluence of Kalind\={\i} (the Jumn\=a, the daughter of Kalinda) and the Tamas\=a. What is there in the three worlds that can be happy without that Mother of the worlds and the Giver of all desires? O Lotus-eyed One! What pleasure do You feel in abandoning Her? O All-pervading One! Even he who has no riches or who is very weak maintains his wife; then why have You, being the Lord of the worlds, forsaken your wife, without any offence, Who is worshipped by the whole universe. O Lord of the world! What advice shall I give to You? He whose wife suffers in the world, is blamed by his enemies. O Omnipresent One! Fie on his such a life! O Lord of the worlds! Your enemie\'s desires are satisfied when they see Her very miserable. They are laughing and mocking and saying :-- O Dev\={\i}, Ke\'sava has now forsaken you; you can spend happily your time with us now. Therefore, O Lord of the Devas! Bring that Lady back unto your palace and place Her unto your lap, Who is of good demeanour, beautiful, par excellence and endowed with all auspicious signs. O Deva! Accept, please, your sweet-smiling wife and be happy. Though I am at present not in bereavement of my dear wife, yet when I remember my former state of bereavement, I feel very much trouble. O Lotus-eyed One! When Sat\={\i} Dev\={\i}, my beloved Wife, quitted Her life, in Dak\d{s}a's house, I felt unbearable pain, O Ke\'sava! Let no other body in this world suffer such pain, I now remember only the suffering and mental agonies that I felt on Her bereavement; I do not give it out to others. After a long time, practising severe Tapasy\=a (asceticism) I got Her back in the form of Girij\=a, who felt herself burnt up as it were by the anger She felt on account of censure cast on Me in the Dak\d{s}a's house and thus quitted Her life. O Mur\=ari! What happiness you have felt in forsaking your dear wife and in remaining thus alone for one thousand years. Console your fortunate young wife with good teeth and bring her back to your place. O Bhagav\=an! Lastly, the Lord Bhav\=an\={\i}, the Originatrix of these worlds, told me to speak thus to you :-- O Destroyer of Kamsa! Let nobody remain even, for a moment, without Lak\d{s}m\={\i}, the Highest Goddess. O Long-lived One! You better assume the form of a horse and go and worship her. Then have a child of yours in the womb of your sweet-smiling wife and bring her back to your house.''

35-42. Vy\=asa said :-- O Ornament of Bh\=arata's race! Hearing thus the words of Chitrar\=upa, Bhagav\=an Hari told that he would do what \'Sankara had told him to do and thus sent the messenger back to \'Sankara. The messenger departing, Hari assumed the form of a beautiful horse

and immediately left Vaikuntha with a passionate intent for the place where Lak\d{s}m\={\i} was staying in the form of a mare and practising her austerities. Coming there, he saw that the Dev\={\i} Bimal\=a was staying in the form of a mare. The mare, too, seeing the horse form of her husband Govinda, recognised him and, chaste as she was, remained there with astonishment and tears in her eyes. Then those two copulated on the famous confluence. The wife of Hari, in the shape of a mare, became pregnant and, in due time, gave birth to a beautiful well qualified child. The Bhagav\=an then graciously smiled on her and spoke in words suited to the time, ``O Dear! Now quit this mare form and assume your former appearance. O Lovely-eyed One! Let both of us assume our own forms and go to Vaikuntha; and let your child remain in this place.''

43-48. Lak\d{s}m\={\i} said :-- ``O Lord! How can I go leaving here this child, born of my womb. It is very difficult to quit the attachments for one's own child. Know this, O Lord! O High-souled One! This child is young and of small body; therefore it is quite incapable to protect itself. If I forsake it on the bank of this river, it will be an orphan, what will happen to it then? O Lotus-eyed One! My mind is now in full attachment towards it. How can I quit this helpless child and go?'' When Lak\d{s}m\={\i} and N\=ar\=aya\d{n}a resumed their divine bodies and mounted on the excellent Vim\=anas, the Devas began to praise them with hymns. When N\=ar\=aya\d{n}a expressed his desire to go, Kamal\=a said :-- ``O Lord! You better take this child; I am unable to forsake it. O Lord! O Slayer of Madhu! This child is dearer to me than my life; see its body resembles exactly like you. Therefore we would take this child with us to Vaikuntha.''

49-54. Hari spoke :-- ``O Dear! You need not be sorry; let this child remain here happily; I have arranged for its preservation and safety. O Beautiful One! There is a great act to do in this world. That will be executed by your child. For this reason I am leaving it here. I am now describing to you the above story. The famous King Yay\=ati had a son named Turvasu; his father kept his name as Hari Varm\=a; he is known by this name. That king is now practising asceticism for getting a son for one hundred years in a place of pilgrimage. O Lak\d{s}m\={\i}! I have begot this son for him. I will go there and send the King here. O Beautiful-faced One! I will give this son to that King, desirous of an issue. He will take this son and go back to his house.''

55. Vy\=asa said :-- O King! Thus consoling his beloved, whose abode is in the Lotus and placing the child there in that position, He mounted on an excellent car with Lak\d{s}m\={\i} and went to Vaikuntha.

Here ends the Nineteenth Chapter in the Sixth Book on the origin of Haihayas from a mare in the Mah\=apur\=a\d{n}am \'Sr\={\i} Mad Dev\={\i} Bh\=agavatam of 18,000 verses by Mahar\d{s}i Veda Vy\=asa.



