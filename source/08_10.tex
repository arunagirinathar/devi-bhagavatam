\chapter{On the description of Bhuvanako\d{s}a}

1-7. N\=ar\=aya\d{n}a said :-- In Hira\d{n}maya Var\d{s}a, the Bhagav\=an is remaining in the form of Kurma, the Tortoise, as the Lord of Yoga. He is thus praised and worshipped by Aryam\=a, the Ruler of the Pitris. Aryam\=a said :-- Om namo Bhagavate Ak\=up\=ar\=aya; (King of tortoises, sustaining the world) Obeisance to Thee,

the Lord of all prosperities, in the form of Tortoise (Kurma); Thou art built of Sattva Gu\d{n}a Incarnate; no one can make out where Thou dost dwell; Thou art not encompassed by Time; (Thou art in the Present, Past and Future); so obeisance to Thee. Thou dost pervade all things; we bow down to Thee. All are established in Thee; so obeisance to Thee. By Thy extraordinary M\=ay\=a (power) Thou hast made manifest this universe that is seen. This is Thy Form. It is by no means distinct from Thee. This Thy Form is seen in so many forms. So the true reality being not known like the mirage, these cannot be counted really speaking, what is Thy form, no one can definitely say. The beings generated by heat and moisture (said of insects and worms), those that are born of eggs, from wombs and the plants and other moving, non-moving beings, the Devas, \d{R}i\d{s}is, Pitris, Bh\=utas, and these senses; the sky, the heavens, earth, mountains, rivers, oceans, islands, planets, and stars all these art Thou and Thou alone. Thy name, form, and appearance, are as varied; and their numbers cannot be counted. Still, Kapila and others have determined their numbers, by the knowledge of which Thou canst become visible to the Eye of Knowledge. Thy form and nature are determined by these S\=ankhyas ascertained by Kapila. So we bow down to Thee. Thus Aryam\=a, and the other rulers of the Var\d{s}a all united sing, praise, and worship the Bhagav\=an Kurma Deva, the Controller of all and the Generator of all. All Hail to Thee! The Bhagav\=an Yaj\~na Puru\d{s}a is manifest in Uttara Kuru Mandala in the form of \=Adi Var\=aha. The Earth Herself worships Him always. The Goddess Earth praises Hari, the Yaj\~na Var\=aha, the Destroyer of the Daityas and worships duly that Deva, with Her heart lotus, naturally devoted, rendered more devoted by Her attachment to the Lord.

8-13. The Goddess Earth spoke :-- ``Om Namo Bhagavate Mantratattva Li\d{n}g\=aya Yaj\~na Kratave'' I bow down to the Bhagav\=an, the Great Boar; Thou art Om; Thy real form and nature can be known by only the Mantra and Tattva. Thou art Yaj\~na and Kratu (sacrifice) incarnate; therefore all the great sacrifices are Thy limbs. Thou art the Three Yugas (there being no Yaj\~na in the Satya Yuga); Thou art that which is left as Pure, after doing Karma (so as to be fit for performing Yaj\~nas). So obeisance to Thee. The sages, versed in J\~n\=ana and Vij\~n\=ana say that Thou art hidden in the body and in the senses as fire is hidden in the wood. So they, ardent to see Thee, seek for Thee with a discriminative and dispassionate mind, judging Karmas and their fruits; and then Thy Nature is revealed, I bow down to Thee, Thy Form can be ascertained by the cause and effect of the Karmas and other Gu\d{n}as of M\=ay\=a, sense objects, senses, actions, Devas, body, time, Ahamk\=ara and others. I bow

down to Thee, Those can see thus Thy form, whose mind is firmly established in Thee, by their discrimination and Yama, Niyama, etc., and who have abandoned all sorts of fickleness and changeability of their tempers. So obeisance to Thee. As iron goes attracted towards the magnet, so M\=ay\=a dances before Thee with Her Gu\d{n}as and Her works in the way of the creation, preservation and destruction of this universe; but Thou art totally indifferent to it. For the sake of the J\={\i}vas (embodied souls), desire comes to Thee, though Thou art not quite willing! Thou art the Witness of the J\={\i}vas and their Adrista (the Fate). I bow down to Thee. The Yaj\~na Var\=aha, the Cause of this universe, has lifted me up from the Ras\=atala and placing me on His big tusks, has come out from the Pralaya, the great ocean, after overpowering in battle His enemy, the powerful Daitya, like an elephant, I bow down to that Controller of all, to Thee. In the Kimpuru\d{s}a Var\d{s}a, the Bhagav\=an \=Adi Puru\d{s}a (the Prime Man), the Self-manifest, and the Lord of all, is residing in the form of R\=ama, the son of Da\'saratha and the Joy of the heart of S\={\i}t\=a Dev\={\i}.

14-18. \'Sr\={\i} Hanum\=ana thus spoke:-- ``Om namo Bhagavate Uttama Slokaya'' I bow down to the Bhagav\=an, who art sung by the excellent verses, purifying all. I bow down to Thee, the incarnate of modesty, good temper, vow and good signs; Thy mind is always under control; Thou dost imitate, as Thy nature is good, the actions of all persons; obeisance to Thee. Thou art the Supreme Place to award praise. Obeisance to Thee. Thou art Brahma\d{n}ya Deva (in the creation of the universe), the high souled Person Mah\=apuru\d{s}a. Thou gettest the First Share, above all the persons! Thou art the One Tattva and That Alone, as established in the Ved\=anta. The holy realisation is the only guide to it. This Tattva dominates over all the Gu\d{n}as. It can never be an object. Only by pure intellect, It can be realised. There is no name, no form of It. It is always beyond the pale of Ahamk\=ara. I take refuge to this Tattva, the most Peaceful, with my body and mind. Thy incarnation in human shape in this world is not simply for killing R\=ava\d{n}a but for giving instructions to the mortals. The contact with woman and the suffering thereof are very difficult to avoid; to give this lesson also He took this incarnation, He Who is merged in enjoying the Supreme Bliss of His Own Nature and He Who is the Lord of all, how can He suffer miseries in the bereavement of S\={\i}t\=a. He is the best friend and the very \=Atman of those who have conquered their minds and senses. Especially He is the receptacle of all the qualities and is in enjoyment of other divine extraordinary powers. So He is not attached to the worldly objects. How can the delusion due to His wife come and darken Him? and why will He send Lak\d{s}ma\d{n}a

in exile? He is the Mahat Tattva and the Parama Puru\d{s}a; so good birth, beauty, intelligence, oratory or good form, nothing can please Him. Bhakti (devotion) can only attract Him. If that be not the case, then why will He, the elder of Lak\d{s}ma\d{n}a, the Bhagav\=an, the son of Da\'saratha make friendship with us, the wanderers of the forest and who are by nature, not the receptacles of any beauty, etc. So everyone, be he a Sura or Asura, man, or not man, should worship the Hari manifest in R\=ama, in the human body with all his heart. He is so good that if anybody worships Him even to a very small extent, He always considers it to be much; what more can be said than this that He took all the inhabitants of Ko\'sala to Heaven!

19-20. N\=ar\=aya\d{n}a said :-- Thus Hanum\=an, the best of monkeys, sings the praises and worships duly in Kimpuru\d{s}a Var\d{s}a, the lotus-eyed R\=ama, truthful, and determined in his vows. He who hears this wonderful description of R\=ama, is freed of all his sins and goes with his body pure to the abode of R\=ama.

Here ends the Tenth Chapter of the Eighth Book on the description of Bhuvanako\d{s}a in the Mah\=apur\=a\d{n}am \'Sr\={\i} Mad Dev\={\i} Bh\=agavatam, of 18,000 verses, by Mahar\d{s}i Veda Vy\=asa.



