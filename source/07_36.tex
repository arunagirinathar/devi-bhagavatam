\chapter{1-4. \'Sri Dev\={\i} said :-- "O Him\=alay\=as! Thus making one's own self attached to the Yoga by the above-mentioned process and sitting on a Yoga posture, one should meditate on My Brahma Nature with}

an unfeigned devotion. (How the knowledge of that Formless Existence and Imperishable Brahman arises, now hear.) He is manifest, near, yea, even moving in the hearts of all beings. He is the well-known Highest Goal. Know that all this whatever, waking, dreaming, or sleeping, which moves, breathes or blinks, is founded on Him. He is higher than Being and Non-being: higher than the Wisdom, He is the Best Object of adoration for all creatures. He is brilliant, smaller than the smallest and in Him the worlds are founded and the Rulers thereof. He is the Imperishable Brahman. He is the Creator (Life), the Revealer of Sacred Knowledge (speech) and Omniscient (or the Cosmic Mind). This is the Truth. He is immortal, O Saumya! Know that He is the target to be hit.
Note. -- The words "higher than wisdom" mean higher than Brahm\=a. (Brahm\=a is the highest of all J\={\i}vas, higher than Brahm\=a means higher than all creatures. The word Vij\~n\=ana denotes Brahm\=a as we find in the following speech of Brahm\=a in the Bh\=agavat Pur\=a\d{n}a) "I, the Wisdom Energy (Vij\~n\=ana-\'Sakti) was born from the navel of this Being resting on the Waters and possessed of the Infinite Powers."
Vi\d{s}\d{n}u is called "Pra\d{n}a," because he is the leader of all (Pr\=a\d{n}a-netri). He is called V\=ak, because He is the Teacher of all; Vi\d{s}\d{n}u is called Manas because He is the adviser of all (Mantri). He is the Controller of all the J\={\i}vas.
The third verse lays down that Brahman is to be meditated upon or that the Manana should be performed; as the second verse teaches that Dhy\=ana or concentration also is necessary.
5-6. Take hold of the Mystic Name as the bow, and know that the Brahman is the aim to be hit. Put on this the great weapon (Om), the arrow (of the mind) sharpened by meditation. Withdraw thyself from all objects, and with the mind absorbed in the idea of Brahman, hit the aim; for know, O Saumya! That Imperishable alone to be the Mark. The Great name"Om" is the bow, the mind is the arrow, and the Brahman is said to be the mark. It is to be hit by a man whose thoughts are concentrated, for then he enters the target.
Note. -- Thus \'Srava\d{n}a, Manana, and Dhy\=ana of Brahman have been taught. This is the method of Brahma-up\=asan\=a.
7. In Him are woven the heavens, and the interspaces, and mingle also with the senses. Know Him to be the one Support of all, the

\=Atman. Leave off all other words (as well as the worship of other deities). This (\=Atman) is the refuge of the Immortals.
"He is the bridge of the Immortal"--the words Amrita or Immortal means Mukta J\={\i}vas. In the Ved\=anta S\=utra I, 3-2, it has been taught that the Lord is the refuge of the Muktas. So also that "He is the Highest Goal of the Muktas."
8-9. In Him the life-webs (n\=adis) are fastened, as the spokes to the nave of a chariot; He is this (\=Atman) that pervades the heart, and by his own free will manifests Himself in diverse ways (as Visva, Taijasa, etc., in waking, sleeping, etc., states); and also as One as Pr\=aj\~na in the dreamless state. Meditate on the \=Atman as Om (full of all auspicious qualities and who is the chief aim of the Vedas), in order to acquire the knowledge of the Param\=atman, Who is beyond the Prakriti and the \'Sri Tattva. Your welfare consists in such knowledge.
Note. -- This shows that Brahman is the Antary\=amin Puru\d{s}a. He resides in the heart where all the 72,000 N\=adis meet, as the spokes meet in the navel of the wheel. He moves within the organs, not for His own pleasure, but to give life and energy to them all. The Om with all its attributes must be constantly meditated upon. He manifests Himself in manifold ways in the waking and dreaming stews as Vi\'sva and Taijasa; while He manifests as One in the state of Su\d{s}upti or Dreamless sleep as Pr\=aj\~n\=a. He is beyond darkness; He has no mortal body. Meditate on such Vi\d{s}\d{n}u in the heart in order to get the Supreme Brahman, with the help of the Mantra Om. The result of such meditation is that there is the welfare of yours--all evils will cease, and you will get the bliss of the manifestation of the Divinity--your Real Self within your Heart.
10. He who is All-Wise, and All-Knowing, whose Greatness is thus manifested in the worlds, is to be meditated upon as the \=Atman residing in the Ether, in the Fourth Dimensional Space, in the shining city of Brahman (the Heart). He is the Controller of the mind and the Guide of the senses and the body. He abides in the dense body, controlling the heart. He, the \=Atman, when manifesting Himself as the Blissful and Immortal, is seen by the wise through the purity of the heart.
11. The fetters of the J\={\i}vas are cut asunder, the ties of Linga-dehas and Prakriti are removed (the effects of all) his works perish, when He is seen who is Supremely High (or when the Supremely High looks at the J\={\i}va.) [Note.--Vi\d{s}\d{n}u is Par\=avare, because Par\=a or High Beings like Ram\=a; Brahm\=as, etc., are Avara or inferior in His comparsion.]

[Note. -- This shows the result of Divine Wisdom in the last verse. The Avidy\=a covers both \=I\'svar\=a and J\={\i}va. It prevents \=I\'svara being, seen by J\={\i}va, and J\={\i}va, seeing \=I\'svara. It is a direct bondage of J\={\i}va and a metaphorical fetter of \=I\'svara. Avidy\=a is the name given to Prakriti in Her active state. When Her three qualities Sattva, Rajas and Tamas, are actively manifest. Destruction of Avidy\=a means putting these Gu\d{n}as in their latent state. There is a great difference between the destruction of the Avidy\=a--fetters as taught in this verse, and the unloosening of them as previously described in this verse! There Avidy\=a still remained, for it was merely a Parok\d{s}a or intellectual apprehension of Truth. Here Avidy\=a itself is destroyed by Aparok\d{s}a or Intuitive Knowledge of Brahman.
The bonds are five :-- The lowest is the Avidy\=a bond, then the Lingadeha bond, then the Pram\=achch\=adaka Prakriti bond, the K\=ama bond, and the Karma bond. When all these bonds are destroyed, then the J\~n\=an\={\i} goes by the Path of Light to the S\=ant\=amka Loka. Before proceeding further all have to salute the \'Si\'su-m\=ara--the Dweller on the threshhold--the hub of the Universe.
The Si\'sum\=ara literally means the Infant Killer and means the porpoise and is the name of a constellation, in the north, near the Pole. It corresponds perhaps with the Draco or the Urs\=a Minor. For a fuller description of it, see Bhagam Pur\=a\d{n}a Book 5, Chapter 23. Here it is a mystical reference to a Being of an exalted order, which every J\~n\=an\={\i} passes by, in his way beyond this Universe. It may correspond with the ring-pass-not of the 'Secret Doctrine'! It is the name of Hari, also, as we find the following verse "The Supreme Hari, the Support of infinity of worlds and who is called Si\'sum\=ara, is saluted by all knowers of Brahma, on their way to the Supreme God.]
12. The Brahman (called Si\'sum\=aram) free from all passions and parts (manifests in the external world) in the highest Golden Sheath (the Cosmic Egg). That is pure, that is the highest of Lights, it is that which the knowers of \=Atm\=an know. [Note. -- "He is in the Centre of the Cosmic (as Si\'sum\=ara, the Light of all Cosmic Suns). He is even in the centre of our Sun and illumining all planets."]
In the first respect He is meditated upon as Si\'sumara and in the second as G\=ayatr\={\i}." [Note.--In man, the Brahman manifests in the heart or the Auric Egg, called the city of Brahman. In the Universe, He manifests Himself in the Cosmic Egg, called the "Golden Sheath." These are the two places where Brahman may be meditated upon.

This verse has been explained in two different ways: First, as applying to \'Si\'sum\=ara and secondly, as teaching how to meditate on N\=ar\=aya\d{n}a in the Sun. The "Golden Sheath" would then mean the Solar sphere. The Supremely High Brahman resides in the excellent Golden Sheath. He is Pure and Without parts.]
13. The Sun does not shine there in His Presence nor the Moon and the Stars (for His Light is greater than theirs, they appear as if dark in that Effulgence, like the candle-light in the Sun. Nor do these lightnings, and much less this fire shine there. When He shines, everything shines after Him; by His Light all this becomes manifest.
Him the Sun does not illumine nor the moon and the stars. Nor do these lightnings; much less this Fire illumines Him. When He illumines all (the Sun, etc.,) then they shine after (Him with His light). This whole Universe reveals His Light (is His Light and its Light is His). Note.--The Sun, etc., do not illumine Him, i.e., cannot make Him manifest.
14. The Eternally Free is verily this Brahman only. He is in the West, in the North and the South, in the Zenith and the Nadir. The Brahman alone is; it is He who pervades all directions. This Brahman alone is it who pervades. This Brahman alone is the Full (that exists in all time the Eternity). This Brahman is the Best:--
This (idam) Brahman is alone the Vi\'svam or Infinity or Full (p\=ur\d{n}\=am). This alone is the Best, the Highest of all. As the word "idam" is used several times in this verse, it qualifies the word Brahman and not "vi\'svam," [Note.--The Brahman was taught to be meditated upon fully in the Heart and the Hira\d{n}maya Ko\'sa. But lest one should mistake that He is thus limited in those two places, one is to infer that they are selected as the best.]
15-16. The man who realises thus is satisfied and has all that he wants to do and is considered as the best. He becomes Brahman and his Self is pleased and he neither wants anything nor becomes sorry. O King! Fear comes from the idea of a second; where there is no second, fear does not exist. No danger then arises for him to be separated from Me. Nor I also get separated from him.
17. O Him\=alay\=as! Know that I am he and he is I. Know that I am seen there where My J\~n\=an\={\i} resides.
18. Neither I dwell in any sacred place of pilgrimage, nor do I live in Kail\=asa nor in Vaikuntha nor in any other place. I dwell in the heart lotus of My J\~n\=an\={\i}.

19. The blessed man who worships once My J\~n\=an\={\i}, gets Koti times the fruit of worshipping Me. His family is rendered pure and his mother becomes blessed. He whose heart is diluted in the all-pervading Brahma Consciousness, purifies this whole world. There is no doubt in this.
20. O Best of Mountains! I have now told everything that you asked about Brahma J\~n\=ana. Nothing now remains to be further described.
21. This Brahma Vidy\=a (science of the knowledge of Brahma) is to be imparted to the eldest son, who is devoted and of good character and to him who is endowed with the good qualities as enumerated in the \'S\=astras and not to be given to any other person.
22. He who is fully devoted to his Ista Deva and who is equally devoted to his Guru, to him the high-minded persons should declare the Brahma Vidy\=a.
23. Verily, he is God himself, who advises this Brahma Vidy\=a; no one is able to repay the debts due to him.
24. He who gives birth to a man in Brahma, is, no doubt, superior to the ordinary father; for the birth that a father gives is destroyed; but the birth in Brahma that is given by the Guru is never destroyed.
25. So the \'Sruti says :-- Never do harm to the Guru who imparts the knowledge of Brahma.
26. In all the Siddh\=antas (decided conclusions) of the \'S\=astras, it is stated that the Guru who imparts the knowledge of Brahman is the best and the most honourable. If \'Siva, becomes angry, the Guru can save; but when the Guru becomes angry, \'Sankara cannot save. So the Guru should be served with the utmost care.
27. So the Guru must be served with all the cares that are possible by body, mind, and word one should always please Him. Otherwise he becomes ungrateful and he is not saved.
28. O Best of Mountains! It is very difficult to acquire Brahma J\~n\=ana. Hear a story. A Muni named Dadhyam of Atharvana family went to Indra and prayed to him to give Brahma J\~n\=ana. Indra said: "I would give you Brahma-J\~n\=ana, but if you impart it to any other body, I would sever your head." Dadhyarna agreed to this and Indra gave him the Brahma-J\~n\=ana. After a few days, the two A\'svins came to the Muni and prayed for Brahma Vidy\=a, The Muni said :-- "If I give

you the Brahma-Vidyi, Indra, will cut off my head." Hearing this the two A\'svins said :-- "We will cut your head and keep it elsewhere and we will attach the head of a horse to your body. Instruct us with the mouth of this horse and when Indra will cut off your this mouth, we will replace your former head." When they said so, the Muni gave them the Brahma-Vidy\=a. Indra cut off his head by his thunderbolt. When the horse-head of the Muni was cut off, the two physicians of the Devas replaced his original head. This is widely known in all the Vedas.
O Chief of Mountains! He becomes blessed who gets this the Brahma-Vidy\=a.
Here ends theThirty-sixth Chapter of the Seventh Book on the Highest Knowledge of Brahma in the Mahapur\=a\d{n}am, \'Sr\={\i} Mad Dev\={\i} Bh\=agavatam, of 18,000 verses, by Mahar\d{s}i Veda Vy\=asa.



