\chapter{On the Solar and Lunar Kings}

1-5. S\=uta said :-- Glad to hear the excellent divine stories of the Solar and Lunar races, the virtuous King Janamejaya, the son of Par\={\i}k\d{s}it, again asked :-- ``O Lord! I am now very eager to hear the increase of the two lines of Kings. O Sinless One! You know everything. So kindly describe, in detail, the pure histories, capable to destroy sins, of the kings and their characters. The kings of the Lunar and the Solar races were great Bhaktas of the Highest \'Sakti, \'Sr\={\i} Bhagavat\={\i} Dev\={\i}; this I have heard. O Muni! Who wants not to hear further on the glorious anecdotes of the Bhaktas of the \'Sr\={\i} Dev\={\i}?'' When the R\=ajar\d{s}i asked thus, the Muni Kri\d{s}\d{n}a Dvaip\=ayan, the son of Satyavat\={\i} began to narrate gladly the several deeds of the Kings.

6-13. O King! I am now narrating to you in detail the origin, etc., of the Lunar and the Solar dynasties as well as of other kings in their connection. Hear attentively. The four-faced Brahm\=a sprang from the navel of Vi\d{s}\d{n}u; engaged in practising Tapasy\=a, he began to worship the Mah\=a Dev\={\i} Durg\=a, extremely hard to conceive. Mah\=a Dev\={\i}, pleased at his worship, granted boon to Brahm\=a; Brahm\=a, the Grandsire of all the Lokas on thus getting the boon, became ready to create the world; but he could not create all on a sudden the human beings. Though the creation was eternally fashioned by the Pram\=atm\=a Bhagavat\={\i}, the four-faced Brahm\=a thinking over in his mind variously, could not quickly spread it out and accomplish it as a veritable fact. Therefore He, the Praj\=apati, first created mentally the seven mind-born sons. These were known by the names of Mar\={\i}chi, Atri, Angir\=a, Pulastya, Pulaha, Kratu and Vai\'sistha. Next sprang Rudra from the anger of Praj\=apati, N\=arada from his lap; Dak\d{s}a from his right thumb. Thus Sanaka and the other \d{R}i\d{s}is were also his mind-born sons. O King! The wife of Dak\d{s}a was born from the left thumb of Praj\=apati; this all-beautiful daughter is well known in all the Pur\=a\d{n}as by the name of V\={\i}ri\d{n}\={\i} and Asikn\={\i}. N\=arada, the chief of the Devar\d{s}is, was born, on some other occasion in her womb.

14-17. Janamejaya said :-- ``O Br\=ahm\=a\d{n}! A great doubt arises in me to hear that the great ascetic N\=arada was born of Dak\d{s}a in the womb of V\={\i}ri\d{n}\={\i}. The Muni N\=arada indeed, was the son of Brahm\=a; moreover be was the foremost of the ascetics and especially endowed with the

knowledge of Dharma; how, then, can he be born of the womb of the Daksha's wife V\={\i}ri\d{n}\={\i}. Well, if that be so, then describe, in detail, that wonderful story of the birth of N\=arada in the womb of V\={\i}ri\d{n}\={\i}.

18-31. O Muni! Under whose curse, the high-souled N\=arada though very wise, had to leave his first body and be born again.'' Vy\=asa said :-- O King, Brahm\=a, the Self-born, with a view to create offspring, ordered first Dak\d{s}a :-- ``Go and multiply innumerable children for the increase of the world.'' Thus ordered by his father the Praj\=apati Dak\d{s}a produced five thousand powerful and heroic sons in the womb of V\={\i}ri\d{n}\={\i}. Seeing all the sons of Dak\d{s}a, desirous to multiply, the Devar\d{s}i N\=arada urged on, as it were, by Fate, began to laugh at them. How do you desire to multiply when you know not the dimensions and capacity of earth; so you will, no doubt, be put to ridicule and laughter. Rather, if you proceed on work, knowing beforehand the earth's capacity, your efforts will be fruitful. Otherwise, your attempt will no doubt, end in failures. Alas! You are awfully illiterate! Not knowing the dimensions of the world, you are ready to multiply your progeny; how, then, can you meet with success! Vy\=asa said :-- O King! Hearing, all on a sudden, these words, Harya\'sva and other sons began to speak with each other, ``What this Muni has told, is very true. Let us then ascertain the earth's dimensions; we can easily multiply afterwards.'' Thus saying, they all went out to reconnoitre the earth. Thus excited at N\=arada's words, some went eastward, some southwards, some towards the north and some went to west all simultaneously and, as they liked, to make a survey of the earth. When the sons went away, Dak\d{s}a became exceedingly sorry on their absence. Bent again on multiplying, he begat other sons; those sons again wanted to procreate. Seeing them, N\=arada again laughed and said :--Alas! What fools are you! Not knowing the dimensions of the earth, why are you ready to procreate? They were deluded by N\=arada's words, took them as true, and went out as their elder brothers did. Not being able to see those sons, Praj\=apati Dak\d{s}a became very sorrowful for them and cursed N\=arada in rage.

32-38. Dak\d{s}a said :-- ``O Evil-minded One! You have destroyed my sons; so be yourself destroyed; you will have to be born in the womb for your sin in causing the death of my sons; you have caused my sons to go abroad; so you must be born as my son.'' Thus cursed by Dak\d{s}a, N\=arada had to take his birth in the womb of V\={\i}ri\d{n}\={\i}. I heard also that the Praj\=apati Dak\d{s}a begat afterwards sixty daughters in her womb. O King! Dak\d{s}a, the great knower of Dharma, then gave up the sorrows for his sons and married his thirteen daughters to the high-souled Ka\'syapa,

ten daughters to Dharma, twenty-seven daughters to the Moon, two to Bhrigu, four to Aristanemi, two to Kri\'s\=a\'sva and the remaining two to Angir\=a. Their sons and grandsons, the Devas and D\=anavas, became powerful but antagonistic towards each other. All of them were heroes and very M\=ay\=avis; so, deluded by their greed and jealousy, they quarrelled amongst each other.

Here ends the First Chapter in the Seventh Book on the beginning of the narrative of the Solar and the Lunar lines of kings in the Mah\=a Pur\=a\d{n}am \'Sr\={\i} Mad Dev\={\i} Bh\=agavatam of 18,000 verses by Mahar\d{s}i Veda Vy\=asa.



