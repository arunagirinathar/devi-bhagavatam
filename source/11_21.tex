\chapter{On G\=ayatr\={\i} Pura\'schara\d{n}am}

1-55. N\=ar\=aya\d{n}a said :-- Now I shall speak of the G\=ayatr\={\i}-pura\'schara\d{n}am. Hear. By its performance all the desires are obtained and all the sins are destroyed. On the tops of mountains, on the banks of the rivers, on the roots of Bel trees, on the edges of tanks, within the enclosures of the cows (cow-stalls), in temples, on the root of A\'svattha trees, in gardens, in the Tulas\={\i} groves, in the Pu\d{n}ya K\d{s}etrams (holy places), before one's Guru, or wherever the mind feels exalted and cheerful, and gets strength, the Pura\'schara\d{n}am if performed, lead to a speedy success. Before commencing the Pura\'schara\d{n}am of a mantra (the Pura\'schara\d{n}am means repetition of the name of a deity or of a mantra attended with burnt offerings, oblations, etc.,) first Pr\=aya\'schitta (penance) is done in the shape of repeating one million times the G\=ayatr\={\i} with the Vy\=arhitis. In any Vaidic Karma or in making Pura\'schara\d{n}am of the mantra of the Devat\=as Nrisi\d{n}ha, S\=urya, Var\=aha, etc., the first thing done is to repeat the G\=ayatr\={\i}. Without the japam of G\=ayatr\={\i}, no action is attended with success. The reason is this :-- Every Br\=ahma\d{n} is a \'S\=akta (a follower of \'Sakti); he cannot be a Vai\d{s}\d{n}ava or \'Saiva; for he is the worshipper of the Prime Force Vedam\=at\=a G\=ayatr\={\i}. Therefore obtain first the Grace of one's own \=Ista Devat\=a G\=ayatr\={\i} by Her Japam. Then worship the other Deities.

Thus one should purify one's j\=apya mantra (the mantra that is to be repeated) by first repeating one million times the G\=ayatr\={\i}; then one is to commence Pura\'schara\d{n}am. Again before purifying the mantra, one is to purify one's \=Atman (Self). In this purification of one's \=Atman three lakh times, in case of inability, one lakh times G\=ayatr\={\i} is to be repeated without one's \=Atman's purification, the Japam, Homa and other actions all become useless. This is specially noted in the Vedas. By Tapas e.g., Japam, Ch\=andr\=aya\d{n}a and Vrata, (asceticism) mortify your body. By offering Tarpa\d{n}am (peace-offerings) to the Fathers and the Devas, one can get self purification. If you want to get the Heavens and if you want to become great, practise Tapasy\=a. There is no other way. Tapasy\=a is the intent calling of the Mother, That Call which penetrates through and through the

Brahm\=anda. The K\d{s}attriyas should cross difficulties and dangers by force of arms; the Vai\'syas, by wealth; the \'S\=udras, by serving the twice born; and the Br\=ahma\d{n}as should cross difficulties and dangers, by Tapasy\=a, Homa, Japam, etc. So the Br\=ahma\d{n}as should always be cheerful and in prompt readiness to do Tapasy\=a. Of all sorts of tapasy\=as, mortifying the body by observing vows and fastings is the best. So say the \d{R}i\d{s}is. (This mortification of the body gives self-reliance and self intuition more surely and speedily than all the other studies and other practices.) The Br\=ahma\d{n}as should purify themselves by following duly Krichhra Ch\=andr\=aya\d{n}a vratas, etc., O N\=arada! Now I am speaking of the purification of food. Hear. The following four occupations of the Br\=ahma\d{n}as are the best :-- Ay\=achita, (without begging or asking for anything), Unchha, (the gathering in of handfuls of the corn left by the reapers), \'Sukla (the maintenance derived by a Br\=ahma\d{n}a from other Br\=ahma\d{n}as; a pure mode of life). And Bhik\d{s}\=u (begging). Whether according to the Tantras or according to the Vedas, the food obtained by the above four means is pure. What is earned by Bhik\d{s}\=a (begging) is divided into four parts :-- one part is given to the Br\=ahma\d{n}as; the second part is given to the cows; the third part is given to the guests, the fourth part is to be taken by him and his wife. Whatever is fixed for taking (swallowing) mouthfuls of food, that is to be taken on a tray or a platter. First throw a little cow-urine over that and count duly the number of mouthfuls. The mouthfuls are to be of the size of an egg; the house-holders are to take eight such mouthfuls and the V\=anaprasth\={\i}s are to take four such mouthfuls. The Brahmach\=arins can sprinkle their food with cow-urine nine times, six times, or three times as they like; while sprinkling, the fingers are to remain intact. The G\=ayatr\={\i} is to be repeated also. The food offered by a thief, Ch\=and\=ala, K\d{s}attriya or Vai\'sya is very inferior. The food of a \'S\=udra, or the companion with a \'S\=udra or taking food in the same line with a \'S\=udra leads one to suffer in the terrible hells as long as there are the Sun and Moon. The Pura\'schara\d{n}am of G\=ayatr\={\i} is repeating this twenty four lakh times (i.e., as many lakh times as there are syllables in the G\=ayatr\={\i}). But, according to Vi\'sv\=amitra, repeating thirty two lakh times is the Pura\'schara\d{n}am of G\=ayatr\={\i}. As the body becomes useless when the soul leaves the body, so the mantra without Pura\'schara\d{n}am is useless. The Pura\'schara\d{n}am is prohibited in the months of Jyaistha, \=A\d{s}\=adha, Pau\d{s}a and Mala (dirty) months. Also on Tuesday, Saturday; in the Vyat\={\i}p\=ata and Vaidhriti Yogas; also in Astam\={\i} (eighth), Navam\={\i} (ninth), Sasth\={\i} (sixth), Chaturth\={\i} (fourth) Trayoda\'s\={\i} (thirteenth), Chaturdas\={\i} (fourteenth) and Am\=av\=asy\=a (New Moon), Tithis (lunar days); in the evening twilight and in the night; while

the star Bhara\d{n}\={\i}, Krittik\=a, \=Ardr\=a, A\'sle\d{s}\=a, Jyesth\=a, Dhanisth\=a, \'Srava\d{n}\=a, or the Janma nak\d{s}atra (Birth time star) is with the Moon; while the signs Me\d{s}a, Karkata, Tul\=a, Kumbha, and Makara are the Lagnas (signs in the ascendant). When the moon and the start are auspicious, especially in the bright fortnight, the Pura\'schara\d{n}am performed, gives the Mantra Siddhi. First of all repeat Svasti v\=achan and perform duly the N\=andi mukha \'Sraddha and give food and clothing to the Br\=ahma\d{n}as. Take the permission of the Br\=ahma\d{n}as and begin the Pura\'schara\d{n}am. Where the \'Siva Lingam exists, facing west, or in any \'Siva temple, commence repeating the mantra. The other \'Siva K\d{s}ettrams are :-- K\=a\'s\={\i}, Ked\=ara, Mah\=a K\=ala, \'Sr\={\i} K\d{s}ettra, and Tryamvakam. These five are the Great K\d{s}ettrams, known widely on this earth, for the fructification and the siddhis of the Mantras. At all other places than these, the Karma Chakra is to be drawn according to the principles of the Tantra. And then they will be fit for Pura\'schara\d{n}am. The number of times that the Pura\'schara\d{n}am (the repeating of the mantra) is done on the first day, the same number is to be continued every day until completion; not greater nor less than that and also no intermission or stoppage should occur in the interval.

The repeating of the Mantra is to be commenced in the morning and should be done up to midday. While doing this, the mind is to be kept free from other subjects, and it is to be kept pure; one is to meditate on one's own Deity and on the meaning of the mantra and one should be particularly careful that no inaccuracies nor omissions should occur in the G\=ayatr\={\i}, Chhandas and in the repetition of the Mantra. One tenth of the total number of Pura\'schara\d{n}ams that are repeated is to be used for the Homa purpose. The Charu is to be prepared with ghee, til, the Bel leaves, flowers, jaya grain, honey and sugar; all mixed, are to be offered as oblations to the fire in the Homa. Then the success in the Mantra comes (i.e., mantra siddhi is obtained or the Mantra becomes manifested). After the Pura\'schara\d{n}am one should do properly the daily and occasional duties and worship the G\=ayatr\={\i} that brings in dharma, wealth, objects of desire and liberation. There is nothing superior an object of worship to this G\=ayatr\={\i}, whether in this world or in the next. The devotee, engaged in the Pura\'schara\d{n}am, should eat moderately, observe silence, bathe thrice in the three Sandhy\=a times, should be engaged in worshipping one's Deity, should not he unmindful and should not do any other work. He is to remain, while in water, to repeat the G\=ayatr\={\i} three lakhs of times. In case the devotee repeats the

mantra for achieving success in any other desired work (k\=amya karma), then he should willingly stick to it until the desired success is attained. Now is being told how to get success in ordinary K\=amya karmas. When the sun is rising, repeat the Puraschara\d{n}am mantra daily thousand times. Then one's life will be lengthened, no disease will occur, and wealth and prosperity will be obtained. If it be done this way, success is surely attained within three months, six months or at the end of one year. If the Homa (offering oblations to the fire) be offered one lakh times with lotuses besmeared with ghee (clarified butter), Mok\d{s}a (liberation) is attained. If, before the Mantra-Siddhi, or the success in realising the Mantra, is attained, one performs Japam or Homam for K\=amya Siddhi (to get certain desires) or mok\d{s}a, then all his actions become useless. If anybody performs twenty-five lakh Homas by curd and milk, be gets success (Siddhi) in this very birth. So all the Mahar\d{s}is say. By this the same result is attained that is got by the aforesaid means, i.e., by the eight-limbed Yoga, whereby the Yogis become perfect.

He will attain Siddhi if he be devoted to his Guru and keep himself under restraint for six months only (i.e., practise Samyama) as regards taking food, etc., whether he be incapable or his mind be attached to other sensual objects. One should drink Pa\~ncha gavya (cow-urine, cow-dung, milk, curd, ghee) one day, fast one day, take Br\=ahma\d{n}a's food one day and be mindful in repeating the G\=ayatr\={\i}. First bathe in the Ganges or in other sacred places and while in water repeat one hundred G\=ayatr\={\i}s. If one drinks water on which one hundred G\=ayatr\={\i}s are repeated, one is freed from all one's sins. He gets the fruit of performing the Krichhra vrata, the Ch\=andr\=aya\d{n}a vrata and others. Be he a K\d{s}attriya King, or a Br\=ahma\d{n}a, if he is to remain in his own house, hold \=A\'srama and be engaged in performing Tapasy\=a then he will be certainly freed of all his sins. Be he a house holder or a Brahmach\=ar\={\i} or V\=anaprasth\={\i}, he should perform sacrifices, etc., according to his Adhik\=ara (or his rights) and he will get fruits according to his desires. The S\=agnik man (who keeps the Holy Fire) and other persons of good conduct and of learning and of good education should perform actions as prescribed in the Vedas and Smritis with a desire to attain Mok\d{s}a. Thus one should eat fruits and vegetables and and water or take eight mouthfuls of Bhik\d{s}\=anna (the food got by begging). If the Pura\'schara\d{n}am be performed this way, then the Mantra Siddhi is obtained. O N\=arada ! If the Pura\'schara\d{n}am be done with the mantra thus, his poverty is removed entirely. What more shall I say than this that if anybody hears this simply, his merits get increased and he attains great success.

Here ends the Twenty First Chapter of the Eleventh Book on G\=ayatr\={\i} Pura\'schara\d{n}am in the Mah\=apur\=a\d{n}am \'Sr\={\i} Mad Devi Bh\=agavatam of 18,000 verses by Mahar\d{s}i Veda Vy\=asa.



