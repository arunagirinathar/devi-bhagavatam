\chapter{On the continuance of the family of Bhrigu}

1-3. Janamejaya said :-- ``Munis! How did the Bh\=argava wives cross this endless sea of troubles and how was the family of Bhrigu re-established on the surface of this earth? And what did the greedy Haihayas, the vilest of the K\d{s}attriyas, do after they killed the Bh\=argavas? Describe all these in detail and satisfy my curiosity. O Thou, Ocean of austerities! I am not satisfied with the drink of your nectar like

words, very holy and leading to happiness in this world and to good merits in the next.''

4-28. Vy\=asa said :-- O King! I will now narrate to you the sin destroying virtuous story how the Bh\=argava wives crossed their great hardships and the ocean of troubles, very difficult to cross. The Bh\=argava wives, when they were very much harassed by the Haihayas, went to the Him\=alay\=as, overwhelmed with terror and disappointment. There on that mountain they erected an earthen image of \'Sr\={\i} Gaur\={\i} Dev\={\i} by the banks of the Ganges and worshipped Her and, firmly resolved to die, began to fast. The Dev\={\i} Jagadambik\=a appeared to those religious women in their dreams and said :-- ``A son will be born of My essence to one of you from one of her thighs; that son will redress all your wants.'' Thus speaking, the Dev\={\i} Bhagavat\={\i} disappeared. Those women when they woke up were very glad; one of them that appeared very clever, becoming very much anxious out of the fear of the K\d{s}attriyas; preserved the foetus in one of her thighs for the propagation of the family. Her body became luminous; she then fled, overwhelmed with terror. The K\d{s}attriyas, seeing that Br\=ahma\d{n}\={\i}, came quickly upon her and said :-- ``See! This pregnant Bh\=argava wife is flying away hastily; seize her and take away her life.'' Thus saying, all of them raised their axes, and pursued her. Then that woman seeing them coming, wept out of fear. She cried, out of terror, for the preservation of the child in her womb; and the child seeing her mother helpless and distressed, trembling with fear and with tears in her eyes having no one to protect her and awfully oppressed by the K\d{s}attriyas as if a pregnant deer has been attacked by a lion and is crying about, angrily burst out of the thigh of his mother, and quickly came out like a second Sun. That good looking boy took away the power of sight of those K\d{s}attriyas by his brilliant lustrous light; no sooner the Haihayas saw that boy than they got blind. Like those that are born blind; they then began to roam in the caves of mountains and thought within themselves, what an evil turn of Fate had overtaken them! They thought thus :-- ``Oh! The mere sight of that boy has turned us blind; what a great wonder is this! Certainly this is due to the influence of the Br\=ahmi\d{n}\={\i} wife; this is, no doubt, the great effect of her virtue of chastity. We have greatly oppressed the Bh\=argava women. They have become very sorry and distressed; now we cannot tell what more evils do these women, of true resolve, inflict on us!'' Thus pondering, those K\d{s}attriyas
deprived of their eyes, helpless, and their minds bewildered, took refuge of those Br\=ahmi\d{n} ladies. The ladies, seeing them again come, were the more terrified; but those K\d{s}attriyas bowed down before them with

folded hands for the restoration of their sights and said :-- ``O Mother! We are your servants. Be gracious unto us. O Auspicious Ones! We are vicious K\d{s}attriyas; O Mothers! What an amount of offence we have committed to you. O Beautiful Ones! We have become blind, no sooner we have seen you. O Angry Ones! No more we can see your lotus-like faces, as if we are born blind; O Mother! The spirit of your asceticism is so very wonderful! We are sinners; therefore by no means we can get our sight; therefore we have taken refuge unto you all; better give us back our eyesight and preserve our honour. O Mother! Blindness is more dreadful than death; therefore do you show your mercy on us. Be pleased unto us and restore our eyesights and make us your slaves; no sooner we get back our sights, we will cease from these vicious acts and go to our homes. In future, we will never commit such heinous acts; from today we all become servants of the Bh\=argavas and we will serve them. Forgive all our sins that we committed unconsciously; we promise that, in future, there will no more be any enmity between the Bh\=argavas and K\d{s}attriyas. O good-looking Ones! You pass your days happily with your sons; we ever bow down before you. O Auspicious Ones! Be graciously pleased unto us; no more we will cherish any inimical feelings towards you.''

29-44. Vy\=asa said :-- O King! The Bh\=argava lady heard their words and was thunderstruck and seeing those K\d{s}attriyas bowing down before her, blind and distressed, consoled them and said, ``O K\d{s}attriyas! I have not taken away your sights nor am I displeased in any way with you. Now hear what is the real cause. This child of Bh\=argava, born of my thigh, is exceedingly angry towards you and has therefore made your eyesight still and to no purpose. For the greed of wealth, you have slain the close relatives of this boy, those that were quite innocent and virtuous ascetics and you have slain their children that were in their mother\'s wombs; this boy has come to know all those things. O children! When you were slaying the children of the Bh\=argavas in their mother\'s wombs, I then bore within my thighs this child for one hundred years. This son of mine though as yet in the womb, has mastered all the Vedas within so very short a time for the propagation of the Bh\=argava clan. Now this Bh\=argava son is infuriated with anger for your slaying his father and is now ready to kill you all. My son! Whose divine effulgence has destroyed your eyesights, is born of grace of the Highest Goddess, the Bhagavat\={\i} Bhuvane\'svar\={\i}; therefore do not consider this boy as an ordinary being. Now bow down with humility before this my son Aurvya (born from the thighs); this son may be pleased by your bowing down and may restore you your eyesights.

Vy\=asa said :-- O King! Hearing thus the words of the Br\=ahmi\d{n} lady, the Haihayas began to praise the boy with hymns. With great humility, they bowed down to the best of the Munis, born of the thighs. The \d{R}i\d{s}i Aurvya, then, became pleased and spoke thus to the Haihayas who were deprived of their eyesights :-- ``Better go back to your own homes. O Kings! And read these following words derived from my this story. Whatever is inevitable and created by the hands of gods must come to pass. Knowing this, no one ought to be sorrowful on any such matters. Let you all regain your eyesights as before and forego your anger and go to your own homes respectively at your own will. Let the \d{R}i\d{s}is, too, get peace and happiness as before.'' When the Mahar\d{s}i Aurvya ordered thus, the Haihayas got back their eyesights and went at their leisure to their own homes; on the other hand the Br\=ahmi\d{n} lady went to her own hermitage, with her Divine-spirited child and began to nourish him. O King! Thus I have described to you the story of the killing of the Bh\=argavas and how the K\d{s}attriyas, actuated by greed, did so very vicious acts.

45-48. Janamejaya said :-- ``O Ascetic! Hearing this exceedingly heart-rending act of the K\d{s}attriyas, I come to know, that greed is the sole cause of it and both the parties had suffered so much, simply out of this insatiable greed. O King of Munis! I want to ask you one more question in regard to this point. How the sons of the Kings came to be known Haihayas in this world? Amongst the K\d{s}attriyas, some are called Y\=adavas for they ware descended from the family of Yadu; some were known as Bh\=arata, for they were descended from Bh\=arata. But was some king named Haihaya born before in their family or were they known as such on account of other actions? I desire to hear of it. Kindly describe
this to me and oblige.''

49-56. Vy\=asa said :-- O King! I am describing in detail to you of the origin of the Haihayas. Hear. The sins are destroyed and the merits accrue on hearing this story. O King! Once on a time Revanta, the son of the Sun, very beautiful and of boundless lustre, was going to Vi\d{s}\d{n}u in Vaikuntha, mounted on the beautiful Uchchai\'srava, the jewel of the horses. When he was going on horseback with a desire to see the God Vi\d{s}\d{n}u, the Goddess Lak\d{s}m\={\i} saw that child of the Sun. The Goddess Lak\d{s}m\={\i}, born out of the churning of the ocean, on looking at the beautiful appearance of her brother Horse, also born out of the churning of the ocean, became very much astonished and steadily gazed on him. The Bhagav\=an Vi\d{s}\d{n}u, capable to show both favour and disfavour, saw the beautiful Revanta, of good figure, coming on horseback; and lovingly asked Lak\d{s}m\={\i} :-- ``O Beautiful One! Who is coming here on horseback, as

it were, enchanting to the three worlds!'' At that time, the Goddess Lak\d{s}m\={\i} was accidentally looking intently on the horse; so she did not reply, though repeatedly asked by the Bhagav\=an.

57-68. The Lak\d{s}m\={\i} Dev\={\i}, always restless, was very much intent on the horse and was enchanted and She was looking steadily with great affection on the horse. Seeing this, the Bhagav\=an became angry and said :-- ``O Beautiful-eyed One! What you are looking at so intently? Are you so much enchanted with the sight of the horse that you are not speaking to me a single word, though I am repeatedly asking you so often! You lovingly dwell on all the objects; hence your name is Ram\=a; your mind is also very restless, therefore you would be known as Chanchal\=a Dev\={\i} (the restless Dev\={\i}). O Auspicious One! You are restless like ordinary women; you can never stay steadily for a certain time at any one place. While sitting before Me, you are enchanted with the sight of a horse; then you be born as a mare in that world of men, full of dreadful troubles, on the surface of the earth. The Goddess Lak\d{s}m\={\i} became very much affrightened at the sudden curse given by Hari, a matter as it were ordained by the Devas, and began to cry aloud, shuddering with pain and sorrows. Lak\d{s}m\={\i} Dev\={\i}, then of sweet smiles, frightened, bowed down with great humility to her own lord N\=ar\=aya\d{n}a and said thus :-- O Deva of the Devas! O Govinda! You are the Lord of this world and the Ocean of mercy. O Ke\'sava! Why have you inflicted on me so dreadful a curse for such a minor fault of mine! O Lord! I never saw you before so very angry; Alas! Where has now gone that affection, so natural and undying, that you showed towards me! O Lord! It is not proper to hurl a thunderbolt on one's own relations; but it is advisable to cast it on the enemies. I am always fit for receiving boons from you. Why have you made me now an object, fit for curse. O Govinda! I will quit this life in Your presence. I will never be able to live, separated from You. O Lord! Be graciously pleased and say when shall I be free from this dreadful curse and regain Your happy companion?

69. The Bhagav\=an said :-- ``O Dev\={\i}! When you will get a son in the world like me, you will no doubt come again to be my companion.''

Here ends the Seventeenth Chapter on the continuance of the family of Bhrigu in the Sixth Book in the Mah\=apur\=a\d{n}am, \'Sr\={\i} Mad Dev\={\i} Bh\=agavatam of 18,000 verses by Mahar\d{s}i Veda Vy\=asa.



