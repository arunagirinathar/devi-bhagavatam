\chapter{On the description of the receptacle of beings and on the mountains and on the origin of rivers}

1-31. \'Sr\={\i} N\=ar\=aya\d{n}a said :-- O Child N\=arada! Now hear in detail about the divisions of the earth into the Dv\={\i}pas and the Var\d{s}as as marked out by the Devas. In brief, I describe about them; no one can speak about this in details. First, the Jambu Dv\={\i}pa is one l\=akh Yoyanas in its dimensions. This Jambu Dv\={\i}pa is round like a lotus. There are nine Var\d{s}as in it and excepting the Bhadr\=a\'sva and Ketum\=ala, each is nine thousand Yoyanas in its dimensions (i.e., in its diameter or circumference?) and there are eight very lofty mountains, in those Var\d{s}as, forming their boundaries. Of the Var\d{s}as, the two Var\d{s}as that are situated in the North and South, are of the size of a bow (segmental); and the four others are elongated in their size. The centre of all these Var\d{s}as is named \=Il\=avrita Var\d{s}a and its size is rectangular. In the centre of this \=Il\=avar\d{s}a is situated the golden Sumeru Mountain, the King of all the mountains, one lakh Yoyanas high. It forms the pericarp of the lotus earth. The top of this mountain is thirty Yoyanas wide. O Child! The sixteen thousand Yoyanas of this mountain is under the ground and the eighty four Yoyanas are visible outside. In the north of this \=Il\=avar\d{s}a are the three mountains the N\={\i}lagiri, the \'Svetagiri and the \'Sringavau, forming the boundaries respectively of the three Var\d{s}as named Ramyaka, Hira\d{n}maya and Kuru respectively. These run along from the east and gradually extend at their base and towards the salt ocean (Lavana Samudra).

These three mountains, that form the boundaries, are each two thousand Yoyanas wide. The length of each from the east towards north is less by one-tenth (1/10) of the above dimensions. Many rivers take their source and flow from them. On the south of \=Il\=avar\d{s}a three beautiful mountain ranges, named Ni\d{s}adha, Hemak\=uta, and Him\=alay\=as, are situated, extending from the east. They are each one Ayuta Yoyanas high. These three mountains form the boundaries again of Kimpuru\d{s}a and Bh\=arata Var\d{s}a.

To the west of \=Il\=avrita is situated the mountain called M\=alyav\=an and to the east are situated the mountains Gandham\=adan, N\={\i}la, and Ni\d{s}adha, the centres of the highest sublime grandeur and beauty. The length and breadth of these the boundary (limiting) mountains are each two thousand Yoyanas. Then the mountains Mandara, Sup\=ar\'svak, and Kumuda and others are situated in the Ketum\=ala and Bhadr\=a\'sva Var\d{s}as; but these all are reckoned as the P\=ada Parvatas (mountains at the foot) of the Sumeru mountain. The height and breadth of each of these is one Ayuta Yoyanas. These form the pillars, as it were, of Meru on the four sides. On these mountains, the mangoe, the jack, plantain, and the fig trees and various others are situated, four hundred (400) Yoyanas wide and eleven hundred (1,100) Yoyanas high; they seem to extend to the Heavens and form, as it were, the flagstaffs on the top. The roots, bases of these trees as well as their branches are wonderfully equally thick and extend to enormous distances. On those mountain tops are situated again, the four very capacious lakes. Of these, one lake is all milk; the other lake is all honey; the third lake is all sugarcane juice and the fourth lake is all sweet water. There are, then, again the four very lovely gardens named Nandana, Chaitrarath, Vaibhr\=ajaka, and Sarvatobhadra, very lovely, enchanting and pleasing to the delicate female sex and where the Devas enjoy the wealth and prosperity and their other Yogic powers. Here the Devas live always with numerous hordes of women and have their free amorous, dealings with them, to their heart's contents and they hear the sweet songs sung by the Gandharbas and Kinnaras, the Upa Devat\=as about their own glorious deeds. On the top of the Mandara mountain, there are the Heavenly mangoe trees eleven hundred Yoyanas high; the sweet delicious nectarlike mangoe fruits, very soft and each of the size as the summit of a mountain, fall to the ground; and out of their juices of a colour of the rising sun, a great river named Aru\d{n}od\=a takes her origin. Here the Devas always worship the great Dev\={\i} Bhagavat\={\i} named Aru\d{n}\=a, the Destructrix of all sins, the Grantrix of all desires, and the Bestower of all fearlessness with various offerings and with the lovely water of this Aru\d{n}od\=a river, with great devotion. O Child! In ancient days, the King of the Daityas worshipped always this Mah\=a

M\=ay\=a Aru\d{n}\=a Dev\={\i} (and obtained immense wealth and prosperity). He who worships Her becomes cured of all diseases, gets his health and other happiness by Her grace. Therefore She is named \=Ady\=a, M\=ay\=a, Atul\=a, Anant\=a, Pust\={\i}, \=I\'svaram\=alin\={\i}, the Destroyer of the wicked and the Giver of lustre and beauty and thus remembered on this capacious earth. The river J\=amb\=unada has come out, as a result of Her worship, containing divine gold.

Here ends the Fifth Chapter of the Eighth Book on the description of the receptacle of beings and on the mountains and on the origin of rivers in the Mah\=a Pur\=a\d{n}am \'Sr\={\i} Mad Dev\={\i} Bh\=agavatam, of 18,000 verses, by Mahar\d{s}i Veda Vy\=asa.



