\chapter{On the merits of the Dev\={\i} in the story of Satyavrata}

1-5. Loma\'sa said :-- O Munis! Utatthya, the son of Devadatta, was quite ignorant of anything of the Vedas, Japam (muttering of mantra), meditation of the deity, worship of the Devas, \=Asana (Posture), Pr\=a\d{n}\=ay\=ama (withholding the breath by way of religious austerity), Praty\=ah\=ara (restraint of mind), Bh\=uta\'suddhi (purification of the elements of the body by respiratory attraction and replacement), mantra (a mystical formula regarding some deity), K\={\i}laka (chanting of a mantra to serve as a pin of protection), G\=ayattr\={\i} (the famous mantra of the Br\=ahmi\d{n}s), Saucha (cleanliness, external and internal), rules how to bathe, \=Achamana (sipping of water and reciting mantrams before worship), Pr\=a\d{n}\=agnihotra (offering of oblations to the fire of Pr\=a\d{n}a or to the fire of life), the offering of a sacrifice, hospitality, Sandhy\=a (the morning, mid-day or evening prayer), collecting fuels for oblations, and offering of oblations. Daily he rose in the morning and somehow rinsed his mouth and washed his teeth and bathed in the Ganges river without any mantrams (like a S\=udra).

6. That stupid fellow ate indiscriminately, did not know what to eat and what not to eat. During the mid-day he collected the fruits from the forest and used to eat them.

7. But he always spoke truth while he stayed there; never did he say any untruth. The people of that place, seeing this, named him Satyatap\=a.

8-9. That Utatthya did no good or bad to anybody; he slept peacefully and blissfully; but be used to think when he would die; thus his troubles would be ended; he felt that the life of an illiterate Br\=ahma\d{n} is a curse; his death would be a better alternative.

10. He used to think thus :-- Fate has made me a fool; I do not find any other cause for it. Oh! I got the exceedingly good birth amongst men; but all this has been rendered in vain by Fate.

11. Oh! As a fair woman, if barren, a cow if giving no milk, and a tree without any fruits are all useless, so Fate has rendered my life, too, quite useless.

12. Why am I cursing Fate? This is all the fruits of my past Karma. In my previous life I never wrote a book and presented to a good Br\=ahmi\d{n}; hence I am illiterate in this birth.

13. In my former birth I did not impart any knowledge to my favourite pupils; hence I am wicked and a cursed Br\=ahmi\d{n} in this birth.

14. I never performed any religious asceticism in any holy place, I did not serve the saints, I never worshipped the Br\=ahmi\d{n}s with any offerings. For all these reasons I am now born of perverted intellect in the present birth.

15. Many a son of the Munis have learnt the meanings of the Vedas and the \'S\=astras; and I am whiling away my time thus in a quite illiterate condition by some wretched combinations of incidents.

16. I do not know how to perform Tapasy\=a; what is the use, then, of my attempting to do so? I am of very bad luck, and thus my good resolve will not be crowned with success.

17. I consider Fate to be the strongest of all; Fie on one's own prowess! For actions done with effort and hard labour are frustrated entirely by Fate.

18. Time can never be overstepped; See! Brahm\=a, Vi\d{s}\d{n}u, Rudra, Indra, and others are all under the influence of the Great Time.

19. O \d{R}i\d{s}is! Thus arguing in his mind, that Br\=ahmi\d{n} son Utatthya stayed there in that hermitage on the bank of the holy Ganges.

20. And gradually he became thoroughly unattached to all the things and, being peaceful, passed away his time in that forest without any habitations and men, with great difficulty.

21. Thus passed away fourteen years in that forest where the Ganges was flowing. Still he did not learn how to worship the Supreme Deity, how to make Japam, nor did he learn any mantrams. Simply he lived there and whiled away his time.

22. People surrounding that place knew this much only of him that this Muni spoke truth only and hence his name was Satyavrata. This one name made him celebrated that he is Satyavrata; never did he say any untruth.

23. Once on a time, a hunter named Ni\d{s}\=ada, exceedingly clever in hunting, came accidentally with bows and arms in his hands, while hunting a deer in that wide forest. He looked like a second God of Death (Yama) and seemed to be very cruel.

24. That savage mountaineer, drawing his bow so as to touch the ear, pierced a boar with his sharp arrows. The boar, being very much terrified, fled with enormous rapidity to the Muni Satyavrata.

25. On seeing the distressed condition of the boar trembling with fear and his body besmeared with blood, the Muni was moved with mercy.

26. While the boar, pierced with arrows and besmeared with blood, was running away in front of him, mercy took possession of the Muni, therefore the Muni began to tremble and agreeably to the human nature exclaimed ``Ai'' ``Ai'' (go to that direction), the seed mantram of the Goddess of learning with ``m'' left out (Aim, Aim).

27. That illiterate Br\=ahmi\d{n} son never heard before that ``Ai'' was the seed mantram of the Sarasvat\={\i} Dev\={\i}; nor did he come to know of it by any other means. Accidentally it came out of his mouth, and he uttered. And afterwards that Mah\=atm\=a seeing the boar's distressed condition was merged in deep sorrow.

28. The boar entered trembling into the Muni's hermitage very much distracted and being very much pained with arrows. Being unable to find any other way the boar hid himself in the dense bushes.

29. Instantly there appeared then, before the Muni, the terrible savage hunter, like a second God of Death, with string stretched to his ear, in pursuit of that boar.

30-33. On seeing the Muni Satyavrata sitting there alone and silent on the Ku\'sa grass seat, the hunter bowed down to him and asked ``O Br\=ahmin! Whither has that boar gone. I know very well everything about you that you never speak untruth; therefore I am enquiring about the boar pierced by my arrows. My family members are all very hungry; and to feed them, I am come out in this hunting. This is the living, ordained by the Fate; I have got no other means of maintaining the livelihood of my family. This I speak truly to you; whether it is bad or good, I will have to maintain my family with it. O Br\=ahma\d{n}! You are famous as Satyavrata; my family members are starving; kindly reply quickly where that boar has gone?''

34. Thus asked by the hunter, the Mah\=atm\=a Satyavrata was merged in an ocean of doubt; he began to argue ``If I say I have not seen the boar then my vow to speak the truth will certainly be broken.

35. The boar struck with arrows has gone this way, it is true. How can I tell a lie? Again this man is hungry and is therefore asking, he will instantly kill the boar no sooner he finds him. How then can I speak truth?

36. Where speaking out the truth causes injury and the loss of lives, that truth is no truth at all; moreover, even untruth, when tempered with mercy for the welfare of others, is recognised as truth. Really speaking, whatever

leads to the welfare of all the beings in this world, that is truth; and every thing else is not truth.

37. O Jamadagni! Thus placed between the horns of a religious dilemma what shall I do now so as to meet both the ends -- to save the life of the boar, to do the welfare, as well as not to speak untruth.''

38. When Satyavrata saw the boar wounded by the arrow of the hunter, he, moved with pity, uttered the seed mantra of the Goddess of Learning; and now that most auspicious Goddess, on account of his uttering Her seed mantram, was very pleased and gave him the knowledge, difficult to be attained otherwise.

39. The door of all his knowledge opened out at once, and he became at once instantly the seer, the poet like the ancient Muni V\=almik\={\i}.

40. Then that religiously disposed, merciful Br\=ahma\d{n}, aiming at Truth, addressed that hunter before him with bows in his arms, thus :--

41. That force which sees (as witness) never speaks; and that force which speaks, never sees. O hunter! Why are you asking me repeatedly, impelled by your own selfish desire?

42. The hunter, the killer of the animals, on hearing this was disappointed in the matter of finding out the boar and went back to his home.

43. That Br\=ahmi\d{n} turned out a poet like Varu\d{n}a and he became celebrated as Satyavrata, the speaker of truth, in all the worlds.

44. He began to recite the Satyavrata mantram duly, and, by its influence, became a Pundit, rivalled by none in this world.

45. During every festival the Br\=ahma\d{n}s chanted his praise and the Munis used to narrate his story in detail.

46. On hearing his fame spreading all around, his father Devadatta who forsook him before, recalled him to his hermitage and took him again in his family with great honour and affection.

47. Therefore O King! You should always worship and serve that Great Goddess, the Prime Energy, the Cause of all this Universe.

48. O King! With due Vedic rites you perform that sacrifice to that Goddess which will surely yield results at all times and all desires. I already spoke to you about this.

49. That Great Goddess is known as K\=amad\=a (the giver of all desires); for She grants all desires when men with devotion remember Her, worship Her, take Her name, meditate Her and eulogise Her.

50-56. O King! The wise sages ought to see the persons diseased, distressed, hungry, those without any wealth, the hypocrite, the cheat, the afflicted,

the sensual, the covetous, the incapable, always suffering from mental troubles; again those who are wealthy with their children and grand-children, prosperous, healthy, with enjoyments, versed in the Vedas, literary, kings, heroes, those who command over many, those attended with relations and kinsmen and endowed with all good qualities; and then judge for themselves that those people did not worship the Goddess and therefore they were sufferers and these people worshipped the Goddess and hence they were happy in this world.

57. Vy\=asa said :-- Thus I heard from the mouth of Loma\'sa Muni, in assembly of the sages, the good merits of the Great Goddess.

58. O King! Consider all these and you will find that the Highest Goddess, the Bh\=agavat\={\i} is to be worshipped always with devotion and unselfish love.

Here ends the Eleventh Chapter on the merits of the Dev\={\i} in the story of Satyavrata in the Third Skandha of the Mah\=a Pur\=a\d{n}am \'Sr\={\i} mad Dev\={\i} Bh\=agavatam of 18,000 verses by Mahar\d{s}i Veda Vy\=asa.



