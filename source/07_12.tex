\chapter{On the description of Va\'sistha's curse on Tri\'sanku}

1-6. Vy\=asa said :-- O King! Thus giving the advice to his son, the King Tri\'sanku was excited with feelings of love and, in a choked voice, said to his father that he would fulfil what he had been ordered. The King then called the Br\=ahmi\d{n}s, versed in the Vedas and Mantrams, and had all the materials for installation collected quickly. He brought the waters from all the sacred places of prigrimages; he then called together with great respect all the kings. On a sacred day, the father installed his son on the throne and gave him, in accordance with due rites and ceremonies, the royal throne. The King then adopted with his wife the third V\=anaprastha stage of life and practised a severe tapasy\=a on the

banks of the Ganges. Then in due course of time the King went to the Heavens. There he began to shine like a second Sun by the side of Indra, respected by all the gods.

7-10. Janamejaya said :-- ``O Bhagav\=an! You spoke before in course of conversation that Satyavrata was cursed by Va\'sistha on the killing of his cow to become a Pi\'s\=acha; how then he got himself freed of this curse. There is a doubt on this point. Kindly clear it and oblige. Satyavrata was cursed; hence pronounced unfit to succeed to the throne. How was the Muni, by what actions, was he freed of the curse? How could the father bring back to his home his son of the form of a Pi\'s\=acha? O Viprar\d{s}i! Kindly narrate to me how the Muni was freed of his curse.''

11-18. Vy\=asa said :-- Cursed by Va\'sistha, Satyavrata became then and there transformed into a Pi\'s\=acha, very ugly, violent and terrible to all; but when he worshipped the Dev\={\i} with devotion, immediately the Dev\={\i} gave him a beautiful divine body. By the grace of the Dev\={\i}, his sins were all washed away and his Pi\'s\=acha form vanished. Satyavrata, then, freed from his sins became very much vigorous and energetic. Va\'sistha also became pleased with him, blessed thus by the Supreme Force and so was his father, too. When his father died, the virtuous Satyavrata became King, governed his subjects and performed various sacrifices and worshipped, too, the Eternal Mother of the Gods. O King! Tri\'sanku had a very beautiful son born to him, named Hari\'schandra, endowed in all his limbs with auspicious signs. The King Tri\'sanku wanted to make his son Yuvar\=aja (the Crown prince) and then in his that very body while living, enjoy the Heavens. The King went to the \=A\'srama of Va\'sistha and gladly asked him, with folded palms, bowing down before him duly.

19-23. O Ascetic! You are the son of Brahm\=a, versed in all the Vaidik Mantrams; so you are exceedingly fortunate; now I beg to inform you one thing; hear it gladly. I now desire to enjoy the happiness of the Heavens and all the enjoyments of the Devas, while I am in this body. To enjoy in the Nandana Garden, to live with the Apsar\=as and to hear the sweet music of the Devas and the Gandharbas, these ideas now have taken a strong hold of my heart. Therefore, O Great Muni! Engage me in such a sacrifice as will enable me, in this very body to live in the Svarloka. O Muni! You are fully competent to do this; therefore be ready for this. Have the sacrifice done and let me have quickly the Devaloka, so difficult to be obtained!

24-26. Va\'sistha said :-- ``O King! It is exceedingly hard to live in the Heavens while in this mortal body. The departed only live in the

Heavens by their merits, this is a known fact. Therefore, O Omniscient One! Your desire is hard to be attained. I am afraid of this. O King! The living men can hardly enjoy the Apsar\=as. Therefore, O Blessed One! Do the sacrifice first. Then, when you leave this body, you will go to the Heavens.''

27-31. Vy\=asa said :-- O King! The Mahar\d{s}i Va\'sistha was already angry with the King; therefore when he spoke these words, the King heard and became absent-minded. He again spoke to the Mahar\d{s}i :-- O Br\=ahma\d{n}a! If you do not allow me to do the sacrifice, on account of your haughtiness, I will have the sacrifice performed now by another priest. Va\'sistha became very angry at the words of the King and cursed him :-- ``O evilminded One! Be as soon as possible a Ch\=and\=ala in this body. You have committed acts by which your path to the Heaven is obstructed. You have stolen a Br\=ahmi\d{n}i's wife, and defiled the path of religion; you have killed the Surabhi Cow and you are a libertine. Therefore, O Sinner! Never you will go to the Heavens, even after your death.''

32-56. Vy\=asa said :-- O King! Hearing these harsh words from the Guru, Tri\'sanku became immediately Ch\=and\=ala in that very body. His golden earrings became turned into iron; the sweet sandal smell over his body smelled like faeces; his beautiful yellow clothings became blue, the colour of his body became like that of an elephant, due to his curse. O King! Those who are the worshippers of the Supreme Force can produce such things when they are angry; there is not the slightest doubt in this. Therefore one ought never to insult any devotee of the Supreme Force. The Muni Va\'sistha is always engaged in repeating silently the G\=ayatr\={\i} of the Dev\={\i}. So what wonder is there that the body of the King will be reduced to such a wretched state by his rage. The King Tri\'sanku became very sorry to see his ugly body; he did not go home; rather he remained in the forest in that form and poor dress. He began to think, distressed with sorrow and over-powered with misery :-- ``My body is now blameable to the extreme, so what to do and where to go in this wretched state! I find no remedy to exhaust all my sufferings. If I go home, my son will be, no doubt, very much pained with sorrow. My wife, when she will see my Ch\=and\=ala appearance, she won't accept me; my ministers will not regard me as they used to do before. My friends and relations, when they will come to me, will not serve me with the former care. So it is far better to die than to live, thus despised. I will drink poison or drown myself in waters or hang myself. Or I will burn myself in the funeral pyre duly or I will quit this blameable life by starvation. But, Alas! I will be guilty of

suicide; so again due to this sin I will be born a Ch\=and\=ala and I will be again cursed.'' Thus thinking, the King again thought that at present he ought not to commit suicide by any means. ``I will have to suffer for my Karma; and, after due suffering, this Karma will be exhausted. So I will suffer in this forest for my Karma in this my body. Without the enjoyment of the fruits, the past actions can never die out; therefore all actions done by me, auspicious or inauspicious, I will enjoy or suffer in this place. Always to remain close to a holy \=A\'srama, to wander in holy places of pilgrimage, to remember the Dev\={\i} Ambik\=a, and to serve the saints will now be my duties. Thus I will no doubt exhaust all my actions, residing in this forest; then, if chance permits, and if I meet with a saintly person, all my intentions will be crowned with success.'' Thus thinking, the King quitting his city went to the banks of the Ganges and repenting very much, remained there on the Ganges. The King Hari\'schandra came to know the cause of his father's curse and with a sorrowful heart sent ministers to him. Like a Ch\=and\=ala, the King was respiring frequently; at this time the ministers went to him and bowing humbly, said : -- O King! Your son has ordered us to come here; we have come at his command; we are the ministers of the King Hari\'schandra. Know this verily, O King! Kindly hear what the Crown Prince has said :-- ``Go and bring my Father here without any delay.'' Therefore, O King! Cast aside your mental agonies and come to the city. The ministers, the subjects all will be always at your service. We will all try our best to please Va\'sistha, so that he may favour you. And that greatly illustrious Muni being pleased will certainly remove your sorrows quickly. O King! Thus your son has spoken to us many words; so now be pleased to go to your own abode.

57-64. Vy\=asa said :-- O King! That Ch\=and\=ala-like King, hearing even their words thus, did not consent to go back to his house. Rather he told them :-- ``Ministers, go back, all of you to the city; and at my word, tell my son that I won't go back to my house. Better leaving off all idleness, you better govern the Kingdom carefully. Shew your respect specially to the Br\=ahmi\d{n}s and perform various sacrifices and worship the Devas. I do not like in this blameable Ch\=and\=ala form to go to the city of Ayodhy\=a with the high-souled ones; so you all go back to Ayodhy\=a without any further delay. Install, at my order, my powerful son Hari\'schandra on the throne and do all these stately duties.'' When the ministers heard thus the King ordering them, they began to cry very much, and, bowing down, they went away early out of

the hermitage. On coming back to Ayodhy\=a they regularly installed on a sacred day the King Hari\'schandra with Abhi\d{s}eka water, purified with Mantrams. Thus the powerful virtuous Hari\'schandra, on being installed on the royal throne by the command of the King, remembered always his father and began to govern his Kingdom with his ministers according to the dictates of Dharma.

Here ends the Twelfth Chapter of the Seventh Book on the description of Va\'sistha's curse on Tri\'sanku in the Mah\=a Pur\=a\d{n}am \'Sr\={\i} Mad Dev\={\i} Bh\=agavatam of 18,000 verses by Mahar\d{s}i Veda Vy\=asa.



