\chapter{On Tri\'sir\=a's austerities}

1-12. The \d{R}i\d{s}is (of the Naimi\d{s}a forest) addressed S\=uta (fondly) :-- O highly Fortunate One! Your nectar-like words are very sweet. We are not satiated with what you have described to us as the auspicious sayings of Dvaip\=ayana Vy\=asa. O S\=uta! We desire to ask you again to narrate to us the auspicious sayings of this Pur\=a\d{n}a, beautiful, famous, and sin-destroying and authorised by the holy Vedas. Vi\'svakarm\=a had a son, named Vritr\=asura, who was very well known, and very powerful. How was it that he had been slain by the high-souled Indra? Vi\'svakarm\=a was a powerful Br\=ahmi\d{n} and belonged to the god\'s party; his son was stronger. How was it that he had been killed by Indra! The Devas are born of the Sattva qualities; men are born from the R\=ajasic qualities; and all the birds, etc., are born of the T\=amasic qualities. This is the opinion of the Pundits, versed in the Pur\=a\d{n}as and \=Agamas. But in this act of slaying Vritr\=asura, a great contradiction arises; for the powerful Vritra was killed merely under a pretext by Indra, the performer of the hundred sacrifices, and endowed with Sattva qualities. And Indra was prompted to do so by Vi\d{s}\d{n}u, the head of those who possess Sattva qualities; while Vi\d{s}\d{n}u himself entered in disguise into the thunderbolt so that he could kill Vritra. The powerful Vritra entered into a treaty and kept himself peaceful when Indra and Vi\d{s}\d{n}u violated truth and treacherously killed him by Jalaphena (the watery foams). O S\=uta! The great wonder is this :-- That Indra and Vi\d{s}\d{n}u turned out so bold as to forsake the truth. This, then, is therefore very clear that the high souled persons become deluded and act sinfully. The Heads of the Devas act very wrongly; they are reckoned as polite simply because they observe the mere outward forms of good conduct as approved by the \'S\=astras. How can the mere observance of outward forms constitute politeness? Had Indra, who killed in disguise Vritra relying on his words, to suffer any punishment for the sin that he incurred in killing a Br\=ahma\d{n}a? It was told by you before that Vritra had been slain by the Dev\={\i} Bhagavat\={\i}; but the general belief is that Indra killed him. Our minds are puzzled on this point. (So clear our doubts on this point.)

13-14. S\=uta said :-- O Munis! Hear the incident of the killing of Vritr\=asura and the punishment that Indra had to suffer due to his sin of Brahmahaty\=a (killing a Br\=ahmi\d{n}). This question was asked by the King P\=arik\d{s}it and replied by Vy\=asa, the son of Satyavat\={\i}. I will tell you what Vy\=asa had told before.

15-18. Janamejaya asked :-- O Best of Munis! How was it that in former days Indra, endowed with the Sattva qualities, killed Vritr\=asura, with the aid of Vi\d{s}\d{n}u? And how and why was it that he was killed again by the Goddess Bhagavat\={\i}? O Lord of Munis! How could one body be killed by the two; our curiosity has been excited to hear the truth. What man is there that does not like to hear any more of the glorious deeds of the high-souled persons! Kindly narrate to us the slaying of Vritra by the Dev\={\i} Bhagavat\={\i}.

19-26. Vy\=asa said :-- O King! You are blessed, since your taste to hear the events of Pur\=a\d{n}a has grown so much; the Devas even get their thirst for drinking nectar; but when quenched, they do not like to drink any more. O King! Your name and fame are widely spread. Your Bhakti (devotion) to the Pur\=a\d{n}as is growing more and more daily. A speaker gets very much delighted when his audience hears him with undivided attention. O Lord of the earth! The fight between Vritra and V\=asava that occurred in days of yore is famous in the passages of the Vedas and the Pur\=a\d{n}as; as well as the suffering that Indra had to encounter as his punishment when he had killed the innocent son of Visvakarma. O King! The Munis, who fear sin very much, commit yet blameable acts under M\=ay\=a; then what wonder is there that Vi\d{s}\d{n}u, and Indra would kill Tri\'sir\=a and Vritra merely under a plea. When Vi\d{s}\d{n}u, the incarnate of Sattva qualities, gets deluded by M\=ay\=a and kills deceitfully the Daityas always, then how can you expect any other man to conquer mentally even the Maha M\=ay\=a Bhav\=ani, Who deludes all the beings! O King! It is under the compulsion of this M\=ay\=a that the Bhagav\=an, the Infinite, the friend of Nara, N\=ar\=ayana, takes incarnations in thousands and thousands of Yugas in this Sams\=ara as Fish, etc., and does deeds sometimes lawful and sometimes unlawful. The Devas and men, being confounded by his M\=ay\=a, become upset and disordered and say ``that this body, wealth, house, sons, wife and relatives are all mine'' and being thus deluded sometimes do virtuous and sometimes sinful deeds. O King! There is not even one, on the surface of this earth, though he may be well versed in finding out cause and effect, the knowledge of the high and low, that can be free from this Great Delusion; he is from the very beginning tied up by the three Gu\d{n}as of this M\=ay\=a and that remains under Her control.

27-35. This explains that Vi\d{s}\d{n}u and Indra both were deluded by M\=ay\=a and engaged in fulfilling their own selfish ends. They killed Vritr\=a\'s\=ura under a pretext. O King! Hear! I am now describing to you the cause of enmity between lndra and Vritra. Vi\'svakarm\=a, the Praj\=apati, was great architect of the Gods, he was skilled, he was superior amongst the gods, a great ascetic and endeared by the Brahmi\d{n}s. He had enmity with Indra; and out of this enmity he created a son, very beautiful named him Tri\'siraska Visvar\=upa. That son had three faces very beautiful and lovely. Visvar\=upa performed three different functions with his three different faces; with one, he used to study the Vedas, with the second he used to drink nectar (wine), and with the third he used to see simultaneously all the directions. Tri\'sir\=a renounced the pleasures of the world and began to practise a hard tapasy\=a; he became a great ascetic, gentle, restrained in his passions and entirely devoted to his religion. He practised Panch\=agni-S\=adhan in the summer season, tying his feet upwards on the branch of a tree with his head downwards; he remained in dew in the cold season, under water in the winter season. Thus he abstained from food and conquered his self and, forsaking all the worldly connections, practised a very hard tapasy\=a; very difficult, indeed, for those who are of dull intellects.

36-49. Indra became very sad and dispirited to see him practise such a Tapasy\=a and thought of the means so that he might not acquire his Indraship. The P\=akas\=a\'sana Indra remained always very anxious see the energetic penance practised by that ascetic of unbounded glory and his steady attachment towards it. He thought thus :-- ``This Tri\'sir\=a is becoming stronger day by day by his penance, so he will kill me. The wise never look an enemy with indifference whose strength daily becomes greater and greater.'' It is now my urgent duty to invent means how to baffle his Tapasy\=a and he at last settled that lust is the great enemy of asceticism; the practice of devout austerities is destroyed complete by lust; so I must try this very day how the Muni becomes attached to worldly lust and enjoyments. The intelligent Indra, thinking thus, called the Apsar\=as Urva\'s\={\i}, Menak\=a, Rambh\=a, Ghrit\=ach\={\i}, and Tilottam\=a and others proud of their beauties so that they might seduce Tri'sir\=a, the son of Vi'svakarm\=a. O Apsar\=as! I have now got a very grave task to fulfil; all of you help me in this respect. A great enemy of mine, difficult to conquer, is practising penance with his self-controlled. Start at once and with your dress suited to various amorous gestures and try hard to seduce him. Be all well with you; seduce him and remove the fever of my heart. O Apsar\=as! What more shall I say, I am restless since I have heard of his strength performing such hard austerities. O Weak Ones ! That powerful

ascetic may acquire my place and thus dispossess me; this fear has possessed me. Therefore destroy my fear as quickly as possible. This is the task now given to you; get united and do this good to me. The Apsar\=as, hearing him, bowed down and said :-- ``O Lord of the Devas! Do not be afraid! We will try our best to seduce him. O highly Lustrous One! For the enticing away of the Muni, we will do all the things, dancing, music and other amorous gestures and practices, that will discard your fear. O King of the Gods! We will unsettle the mind of the Muni by our side glances and passionate gestures and postures, delude and tie him and then bring him under our control.''

50-60. Vy\=asa said :-- O King! Thus saying, the Apsar\=as went to Tri\'sir\=a and began to exhibit various amorous gestures and postures as stated in the K\=ama S\=astra. They began to sing sometimes, sometimes to dance in tune with musical measures before the Muni. In short, they practised various amorous gestures to entice him away. But that ascetic, blazing with the fire of Tapas, did not notice even the Apsara\'s various attempts; rather he kept all his senses under the control and remained like a deaf, dumb, and blind man. In that lovely hermitage of the Muni, the Apsar\=as sang and danced ravishingly and remained a few days there. But when they saw that the Muni Tri\'sir\=a did not swerve a bit from his meditative posture they returned tired, distressed to Indra and all, very fearful, began to address Indra with folded hands :-- ``O King! We tried our best and we could not in any way make the Muni unsteady, very hard to surmount. O P\=aka\'sasana! Please invent other means; we could not make the self-controlled Muni move away an inch from his position; it is our good luck that that high-souled Muni, an incarnate of blazing fire have not cursed us!'' Then dismissing the Apsar\=as, the evil-minded and dull Indra began to dev\={\i}se means, though totally unlawful, how to kill that good Muni. O King! That Indra abandoned all shame, and fear of sin and ultimately came to a highly blameable and sinful conclusion how to kill him.

Here ends the First Chapter of the Sixth Book on Tri\'sir\=a's austerities in \'Sr\={\i} Mad Dev\={\i} Bh\=agavatam the Mah\=a Pur\=a\d{n}am, of 18,000 verses by Mahar\d{s}i Veda Vy\=asa.



