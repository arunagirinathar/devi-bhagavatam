\chapter{On the description of Sandhy\=a Up\=as\=an\=a}

1-24. N\=ar\=aya\d{n}a said :-- Now I am speaking of the very holy Sandhyop\=asan\=a method of Sandhy\=a worship of G\=ayatr\={\i}, the Presiding Deity of the morning, midday and evening, and of the twice-born. Listen. The greatness of using Bhasma has been described in detail. No further need be stated on the subject. I shall talk, first of all, of the morning Sandhy\=a. The morning Sandhy\=a is to be done early in the morning while the stars are visible. When the Sun is in the meridian, the midday Sandhy\=a is to be performed; and while the Sun is visibly going down, the

evening Sandhy\=a is to be recited over. Now again, the distinctions are made in the above three Sandhy\=as :-- The morning Sandhy\=a with stars seen is the best; with stars disappeared, middling; and with the Sun risen above the horizon, inferior. So the evening Sandhy\=a, again, is of three kinds :-- best, middling, and inferior. When the Sun is visibly disappearing, the evening Sandhy\=a is the best; when the Sun has gone down the horizon, it is middling and when the stars are visible, it is inferior. The Br\=ahma\d{n}as are the root of the Tree, the Sandhy\=a Vandanam; the Vedas are the branches; the religious actions are the leaves. Therefore its root should be carefully preserved. If the root be cut, no branches or leaves of the tree will remain. That Br\=ahma\d{n}a who knows not his Sandhy\=a or who does not perform the Sandhy\=as is a living \'S\=udra. That Br\=ahma\d{n}a after his death verily becomes a dog. Therefore the Sandhy\=as must be observed every day. Otherwise no right comes at all to do any action. At the sunrise and the sunset the time for Sandhy\=a is two Dandas (48 minutes) and if Sandhy\=a be not done or rather neglected in the interval, the Pr\=aya\'schitta (penance) is to he paid duly (performed duly). If the proper time for Sandhy\=a expires, one more offering of Arghya is to be made in addition to the three Arghayas daily made or the G\=ayatr\={\i} is to be repeated one hundred and eight times before the Sandhy\=a is commenced. In whichever time any action ought to be done, worship, first of all, the Sandhy\=a Dev\={\i}, the Presiding Deity of that time and do the actions proper to that time afterwards. The Sandhy\=a performed in dwelling houses is ordinary; the Sandhy\=a done in enclosures of cows is middling and on the banks of the rivers is good and the Sandhy\=a performed before the Dev\={\i}'s temple or the Dev\={\i}'s seat is very excellent. The Sandhyop\=asan\=a ought to be done before the Dev\={\i}, because that is the worship of the very Dev\={\i}. The three Sandhy\=as done before the Dev\={\i} give infinitely excellent fruits. There is no other work of the Br\=ahma\d{n}as better than this Sandhy\=a. One can rather avoid worshipping \'Siva or Vi\d{s}\d{n}u; because that is not daily done as obligatory; but the Sandhyop\=asan\=a ought to be done daily. The G\=ayatr\={\i} of the Great Dev\={\i} is the Essence of all the mantras in the Vedas. In the Veda \'S\=astras, the worship of G\=ayatr\={\i} is most definitely pronounced. Brahm\=a and the other Devas meditate in the Sandhy\=a times on this Dev\={\i} G\=ayatr\={\i} and make a japam of that. The Vedas always make japams of Her. For this reason the G\=ayatr\={\i} has been mentioned as the object of worship by the Vedas. The Br\=ahma\d{n}as are called \'S\=aktas inasmuch as they worship the Primal \'Sakti (Force) G\=ayatr\={\i}, the Mother of the Vedas. They are not \'Saivas nor Vai\d{s}\d{n}avas.

Firstly make the ordinary \=Achaman three times, and, while inhaling, drink a little of the water of \=Achaman, repeating ``Om Ke\'sav\=aya Sv\=ah\=a, Om N\=ar\=aya\d{n}\=aya Sv\=ah\=a, Om M\=adhav\=aya Sv\=ah\=a.'' Then wash your two hands, repeating ``Om Gobind\=aya Namah, Om Vi\d{s}\d{n}ave Namah.'' Then by the root of the thumb rub the lips repeating ``Om Madh\=u s\=udan\=aya Namah, Om Trivikram\=aya Namah.'' So rub the mouth, repeating ``Om V\=amam\=aya Namah, Om \'Sr\={\i}dhar\=aya Namah.'' Then sprinkle water on the left hand, saying ``Om Hris\={\i}ke\'s\=aya Namah.'' Sprinkle water on the legs, saying ``Om Padman\=abh\=aya Namah.'' Sprinkle water on the head, saying ``Om D\=amodar\=aya Namah.'' Touch the mouth with the three fingers of the right hand, saying ``Om Samkar\d{s}a\d{n}\=aya Namah.'' Touch the nostrils with the thumb and forefinger saying ``Om V\=asudev\=aya Namah, Om Pradyumn\=aya Namah.'' Touch the eyes with the thumb and ring-finger, saying ``Om Aniruddh\=aya Namah, Om Puru\d{s}ottam\=aya Namah.'' Touch the ears with the thumb and ringfinger saying ``Om Adhok\d{s}aj\=aya Namah, Om N\=arasimh\=aya Namah.'' Touch the navel with the thumb and little finger saying ``Om Achyut\=aya Namah.'' Touch the breast with the palm, saying ``Om Jan\=ardan\=aya Namah.'' Touch the head saying ``Om Upendr\=aya Namah.'' Touch the roots of the two arms saying ``Om Haraye Namah, Om Kri\d{s}\d{n}\=aya Namah.''

25-50. While sipping the \=Achaman water on the right hand, touch the right hand with your left hand; otherwise; the water does not become pure. While doing \=Achaman, make the palm and the fingers all united and close, of the form of a Gokar\d{n}a (the ear of a cow) and spreading the thumb and the little finger, drink the water of the measure of a pea. If a greater or less quantity be sipped, then that would amount to drinking liquor. Then thinking of the Pra\d{n}ava, make the Pr\=a\d{n}\=ay\=ama, and repeat mentally the G\=ayatr\={\i} with her head and the Tur\={\i}ya p\=ada, i.e., \=Apojyotih rasomritam Brahm\=a Bhurbhuvah svarom. Inhale the air by the left nostril (P\=urak), close both the nostrils (Kumbhak) and exhale the air, by the right nostril (Rechak). Thus Pr\=a\d{n}\=ay\=ama is effected. While doing P\=urak, Kumbhak and Rechak repeat the G\=ayatr\={\i} every time; hold the right nostril with the right thumb and hold the left nostril with the ringfinger and little finger (i.e., don't use forefinger and middle finger).

The Yogis who have controlled their minds say that Pr\=a\d{n}\=ay\=ama is effected by the three processes P\=uraka, K\=umbhaka and Rechaka. The external air is inhaled in P\=uraka; air is not exhaled nor inhaled (it is retained inside) in K\=umbhaka; and air is exhaled in Rechaka. While

doing P\=uraka, meditate on the navel, the four-armed high-souled Vi\d{s}\d{n}u, of the blue colour (Sy\=ama) like the blue lotus. While doing K\=umbhaka, meditate in the heart lotus the four-faced grandsire Brahm\=a Praj\=apati, the Creator seated on the lotus and while doing Rechaka meditate, on the fore-head, on the white sin-destroying \'Sankara, pure as crystal. In P\=uraka, the union with Vi\d{s}\d{n}u is obtained; in K\=umbhaka, the knowledge of Brahm\=a is attained and in Rechaka, the highest position of \=I\'svara (\'Siva) is attained. This is the method of \=Achaman according to the Pur\=a\d{n}as. Now I am speaking of the all sin destroying Vaidik \=Achaman. Listen. Reciting the G\=ayatr\={\i} mantra ``Om Bhurbhuvah,'' sip a little water; this is the Vaidik \=Achaman after repeating the seven great Vy\=ahritis Om Bhuh; Om Bhuvah, Om Svah, Om Mahah, Om Janah, Om Tapah, Om Satyam, repeat G\=ayatr\={\i} and the head of the G\=ayatr\={\i} \=Apojyoti Rasomritam Brahm\=a Bhurbhuvah svarom and practise Pr\=a\d{n}\=ay\=ama three times. Hereby all sins are destroyed and all virtues spring. Now another sort of Pr\=a\d{n}\=ay\=ama Mudr\=a is described :-- The V\=anaprasth\={\i}s and Grihasthas would do Pr\=a\d{n}\=ay\=ama with five fingers, holding the tip of the nose; the Brahm\=ach\=aris and Yatis would do Pr\=a\d{n}\=ay\=ama with the thumbs, little finger, and ring finger (avoiding middle and fore). Now I am speaking of the \=Aghama\d{s}a\d{n}a M\=arjana mantra. Listen. The Mantra of this M\=arjana is ``\=Apohisth\=a Mayobhuvah, etc.'' There are three mantras in this. There are three P\=adas in every mantra, prefix Om to every p\=ada (thus nine times Om is to be prefixed); at the end of every p\=ada sprinkle water on the head with the sacrificial thread and the Ku\'sa grass. Or at the end of every mantra do so. By the above M\=arjana (cleaning) the sins of one hundred years are instantly destroyed. Then making \=Achaman (taking a sip of water to rinse the mouth before worship), repeat the three Mantras ``Om S\=urya\'scha m\=a manyu\'scha, etc.'' By this act, the mental sins are destroyed. As m\=arjana is done with Pra\d{n}ava, Vy\=arhitis, and G\=ayatr\={\i}, so make M\=arjana by the three mantras ``\=Apohisth\=a, etc.'' Make your right palm of the shape of a cow's ear; take water in it and carry it before your nose and think thus :-- ``There is a terrible sinful person in my left abdomen, his colour is dark black and he is horrible looking.'' Recite, then, the mantras ``Om ritamcha satyamch\=abh\={\i}dhy\=at, etc.'' and ``Drup\=ad\=adiva Mumuch\=ana, etc.'' and bring that Sinful Person through your right nostril to the water in the palm. Don't look at that water; throw it away on a bit of stone to your left. And think that you are now sinless. Next, rising from the seat, keep your two feet horizontal and with the fingers save forefinger and thumb, take a palmful of water and with your face towards the Sun, recite the G\=ayatr\={\i} three times

and offer water to the Sun three times. Thus, O Muni! The method of offering the Arghyas has been mentioned to you.

51-80. Then circumambulate, repeating the S\=urya Mantra. The one thing to be noted in offering Arghyas is this :-- Offer once in the midday, and three times in the morning and three times in the evening. While offering the Arghya in the morning, bend yourself a little low; in offering the arghya in the midday, stand up; and while offering the arghya in the evening, it can be done while sitting. Now I will tell you why the Arghya is offered to the Sun. Hear. Thirty Koti R\=ak\d{s}asas known as the Mandehas, always roam on the path of the Sun (the mental Sun also). They are great heroes, treacherous and ferocious. They always try to devour the Sun, while they assume terrible forms. For this reason the Devas and the \d{R}i\d{s}is combined offer the water with their folded hands to the Sun, while they perform the great Sandhy\=a Up\=asan\=a. The water thus offered, becomes transformed into the thunderbolt and burns the heads of the cruel demons (and throws them on the island Mandeh\=aru\d{n}a) Therefore the Br\=ahma\d{n}as daily do their Sandhyop\=asana. Infinite merits accrue from this Sandhy\=a Up\=asan\=a. O N\=arada! Now I am speaking to you of the Mantras pertaining to the Arghya. No sooner they are pronounced the full effects of performing the Sandhy\=as are obtained. I am That Sun; I am That Light; I am That \=Atman (Self); I am \'Siva; I am the Light of \=Atman; I am clear and transparently white; I am of the nature of all energy; and I am of the nature of R\=asa (the sweetness, all the sweet sentiments). O Dev\={\i}! O G\=ayatr\={\i}! O Thou! Who art of the nature of Brahm\=a! Let Thee come and preside in my heart to grant me success in this Japa Karma. O Dev\={\i}! O G\=ayatr\={\i}! Entering into my heart, go out again with this water. But Thou wouldst have to come again. Sit thus on a pure seat and with a single intent repeat the G\=ayatr\={\i}, the Mother of the Vedas. O Muni! In this Sandhyop\=asan\=a, the Khhechar\={\i} Mudr\=a ought to be done after practising the Pr\=a\d{n}\=ay\=ama. Hear now the meaning of the Khhech\=ari Mudr\=a. When the soul of a being leaves the objects of senses, it roams in the \=Ak\=a\'sa, i.e., it becomes aimless when the tongue also goes to the \=Ak\=a\'sa and roams there; and then the sight is fixed between the eyebrows; this is called the Khhech\=ari Mudr\=a. There is no \=Asana (seat) equal to Siddh\=asana and there is no V\=ayu (air) equal to the Khumbaka V\=aya (suspension of air in the body).

O N\=arada! There is no Mudr\=a equal to the Khhech\=ari Mudr\=a. One is to pronounce Pra\d{n}ava in Pluta (protracted) accents like the sound of a bell and, suspending his breath, sit quiet motionless in Sthir\=asana without any Ahamk\=ara (egoism). O N\=arada! I am now talking of

Siddh\=asana and its characteristic qualities. Hear. Keep one heel below the root of the genital and the other heel below the scrotum; keep the whole body and breast straight and motionless; withdraw the senses from their objects and look at the point, the pituitary body, between the eyebrows. This posture is called the Siddh\=asan and is pleasant to the yogis. After taking this seat, invoke the G\=ayatr\={\i}. ``O Mother of the Vedas! O G\=ayatr\={\i}! Thou art the Dev\={\i} granting boons to the Bhaktas. Thou art of the nature of Brahm\=a. Be gracious unto Me. O Dev\={\i}! Whoever worships Thee in the day gets his day sins destroyed and in the night, night sins destroyed. O Thou! Who art all the letters of the alphabet! O Dev\={\i}! O Sandhye! O Thou who art of the nature of Vidy\=a! O Sarasvat\={\i}! O Ajaye! O Thou immortal! Free from disease and decay. O Mother! Who art all the Devas! I bow down to Thee.'' Invoke the Dev\={\i} again by the mantra ``Ojosi, etc,'' and then pray :-- ``O Mother! Let my japam and other acts in Thy worship be fulfilled with success by Thy Grace.'' Next for the freedom of the curse of G\=ayatr\={\i}, do the things properly. Brahm\=a gave a curse to G\=ayatr\={\i}; Vi\'sv\=amitra gave a curse to Her and Va\'sistha also cursed Her. These are the three curses; they are removed in due order by recollecting Brahm\=a, Vi\'sv\=amitra and Va\'sistha. Before doing Ny\=asa, one ought to collect oneself and remember the Highest Self; think in the lotus of the heart that Puru\d{s}a (Person) who is Truth; who is all this Universe, who is the Highest Self and who is All knowledge and who cannot be comprehended by words. Now I am speaking of the Amgany\=asa of Sandhy\=a; Hear. First utter Om and then utter the mantra.

Touch the two legs, saying ``Om Bhuhp\=ad\=abhy\=am namah''
Touch the knees, saying ``Om Bhuva J\=anubhy\=am namah''
Touch the hip, saying ``Om Svah Katibhy\=am namah''
Touch the navel, saying ``Om Maharn\=abhyai namah''
Touch the heart, saying ``Om Janah Hriday\=aya namah''
Touch the throat, saying ``Om Tapah Kanth\=aya namah''
Touch the forehead, saying ``Om Satyam Lal\=at\=aya namah''
Thus perform the Vy\=arhiti ny\=asa.

Next perform the Kar\=amgany\=asa thus :-- Om Tat savituh ramgusth\=abhy\=am namah (referring to the thumb); Om Varenyam Tarjan\={\i}bhy\=am namah (referring to the forefinger); Om bhargo devasya madhyam\=a bhy\=am namah (referring to the middle finger); Om Dh\={\i}mahi an\=amik\=abhy\=am namah (referring to the ringfinger); Om dh\={\i}yo yonah, Kanisth\=aby\=am namah (referring to the little finger); Om prachoday\=at kara tal pristh\=abhy\=am namah (referring to the upper part and lower part of the palm and all over the body).

81-106. Now I am speaking of the Amgany\=asa. Hear. Om tat savitur Brahm\=a tmane hriday\=aya namah (referring to the heart.)

Om Varenyam Vi\d{s}\d{n}v\=a tmane \'Sirase namah (referring to the head); Om bhargo devasya Rudr\=atmane \'Sikh\=ayai namah (referring to the crown of the head); Om dh\={\i}mahi \'Sakty\=atmane Kavach\=aya namah (referring to the Kavacha); Om dh\={\i}yoyonah K\=al\=atmane netratray\=aya namah (referring to the three eyes); Om prachoday\=at sarv\=atmane astr\=aya namah (referring to the Astra or armour, protecting the body). Now I am speaking of the Var\d{n}any\=asa. O Great Muni! Hear. This Var\d{n}any\=asa is performed by the letters in the G\=ayatr\={\i} mantra. If anybody does this, he becomes freed of sins.

``Om Tat namah'' on the two toes; (touching them).
``Om Sa namah'' on the two heels; (touching them).
``Om Vi namah'' on the legs;
``Om Tu namah'' on the two knees;
``Om Va namah'' on the two thighs;
``Om re namah'' on the anus;
``Om \d{n}i namah'' on the generative organ;
``Om ya namah'' on the hip;
``Om bha namah'' on the navel;
``Om Rgo namah on the heart;
``Om De namah'' on the breasts;
``Om va namah'' on the heart;
``Om sya namah'' on the throat;
``Om dh\={\i} namah'' on the mouth;
``Om ma namah'' on the palate;
``Om hi namah'' on the tip of the nose;
``Om dhi namah'' on the two eyes;
``Om yo namah'' on the space between the eye-brows;
``Om yo namah'' on the forehead;
``Om nah namah'' to the east;
``Om pra namah'' to the south;
``Om cho namah'' on the west;
``Om da namah'' on the north;
``Om y\=a namah'' on the head;
``Om ta namah'' on the whole body from head to foot.

Some J\=apakas (those who do the Japam) do not approve of the above ny\=asa. Thus the Ny\=asa is to be done. Then meditate on the G\=ayatr\={\i} or the World-Mother. The beauty of the body of the G\=ayatr\={\i} Dev\={\i} is like that of the full blown Jav\=a flower. She is seated on the big red lotus on the back of the Ha\d{n}sa (Flamingo); She is holding the red coloured garland on Her neck and anointed with red coloured unguent. She has four faces;

every face has two eyes. On her four hands are a wreath of flowers, a sacrificial ladle, a bead, and a Kamandalu. She is blazing with all sorts of ornaments. From the Dev\={\i} G\=ayatr\={\i} has originated first the Rig veda. Brahm\=a worships the virgin G\=ayatr\={\i}; on the idea of \'Sr\={\i} Parame\'svar\={\i} G\=ayatr\={\i} has four feet. The Rig Veda is one; the Yajurveda is the second, the S\=amaveda is the third and the Atharva veda is the fourth foot. The G\=ayatr\={\i} has eight bellies; the east side is the one; the south is the second; the west is the third; the north is the fourth; the zenith is the fifth; the nadir is the sixth; the intermediate space is the seventh and all the corners are the eighth belly. G\=ayatr\={\i} has seven \'Siras (heads); Vy\=akara\d{n}am (Grammar) is one; \'Sik\d{s}\=a is the second (that Amga of the Veda, the science which teaches the proper pronunciation of words and laws of euphony); Kalpa is the third (the Ved\=anga which lays down the ritual and prescribes rules for ceremonial and sacrificial acts); Nirukta is the fourth (the Ved\=anga that contains glossarial explanation of obscure words, especially those occurring in the Vedas); Jyotish or astronomy is the fifth; Itah\=asa (history) and Pur\=a\d{n}as is the sixth head; and Upani\d{s}adas is the seventh head. Ag\d{n}i (fire) is the mouth of G\=ayatr\={\i}; Rudra is the \'Sikh\=a (the chief part); Her gotra (lineage) is S\=amkhy\=aya\d{n}a; Vi\d{s}\d{n}u is the heart of G\=ayatr\={\i} and Brahm\=a is the armour of G\=ayatr\={\i}. Think of this Mahe\'svar\={\i} G\=ayatr\={\i} in the middle of the Solar Orb. Meditating on the G\=ayatr\={\i} Dev\={\i} as above, the devotee should shew the following twenty-four Mudr\=as (signs by the fingers, etc., in religious worship) for the satisfaction of the Dev\={\i} :-- (1) Sanmukh; (2) Samp\=ut; (3) Vitata (4) Vistrita; (5) Dv\={\i}mukha; (6) Trimukha; (7) Chaturmukha; (8) Panchamukha; (9) Sa\d{n}mukha; (10) Adhomukha; (11) Vy\=apaka; (12) Anjali; (13) \'Sakata (14) Yamap\=a\'sa; (15) fingers intertwined end to end; (16) Vilamba (17) Mustika; (18) Matsya; (19) K\=urma; (20) Var\=aha; (21) Simh\=akr\=anta; (22) Mah\=akr\=anta; (23) Mudgara; (24) Pallava. Next make japam once only of one hundred syllabled G\=ayatr\={\i}. Thus twenty-four syllabled S\=avitr\={\i}, ``J\=atavedase sunav\=ama, etc.,'' forty-four syllabled mantra; and the thirty two syllabled mantra, ``Tryamvakam Jaj\=amahe, etc.'' These three mantras united make up one hundred lettered G\=ayatr\={\i}. (The full context of the last Mantra is this :-- Om Haum Om yum sah - Trayamvakam yaj\=amahe Sugandhim Pusti Vardhanam. Urbh\=arukamiva bandhan\=an mrityo m\=uksiya ma mrit\=at Bhur Bhhuvah. Svarom Yum Svah Bhurbhuvah Svarom Haum.) Next make japam of Bhurbhuvah Svah, twenty four lettered G\=ayatr\={\i} with Om. O N\=arada! The Br\=ahma\d{n}as are to perform daily the Sandhyo p\=as\=an\=a repeating G\=ayatr\={\i}, completely adopting the rules above prescribed and then he will be able to enjoy completely pleasures, happiness and bliss.

Here ends the Sixteenth Chapter of the Eleventh Book on the description of Sandhy\=a Up\=as\=an\=a in the Mah\=apur\=a\d{n}am \'Sr\={\i} Mad Dev\={\i} Bh\=agvatam of 18,000 verses by Mahar\d{s}i Veda Vy\=asa.



