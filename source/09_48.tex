\chapter{On the anecdote of Manas\=a}

1-30. N\=ar\=aya\d{n}a said :-- O N\=arada! I will now speak of the Dhy\=anam and the method of worship of \'Sr\={\i} Dev\={\i} Manas\=a, as stated in the S\=ama Veda. Hear. ``I meditate on the Dev\={\i} Manas\=a, Whose colour is fair like that of the white champaka flower, whose body is decked all over with jewel ornaments, whose clothing is purified by fire, whose sacred thread is the N\=agas (serpent), who is full of wisdom, who is the foremost of great J\~nanins, who is the Presiding deity of the Siddhas, Who Herself is a Siddha and who bestows Siddhis to all.'' O Muni! Thus meditating on Her, one should present Her, flowers, scents, ornaments, offerings of food and various other articles, pronouncing the principal Seed Mantra. O N\=arada! The twelve lettered Siddha Mantra, to be mentioned below, yields to the Bhaktas their desires like the Kalpa Tree. Now the Radical mantra as stated in the Vedas is ``Om Hr\={\i}m \'Sr\={\i}m Kl\={\i}m Aim Manas\=a Devyai Sv\=ah\=a.'' Repetition of this, five lakhs of times, yields success to one who repeats. He who attains success in this mantra gets unbounded name and fame in this world. Poison becomes nectar to him and he himself becomes famous like Dhanvantari. O N\=arada! If anybody bathes on any Samkr\=anti day (when the sun enters from one sign to another) and going to a private room (hidden room), invokes the Dev\={\i}

Manas\=a \=I\'s\=an\=a and worships Her with devotion, or makes sacrifices of animals before the Dev\={\i} on the fifth day of the fortnight, he becomes certainly wealthy, endowed with sons and name and fame. Thus I have described to you the method of worship of Manas\=a Dev\={\i}. Now hear the anecdote of the Dev\={\i} as I heard from Dharma. In olden days, men became greatly terrified on earth from snakes and took refuge of Ka\'syapa, the supreme amongst the Munis. The Mahar\d{s}i Ka\'syapa became very afraid. He then with Brahm\=a, and by His command composed a mantra following the principal motto of the Vedas. While composing this mantra, he intensely thought of the Dev\={\i}, the Presiding Deity of that Mantra, through the power of his Tapasy\=a and through the mental power, the Dev\={\i} Manas\=a appeared and was named so, as She was produced from the sheer influence of mind. On being born, the girl went to the abode of \'Sankara in Kail\=a\'sa and began to worship Him and chant hymns to Him with devotion. For one thousand Divine years, the daughter of Ka\'syapa served Mah\=adeva when He became pleased. He gave her the Great Knowledge, made Her recite the S\=ama Veda and bestowed to her the eight-lettered Kri\d{s}\d{n}a mantra which is like the Kalpa Tree. \'Sr\={\i}m Hr\={\i}m Kl\={\i}m Kri\d{s}\d{n}\=aya Namah was the eight lettered Mantra. She got from Him the Kavacha (amulet) auspicious to the three worlds, the method of worship and all the rules of Pura\d{s}chara\d{n}a (repetition of the name of a deity attended with burnt offerings, oblations, etc.) and went by His command to perform in Pu\d{s}kara very hard austerities. There she worshipped Kri\d{s}\d{n}a for the three Yugas. \'Sr\={\i} Kri\d{s}\d{n}a then appeared before Her. On seeing Kri\d{s}\d{n}a, immediately the girl, worn out by austerities, worshipped Him, and she was also worshipped by \'Sr\={\i} Kri\d{s}\d{n}a. Kri\d{s}\d{n}a granted her the boon, ``Let you be worshipped throughout the world'' and departed. O N\=arada! She was thus first worshipped by the Supreme Spirit, the Dev\={\i} Kri\d{s}\d{n}a; secondly by \'Sankara; thirdly by the Mahar\d{s}i Ka\'syapa and the Devas. Then she was worshipped by the Munis, Manus, N\=agas, and men; and She became widely renowned in the three worlds. Ka\'syapa gave Her over to the hands of Jaratk\=aru Muni. At the request of the Br\=ahmi\d{n} Ka\'syapa, the Muni Jarat K\=aru married Her. After the marriage, one day, being tired with his long work of Tapasy\=a, Jarat K\=aru laid his head on the hip and loins of his married wife and fell fast asleep. Gradually the evening came in. The sun set. Then Manas\=a thought, ``If my husband fails to perform the Sandhy\=a, the daily duty of the Br\=ahma\d{n}as, he would be involved in the sin of Brahmahaty\=a. It is definitely stated in the \'S\=astras, that if any Br\=ahma\d{n}a does not perform his Sandhy\=a in the morning and in the evening, he becomes wholly impure and the sins

Brahmahatty\=a and other crimes come down on his head.'' Arguing thus, these thoughts in her mind, as commanded by the Vedas, at last she awakened her husband, who then got up from his sleep.

31-60. The Muni Jarat K\=aru said :-- ``O Chaste One! I was sleeping happily. Why have you thus interrupted my sleep? All his vows turn out useless who injures her husband. Her tapas, fastings, gifts, and other meritorious works all come to vain who do things unpleasant to her husband. If she worships her husband, she is said to have worshipped \'Sr\={\i} Kri\d{s}\d{n}a. For the sake of fulfilling the vows of the chaste women, Hari himself becomes their husbands. All sorts of charities, gifts, all sacrifices, fastings, practising all the virtues, keeping to truth, worshipping all the Devas, nothing can turn out equal to even one-sixteenth part of serving one's husband. She ultimately goes with her husband to the region of Vaikuntha, who serves her husband in this holy land Bh\=arata. She comes certainly of a bad family who does unpleasant acts to her husband or who uses unpleasant words to her husband. She goes to the Kumbh\={\i}p\=aka hell as long as the Sun and Moon last and then she becomes born as a Chand\=al\={\i}, without husband and son.'' Speaking thus, Jarat K\=aru, the best of the Munis, became angry and his lips began to tremble. Seeing this, the best Manas\=a, shivering with fear, addressed her husband :--

I have broken your sleep and awakened you, fearing you might miss your time of Sandhy\=a. I have committed an offence. Punish me as you think. I know that a man goes to the K\=alas\=utra hell as long as the Sun and Moon last in this world, who throws an obstacle when any man eats, sleeps or enjoys with the opposite sex. O N\=arada! Thus saying, the Dev\={\i} Manas\=a fell down at the feet of her husband and cried again and again. On the other hand, knowing the Muni angry, and ready to curse her, the Sun came there with Sandhy\=a Dev\={\i}. And He humbly spoke to him with fear :-- ``O Bhagav\=an! Seeing Me going to set, and fearing that you may miss Dharma, your chaste wife has awakened you. O Br\=ahmi\d{n}! Now I am also under your refuge; forgive me. O Bhagav\=an! You should not curse Me. The more so, a Br\=ahma\d{n}a's heart is as tender as the fresh butter. The anger of a Br\=ahma\d{n} lasts only half the twinkling of an eye (K\d{s}a\d{n}). When a Br\=ahma\d{n}a becomes angry, he can burn all this world and can make a new creation. So who can possess an influence like a Br\=ahma\d{n}a. A Br\=ahmi\d{n} is a part of Brahm\=a; he is shining day and night with the Tejas of Brahm\=a. A Br\=ahma\d{n}a meditates always on the Eternal Light of Brahm\=a.'' O N\=arada! Hearing the words of the Sun, the Br\=ahmi\d{n} became satisfied and blessed Him. The Sun also went to His own place, thus blessed daily. To keep his promise, the Br\=ahmin Jaratk\=aru quitted

Manas\=a. She became very sorry and began to cry aloud with pain and anguish. Being very much distressed by the then danger, she remembered Her \=Ista Deva, Mah\=adeva, Brahm\=a, Hari and Her father Mahar\d{s}i Ka\'syapa. On the very instant when Manas\=a remembered \'Sr\={\i} Kri\d{s}\d{n}a, the Lord of the Gopis, Mah\=adeva, Brahm\=a and Mahar\d{s}i Ka\'syapa appeared there. Then seeing his own desired Deity \'Sr\={\i} Kri\d{s}\d{n}a, superior to Prakriti, beyond the attributes, Jaratk\=aru began to praise Him and bowed down to Him repeatedly. Then bowing down to Mah\=adeva, Brahm\=a and Ka\'syapa, he enquired why they had come there. Brahm\=a, then, instantly bowed down at the lotus feet of Hri\d{s}\={\i}ke\'sa and spoke in befitting words at that time if the Br\=ahmi\d{n} Jaratk\=aru leaves at all his legal wife, devoted to her own Dharma, he should first of all have a son born of her to fulfil his Dharma. O Muni! Any man can quit his wife, after he has impregnated her and got a son. But if without having a son, he leaves his wife, then all his merits are lost as all water leaks out of a sieve or a strainer. O N\=arada! Hearing thus the words of Brahm\=a, the Muni Jaratk\=aru by his Yogic power recited a Mantra and touching the navel of Manas\=a spoke to her :-- ``O Manas\=a! A son will be born in your womb self-controlled, religious, and best of the Br\=ahma\d{n}as.

61-77. That son will be fiery, energetic, renowned, well-qualified the foremost of the Knowers of the Vedas, a great J\~n\=anin and the best of the Yogis. That son is a true son, indeed, who uplifts his family who is religious and devoted to Hari. At his birth all the Pitris dance with great joy. And the wife is a true wife who is devoted to her husband, good-natured and sweet-speaking and she is religious, she is the mother of sons, she is the woman of the family and she is the preserver of the family. He is the true friend, indeed, the giver of one's desired fruits, who imparts devotion to Hari. That father is a true father who shows the way to devotion to Hari. And She is the True Mother, through whom this entering into wombs ceases for ever, yea, for ever! That sister is the true kind sister from whom the fear of Death vanishes. That Guru is the Guru who gives the Vi\d{s}\d{n}u Mantra and the true devotion to Vi\d{s}\d{n}u. That Guru is the real bestower of knowledge who gives the J\~n\=anam by which \'Sr\={\i} Kri\d{s}\d{n}a is meditated in whom this whole universe, moving and non-moving from the Brahm\=a down to a blade of grass, is appearing and disappearing. There is no doubt in this. What knowledge can be superior to that of \'Sr\={\i} Kri\d{s}\d{n}a. The knowledge derived from the Vedas, or from the sacrifices or from any other source is not superior to the service to \'Sr\={\i} Kri\d{s}\d{n}a. The devotion and knowledge of \'Sr\={\i} Hari is the Essence of all knowledge; all else is vain and mockery. It is through this Real Knowledge; that this bondage from this world is severed. But the Guru who does not impart this devotion

and knowledge of \'Sr\={\i} Hari is not the real Guru; rather he is an enemy that leads one to bondage. Verily he kills his disciple when he does not free him. He can never be called a Guru, father or friend who does not free his disciple from the pains in the various wombs and from the pains of death. Verily he can never be called a friend who does not show the way to the Undecaying \'Sr\={\i} Kri\d{s}\d{n}a, the Source of the Highest Bliss. So, O Chaste One! You better worship that Undecaying Para Brahm\=a \'Sr\={\i} Kri\d{s}\d{n}a, Who is beyond the attributes. O Beloved! I have left you out of a pretence; please excuse me for this. The chaste women are always forgiving; never they become angry because they are born of Sattvagu\d{n}as. Now I go to Pu\d{s}kara for Tapasy\=a; you better go wherever you like. Those who have no desire have their minds always attached to the lotus feet of \'Sr\={\i} Kri\d{s}\d{n}a.'' O N\=arada! Hearing the words of Jaratk\=aru, the Dev\={\i} Manas\=a became very much distressed and bewildered with great sorrow. Tears began to flow from her eyes. She then humbly spoke to her dearest husband :-- ``O Lord! I have not committed any such offence, as you leave me altogether when I have thus broken your sleep.

78-115. However kindly show Thyself to me when I will recollect you. The bereavement of one's friend is painful; more than that is the bereavement of a son. Again one's husband is dearer than one hundred sons; so the bereavement of one's husband is the heaviest of all. To women, the husband is the most beloved of all earthly things; hence he is called Priya, i.e., dear. As the heart of one who has only one son is attached to that son, as the heart of a Vai\d{s}\d{n}ava is attached to \'Sr\={\i} Hari; as the mind of one-eyed man to his one eye, as the mind of the thirsty is attached to water, as the mind of the hungry is attached to food, as the mind of the passionate is attached to lust, as the mind of a thief is attached to the properties of others, as the mind of a lewd man to his prostitute, as the mind of the learned is attached to the \'S\=astras, as the mind of a trader is attached to his trade, so the minds of chaste women are attached to their husbands.'' Thus saying, Manas\=a fell down at the feet of her husband. Jaratk\=aru, the ocean of mercy, then, took her for a moment on his lap and drenched her body with tears from his eyes. The Dev\={\i} Manas\=a, too, distressed at the bereavement of her husband also drenched the lap of the Muni with tears from her eyes. Some time after, the true knowledge arose in them and they both became free from fear. Jaratk\=aru then enlightened his wife and asked her to meditate on the lotus feet of \'Sr\={\i} Kri\d{s}\d{n}a the Supreme Spirit repeatedly; thus saying he went away for his Tapasy\=a. Manas\=a, distressed with sorrow, went to his \=Ista Deva Mah\=adeva on Kail\=a\'sa. The auspicious \'Siva and P\=arvat\={\i} both consoled her with knowledge and advice. Some days after, on an

auspicious they and on an auspicious moment she gave birth to a son born in part of N\=ar\=aya\d{n}a, and as the Guru of the Yogis and as the Preceptor of the J\~n\=anins. When the child was in mother's womb, he heard the highest knowledge from the mouth of Mah\=adeva; therefore he was born as a Yog\={\i}ndra and the Spiritual Teacher of the J\~n\=anins. On his birth, Bhagav\=an \'Sankara performed his natal ceremonies and performed various auspicious ceremonies. The Br\=ahma\d{n}as chanted the Vedas for the welfare of the child; various wealth and jewels and Kir\={\i}tas and invaluable gems were distributed by \'Sankara to the Br\=ahma\d{n}as; and P\=arvat\={\i} gave one lakh cows and various jewels to others. After some days, Mah\=adeva taught him the four Vedas with their Angas (six limbs) and gave him, at last, the Mrityumjaya Mantra. As in Manas\=a's mind there reigned the devotion to her husband, the devotion to her \=Ista Deva and Guru, the child's name was kept \=Astika.

\=Astika then got the Mah\=a Mantra from \'Sankara and by his command went to Pu\d{s}kara to worship Vi\d{s}\d{n}u, the Supreme Spirit. There he practised tapasy\=a for three lakh divine years. And then he returned to Kail\=a\'sa, to bow down to the great Yogi and the Lord \'Sankara, Then, bowing down to \'Sankara, he remained there for some time when Manas\=a with her son \=Astika went to the hermitage of Ka\'syapa, her father. Seeing Manas\=a with son, the Mahar\d{s}i's gladness knew no bounds. He fed innumerable Br\=ahma\d{n}as for the welfare of the child, and distributed lakhs and lakhs of jewels. The joy of Aditi and Diti (the wives of Ka\'syapa) knew no bounds; Manas\=a remained there for a long, long time with his son. O Child! Hear now an anecdote on this. One day due to a bad Karma, a Br\=ahma\d{n}a cursed the king Parik\d{s}it, the son of Abhimanyu; one \d{R}i\d{s}i's son named \'Sring\={\i}, sipping the water of the river Kau\'sik\={\i} cursed thus :-- ``When a week expires, the snake Tak\d{s}aka will bite you, and you will be burnt with the poison of that snake Tak\d{s}aka.'' Hearing this, the King Parik\d{s}it, to preserve his life, went to a place, solitary where wind even can have no access and he lived there. When the week was over, Dhanvantari saw, while he was going on the road, the snake Tak\d{s}aka who was also going to bite the king. A conversation and a great friendship arose between them; Tak\d{s}aka gave him voluntarily a gem; and Dhanvantari, getting it, became pleased and went back gladly to his house. The king Parik\d{s}it was lying on his bed-stead when Tak\d{s}aka bit the king. The king died soon and went to the next world. The king Janamejaya then performed the funeral obsequies of his father and commenced afterwards the Sarpa Yaj\~n\=a (a sacrifice where the snakes are the victims). In that sacrifice, innumerable snakes gave up their lives by the Brahm\=a Teja (the fire of the Br\=ahmi\d{n}s). At this, Tak\d{s}aka became

terrified and took refuge of Indra. The Br\=ahmi\d{n}s, then, in a body, became, ready to burn Tak\d{s}aka along with Indra, when, Indra and the other Devas went to Manas\=a. Mahendra, bewildered with fear, began to chant hymns to Manas\=a. Manas\=a called his own son \=Astika who then went to the sacrificial assembly of the king Janamejaya and begged that the lives of Indra and Tak\d{s}aka be spared. The king, then, at the command of the Br\=ahma\d{n}as, granted their lives. The king, then, completed his sacrifice and gladly gave the Dak\d{s}i\d{n}\=as to the Br\=ahmi\d{n}s. The Br\=ahma\d{n}as, Munis, and Devas collected and went to Manas\=a and worshipped Her separately and chanted hymns to Her. Indra went there with the various articles and He worshipped Manas\=a with devotion and with great love and care; and He chanted hymns to Her. Then bowing down before Her, and under the instructions of Brahm\=a, Vi\d{s}\d{n}u and Mahe\'sa, offered her sixteen articles, sacrifices and various other good and pleasant things. O N\=arada! Thus worshipping Her, they all went to their respective places. Thus I have told you the anecdote of Manas\=a. What more do you want to hear. Say.

N\=arada said :-- ``O Lord! How did Indra praise Her and what was the method of His worshipping Her; I want to hear all this.''

116-124. N\=ar\=aya\d{n}a said :-- Indra first took his bath; and, performing \=Achamana and becoming pure, He put on a fresh and clean clothing and placed Manas\=a Dev\={\i} on a jewel throne. Then reciting the Vedic mantras he made Her perform Her bath by the water of the Mand\=akin\={\i}, the celestial river Ganges, poured from a jewel jar and then He made Her put on the beautiful clothing, uninflammable by fire. Then He caused sandalpaste to be applied to Her body all over with devotion and offered water for washing Her feet and Arghya, an offering of grass and flowers and rice, etc., as a token of preliminary worship. First of all the six Devat\=as Gane\'sa, Sun, Fire, Vi\d{s}\d{n}u, \'Siva, and \'Siv\=a were worshipped. Then with the ten lettered mantra, ``Om Hr\={\i}m \'Sr\={\i}m Manas\=a Devyai Sv\=ah\=a'' offered all the offerings to Her. Stimulated by the God Vi\d{s}\d{n}u, Indra worshipped with great joy the Dev\={\i} with sixteen articles so very rare to any other person. Drums and instruments were sounded. From the celestial heavens, a shower of flowers was thrown on the head of Manas\=a. Then, at the advice of Brahm\=a, Vi\d{s}\d{n}u and Mahe\'sa, the Devas and the Br\=ahma\d{n}as, Indra, with tears in his eyes, began to chant hymns to Manas\=a, when his whole body was thrilled with joy and hairs stood on their ends.

125-145. Indra said :-- ``O Dev\={\i} Manase! Thou standest the highest amongst the chaste women. Therefore I want to chant hymns to

Thee. Thou art higher than the highest. Thou art most supreme. What I now praise Thee? Chanting hymns is characterised by the description of one's nature; so it is said in the Vedas. But, O Prakriti! I am unable to ascertain and describe Thy qualities. Thou art of the nature of \'Suddha Sattva (higher than the pure sattva unmixed with any other Gu\d{n}as); Thou art free from anger and malice. The Muni Jaratk\=aru could not forsake Thee; therefore it was that he prayed for Thy separation before. O Chaste One! I have now worshipped Thee. Thou art an object of worship as my mother Aditi is. Thou art my sister full of mercy; Thou art the mother full of forgiveness. O Sure\'svar\={\i}! It is through Thee that my wife, sons and my life are saved. I am worshipping Thee. Let Thy love be increased. O World-Mother! Thou art eternal; though Thy worship is extant everywhere in the universe; yet I worship Thee to have it extended further and further. O Mother! Those who worship Thee with devotion on the Sankr\=anti day of the month of \=A\d{s}\=udha, or on the N\=aga Pa\~ncham\={\i} day, or on the Sankr\=anti day of every month or on every day, they get their sons and grandsons, wealth and grains increased and become themselves famous, well gratified, learned and renowned. If anybody does not worship Thee out of ignorance, rather if he censures Thee, he will be bereft of Lak\d{s}m\={\i} and he will be always afraid of snakes. Thou art the Griha Lak\d{s}m\={\i} of all the householders and the R\=aja Lak\d{s}m\={\i} of Vaikuntha. Bhagav\=an Jarat K\=aru, the great Muni, born in part of N\=ar\=aya\d{n}a, is Thy husband. Father Ka\'syapa has created Thee mentally by his power of Tapas and fire to preserve us; Thou art his mental creation; hence thy name is Manas\=a. Thou Thyself hast become Siddha Yogin\={\i} in this world by thy mental power; hence thou art widely known as Manas\=a Dev\={\i} in this world and worshipped by all. The Devas always worship Thee mentally with devotion; hence the Pundits call Thee by the name of Manas\=a. O Dev\={\i}! Thou always servest Truth, hence Thou art of the nature of Truth. He certainly gets Thee who always thinks of Thee verily as of the nature of truth.'' O N\=arada! Thus praising his sister Manas\=a and receiving from her the desired boon, Indra went back, dressed in his own proper dress, to his own abode. The Dev\={\i} Manas\=a, then, honored and worshipped everywhere, and thus worshipped by her brother, long lived in Her father's house, with Her son.

One day Surabhi (the heavenly cow) came from the Goloka and bathed Manas\=a with milk and worshipped Her with great devotion and revealed to Her all the Tattva J\~n\=anas, to be kept very secret. (This is now made the current story wherever any Lingam suddenly becomes visible.) O N\=arada! Thus worshipped by the Devas and Surabhi, the Dev\={\i} Manas\=a went to the Heavenly regions. O Muni! One gets no fear from snakes who recites

this holy Stotra composed by Indra and worships Manas\=a; his family descendants are freed from the fear due to snakes. If anybody becomes Siddha in this Stotra, poison becomes nectar to him. Reciting the stotra five lakhs of times makes a man Siddha in this Stotra. So much so that he can sleep on a bed of snakes and he can ride on snakes.

Here ends the Forty-eighth Chapter of the Ninth Book on the anecdote of Manas\=a in the Mah\=a Pur\=a\d{n}am \'Sr\={\i} Mad Dev\={\i} Bh\=agavatam of 18,000 verses by Mahar\d{s}i Veda Vy\=asa.



