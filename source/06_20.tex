\chapter{On the son born of mare by Hari}

1-2. Janamejaya said :-- ``O Bhagav\=an! A great doubt has arisen in my mind on this subject. Who was it that took away that son, when both Lak\d{s}m\={\i} and N\=ar\=aya\d{n}a left it, in that forlorn state, in a forest without any person there to look after?''

3-11. Vy\=asa said :-- O King! No sooner Lak\d{s}m\={\i} and N\=ar\=aya\d{n}a departed from that place, one Vidy\=adhara, named Champaka, mounting on a beautiful celestial car came there at his free will, sporting with a woman named Madan\=alas\=a. There they saw that one lovely child, exquisitely beautiful like a Deva's son, was playing alone as it liked. They then, quickly descended from their chariot and picked it up. Vidy\=adhara became very glad as a beggar becomes glad, when he gets a hoard of jewels. On taking that newly born beautiful child like a Cupid, Champaka gave it to the Dev\={\i} Madan\=alas\=a. Madan\=alas\=a took it and became very much astonished; and her hairs stood at their ends. She clasped it to her bosom and kissed it frequently. O Bh\=arata! Taking that child on her lap as if her own child, Madan\=alas\=a embraced it and kissed it and got the highest happiness. Then both of them took that child and mounted on the car. The lean Madan\=alas\=a then laughingly queried :-- ``O Lord! Whose child is this? Who has left it in this forest? It seems to me Mah\=a Deva, desirous to give me a son, has given it unto me.''

12-18. Champaka said :-- I will just now go and ask the all-knowing Indra whose child is this, whether it is of a Deva, D\=anava or Gandharva. If he orders, I will purify this child found thus in this forest by the Veda Mantrams and then accept it as my own. It is not advisable to do a thing suddenly without knowing all the details. Thus saying to his wife Madan\=alas\=a, Champaka went with a gladdened heart hurriedly to the city of Indra with that child in his arms. Champaka gladly bowed down at the feet of Indra and gave him all the information he knew about the child and stood at one side with folded hands and spoke, ``O Lord of the Devas! I have got this child, beautiful as Cupid, in the sacred place of pilgrimage at the confluence of the Jumn\=a and the Tamas\=a. O Lord of \'Sach\={\i}! Whose child is this? and why did they forsake it there? If

you kindly permit, I will take this child as my own son. This child is very beautiful and liked very much by my wife; it is also the rule laid down in the \'S\=astras that one can accept any child as the Kritrima son. Therefore it is my earnest desire that I purify this child by the Veda Mantrams and take it legally as my own son.''

19-24. Indra said :-- O Highly Fortunate One! Bhagav\=an V\=asudeva, assuming the form of a horse, has produced this child out of the womb of Kamal\=a in the form of a mare. He intends to give over the child, capable to destroy enemies to Turvasu, the son of Yay\=ati, and thus will get a great purpose achieved by the child. That King, very religious, will be sent by Hari today and he will come for the child in that beautiful sacred place of pilgrimage. You better go back as early as possible and keep the child there as it was before till that king comes to that spot at the instance of the Devadeva Vi\d{s}\d{n}u. Do not waste a minute more. The King will be very sorry if he does not find the child there. Therefore O Champaka! Quit the attachment that you have for this child. You should know that this child will be famous in this earth as Ekav\={\i}ra (only one hero).

25-30. Vy\=asa said :-- O King! Thus hearing the Indra's words, Champaka took the child and went back immediately to the spot whence he picked it up and keeping the child there as it laid, mounted on his car and went to his abode. At that instant, the husband of Lak\d{s}m\={\i}, the Lord of the three worlds, went to the King, mounted on His car, beaming with effulgent rays. When the Bhagav\=an was descending from His aerial car, the King Turvasu was very glad to see Him and bowed down and laid himself prostrate on the ground. The Bhagav\=an, then, comforted the King, his own devotee, and said, ``Get up, my child! Do away with your mental distress.'' The King also eagerly and full of devotion, began to utter verses in praise of the Bhagav\=an. O Lord of Ram\=a! You are the presiding Deity of the Devas; Lord of the whole worlds, Ocean of Mercy and Giver of advice to all men. O Lord! Your sight is very rare even to the Yogis; being myself of a very slow dull intellect; I have been fortunate enough to see you. O Lord! This shews Your mercy.

31-54. Vy\=asa said :-- O Bhagav\=an! O Infinite One! Those who are free from any desires and free from any attachment to worldly things, they alone are entitled to see Thee. O Deva of the Devas! I am bound in thousand and one desires. I am quite unfit to see Thee. There is no doubt in this. When Turvasu, the best of the kings, praised thus, Bhagav\=an Vi\d{s}\d{n}u became pleased and began to speak in the following pleasant words :-- ``O King! I am pleased with your asceticism; now ask your

desired boon; I will grant it immediately.'' The King bowed down again to the feet of Vi\d{s}\d{n}u and said :-- ``O Mur\=ari! For the sake of a son, I have practised this tapasy\=a; grant me a son like my Self.'' N\=ar\=aya\d{n}a, the First-born of the Devas, hearing this King's request spoke to him in infallible words :-- ``O son of Yay\=ati! Go to the confluence of the Yamun\=a and Tamas\=a. For you I have kept there today a son as you like and of indomitable prowess. O King! That child is begotten by me in the womb of Lak\d{s}m\={\i}.'' The King became very glad to hear the sweet pure words of the Bhagav\=an. Thus granting him the boon, Vi\d{s}\d{n}u went with Ram\=a to Vaikuntha. The King Turvasu, the son of Yay\=ati, hearing these words, became exceedingly gladdened in his heart and mounting on a chariot, whose speed cannot be checked, went to the spot where lay the child. The king, of extraordinary genius, went there and saw that the exceedingly beautiful child, catching hold of his toe by one of his soft hands was sucking it by his mouth and was playing on the ground. The child was born of N\=ar\=aya\d{n}a out of the womb of Kamal\=a. Therefore it resembled like Him. On looking at that beautiful lovely child, the famous King Harivarm\=a's face got cheered up with the intensest delight. The King took it up with both of his hands and got merged in the Ocean of Bliss and taking gladly the scent of its head embraced it happily. On looking at the beautiful lotus-face of the child, the King, choked with tears from his eyes and with feelings of joy said :-- ``O Child! N\=ar\=aya\d{n}a has given me, the child jewel in you; so save me from the terrors of the hell named Put. O Child! For full one hundred years I have practised a very hard tapasy\=a for the sake of you. Pleased with that, the Lord of Kamal\=a has given you to me for the happiness of my worldly career. Your Mother Ram\=a Dev\={\i} has forsaken Her own child for the sake of me and has gone away with Hari. O Child! That Mother is blessed whose face beams with joy by seeing the smiles in your lotus-face. O Delighter of my heart! The Lord of Ram\=a, the Deva of the Devas, has made you, as it were, to serve as a boat for me for crossing to the other side of this Ocean of World.'' Thus saying, the King took the child and gladly went home. Knowing that the King had come very close to his city, the King's Minister and the city people, the subjects came forward with the priest and many other presents and offerings. The bards, singers and S\=utas came in front of the King. The King as he entered into his city looked affectionately on his subjects and gladdened their spirits by enquiries of welfare. Then worshipped by the citizens, the King entered into the city with his child. As the King went along the royal road, the subjects showered on his head the flowers and fried rice. Then taking the child by his two arms, the King entered into his prosperous palace with his ministers.

The king next handed over the newly-born lovely child, as beautiful as Cupid, to the hands of his queen. The good queen took the child and asked the king :-- ``O King! Whence have you got this new born child as fascinating as the God of Love? Who has given this child to you? O Lord! Speak quickly. This child has stolen away my mind.'' The King gladly replied :-- ``O Beloved! The Lord of Kamal\=a, the Ocean of Mercy has given me this child; O Quick-eyed One! This child is born of N\=ar\=aya\d{n}a's part and out of the womb of Kamal\=a. O Dev\={\i}! This child has strength, energy, patience, gravity and all other good qualities.'' Then the queen took the child in her arms and got the unbounded bliss. Great festivities began to be performed in the palace of the King Turvasu. Charities were given to those that wanted; music and singing of various sorts were performed. In this ceremony for the sake of his child, the king Turvasu put the name of the child as ``Ekav\={\i}ra.'' Getting thus the child equivalent in form and qualities to Hari, the powerful Indra-like king became happy and freed from his debt due to his family line, became very cheerful and glad. O King! The king, powerful like his enemies, began to enjoy in his own palace with his all-qualified child, that was given to him by N\=ar\=aya\d{n}a, the Lord of all the Devas. He was always served by his dear wife and all sorts of pleasures and he felt himself enjoying as a King would do.

Here ends the Twentieth Chapter in the Sixth Book on the son born of mare by Hari, in the Mah\=apur\=a\d{n}am in \'Sr\={\i} Mad Dev\={\i} Bh\=agavatam of 18,000 verses by Mahar\d{s}i Veda Vy\=asa.



