\chapter{On the coming of Vi\'sv\=amitra to Tri\'sanku}

1-3. Janamejaya said :-- ``O Muni! I see that at the command of the King, the ministers installed Hari\'schandra on the royal throne; but how Tri\'sanku got rid of his Ch\=and\=ala body, kindly say. Was it that he bathed in the holy waters of the Ganges and lived in the forest and when he died he was freed of the curse; or was it that the Guru Va\'sistha favoured him by his grace and freed him of the curse? O best of \d{R}i\d{s}is! I am extremely eager to hear the life of the King; therefore kindly describe to me in detail his wonderful career.''

4-16. Vy\=asa said :-- O King! The King became gladdened in his heart to install his son on the throne and began to pass his days in that forest in the meditation of Bhagavat\={\i} Bhav\=an\={\i}. Thus some time passed when Vi\'sv\=amitra, the son of Kau\'sika, completing his course of Tapasy\=a with an intent mind returned to his home to see his wife and sons. On coming back to his house, the intelligent Muni found his sons and other members of the family happy and well conditioned, became very glad and when his wife came to him for his service, asked her :-- O Fair-eyed One! How did you spend your time in days of famine? There was nothing whatsoever of the stock of rice, etc., in the house; how then did you nourish these boys? Please speak to me. O Fair One! I was very busy with my austerities, I could not therefore come to you and see my boys; how then, O Beloved, and what measures did you resort to for their maintenance? O good and auspicious One! When I heard of the dire famine, I thought then ``I have no wealth; so what shall I do if I go there?'' Thus thinking I did not come then. O Beautiful One! At that time, one day I was very hungry and being very much

tired I entered into the house of a Ch\=and\=ala, with the object of stealing. On entering the house I found the Ch\=and\=ala sleeping; then being extremely distressed with hunger, I entered into his kitchen if I could find anything there. When the dishes were sought and turned, and when I was going to take cooked dog's flesh I immediately fell into the sight of that Ch\=and\=ala. He asked me very affectionately ``Who are you? Why have you entered here at this hour of night? Why are your looking after the dishes? Speak what you want.'' O Beautiful One! When the Ch\=and\=ala asked me these questions, I was very much pressed by hunger and I spoke out my wants in a tremulous voice :-- O Fortunate One! I am an ascetic Br\=ahmi\d{n} very much pained by hunger; I have entered your house stealthily and am looking out for some eatables from your cooking pots. O Intelligent One! I am now your guest in the form of a thief; I am now specially very hungry; so I will now eat your cooked meat; kindly permit me. Hearing these words, the Ch\=and\=ala spoke to me in words authorised by the \'S\=astras :-- O One of the Superior Var\d{n}a! Know this to be the house of a Ch\=and\=ala; so never eat that flesh.

17-28. The human birth is very rare in this world; then again to be born a Dv\={\i}ja is more difficult; and to get Br\=ahma\d{n}hood again in the Dv\={\i}jas is exceedingly difficult. Are you not aware of this? They ought never to eat the defiled food who desire to attain to the Heavens; owing to Karma, the Mahar\d{s}i Manu has denominated the seventh caste as Antyaja and has discarded them altogether. So, O Br\=ahmi\d{n}! I am now by my actions turned into a Ch\=and\=ala and so forsaken by all; there is no doubt in this. I am forbidding you so that this fault of Var\d{n}a \'Sa\d{n}kara may not suddenly attack you. Vi\'sv\=amitra said :-- ``O Knower of Dharma! What you are speaking is quite true; though a Ch\=and\=ala, your intelligence is very clear; hear, I will now speak to you the subtleties of the Dharma in times of danger. O Giver of respect! Always and by all means it is advisable to keep up the body if sin be thereby incurred, one ought to perform Pr\=aya\'schitta (penance) for its purification when the time of danger is over. But if one commits sin when the time is not one of danger, one gets degraded; not so in the time of danger. The man that dies out of hunger, goes to hell, no doubt. Therefore every man seeking for his welfare must satisfy his hunger. Therefore I intend to steal for preserving my body. O Ch\=and\=ala See! The sin, incurred in stealing during famine, which the Pundits have declared, goes to the God of rains until he does not pour forth rain.'' O Beloved! Just when I spoke these words, the God of Rains began to pour forth rain. O Beloved! Just when I spoke these words, the God of Rains began to pour forth rain so desired by all, like that coming out of the elephant's trunk. When

the clouds thus poured forth rains with the glitterings of the lightnings, I felt very glad and left the house of the Ch\=and\=ala. O Beautiful One! Now speak out to me, how did you behave in that famine time, so terrible to all the beings.

29-48. Vy\=asa said :-- O King! Hearing the above words of the husband, the sweet speaking lady spoke :-- Hear, how I passed my time in times of famine. O Muni! After you had gone to practise Tapasy\=a, the dire famine raged; and my sons, exhausted of hunger, became very anxious for food. I became very anxious to see the sons hungry; I then went out to the forest in quest of wild rice; and I got some fruits. Thus I spent some months by collecting the rice growing wildly in the forest; then in times these also could not be got and I became again anxious. The Nib\=ara rice, too, is now not available and nothing is obtained also by begging; there are no fruits on the trees and no roots are found under the earth. The sons are crying in agony of hunger. What to do? And where to go? What am I to say now to the hungry boys? Oh God! Thus thinking on various ways, I at last came to this conclusion that I would sell one of my sons to a rich man and whatever price I can fetch, with that I will preserve the lives of the other sons. O Dear! Thus thinking, I became ready and went out. O Fortunate One! Then this boy began to cry aloud and became very distressed; yet I was so shameless that I took the crying boy and got out of my \=A\'srama. At this time one R\=ajar\d{s}i Satyavrata seeing me very distressed, asked me ``O One of good vows! Why is this boy weeping?'' O Muni! I spoke to him ``Today I am going to sell this boy.'' The King's heart became overfilled with pity, and spoke to me :-- ``Take back to your \=A\'srama this boy. Daily I will supply you with meat for the food of your boys until the Muni returns home.'' O Muni! The King from that time used to bring, with great pity, daily the flesh of deer and boar killed by him in the forest and he used to tie that on this tree. O Beloved! Thus I could protect my sons in that fearful ocean of crisis; but that King was cursed by Va\'sistha only for my sake. One day that King did not get any meat in the forest; so he slaughtered the K\=ama Dhenu (the cow giving all desires) of Va\'sistha and the Muni became therefore very angry with him. The high-souled Muni, angry on account of the killing of his cow, called the King by the name of Tri\'sanku and made him a Ch\=and\=ala. O Kau\'sika! The prince turned into a Ch\=and\=ala because he came forward to do good to me, so I am very sorry for his sake. So it is your urgent duty to save the King from his terrible position by any means or by the influence of your powerful Tapasy\=a.

49. Vy\=asa said :-- O King! Hearing these words from his wife the Muni Kau\'sika consoled her and said :--

50-55. O Lotus-eyed One! I will free the King of his curse, who saved you at that critical moment; what more than this that I promise to you that I will remove his sufferings whether it be by my learning or it be by my Tapas. Thus consoling his wife at that moment, Kau\'sika, the Knower of the Highest Reality, began to think how he could destroy the pains and miseries of the King. Thus thinking, the Muni went to the King Tri\'sanku, who was staying at that time very humbly in a village of the Ch\=and\=alas, in the garb of a Ch\=and\=ala. Seeing the Muni coming, the King was greatly astonished and instantly threw himself before his feet like a piece of stick. Kau\'sika raised the fallen King and consoling him said :-- O King! You are cursed, on my account, by the Muni Va\'sistha. I will, therefore, fulfil your desires. Now speak what I am to do.

56-62. The King said :-- With a view to perform a sacrifice I prayed to Va\'sistha that I would perform a sacrifice, kindly do this for me. O Muni! Do that sacrifice, by which I can go to the Heavens in this my present body.'' Va\'sistha became angry and said :-- ``O Villain! How can you go and live in the Heavens in this your human body?'' I was very anxious to go to the Svarga (Heaven)
so I again spoke to him :-- ``O Sinless One! I will then have the excellent sacrifice done by another priest.'' Hearing this, Va\'sistha Deva cursed me, saying ``Be a Ch\=and\=ala.'' O Muni! Thus I have described to you all about my curse. You are the one quite able to remove now my grievances. Distressed in pain and agony, the King informed him and became quiet. Vi\'sv\=amitra, too, thought how he could free him of his curse.

Here ends the Thirteenth Chapter of the Seventh Book on the coming of Vi\'sv\=amitra to Tri\'sanku in the Mah\=a Pur\=a\d{n}am \'Sr\={\i} Mad Dev\={\i} Bh\=agavatam of 18,000 verses by Mahar\d{s}i Veda Vy\=asa.



