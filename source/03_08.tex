\chapter{On the Gu\d{n}as and their forms}

1. Brahm\=a said :--O N\=arada! I have described to you what you asked me just now about the creation of this universe, etc. Now hear with attention the colour of the three qualities, as well their configuration and how they are seen to exist.

2-3. The Sattva Gu\d{n}a is the source of pleasure and happiness; and when happiness comes, everything seems delightening. When integrity, truthfulness, cleanliness, faith, forgiveness, fortitude, mercy, bashfulness, peace and contentment arise in one's heart, know certainly that there has arisen firmly the Sattva Gu\d{n}a in that man.

4. The colour of the Sattva quality is white; it makes one always like religion, and have faith towards good purposes and discard one's tendencies towards bad objects.

5. The \d{R}i\d{s}is, the seers of truth classify Sraddh\=a (faith) under the three headings: S\=attvik, R\=ajasik and T\=amasik.

6. The quality Rajas is of red colour, wonderful and is not pleasant; it is the source of all troubles; there is no doubt in this.

7-8. The intelligent should understand that Rajas has certainly arisen in him, when his mind is filled with hatred, enmity, quarrelsome feeling, pride, stupification, uneasiness, sleeplessness, want of faith, egoism, vanity and arrogance.

9-11. The quality Tamas is of black colour. From Tamas arises laziness, ignorance, sleep, poverty, fear, quarrels, miserliness, insincerity, anger, aberration of intellect, violent atheism, and finding fault with others. The wise should think that Tamas has overpowered him when the above

qualities are found to possess him. When this Tamas quality is attended with the T\=amas\={\i} faith, then it becomes the source of pain to others.

12. The well wishers should manifest in themselves the Sattva qualities, control the R\=ajasic qualities, and destroy the T\=amasic qualities.

13. These three qualities are always found to remain intermingled with another, and each of them has always an inherent tendency to overcome the others; and therefore they are always, as it were, at war with another. They never have a separate existence from one another.

14. Never is found anywhere only one Sattva quality to the exclusion of others, the Rajas and Tamas; similar is the case with the Rajas or Tamas. They remain intermingled and depend on one another.

15. O N\=arada! Now hear, in detail, which two qualities remain in twins, knowing which, one is freed from this ocean of the transmigration of existence.

16. I have realised these; therefore you ought not to have any uncertainties on these points. The reality of these is especially felt, when it is really understood and when its effects begin to manifest themselves.

17. O high-minded! No one is able to realise these at once; it requires be heard, and then meditated upon. It also depends on one's natural capability and merits, due to the past actions.

18-21. Suppose one hears of the sacred places of pilgrimages and is filled with the R\=ajasic devotion. He goes out to those places and sees what he had heard before. There he performs his ablutions, makes offerings and the R\=ajasic gifts, stays there for some time; but all this he does under the influence of the R\=ajasic quality. And when he returns home, he finds himself not free from lust, anger, love and hatred; he remains the same that he was before. Therefore, in this case, O N\=arada! man hears but he does not realise the purifying effects of those holy places. O best of Munis! And when he does not find any benefit from the holy place of pilgrimage, it is equivalent to his not at all hearing of the place.

22. O best of Munis! The effect of visiting the sacred places of pilgrimages is then said to accrue to any individual, when he becomes freed from his sins, just as the fruit of cultivating fields is then said to occur, when the cultivator gets the ripened harvest out of his labour and enjoys the produce of his fields.

23. O N\=arada! Lust, anger, covetousness, delusion, thirst, hatred, love, vanity, malice, jealousy, non-forgiveness, unrest all these indicate that there is sin; and until these are purged out of one's body and mind,

man lives in sin. If the visiting of the sacred places of pilgrimages does not enable one to overcome the above passions, then the labours in going to those places are in vain, i.e., those labours merely are the results just as the toil only undergone by the cultivator is his only result, and is not met with any reward when there is no harvest at all.

24-28. Lo! The cultivator takes hard labour to clear his fields and cultivate the hard soil; he then sows the valuable seeds, because this is considered as doing good. Next, in expectation of the harvest, he undergoes a good deal of pains, day and night, to protect his fields and goes down to sleep, in the cold season, in the forest surrounded by tigers and other dangerous animals; but alas! locusts coming eat away and destroy all the crops, to the utter disappointment of the cultivator. All his labours are spent in vain. So, O N\=arada! The labour taken by one in going to the holy places yields pains, and pains only, instead of success and happiness.

29-32. When the Sattva quality grows in abundance, as a consequence of reading the Ved\=anta and the other \'S\=astras, dispassion comes towards the R\=ajasic and the T\=amasic qualities and things, and the Sattva quality overpowers the Rajas and Tamas. Similarly when the R\=ajasic quality grows in abundance, as a natural consequence of greed and avarice, then it overpowers Sattva and Tamas; so, by delusion, when the T\=amasic quality grows in abundance, it overpowers the Sattva and the R\=ajasic qualities. O N\=arada! I will now speak to you, in detail, about the overpowering of these qualities by one another.

33-35. When the Sattva quality grows in preponderance, the mind rests in religious ideas and things; it no more thinks of those external things, the products of the Rajas and Tamas qualities. Rather it wants to enjoy the S\=attvic things; wealth, religious affairs, sacrifices that can be acquired or performed without any trouble. Then that individual yearns after salvation and renounces his pursuits after the R\=ajasic and T\=amasic objects.

36. Thus, O N\=arada! first try to conquer the Rajas and then the Tamas; then the Sattva becomes pure.

37. When the R\=ajasic quality grows in preponderance, the individual imbibes the R\=ajasic faith, abandons his own San\=atan Dharma (settled eternal religion) and practises against his religious instructions.

38. Under the R\=ajasic propensities, one is eager to amass wealth and enjoy the R\=ajasic things. The Rajas drives away the Sattva and curbs the Tamas.

39-41. N\=arada! So when the T\=amasic quality grows in preponderance,

the faith in the Vedas and in the religious \'S\=astras entirely disappears. Imbibing the T\=amasic faith, the individual squanders away his wealth and is always engaged in quarrels, and party feelings, envy, violence and never enjoys peace. The individual with the T\=amasic quality in excess overpowers the R\=ajasic and S\=attvic qualities and becomes angry, wicked, and a great cheat and does everything as he likes, without any regard to his superiors.

42. N\=arada! Thus you see that, of these three qualities, no one can remain entirely alone, free from the other qualities. These remain always in twos or threes.

43-44. The Sattva can never exist without the Rajas; the Rajas can never exist without the Tamas; and these two qualities can never exist without Tamas. Again Tamas cannot exist without Rajas and Sattva. These qualities act and react always in twos or threes.

45-47. They never exist separately; they live in pairs or threes and are the originators of each other; these qualities are of the nature of procreating things; in other words, Sattva originates the Rajas or Tamas; again the Rajas originates sometimes Sattva and Tamas. Again the Tamas sometimes originates Sattva and Rajas. Thus they generate each other as the earthen pots and earth are their mutual causes.

48-49. Deva Datta, Vi\d{s}\d{n}u Mitra, and Yaj\~na Datta these three united perform any action, so these three qualities united reside in the buddhi (intellect) of the J\={\i}vas and generate their sense perceptions.

Just as the husband and wife get into a couple, the qualities get into couples.

50. The Sattva with Rajas forms the couple Rajas Sattva; so Sattva Rajas forms another couple, where the Sattva predominates. So Sattva end Rajas forms each with Tamas the other couples.

51. N\=arada said! O Dvaip\=ayana! Hearing thus about these three qualities from my father, I asked him again these questions.

Thus ends the eighth chapter of the Mah\=a Pur\=a\d{n}am \'Sr\={\i}mad Dev\={\i} Bh\=agavatam containing the description of the Gu\d{n}as, of 18,000 verses by Mahar\d{s}i Veda Vy\=asa.



