\chapter{On the anecdote of Tulas\={\i}}

1-6. N\=arada said :-- O Bhagav\=an! How came the pure chaste Tulas\={\i} to be the wife of N\=ar\=aya\d{n}a? Where was Her birth place? And what was She in Her previous birth? What family did She belong to? Whose daughter was She? And what austerities did She practise, that She got

N\=ar\=aya\d{n}a for Her husband, Who is above Prakriti, not liable to change without any effort, the Universal Self, Para Brahm\=a and the Highest God; Who is the Lord of all, omniscient, the Cause of all, the Receptacle of all, Omnipresent, and the Preserver of all. And how did Tulas\={\i}, the chief Dev\={\i} of N\=ar\=aya\d{n}a, turn out into a tree? Herself quite innocent, how She was attacked by the fierce Asura? O Remover of all doubts! My mind, plain and simple, has become restless. I am eager to hear all this. So kindly cut asunder all my doubts.

7-40. N\=ar\=aya\d{n}a said :-- O N\=arada! The Manu Dak\d{s}a S\=avar\d{n}i was very religious, devoted to Vi\d{s}\d{n}u, of wide renown, of a great name, and born with Vi\d{s}\d{n}u's parts. Dak\d{s}a S\=avar\d{n}i's son Brahm\=a S\=avar\d{n}i was also very religious, devoted to Vi\d{s}\d{n}u and of a pure \'Suddha Sattva Gu\d{n}a. Brahm\=a S\=avar\d{n}i's son, Dharma S\=avar\d{n}i was devoted to Vi\d{s}\d{n}u and He was the master of his senses. Dharma S\=avar\d{n}i's sons Rudra S\=avar\d{n}i was also a man of restraint and very devoted. Rudra S\=avar\d{n}i's son was Deva S\=avar\d{n}i, devoted to Vi\d{s}\d{n}u. Deva S\=avar\d{n}i's son was Indra S\=avar\d{n}i. He was a great Bhakta of Vi\d{s}\d{n}u. His son was Vri\d{s}adhvaja. But He was a fanatic \'Saiva (devoted to \'Siva). At his house \'Siva Himself remained for three Yugas according to the Deva measure. So much so that Bhagav\=an Bh\=utan\=atha loved him more than His own son. Vri\d{s}adhvaja did not recognise N\=ar\=aya\d{n}a, nor Lak\d{s}m\={\i} nor Sarasvat\={\i} nor another body. He discarded the worship of all the Devas. He worshipped \'Sankara only. The greatly exciting Lak\d{s}m\={\i} Puja (worship of Mah\=a Lak\d{s}m\={\i}) in the month of Bh\=adra and \'Sr\={\i} Pa\~ncham\={\i} Puja in the month of M\=agha, which are approved of by the Vedas, Vri\d{s}adhvaja put an entire stop to these and the Sarasvat\={\i} Puja. At this the Sun became angry with the King Vri\d{s}adhvaja, the discarder of the holy thread, the hater of Vi\d{s}\d{n}u, and cursed Him thus :-- ``O King! As you are purely devoted to \'Siva and \'Siva alone, and as you do not recognise any other Devas, I say within no time, you will be deprived of all your wealth and prosperity.'' \'Sankara, hearing this curse, became very angry and taking His trident, ran after the Sun. The Sun, becoming afraid, accompanied His father Ka\'syapa and took refuge of Brahm\=a. Bhagav\=an \'Sankara went to the Brahm\=a Loka, with trident in His hands. Brahm\=a became afraid of Mah\=adeva and took Sun to the region of Vaikuntha. Out of terror, the throats of Brahm\=a, Ka\'syapa, and Sun became parched and dry and they all went afraid for refuge to N\=ar\=aya\d{n}a, the Lord of all. They all bowed down to Him and praised Him frequently and finally informed Him of the cause of their coming and why they were so much afraid. N\=ar\=aya\d{n}a showed them mercy and granted them ``Abhaya'' (no fear). O You! Who are afraid, take rest. What cause of fear there can be to you, when I am here!

Whoever remembers Me, wherever he may be, involved in danger or fear, I go there with the Sudar\'san disc in My hand and save him. O Devas! I am always the Creator, Preserver and Destroyer of this universe. In the form of Vi\d{s}\d{n}u, I am the Preserver; in the form of Brahm\=a, I am the Creator; and in the form of Mahe\'sa, I am the Destroyer. I am \'Siva; I am you; and I am the S\=urya, composed of the three qualities. It is I who assumes many forms and preserves the universe. Better go to your respective places. What fear can ye suspect? I say, all your fears due to \'Sankara, are verily removed from this day. Bhagav\=an \'Sankara, the Lord of all, is the Lord of the S\=adhus. He always hears the words of His Bhaktas; and He is kind to them. He is their Self. Both the Sun and \'Siva are dearer to Me than My life. No one is more energetic than \'Sankara and the Sun. Mah\=adeva can easily create ten million Suns and ten million Brahm\=as. There is nothing impossible with \'S\=ulap\=ani. Having no consciousness of any outer thing, immersed, day and night, in meditating on Me with His whole heart concentrated, He is repeating with devotion My Mantra from His five faces and He always sings My glories. I am also thinking, day and night, of His welfare. Whoever worships Me in whichever way, I also favour him similarly. Bhagav\=an Mah\=a Deva is of the nature of \'Siva, all auspiciousness; He is the presiding deity of \'Siva, that is, liberation. It is because liberation is obtained from Him, He is called \'Siva. O dear N\=arada! While N\=ar\=aya\d{n}a was thus speaking, the trident bolder Mah\=adeva, with his eyes red like reddened lotuses, mounting on His bull, came up there and getting down from His Bull, humbly bowed down with devotion to the Lord of Lak\d{s}m\={\i}, peaceful and higher than the highest. N\=ar\=aya\d{n}a was then seated on His throne, decked with jewel ornaments. There was a crown on His crest; two earrings were hanging from His ears; the disc was in His hand, forest flower's garlands on His neck; of the colour of fresh blue rain cloud; His form exceedingly beautiful. The four-armed attendants were fanning Him with their four hands; His body smeared all over with sandal-paste and He is wearing the yellow garment. That Bhagav\=an, distressed with the thought of welfare for His Bhaktas, the Highest Self was sitting on a jewel throne and chewing the betel offered by Padm\=a and with smiling countenance, seeing and hearing the dancing and singing of the Vidy\=adhar\={\i}s. When Mah\=adeva bowed down to N\=ar\=aya\d{n}a, Brahm\=a also bowed down to Mah\=adeva. The Sun, too, surprised, bowed down to Mah\=adeva with devotion. Ka\'syapa, too, bowed and with great bhakti, began to praise Mah\=adeva. On the other hand, \'Sankara praised N\=ar\=aya\d{n}a and took His seat on

the throne. The attendants of N\=ar\=aya\d{n}a began to fan Mah\=adeva with white chowries. Then Vi\d{s}\d{n}u addressed Him with sweet nectar like voice and said :-- ``O Mahe\'svara! What brings Thee here? Hast Thou been angry?''

41-45. Mah\=adeva said : -- ``O Vi\d{s}\d{n}u! The King Vri\d{s}adhavaja is My great devotee; he is dearer to Me than My life. The Sun has cursed him and so I am angry. Out of the affection for a son I am ready to kill S\=urya. S\=urya took Brahm\=a's refuge and now he and Brahm\=a have taken Thy refuge. And Those who being distressed take Thy refuge, either in mind or in word, become entirely safe and free from danger. They conquer death and old age. What to speak of them, then, of those who come personally to Thee and take Thy refuge. The remembrance of Hari takes away all dangers. All good comes to them. O Lord of the world! Now tell me what becomes of My stupid Bhakta who has become devoid of fortune and prosperity by the curse of S\=urya.''

46-51. Vi\d{s}\d{n}u said :-- ``O \'Sankara! Twenty-one yugas elapsed within this one-half Ghatik\=a, by the coincidence of Fate (Daiva). Now go quickly to Thy abode. Through the unavoidable coincidence of the cruel Fate, Vri\d{s}adhvaja died. His son Rathadhvaja, too, died. Rathadhvaja had two noble sons Dharmadhvaja and Ku\'sadhvaja. Both of them are great Vai\d{s}\d{n}avas; but, through S\=urya's curse, they have become luckless. Their kingdoms are lost; they have become destitute of all property, prosperity and they are now engaged in worshipping Mah\=a Lak\d{s}m\={\i}. Mah\=a Lak\d{s}m\={\i} will be born in parts of their two wives. Then again, by the grace of Lak\d{s}m\={\i}, Dharmadhvaja and Ku\'sadhvaja will be prosperous and become great Kings. O \'Sambhu Your worshipper Vri\d{s}adhvaja is dead. Therefore Thou dost go back to Thy place. O Brahm\=a, O Sun! O Ka\'syapa! You all also better go to your places respectively.'' O N\=arada! Thus saying, Bhagav\=an Vi\d{s}\d{n}u went with His wife to the inner rooms. The Devas also went gladly to their own places respectively. And Mah\=adeva, too, Who is always quite full within Himself, departed quickly to perform His Tapas.

Here ends the Fifteenth Chapter on the question of anecdote of Tulas\={\i} in the Ninth Book in the Mah\=apura\d{n}am \'Sr\={\i} Mad Dev\={\i} Bh\=agavatam of 18,000 verses by Mahar\d{s}i Veda Vy\=asa.



