\chapter{On the rules of \'Sirovrata}

1-43. \'Sr\={\i} N\=ar\=aya\d{n}a said :-- The Br\=ahma\d{n}as that will perform duly the \'Sirovrata, to be described in the following, are the only ones who will attain very easily the highest knowledge, destroying all Avidy\=a or Ignorance. So much so that the rules of right living and right conduct as ordained in the \'Srutis and Smritis are not necessary to be observed by those who duly and devotedly perform the \'Sirovrata (i.e., vow of the head; i.e., vow to apply ashes on the forehead). O Learned One! It is through this \'Sirovrata that Brahm\=a and the other Devas have been able to get their Brahm\=ahood and the Devahood. The ancient sages glorified highly this \'Sirovrata. Brahm\=a, Vi\d{s}\d{n}u, Rudra and the other Devas all performed this \'Sirovrata. O Wise One! Those that performed

duly this \'Sirovrata, all became sinless though they were sinful in every way. Its name is \'Sirovrata, inasmuch as it is mentioned in the first part of the Atharva Veda. Only this vrata (vow) is called \'Sirovrata; no other thing is denominated by this name. By no other merit can this be acquired. O Muni! Different names are assigned to this vrata in different \'S\=akh\=as; in fact, they are all one and the same.

N. B. -- P\=asupata vrata, \'Sivavrata, etc., are the different names assigned to it. In all the \'S\=akh\=as, the One Substance, Intelligence solidified named \'Siva and the knowledge thereof is mentioned. This is ``\'Sirovrata.'' He who does not perform this \'Sirovrata, is irreligious and he is banished from all religious acts, though he is well-qualified in all branches of learning. There is no manner of doubt in this. This \'Sirovrata is like the blazing fire in destroying wholly the forest of sins. All knowledge fleshes before him who performs this \'Sirovrata. The Atharva \'Sruti expounds the subtle and particularly incomprehensible things; this \'Sruti declares the above \'Sirovrata as daily to be done; so it is one of the daily observances. ``Fire is ashes,'' ``water is ashes,'' ``earth is ashes,'' ``air is ashes,'' ``ether or \=Ak\=a\'sa is ashes,'' ``all this manifest Universe is ashes.'' These six mantras stated in the Atharva Veda are to be recited; after this, ashes are to be besmeared all over the body. This is named the \'Sirovrata. The devotee is to put on these ashes named \'Sirovrata during his Sandhyop\=asan\=a (practising Sandhy\=a thrice a day); so long as the Brahm\=a Vidy\=a (the knowledge of Brahm\=a) does not arise in him. One is to make a Sankalpa (resolve) of twelve years before one starts with this Vrata. In cases of incapability, a period of one year or six months, or three months or at least twelve days are to be adopted. That Guru is considered very cruel and his knowledge will come to an end who hesitates and does not impart the knowledge of the Vedas and other things to him who is purified by observing this \'Sirovrata. Know him certainly as a very merciful Guru who illumines the heart by Brahm\=a Vidy\=a just as God is very merciful and compassionate to all the living beings. One who performs one's own Dharmas for many births, acquires particular faith in this \'Sirovrata; others can have no faith in this. Rather he gets animosity for this vrata, because of the abundance of ignorance in him. So one ought never to advise on spiritual knowledge to an enemy who has no faith, rather who has hatred for any such thing. Those only that are purified by the observance of \'Sirovrata are entitled to Brahm\=a Vidy\=a; and none others. So the Vedas command :-- Those are to be advised on Brahm\=a Vidy\=a who have performed \'Sirovrata. Even the animal becomes freed of his animalism, as a result of this vrata; no sin occurs in killing that animal; this is

the decision of the Ved\=anta. It has been repeatedly uttered by J\=av\=ala \d{R}i\d{s}i that the Dharma of the Br\=ahma\d{n}as is to put on the Tripundra (three curved lines of ashes on the forehead). The householders are instructed to put on this Tripundra by repeating the mantra ``triyamvaka''' with Om prefixed. Those that are in the stage of the Bhik\d{s}us (Sanny\=asis, etc.,) are to put on this Tripundra uttering thrice the mantra ``Om Hasah.'' Such is regularly stated in J\=av\=ala \'Sruti. The house holders and the V\=anaprasthis (foresters) are to put on this Tripundra, uttering Triyamvaka mantra purified with ``Haum'' the pra\d{n}ava of \'Siva prefixed.

Those that are the Brahm\=ach\=aris are to use daily this Tripundra uttering the mantra ``Medh\=av\={\i},'' etc. The Br\=ahma\d{n}as are to apply the ashes in three curved lines on the forehead. The God \'Siva is always hidden under the cover of ashes; so the \'Saivas, the devotees of \'Siva are to use the Tripundra. The Br\=ahma\d{n}as are to use daily this Tripundra. Brahm\=a is the Prime Br\=ahmi\d{n}. When He used Tripundra on His forehead, what need to tell, then, that every Br\=ahma\d{n} ought always to use it! Never fail, out of error, to besmear your body with the ashes as prescribed in the Vedas and worship the \'Siva Lingam. The Sanny\=asins are to apply Tripundra on their forehead, arms, chest, uttering the Triyamvaka mantra with Om prefixed and also the five lettered mantra of \'Siva ``Om Namah \'Siv\=aya.'' The Brahm\=ach\=aris should use Tripundra of ashes, obtained from their own fire, uttering the mantra ``Triy\=ayu\d{s}am Jamadagneh,'' etc., or the mantra ``Medh\=av\={\i}'', etc. The \'S\=udras in the service of the Br\=ahmi\d{n}s are to use the ashes with devotion, with the mantra ``Namah \'Siv\=aya.'' The other ordinary persons can use the Tripundra without any mantra. To besmear the body all over with ashes and to put on the Tripundra is the essence of all Dharma; therefore this should be used always. The ashes from the Agnihotra Sacrifice or from Viraj\=agni (Viraj\=a fire) are to be carefully placed on a clean and pure basin. Cleansing hands and feet, one is to sip (perform \=Achamana) twice, and then, taking the ashes in the hand, utter the five Brahm\=a mantras ``Sadyoy\=atam prapady\=ami,'' etc., and perform short Pr\=a\d{n}\=ay\=ama thrice; he is, then, to utter the seven mantras ``Fire is ashes,'' ``water is ashes,'' ``earth is ashe\'s' ``Teja is ashes,'' ``wind is ashes,'' ``ether is ashes,'' ``All this whatsoever is ashe\'s' and purify and impregnate the ashes with the mantra by blowing out air through the mouth. Then one is to think of Mah\=a Deva, repeating the mantra ``Om Apojyoti,'' etc., and apply dry ashes of white colour all over the body and become sinless. After this he is to meditate on the Mah\=a Vi\d{s}\d{n}u, the Lord of the universe and on the Lord of the waters and repeat again the mantras ``Fire is ashe\'s' and mix water with the ashes. He is, then, to think of \'Siva and apply ashes on his forehead. He is to think of the ashes as \'Siva Himself and

then, with mantras appropriate to his own \=A\'srama (stages of life) use the Tripundra on his forehead, chest and shoulders.

By the middle finger and ringfinger he is to draw the two lines of the ashes from the left to the right and by his thumb draw a third line of ashes from the right to the left. These Tripundras are to be used in the morning, midday and in the evening.

Here ends the Ninth chapter of the Eleventh Book on the rules of \'Sirovrata in the Mah\=apur\=a\d{n}am \'Sr\={\i} Mad Dev\={\i} Bh\=agavatam of 18,000 verses by Mah\=ar\d{s}i Veda Vy\=asa.



