\chapter{On the killing of the enemy of Sudar\'sana in the great war}

1. Vy\=asa said :-- After paying due respects to his new son-in-law, the king Sub\=ahu cheerfully entertained him for six days with variety of good dishes.

2. Thus finishing off the marriage ceremony, the king after consulting with his ministers, presented the bridegroom and the bride various jewels and ornaments and other things given naturally on marriage occasions.

3. Then the king of Benares, of brilliant splendour, heard from his messengers that the kings had obstructed the way back of Sudar\'sana and became very absent minded.

4. Then Sudar\'sana, of firm resolve, told his father-in-law ``O king! better now give us order that we may depart. We will go without any fear.

5. O king! First we will halt at the holy hermitage of Bh\=aradv\=aja Muni; and next we will, after due considerations, settle where we would go.

6. O pure one! You need not fear a bit from these kings; the Mother of the Universe, the Bhagavat\={\i} Bhav\=an\={\i} will surely protect us.''

7. Vy\=asa said :-- O king Janamejaya! Hearing thus his son-in-law's orders, the king Sub\=ahu gave him a vast amount of wealth and bade good-bye to him. Sudar\'sana, too, quickly departed.

8. The king Sub\=ahu followed him with a Iong train of soldiers. Thus Sudar\'sana went on, in his journey, fearless.

9. The great hero Sudar\'sana, the descendant of Raghu, with his new consort in the chariot and followed by many other chariots, saw the soldiers of the several kings.

10. The king Sub\=ahu, seeing them, became anxious. But Sudar\'sana, gladly took refuge, with his whole heart, of the all-auspicious Goddess \'Sankar\={\i}.

11. Sudar\'sana began to recite silently the excellent one word seed mantra of the King of Desires (K\=amar\=aja) and, out of its power, he and his wife remained in the chariot without any fear and sorrow.

12. Then all the kings came there with their soldiers to fight with Sudar\'sana and to carry away by force the bride. Thus a loud uproar arose.

13. The king of Benares seeing them wanted to kill them. But Sudar\'sana, the descent of Raghu, desirous of victory, repeatedly asked him not do so.

14. Loud arose, then, the uproar, caused by the sounds of conchshells, bherri, and war drums of the kings on one side and Sub\=ahu on the other, each of the two parties determining to extirpate the other.

15. \'Satrujit prepared himself for the war to destroy his enemy. Yudh\=ajit came there for his help, well equipped with army, etc.

16-17. Some warrior kings remained there as witnesses with their soldiers. Then Yudh\=ajit went in front of Sudar\'sana. His younger brother \'Satrujit, too, attended Yudh\=ajit to kill his brother in the battle field. Then the warriors, overpowered with anger, shot each other with arrows.

18. A great encounter then ensued in the battle field with sharp arrows. The king of Benares hurriedly advanced there, with a great body of army, to relieve his son-in-law.

19. Thus when the dreadful war began to grow more and more horrible, the Goddess Bhagavat\={\i} suddenly appeared there, mounted on Her lion.

20-21. The beauty of Her body was exceedingly lovely; She was adorned with various excellent ornaments and She held various weapons. She wore divine clothings and the beautiful Mand\=ara garland suspended from Her neck up to Her knees. The kings were greatly astonished to see Her. They began to argue ``Who in this Lady, mounted on a lion? Whence has She so suddenly come?''

22-23. Beholding Her, Sudar\'sana told the king of Benares ``O king! Behold! The Divine Mah\=a Dev\={\i} has come here to favour us. She is very merciful. Now I am completely fearless.''

24. Sudar\'sana and Sub\=ahu were highly delighted to see the Beautiful Goddess and bowed down to Her feet with great devotion.

25. Then the lion, the vehicle of the Goddess, roared, making tremendous noise. Hearing the roaring of the lion, all the elephants trembled. At that

time, the winds began to blow violently and the four quarters assumed an awful appearance.

26-27. Then Sudar\'sana told his general to carry soon his forces where the kings were staying, blocking his way. ``What could the vicious kings do now, though they had become very angry? The Goddess Bhagavat\={\i} had come there to save us.

28. Now you all go safely and calmly through the midst of the kings. See! At my remembering Her, She has come here mercifully to save us.''

29-30. The general, on hearing these words, became ready to march by that route. Then Yudh\=ajit, very much infuriated with anger, said to all the kings :-- ``Why are you all so much fear stricken? Kill this Sudar\'sana, stealing away this girl.

31. This lad, weak and without any support, will carry away by force and fearlessly the girl, spiting all the kings; and won't you be able to do anything? This is very strange!

32. Are you afraid to see this one lady on a lion? O high minded kings! Never trifle away this boy; kill him with all attention.

33. Killing him, we will then take away this girl. The jackal can never snatch away the lady under the grasp of a lion.''

34. Thus saying, the king Yudh\=ajit, filled with anger, came to the battle field with \'Satrujit and all his forces.

35-36. That wicked king, drew his bow string well nigh to his ear and shot arrows after arrows, sharpened under stone and by blacksmith at Sudar\'sana, with the object of killing him. Sudar\'sana cut off all those arrows quickly with his own quick going arrows.

37. Thus when the fight grew intense, the Goddess Chandik\=a became very mach enraged and shot arrows at Yudh\=ajit.

38. Assuming diverse forms, the Goddess Durg\=a, holding various weapons the auspicious Mother of the Universe, began to fight terribly in the battle field.

39. \'Satrujit and the king Yudh\=ajit were killed in that terrible battle. Both of them fell dead from their chariots; and a shout of victory arose from the side of Sudar\'san.

40. The uncle and cousin of the king Sub\=ahu were on the side of Yudh\=ajit and were killed. The kings were very much astonished to see them thus lying dead.

41. The king Sub\=ahu, seeing them dead in the battlefield became very glad and began to praise and sing hymns in honour of Durg\=a Dev\={\i} the Destroyer of all difficulties.

42-43. I bow to the auspicious Goddess Jagaddh\=atr\={\i}, again and again; I bow to the Bhagavat\={\i} Durg\=a the bestower of all desires; I always bow down to Her Who is auspicious, peace giving, and the Higher Vidy\=a. O Mother! O Giver of salvation! O Auspicious One! You are pervading the whole Universe, O World Mother! and Upholder of the Universe! I bow down to Thee.

44. O World-mother! O Dev\={\i}! you are devoid of Pr\=akritic qualities; you are full of qualities; beyond mind and speech; one cannot think out your prowess, etc., by one's mind. Mother! you are the Highest Force; ever willing to destroy the miseries of your devoted persons. Your influence is manifest everywhere; what eulogy can I sing of Thee.

45. O Dev\={\i}! You are the Goddess of V\=ak (speech) of all beings; you are the all pervading intelligence, mind, effort, and movements; you are the controller of the minds of all; therefore how can I praise You? O Goddess; You are the Self of all; how can I sing eulogies to You, who are beyond speech and mind, and to the Universal Self.

46. Brahm\=a, Hari and Hara and other higher Devas have not been able to find the limits of your qualities, though they are incessantly chanting your praises; O Goddess! I am the small of the smallest, I am without qualification, and bound by Pr\=akritic qualities; I am ignorant as regards J\={\i}va and Brahm\=a. O Mother! I will never be able to describe Your characteristics that are unfathomable.

47. O Mother! why not good companionships effect the fulfilment of one's desires. The purification of my heart has been effected incidentally. O Mother! my son-in-law is wholly devoted to you; accidentally there has arisen the connection between him and me and it is on account of his connection that I have been able to see You.

48. O Mother! Today I have got without any restraint and control of passions, and sam\=adhi, the rare vision of You, who is wanted to be seen even by Brahm\=a, Hari and Hara, Indra and the other Devas and by the Munis, who have attained their realisation. Therefore who is there in this Trilok\={\i}, that is so fortunate as I am.

49. O Bhav\=an\={\i}! Where am I, void of intelligence and where is the rare vision of You, Who is the only medicine of this disease of the ocean of world? Still, O Mother! Who is worshipped by the Devas, I have got Your vision. Now I have come to know that You always show mercy to Your Bhaktas, who are in their Bhavas (mental images of your Self).

50. O Goddess! You have saved Sudar\'san in this great war crisis and You have slain these two powerful enemies. How can I describe your

prowess in this matter? This I have understood that Your Holy Character ever shows mercy on Your devotees.

51. O Goddess! Again this is not a matter to be wondered at, if one considers; for You are protecting this whole universe, moving and unmoving; and accordingly You have now protected, out of Your mercy, your Bhakta Sudar\'sana, the son of Dhruvasandhi, by killing his enemy.

52. O Bhav\=an\={\i}! It is not merely for the protection of your Bhaktas, engaged in Your service, that You shew this favour but also to extol the meritorious deeds of your Bhaktas that You do such things; otherwise how is it that this Bhakta saintlike Sudar\'sana, by marrying my daughter, has got victory in this battle field?

53. O Mother! You are fully capable to destroy the fear of birth and death. What wonder is there that you fulfill the desires of your Bhaktas? The Bhaktas extol You by characterising You as Sagu\d{n}a (full of qualities), Nirgu\d{n}a (devoid of any quality) and Ap\=ar\=a, beyond all merits and demerits.

54. O Goddess! O Bhuvane\'svar\={\i}! I am fortunate that I have been able to see You, and thus all my duties have become crowned with success. O Mother! I have no practices in the shape of Your meditation, etc. nor do I know any seed mantras of Yours; today I have fully seen Your glory manifested.

55. Vy\=asa said :-- Thus extolled by the king Sub\=ahu, the Goddess Bhagavat\={\i}, the Bestower of the Absolute Freedom, was pleased and said ``O thou, practiser of good vows! Ask boon from Me.''

Thus ends the twenty third chapter on the killing of the enemy of Sudar\'sana in the great war, in \'Sr\={\i}mad Dev\={\i} Bh\=agavatam of 18,000 verses by Mahar\d{s}i Veda Vy\=asa.



