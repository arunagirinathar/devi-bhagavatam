\chapter{1-22. The Dev\={\i} said :-- "O Devas! You are not at all worthy to see this My Wonderful Cosmic Form. Where are Ye! and where is this My Form! But it is my affection towards the Bhaktas that I have shewn to you all this great form of mine. Nobody can see this form without My Grace; the study of the Vedas, the Yoga, the gift, the Sacrifice, the austerities or any other S\=adhanas are quite incompetent to make this form visible to anybody. O King of mountains! Now hear the real instructions. The Great Self is the only Supreme Thing in this world of M\=ay\=a (Illusions). He it is that under the various Up\=adhis of an actor and enjoyer performs various functions leading to the Dharma (righteousness) and the Adharma (unrighteouss). Then he goes into various wombs and enjoys pleasure or pain according to his Karma. Then again owing to the tendencies pertaining to these births he becomes engaged in various functions and gets again various bodies and enjoys varieties of pleasures and pains. O Best of Mountains! There is no cessation of these births and deaths; it is like a regular clockwork machine; it has no beginning and it goes on working to an endless period. Ignorance or Avidy\=a is the Cause of this Sams\=ara. Desire comes out of this and action flows thence. So men ought to try their best to get rid of this Ignorance. O King of Mountains! What more to say than this that the Goal of life is attained when this Ignorance is destroyed. The highest goal is attained by a J\={\i}va, when he becomes liberated, while living. And Vidy\=a is the only thing that is able and skilful in destroying this Ignorance. (As darkness cannot dispel darkness, so) the Karma done out of Ignorance is Ignorance itself; and such a work cannot destroy Ignorance. So it is not proper to expect that this Avidy\=a can be destroyed by doing works. The works are entirely futile. The J\={\i}vas want again and again the sensual enjoyments out of this Karma. Attachment arises out of this desire; discrepancies creep in and out of this ignorant attachment great calamities befall when such faults or discrepancies are committed. So every sane man ought to make his best effort to get this J\~n\=anam (knowledge). And as it is also enjoined in the \'Srutis that one ought to do actions (and try to live one hundred years) so it is advisable to do works also. Again the \'Srutis declate that the "final liberation comes from}

Knowledge" so one ought to acquire J\~n\=anam. If both these be collectively followed, then works become beneficial and helping to J\~n\=anam. (Therefore the J\={\i}vas should take up both of these.) Others say that this is impossible owing to their contradictory natures. The knots of heart are let loose by J\~n\=anam and the knots are knit more by Karma. So how can they be reconciled? They are so very diametrically opposite. Darkness and light cannot be brought together, so J\~n\=anam and Karma cannot be brought together. Therefore one ought to do all the Karmas as best as one can, as enjoined in the Vedas, until one gets Chitta\'suddhi (the purification of one's heart and mind). Karmas are to be done until \'Sama (the control of the inner organs of senses), Dama (the control of the outer organs of senses), Titik\d{s}\=a (the power to endure heat and cold and other dualities), Vair\=agyam (Dispassion), Sattva Sambhava (the birth of pure Sattva Gu\d{n}a in one's own heart) take place. After these, the Karmas cease for that man. Then one ought to take Sanny\=asa from a Guru (Spiritual Teacher) who has got his senses under control, who is versed in the \'Srutis, attached to Brahma (practising the Yogic union with Brahma). He should approach to him with an unfeigned Bhakti. He should day and night, without any laziness, do \'Srava\d{n}am, Mananam, and Nididhy\=asanam (hearing, thinking and deeply realising) the Ved\=anta sayings. He should constantly ponder over the meanings of the Mah\=av\=akyam "Tat Tyam Asi." "Tat Tyam Asi" means Thou art That; it asserts the identity of the Supreme Self (Brahma) and Embodied Self (J\={\i}v\=atm\=a). When this identity is realised, fearlessness comes and he then gets My nature. First of all, he should try to realise (by reasoning) the idea conveyed by that sentence. By the word "Tat" is meant Myself, of the nature of Brahman; and by the word "Tvam" is meant "J\={\i}va" embodied self and the word "Asi" indicates, no doubt, the identity of these two. The two words "Tat" and "Tvam" cannot be apparently identified, as they seem to convey contradictory meanings ("Tat" implying omniscience, omnipresence, and other universal qualities and "Tvam" implying non-omniscience and other qualities of a limited nature). So to establish the identity between the two, one ought to adopt Bh\=agalak\d{s}man\=a and Ty\=agalak\d{s}man\=a. [N. B.--Bh\=agalak\d{s}man\=a -- kind of Lak\d{s}ma\d{n}\=a or secondary use of a word by which it partly loses and partly retains its primary meaning also called Jahadajahallak\d{s}a\d{n}\=a. Ty\=aga Laksman\=a -- a secondary use of a word by which it loses partly its primary meaning.
23-40. The Supreme Self is Brahma -- Consciousness, endowed with the omniscience, etc., and.the Embodied Self is Limited J\={\i}va Consciousness, etc.) Leaving aside their both the adjuncts, we take the Consciousness,

when both of them are indentical and we come to Brahma, without a second. The example is now quoted to illustrate what is called Bh\=agalak\d{s}a\d{n}\=a and Ty\=agalak\d{s}a\d{n}\=a. "This is that Devadatta" means Devadatta seen before and Devadatta seen now means one and the same person, if we leave aside the time past and the time present take the body of Devadatta only. This gross body arises from the Pa\d{n}ch\={\i}krita gross elements. It is the receptacle of enjoying the fruits of its Karma and liable to disease and old age. This body is all M\=ay\=a; therefore it has certainly no real existence. O Lord of Mountains! Know this to be the gross Up\=adhi (limitation) of My real Self. The five J\~nanendriyas (organs of senses), five Karmendriyas (working organs), the Pr\=a\d{n}a V\=ayus, mind and Buddhi (rational intellect), in all, these seventeen go to form the subtle body, S\=uk\d{s}ma Deha. So the Pundits say. This body of the Supreme Self is caused by the Apanch\={\i}krita five original elements. Through this body, pain and pleasure are felt in the heart. This is the second Up\=adhi of the \=Atman. The Aj\~n\=ana or Primeval Ignoranee, without beginning and indescribable, is the third body of the \=Atman. Know this also to be my third Up\=adhi. When all these Up\=adhis subside, only the Supreme Self, the Brahman remains. Within these three gross and subtle bodies, the five sheaths, Annamaya, Pr\=a\d{n}amaya, Vij\~n\=anamaya, and \=Anandamaya always exist. When these are renounced, Brahmapuchcha is obtained. That is Brahma and My Nature, too. This is the Goal of "Not this, Not this" the Ved\=anta words. This Self is not born nor It dies. It does not live also, being born. (But it remains constant, though It is not born). This Self is unborn, eternal, everlasting, ancient. It is not killed, when the body is killed. If one wants to kill it or thinks It as slain, both of them do not know; this does not kill nor is it killed. This \=Atman, subtler than the subtlest, and greater than the greatest, resides within the cave (the Buddhi) of the J\={\i}vas. He whose heart is purified and who is free from Sankalpa and Vikalpa (doubt and mental phenomena), knows It and Its glory and is free from sorrows and troubles. Know this \=Atman and Buddhi as the charioteer, this body as the chariot, and the mind as the reins. The senses and their organs are the horses and the objects of enjoyments are their aims. The sages declare that the \=Atman united with mind and organs of senses enjoys the objects. He who is non-discriminating, unmindful, and always impure, does not realise his \=Atman; rather he is bound in this world. He who is discriminating, mindful, and always pure reaches the Goal, realises the Highest Self; and he is not fallen again from That. That man becomes able to cross the Ocean of Sams\=ara and gets My

Highest Abode, of the nature of everlasting Existence, Intelligence and Bliss, whose charioteer is Discrimination, and who keeps his senses under control by keeping tight the reins of his mind. Thus one should always meditate intensely on Me to realise the nature of Self by \'Sravanam (hearing), Mananam thinking and realising one's own self by one's Self (pure heart).
41-44. When by the constant practice, as mentioned above, one's heart is fit for Sam\=adhi (being absorbed in the Spirit), just before that, he should understand the meanings of the separate letters in the seed Mantra of Mah\=am\=ay\=a. The letter "Ha" means gross body and the letter "Ra" means subtle body and the letter "\=I" means the causal body; the (dot over the semicircle) is the fourth "Tur\={\i}ya" state of Mine. Thus meditating on the separate differentiated states, the intelligent man should meditate on the aforesaid three V\={\i}jas in the Cosmic body also and he should then try to establish the identity between the two. Before enteriing into Sam\=adhi, after very carefully thinking the above, one should close one's eyes and meditate on Me, the Supreme Deity of the Universe, the Luminous and Self-Ellulgent Brahm\=a.
45-50. O Chief of Mountains! Putting a stop to all worldly desires, free from jealousy and other evils, he should (by constant practice of Pr\=a\d{n}\=ay\=ama) make equal according to the rules of Pr\=a\d{n}\=ay\=ama, the Pr\=a\d{n}a (the inhaled breath) and Ap\=a\d{n}a (the exhaled breath) V\=ay\=us and with an unfeigned devotion get the gross body (Vai\'sv\=anara) indicated by the letter "Ha" dissolved in the subtle body Taijasa. The Taijasa body, the letter "Ra" is in a cave where there is no noise (in the Su\d{s}um\d{n}\=a cave). After that He should dissolve the Taijasa, "Ra" into the Causal body "\=I". He should then dissolve the Causal body, the Pr\=aj\~na "\=I" into the Tur\={\i}ya state Hr\={\i}m. Then he should go into a region where there is no speech or the thing spoken, which is absolutely free from dualities, that Akhanda Sachchid\=ananda and meditate on that Highest Self in the midst of the Fiery Flame of Consciousness. O King of Mountains! Thus men by the meditation mentioned above, should realise the identity between the J\={\i}va and Brahma and see Me and get My Nature. O Lord of Mountains! Thus the firmly resolved intelligent man, by the practice of this Yoga sees and realises the nature of My Highest Self and destroys immediately the Ignorance and all the actions thereof.
Here ends the Thirty-fourth Chapter of the Seventh Book on the Knowledge, Final Emancipation in the Mah\=a Pur\=a\d{n}am, \'Sr\={\i} Mad Dev\={\i} Bh\=agavatam, of 18,000 verses, by Mahar\d{s}i Veda Vy\=asa.



