\chapter{On the rules of Vai\'svadeva}

1-45. N\=ar\=aya\d{n}a said :-- O N\=arada! In connection with this Pura\'schara\d{n}am it comes now to my memory about the rules concerning the Vai\'sva Deva worship (An offering made to the Vi\'svadev\=as; an offering to all deities made by presenting oblations to fire before meals). Hear. The five yaj\~n\=as are the following :-- (1) The Devayaj\~n\=a, (2) Brahm\=a yaj\~n\=a, (3) Bh\=uta yaj\~n\=a, (4) Pitri yaj\~n\=a, and (5) Manu\d{s}ya yaj\~n\=a. Fireplace, the pair of stone pestles, brooms (for sweeping, etc.), sieves and other house-hold things of the sort, wooden mortars (used for cleansing grains from husk) and water-jars, these five are the sources of evils inasmuch as they are the means of killing. So to free one's self from the above sins, one is to sacrifice before the Vai\'svadeva. Never offer oblations of Vai\'svadeva on hearths, on any iron vessel, on the ground or on broken tiles. They are to be offered in any sacrificial pit (Kunda) or on any sacrificial altar. Do not fire the hearth by fanning with hands, with winnowing baskets, or with holy deer skin, etc., but you can do so by blowing by your mouth. For the mouth is the origin of fire. If the fire be ignited by clothes, one is liable to get disease; if by winnowing baskets, then less of wealth comes; if by hands, one's death ensues. But if it be done by blowing, then one's success comes. (There is the danger of catching fire.)

One should sacrifice with curd, ghee, fruits, roots and vegetables, and water and in their absence with fuel, grass, etc., or with any other substances soaked with ghee, curd, P\=ayasa or lastly with water. But never with oil or with salty substances. If one performs the Homa with dry or stale substances, one is attacked with leprosy; if anybody performs Homa with leavings of other food he becomes subdued by his enemy; if one does so with rude and harsh substances, he becomes poor and if one does with salty substances, he meets with a downward course, gets degraded in position and honour. You can offer oblations to Vai\'svadeva with burning coals and ashes from the north side of the fire of the hearth after the preparation of the meals. But you should never offer sacrifices with salty things. The

illiterate Br\=ahma\d{n}a who eats before offering oblations to Vai\'svadeva goes headlong downwards into the K\=ala S\=utra hell. Whatever food that you are intending to prepare, whether they be vegetables, leaves, roots or fruits, offer oblations to Vai\'svadeva with that if, before the Homa be performed of Vai\'svadeva, any Brahmach\=ar\={\i} comes, then take off, for the Homa, first something; and then give to the beggar and satisfy him and tell him to be off. For the Brahm\=ach\=ar\={\i} mendicant can remove any defects that may occur to Vai\'svadeva but Vai\'svadeva is unable to remove any defects that may occur regarding the mendicant Brahmach\=ar\={\i}. Both the Paramahansa or Brahmach\=ar\={\i} mendicant are the masters of the prepared food (Pakk\=anna); so when anybody takes one's food without giving to any of these two, if they happen to come there, he will have to make the Ch\=andr\=aya\d{n}a (religious or expiatory penance regulated by the moon's age, that is, waxing or waning). O N\=arada! After the offering given to Vai\'svanara, one is to offer Gogr\=asa, that is, mouthfuls of food to the cows. Hear now how that is done. The mother Surabhi, the beloved of Vi\d{s}\d{n}u, is always stationed in the region of Vi\d{s}\d{n}u (Vi\d{s}\d{n}upada); so O Surabhi! I am offering you mouthfuls of food. Accept it. ``Salutation to the cows,'' saying this, one is to worship the cows and offer food to them. Hereby Surabhi, the Mother of the cows, becomes pleased. After this, one is to wait outside for a period that is taken to milch a cow, whether any guests are coming. For if any guest goes back disappointed from any house without any food; he takes away all the pu\d{n}yams (merits) of the house-holder and gives him back his own sin. The house-holder is to support mother, father, Guru, brother, son, servants, dependants, guests, those that have come, and Ag\d{n}i (Fire). Knowing all these, he who does not perform the functions of the house-hold is reckoned as fallen from his Dharma both in this world and in the next. The poor house-holder gets the same fruit by performing these five Mah\=a jaj\~n\=as that a rich Br\=ahma\d{n}a gets by performing the Soma Yaj\~n\=a. O Best of the Munis! Now I am talking of the Pr\=an\=agni Hotra or about taking food, knowing the rules of which makes a man free from birth, old age and death and from all sorts of sins. He who takes his food according to proper rules, is freed of the threefold debts, delivers his twenty one generations from the hells, obtains the fruits of all the Yaj\~n\=as and goes unhampered to all the regions of the righteous. Think of the belly as Ara\d{n}i or the piece of wood for kindling the fire
(by attrition), think of the mind as the churning rod, and think of the wind as the rope, and then kindle the fire, residing in the belly; the eyes are to be considered as the sacrificer, (the

Addharyu), and consider fire in the belly as the result of churning. In this fire of the belly, one is to offer oblations for the satisfaction of Pr\=a\d{n}a, etc., the five deities. First of all offer oblations to the Pr\=a\d{n}a V\=ayu with food taken by the forefinger, middlefinger and thumb; next offer oblations to the Ap\=ana V\=ayu with the thumb, middle finger and the nameless (an\=am\=a) finger; next offer oblations to the Vy\=ana V\=ayu (breath) with the thumb, nameless finger and the little finger; next offer oblations to the Ud\=ana V\=ayu with the thumb, forefinger and the little finger and lastly offer oblations to the Sam\=ana V\=ayu with food taken by all the fingers. At the same time repeat respectively the mantras :--

``Om Pr\=a\d{n}\=aya Sv\=ah\=a,''
``Om Ap\=an\=aya Sv\=ah\=a,''
``Om Sam\=an\=aya Sv\=ah\=a,''
``Om Ud\=an\=aya Sv\=ah\=a,''
``Om Vy\=an\=aya Sv\=ah\=a.''

Within the mouth, there is the \=Ahavan\={\i}ya fire; within the heart, there is the G\=arhapatya fire; in the navel, there is the Dhak\d{s}i\d{n}\=agni fire; below the navel, there is the Sabhy\=agni fire and below that there is the \=Avasathy\=agni fire. Think thus. Next consider the Speech as the Hot\=a, the Pr\=a\d{n}a as the Udg\=ath\=a, the eyes as the Addharyu, the mind as the Brahm\=a, the ears as the Hot\=a and the keeper of the Ag\d{n}i, the Ahamk\=ara (egoism) as beast (Pa\'su), Om K\=ara as water, the Buddhi (intellect) of the house-holder as the legal wife, the heart as the sacrificial altar, the hairs and pores as the Ku\'sa grass, and the two hands as the sacrificial ladles and spoons (Sruk and Sruva.) Then think of the colour of the Pr\=a\d{n}a mantra as golden the fire of hunger as the \d{R}i\d{s}i (seer), S\=urya (the sun) as Devat\=a, the chhandas as G\=ayatr\={\i} and Pr\=an\=aya Sv\=ah\=a as the Mantra uttered; also repeat ``Idam\=adityadev\=aya namah'' and offer oblations to the Pr\=a\d{n}a. The colour of the Ap\=ana mantra is milk white. \'Sraddh\=agni is the \d{R}i\d{s}i, the Moon is the Devat\=a, U\d{s}\d{n}ik is the chhandas, and ``Ap\=an\=aya Sv\=ah\=a,'' ``Idam Som\=aya na namah'' are the mantras. The colour of the Vy\=ana mantra is red like red lotuses; the fire Deity Hut\=asana is the \d{R}i\d{s}i, the fire is the Devat\=a; Anustup is the chhandas, ``Vy\=an\=aya Sv\=ah\=a'' and ``Idamagnaye na namah'' are the mantras. The colour of the Ud\=ana mantra is like that of the worm Indra Gopa; fire is the \d{R}i\d{s}i; V\=ayu is the Devat\=a, Brihat\={\i} is the chhandas; ``Ud\=an\=aya Sv\=ah\=a'' and ``Idam V\=ayave na namah'' are the mantras. The colour of the Sam\=ana mantra is like lightning; Ag\d{n}i is the \d{R}i\d{s}i; Parjanya (the rains, water) is the Devat\=a; Pankti is the chhanda; ``Sam\=an\=aya

Sv\=ah\=a'' and ``Idam Parjany\=aya na namah'' are the mantras. O N\=arada! Thus offering the five oblations to the five breaths, next offer oblations to the \=Atman; the Bh\={\i}\d{s}a\d{n}a Vah\d{n}i is the \d{R}i\d{s}i; the G\=ayatr\={\i} is the chhanda; the Self is the Devat\=a; ``\=Atmane Sv\=ah\=a,'' and ``Idam\=atmane na namah'' are the mantras. O N\=arada! He who knows this Homa of Pr\=an\=agnihotra attains the state of Brahm\=a. Thus I have spoken to you in brief the rules of the Pr\=an\=agni hotra Homa.

Here ends the Twenty-Second Chapter of the Eleventh Book on the rules of Vai\'svadeva in the Mah\=apur\=a\d{n}am \'Sr\={\i} Mad Dev\={\i} Bh\=agavatam of 18,000 verses by Mahar\d{s}i Veda Vy\=asa.



