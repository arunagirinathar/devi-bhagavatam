\chapter{On the story of S\=avitr\={\i}}

1-4. N\=ar\=aya\d{n}a said :-- O N\=arada! Hearing the words of Yama, the chaste intelligent S\=avitr\={\i}, replied with great devotion :-- ``O Dharmar\=ajan! What is Karma? Why and how is its origin? What is the cause of Karma? Who is the embodied soul (bound by Karma)? What is this body? And who is it that does Karma? What is J\~n\=ana? What is Buddhi? What is this Pr\=a\d{n}a of this embodied J\={\i}va? What are the Indriyas? And what are their characteristics? And what are the Devat\=as thereof? Who is it that enjoys and who is it that makes one enjoy? What is this enjoyment (Bhoga)? And what is the means of escape from it? And what is the nature of that State when one escapes from enjoyment? What is the nature of J\={\i}vatm\=a? And what of Param\=atm\=a? O Deva! Speak all these in detail to me.''

5-21. Dharma said :-- Karma is of two kinds: good and bad. The Karma that is stated in the Vedas as leading to Dharma is good; all other actions are bad. The God's service, without any selfish ends (Sankalpa) and without the hope of any fruits thereof (ahaituk\={\i}), is of such a nature as to root out all the Karmas and gives rise to the highest devotion to God. A man who is such a Bhakta of Brahm\=a becomes liberated, so the \'Srutis say. Who then does the Karma and who is it that enjoys? (i.e., no such body). To such a Bhakta to Brahm\=a, there is no birth, death, old age, disease, sorrow nor any fear. O Chaste One! Bhakti is two-fold. This is stated by all in the \'Srutis. The one leads to Nirv\=a\d{n}a and the other leads to the nature of Hari. The Vai\d{s}\d{n}avas want the Bhakti to Hari, i.e., the Sagu\d{n}a Bhakti. The other Yogis and the best knowers of Brahm\=a want the Nirgu\d{n}a Bhakti. He who is the Seed of Karma, and the Bestower for ever the fruits of Karma, Who is the Karma Incarnate and the M\=ula Prakriti, is the Bhagav\=an; He is the Highest Self. He is the Material Cause of Karma. Know this body to be by nature liable to dissolve and die. Earth, air, \=ak\=a\'sa, water, and fire; these are the threads, as it were, of the work of creation of Brahm\=a, Who is of the nature of Being. ``Deh\={\i}'' or The Embodied Soul is the Doer of Karma, the Kart\=a; he is the enjoyer; and \=Atm\=a (self) is the prompter, the stimulator within to do the Karma and enjoy the fruits thereof. The experiencing of pleasures and pains and the varieties thereof is known as Bhoga (enjoyment). Liberation, Mukti is the escape therefrom.

The knowledge by which \=Atm\=a (sat) and M\=ay\=a (Asat) are discriminated is called J\~n\=anam (Brahm\=a J\~n\=anam). The knowledge is considered as the root discriminator of various objects of enjoyments (i.e., by which the various objects are at once recognised as different from \=Atm\=an). By Buddhi is meant the right seeing of things, (as certain) and is considered as the seed of J\~n\=anam. By Pr\=a\d{n}a is known as the different V\=ayus in the body. And this Pr\=a\d{n}a is the strength of the embodied. Mind is the chief, the best, of the senses, it is a part of \=I\'svara; its characteristic is its doubtful uncertain state. It impels to all actions, irresistible. It is inascertainable, invisible; it obstructs the J\~n\=ana. The senses are seeing, hearing, smelling, touching and tasting. These are the several limbs, as it were, of the embodied and the impellers to all actions. They are both enemies and friends as they give pain (when attached to worldly objects) and happiness (when attached to virtuous objects) both. The Sun, V\=ayu, Earth, Brahm\=a and others are their Devat\=as. The J\={\i}va is the holder, the sustainer of Pr\=a\d{n}a, body, etc. The Param\=atm\=a, the Highest Self, is the Best of all, Omnipresent, transcending the the Gu\d{n}as, and beyond Prakriti. He is the Cause of all causes and He is the Brahm\=a Itself. O Chaste One! I have replied, according to the \'S\=astras to all your questions. These are J\~n\=anas of the J\~n\=anins. O Child! Now go back to your house at pleasure.

22-30. S\=avitr\={\i} said :-- Whither shall I go, leaving my Husband and Thee, the Ocean of Knowledge? Please oughtest to answer the queries that I now put to Thee. What wombs do the J\={\i}vas get in response to which Karmas? What Karmas lead to the Heavens? And what Karmas lead to various hells? Which Karmas lead to Mukti? And which Karmas give Bhakti? What Karmas make one Yogi and what Karmas inflict diseases? Which Karmas make one's life long? or short? Which Karmas again make one happy? And what Karmas make one miserable? Which Karmas make one deformed in one's limbs, one-eyed, blind, deaf, lame or idiotic? Which Karmas again make one mad? Make one very much avaricious or of a stealing habit? What Karmas make one possess Siddhis? Or make one earn the four Lokas S\=alokya, etc.? What Karmas make one a Br\=ahmi\d{n} or an ascetic? Or make one go to Heaven or Vaikuntha? What Karmas enable one to go to Goloka, the par excellence and free from all diseases? How many are the hells? What are their names and how do they appear? How long will one have to remain in each hell? and what Karmas lead to what diseases? O Deva, now tell me about these that I have asked to you and oblige.

Here ends the Twenty-Eighth Chapter of the Ninth Book on the story of S\=avitr\={\i} in \'Sr\={\i} Mad Dev\={\i} Bh\=agavatam of 18,000 verses by Mahar\d{s}i Veda Vy\=asa.



