\chapter{On the story of the King Hari\'schandra}

1-7. Vy\=asa said :-- O King! When there was going on in the King's palace, the grand festivities for the son's birth ceremonies, Varu\d{n}a Deva came there in the holy Br\=ahmi\d{n} form. ``Let welfare be on you.'' Saying this, Varu\d{n}a began to say :-- ``O King! Know me to he Varu\d{n}a. Now hear what I say. O King! Your son is now born; therefore perform sacrifices in honor to me with your son. O King! Your defect of not having a son is now removed; so fulfil what you promised before.'' Hearing these words, the King began to think ``Oh! Only one lotus-faced son is born to me; how can I kill it. On the other hand, the powerful Regent (Lokap\=ala) of one quarter is present in Br\=ahma\d{n}a form; and it never behoves one to show disrespect to a Deva or to a man who wishes welfare to us. Again it is very difficult to root out the affection for a son; so what am I to do now? How shall I preserve my happiness due to the birth of my son.'' The King, then, with patience bowed down to him and worshipped him duly and humbly spoke to him in beautiful words, pregnant with reason.

8-10. O Deva of the Devas! I will obey your order no doubt and I will perform your sacrifice according to the Vedic rites and with profuse Dak\d{s}i\d{n}\=as (remuneration to priests, etc.) But, when in a sacrifice human beings are immolated as victims, both the husband and wife are entitled to the ceremony. Father becomes purified on the tenth day and mother on the expiration of one month after the son's birth; so how can I perform the sacrifice until one month expires! You are omniscient and the master of all the beings; and you know what is Nitya Dharma. So, O Varu\d{n}a Deva! I want one month time; and show mercy thus on me.

11-19. Vy\=asa said :-- O King! The King Hari\'schandra saying thus, Varu\d{n}a Deva spoke to the King :-- ``O King! Welfare be unto

you! Do your duties; I am now going back to my place. O King! I will come again after one month. Better finish the natal ceremonies and the N\=amakara\d{n}a ceremony regularly and then perform my sacrifice.'' O King! When Varu\d{n}a Deva turned his back, the King began to feel happiness. Then the King gave as gifts millions of cows, yielding plenty of milk and ornamented with gold, and mountains of Til, sesamums to the Br\=ahmi\d{n}s versed in the Vedas and kept his name, with formal ceremonies as Rohit\=a\'sva. When one month became complete, Varu\d{n}a Deva came again in a Br\=ahmi\d{n} form and frequently said :-- ``O King! Start the sacrifice just now!'' The King, on seeing the God of Waters, at once fell into an ocean of anxieties and sorrows; he then bowed down and worshipping him as a guest, spoke to him with folded palms :-- ``O Deva! It is to my great fortune that you have landed your feet at my place; O Lord! My house has been sanctified to day. O Deva! I will do, no doubt, your desired sacrifice according to the rites and ceremonies. But see, the victims that have not their teeth come as yet are not fit for a sacrifice; so the versed Pundits say; so I have settled I would perform your great sacrifice, as desired by you, when the teeth will come out of my son.''

20-41. Vy\=asa said :-- O Lord of men! Hearing thus, Varu\d{n}a spoke ``Let it be so'' and went away. The King Hari\'schandra became glad and passed his days in enjoyments in his household. When the teeth of the child got out, Varu\d{n}a knew it and came again in a Br\=ahmi\d{n} garb in the palace and spoke ``O King! Now commence my sacrifice.'' Seeing the Br\=ahmi\d{n} Varu\d{n}a there, the King, too, bowed down and gave him a seat and shewing all respects to him, worshipped him. He sang hymns to him and very humbly said with his head bent low :-- ``O Deva! I will perform your desired sacrifice with plenty of Dak\d{s}'i\d{n}\=as according to rites and ceremonies. But the child's Ch\=ud\=akara\d{n}a (the ceremony of tonsure) is not yet done; so the hairs that were at the birth time are still there and the child cannot be fit for sacrifice as long as those hairs exist. So I have heard from the elderly persons. O Lord of Waters! You know the \'S\=astric rules; kindly wait till the Ch\=ud\=akara\d{n}a is over. When the child will have his head shaven, I will certainly perform your sacrifice; there is no doubt in this.'' Hearing these words, Varu\d{n}a spoke to him again :-- ``O King! Why are you deceiving me like this so often? O King! Now you have all the materials ready for the sacrifice; only for your filial affection you are deceiving me. However, if, after the ceremony of tonsure, you do not perform my sacrifice, I will be angry and I will curse you. O King! I am going for the present; but see do not tell lies, being born in the family of Ik\d{s}\=aku.'' Instantly Varu\d{n}a

disappeared; the King, too, felt himself happy in his household. When the ceremony of tonsure was commenced and grand festivities were held, on the occasion Varu\d{n}a soon came again to the King's palace. The queen was then sitting before the King with the child in her lap when Varu\d{n}a came up there. The Br\=ahmi\d{n} Varu\d{n}a then appeared like a Flaming Fire and spoke to the King in a clear voice :-- ``O King! Start the sacrifice.'' Seeing him, the King was confused with terror and with folded palms, quickly bowed down to him. After worshipping him duly, he very humbly said :-- ``O Lord! Today I will perform your sacrifice. But kindly hear with attention my saying and then do what is advisable. O Lord! If you approve of this as reasonable, I then open my heart to you. The three Var\d{n}as Br\=ahma\d{n}as, K\d{s}attriyas, and Vai\'syas become Dv\={\i}jas (twice-born) only when they are duly purified according to proper rules and ceremonies; without any such purifications they are certainly \'S\=udras. So the Pundits versed in the Vedas declare. My child is now an infant only; so it is like a \'S\=udra. When his thread ceremony (Upanayan) will be performed, he will then be fit for the sacrifice; this the Veda \'S\=astras declare. The K\d{s}atttriyas are so purified in their eleventh year; the Br\=ahma\d{n}as in their eighth year and the Vai\'syas in their twelfth year. So, O Lord of the Devas! If you feel pity for your this humble servant, then wait till the Upanayana ceremony is over, when I will perform your grand sacrifice with my son. O Bibhu! You are the Lokap\=ala; specially you are conversant with all the \'S\=astric rules and have acquired the knowledge of Dharma. If you think my saying as true, then go to your home.''

42-51. Vy\=asa said :-- Hearing these words, Varu\d{n}a's heart was filled with pity and so he went away instantly, saying ``Let it be so.'' Varu\d{n}a going away, the King felt very glad and the queen, knowing the welfare of the son became glad too. Then the King gladly performed his state duties. After some time, the child grew ten years old. Consulting with the peaceful Br\=ahma\d{n}as as well as his ministers, he collected materials for the Upanayana ceremony befitting his position. When the eleventh year was completed by his son, the King arranged everything for the thread ceremony but when his thoughts turned to Varu\d{n}a's sacrifice, he became very sad and anxious. When the thread ceremony began to be performed, the Br\=ahmi\d{n} Varu\d{n}a came there. Seeing him, the King instantly bowed down and standing before him with clasped palms, gladly spoke to him :-- O Deva! My son's Upanayana being over, now my son is fit for the victim in the sacrifice; and by your grace, my sorrow that was within me as not having a son, has vanished. I speak truly before

you that, O Knower of Virtue! after some mere time I have desired to perform your sacrifice with plenty of Dak\d{s}i\d{n}\=as. In fact, when the Sam\=avartan ceremony will be over, I will do as you like. Kindly wait till then.

52-62. Varu\d{n}a said :-- O Intelligent One! You are very much attached to your son now and so by various reasonable plays of intellect, you are repeatedly deceiving me. However, I am going home today at your request but know certain that I will come again at the time of the Sam\=avartan ceremony. (N. B.:-- Sam\=avartan means the return home especially of a pupil from his tutor's house after finishing his course of study there.) O King! Thus saying, Varu\d{n}a went away and the King became glad and began to perform duly his various duties. The prince was very intelligent; and as he used to see Varu\d{n}a coming, now and then, at the time of the ceremonies, he became very anxious. He then made enquiries outside hither and thither and came to know of his own being about to be killed and he desired to quit the house instantly. He then consulted with the minister's sons and came to a final conclusion and went out of the city to the forest. When the son had gone to the forest, the King became very much afflicted with sorrow and sent messengers in quest of him. When some time passed away, Varu\d{n}a came to his house and spoke to the distressed King :-- ``O King! Now perform your desired Sacrifice.'' The King bowed down to him and said :-- ``O Deva! What shall I do now? My son has become afraid and has gone away. I do not know where he has gone. O Deva! My messengers have searched for him in difficult places in mountains, in the hermitages of the Munis, in fact, in all the places; but they have not been able to find him out anywhere. My son has left his home; order now what I can do. O Deva! You know everything; so judge I have got no fault in this matter. It is certainly luck and nothing else.''

63-66. Vy\=asa said :-- O King! Hearing these words of the King, Varu\d{n}a became very much angry and when he saw that he was deceived so many times by the King, he then cursed, saying :-- ``O King! As you have cheated me by your deceitful words, so you be attacked by dropsy and be severely pained by it.'' Thus cursed by Varu\d{n}a, the King was attacked with that disease and began to suffer much. Cursing thus, Varu\d{n}a went back to his own place and the King was much afflicted with that terrible disease.

Here ends the Fifteenth Chapter in the Seventh Book on the story of the King Hari\'schandra in the Mah\=apur\=a\d{n}am \'Sr\={\i} Mad Dev\={\i} Bh\=agavatam of 18,000 verses by Mahar\d{s}i Veda Vy\=asa.



