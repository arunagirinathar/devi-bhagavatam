\chapter{On the description by N\=arada of his own Moha}

1-13. Vy\=asa said :-- O King! When I asked him why this delusion overtook me, Mahar\d{s}i N\=arada smiled and said :-- ``O son of Par\=a\'sara! You are thoroughly acquainted with all the Pur\=a\d{n}as. Why then are you making this question about the cause of my Moha (delusion). No embodied soul can exist in this Sams\=ara without this Moha. Brahm\=a, Vi\d{s}\d{n}u, Rudra, and the other Devas, \'Sanaka, Kapila and the other \d{R}i\d{s}is, all these are surrounded by M\=ay\=a and are thus travelling in this path of Sams\=ara. The people know me as a J\~n\=anin; but I, too, am deluded like an ordinary man. I am now speaking to you as certain as anything my of previous history now. I was deluded by M\=ay\=a; hear it attentively. O Son of V\=asav\={\i}! Great troubles and pains were felt by me before, due to this Moha, for my wife. One day Parvata and I, the two Devar\d{s}is, went out together from the Devaloka to see the excellent portion of the earth named Bh\=arata and came to the Martyaloka or the land of the mortals. We then began to travel over various places and saw the places of pilgrimages and the holy places and the beautiful hermitages of the Munis. Before we went out from the Devaloka, we consulted with each other and entered into this agreement that we would not hide our feelings from each other, whether they be good or bad, while we would travel over the face of the earth. Whether it be our desire to get food, or wealth or women for enjoyment, whatever arises in the mind of any of us, we would express that freely amongst ourselves. Thus making an agreement, we went out in right earnest as Munis to travel over the face of this earth. Thus roaming all over the face of the earth, at the end of the summer season, when the rainy season commenced we came to the beautiful city of the King named \'Sanjaya. The King showed us great respect and worshipped us with devotion. Since then we remained for four months at his house.

14-33. During the four months of the rainy season, the roads are always almost impassable; it is, therefore, wise to stay at one place. For eight months, the Dv\={\i}jas should always remain abroad on some work

or other. Thinking all these, we two began to stay in the house of the King \'Sanjaya. That liberal minded King gladly and with respect kept us as his guests and tendered to us all our requirements. The King had a very beautiful daughter named Damayant\={\i}, with good teeth. The King ordered her to take care of us. That large-eyed princess, of great discrimination, was very energetic, day and night. She began to serve both of us. In due time she gave us water for our bath, excellent meat, food, towels for cleaning and rubbing our faces, in fact, everything what we desired. She kept ready for us whatever we desired, fans, seats, beds, whatever were necessary for us. Thus she began to serve. We were also engaged in the study of our Vedas and in those practises that were approved by the Vedas. O Dvaip\=ayana! I used to sing, then, with lute in my hands, the sweet lovely S\=ama G\=ayatr\={\i} songs in tunes and good Svaras. The princess herself appreciated the songs and when she heard these S\=ama songs ravishing to one's mind, she became attached to me and showed signs of affection. Day by day the attachment towards me grew stronger. Seeing her attached to me, my mind also became attached to her. Thus that princess indulged in amorous sentiments towards me and began to make slight distinctions between the food and other things offered to me and Parvata. I got warm water for my bath and Parvata used to get cold water; I got nice curds when food was served to me whereas Parvata got only whey. I got nice white bedding for myself to sleep on whereas Parvata had merely a dirty sheet to lie down. Thus the princess began to serve me with great love and devotion but not so she served Parvata. The fair lady began to look at me with eyes of love; not so towards Parvata. Parvata was very much surprised to see all this and thought within himself, ``What is this?'' Parvata, then, asked me in private :-- ``O N\=arada! Speak out to me truly in detail. The princess shews with much gladness and affection her deep love towards you; she serves you with dainty dishes but she behaves not so with me. I therefore suspect when I see all these distinctions made between you and me, that the daughter of the King \'Sanjaya wants with her heart and soul to make you her husband. And you also want to make her your wife. 1 have come to know this by signs and symptoms; for affection and love reigning inside can be made out by outward expressions of eyes and face. Whatever this be, O Muni! Now speak truly to me; do never tell a lie. When we went out from the Heavens, we made out that agreement; now remember that.''

34-42. N\=arada said :-- Thus questioned suddenly by Parvata, I became very much abashed and said :-- ``O Parvata! This large-eyes princess is ready to marry me and I am also very much attracted towards

her.'' When Parvata heard all these, he became very much angry and uttered repeatedly, ``Fie! O N\=arada! Fie! O N\=arada! First you swore on oath and then you deceived me afterwards. Therefore, O Deceiver of friends! I curse you and let your face become that of a monkey.'' When the high-souled Parvata cursed thus, the face turned immediately into that of a monkey, elongated and distorted. I did not excuse him, though he was my sister's son. I also got angry and cursed him, ``Certainly, your journey to the Heavens will be stopped. You will not be able to go to Heaven. O Parvata! When you cursed me so heavily for so trivial a fault of mine, I see you are very mean. Whatever it be, you will have to remain on earth so long.'' At this Parvata became very sad and went out of the city. My face became immediately like that of a monkey. The daughter of the King became very sorry to see my face thus distorted into that of a monkey. I did not see her glad as she was before; but her desire to hear my playing with my lute remained the same as before.

43-52. Vy\=asa said :-- O Muni! What happened next? How did you get yourself rid of your curse and how did you get your man-like face? Whither did Parvata \d{R}i\d{s}i go! When and how did you again re-unite with each other? Kindly describe all these to me in detail. N\=arada said :-- ``O Highly Intelligent One! What shall I say about the nature of M\=ay\=a? When Parvata went away angrily, the daughter of the King began to serve me with greater care than before. I remained there, though Parvata went away, and seeing my face monkey-like, I became very dejected and sorry and was specially troubled with the care and anxiety what would happen to me hereafter? The King \'Sanjaya saw that his daughter Damayant\={\i} was slipping into her youth and asked the prime minister about her marriage. He said :-- ``The time of marriage of my dear daughter has now come; I will now marry her in accordance with due rites and ceremonies. Now tell me particularly about a prince worthy of her, as we like, in beauty, qualifications, largeheartedness, calmness, patience and heroism and who is of a good family.'' The minister said :-- ``O King! There are many princes on the face of this earth, worthy in all respects, of your daughter. Whomever you like, you can call on him and give him your daughter with elephants, horses, chariots, wealth, gems and jewels.''

53-57. Damayant\={\i}, knowing the intention of his father informed the King of her own desire by her nurse and attendant. The nurse went to the King and said :-- ``When my father will sit at his ease and comfort you would go and speak to him in private that I am enchanted with the

enchanting N\=ada sound of the great lute played by the Mahar\d{s}i N\=arada and have selected him as my bridegroom. No other person will be dear to me. O Father! Marry me with N\=arada and thus fulfil my desire; O Knower of Dharma! I won't marry anybody but N\=arada. O Father! I am now merged in the N\=ada-ocean (sound ocean) of bliss, sweet and joyful, void of anything destructive of happiness, void of Nakra, alligators, and fishes, Timingala, etc. (injurious animals) and without any salty taste; my mind won't be satisfied with any other thing.''

Here ends the Twenty sixth Chapter of the Sixth Book on the description by N\=arada of his own Moha in the Mah\=apur\=a\d{n}am \'Sr\={\i} Mad Dev\={\i} Bh\=agavatam of 18,000 verses by Mahar\d{s}i Veda Vy\=asa



