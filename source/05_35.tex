\chapter{On the receiving of the boons by the King Suratha and the Vai\'sya Sam\=adhi}

1-12. Vy\=asa said :-- O King! Hearing thus the \d{R}i\d{s}i's words, the king Suratha and Vai\'sya, who were very distressed in their minds, became very much comforted and bowed down to the Muni with great humility and modesty. Their eyes expressed their gladness and their hearts were filled with loving devotion. Both of them, then, clever in speaking and of calm and quiet temper, began to address him with their folded hands. O Bhagav\=an! We were passing our days in a very humble and distressed spot; we are today purified by your good words, just as the country was rendered pure by Bhag\={\i}ratha when he brought down the river Ganges here. The saints, adorned with purely good qualities, are incessantly engaged in doing good to others and how the people can be made happy. O Intelligent One! Surely we have come to this auspicious \=A\'srama owing to our past good deeds (in previous births) and all our miseries are therefore brought to their ends today. There are good many persons that roam in this world for their selfish ends; very few there exist like you who are always ready to do good to others. O Muni! True that I am very much distressed

but this Vai\'sya is more distressed than me. Both of us, very much afflicted by the miseries of the world, have come gladly to your \=A\'srama and are relieved of our bodily sufferings by your sight; and now, hearing your words, we are relieved also of our mental pain and sufferings. O Br\=ahma\d{n}a! We are very much blessed and our objects have been gained by your nectar-like words; O Thou, the Ocean of mercy! You have purified us, out of your unbounded mercy. We are quite tired of this world; knowing this, do you lead us beyond this world by holding our hands and by initiating us with Mantrams. O Best of Munis! We will first of all practise a very hard Tapasy\=a (asceticism) and worship Bhagavat\={\i}, the Awarder of happiness; then, seeing Her, we will go to our respective abodes. Now we expect the nine-lettered Mantram of the Dev\={\i} from your mouth and practising the Navar\=atra varam we will fast and meditate on the Mantram.

[Note :-- The nine-lettered Mantram is ``Om Mahi\d{s}amardinyai Sv\=ah\=a.'' Instead of Om, any of the following may be used :-- Hr\={\i}m, Kl\={\i}m, Aim, Str\={\i}m, or H\=um mentioned in S\=arad\=a Tilaka, N\=ar\=ayan\={\i} Tantra, or in Vi\'svas\=ara Tantra (see page 125 of Tantra S\=ara ).]

13-30. Vy\=asa said :-- O King! When the king and Vai\'sya prayed thus to the Muni Sumedha, the best of the Munis, gave them the auspicious Mantram with its seed (V\={\i}ja) and as well what is to be meditated (Dhy\=an). On getting the Mantram (with \d{R}i\d{s}i, Chhanda, seed \'Sakti, and Devat\=a) duly, they welcomed the Muni and with his permission went to the holy bank of a river. Both of them were of delicate frames and both of them were fully determined; they went to a very solitary place and selected their place and took their seats there. There they spent one month in repeating silently the Mantram and in chanting the three glorious deeds of Chand\={\i}. In this short period of one month, they became very much attached to the lotus-feet of Bhav\=an\={\i} and their minds were also much pacified. They attended to no other business; only they used to go to the Muni once a day and bowing down before him they returned to their own seats of Ku\'sa grass and gave themselves up to the meditation of the Dev\={\i} and always repeated silently their Mantrams. O King! One year thus passed away; they then abstained from taking fruits and subsisted on the leaves of trees. Thus engaged in meditation and asceticism they passed away another year sustaining themselves with dry leaves only. O King! When the two years thus passed, they got in their dreams the beautiful vision of the Goddess Bhagavat\={\i}. They were very much delighted to see in their dreams the Ambik\=a Dev\={\i} in red robes and decorated with various ornaments. They practised tapasy\=a in the third year with water as

their only food. Thus when they found that, after practising the tapas for three years, they could not see face to face the Dev\={\i} they became very anxious to see the Dev\={\i} and thought thus :-- ``When we have not been so fortunate as to see the Dev\={\i}, Who art the Bestower of peace and happiness to the human beings, we will then leave our bodies, in deep distress and sorrow!'' Thus thinking, the King prepared a beautiful triangular Kunda (pit), firm and of one hand measure. Lighting a fire in that pit, the King began to cut off slices of flesh from his own body and offered them as oblations to the fire. The Vai\'sya, too, then did the same. O King! Both of them were very much excited and began to offer their blood as oblation to the Dev\={\i}. The Dev\={\i} Bhagavat\={\i}, then, seeing them thus grieved, and that their hearts were over flown with devotion towards Her, appeared direct before them and said thus :--

31-32. O King! You are my favourite devotees; I am pleased with your Tapasy\=a; now ask whatever you desire; I will grant you that boon. Then She spoke to the Vai\'sya :-- ``O Highly Fortunate One! I am pleased; ask without any delay any boon; I will grant that just now.''

33-52. Vy\=asa said :-- O King! Hearing thus the words of the Dev\={\i}, the king Suratha was very much delighted and said thus :-- ``O Dev\={\i}! Grant me this boon that I be able today to conquer my enemies with my own power and that I may regain my kingdom.'' The Dev\={\i} then spoke to him thus :-- ``O King! Go to your own abode; your enemies are now enfeebled and will certainly be defeated.

(Note :-- The Dev\={\i} has now withdrawn Her own power from the enemies with which they were filled before. This is the result of the real sacrifice to the Dev\={\i}.)

O Fortunate One! Your ministers will all come and prostrate themselves before your feet and will be obedient to you; you can now go back to your city and govern your subjects happily. O King! Thus reign for Ajuta years (10,000 years) over your widely extended dominion; then, when you quit your body, you will again be born from S\=urya, and be known widely as S\=avar\d{n}i Manu.'' Vy\=asa said :-- O King! The pure-natured Vai\'sya said with folded hands :-- ``O Dev\={\i}! I have nothing to do with house, sons, nor wealth. O Mother! The house, wealth and sons, all these are so many sources of bondage to this world and are very transitory like dreams. Therefore give me knowledge so that my ties to this world be cut asunder. Persons who are devoid of knowledge, those fools are merged in this ocean of world. The wise never prefer this Sams\=ara; therefore they can cross this world. Vy\=asa

said :-- O King! Hearing this, the Mah\=am\=ay\=a said to the Vai\'sya, that stood in front of Her thus :-- ``O Vai\'sya! No doubt you will acquire knowledge.'' Thus granting boons to them, the Dev\={\i} then and there disappeared. After the Dev\={\i} had disappeared, the King bowed down to the Muni, mounted on his horse and expressed a desire to go back to his kingdom. Just at that time all his ministers and subjects came humbly before him, bowed down to him and standing before him with folded hands, said :-- ``O King! Your enemies all had acted very sinfully; hence they were all slain in battle; you be pleased now to remain in your city, free from any enemy and govern your subjects.'' The King, hearing thus, bowed down to the Muni and with his permission, started towards his kingdom, surrounded by his ministers. On regaining his own kingdom, wife, relatives and kinsmen he began to enjoy the sea-girt earth. On the other hand, the Vai\'sya became illumined with the Spiritual Knowledge and all his connections and attachments being completely severed, became free from all bondages. He became liberated in his lifetime and travelled always from one place of pilgrimage to another and passed away his time in singing the glorious deeds of the Dev\={\i}. O King! Thus I have described to you the most wonderful character of the Dev\={\i}, what fruits were obtained by the King and the Vai\'sya on their worshipping Her, how the Daityas were killed by Her and about Her auspicious appearances on this earth. Oh! Such is the glory of the Dev\={\i}, leading to fearlessness amongst Her devotees. The mortal who hears constantly this excellent pure narrative of the Dev\={\i} Bhagavat\={\i}, gets truly all the best and wonderful pleasures of this world. No doubt anybody who hears this wonderful incident, will obtain knowledge, liberation, fame, happiness and purity. The essence of all religions lies in this narration; therefore it leads, above all, to Dharma, Artha, Kama, and Mok\d{s}a (religion, wealth, desire and liberation). It grants all desires to human beings.

53-54. S\=uta said :-- O \d{R}i\d{s}is! The Mahar\d{s}i Vy\=asa, the son of Satyavat\={\i}, versed in all the departments of knowledge, asked by the King Janamejaya, narrated to him this divine Samhit\=a. The character of Chandik\=a, the killing of the Daitya \'Sumbha, were thus narrated by the merciful Muni Veda Vy\=asa. O Munis! I, too, have described to you the main points of this Pur\=a\d{n}a. Here ends the Fifth Book.

Here ends the Thirty-fifth Chapter of the Fifth Book on the receiving of the boons by the King Suratha and the Vai\'sya Sam\=adhi in the Dev\={\i} Bh\=agavatam, the Mah\=a Pur\=a\d{n}am, of 18,000 verses by Mahar\d{s}i Veda Vy\=asa.

The Fifth Book Completed.



