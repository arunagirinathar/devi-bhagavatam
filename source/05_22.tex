\chapter{On the eulogising of the Dev\={\i} by the Devas}

1-7. Vy\=asa said :-- O King! When the Devas were all defeated, \'Sumbha began to govern all their kingdoms; thus one thousand years passed away. The Devas, on the other hand, deprived of their kingdoms, were all drowned in an ocean of cares and anxieties; at last they began to feel very much and were greatly afflicted. They asked with reverence their own Guru Brihaspati, ``O Guru! What are we to do now? O All knowing! You are the Great Muni; kindly say unto us if there be any means by which we can get rid of this our present crisis. There are thousands of Vedic Mantrams which yield the desired results, if they are worshipped with due rites and ceremonies and if all the rules be observed thereof. O best of Munis! Many Yaj\~n\=as are mentioned in the Vedas that yield all the desired results; you know them all; so kindly perform those Yaj\~n\=as. Do all those ceremonies duly that are ordained in the Vedas for the killing of enemies; O Descendant of \=Angirasa! You ought to perform as early as possible those sacrifices for magical purposes to destroy the D\=anavas so that all our miseries come to an end.''

8-22. Brihaspati said :-- ``O Lord of the Suras! All the mantras mentioned in the Vedas yield the desired results, but subservient to the Great Destiny only; they do not give results of themselves but do so in obedience to to the laws ordained by Nature. You all are the presiding Deities of the Vedic Mantras; but, now, by the strange irony of Time, you are put to difficulties and troubles; what can I do now in this case? See! Indra, Ag\d{n}i, Varu\d{n}a, and other gods are invoked in sacrifices; how, then, can sacrificial ceremonies do good when you are put to so great difficulties. Therefore there is no remedy to those which will take place unavoidably; but those who are wise declare that in such cases means are to be adopted. Some sages say that Fate is strong but those who advocate the cause of taking remedial means say that Fate is powerless; remedies or manly exertions lead to all success. But, O King of the Devas! The embodied souls ought to resort to both Fate and Remedies; it is never advisable to depend solely on Fate. Therefore, it is advisable to think out again and again as far as one's own Intellect goes, the best remedies. O Devas! I have thought over again and again on this subject and say to you my opinion. Hear. In days of yore, the Bhagavat\={\i}, being appeased, killed Mahi\d{s}\=asura; and when you

all praised and chanted hymns to Her, She gave you this boon that She will remove all your sorrows and troubles no sooner you remember Her, and She told that you all must remember Her whenever any difficulty would arise to you out of this Great Destiny. She would, then, free you all of your ocean of great difficulties. Therefore do you all now go to the highly sacred and exquisitely beautiful Him\=alay\=an mountains and worship the most worshipful Chandik\=a Dev\={\i} with your love and devotion. Know all the rules of the Seed mantra of M\=ay\=a and be engaged in taking Her name accompanied with burnt offerings. I have come to know, by Yogic power, that She will be pleased with You. I see that today your difficulties will come to an end; there is not the least doubt in this. I have heard that the Dev\={\i} resides always in the Him\=achal; if you worship and praise and chant hymns to Her, She will certainly grant you your desired boons. Therefore fully decide on this thing and go to the Him\=alay\=as. O Devas! She will fulfil all your desires and carry out all your intentions.''

23-24. Vy\=asa said :-- O King! Hearing thus his words, the Devas departed to the Him\=alay\=as and they became all merged in the devotional worship of the Supreme Goddess and began to meditate constantly in their hearts the Seed mantra of M\=ay\=a (Hr\={\i}m). They bowed down to the Goddess Mah\=a M\=ay\=a, the Discarder of all the fears of Her Bhaktas and began to chant hymns to Her with perfect devotion.

25-42. O Goddess! Salutation to Thee! O Thou, the Lord of the Universe! the Lord of our hearts! Thou art the Everlasting Bliss and the Giver of bliss to the Devas! Salutation to Thee! Thou art the Destroyer of the D\=anavas and Thou art the Giver of all desires of human beings. Thou canst be approached with devotion. Salutation to Thee! O Thou, the Incarnate of all the Devas! Thy names are endless; Thy forms are endless; none can count them. Thou residest always as the Force Incarnate in all the actions, in the Creation, Preservation and Dissolution of Beings. O Goddess! Thou art the Memory, Constancy, Intelligence, Old Age. Thou art the nourishment, contentment; Thou upholdest all; Thou art the beauty, peace, good knowledge, prosperity and happiness, Thou art the Goal, fame, and intellect and Thou art the Eternal Seed unmanifested. We now bow down to those forms of Thine through which Thou dost serve the purpose of the Devas in this world as we are now in need of peace. Thou art forgiveness and mercy; Thou art the Yoga Nidr\=a (a state between sleep and wakefulness); Thou art the kindness and Thou residest in all the beings in so many forms, great and grand, and so very celebrated; O Goddess! Thou hadst already served the cause of the gods in killing our

great enemy Mahi\d{s}\=asura, puffed up with vanity. Therefore Thy mercy is well known amongst the gods; what more, Thy mercy is known, since very ancient times and it is narrated in the Vedas. What wonder is there that a mother nourishes gladly her own sons and preserves them carefully! For Thou art the Mother of the Devas; Thou art the great source of help to them; therefore dost Thou fulfill all their desires with Thine whole heart. O Dev\={\i}! We do not know the limit of Thy qualities nor of Thy forms; O Goddess! Thou art worshipped by the whole Universe. Thou art fully competent to save all from dangers; we are objects of Thy pity; dost Thou save us from our present troubles! Thou art capable to kill enemies without shooting any arrrows, without striking any blows, without hurling any trident, axes, \'Saktis, clubs, or any other weapons; merely by Thy mere will Thou canst kill; still for sports and for the good of all beings Thou incarnatest and fightest for the sake of L\={\i}l\=a. The ignorant persons know such things as birth, death, etc., that this world is not eternal; that no actions can be without any cause; we, therefore, ascertain by reasoning and inference that Thou art the Supreme Cause of this whole Universe. Brahm\=a is the Creator, Vi\d{s}\d{n}u is the Preserver, and Mahe\'sa is the Destroyer; so it is related in the Pur\=a\d{n}as. Thou again hast given birth to these three Gods in the respective cycles; therefore Thou art the Mother of all; there is no doubt in this. O Dev\={\i}! In days of yore, these three Devas worshipped Thee; Thou Wert pleased and gavest them all the best powers. Being thus endowed with Thy powers, they have been able to create, preserve and destroy this Universe beautifully. Art they not foolish, though they be Yatis (persons of self-controlled nature), who do not worship the Universal Mother, the Consciousness Incarnate, the Giver of liberation, on Whose feet are worshipped by the Devas, and worshipping Whom, one gets the fruits of all one's desires? Certainly those Vai\d{s}\d{n}avas, Sauras (worshippers of the Sun) and Pa\'supatas (worshippers of \'Siva) are foolish braggarts who do not meditate Thee as the embodiment of Kamal\=a (prosperity), modesty, beauty, continuancy, Fame, nourishment. O Mother! The Asuras, Hari, Hara and other great Devas worship Thee in this world; therefore those mortals are certainly deceived by their Creator that do not worship Thee on the surface of this earth. O Dev\={\i}! Hari himself serves the lotus feet of Lak\d{s}m\={\i} by colouring them (toes and other fingers of the feet) red with lac juice; Hara is very anxious to serve the lotus feet and take the dust thereof of Parvat\={\i}; Lak\d{s}m\={\i} and Parvat\={\i} are but Thy part manifestations; therefore to serve them is, in other words to serve Thee. What to speak of other persons, even those who can discriminate between real and unreal and those who have left their worldly homes and have become dispassionate towards worldly objects, even those Munis worship forgiveness and mercy, that are but Thy parts; therefore who is there

in this world that does not serve Thy lotus-feet! O Dev\={\i}! Those human beings plunge into the dreadful wells of this Sams\=ara, the round of birth and death, and are deprived of all pleasures, who do not serve Thy lotus feet. What more can be said than the fact that those fallen beings suffer terribly from poverty, humility, leprosy, headache, and the chronic enlargement of spleen. O Mother! Those persons are void of any wealth and wife; they are the carriers of loads of wood and collect grass and leaves and show their skill in such acts; they are of little understanding and never they served in their previous births Thy lotus-feet. This we have come to know very well within our heart of hearts.

43-47. Vy\=asa said :-- O King! When all the Devas thus eulogised, instantly the Dev\={\i} Ambik\=a, full of youth and beauty appeared there out of mercy. That extraordinary beautiful Bhagavat\={\i}, endowed with all auspicious signs, and adorned with the Divine clothings, ornaments, and garlands and sandal paste, etc., appeared before the Devas. Before Whom, even the world enchanter Cupid bows down; with such beautiful, Divine appearance, the Dev\={\i} emerged from the mountain cave in order to take Her ablutions in the Ganges. That Dev\={\i}, sweet voiced like a cuckoo, gladly smiling began to say to the Devas, singing hymns to Her, in a voice deep like that of a rumbling cloud.

48. The Dev\={\i} said :-- O Best of Suras! Whom are you praising constantly in this place? What do you want! Why are you so anxious and seem to be so much care-worn? Do please tell all this to Me in detail.

49. Vy\=asa said :-- O King! The Devas were first enchanted by Her beauty and softness; then, being encouraged by Her sweet words, began to speak with great joy.

50-57. O Dev\={\i}! We pray toThee, O Lord of his Universe! We bow down to Thee. O Thou, the Ocean of mercy! Protect us from all the troubles; we are very much care-worn and tormented by the Daityas. O Great Goddess! In ancient times Thou didst kill Mahi\d{s}\=asura, the source of troubles to all and then told us to remember Thee whenever any difficulty would arise. Then Thou wouldst undoubtedly remove all the troubles arising from the Daityas no sooner we remember Thee. O Dev\={\i}! We have now remembered Thee for that very reason. At present the two dreadful Asuras, \'Sumbha and Ni\'sumbha have sprang up and are creating great disturbances; and they cannot be killed by any male beings. The powerful Raktav\={\i}ja and Chanda Munda and other Asuras united have dispossessed the Devas of their Heavens. Thou alone art our goal and refuge; without Thee there is none other to save us. Therefore, O Beautiful One! Thou dost do this work for the Devas who are extremely troubled and distressed. O Powerful Dev\={\i}! The Devas are always at the services of Thy lotus feet; still the very powerful D\=anavas are throwing them into dangers; O Mother! Thou art the

Preserver of the distressed; therefore dost Thou preserve the Devas, devoted to Thee. O Mother! The D\=anavas, being very much emboldened by their powers, are creating many havocs on the surface of the Earth; now remembering that, in the beginning of the Yugas, Thou didst create all this Universe, Thou dost now ought to protect all this Universe.

Here ends the Twenty-second Chapter of the Fifth Book on the eulogising of the Dev\={\i} by the Devas in \'Sr\={\i} Mad Dev\={\i} Bh\=agavatam of 18,000 verses by Mahar\d{s}i Veda Vy\=asa.



