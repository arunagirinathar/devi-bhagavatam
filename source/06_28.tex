\chapter{On N\=arada's getting the feminine form}

1-11. N\=arada said :-- O Thou whose only wealth consists in asceticism! I am now describing to you all those good stories; hear attentively. O Muni! This M\=ay\=a and Her Power are incomprehensible even by those who are the foremost amongst the Yogins. This whole Universe, moving and non-moving, from Brahm\=a to the blade of grass, is enchanted by that Unborn and Incomprehensible M\=ay\=a; therefore no one can escape from the hands of that M\=ay\=a. One day I wanted to see Hari, of wonderful deeds, and went out with lute in my hand from Satyaloka, to the lovely \'Sveta Dv\={\i}pa (the residence of Vi\d{s}\d{n}u) singing the beautiful S\=ama hymns in tune with the seven Svaras. I saw there Gad\=adhara, the Deva of the Devas, with four arms holding disc in one of his hands. He resembled a newly-formed rain-cloud of \'Sy\=ama colour. He was illumined with the lustre of the Kaustubha jewel in his breast. He was wearing an yellow apparel. His head was beautified with a lustrous crown. Thus the Bhagav\=an N\=ar\=aya\d{n}a was playing in amorous movements with the daughter of the ocean, fully capable to give one delight and enjoyment. Seeing me, the lovely Dev\={\i} Kamal\=a, dear to V\=asudeva, full of youth and beauty, decorated with ornaments, endowed with all auspicious signs, superior to all the women, went away at once (to another room) from the presence of Jan\=ardana. The breast of Lak\d{s}m\={\i} Dev\={\i} was becoming visible even through the cloth thrown over it; therefore she went hurriedly to the inner compartment. Seeing this I asked Jan\=ardana, the Deva of the Devas, the Lord of the worlds, and holding a garland of forest grown flowers thus :-- ``O Bhagav\=an! O Slayer of Mura! O Padman\=abha! Why has Kamal\=a Dev\={\i}, the Mother of all the Lokas, on seeing me coming here, gone out of Your presence. O Lord of the worlds! I am not a rogue nor a cheat; I have conquered my passions and am become an ascetic; I have conquered even M\=ay\=a. Therefore O Deva! What is the cause of the departure of the Kamal\=a Dev\={\i} from here? Kindly explain this to me.''

12-20. N\=arada said :-- O Dvaip\=ayana! Hearing my words, expressive of my pride, Jan\=ardana smiled and spoke to me in words sweet like the sound of a lute :-- ``O N\=arada! The rule in such cases is this :-- The wife of any man whatsoever ought not to stay before any other male outsider than her husband. O N\=arada! It is very hard to conquer M\=ay\=a; even those, who by Pr\=a\d{n}\=ay\=ama have conquered their Pr\=a\d{n}a V\=ayu, their organs of senses and their food, even those S\=amkhya Yogins and the Devas are not able to conquer M\=ay\=a. The words that you have just now uttered that you

have conquered M\=ay\=a are not fit to come out of your mouth; for by your knowledge of music, it seems that you are enchanted with the sounds of the music. Brahm\=a, I, \'Siva, and the other Munis, none of us has been able as yet to conquer that Unborn M\=ay\=a; how, then, can it be possible that you or any other man can conquer that M\=ay\=a! Any embodied being, be he a Deva, a human being, or a bird, no one is able to conquer that M\=ay\=a Unborn. Whoever is endowed with the three Gu\d{n}as, be he a knower of the Vedas, or a Yogin, or conqueror of his passions, or all knowing, is not able to conquer M\=ay\=a. The Great Time (K\=ala) though formless, is one form of M\=ay\=a and fashions this universe. All the J\={\i}vas are subservient to this K\=ala, be he a good literary person, or of a mediocre nature, or an illiterate brute. This K\=ala sometimes makes even a religious man that knows Dharma confounded and deluded; so you know the nature of M\=ay\=a is very incomprehensible and Her ways mysterious.'' (Note: This K\=ala is of the fourth dimension, time and space.)

21-23. O Dvaip\=ayana! Thus saying, Vi\d{s}\d{n}u stopped. I was greatly astonished and asked that Eternal V\=asudeva, the Deva of the Devas, the Lord of the World, ``O Lord of Ram\=a! What is the form of M\=ay\=a? How is She? What is the measure of Her strength? Where She resides? Whose substratum is She? Kindly tell these to me. O Preserver of the Universe! I am greatly desirous to see M\=ay\=a; Shew Her to me quickly. O Lord of Ram\=a! I am very eager to know M\=ay\=a. Be graciously pleased to describe tome the glory of M\=ay\=a.''

24-36. Vi\d{s}\d{n}u said :-- M\=ay\=a resides everywhere throughout this whole Universe; Her nature consists of the three Gu\d{n}as; She is the substratum of all; She is omniscient, and acknowledged by all; invisible, and of diverse forms. O N\=arada! If you want to see M\=ay\=a, then come quickly and mount with me on Garuda; we both will go elsewhere and I will shew you that M\=ay\=a, invincible by those who have not conquered themselves. O Son of Brahm\=a! Don't be depressed when you see M\=ay\=a. Thus saying, Jan\=ardana Hari remembered Garuda and instantly he came to Hari. Jan\=ardana mounted on him and gladly made me also get up on his back and took me with Him. In a moment Garuda, went, at his command, with the speed of wind to the forest where the Bhagav\=an desired to go. Mounting on Garuda we passed and saw on our way beautiful forests, nice lakes, rivers, towns, villages, huts of cultivators, towns close to the mountains, huts for cow-keepers in cowsheds, the beautiful hermitages of the Munis, lovely Jhils, tanks and lakes beautified with big lotuses, flocks of ewes, packs of wild boars, etc., till, at last, we came to a place close to Kanauj. I saw there a beautiful divine tank; nice lotuses blossomed there, spreading their sweet fragrance all around; the bees

were making lovely humming noise and ravishing away the minds of men; various flowers, lilies, etc., were beautifying the place; Geese, K\=arandavas, and Chakrav\=akas and other acquatic fowls were playing with their cackling noise, the water was very sweet like milk; the tank was defying, as it were the ocean. Seeing such a wonderful tank, the Bhagav\=an told me :-- ``O N\=arada! See, how beautiful is this deep tank with its clear waters, and adorned all over with lotuses! The sweet voiced flamingoes are roaming on the lake making lovely sounds!

37-54. We will bathe in this tank and then go to the city Kanauj. Thus saying, He made me descend quickly from Garuda and He himself also got down. Then the Bhagav\=an smilingly caught hold of my fore-finger and repeatedly praising the glory of the tank took me to its bank. We rested a while on the cool umbrageous beautiful bank when \'Sr\={\i} Bhagav\=an said :-- ``O Muni! Better bathe you first in this tank; next I will bathe in this very holy pool of water. O N\=arada! Look! Look! How clear crystal-like is the water of this pool like the heart of a saint; see how it smells also fragrantly in contact with the lotuses on it.'' When the Bhagav\=an spoke thus to me; I kept my lute and deer skin aside and gladly went to the edge of the tank. Washing then my hands and feet I tied my hair lock and, taking Ku\'sa grass, I performed my \=Achaman and, purifying myself, began to bathe myself in that tank. While I was bathing, Hari was looking at me; by the time I took a dip, I saw that I quitted my male form and got a beautiful female form. Hari took away, then, my deer skin and lute and mounting on Garuda went away in a moment to His own residence. Getting the female form and decorated with excellent ornaments, my memory of my previous male form vanished at once; I forgot all about my famous lute and forgot also Jagann\=atha, the Deva of the Devas. I then came out of the tank in that enchanting woman form, saw the pool of water filled with clear limpid water and adorned with lotuses. Seeing that, I began to think :-- ``What is this?'' and I became very much astonished. While I was thus meditating in my woman form, a king, named T\=aladhvaja, came there, all on a sudden, on a chariot, accompanied by numerous elephants and horses. The King looked like a second Cupid; he was decorated with various ornaments on his various limbs; he was just entering into his youth and he looked very enchanting. The King saw me at once and looking at me decked with divine ornaments and my moon-like face, was greatly astonished and asked me :-- ``O Kaly\=a\d{n}i! Who are you? Are you the daughter of a man or of a N\=aga (serpent) or of a Gandharva or of a Deva? I see you are now in your youth; why are you alone here? O Lovely-eyed!

Has any fortunate person married you? Or are you still unmarried? Speak all these truly to me. O Fair-haired One! What are you looking at in this tank? O One enchanting, as it were, like the Cupid! What is your desire? Say, O Slanting-eyed! My mind is ravished to hear your cuckoo-like voice. O One of thin waist! Choose me as your husband and enjoy various excellent things as you like.''

Here ends the Twenty-eighth Chapter of the Sixth Book on N\=arada's getting the feminine form in the Mah\=apur\=a\d{n}am \'Sr\={\i} Mad Dev\={\i} Bh\=agavatam of 18,000 verses by Mahar\d{s}i Veda Vy\=asa.



