\chapter{On the method of worship of Tulas\={\i} Dev\={\i}}

1-2. N\=arada said :-- When the Dev\={\i} Tulas\={\i} has been made so dear to N\=ar\=aya\d{n}a and thus an object for worship, then describe Her worship and Stotra (the hymn of Tulas\={\i}) now. O Muni! By whom was She first worshipped? By whom were Her glories first sung? And how did She become therefore an object of worship? Speak out all these to me.

3. S\=uta said :-- Hearing these words of N\=arada, N\=ar\=aya\d{n}a, laughing, began to describe this very holy and sin-destroying account of Tulas\={\i}.

4-15. N\=ar\=aya\d{n}a said :-- Bhagav\=an Hari duly worshipped Tulas\={\i}, and began to enjoy her with Lak\d{s}m\={\i}; He raised Tulas\={\i} to the rank of Lak\d{s}m\={\i} and thus made her fortunate and glorious Lak\d{s}m\={\i} and Gang\=a allowed and bore this new union of N\=ar\=aya\d{n}a and Tulas\={\i}. But Sarasvat\={\i} could not endure this high position of Tulas\={\i} owing to Her anger. She became self-conceited and beat Tulas\={\i} on some quarrel before Hari. Tulas\={\i} became abashed and insulted and vanished off. Being the \=I\'svar\={\i} of all the Siddhis, the Dev\={\i}, the Self-manifest and the Giver of the Siddhiyoga to the J\~n\=anins, Tulas\={\i}, Oh! what a wonder, became angry and turned out as invisible to \'Sr\={\i} Hari even.

Not seeing Tulas\={\i}, Hari appeased Sarasvat\={\i} and getting Her permission went to the Tulas\={\i} forest. Going there and taking a bath in due accord, and with due rites, worshipped with His whole heart the chaste Tulas\={\i} and then began to meditate on Her with devotion. O N\=arada! He gets certainly all siddhis who worships Tulas\={\i} duly with the ten lettered mantra :-- ``\'Sr\={\i}m Hr\={\i}m Kl\={\i}m Aim Vrind\=avanyai Sv\=ah\=a,'' the King of mantras, yielding fruits and all gratifications like the Kalpa Tree. O N\=arada! At the time of worship, the lamp of ghee, was

lighted and dh\=up, sind\=ura, sandal, offerings of food, flowers, etc., were offered to Her. Thus hymned by Hari, Tulas\={\i} came out of the tree, pleased. And She gladly took refuge at His lotus feet. Vi\d{s}\d{n}u, then, granted her boon that, ``You will be worshipped by all; I will keep you in My breast and in My head and the Devas also will hold you on their heads.'' And He then took her to His own abode.

16. N\=arada said :-- ``O Highly Fortunate One! What is Tulas\={\i}'s dhy\=an, stotra and method of worship? Kindly describe all these.''

17. N\=ar\=aya\d{n}a said :-- When Tulas\={\i} vanished, Hari became very much agitated at her bereavement and went to Vrind\=avana and began to praise her.

18-44. The Bhagav\=an said :-- The Tulas\={\i} trees collect in multitudinous groups; hence the Pundits call it Vrind\=a. I praise that dear Tulas\={\i}. Of old, She appeared in the Vrind\=avana forest and therefore known as Vrind\=avan\={\i}. I worship that fortunate and glorious One. She is worshipped always in innumerable universes and is, therefore, known as Vi\'svap\=ujit\=a (worshipped by all). I worship that Vi\'svap\=ujit\=a. By whose contact, those countless universes are always rendered pure and holy; and therefore She is called Vi\'svap\=avani (purifying the whole universe). I am suffering from her bereavement, I remember the Dev\={\i}. Without Tulas\={\i}, the Devas do not get pleased, though other flowers be heaped on them; therefore She is considered as the essence of all the flowers. Now I am in sorrow and trouble and I am very eager to see her, who is of the nature of purity incarnate. The whole universe gets delighted when the Bhaktas receive her; hence She is called Nandin\={\i}; so may She be pleased with me. There is nothing in the universe that can be compared to Her; hence She is called Tulas\={\i}; I take refuge of that clear Tulas\={\i}. That chaste dear one is the life of Kri\d{s}\d{n}a, hence She is known as Kri\d{s}\d{n}aj\={\i}van\={\i}. Now may She save my life. O N\=arada! Thus praising, Ram\=apat\={\i} remained there. The chaste Tulas\={\i} then came to His sight and bowed down to His lotus feet; when She becoming sensitive out of the insult, began to weep. Bhagav\=an Vi\d{s}\d{n}u, seeing that sensitive dear one, immediately took her to His breast. Taking, then, Sarasvat\={\i}'s permission, He took her to His own home and brought about, first of all, the agreement between her and Sarasvat\={\i}. Then He granted her the boon, ``You will be worshipped by all, respected by all, and honoured by all; and all will carry you on their heads.'' I will also worship, respect and honour you and keep you on My head. Receiving this boon from Vi\d{s}\d{n}u, the Dev\={\i} Tulas\={\i} became very glad.

Sarasvat\={\i} then attracted her to her side, made her sit close to her. Lak\d{s}m\={\i} and Gang\=a both with smiling faces attracted her and made her enter into the house. O N\=arada! Whosoever worships her with her eight names Vrind\=a, Vrind\=avan\={\i}, Vi\'svap\=ujit\=a, Vi\'svap\=avan\={\i}, Tulas\={\i}, Pu\d{s}pas\=ar\=a, Nandan\={\i} and Kri\d{s}\d{n}a J\={\i}van\={\i} and their meanings and sings this hymn of eight verses duly, acquires the merit of performing A\'svamedha Yaj\~na (horse sacrifice). Specially, on the Full Moon night of the month of K\=artik, the auspicious birth ceremony of Tulas\={\i} is performed. Of old Vi\d{s}\d{n}u worshipped her at that time. Whoever worships with devotion on that Full Moon combination, the universe purifying Tulas\={\i}, becomes freed of all sins and goes up to the Vi\d{s}\d{n}u Loka. Offerings of Tulas\={\i} leaves to Vi\d{s}\d{n}u in the month of K\=artik bring merits equal to those in giving away Ayuta Cows. Hearing this stotra at that period gives sons to the sonless persons, wives to the wifeless persons and friends to friendless persons. On hearing this stotra, the diseased become free of their diseases, the persons in bondage become free, the terrified become fearless, and the sinners are freed of their sins. O N\=arada! Thus it has been mentioned how to chant stotra to her. Now hear her dhy\=an and method of worship. In the Vedas, in the K\=a\d{n}va \'S\=akh\=a branch, the method of worship is given. You know that one is to meditate on the Tulas\={\i} plant, without any invocation (\=av\=ahana) and then worship her with devotion, presenting all sorts of offerings as required to her. Now hear Her Dhy\=anam. Of all the flowers, Tulas\={\i} (the holy basil) is the best, very holy, and captivating the mind. It is a flame burning away all the fuel of sins committed by man. In the Vedas it is stated that this plant is called Tulas\={\i}, because there can be made no comparison with Her amongst all the flowers. She is the holiest of them all. She is placed on the heads of all and desired by all and gives holiness to the universe. She gives J\={\i}vanmukti, mukti and devotion to \'Sr\={\i} Hari. I worship Her. Thus meditating on Her and worshipping Her according to due rites, one is to bow down to Her. O N\=arada! I have described to you the full history of \'Sr\={\i} Tulas\={\i} Dev\={\i}. What more do yo want to hear now, say.

Here ends the Twenty-fifth Chapter of the Ninth Book on the method of worship of Tulas\={\i} Dev\={\i} in the Mah\=apur\=a\d{n}am \'Sr\={\i} Mad Dev\={\i} Bh\=agavatam of 18,000 verses by Mahar\d{s}i Veda Vy\=asa.



