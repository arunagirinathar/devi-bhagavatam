\chapter{On the rules of using the Tripundra and \=Urdhapundra marks}

1-10. N\=ar\=aya\d{n}a said :-- Only the twice born are to take this Tripundra on the forehead and the other parts of the body after carefully purifying the ashes by the mantra Agniriti Bhasma, etc. The Br\=ahma\d{n}s, K\d{s}attriyas, and Vai\'syas are known as the twiceborn, (the Dv\={\i}jas). So the Dv\={\i}jas ought to take daily this Tripundra with great care. O Br\=ahma\d{n}a! Those who are purified with the ceremony of the holy thread, are called the Dv\={\i}jas. For these the taking of Tripundra as per \'Sruti is very necessary. Without taking this Vibh\=uti, any good work done is as it were not done. There is no doubt in this. Even the japam of G\=ayatr\={\i} is not well performed if this Bhasma be not used. O Best of Munis! The G\=ayatr\={\i} is the most important and the chief thing of the Br\=ahma\d{n}hood. But that is not advised if the Tripundra be not taken. O Munis! As long as the ashes

born of Agni are not applied on the forehead, one is not entitled to be initiated in the G\=ayatr\={\i} Mantra. O Br\=ahma\d{n}! Unless ashes be applied on the forehead, no one will recognise you as a Br\=ahma\d{n}a. For this reason I take this holding of the merit-giving Tripundra as the cause of the Br\=ahma\d{n}hood. I speak this verily unto you, that he is recognised as a Br\=ahma\d{n}a and literary on whose forehead there is seen the white ashes purified by the mantra. He is entitled to the state of a Br\=ahma\d{n}a who is naturally very eager to collect the ashes as he collects the invaluable gems and jewels.

11-20. Those who are not naturally eager to collect the Bhasma as they are naturally eager to collect gems and jewel, are to be known as Ch\=and\=alas in some of their previous births. Those who are not naturally joyous in holding Tripundra, were verily Ch\=and\=alas in their previous births; this I tell you truly very truly.

Those who eat roots and fruits without holding ashes go to the terrible hells. He who worships \'Siva without having Bibh\=uti on his forehead, that wretch is a \'Siva hater and goes to hell after his death. He who does not hold Bibh\=uti is not entitled to any religious act.

Without taking Bibh\=uti, if you make a gift of Tul\=a Puru\d{s}a made of gold, you won't get any fruits. Rather you will have to go to hell!

As the Br\=ahma\d{n}as are not to perform their Sandhy\=as without their holy threads, so without this Bibh\=uti, one ought not also to perform one's Sandhy\=a.

If at times a man by chance has no holy thread, he can do his Sandhy\=a by muttering the G\=ayatr\={\i} or by fasting. But there is no such rule in holding Bhasma.

If one performs Sandhy\=a, without having any Bibh\=uti, he is liable to incur a sin; as without holding this Bhasma, no right can come to him to perform his Sandhy\=a.

As a man of a lowest caste acts contrary and incurs a sin if he hears the Veda mantra, so a twice-born incurs a sin if he performs Sandhy\=a without having his Tripundra. The twiceborn must therefore collect his thoughts with his heart intent on this Tripundra whether it be according to \'Srauta or Sm\=arta method; or in absence thereof the Laukika Bhasma. Of whatsoever sort is the Bhasma, it is always pure. In the Sandhy\=a and other actions of worship, the twiceborn ought to be very careful and punctilious in using this Bhasma.

21-31. No sin can enter into the body of one besmeared with ashes. For this reason, the Br\=ahma\d{n}as ought always to use ashes with great care. One is to hold the Tripundra, six Angulas high or greater by the fore, middle and ring fingers of the right hand. If anybody uses Tripundra, shining and brilliant, and extending from eye to eye, he becomes, no doubt, a Rudra. The ring-finger is the letter ``A,'' the middle finger is ``U'' and the forefinger is ``M''; so the Tripundra marks drawn by the above three fingers is of the nature of the three gu\d{n}as. The Tripundra should be drawn by the middle, fore, and ring fingers in a reverse way (from the left of the forehead to its right). I will now tell you an anecdote, very ancient. Listen. Once Durv\=as\=a, the head of the ascetics, with his body besmeared with ashes and with Rudr\=aksam, all over, on his body went to the region of the Pitris, uttering loudly, ``O \'Sankara, of the Form of All! O \'Siva! O Mother Jagadambe, the Source of all auspiciousness!'' The Pitris Kavya-V\=al\=as, etc., (Kavya V\=alanalah Somah Yamah schaiv\=aryam\=a Tath\=a, Agnisv\=attv\=a, Varhisadah, Somap\=ah Pitri Devat\=ah) got up, received him heartily and gave him seats and shewed him great honours and respect and held many pure conversations with the Muni. During their talk, the sinners of the Kumbh\={\i}p\=aka hell were crying, ``Oh! Alas! We are killed, we are being killed. Oh! We are being burnt!'' some others cried, ``Oh! Oh! We are cut down.'' Thus various cries and lamentations reached their ears.

32-40. Hearing their piteous cries, Durv\=as\=a, the prince of the \d{R}i\d{s}is, asked with a grievous heart the Pitris, ``Who are those crying?'' The Pitris replied :-- There is a city close to our place called ``Samyaman\={\i} Pur\={\i}'' of the King Yama where the sinners are punished. Yama gives punishment to the sinners there. O Sinless One! In that city the King Yama lives with his terrible black-coloured messengers, the personifications of K\=ala (the Destruction). For the punishment of the sinners, eighty-six hells exist there. The place is being guarded always by the horrible messengers of Yama. Out of those hells, the hell named Kumbh\={\i}p\=aka is very big and that is the chief of the hells. The ailings and torments of the sinners in the Kumbh\={\i}p\=aka hell cannot be described in hundred years. O Muni! The \'Siva-haters, the Vi\d{s}\d{n}u-haters, the Dev\={\i}-haters are made to fall to this Kunda. Those who find fault with the Vedas, and blame the Sun, Ga\d{n}e\'sa and tyrannise the Br\=ahma\d{n}as fall down to this hell. Those who blame their mothers, fathers, Gurus, elder brothers, the Smritis and Pur\=a\d{n}as and those as well who take the Tapta Mudr\=as (hot marks on their bodies) and

Tapta \'S\=ulas (i.e., those who being \'Saivas act as they like) those who blame the religion (Dharma) go down to that hell.

41-50. We hear constantly their loud piteous cries, very painful to hear; hearing which naturally gives rise to feelings of indifference (Vair\=agyam). Hearing the above words of the Pitris, Durv\=as\=a, the prince of the Munis, went to the hell to see the sinners. O Muni! Going there, the Muni bent his head downwards and saw the sinners when, instantly the sinners began to enjoy pleasures more than those who enjoy in the Heavens. The sinners became exceedingly glad. Some began to sing, some began to dance, some began to laugh; some sinners began to play one with one another in great ecstasy. The musical instruments Mrida\d{n}ga, Muraja, lute, Dhakk\=a, Dundubhis, etc., resounded with sweet sonorous tones (in accordance with five resonants). The sweet fragrant smell of the flowers of V\=asanti creepers spread all round. Durv\=as\=a Muni became surprised to see all this. The messengers of Yama were startled and immediately went to their King Yama and said :-- ``O Lord! Our King! A wondrous event occurred lately. The sinners in the Kumbh\={\i}p\=aka hell are now enjoying pleasures more than those in the Heavens. O Bibhu! How can this take place! We cannot make out the cause of this. O Deva! We all have become terrified and have come to you.'' Hearing the words of the messengers, Dharmar\=aja, mounting on his great buffalo, came there instantly and seeing the state of the sinners sent news immediately to the Heavens.

51-60. Hearing the news Indra came there with all the Devas, Brahm\=a came there from His Brahmaloka; and N\=ar\=aya\d{n}a came there from Vaikuntha. Hearing this, the regents of the quarters, the Dikp\=alas came there with all their attendants from their respective abodes. They all came there to the Kumbh\={\i}p\=aka hell and saw that all the beings there are enjoying greater pleasures than those in the Heavens. They all were astonished to see this; and they could not make out why this had happened. ``What a wonder is this! This Kunda has been built for the punishment of the sinners. When such a pleasure is now being felt here, the people won't fear anything henceforth to commit sins. Why is this order of the Vedas created by God reversed? Why has God undone His own doing? What a wonder is this! Now a great miracle is before our sight.'' Thus speaking, they remained at a fix. They could not make out the cause of this. In the meanwhile Bhagav\=an N\=ar\=aya\d{n}a after consulting with the other Devas went with some Devas to the abode of

\'Sankara in Kail\=a\'sa. They saw there that \'Sr\={\i} Bhagav\=an \'Sankara (with crescent of the Moon on His forehead) was playing there attended always by the Pramathas and adorned with various ornaments like a youth, sixteen years old. His parts of the body were very beautiful as if the mine of loveliness. He was conversing on various delightful subjects with His consort P\=arvat\={\i} and pleasing Her mind. The four Vedas were there personified. Seeing Him, N\=ar\=aya\d{n}a bowed down and informed him clearly of all the wonderful events. He said :--

61-75. ``O Deva! What is the cause of all this? We cannot make out anything! O Lord! Thou art omniscient. Thou knowest everything. So kindly mention how is this brought about!'' Hearing Vi\d{s}\d{n}u's words. Bhagav\=an \'Sankara spoke graciously in sweet words, grave as the rumbling of a rain-cloud :-- ``O Vi\d{s}\d{n}u! Hear the cause of this. What wonder is there? This is all due to the greatness of Bhasma (ashes)! What cannot be brought about by Bhasma! The great \'Saiva Durv\=as\=a went to see the Kumbh\={\i}p\=aka hell, besmearing his whole body with Bhasma and looked downwards while he was looking at the sinners. At that time, accidentally a particle of Bhasma from his forehead was blown by air to the bodies of the sinners in the hell. Thereby they were freed of their sins and they got so much pleasure! Such is the greatness of Bhasma! Henceforth the Kumbh\={\i}p\=aka will no more be a hell. It will be a T\={\i}rtha (holy place of pilgrimage) of the residents of the Pitrilokas. Whoever will bathe there will be very happy. There is no doubt in this. Its name will be henceforth the Pitri T\={\i}rtha.

O Sattama! My Lingam and the form of Bhagavat\={\i} ought to be placed there. The inhabitants of the Pitri Loka would worship them. This will be the best of all the T\={\i}rthas extant in the three Lokas. And if the Pitri\'svar\={\i} there be worshipped, know that the worship of the Trilok\={\i} is done. N\=ar\=aya\d{n}a said :-- Hearing thus the words of \'Sankara, the Deva of the Devas, He thanked Him and, taking His permission came to the Devas and informed them of everything what \'Sankara had said. Hearing this, the Devas nodded their heads and said, ``Sadhu (well, very well)'' and began to glorify the greatness of Bhasma. O Tormenter of the enemies! Hari, Brahm\=a and the other Devas began to eulogise the glories of ashes. The Pitris became very glad to get a new T\={\i}rtha. The Devas planted a \'Siva Lingam and the form of the Dev\={\i} on the banks of the new T\={\i}rtha, and began to worship them regularly day by day. The sinners that were there suffering, all ascended on the celestial chariot and got up to Kail\=a\'sa. Even today they are

all dwelling in Kail\=a\'sa and are known by the name of the Bhadras. The hell Kumbh\={\i}p\=aka came to be built afterwards in another place.

76-84. Since that day the Devas did not allow any other devotee of \'Siva to go to the newly created hell Kumbh\={\i}p\=aka. Thus I have described to you the excellent greatness of the Bhasma. O Muni! What more can there be than the glories of the Bhasma! O Best of Munis! Now I am telling you of the usage of \=Urdhapundra (the vertical marks) according to the proper province of the devotees. Listen. I will now speak what I have ascertained from the study of the Vai\d{s}\d{n}ava \'S\=astras, the measure of \=Urdhapundra, according to the Anguli measurements, the colour, mantra, Devat\=a and the fruits thereof. Hear. The earth required is to be seen red from the crests of hills, the banks of the rivers, the place of \'Siva (\'Siva K\d{s}ettram), the ocean beaches, the ant-hill, or from the roots of the Tulas\={\i} plants. The earth is not to be had from any other places. The black coloured earth brings in peace, the red-colour earth brings in powers to bring another to one's control; the yellow-coloured earth increases prosperity and the white-coloured earth gives Dharma (religion). If the \=Urdhapundra be drawn by the thumb, nourishment is obtained; if it be drawn by the middle finger, longevity is increased; if it be drawn by nameless or ring finger, food is obtained and if it be drawn by the fore finger, liberation is attained. So the \=Urdhapundras ought to be drawn by these fingers, only be careful to see that the nails do not touch at the time of making the mark. The shape of the \=Urdhapundra (the vertical mark or sign on the fore-head) is like a flame or like the opening bud of a lotus, or like the leaf of a bamboo, or like a fish, or like a tortoise or like a conch-shell.

85-95. The \=Urdhapundra, ten Angulis high is the super best; nine Angulis high, is best; eight Angulis high, is good; the middling \=Urdhapundra is of three kinds as it is of seven Angulas, six Angulas, or five Angulas. The lowest \=Urdhapundra is again of three kinds as it is four Angulas, three Angulas or two Angulas high. On the \=Urdhapundra of the forehead, you must meditate Ke\'sava, on the belly you must think of N\=ar\=aya\d{n}a; on the heart, you must meditate on M\=adhava; and on the neck, you must meditate on Govinda. So on the right side of the belly, you must meditate on Madh\=us\=udana; on the roots of the ears, on Trivikrama; on the left belly, on V\=amana; on the arms, on \'Sr\={\i}dhara; on the ears, Hri\d{s}\={\i}ke\'sa; on the back, Padman\=abha; on the shoulders D\=amodara; and on the head Brahmarandhra you must meditate on V\=asudeva. Thus the twelve

names are to be meditated. In the morning or in the evening time when you are going to make the P\=uj\=a or Homa, you are to take duly, single-in-intent, the above names and make the marks of \=Urdhapundras. Any man, with \=Urdhapundra on his head, is always pure, whether he be impure, or of unrighteous conduct or whether he commits a sin mentally. Wherever he dies, he comes to My Abode even if he be of a Ch\=and\=ala caste. My devotees ( V\={\i}ra Vai\d{s}\d{n}avas or Mah\=av\={\i}ra Vai\d{s}\d{n}avas) who know My Nature must keep an empty space between the two lines of \=Urdhapundra of the form of the Vi\d{s}\d{n}upada (the feet of Vi\d{s}\d{n}u) and those who are my best devotees are to use nice \=Urdhapundras, made of turmeric powder, of the size of a spear (\'S\=ula), of the form of the feet of Vi\d{s}\d{n}u (Vi\d{s}\d{n}u padah).

96. The ordinary Vai\d{s}\d{n}avas are to use with Bhakti, the \=Urdhapundras without any empty space, but the form of it is to be like a flame, the blossom of a lily or like a bamboo leaf.

97-110. Those who are Vai\d{s}\d{n}avas in name only can use \=Urdhapundra of both the kinds, with or without any empty space. They incur no sin if they use one without an empty space. But those who are My good devotees, incur sin if they do not keep an empty space between the two vertical lines (in the \=Urdhapundra three vertical lines are used). The Vai\d{s}\d{n}avas who use excellent vertical rod like \=Urdhapundras keeping an empty space in the middle and uttering the mantra ``Kesv\=aya Namah'' build My Temple there. In the beautiful middle space of \=Urdhapundra, the Undecaying Vi\d{s}\d{n}u is playing with Lak\d{s}m\={\i}. That wretch, the twice-born who uses \=Urdhapundra without any empty apace kills Vi\d{s}\d{n}u and Lak\d{s}m\={\i}, seated there. The stupid who uses \=Urdhapundra without a vacant space goes successively to twenty-one hells. The \=Urdhapundra should be of the size of a clear straight rod, lotus, flame, a fish with sharp straight edges and with vacant spaces between them. O Great Muni! The Br\=ahma\d{n}a should always use the Tripundra like the lock of hair on the crown of his head and like his Sacrificial thread; otherwise all his actions will be fruitless. Therefore in all ceremonies and actions the Br\=ahma\d{n}as ought to use \=Urdhapundras of the form of a trident, a circle or of a square form. The Br\=ahma\d{n}a who knows the Vedas is never to use the semi-moonlike mark (Tilak) on his bead. The man who is of the Br\=ahmi\d{n} caste and follows the path of the Vedas should not even by mistake use any other mark than those above-mentioned. Other sorts of pundras (marks) that are mentioned in other Vai\d{s}\d{n}ava \'S\=astras for the attainment of fame, beauty, etc., the Veda-knowing Br\=ahma\d{n}as should not use them. The Vaidik Br\=ahma\d{n}as should not use even in error any other Tilaks than the curved Tripundras.

If, out of delusion, the man, following the path of the Vedas, uses other sorts of Tripundras, he would certainly go down to hell.

111-118. The Veda-knowing Br\=ahma\d{n}as would certainly go down to hell if they use other sorts of Tripundras on their bodies. Only the Tilakas, prescribed in the Vedas ought to be used by those who are devoted to the Vedas. Those who do not observe the duties of the Vedas would use Tilaks approved of by other \'S\=astras. Those should use marks approved of by the Vedas whose Deity is that of the Vedas. Those who follow the Tantra \'S\=astras different from the Vedas, should use marks approved of by the Tantras.

Mah\=a Deva is the Veda's Deity and, ready to deliver from the bondages of the world, He has prescribed the Tilakas prescribed in the Vedas for the benefit of the devotees. The marks prescribed by Vi\d{s}\d{n}u, also a Deity of the Vedas, are also those of the Vedas. His other Avataras also use marks approved of by the Vedas. The Tripundras and the besmearing of the body with ashes are according to the Vedas. In the Tantra \'S\=astra different from the Vedas, there is the usage of Tripundra and other marks. But they are not to be used by the Vaidiks. No never.

Those who follow the path of the Vedas should use the curved Tripundras and Bhasma on their foreheads according to the rules prescribed in the Vedas.

He who has obtained the highest state of N\=ar\=aya\d{n}a, i.e., who has realised My Nature, ought to use always on their foreheads \'S\=ula marks scented with fragrant sandalpaste.

Here ends the Fifteenth Chapter of the Eleventh Book on the rules of using the Tripundra and \=Urdhapundra marks in the Mah\=apur\=a\d{n}am \'Sr\={\i} Mad Dev\={\i} Bh\=agavatam of 18,000 verses by Mahar\d{s}i Veda Vy\=asa.



