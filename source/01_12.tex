\chapter{On the birth of Pururav\=a}

1-53. S\=uta said :-- O Mah\=ar\d{s}is! The son of the above mentioned Budha is the greatly religious Pururav\=a, of a very charitable disposition, and always ready to perform sacrificial acts; he was born of a kshattriya woman named Il\=a; and I will now describe how this Pururav\=a was born of Il\=a and Budha, kindly listen.

In days of yore there was a king named Sudyumna, very truthful and highly capable in keeping his senses under control. Once on a time, wearing beautiful ear-rings, with extraordinary bow named \=Ajagab and holding the arrow-case full of arrows on his back, he went out on hunt to a forest, riding on a horse, born of the country Sindhu, surrounded by a few of his ministers. Going about in the regions of forest, the king got for his shik\=ar, buck, hare, boar, rhinoceros, bison, buffalo, young elephant, Srimar deer, wild fowl and various other animals fit for sacrificial purposes; thus he went on deep into the interior of the forest. This divine forest was adorned with rows of Mand\=ara trees and situated at the bottom of the mount Sumeru. Various trees and flowers were spreading the beauty of the forest all around; at places were Asoka creepers, Vakula, S\=al, T\=al, Tam\=al, Champak, Panasa, mangoe, N\={\i}p, Madh\=uka, pomegranate, cocoanut, Y\=uthik\=a, plantain, kunda creeper, and various other trees and flowers; at some other places the bowers formed of M\=adhavi creepers enhanced the beauty and shed the lustre all around. There were tanks and reservoirs of water in the forest filled with swans, k\=arandavas, and other aquatic birds. The bamboo trees on their banks becoming filled with air were emitting sweet musical sounds; and at many places of that all blissgiving forest, there were bees humming sweetly and delighting the minds of people there. Now the R\=ajar\d{s}\={\i} Pradyumna became highly gladdened in his heart to see this interior of the forest, resonated with the cooing of the cuckoos and beautified by various lovely flowers; and no sooner he entered there than he was turned into a female and his horse, also, turned

into a mare; the king, then, became seriously anxious. He began to think over and over again ``What is this? How has this come to pass?'' and became very ashamed and sorry and pondered over thus :--``What am I to do now? How can I in this woman condition go back to my house and how shall I govern my kingdom? Alas! Who has deceived me thus!'' Hearing these astounding words of S\=uta, the \d{R}i\d{s}is said :-- ``O S\=uta! You have mentioned that the king Sudyumna equal to god has been turned into a woman; this is very strange indeed! Therefore, O Suvrata! What is the reason of his being turned into a mare? Kindly describe fully what that beautiful king did in the forest?''

S\=uta, said :-- Once on a time, Sanaka and other \d{R}i\d{s}is went to this forest to pay a visit to \'Sankara, illuminating the ten quarters by their holy aura. But then Bhagav\=an \'Sankara was in amorous dealings with \'Sankar\={\i}. The beautiful Dev\={\i} Ambik\=a was then naked and sitting on the lap of \'Sankara and therefore became very much ashamed at their sight; She got up hurriedly, and putting on her cloth, remained there shuddering, with great shame and sensitiveness. The \d{R}i\d{s}is, also, seeing them engaged in amorous dealings, went away quickly to the hermitage of Nara N\=ar\=aya\d{n}a.

Then Bhagav\=an \'Sankara, seeing \'Sankari too much sensitive, said :-- ``Why are you so much bashful and shy? I am doing just now what will give you pleasure. O Beautiful one! Whoever male will enter from to-day and hereafter, within the precincts of this forest, will be instantly converted into a woman.'' O Munis! Though the forest gave all delights to all, yet, having this defect, all the persons that know of this curse, carefully avoid the forest. No sooner did the King Sudyumna enter into the forest, out of ignorance, than he, as well as his attendants, were instantly turned into women; there could be no doubt in this. The king became overpowered with great care and anxiety and did not go back, out of shame, to his palace; but he began to wander to and fro on the outer skirts of that forest. He became known afterwards as the woman Il\=a. Now, once on an occasion, Budh, while he was wandering at his will, came up there and seeing the beautiful Il\=a with gestures and pastures and surrounded by women, became passionately attached towards her; and Il\=a, too, seeing the beautiful Budh, the son of Chandra deva, became desirous to have him as her husband. They became so very much tied in love towards each other, that intercourse took place there. Thus Bhagav\=an Budh generated, in the womb of Il\=a, Pururav\=a; and Il\=a gave birth, in due time, the son Pururav\=a in that forest. She then, with an anxious heart, recollected, while in the forest, her (rather his), family priest Va\d{s}istha Deva. Now then Va\d{s}istha Deva, seeing the distressed condition of the king Sudyumna

became affected with pity and pleased Mah\=adeva, \'Sankara, the most auspicious Deva of all, by hymns and praises. When Bhagav\=an \'Sankara wanted to grant him the boon that he desired, Va\d{s}istha Deva wanted that the king would be turned again into man as before. At this Bhagav\=an \'Sankara said, in recognition of His promise, that the king Sudyumna would be alternately one month a man and the second month a woman and so on. Thus, by the favour of Va\d{s}istha Deva, the king Sudyumna got this boon and returned to his kingdom and began to govern it. When he used to be turned into a woman, he used to remain in the interior, and when he used to become a man, he governed his kingdom. At this the subjects became very anxious and did not welcome the king as they used to do before. Some days passed away in this way when the prince Pururav\=a grew up into manhood. Then the king Sudyumna gave over to him the kingdom and made him the king of the new capital named Pratisth\=an; and started out to an hermitage to perform tapasy\=a. He went to a beautiful forest, variegated with all sorts of trees, and got from the Devar\d{s}i N\=arada the excellent mantra of the Bhagavat\={\i} Dev\={\i}, consisting of nine letters. He began to repeat it incessantly, with an heart filled with love. Thus some days passed away when the all-auspicious Dev\={\i} Bhagavat\={\i}, the Saviour of the whole Universe, became pleased with the king and appeared before the king, assuming the divine beautiful form, composed of attributes, intoxicated with the drink, and with eyes rolling with pride, and riding on v\=ahana, the lion. Seeing this divine form of the Mother of the Universe, the king Il\=a (in this form) bowed down before Her with eyes filled with love and gladly praised Her with hymns thus :-- ``O Bhagavat\={\i}! What a fortunate being I am! That I have seen today the extraordinary world renowned benignant form of Thine granting grace and benefit to all the Lokas, I, therefore, bow down to Thy lotus-feet, granting desires and liberation, and served by the whole host of the Devas. O Mother! What mortal is there on this earth, who can fully comprehend Thy glories when all the Devas and Munis get bewildered in trying to know of them.

O Dev\={\i}! I am thoroughly astonished to see Thy glories and Thy compassion towards the distressed and poor and helpless people. How can a human being, who is devoid of attributes comprehend Thy attributes when Brahm\=a, Vi\d{s}\d{n}u, Mahe\'svara, Indra, Chandra (moon), Pavana (wind), S\=urya, Kuvera, and the eight Vasus know not Thy powers. O Mother! Bhagav\=an Vi\d{s}\d{n}u, of unrivalled brilliancy, knows Thee as a part of Thine only, as Kamal\=a of Sattva Gu\d{n}as and giving one all one's desires; Bhagav\=an Brahm\=a knows Thy part only as the form made of Rajo gu\d{n}a and

Bhagav\=an \'Sankara knows Thee as Um\=a only made of Tamo Gu\d{n}a; but, O Mother! none of them knows Thy tur\={\i}ya form, transcending all the Gu\d{n}as.

O Mother! where is my humble self, that is of very dull intellect and powerless, and where is Thy extremely propitious serenity and graciousness! Indeed such a gracious favour on me is certainly beyond expectation. Therefore, O Bhav\=ani! I have come to realise, in particular, that Thy heart is full of unbounded mercy; for Thou dost certainly feel compassion for these Bhaktas that are full of Bhakti towards Thee. O Mother! what more shall I say than this, that Bhagav\=an Madhus\=udan Vi\d{s}\d{n}u, though married to Kamal\=a, born from only a part of Thine, considers Himself unfit of Her and is therefore not happy; then the fact that He, the \=Adi Puru\d{s}a gets his feet shampooed by Kamal\=a merely corroborates the fact that He wants His feet to become pure and all auspicious to the world by the holy touch of Kamal\=a's hands. O Mother! It seems to me that the ancient Puru\d{s}a Bhagav\=an Vi\d{s}\d{n}u wants gladly to be kicked by Thee like A\'soka tree, for his own improvement and pleasure; and therefore it is that Thou dost want, as if Thou hast become angry to kick (beat with one's legs) Thy husband, stricken by Smara (cupid, love) and worshipped by all the Devas, who lies prostrate below Thy feet.

O Dev\={\i}; when Thou always residest on the calm broad chest, as if on a great cot, adorned beautifully of Bhagav\=an Vi\d{s}\d{n}u, as lightning in deep dense blue clouds, then it is without doubt that He, becoming the Lord of the Universe, has surely become Thy v\=ahan (vehicle) (on account of carrying Thee on His breast), O Mother! If Thou forsakest Madhus\=udana, out of wrath, He becomes at once powerless and is not worshipped by any body; for it is seen everywhere that persons, though calm and serene, if devoid of \'Sr\={\i} (wealth and power) are forsaken by their relatives as reduced to a state having no qualities. O Mother! I am not to be ignored by Thee, on account of my being a woman, for was it not the fact, that Brahm\=a and the other Devas who always take shelter of Thy lotus feet, had not all to assume once youthful feminine forms, while in Manidv\={\i}pa, and I know this surely that Thou again didst make them of male forms. Therefore, O Thou of unbounded power! What shall I describe about Thy power? Indeed, there is great doubt in my mind whether Thou art masculine or feminine? O Dev\={\i}! Whoever Thou mayst be, whether with attributes on transcending the attributes, whether male or female, I always bow down to Thee, with heart full of devotion towards Thee. O Mother! I want that I may have one unflinching devotion, towards Thee in my final state.''

S\=uta said :-- Thus praising the Dev\={\i}, the king Sudyumna, in the form of the feminine Il\=a, took refuge of the World Mother; and the Dev\={\i}, becoming greatly pleased, gave to the king, then and there, union with Her own

Self. Thus the king got the highest steady place, so very rare even to the Munis, by the grace of the Prime Force, the Dev\={\i} Brahm\=amay\={\i}.

Thus ends the Twelfth Chapter of the first Skandha on the birth of Pururav\=a, in the Mah\=apur\=a\d{n}am \'Sr\={\i}mad Dev\={\i} Bh\=agavatam of 18,000 verses by Mahar\d{s}i Veda Vy\=as.



