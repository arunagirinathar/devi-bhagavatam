\chapter{On the story of Tri\'sanku}

1. Janamejaya said :-- ``O Intelligent One! Did the prince Tri\'sanku free himself afterwards of the curse inflicted on him by the Muni Va\'sistha.''

2-8. Vy\=asa said :-- O King! Satyavrata, cursed by Va\'sistha, was transformed into a demoniacal state (Pi\'s\=achatva); but he became a great devotee of the Dev\={\i} and passed away his time in that \=A\'srama. One day he repeating slowly the nine-lettered Mantram of the Bhagavat\={\i}, wished to perform the Pura\'schara\d{n}a ceremony (repeating the name of a deity attended with burnt offerings, oblations, etc.) of the said Mantra, came to the Br\=ahmi\d{n}s, bowed down to them with great devotion and purity and said :-- ``O venerable gods of the earth! Kindly hear me;

I with my head bowed down pray to you, that you all be my priests (Ritt-vigs). You are all versed in the Vedas; so kindly do for me duly the Homa ceremony equal to one-tenth part of Japam, for my success. O Br\=ahma\d{n}as! My name is Satyavrata; I am a prince; you ought to do this work for me for my welfare.'' Thus hearing the prince's words the Br\=ahma\d{n}as said :-- ``O Prince! You are cursed by your Guru and you are now turned into a demoniacal state. You have now no right to the Vedas; especially you are now in the Pi\'s\=acha state; it is blamed by all the persons; so now you are not fit to be initiated into the ceremony.''

9-14. Vy\=asa said :-- O King! Hearing them, the prince got very sad and dejected and thought ``Fie on my life! What shall I do now in living even in the forest. My father has forsaken me; I am banished from the kingdom; again, by the Guru's curse, I have got this Pi\'s\=acha state; I therefore can't decide what to do.'' The prince, then, collecting fuel, prepared the funeral pile for himself, remembered the Chandik\=a Dev\={\i} and repeating Her Mantram, resolved to jump into the fire. Lighting the pyre in front, the prince bathed and standing, with clasped palms, began to chant the hymns to Mah\=a M\=ay\=a before entering into the fire. At this moment, the Dev\={\i} Bhagavat\={\i}, knowing that the prince was ready to burn himself, came instantly to the spot on the back of the lion, by the aerial route. She manifested Henself before him and spoke in a voice deep like a rain-cloud.

15-17. ``O Virtuous One What is all this? What have you settled all these? Never throw yourself in fire; be patient. O Fortunate One! Your father is now aged; he will give you his kingdom and will go to the forest for tapasy\=a; therefore, O Hero! Do leave your depression of spirits. O King! Tomorrow the ministers of your father will came to you to take you there. By My Grace, your father will install you on the throne and, in due time, he will conquer his desires and will go undoubtedly to the Brahm\=a loka.''

18-32. Vy\=asa said :-- O Fortunate One! Thus saying, the Dev\={\i} vanished at that spot; the prince, too, desisted from his purpose of entering into the fire. In the meanwhile, the highsouled N\=arada went to Ayodhy\=a and at once informed everything to the King. The King became very sad and began to repent very much, hearing the son's resolve to burn himself. The virtuous King, grieved at heart, for his son, said to his ministers :-- ``You all are aware of the turning out of my son. I have forsaken my intelligent son Satyavrata; though he was very spiritual and worthy to get the kingdom; yet, at my command, he

instantaneously went away to the forest. Void of wealth, he, practising forgiveness, passed his time in study, particularly in spiritual knowledge; but Va\'sistha Deva, cursed him and made him like a Pi\'s\=acha. Very much distressed by pain and sorow, he was ready to burn himself but the Mah\=a Dev\={\i} preventing him, he desisted from this purpose. So go hurriedly and, consoling my powerful eldest son, bring him at once to me. I am now calm and quiet and of a retiring disposition; so I am determined to practise tapasy\=a. My son is now capable to govern the subjects; I will now install my son on the throne and retire to the forest.'' So he gladly sent his ministers to his son. The ministers, too, gladly went there and consoled the prince and, with respect, brought him to the Ayodhy\=a city. Seeing Satyavrata with matted hair on his head, with dirty clothes, and thin and worn out with cares, the King began to think within himself ``Oh! What a cruel act have I done, though I know everything about religion, in banishing my intelligent son, quite fit to govern my kingdom.'' Thus thinking, he embraced his son by his arms and consoling him, made him sit by his throne. The King, versed in politics, then began to speak gladly with suffocated feelings of love to his son sitting by the side of him.

33-53. O Son! Your highest duty is to keep your mind always on religion and to respect the Br\=ahmi\d{n}s. Never speak falsely anywhere nor follow any bad course in any way. Rather the words of the spiritual good persons ought to be fully observed; the ascetics ought to be worshipped. Senses must be controlled and the wicked cruel robbers are certainly to be slain. O Son! For one's success, one should consult with one's ministers and keep that as secret by all means. Any enemy howsoever insignificant he may be, a clever King should never overlook him. The ministers, if they be attached to other masters and if they come round afterwards, don't trust them. Spies should be kept to watch friends and foes alike. Show your living regards to the religion always, and make charitable gifts. One ought not to argue in vain and always avoid the company of the wicked. O Son! You should worship the Mahar\d{s}is and perform various sacrifices. Never trust women, those who are inordinately addicted to women, and the gamblers. Never is it advisable to be addicted too much to hunting. Always shew your back to gambling, drinking, music and to the prostitutes and try to make your subjects follow the same. Early in the morning at the Brahm\=a Muh\=urta everyday you should get up from your bed and bathe and perform other analogous duties. O Son! Be initiated by the Guru in the Dev\={\i} Mantra, and worship with devotion the Supreme Force, the Bhagavat\={\i}. Human birth is crowned with success by worshipping Her Lotus Feet, O Son!

He who performs once the great P\=uj\=a of the Mah\=a Dev\={\i} and drinks the Chara\d{n}\=amrita water (water with which Her feet are worshipped) has never to enter again in the womb of his mother; know this as certain. That Mah\=a Dev\={\i} is all that is seen and She Herself is again the Seer and Witness, of the nature of Intelligence. Filled with these ideas, rest fearless like the Universal Soul. Do your daily Naimittik (occasional) duties, go to the Br\=ahmi\d{n}'s assembly and calling on them ask the conclusions of the Dharma \'S\=astras. The Br\=ahmi\d{n}s, versed in the Vedas and Vedantas, are objects of venerations and must be worshipped. Give, then, them always according to merits, cows, lands, gold, etc. Don't worship any Br\=ahmi\d{n} who is illiterate. Don't give to illiterates more than their belliful wants. O Child! Never trespass Dharma, out of covetousness, and remember always not to insult ever afterwards any Br\=ahma\d{n}as. The Br\=ahmi\d{n}s are the cause of the K\d{s}attriyas, the more so they are the terrestrial gods; honour them with all your care! In this never flinch from your duties. Fire comes out of water; the K\d{s}attriyas come out of the Br\=ahma\d{n}as; iron comes out of stones. The powers of these flow everywhere. But if there be any clash between one thing and its source, then that clash dies away in the source. Know this as quite certain. The King who wants his own welfare and improvement must by gift and humility shew his respect especially to the Br\=ahmi\d{n}s. Follow the maxims of morality as dictated in the Dharma \'S\=astras. Amass wealth according to rules of justice and fill the treasury.

Here ends the Eleventh Chapter of the Seventh Book about the story of Tri\'sanku in the Mah\=apur\=a\d{n}am \'Sr\={\i} Mad Dev\={\i} Bh\=agavatam of 18,000 verses by Mahar\d{s}i Veda Vy\=asa.



