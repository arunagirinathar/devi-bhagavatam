\chapter{On the account of Bhr\=amar\={\i} Dev\={\i}}

1-21. \'Sr\={\i} N\=ar\=aya\d{n}a spoke :-- O Child N\=arada! Hear now the wonderful anecdotes of the births of the remaining other Manus. The mere remembrance of these birth anecdotes causes Bhakti to grow and well up towards the Dev\={\i}. Vaivasvata Manu had six sons :-- viz., Karu\d{s}a, Pri\d{s}adhra, N\=abh\=aga, Dista, Sary\=ati, and Tri\'sa\d{n}ku. All of them were stout and strong. Once they all united went to the excellent banks of the Jumn\=a and began to practise Pr\=a\d{n}\=ayama without taking any food and became engaged in worshipping the Dev\={\i}. Each of them built separately an earthen image of the Dev\={\i} and worshipped Her with devotion and with various offerings. In the beginning, they took the dry leaves of the trees that dropped of themselves for their food; then they drank water only, then breathed air only; then the smoke from the fire of the Homa; then they depended on the Solar Rays. Thus they practised tapasy\=a with great difficulties. The continual worship of the Dev\={\i} with the greatest devotion made them conscious of their clear intellect, destructive of all sorts of vanities and delusions, and the Manu's sons thought only of the Hallowed Feet of the Dev\={\i}; their intellects were purified and they were greatly wondered to see within their Self the whole Universe. Thus they practised their Tapasy\=a full twelve years when Bhagavat\={\i}, the Ruling Principle of this Universe resplendent wiih the brilliance of the thousand Suns, appeared before them. The princes with their intelligences thus purified saw Her, bowed down and, with their lowly hearts, began to chant hymns to Her with greatest devotion. ``O \=I\'s\=an\={\i}! O Merciful! Thou art the Dev\={\i} presiding over all. Thou art the Best. So Victory to Thee! Thou art known by the V\=agbhava Mantra. Thou gettest pleased when the V\=agbhava Mantra is repeated. O Dev\={\i}! Thou art of the nature of Kl\={\i}m K\=ara (of the form of Kl\={\i}m). Thou gettest pleased with the repetition of Kl\={\i}m Mantra. O Thou, that gladdenest the Lord! Thou bestowest joy and pleasure in the heart of the King of K\=ama. O Mah\=a M\=ay\=a! When Thou art pleased, Thou givest that Unequalled Kingdom. O Thou that increasest the enjoyments! Thou art Vi\d{s}\d{n}u, S\=urya, Hara, Indra and the other Devas.'' When the highsouled princes praised Her thus, Bhagavat\={\i} became pleased and spoke to them the following sweet words :-- ``O Highsouled Princes! You all have worshipped Me and practised, indeed, very hard tapasy\=as and thus you have become sinless and your intellects and hearts have become thoroughly purged and thus purified. Now ask boons that you

desire. I will grant them ere long to you.'' The Princes said :-- ``O Dev\={\i}! We want unrivalled Kingdoms, many sons of long longevity, continual enjoyment of pleasures, fame, energy, freedom in all actions, and as well the good and keen intelligence. These will be beneficial to us.'' The Dev\={\i} said :-- Whatever you have desired, I grant them to you all. Besides I give you another boon. Listen attentively. By My Grace you all will be the Lords of the Manvantaras and acquire strength that will experience no defeat, and you will get prosperity, fame, energy, powers, and a continual line of descent and abundant full enjoyments.

22-32. N\=ar\=aya\d{n}a said :-- After the World Mother Bhr\=amar\={\i} Dev\={\i} granted them these boons, the princes chanted hymns to Her and then She instantly vanished. The very energetic princes acquired in that birth excellent kingdoms and abundance of wealth. They all had sons and thus established their families, and became the Lords of Manvantara in their next births. By the Grace of the Dev\={\i}, the first of the princes Karu\d{s}a became the Ninth Manu, the exceedingly powerful Dak\d{s}a S\=avar\d{n}i; the second prince Pri\d{s}adhra became the Tenth Manu, named Meru S\=avar\d{n}i; the third prince, the highly enthusiastic N\=abh\=aga became the Eleventh Manu, named S\=urya S\=avar\d{n}i; the fourth prince Dista became the Twelfth Manu, named Chandra S\=avar\d{n}i; the powerful fifth prince \'Sary\=ati became the Thirteenth Manu named Rudra S\=avar\d{n}i and the sixth prince Tri\'sanku became the Fourteenth Manu named Vi\d{s}\d{n}u S\=avar\d{n}i and became the celebrated Lord of the world.

33-41. N\=arada questioned :-- ``O Wise One! Who is that Bhr\=amari Dev\={\i}? What is Her Nature? What for She takes birth? Kindly describe all this beautiful and pain destroying anecdotes to me. I am not satiated with the drinking of the nectar of the Glories of the Dev\={\i}; my desire to hear further more is as strong as ever. As the drink of the nectar takes away death, so the drink of this anecdote of the Dev\={\i} takes away the fear of death.'' N\=ar\=aya\d{n}a said :-- O N\=arada! I will now narrate the wonderful glories of that unthinkable, unmanifested World Mother, leading to Mukti. Hear, as a Mother behaves towards Her child kindly and without any hypocrisy, so the World Mother in all Her lives manifests Her merciful sincere dealings for the welfare of the humanity. In days gone by, in the nether regions, in the city of the Daityas, there lived a powerful Daitya named Aru\d{n}a. He was a furious Deva Hater and a p\=akk\=a hyprocrite. With a view to conquer the Devas, he went to the banks of the Ganges in the Him\=alay\=as, practised a very bard Tapasy\=a, to Brahm\=a, taking Him to be the Protector of the Daityas. First influenced by Tamo Gu\d{n}a, he withheld in his body the five V\=ayus

and partook only the dry leaves and repeated, the G\=ayatr\={\i} Mantra and practised austerities. Thus he practised for full ten thousand years. Then for another ten thousand years the Daitya lived drinking some drops of water only; then for another ten thousand years he remained by inhaling air only; and then for another ten thousand years he did not take any thing and thus practised he his wonderful Tapasy\=a.

42-49. Thus practising his Tapasy\=a a sort of wonderful halo of light emitted from his body and began to burn the whole world. This thing then appeared a great wonder. All the Devas then exclaimed. ``Oh! What is this! Oh! What is this!'' And they trembled. All were very much terrified and took refuge of Brahm\=a. Hearing all the news from the Devas, the four-faced Bhagav\=an rode on His vehicle, the Swan, and with the G\=ayatr\={\i} went very gladly to where the Daitya was practising his austerities and saw that the Daitya was immersed in meditation with his eyes closed; and he looked, as it were, blazing with fire, as if a second Fire himself. His belly had become dried up, body withered and the nerves of the bodies, too, became almost visible; only the life breath was lingering there. Brahm\=a then spoke to him :-- ``O Child! Auspices to you! Now ask the boon that you desire.'' Hearing these gladdening nectar-like words from the mouth of Brahm\=a, Aru\d{n}a, the chief of the Daityas opened his eyes and saw Brahm\=a in his front. Seeing Brahm\=a before him with a rosary of beads and K\=amandalu in his hand and attended by G\=ayatr\={\i} and the four Vedas, muttering the name of the Eternal Brahm\=a, the Daitya rose up and bowed down to Him and sang to Him various Stotras.

50-59. Then the intelligent Daitya asked from Brahm\=a the following boon that ``I shall not die. Grant this.'' Brahm\=a then gently explained to him :-- ``O Best of the D\=anavas! See that Brahm\=a, Vi\d{s}\d{n}u, Mahe\'svara and others are not free from this limitation of death! What to speak then for others! I cannot grant you a boon that is an impossibility. Ask what is possible and just. The intelligent persons never show an eagerness to an impossibility.'' Hearing the above words of Brahm\=a, Aru\d{n}a again said with devotion :-- ``O Deva! If Thou art unwilling to grant me the above boon, then, O Lord! Grant me such a boon, as is practicable, that my death shall not be caused by any war, nor by any arms or weapons, nor by any man or any woman, by any biped or quadruped or any combination of two and grant me such a boon, such a large army as I can conquer the Devas.'' Hearing the words of the Daitya, Brahm\=a said ``Let that be so'' and went back instantly to His own abode. Then, puffed up with that boon, the Daitya Aru\d{n}a called on all the other Daityas that lived in

the nether regions. The Daityas, that were under his shelter, came and saluted him, as their king and, by his command, they sent messengers to the Heavens to fight with the Devas. Hearing from the messenger that the Daityas were willing to fight with the Devas, Indra trembled with fear and went instantly with the Devas to the abode of Brahm\=a. Taking Brahm\=a, too, along with them from there, they went to the Vi\d{s}\d{n}u Loka and took Vi\d{s}\d{n}u with them and all went to the \'Siva Loka.

60-70. There they all held a conference how to kill the Daitya, the enemy of the Gods. While, on the other hand, Aru\d{n}a, the king of the Daityas surrounded by his army, went ere long to the Heavens.

O Muni! The Daitya, then, through the power of his Tapas, assumed various forms and seized the rights and possessions of the Moon, the Sun, Yama, Agni and all the othars. All the Devas, then, dislodged from their stations went to the region of Kaila\'sa and represented to \'Sankara about their own troubles and dangers respectively. Then, what was to be done on this subject, on this, great discussions cropped up. When Brahm\=a said, that the death of the Daitya would not ensue from any fight, with any arms or weapons, from any man or woman, biped, quadruped or from any combination of the above two. Then the Devas became all anxious and could not find out any solution at that instant, when the Incorporeal Voice was clearly heard in the Heavens :-- Let you all worship the Queen of the Universe. She will carry out your work to success. If the king of the Daityas, always engaged in muttering the G\=ayatr\={\i}, forsakes the G\=ayatr\={\i} any how, then his death will occur. Hearing this gladdening Celestial Voice, the Devas held the council with great caution. When it was settled what ought to be done, Indra asked Brihaspati and said :-- ``O Guru Deva! You would better go to the Daitya for the carrying out of the Deva\'s ends and do so that he forsakes the Dev\={\i} G\=ayatr\={\i} Parame\'svar\={\i}. We will all now go and meditate on Her. When She will be pleased, She will help us.

71-77. Thus commanding Brihaspati and thinking that the beautiful Protectress of J\=amb\=u Nada would protect them the Devas all started to worship Her and, going there, began the Dev\={\i} Yaj\~na and with great devotion muttered the M\=ay\=a V\={\i}ja and practised asceticism. On the other hand, Brihaspati went ere long in the garb of a Muni to the Daitya Aru\d{n}a. The king of the Daityas then asked him :-- ``O Best of Munis! Whence and why have you come here. Say, O Muni! Where have you come? I am not one of your party. Rather I am your enemy.'' Hearing the above words, Brihaspati said :-- When you are worshipping incessantly the Dev\={\i} whom we too worship, then say how you are not a one on our side! O Saint! The vicious Daitya, hearing the above words and deluded

by the M\=ay\=a of the Devas, forsook the G\=ayatr\={\i} Mantra out of vanity and therefore he became weak, bereft of the Holy Fire.

78-85. Then Brihaspati, having succeeded in his work there, went to the Heavens and saw Indra and told him everything in detail. The Devas became satisfied and worshipped the Highest Deity. O Muni! Thus, a long interval passed, when one day the World Mother, the Auspicious Dev\={\i} appeared before them. She was resplendent with the brilliance of ten million suns and looked beautiful like ten millions of Kandarpas (Gods of love). Her body was anointed with variegated colours, etc.; She wore a pair of clothings; a wonderful garland suspended from Her neck; Her body was decked with various ornaments and in the fists of Her hands there were wonderful rows of hornets ( large black bees). Her one hand was ready to grant boons and Her other hand was ready to hold out ``no fear.'' On the neck of Bhagavat\={\i}, the Ocean of Mercy, and peaceful, were seen the variegated garlands with large black bees all round. Those male and female bees singing incessantly all round Her the Hr\={\i}mk\=ara Mantra (the First Vibration of Force), kotis of black bees surrounded Her. The All-auspicious Bhagavat\={\i}, praised by all the Vedas, Who is all in all, composed of all, Who is all good, the Mother of all, Omniscient, the Protectress of all, was adorned fully with dress.

86-96. Seeing suddenly the Dev\={\i}, in their front Brahm\=a and the other Devas became surprised and by and by they got relieved and gladly began to chant hymns of praise to Bhagavat\={\i}, Whose Glories have been written in the Vedas.

The Devas said :-- ``O Dev\={\i}! Obeisance to Thee! Thou art the Highest Knowledge and the Creatrix, Preservrix and the Destructrix of the Universe. O Thou, the Lotus-eyed! Thou art the Refuge of all! So we bow down to Thee. O Dev\={\i}! Thou art collectively and individually Vi\'sva, Taijasa, Pr\=aj\~na, Vir\=at and S\=utr\=atm\=a. O Bhagavat\={\i}. Thou art differentiated and undifferentiated; Thou art the K\=utastha Chaitanya (the Unmoveable, Unchangeable Consciousness). So we bow down to Thee. O Durge! Thou art unconcerned with the creation, preservation and destruction; yet Thou punishest the wicked and art easily available by the sincere devotion of Thy Bhaktas. O Dev\={\i}! Thou scorchest and destroyest the ignorance and sin of the embodied souls. Hence Thou art named Bharg\=a. So we bow down to Thee. O Mother! Thou art K\=alik\=a, N\={\i}la Sarasvat\={\i}, Ugra T\=ar\=a, Mahogr\=a; Thou assumest many other forms. So we always bow down to Thee. O Dev\={\i}! Thou art Tripura Sundar\={\i}, Bhairab\={\i}, M\=atang\={\i}, Dh\=um\=avat\={\i}, Chhinnamast\=a, \'S\=akambhar\={\i} and Rakta Dantik\=a. Obeisance to Thee! O Bhagavat\={\i}! It is Thou that didst appear as Lak\d{s}m\={\i} out of

the milk ocean (K\d{s}\={\i}ra Samudra). Thou hadst destroyed Vritr\=asura, Chanda, Munda, Dh\=umralochana, Rakta B\={\i}ja, \'Sumbha, Ni\'sumbha and the Exterminator of the D\=anavas and thus, Thou didst do great favours to the Devas. So, O Gracious Countenanced! Thou art V\={\i}jay\=a and Gang\=a; O S\=arade! We bow down to Thee. O Dev\={\i}! Thou art the earth, fire, Pr\=a\d{n}a and other V\=ayus and other substances. O Merciful! Thou art of the form of this Universe; the Deva form, and the Moon, Sun and other Luminous forms and of the Knowledge Form.

97-109. O Dev\={\i}! Thou art S\=avitr\={\i}; Thou art G\=ayatr\={\i}; Thou art Sarasvat\={\i}; Thou art Svadh\=a, Sv\=ah\=a, and Dak\d{s}i\d{n}\=a. So we bow down to Thee. Thou art, in the Vedas, the \=Agamas, ``Not this, not this.'' Thou art what is left after the negation of all this. This all the Vedas declare of Thy True Nature thus as the Absolute Consciousness in all. Thus Thou art the Highest Deity. So we worship Thee. As Thou art surrounded by large black bees, Thou art named Bhr\=amar\={\i}. We always make obeisance to Thee! Obeisance to Thee! Obeisance to Thy sides! Obeisance to Thy back! Obeisance to Thy front! O Mother! Obeisance to Thy above! Obeisance to Thy below! Obeisance to everywhere round of Thee! O Thou, the Dweller in Ma\d{n}\={\i} Dv\={\i}pa! O Mah\=a Dev\={\i}! Thou art the Guide of the innumerable Brahm\=andas! O World Mother! Let Thou be merciful to us. O Dev\={\i}! Thou art higher than the highest. O World Mother! Victory be to Thee! All Hail! O Goddess of the universe! Thou art the Best in the whole universe; Victory to Thee! O Lady of the world! Thou art the mine of all the gems of qualities. O Parame\'svar\={\i}! O World Mother! Let Thou be pleased unto us.'' N\=ar\=aya\d{n}a said :-- Hearing those sweet, ready and confident words of the Devas, the World Mother said in the sweet tone of a Mad Cuckoo:-- ``O Devas! As far as granting boons to others is concerned, I am ever ready. I am always pleased with you. So, O Devas! Say what you want.'' Hearing the words of the Dev\={\i}, the Devas began to express the cause of their sorrows. They informed Her of the wicked nature of the vicious Daitya, the neglect of the Devas, the Brahm\=a\d{n}as and the Vedas and the ruins thereof, and the dispossession of the Devas of their abodes and the receiving by the Daitya of the boon from Brahm\=a; in fact, everything what they had to say, duly and vigorously. Then the Bhagavat\={\i} Bhr\=amar\={\i} Dev\={\i} sent out all sorts of black bees, hornets, etc., from Her sides, front and forepart.

110-120. Innumerable lines of black bees then were generated and they joined themselves with those that got out of the Dev\={\i}'s hands and thus they covered the whole earth. Thus countless bees began to emit from all sides like locusts. The sky was overcast with the bees; and the earth was covered with darkness. The sky, mountain peaks, trees, forests all became filled with bees and the spectacle presented a grand dismal sight. Then the black bees began to tear asunder the breasts of the Daityas as the bees bite those who destroy their beehives. Thus the Daityas could not use their weapons nor could they fight nor exchange any words. Nothing they could do; they had no help but to die. The Daityas remained in the same state where they were and in that state they wondered and died. No one could talk with another. Thus the principal Daityas died within an instant. Thus completing their destruction, the bees came back to the Dev\={\i}. All the people then spoke to one another ``Oh! What a wonder! Oh! What a wonder!'' Or like this :-- ``Whose M\=ay\=a is this! What a wonder that She will do like this!'' Thus Brahm\=a, Vi\d{s}\d{n}u and Mahe\'sa became merged in the ocean of joy and worshipped the Dev\={\i} Bhagavat\={\i} with various offerings and shoutings of chants ``Victory to the Dev\={\i}'' and showered flowers all around. The Munis began to recite the Vedas. The Gandharbas began to sing.

121-127. The various musical instruments, Mridangas, Murajas, the Indian lutes, Dhakk\=as, Damarus, \'Sankhas, bells, etc., all sounded and the three worlds were filled with their echoes. All with folded palms chanted various hymns of praise to the Dev\={\i} and said ``O Mother! \=Is\=an\={\i}! Victory to Thee!'' The Mah\=a Dev\={\i} became glad and gave to each separate boons and when they asked ``for unshakeable devotion to Thy lotus feet,'' She granted them that also and disappeared before them. Thus I have described to you the glorious character of the Bhr\=am\=ar\={\i} Dev\={\i}. If anybody hears this very wonderful anecdote, he crosses at once this ocean of the world. Along with the glories and greatness of the Dev\={\i}, if one hears the accounts of Manus, then all auspiciousness comes to him. He who hears or recites daily this Greatness of the Dev\={\i}, becomes freed from all his sins and he gets himself absorbed in the thoughts of the Dev\={\i} (S\=ajuya). Note :-- The Mantra is here not merely the Seed, the Spiritual Password, but it connotes, besides the idea of the password, the \=Adi First vibration and it exhibits the First Spiritual Form, endowed with the highest feelings of Faith, Wisdom, Bliss and Joy, displayed with the grandest colours, startling thrills, rapt enchanting

signs, gestures, and postures, the shooting forth of all powers, the sources of Siddhis, that cannot be ordinarily conceived in the worldly concerns. Their faint echoes govern this mighty world. The Mantras are seated in the six chakras or plexuses or the six Laya centres in the spinal cord. Within these chakras, the transformations of the Tattvas take place. Some vanish. Some appear and so on. Remark :-- In this chapter we find clearly the mention of the several names of the ten Da\'sa Mah\=a Vidy\=as.

Here ends the Thirteenth Chapter of the Tenth Book of the account of Bhr\=amar\={\i} Dev\={\i} in the Mah\=apur\=a\d{n}am \'Sr\={\i} Mad Dev\={\i} Bh\=agavatam of 18,000 verses by Mah\=ar\d{s}i Veda Vy\=asa and here ends as well the Tenth Book.

[The Tenth Book completed.]



