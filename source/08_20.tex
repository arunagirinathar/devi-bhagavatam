\chapter{On the narrative of the Tal\=atala}

1-37. N\=ar\=aya\d{n}a said :-- O N\=arada! The cave lower down than Sutala is Tal\=atala! The Lord of Tripura, (the three cities) the great M\=ay\=a D\=anava is the Ruler of this region. Mahe\'svara, the Doer of good to the three Bhuvanas, burnt his three cities; but at last, being pleased with his devotion, He rescued him. Thus M\=ay\=a, by the favour of that God, has regained his own kingdom and the enjoyments thereof. This M\=ay\=a D\=anava is the Teacher (\=Ach\=arya) of the M\=ay\=avi sect and the cult thereof; and he is skilled in various M\=ay\=as or all sorts of the magic powers. All the fierce demons, of cruel temper, worship him for their prosperities in their various enterprises. Next to this Tal\=atala is the most renowned Mah\=atala. The sons of Kadru, the very angry Snakes, live here. They are many headed. O Vipra! I now mention to you the names of the famous amongst them :-- Kuhaka, Tak\d{s}aka, Su\d{s}e\d{n}a, and K\=aliya. These all have very wide hoods and they all are very strong; they all are of cruel temper. Their kinsmen also are so. They are always afraid of Garuda, the King of birds. Surrounded with their sons, wives, friends and acquaintances, they live happily, well skilled in various sports and pleasures. Lower down this Mah\=atala is Ras\=atala. The Daityas, D\=anavas and Pa\d{n}i Asuras live here. Besides these, there live the Niv\=ata Kavachas of the Hira\d{n}yapura city and the Asuras named K\=aleyas, the enemies of the Devas. These all are naturally very energetic and brave; their powers are baffled by the Tejas of the Bhagav\=an and they live like snakes in this region. The other Asuras that were driven and were afraid of the Mantras, uttered by Saram\=a, the messenger of Indra, live here too. O N\=arada! Lower down is P\=at\=ala, where live V\=asuki, the Chief of the snakes, and others named \'Sankha, Kulika, \'Sveta, Dhananjaya, Mah\=a\'sankha, Dh\d{r}tar\=a\d{s}\d{t}ra, \'Sankhach\=uda, Kamvala, A\'svatara, and Devopadattaka, all very angry, of wide hoods, and virulently poisonous. Some of these have five heads, some seven hoods, some ten; some hundred, some others have thousand heads, while some others have on their crests exceedingly luminous jewels. By their rays, they dispel the darkness of the nether regions; but they are awfully prone to anger. At the bottom of this P\=at\=ala, and at a distance of the

thirty Yoyanas; the Portion of Bhagav\=an in the shape of the infinite Darkness is reigning there. O Devar\d{s}i! All the Devas worship this Form. The devotees call Him by the name of Sa\d{n}akar\d{s}a\d{n}a, as He is the manifested emblem of ``Aham'' and the common ground where the Seer and the Seen blend into one. He is the thousand-headed Controller of all, moving and non-moving; He is of infinite forms; He is \'Se\d{s}a; this whole universe is being held as a mustard bean on His head; He is of the Nature Intelligence and Bliss and He is Self-manifest. When he wants to destroy all this during the Pralaya, the very powerful Sankar\d{s}a\d{n}a Rudra, well arrayed with the eleven Vy\=uhas, military (squadrons) arrangements, springs up from Him. From His Central Eyebrow, looking wide with His Three Eyes and raising His Trident, resplendent with three flames. All the (prominent) principal snakes, ruling over many others, come to Him during the nights filled with devotion and surrounded with Bhaktas (devotees) and bow down to Him with their heads bent low and look at each other's faces, enlightened with the lights from the jewels shining with clear lustre, on the nails of the red toes of His Lotus-Feet. At that time their faces become brilliant with the rays emitting from the jewels on the top of their very gay encircled hoods; and their cheeks look beautiful and shining. The daughters of the N\=aga R\=aja also do like this; when very beautiful rays come out of their perfectly excellent bodies. Their arms are wide extended; they look very clear and they are beautifully white. They use always Sandalpaste, Aguru and K\=a\d{s}m\={\i}ri unguents. Being overpowered by the amorous passion, due to their contact with those scented things, they look at Him with bashful glances and sweet smiles and expect \=Asiss (benedictions) from Him. And then His eyes roll maddened with love and express signs of kindness and mercy. The Bhagav\=an Ananta Deva is of boundless strength; His attributes are infinite; He is the ocean of infinite qualities. He is the \=Adi Deva, of a very good nature and His Nature is highly luminous. He has abandoned anger and envy and He wants the welfare of all. All the Devas worship Him and He is the repository of all S\=attvic qualities.

The Devas, Siddhas, Asuras, Uragas, Vidy\=adharas, Gandharbas, and Munis always meditate on Him. On account of His constant Mada R\=aga the enthusiasm and intoxication, His sight appears intoxicated and His eyes look perturbed with emotions. He is always pleasing to those who surround him and to the Devas by His sweet nectar-like words.

The Vaijayant\={\i} garland hangs from His neck; it never wanes and it is always decorated with the fresh and clear Tulas\={\i} leaves. The

maddened bees make their humming noises incessantly and thus add to the beauty. He is the Deva of the Devas and He wears a blue coloured cloth and He is ornamented with only one earring. He (the God Vi\d{s}\d{n}u) Undecaying and Immutable; resting His fleshy arms on the Halakakuda (the lofty portion of the plough), He is upholding the golden girdle as the elephant Air\=avata of Indra upholds the golden girdle. O N\=arada! The devotees describe Him as the Source of this Leel\=a of the Universe and the Controller of the Devas.

Here ends the Twentieth Chapter of the Eighth Book on the narrative of the Tal\=atala in the Mah\=apur\=a\d{n}am \'Sr\={\i} Mad Dev\={\i} Bh\=agavatam, of 18,000 verses, by Mahar\d{s}i Veda Vy\=asa.



