\chapter{On the quarrels between Hari\'schandra and Vi\'sv\=amitra}

1-12. S\=uta said :-- Here, on the other hand, one day the boy Rohit\=a\'sva went out with other boys to play at some place close to K\=a\d{s}'\={\i}. He first played with the his comrades; he then began to root out and collect, as far as he could, the Darbha (Ku\'sa) grass, with its ends and which had not deep roots. On being questioned why he was taking the Dharba grass, Rohit\=a told his comrades that his master was a Br\=ahmi\d{n} and that he was collecting them for his satisfaction. Saying this,

he began to collect carefully by his hands the sacrificial fuel (Samidha) and other fuel for the burning purposes. He collected the Pal\=asa wood for Homa purpose and making it into a bundle with other articles already collected, took it on his head, but at every step he seemed to be fatigued. Feeling thirsty he went to a pool of water close by and keeping his load on the ground went down to drink water. Drinking water he rested a while and then as he had kept his load on the anthill, be began to take it back on his head, a very poisonous deadly serpent came out suddenly out of that anthill at the order of Vi\'sv\=amitra. The snake immediately bit the boy who instantly fell down and died. His comrades seeing Rohit\=a\'sva dead went to the house of the Br\=ahmi\d{n}. With much anxiety the boys went soon out of fear, to his mother and said :-- ``O Br\=ahmi\d{n}'s maidservant! Your son went out with us to play outside; but suddenly a poisonous snake bit him and he is dead.'' Rohit\=a's mother, hearing these cruel words like thunder and lightning at once fell down on the ground like a plantain tree, cut off from its roots. The Br\=ahmi\d{n}, then, came and sprinkled water on her face. When she regained her consciousness, the Br\=ahmi\d{n} then angrily spoke :--

13-19. O wicked One! It is very inauspicious to cry at the evening time; especially the disfavour of the Goddess Lak\d{s}m\={\i}; the poverty comes to the householder, you know this; why are you then weeping? Have you not a bit of shame in your heart? She made no reply at this. Rather very much immersed in grief for her son, she wept in a pitiful voice. Her body was covered with dust, hairs were dishevelled and her face covered all over with tears from eyes. She constantly wept out of sorrow. The Br\=ahmi\d{n}, then, became very angry and spoke to the queen :-- ``O Villain! O Wicked! Fie on you. I have bought you for money; yet you are hindering my luck. If you had this thought that you would not work under me, why did you take for nothing my money?'' Thus repeatedly scolded by the Br\=ahmi\d{n}, she pitifully cried and spoke to the Br\=ahmi\d{n} in a voice choked with feelings :-- ``O Lord! My son has fallen into the jaws of death, being smitten by a serpent. O One of good vows! I will never be able to see him. So kindly permit me to go and see my boy.'' Saying thus, that lady began again to weep in a pitiful voice. The Br\=ahmi\d{n} became very angry and spoke thus :--

20-26. O Cheat! Your conduct is extremely blameable; you do not know how one commits a sin. The man who taking his pay from his master spoils his master's work, he goes to the terrible hell Raurava and is being scorched there. Living in the Hell for a short while, he is born as a cock. Or it is useless for me to give you this instruction of the Dharma \'S\=astra, for to speak to such to an illiterate, cruel, low, hypocrite

and liar and to one addicted to sinful acts is to sow seed on an usar land and to see it fruitless. Now if you have any fear for the afterlife, come and do the household affairs. Hearing this, she said to the Br\=ahmi\d{n}, trembling :-- ``O Lord! Be graciously pleased and shew your mercy on a maidservant. Only for a moment I will go to see the dead son of mine; so give me order to go there for a moment.'' That lady was deeply absorbed with sorrows for her son; then she put her head on the feet of the Br\=ahmi\d{n} and with a pitiful voice cried. The angry Br\=ahmi\d{n} with eyes reddened then began to speak.

27-41. What purpose of mine will be served by your son? Don't you know about my anger? Have you forgotten about my whipping? So be ready and do my household work without any delay. Hearing his words, the queen held her patience and began to do the household work. She spent half the night time, when she finished champooing his feet. When this was over, the Br\=ahmi\d{n} spoke to her :-- ``You can go now to your son; but see, finish his burning ceremonies and come back quickly. See that my morning works do not suffer.'' Thus getting the permission, the Queen went at that dead of night to look for her son, alone and weeping. Gradually she went out of the precincts of the city of K\=a\d{s}\={\i} and there she saw her son like a poor man's son lying on the ground over leaves and pieces of woods. Seeing her son dead, the humble Queen was very troubled with sorrow like an antelope, straying from its herd and as a cow missing her calf. The Queen M\=adhav\={\i} then began to lament, in a very pitiful tone, thus :-- ``O my Son! Come once before me; say why you are angry. Oh! My child! You used to come frequently to me, uttering Ma! Ma! Then why are you not coming now?'' Saying thus, she tumbling went and fell over his son. She, regaining her consciousness, embraced her son and placing her face on the face of the child began to weep pitifully. ``Oh! My son! Oh! My child! Oh my Kum\=ara! Oh! My Beautiful! and began to beat her head and her breast with her hands. O King! Where are you now? You used to look upon your son dearer than even your life. Your that son is now lying dead on the ground. Come and behold him once. It seems that the son has got back his life.'' Thus thinking she looked upon his face; but when it looked dead, she fell immediately unconscious. Getting back soon her consciousness, she held his face by her hands and said :-- ``O Child! Rise up from your sleep; awake; now is the dreadful night time; hundreds of jackals are yelling into our ears. Even Pretas, Bhutas, Pi\'s\=achas and D\=akin\={\i}s are roaming in packs and making terrible sounds Hum, Hum. Your comrades returned to their homes just at sunset; Why are you alone remaining here?

42-56. S\=uta said :-- The thin-bodied queen, thus saying, began to lament, ``Oh my Child! Oh! My son, Oh! Rohit\=a\'sva, O Kum\=ara, why are you not replying to my words! Oh my Child! I am your mother; do you not recognise me; look at me once. O Child! I am deprived of my kingdom and exiled from my country; my husband has sold even his body and I am myself reduced to slavery. What man is there that can live in this state! I am living simply by seeing your lotus-face. The astrologer who cast your horoscope at your birth, calculated future events in your life; but where? none of them is fructified. They said that this child will be a hero, warrior, long-lived, very charitable man, and always ready to do the worship of the Devas, Dv\={\i}jas and the Gurus. What more than this that the child will be one paramount sovereign and with his sons and grandsons will enjoy his kingdom. This boy will be the master of his senses and will fulfil the desires of his father and mother. Oh my Son! Now all those predictions have turned out false. O Child! You have on your palms so many auspicious signs, discus, fishes, umbrella, \'Sr\={\i} Vatsa, Svastika, flags, Kala\'sa (earthen jar), Ch\=amara and other signs; besides these, various other auspicious omens exist on your hands. Are all these become in vain today! O Son! You are the Lord of this whole dominion; but where are your that Kingdom now, those ministers, that royal throne, that umbrella, that axe, that vast amount of riches, that Ayodhy\=a city, those palatial buildings, those elephants, horses, and chariots? Where have gone your subjects! O Child! Where have you gone now, quitting all these and even me! O beloved Husband! See the condition of your son who in his early childhood used to move on all fours (the hands and feet) and get up on your broad chest, anointed with Kumkum, and spoil it with dust; O King! Come once and witness the condition of your child who used to press, out of ignorance due to his young age, the Tilak on your forehead, prepared of Mrigan\=abhi, (musk). Alas! Flies are now sitting on the lotus face today which I used to kiss over, covered with dirt; the insects are now stinging that. Oh! This I have got to witness now! O King! Come and see once your child is now sleeping on the ground like a poor man's dead son. O Fate! What bad act did I commit in my past life, that I have got to suffer so much in this life and I do not get an end of them! O Child! O Son! Oh, my Kum\=ara! Oh! My Beautiful! Shall I not be able to see you once any more elsewhere?'' The Queen M\=adhav\={\i} thus lamented very much when the warders of the city, hearing her lamentations awoke and came to her without any delay, greatly astonished. They asked her thus :--

57-77. Who are you? Whose son is this? Where is your husband? Why are you weeping here in this dead of night, without any fear? Though thus questioned, the thin Queen did not reply anything. Being again asked, she remained silent; and in the next moment she was pained with extreme agony and began again to cry. Tears flowed incessantly from her two eyes out of her sorrow. The guards then began to suspect her and were greatly afraid. So much that hairs stood on their ends out of terror. They at once raised their arms and began to talk with each other. When this lady is not giving any sort of reply, she is then certainly not a woman; most probably she will be a R\=ak\d{s}as\={\i}, knowing magic and destroying young children. So she should be killed with great attention. If she be not a R\=ak\d{s}as\={\i}, then why she should stay in this dead of night outside the city? No doubt, this R\=ak\d{s}as\={\i} has brought someone's child to eat here. Thus saying, they, without any delay, tied her hairs closely and some caught hold of her hand and some caught hold of her neck, saying O R\=ak\d{s}as\={\i}! where will you go now? The armed men, then dragged her perforce to the house of the Ch\=and\=ala and handed her over to him. All the people said :-- ``O Chief of the Ch\=and\=alas! We have caught today outside the city this child eating R\=ak\d{s}as\={\i}; so you better take her quickly on the slaughter ground and slaughter her.'' The Ch\=and\=ala looked at her body and said, ``This R\=ak\d{s}as\={\i} is widely celebrated in this world. I know her from before; but nobody is able to see her. This M\=ay\=avin\={\i} has devoured many sons of many persons. You all will acquire great merit when she will be slaughtered and your good name will be known to all and will last long. You better now go back to your own homes. The man who kills women, children, cows and Br\=ahmi\d{n}s, who burns another's house with fire, who destroys the wayfares of others, who steals his Guru's wife, who quarrels with saintly persons, and who drinks wine, if killed, will certainly yield merits to the man who kills him. If such a one be a female or a Br\=ahmi\d{n}, no sin will accrue if he or she be slaughtered.

So it is my paramount duty to kill her.'' Saying this, the Ch\=and\=ala tied her closely and drawing her by her hairs, began to beat her with a rope. Then he told to Hari\'schandra in terse language :-- ``O Slave! Kill her; this woman is by her very nature wicked; so do not judge anything in this matter of killing her.'' Hearing these harsh words, like the falling of a thunderbolt, the King shuddered. When he came back to his nature, he fearing lest a woman be killed, said to the Ch\=and\=ala :-- ``I am not at all able to carry this order out; so kindly make over this task to some other servant of yours. He will kill her. I will certainly carry out any other order that you would task me to do.'' Thus hearing

the King, the Ch\=and\=ala said :-- Discard your fear and take the sword; this M\=ay\=avin\={\i} kills always the children; so to kill her is meritorious; in no way whatsoever ought she to be saved. The King became very sorry and said :-- Women should always be protected with care, never to be killed; the more so as the religious Munis have assigned greater sin in the killing of women. The man who kills consciously or unconsciously females, certainly becomes boiled in the Mah\=a Raurava hell.

78-79. The Ch\=and\=ala said :-- ``Don't you say this; take this sharp sword, lustrous like a lightning; where killing one engenders happiness to many, abundance of merits are acquired in doing that. This wicked fellow has eaten many children of this place; so kill her as early as possible and bring peace and happiness to the K\=a\d{s}\={\i} people.''

80. The King said :-- ``O Chief of the Ch\=and\=alas! I have taken the difficult vow from my childhood, not to kill any woman. Therefore I cannot exert myself in this matter of killing the woman as you order.''

81-82. The Ch\=and\=ala said :-- ``O Wicked Fellow! No work is superior which is not the master's work. Why then are you cancelling today to carry out my order, when you are taking pay from me. The servant that spoils his master's work, taking his money, is not freed from the hell even if he remains for ten thousand years there.''

83-86. The King said :-- ``O Lord of the Ch\=and\=alas! Put me to some other task that is very difficult. I will do that easily. Or if you have an enemy, specify and I will kill him no doubt within an instant. I will give you the whole earth by killing him. Even if Indra comes against you with the other Devas, or D\=anavas, or Uragas, or Kinnaras, or Siddhas, or Gandharbas, I will slay him with my sharpened arrows, but I will never be able to kill a woman.'' The Ch\=and\=ala, then, began to tremble with anger at these words and said to the King.

87-89. You are a servant and what you have spoken is not fit for a servant. Working as a slave of a Ch\=and\=ala, you are speaking the words of the gods. Therefore, O slave! hear now what I say; no need of exchanging any further words. O Shameless One! If you fear sin a bit why then did you accept the slavery in a Ch\=and\=ala's house. Take this sword and cut off her head.'' Thus speaking the Ch\=and\=ala gave him the axe.

Here ends the Twenty-fifth Chapter of the Seventh Book on the quarrels between Hari\'schandra and Vi\'sv\=amitra in the Mah\=a Pur\=a\d{n}am, \'Sr\={\i} Mad Dev\={\i} Bh\=agavatam, of 18,000 verses, by Mahar\d{s}i Veda Vy\=asa.



