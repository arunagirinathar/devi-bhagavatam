\chapter{On the description of the \'Saktis, etc., of the syllables of G\=ayatr\={\i}}

1-18. N\=ar\=aya\d{n}a said :-- O N\=arada! O Great Muni! Now hear which are the \'Saktis in due order of the twenty four syllables of the G\=ayatr\={\i} Dev\={\i}:--

(1) V\=ama Dev\={\i}, (2) Priy\=a, (3) Saty\=a, (4) Vi\'sv\=a, (5) Bhadravil\=asin\={\i}, (6) Prabh\=a Vat\={\i}, (7) Jay\=a, (8) \'S\=ant\=a, (9) K\=ant\=a, (10) Durg\=a, (11) Sarasvat\={\i}, (12) Vidrum\=a, (13) Vi\'s\=ale's\=a, (14) Vy\=apin\={\i}, (15) Vimal\=a, (16) Tamopah\=ari\d{n}\={\i}, (17) S\=uk\d{s}m\=a, (18) Vi\'svayoni, (19) Jay\=a, (20) Va\'s\=a, (21) Padm\=alay\=a, (22) Par\=a\'sobh\=a, (23) Bhadr\=a, and (24) Tripad\=a.

Now hear the respective colours of the several syllables of the G\=ayatr\={\i} Dev\={\i} :-- (1) like Champaka and Atas\={\i} flowers, (2) like Vidruma, (3) like crystal, (4) like lotus; (5) like the Rising Sun; (6) white like conchshell; (7) white like Kunda flower; (8) like Prab\=ala and lotus leaves; (9) like Padmar\=aga, (10) like Indran\={\i}lama\d{n}i; (11) like pearls; (12) like Saffron; (13) like the black collyrium of the eye; (14) red; (15) like the Vaid\=urya ma\d{n}i; (16) like Ksaudra (Champaka tree, honey, water); (17) like turmeric; (18) like Kunda flower; and the milk (19) like the rays of the Sun; (20) like the tail of the bird \'Suka; (21) like \'Satapatra; (22) like Ketak\={\i} flower; (23) like Mallik\=a flower; (24) like Karav\={\i}ra flower.

Now about their Tattvas :-- (1) earth; (2) water; (3) fire; (4) air; (5) \=Ak\=a\'sa (ether); (6) smell; (7) taste; (8) form; (9) sound; (10)

touch; (11) male generative organ; (12) anus; (13) legs, (14) hands; (15) speech; (16) Pr\=a\d{n}a (vital breath); (17) tongue; (18) eyes; (19) skin; (20) ears; (21) Pr\=a\d{n}a (up going breath); (22) Ap\=ana; (23) Vy\=ana, (24) S\=am\=ana.

Now about the Mudr\=as of the syllables:-- (l) Sammukha; (2) Samputa; (3) Vitata; (4) Vistrita; (5) Dvimukha, (6) Trimukha; (7) Chaturmukha; (8) Pa\~nchamukha; (9) Sa\d{n}mukha; (10) Adhomukha; (11) Vy\=apak\=anjali (12) \'Sakata; (13) Yamap\=a\'sa; (14) Grathita; (15) Sanmukhon mukha (16) Vilamba; (17) Mustika; (18) Matsya; (19) K\=urma; (20) Var\=ahaka; (21) Simh\=akr\=anta, (22) Mah\=akr\=anta; (23) Mudgara, and (24) Pallava.

The Mah\=amudr\=as of the fourth foot of G\=ayatr\={\i} are (1) Tris\=ulayon\={\i} (2) Surabhi; (3) Ak\d{s}a m\=al\=a; (4) Li\d{n}ga; and (5) Ambuja. O N\=arada! Thus I have described to you all about the Mudras, etc., of the several syllables of the G\=ayatr\={\i}. If during Japam, one thinks all these and at the same time repeats, all his sins are destroyed and his wealth gets increase and the fame attends on him.

Here ends the Second Chapter of the Twelfth Book on the description of the \'Saktis, etc., of the syllables of G\=ayatr\={\i} in the Mah\=apur\=a\d{n}am \'Sr\={\i} Mad Dev\={\i} Bh\=agavatam of 18,000 Verses by Mahar\d{s}i Veda Vy\=asa.



