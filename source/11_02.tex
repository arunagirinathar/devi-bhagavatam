\chapter{On cleansing the several parts of the body}

1-42. N\=ar\=aya\d{n}a said :-- Even if a man studies the Vedas with six Amgas (limbs of the Vedas), he cannot be pure if he be devoid of the principle of right living (Sad\=ach\=ara) and if he does not practise it. All that is in vain. As soon as the two wings of the young ones of birds appear they leave their nests, so the chhandas (the Vedas) leave such a man devoid of Sad\=ach\=ara at the time of his death. The intelligent man should get up from his bed at the Br\=ahma muh\=urta and should observe all the principles of Sad\=ach\=ara. In the last quarter of night, he should practise in reciting and studying the Vedas. Then for some time he should meditate on his Ista Deva (his Presiding Deity). The Yogi should meditate on Brahm\=a according to the method stated before. O N\=arada! If meditation be done as above, the identity of J\={\i}va and Brahm\=a is at once realised and the man becomes liberated while living. After the fifty-fifth Danda from the preceding sunrise, i.e., 2 hours before the sunrise comes the U\d{s}\=ak\=ala; after the fifty seventh Danda comes the Aru\d{n}odayak\=ala; after fifty eighth Danda comes the morning time; then the Sun rises. On should get up from one's bed in the morning time. He should go then to a distance where an arrow shot at one stretch goes. There in the south-west corner he is to void his urines and faeces. Then the man, if he be a Brahmach\=ar\={\i}, should place his holy thread on his right ear and the householder should suspend it on his neck only. That is, the Brahmach\=ar\={\i}, in the first stage of of his life should place the holy thread over his right ear; the householder and the V\=anaprasth\={\i}s should suspend the holy thread from the neck towards the back and then void their faeces, etc. He is to tie a piece of cloth round his head; and spread earth or leaves on the place where he will evacuate himself. He is not to talk then nor spit nor inhale hard. One is not to evacuate oneself in cultivated lands, that have been tilled, in water, over the burning pyre, on the mountain, in the broken and ruined temples, on the ant-hills, on places covered with grass, on road side, or on holes where living beings exist. One ought not

to do the same also while walking. One ought to keep silence during both the twilights, while one is passing urine or voiding one's faeces, or while one is holding sexual intercourse, or before the presence of one's Guru, during the time of sacrifice, or while making gifts, or while doing Brahm\=a Yaj\~na. One ought to pray before evacuating, thus :-- ``O Devas! O \d{R}i\d{s}is! O Pi\'s\=achas! O Uragas! O R\=ak\d{s}asas! You all who might be existing here unseen by me, are requested to leave this place. I am going to ease myself here duly.'' Never one is to void oneself while one looks at V\=ayu (wind), Agni (fire), a Br\=ahma\d{n}a, the Sun, water or cow. At the day time one is to turn one's face northward and at the night time southward, while easing oneself and then one is to cover the faeces, etc., with stones, pebbles, leaves or grass, etc. Then he is to hold his genital organ with his hand and go to a river or any other watery place; he is to fill his vessel with water then and go to some other place.

The Br\=ahma\d{n}a is to use the white earth, the K\d{s}attriya is to take the red earth, the Vai\'sya is to use the yellow earth and the \'S\=udra is to apply the black earth and with that he is to cleanse himself. The earth under water, the earth of any temple, the earth of an anthill, the earth of a mouse hole, and the remnant of the earth used by another body for washing are not to be used for cleansing purposes. The earth for cleansing faeces is twice as much as that used in case of urine clearance; in the cleansing after sexual intercourse thrice as much. In urine cleansing the earth is to be applied in the organ of generation once, thrice in the hand. And in dirt clearing, twice in the organ of generation, five times in anus, ten times in the left hand and seven times in both the hands. Then apply earth four times first in the left feet and then on the right feet. The house holder should clean thus; the Brahmach\=ari is to do twice and the Yatis four times. At every time the quantity of wet earth that is to be taken is to be of the size of an \=Amalak\={\i} fruit; never it is to be less than that. This is for the clearance in the day time. Half of these can be used in the night time. For the invalids, one-fourth the above measurements; for the passers-by, one-eighth the above dimensions are to be observed. In case of women, \'S\=udras, and incapable children, clearings are to be done till then when the offensive smell vanishes. No numbers are to be observed. Bhagav\=an Manu says for all the Var\d{n}as the clearing is to be done till then when the offensive smell vanishes. The clearing is to be performed by the left hand. The right hand is never to be used. Below the navel, the left hand is to be used; and above the navel the right hand is to be used for clearing. The wise man should never hold his water pot while evacuating himself. If by mistake he catches hold of his waterpot, he will have to perform the penance (pr\=aya\d{s}chitta).

If, out of vanity or sloth, clearing be not done, for three nights, one is to fast, drinking water only, and then to repeat the G\=ayatr\={\i} Mantra and thus be purified. In every matter, in view of the place, time and materials, one's ability and power are to be considered and steps are to be taken accordingly. Knowing all this, one should clear oneself according to rule. Never be lazy here. After evacuating oneself of faeces, one is to rinse one's mouth twelve times; and after passing urine and clearing, one is to rinse four times. Never less than that is to be done. The water after rinsing is to be thrown away slowly downwards on one's left. Next performing \=Achaman one is to wash one's teeth. He is to take a tiny piece, twelve \=Angulas (fingers) long (about one foot) from a tree which is thorny and gummy. The cleansing twig (for teeth) is thick like one's little finger. He is to chew the one end of it to form a tooth brush. Karanja, Udumbara (figtree), Mango, Kadamba, Lodha, Champaka and Vadar\={\i} trees are used for cleansing teeth. While cleansing teeth, one is to recite the following mantra :-- ``O Tree! Wherein resides the Deity Moon for giving food to the beings and for killing the enemies! Let Him wash my mouth to increase my fame and honour! O Tree! Dost Thou please give me long life, power, fame, energy, beauty, sons, cattle, wealth, intellect, and the knowledge of Brahm\=a.'' If the cleansing twig be not available and if there be any prohibition to brush one's teeth that day, (say, Pratipad day, Am\=avas, Sasthi and Navam\={\i}), take mouthfuls of water, gargle twelve times and thus cleanse the teeth. If one brushes one's teeth with a twig on the new moon day, the first, sixth, ninth and eleventh day after the Full or New Moon or on Sunday, one eats the Sun (is it were, by making Him lose his fire), makes his family line extinct and brings his seven generations down into the hell. Next he should wash his feet and sip pure clean water thrice, touch his lips twice with his thumb, and then clear the nostrils by his thumb and fore finger. Then he is to touch his eyes and ears with his thumb and ring finger, touch his navel with his thumb and little finger, touch his breast with his palm and touch his head with all his fingers.

Here ends the Second Chapter of the Eleventh Book on cleansing the several parts of the body in the Mah\=apur\=a\d{n}am \'Sr\={\i} Mad Dev\={\i} Bh\=agavatam of 18,000 verses by Mahar\d{s}i Veda Vy\=asa.



