\chapter{1-2. Janamejaya said :-- "O Muni! You told before that "the Highest Light took Her birth on the top of the Him\=alay\=as." Now describe to me in detail about this Highest Light. What intelligent man}

can desist from hearing these nectar-like words about the \'Sakti? The danger of death may come even to the Devas that drink nectars but no such danger can possibly come to those that drink the nectar of the Dev\={\i}'s glorious deeds.
3-43. Vy\=asa said :-- "O King! You are blessed; you have attained what you are to attain in this life; you are taught by the high-souled men; you are fortunate since you are so sincerely devoted to the Dev\={\i}. O King! Hear the ancient history :--Wherever the Deva of the Devas, the Mahe\'svara rested while He was wandering all over the world in a distracted state, carrying the Sat\={\i}'s body that as burnt by fire, He spent his time there with his senses controlled, in Sam\=adhi, forgetting all his knowledge of Sams\=ara in deep meditation of the form of the Dev\={\i}. At this time, the three worlds, with their objects, moving and immoving, with their oceans, mountains and islands became void of prosperity and power. The hearts of all the embodied beings became dried up, without any trace of joy; they were all burdened with anxious thoughts and remained indifferent. All were merged in the ocean of sorrows and became diseased. Planets retrograded and the Devas had their states reversed. The Kings were attacked with a series of ills and misfortunes. \=Adhibhantik and \=Adhidaivik (from material causes and from divine interference). At this time a great Asura, named T\=araka, became unconquerable owing to his receiving a boon from Brahm\=a. Being intoxicated by his power and heroism, he conquered the three worlds and became the sovereign ruler.The Brahm\=a Praj\=apati, gave him boon to this effect that the legitimate son of \'Siva would be able to kill him. And as at that time \'Siva had no son, the great Asura, elated with joy, became infatuated and carried off all victories. All the Devas were banished from their places by his oppression; they remained always anxious owing to the want felt by them of a son of \'Siva. "\'S\=ankara has now no wife; how can He then have a son! We are very unfortunate; how can our work be accomplished? Thus oppressed with thoughts, all the Devas went to Vaikuntha and informed the Bh\=agava\d{n} Vi\d{s}\d{n}u of all that had happened, in privacy. The Bh\=agavan Vi\d{s}\d{n}u began to tell them the means, thus :-- "O Devas! Why are you all so anxious when the Auspicious Goddess of the Universe, the Dweller in the Ma\d{n}i Dv\={\i}pa, the Yielder of all desires like a Kalpa Vrik\d{s}a is always wakeful for you. It is due to your faults that She is showing Her indifference; it is meant to teach us (not for our destruction but to show Her Infinite mercy). When a mother nourishes and frightens and reprimands a son, it is not that she has became merciless; so the World Mother, the Controller of the Universe, will never be merciless to you as regards your qualifications and defects. A son commits offence at every step who can bear that in these three

worlds except the mother! So soon take refuge to the Highest Mother, the Goddess of the universe, with the sincerest devotion. She will certainly take action and help your cause. Thus ordering the Devas, Vi\d{s}\d{n}u with His consort Lak\d{s}m\={\i} and the other Devas quickly went out to worship the Dev\={\i}. Going to the Him\=alay\=as, they soon engaged themselves in doing the Pura\'schara\d{n}a Karma (act of repeating the names of the Deity, attended with burnt oblations and offerings, etc.). O King! Those who were well versed with the performance of sacrifice to the Mother, began their sacrificial ceremonies and all began to hold vows, viz, Tritiy\=adi Vrat\=a\d{n}\={\i}. Some were engaged in incessantly meditating on the Dev\={\i}; some began to repeat Her names constantly; some began to repeat the Dev\={\i} S\=ukta. Thus some devoted themselves to repeating names; others to repeating mantrams. Again some wore engaged in performing severe (painful) Ch\=andr\=aya\d{n}a and other Vratas. Some were doing Antaray\=agas (inner sacrifices); some wore doing Pr\=an\=agnihotra Y\=agas; whereas others engaged themselves in Ny\=as\=adi, etc. Again some began to worship the Highest \'Sakt\={\i}, the Goddess of the Universe, without any sleep or rest, by the seed mantra of M\=ay\=a. O King! Thus many years of the Devas passed away. When the ninth Tith\={\i} came in the month of Chaitra on Friday, the Highest Light of the Supreme Force suddenly appeared in front of them. That Light was equal to Koti lightnings, of a red colour, and cool like the Koti Moons. Again the lustre was like the Koti Suns. The four Vedas personified, were chanting hymns all round Her. That mass of fire was above, below, on all sides, in the middle; nowhere it was obstructed. It had no beginning, nor end. It was of the form of a female with hands and feet and all the limbs. The appearance was not that of a male nor that of an hermophrodite. The Devas, dazzled by the brilliant lustre, first closed their eyes; but at the next moment, holding patience when they opened again their eyes, they found the Highest Light manifesting in the form of an exceedingly beautiful Divine Woman. Her youth was just blooming and Her rising breasts, plump and prominent, vying as it were, with a lotus bud, added to the beauty all around. Bracelets were on Her hands; armlets on Her four arms; necklace on Her neck; and the garland made of invaluable gems and jewels spread very bright lustre all arouud. Lovely ornaments on Her waist making tinkling sounds and beautiful anklets were on Her feet. The hairs of Her head, flowing between Her ears and cheek sparkled bright like the large black bees shining on the flower leaves of the blooming Ketak\={\i} flower. Her loins were nicely shaped and exquisitely lovely and the hairs on Her navel gave additional beauty. Her exquisitely lively lotus mouth rendered more lustrous and beautiful by the shining golden ear-ornaments, was filled with betel leaves mixed with camphor, etc.; on Her forehead there was

the half crescaut moon; Her eye-brows were extended and Her eyes looked bright and beautifully splendid like the red lotus; Her nose was elevated and Her lips very sweet. Her teeth were very beautiful like the opening buds of Kunda flowers; from Her neck was suspended a necklace of pearls; on Her head was the brilliant crown decked with diamonds and jewels; on Her ears, earrings were suspended like the lines on the Moon; Her hairs were ornamented with Mallik\=a and M\=alat\={\i} flowers; Her forehead was pasted with K\=a\d{s}m\={\i}ra Kunkuma drops; and Her three eyes gave unparallelled lustre to Her face. On Her one hand there was the noose and on Her other hand there was the goad; her two other hands made signs granting boons and dispelling fears; Her body shed lustre like the flowers of a D\=arima tree. Her wearing is a red coloured cloth. All these added great beauty. Thus the Devas saw before them the Mother Goddess, the Incarnate of unpretended mercy, with a face ready to offer Her Grace, the Mother of the Whole Universe, the Enchantress of all, sweet-smiling, saluted by all the Devas, yielding all desires, and wearing a dress, indicative of all lovely feelings. The Devas bowed at once they saw Her; but they could not speak with their voice as it was choked with tears. Then holding their patience, with much difficulty, they began to praise and chant hymns to the World Mother with their eyes filled with tears of love and devotion and with their heads bent low.
44-54. The Devas said :-- We bow down to Thee, the Dev\={\i} and the Mah\=a Dev\={\i}, always obeisance to Thee! Thou art the Prakriti, and the Auspicious One; we always salute to Thee. O Mother! Thou art of a fiery colour (residing as a Red Flame in the heart of a Yog\={\i}) and burning with Asceticism and Wisdom (shedding lustre all around). Thou art specially shining everywhere as the Pure Chaitanya; worshipped by the Devas and all the J\={\i}vas) for the rewards of their actions; We take refuge to Thee, the Durg\=a, the Dev\={\i}, we bow down to Thee, that can well make others cross the ocean of Sams\=ara; so that Thou helpest us in crossing this terrible ocean of world. Mother! The Devas have created the words (i.e., the words conveying ideas are uttered by the five V\=ayus, Pr\=a\d{n}a, etc., which are called the Devas) which are of the nature of Vi\'svar\=upa, pervading everywhere, like the K\=ama Dhenu (the Heavenly Cow yielding all desires, riches, honor, food, etc.), and by which the brutes (the gods) become egotistical, O Mother! Thou art that language to us; so Thou fulfillest our desires when we praise and chant hymns to Thee. O Dev\={\i}! Thou art the Night of Destruction at the end of the world; Thou art worshipped by Brahm\=a; Thou art the Lak\d{s}mi, the \'Sakti of Vi\d{s}\d{n}u; Thou art the Mother of Skanda; the \'Sakt\={\i} of \'Siva; Thou art the \'Sakt\={\i} Sarasvat\={\i} of Brahm\=a. Thou art Aditi,

the Mother of the gods and Thou art Sat\={\i}, the daughter of Dak\d{s}a. Thus Thou art purifying the worlds in various forms and giving peace to all. We bow down to Thee. We know Thee to be the great Mah\=a Lak\d{s}m\={\i}; we meditate on Thee as of the nature of all the \'Saktis as Bhaghavat\={\i}. O Mother! Illumine us so that we can meditate and know Thee. O Dev\={\i}! Obeisance to Thee, the Vir\=at! Obeisance to Thee, the S\=utr\=atm\=a, the Hira\d{n}yagarbha; obeisance to Thee, the transformed into sixteen Vikritis (or transformations). Obeisance to Thee, of the nature of Brahma. We bow down with great devotion to Thee, the Goddess of the Universe, the Creatrix of M\=ayic Avidy\=a (the Nescience) under whose influence this world is mistaken as the rope as a garland is mistaken for a rope and again that mistake is corrected by whose Vidy\=a.
We bow down to Thee who art indicated by both the letters Tat and Tvam in the sentence Tat Tvamasi (Thou art That), Tat indicating the Chit (Intelligence) of the nature of oneness and Tvam indicating the nature of Akhanda Brahma (beyond the Annamaya, Pr\=a\d{n}amaya, Manomaya, Vijn\=anamaya and the \=Anandamaya--the five Ko\'sas, the Witness of the three states of wakefulness, dream, and deep sleep states) and indicating Thee. O Mother! Thou art of the nature of Pra\d{n}ava Om; Thou art Hr\={\i}m; Thou art of the nature of various Mantras and Thou art merciful; we bow down again and again to Thy lotus Feet. When the Devas thus praised the Dev\={\i}, the In-dweller of the Ma\d{n}i Dv\={\i}pa, the Bhagavat\={\i} spoke to them in a sweet cuckoo voice.
55. O Devas! What for have you come here? What do you want? I am always the Tree, yielding all desires to my Bhaktas; and I am ready to grant boons to them.
56-57. You are my devotees; why do you care, when I am on your side? I will rescue you from the ocean of troubles, O Devas! Know this as My true resolve. O King! Hearing these words of deep love, the Devas became very glad and gave out all their causes of troubles.
58-65. O Parame\'svar\={\i}! Thou art omniscient and witness of all these worlds. What is there in the three worlds that is not known to Thee! O Auspicious Mother! The Demon T\=araka is giving us troubles day and night. Brahm\=a has given him boon that he will be killed by the \'Siva's son. O Mahe\'svar\={\i}! Sat\={\i}, the wife of \'Siva has cast aside Her body. It is known to Thee. What will the ignorant low people inform the one, Who is Omniscient? O Mother! We have described in brief all what we had to say. What more shall we say? Thou knowest all our other troubles and causes of sorrows. Bless us so that our devotion remains unflinched at Thy lotus feet; this is our earnest prayer. That Thou

takest the body to have a son of \'Siva is our fervent prayer to Thee. Hearing the Dava's words, Parame\'svar\={\i}, with a graceful countenance, spoke to them, thus :-- "My \'Sakt\={\i} will incarnate as Gaur\={\i} in the house of Him\=alay\=as; She will be the wife of \'Siva and will beget a son that will destroy T\=araka De
mon and will serve your purpose. And your devotion will remain steadfast at My Lotus feet. Him\=alay\=as, too, is worshipping Me with his wholehearted devotion; so to take birth in his house is to my greatest liking; know this.
66-73. Vy\=asa said :-- "O King! Hearing the kind words of the Dev\={\i}, the King of mountains was filled with love; and, with voice choked with feelings and with tears in his eyes spoke to the Goddess of the world, the Queen. of the three worlds. Thou hast raised me much higher, that Thou dost me so great a favour; otherwise where am I inert, and unmoving and where art Thou, of the nature of Existence, Intelligence and Bliss! It manifests the Greatness of Thy Glory. O Sinless One! My becoming the father of Thee indicates nothing less than the merits earned by me for doing, countless A\'svamedha sacrifices or for my endless Sam\=adhi. Oh! What a favour hast Thou shewn towards me! Henceforth my unparalleled fame will be spread throughout the whole Universe of five original elements that "The Upholder of the Universe, the World Mother has become the daughter of this Him\=alay\=as! This man is blessed and fortunate!" Who can be so fortunate, virtuous and merited as he whose daughter She has become, Whose belly contains millions of Brahm\=andas! I cannot describe what pre-eminent heavens are intended for my Pitris, my family predecessors, wherein virtuous persons like myself are born. O Mother! O Parame\'svar\={\i}! Now describe to me Thy Real Self as exemplified in all the Ved\=antas; and also J\~n\=ana with Bhakti approved by the Vedas in the same way that Thou hast shown already this favour to me, so that by That Knowledge I will be able to realise Thy Self.
74. Vy\=asa said :-- "O King! Thus hearing the praise of Him\=alayas, the Goddess of the Universe, with a graceful look, began to speak the very secret essences of the \'Srutis.
Here ends the Thirty-first Chapter of the Seventh Book on the birth of P\=arvat\={\i} in the House of Him\=alay\=as in the Mah\=apur\=a\d{n}am \'Sri Mad Dev\={\i} Bh\=agavatam of 18,000 verses, by Mahar\d{s}i Veda Vy\=asa.



