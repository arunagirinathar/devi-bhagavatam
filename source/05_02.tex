\chapter{On the birth of D\=anava Mahi\d{s}a}

1-2. The king said :-- ``Lord! You have described fully the glory of the Mah\=a May\=a Yoge\'svar\={\i}; now describe Her Life and Character; I am very eager to hear them. This whole universe, moving and nonmoving, has been created by Mahe\'svar\={\i}; who is there that desires not to hear Her Glory!''

3-7. Vy\=asa spoke :-- O king! You are very intelligent; I will describe in detail all this to you; whoever does not describe Her Glory to the peaceful and faithful, is certainly low-minded? In days of yore, a terrible battle ensued between the Devas and D\=anava forces on this earth when Mahi\d{s}\=asura was the Ruler of this world. O king! Mahi\d{s}\=asura went to the mountain of Sumeru and performed a very severe and excellent tapasy\=a, wonderful even to the gods. O king! Meditating on his Ista Devat\=a (the deity for his worship) in his heart, elapsed full ten thousand years, when Brahm\=a the Grandfather of all the Lokas, was pleased with him. The fourfaced Brahm\=a, arrived there on his vehicle the swan, asked Mahi\d{s}\=asura ``O One of virtuous soul! Ask from me what is your desired object; I will grant thee boon.''

8. Mahi\d{s}a said :-- ``O Lord, Lotus-eyed! I want to become immortal! therefore O Thou, the Grandfather of the Devas! Dost thou do for me so that I have no fear of death.''

9-11. Brahm\=a said :-- ``O Mahi\d{s}a! Birth must be followed by death, and death must be followed by birth; this is the eternal law of nature. Then know this as certain that when one is born, one must die; and when one dies, one will be born. O Lord of the D\=anavas! What more to say than this, that high mountains, vast oceans, and all the beings will die when time will come. O Ruler of the earth! You are virtuous; therefore ask any other boon than this immortality; I will grant that to you.''

12-13. Mahi\d{s}a said :-- ``O Grand Sire! Grant, then, that no Deva, D\=anava, nor human being of the male sex can cause my death. There is none among women who can cause my death. Therefore, O Lotus-eyed! Let woman be the cause of my death; how can women slay me! The are too weak to kill me!''

14. Brahm\=a said :-- ``O Lord of the D\=anavas! Your death will certainly occur, at any time, through a woman; O Highly Fortunate One! No man will be able to cause your death.''

15. Vy\=asa said :-- Thus granting him the boon, Brahm\=a went to his own abode; the lord of the D\=anavas, too, returned to his place, very glad.

16. The king said :-- ``O Bhagav\=an! Whose son was this powerful Mahi\d{s}\=asura; how his birth took place? and why, too, did he get a body of a buffalo?''

17-26. Vy\=asa said :-- O king! Rambha and Karambha were the two sons of Danu; these two D\=anavas were far famed in this world for their pre-eminence. O king! They had no issues; hence, desirous of issues, they went to the sacred banks of the Indus (Pa\~ncha Nada) and there performed severe asceticism for long years. Karambha got himself submerged in water and thus began his severe tapasy\=a; while the other, Rambha, had recourse to a juicy peepul tree (haunted by Yakshin\={\i}s) and there began to worship Fire. Rambha remained, engaged in worshipping the Five Fires; knowing this, Indra, the Lord of \'Sach\={\i}, was pained and hurried thither, being very anxious. Going to Pa\~ncha Nada, Indra assumed the form of a crocodile and caught hold of the legs of the wicked Karambha and killed him. Hearing of the death of his brother, Rambha got very much enraged and wishing to offer his own head as an oblation to the Fire, he wanted to cut off his own head; he, being infuriated, held the hairs of his head by his left hand, and, catching hold of a good axe, by his right hand, was on the point of cutting it, when the Fire gave him knowledge, desisted him from this act and spoke thus :-- You are stupid; why have you desired to cut off your own head; killing one's ownself is a great sin; and there is no means of deliverance from this sin. Why are you then ready to execute it? Do not seek your death now; what end will that serve you? Rather ask boons from me; thus you will get your welfare.

27-31. Vy\=asa said :-- O king! Hearing thus the sweet words of Fire, Rambha quitted the hold of his hairs and said :-- O Lord of the Devas! If thou art pleased, grant my desired boon that a son be born unto me, who will destroy the forces of my enemy and who will conquer the three worlds. And that son be invincible in every way by the Devas, D\=anavas and men, very powerful, assuming forms at will, and respected by all. The Fire said :-- O highly Fortunate! You will get your son, as you desire; therefore desist now from your attempting suicide. O highly fortunate Rambha! With any female of whichever species, you will co-habit, you will get a son, more powerful than you; there is no doubt in this.

32-50. Vy\=asa said :-- O king! Hearing thus the sweet words of the Fire as desired, Rambha, the chief of the D\=anavas, went, surrounded by Yaksas, to a beautiful place, adorned with picturesque sceneries; when one lovely she-buffalo, who was very maddened with passion, fell to the sight of Rambha. And he desired to have sexual intercourse with her, in preference to other women. The she-buffallo, too, gladly yielded to his purpose and Rambha had sexual intercourse with her, impelled as it were by the destiny. The she-buffalo became pregnant with his semen virile. The D\=anava, too, carried the she-buffalo, his dear wife, to P\=at\=ala (the lower regions) for her protection. On one occasion, another buffalo got excited and wanted to fall upon the she-buffalo. The D\=anava was also ready to kill him. The D\=anava came hurriedly and struck the buffalo for the safety of his wife; whereon the excited buffalo attacked him with his horns. The buffalo struck him so violently with his sharp horns that Rambha fell down senseless all on a sudden and finally died. Seeing her husband dead, the she-buffalo quickly fled away in distress and, with terror, she quickly went to the peepul tree and took refuge under the Yaksas. But that buffalo, excited very much and maddened with vigour, ran in pursuit of her, desiring intercourse with her. On seeing the miserable plight of the weeping she-buffalo, distressed with fear, and seeing the buffalo in pursuit of her, the Yaksas assembled to protect her. A terrible fight ensued between the buffalo and the Yaksas, when the buffalo, shot with arrows by them, fell down and died. Rambha was very much liked by the Yaksas; so they cremated his dead body for its purification. The she-buffalo, seeing her husband laid in the funeral pyre, expressed her desire to enter also into that fire. The Yaksas resisted; but that chaste wife quickly entered into the burning fire along with her husband. When the she-buffalo died, the powerful Mahi\d{s}a rose from his mother's womb from the midst of the funeral pyre; Rambha, too, emerged from the fire in another form out of his affection towards his son. Rambha was known as Raktav\={\i}ja after he had changed his form. His son was thus born as a very powerful D\=anava and became famous by the name of Mahi\d{s}a. The chief D\=anavas installed Mahi\d{s}a on the throne. O king! The very powerful Raktav\={\i}ja and the D\=anava Mahi\d{s}a, thus took their births and became invincible of the Devas, D\=anavas and human beings. O king! I have now described to you the birth of the highsouled D\=anava Mahi\d{s}a and his getting the boon, all in detail.

Here ends the Second Chapter of the Fifth Book on the birth of Mahi\d{s}a D\=anava in the Dev\={\i} Bh\=agavatam, the Mah\=apur\=a\d{n}am composed of 18,000 verses by Mahar\d{s}i Veda Vy\=asa.



