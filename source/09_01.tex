\chapter{On the description of Prakriti}

1. \'Sri N\=ar\=aya\d{n}a said :-- This (Highest) Prakriti is recognised as five-fold. When She is engaged in the work of Creation, She appears as :-- (1) Durg\=a, the Mother of Ga\d{n}e\'sa, (2) R\=adh\=a, (3) Laksm\={\i}, (4) Sarasvat\={\i} and (5) S\=avitr\={\i}.

2-3. N\=arada replied :-- O Thou, the best of J\~n\=anins! Who is this Prakriti? (Whether She is of the nature of Intelligence or of matter?) Why did She manifest Herself and then again why did She reveal Herself in these five forms? And what are Her characteristics? Now Thou oughtest to describe the lives of all, the different modes of their worship, and the fruits that are accrued thereby. Please also inform me which Forms of them manifested themselves in which different places. Dost Thou please narrate to me all these.

4-18. N\=ar\=ayana said :-- ``O Child! Who is there in this world that can describe fully the characteristics of Prakriti! However I will describe to you that much which I heard from my own father, Dharma. Hear. The prefix ``Pra'' in the word Prakriti means exalted, superior, excellent; and the affix ``Kriti'' denotes creation. So the Goddess, the Dev\={\i} Who is the most excellent in the work of creation is known as the Dev\={\i} Prakriti. To come closer :-- ``Pra'' signifies the Sattva Gu\d{n}a, the most exalted quality, ``Kri'' denotes the Rajo Gu\d{n}a and ``Ti'' denotes the Tamo Gu\d{n}a. (The Sattva Gu\d{n}a is considered as the Highest as it is perfectly clear and free from any impurities whatsoever; the Rajo Gu\d{n}a is considered intermediate as it has this defect :-- that it spreads a veil over the reality of things, so as not to allow men to understand the True Reality, while the Tamo Gu\d{n}a is considered worst as it completely hides the Real Knowledge).

So when this Intelligence of the nature of Brahm\=a, beyond the three attributes, gets tinged with the above three Gu\d{n}as and becomes omnipotent, then She is superior (Pradh\=an\=a) in the work of creation. Hence She is styled as Prakriti.

O Child N\=arada! The state just preceding that of creation is denoted by ``Pra''; and ``Kri'' signifies creation. So the Great Dev\={\i} that exists before creation is called Prakriti after creation. The Param\=atm\=a by His Yoga (i.e., M\=ay\=a \'Sakti, the Holy Ghost) divided Himself into two parts; the right side of which was male and the left side was the female Prakriti.
(Note :-- The Holy Ghost is the principle of Conception and Emanation, Creation). So the Prakriti is of the nature of Brahm\=a. She is eternal. As the fire and its burning power are not different, so there is no separate distinction between \=Atman and His \'Sakti, between Puru\d{s}a and Prakriti. Therefore those that are foremost and the highest of the Yogis do not recognise any difference between a male and a female. All is Br\=ahman. He is everywhere as male and female forever. There is nothing in this world that can exist for a moment even without this Br\=ahman consisting of male and female. (i.e. they are Br\=ahman with M\=ay\=a manifested). Out of the Will of \'Sr\={\i} Kri\d{s}\d{n}a, to create the world Whose Will is all in all, came out at once the M\=ul\=a Prakriti, the Great Dev\={\i} \=I\'svar\={\i}, (the Lady Controller of the Universe) Brahm\=a with M\=ay\=a in a state of equilibrium). By Her command came out five Forms of Her, either for the purpose of creation or for bestowing Favour and Grace to the Bhaktas (devotees). Durg\=a the Mother of Ga\d{n}e\'sa, comes, as the first, the most auspicious, loved by \'Siva. She is N\=ar\=aya\d{n}\={\i}, Vi\d{s}\d{n}u M\=ay\=a, and of the nature of P\=ur\d{n}a Brahm\=a (the Supreme Brahm\=a). This eternal, all auspicious Dev\={\i} is the Presiding Deity of all the Devas and is, therefore, worshipped and praised by Brahm\=a and the other Devas, Munis, and Manus. This Bhagavat\={\i} Durg\=a Dev\={\i}, (when She gets pleased) destroys all the sorrows, pains and troubles of the Bhaktas that have taken Her refuge, and gives them Dharma, everlasting name and fame, all auspicious things and bliss and all the happiness, nay, the Final Liberation! She is the Greatest Refugeof these Bhaktas that come to Her wholly for protection and are in great distress, whom She saves from all their dangers and calamities. In fact, know this Durg\=a Dev\={\i} as, verily, the Presiding Deity of the heart of Kri\d{s}\d{n}a and as His Highest \'Sakti, of the nature of the Holy Fire and the Holy Light. She is Omnipotent and resides always with Kri\d{s}\d{n}a, the Great God. She is worshipped by all the Siddha Puru\d{s}as (those that have attained success); the (eighteen) Siddhis all go to Her and when pleased She gives whatever Siddhis (success) that Her Bhaktas want.

19-40. This Great Dev\={\i} is the intelligence, sleep, hunger, thirst, shadow, drowsiness, fatigue, kindness, memory, caste, forbearance, errors, peace, beauty, and consciousness, contentment, nourishment, prosperity, and fortitude. She is sung in the Vedas and in other \'S\=astras

as the Mah\=a M\=ay\=a, of the nature of the Universe. In reality, She is the All-\'Sakti of the Universe and She is the \'Sakti of Kri\d{s}\d{n}a. All these qualities are also mentioned in the Vedas. What is mentioned here is a tithe merely, in comparison to that of the Vedas. She has infinite qualities. Now hear of other \'Saktis. The second \'Sakti of the Param\=atman is named Padm\=a (Lak\d{s}m\={\i}). She is of the nature of \'Suddha Sattva (Higher than Sattva Gu\d{n}a) and is Kri\d{s}\d{n}a's Presiding Deity of all wealth and prosperity. This very beautiful Lak\d{s}mi Dev\={\i} is the complete master of the senses; She is of a very peaceful temper, of good mood and all-auspicious. She is free from greed, delusion, lust, anger, vanity and egoism. She is devoted to Her husband and to Her Bhaktas; Her words are very sweet and She is very dear to Her husband, indeed, the Life and Soul of Him. This Dev\={\i} is residing in all the grains and vegetables and so She is the Source of Life of all the beings. She is residing in Vaikuntha as Mah\=a Laksm\={\i}, chaste and always in the service of Her husband. She is the Heavenly Lak\d{s}m\={\i}, residing in the Heavens and the royal Lak\d{s}m\={\i} in palaces and the Griha Lak\d{s}m\={\i} in the several families of several householders. O N\=arada! All the lovely beauty that you see in all the living beings and all the things, it is She; She is the glory and fame of those that have done good and pious works and it is She that is the prowess of the powerful kings. She is the trade of merchants, the mercy of the saints, engaged in doing good to others and the seed of dissensions in those sinful and vicious persons as approved of in the Vedas. She is worshipped by all, reverenced by all. Now I will describe to you about the third \'Sakti of the Great God who is the Presiding Deity of knowledge, speech, intelligence, and learning. This third \'Sakti is named Sarasvat\={\i}. She is all the learning of this endless Universe and She resides as medh\=a (intelligence) in the hearts of all the human beings; She is the power in composing poetry; She is the memory and She is the great wit, light, splendour and inventive genius. She gives the power to understand the real meaning of the various difficult Siddh\=anta works; She explains and makes us understand the difficult passages and She is the remover of all doubts and difficulties. She acts when we write books, when we argue and judge, when we sing songs of music; She is the time or measure in music; She holds balance and union in vocal and instrumental music. She is the Goddess of speech; She is the Presiding Deity in the knowledge of various subjects; in argumentations and disputations. In fact all the beings earn their livelihood by taking recourse to Her. She is peaceful and holds in Her hands V\={\i}\d{n}\=a (lute) and books. Her nature is purely S\=attvic (\'Suddha Sattva), modest and very loving to \'Sr\={\i} Hari. Her colour is white like ice-clad mountains, like that of the white sandal, like that of the Kunda flower, like that of the Moon, or white lotus. She always repeats

the name of Param\=atm\=a \'Sr\={\i} Kri\d{s}\d{n}a while She turns Her bead composed of jewels. Her nature is ascetic; She is the bestower of the fruits of the ascetism of the ascetics; She is the Siddhi and Vidy\=a of all; She grants always success to all. Were She not here, the whole host of Br\=ahmi\d{n}s would always remain speechless like the dead cluster of persons. What is recited in the Vedas as the Third Dev\={\i} is the Holy Word, The Third \'Sakti, Sarasvat\={\i}. Thus I have described Her. Now hear the glories of the other Dev\={\i} in accordance with the Vedas. She is the mother of the four colours (castes), the origin of the (six) Ved\=amgas (the limbs of the Vedas and all the Chhandas, the Seed of all the mantrams of Sandhy\=a vandanam and the Root, the Seed of the Tantras; She Herself is versed in all the subjects. Herself an ascetic, She is the Tapas of the Br\=ahmi\d{n}s; She is the Tejas (Fire) and the caste of the Br\=ahmi\d{n} caste and embodies in Herself all sorts of Samsk\=aras (tendencies; inclinations); She is the Japam. Pure, known by the names of S\=avitr\={\i} and G\=ayatr\={\i}, She resides always in the Brahm\=a Loka (the Sphere of Brahm\=a) and is such as all the sacred places of pilgrimages want Her touch for their purification.

41-47. Her colour is perfectly white like the pure crystal. She is purely \'Suddha Sattva, of the nature of the Highest Bliss; She is eternal and superior to all. She is of the nature of Para Brahm\=a and is the bestower of Mok\d{s}a. She is the Fiery \'Sakti and the Presiding Deity of the Brahm\=a Teja (the fiery spirit of Brahm\=a, and the Br\=ahma\d{n}as). The whole world is purified by the touch of Whose Feet, this S\=avitr\={\i} Dev\={\i} is the Fourth \'Sakti. O Child N\=arada! Now I will describe to you about the Fifth \'Sakti, the Dev\={\i} R\=adhik\=a. Hear. She is the Presiding Deity of the five Pr\=a\d{n}as; She Herself is the Life of all; dearer than life even to \'Sr\={\i} Kri\d{s}\d{n}a; and She is highly more beautiful and superior to all the other Prakriti Dev\={\i}s. She dwells in everything; She is very proud of Her good fortune (Saubh\=agyam); Her glory is infinite; and She is the wife, the left body, as it were, of \'Sr\={\i} Kri\d{s}\d{n}a and She is not in any way inferior to Him, either in quality or in the Tejas (Fiery Spirit) or in any other thing. She is higher than the Highest; the Essence of all, infinitely superior, the First of all, Eternal, of the nature of the Highest Bliss, fortunate, highly respected, and worshipped by all. She is, the Presiding Dev\={\i} of the R\=asa L\={\i}l\=a of \'Sr\={\i} Kri\d{s}\d{n}a. From Her has sprung the R\=asa mandalam and She is the Grace and the Ornament of the R\=asa mandalam (the dance in a circle in R\=asa).

[Note :-- Extracts from a paper on Creation as explained by Hon'ble Justice Sir G. Woodroffe.
The lecturer commenced by pointing out that an examination of any doctrine of creation reveals two fundamental concepts: Those of Being

(Kutastha) and Becoming (Bhava); Changelessness and Change; the one and Many. The Brahm\=an or Spirit in its own nature (Svarupa) is and never becomes. It is the evolutes derived from the Principle of Becoming (M\=ul\=a Prakriti) which constitute what is called Nature. The latter principle is essentially Movement. The world is displayed by consciousness (chit) in association with M\=ul\=a Prakriti in cosmic vibration (spandana). Recent Western hypotheses have made scientific ``matter'' into M\=ay\=a in the sense that it is but the varied appearances produced in our mind by vibration of and in the single substance ether. The doctrine of vibration (Spandana) is however in India an ancient inheritance. The whole world is born from the varied forms of the initial movement in M\=ul\=a Prakriti. The problem is how does such multiplicity exist without derogation to the essential unit of its efficient cause, the spirit? The lecturer then made a rapid survey of the S\=ankhya philosophy on this point which assumes two real and independent principles of Being and Becoming which it calls Puru\d{s}a and Prakriti and passed from this the easiest dualistic answer to the pure monism of \'Sankara which asserted that there was but one Principle of Being, the Sadvastu and M\=ay\=a, whether considered as a \'Sakti of \=I\'svara or as the product of such \'Sakti was Avastu or nothing. He then pointed out that the T\=antrik doctrine with which he dealt occupied a middle position between those two points of view. \'Siva in the Kul\=arnava Tantra says ``Some desire Monism'' (Advaitav\=ada), others Dualism (Dvaitav\=ada). Such, however, know not My Truth which is neither Monism nor Dualism (Dvait\=advait\=a Vivarjita). Tantra is not Dvaitav\=ada for it does not recognise Prakriti as an independent unconscious principle (Achit). It differs from \'Sankara's Advaitav\=ada in holding that Prakriti as a conscious principle of Becoming, that is as \'Sakti, is not Avastu, though its displayed picture, the world is M\=ay\=a. It effects a synthesis of the \'S\=ankhya dualism by the conversion of the twin principles of Puru\d{s}a and Prakriti into the unity which is the Ardhan\=ar\={\i}\'svara \'Siva \'Sakti.

As regards other matters it adopts the notions of the S\=ankhya such as the concepts of M\=ul\=a Prakriti with the three Gu\d{n}as, vibration (spandana), evolution (Parin\=ama) of the Vikritis and the order of emanation of the Tattvas. \'Sakti which effects this exists and is Herself never unconscious (Achit) though It has the power to make the J\={\i}va think It is such. If this were understood one would not hear such nonsense as that the \'S\=aktas (whose religion is one of the oldest in the world) worship material force or gross matter (Jada).

The lecturer then shortly explained the nature of \'Sakti (\'Sakti Tattva), a term which derived from the root ``\'Sak'' meant the Divine Power whereby the world was created, manifested and destroyed. In Tantra the power and the Lord who wields it (\'Saktim\=an) are one and the same, \'Siva and \'Sakti are one and the same, \'Siva is Brahm\=an, \'Sakti is Brahm\=an. The first is the transcendent, the second the immanent aspect of the one Brahm\=an, Who is both \'Siva and \'Sakti. The Mother creates (K\=arya-Vibh\=avin\={\i}). The Father wills what She does (K\=arya-Vibh\=avaka). From their union creation comes. \'Sakti is not like the diminutive female figure which is seen on the lap of some Indian images, to which is assigned the subordinate position which some persons consider a Hindu wife should occupy. She is not a handmaid of the Lord but the Lord Himself in Her aspect as Mother of the worlds. This \'Sakti is both Nirgu\d{n}a and Sagu\d{n}a that is Chit \'Sakti and M\=ay\=a \'Sakti.

After this defining the nature of \'Sakti by which the world was created, the lecturer commenced an account of its manifestation as the universe, following in the main the \'S\=arad\=a Tilaka written in the eleventh century by Laksman\=ach\=arya, the guru of the celebrated Kashmiran T\=antrik, Abhinava Gupta. The following is a very abbreviated summary of this, the main portion of the paper. The lecturer first referred to the Aghan\=avasth\=a state which was that Ni\d{s}kala \'Siva and touching upon the question why \'Siva became Sakala (associated with Kal\=a) and creative explained the term Kal\=a and the theory of Adristasristi taught by the Tantra as by other \'Sastras. The former is according to S\=ankhya, M\=ul\=a Prakriti; according to Ved\=anta, Avidy\=a and according to the \'Siva Tantra, \'Sakti. The latter is the doctrine that the impulse to creation is proximately caused by the Karma of the J\={\i}vas. It is the seed of Karma which contains the germ of cosmic will to life. When Karma becomes ripe, there arises the state called \=Ik\d{s}a\d{n}a and other names indicative of creative desire and will. There then takes place a development which is peculiar to the Tantra called Sadri\'sa Pari\d{n}\=ama, which is a kind of Vivartta. The development is only apparent for there is no real change in the \=Anandamaya Ko\d{s}a. \'Sakti which exists in Sakala \'Siva in a purely potential state is said to issue from Him. This is the first kinetic aspect of \'Sakti in which Sattagu\d{n}a is displayed. This is the Param\=ak\=a\d{s}\=avasth\=a. N\=ada (sound, word) then appears. \'Sakti becomes further kinetic through the enlivening of the Rajo Gu\d{n}a. This is the Aksar\=avasth\=a. Then under the influence of Tamas, \=I\d{s}vara becomes Ghanibh\=uta and what is called the Par\=avindu. This is the Avyakt\=avasth\=a. Thus the Supreme Vindu men call by different names, Mah\=a Vi\d{s}\d{n}u, Brahm\=a Pur\=u\d{s}a, or Dev\={\i}. It is compared to a grain of gram which under its sheath contains two seeds in undivided

union. These are \'Siva \'Sakti and their encircling sheath is M\=ay\=a. This Vindu unfolds and displays itself, in the threefold aspect of Vindu, V\={\i}ja, N\=ada; or \'Siva, \'Sakti, and \'Siva \'Sakti; the three \'Saktis of will, knowledge and action. This is the mysterious K\=ama Kal\=a which is the root of all Mantras. These seven :-- Sakala, \'Siva, \'Sakti, N\=ada, Par\=avindu, Vindu, V\={\i}ja, N\=ada are all aspects of \'Sakti which are the seven divisions of the Mantra Om and constitute what is called the creation of Par\=a sound in the \=I\'svara creation.

The lecturer having explained the nature of these \'Saktis which formed part of the sound (\'Sabda), Sadri\d{s}a Parin\=ama, referred to the form or meaning (Artha) creation in the same development by the appearance of the six \'Sivas from \'Sambu to Brahm\=a which were aggregate (Samasti) sound powers. It was he said, on the differentiation of the Par\=avindu that there existed the completed causal \'Sabda which is the Hidden Word. The causal body or Par\=a \'Sabda and Artha being complete, there then appeared the displayed word or \'Sabd\=artha. This is a composite like the Greek Logos. The \'Sabda Brahm\=an or Brahm\=an as cause of \'Sabda is the Chaitanya in all beings. The \'Sabd\=artha in the Vedantin N\=amar\=upa or world of name and form of this \'Sabd\=artha the subtle and gross bodies are constituted, the \'Saktis of which are the Hira\d{n}yagarbha sound, called Madhyam\=a and the Vir\=at sound Vaikh\=ar\={\i}. By \'Sabda is not meant merely physical sound which as a quality of atomic ether is evolved from T\=amasik Ahamk\=ara.

The lecturer then pointed out that there had been Adrista Sristi up to the appearance of \'Sakti and Vivartta development up to the completion of the ``word'' or causal sound. Then there takes place real evolution (Pari\d{n}\=ama) in which the Tattvas (or elements discovered as a result of psychological analysis of our worldly experience) are said to emanate according to the S\=ankhya and not the Vedantic scheme, though there were some peculiarities in the Tantrik exposition which the lecturer noted. Finally Yogika Sristi was accepted in so far as it was the elements which in varied combinations made up the gross world.

In conclusion the lecturer pointed out that Indian \'S\=astra was a mutually connected whole. Such peculiarities as existed in any particular \'S\=astra were due to a variety of standpoints or purpose in view. The main point in this connection to be remembered was that the Tantra was practical \'S\=adhan\=a \'S\=astra. Whilst \'Sankara dealt with the subject from the standpoint of J\~n\=anak\=anda, the Tantra treated it from the point of view of worship (Up\=asan\=ak\=anda) the Tantrik doctrine is compounded of various elements some of which it shared with other \'S\=astras, some of which are its own, the whole being set forth according to a method and terminology which is peculiar to itself.]

48-70. She is the Lady of the R\=asa L\={\i}l\=a, the Foremost of the Jovial, humourous (witty) persons and dwells always in R\=asa. Her abode is in Goloka and from Her have come out all the Gop\={\i}k\=as. R\=asa -- the circular dance of Kri\d{s}\d{n}a and the cow-herdesses of Vrind\=avana. Her nature is the Highest Bliss, the Highest Contentment, and Excessive Joy; She transcends the three Sattva, Rajo and Tamo Gu\d{n}as and is Nir\=ak\=ara (without any particular form); but She dwells everywhere but unconnected with any. She is the soul of all. She is without any effort to do anything and void of Ahamk\=ara. She assumes forms only to show Her favour to Her Bhaktas. The intelligent learned men (Pundits) read Her Mahim\=a (glories) in meditating on Her according to the Vedas. The chief of the Devas and the Munis could never see Her; Her clothings are fire-proof and She is decorated with many ornaments all over Her body. Her body looks as if the crores of moons have risen all at once; She is the Giver of Bhakti (devotion) towards Kri\d{s}\d{n}a, services towards Kri\d{s}\d{n}a; and She bestows all wealth and prosperity. In Var\=aha Kalpa, i.e., when the Var\=aha incarnation took place, She incarnated Herself as the daughter of one Gopa (cow-herd), named Vrisabh\=anu. And Earth was blessed by the touch of Her feet. She is such as Brahm\=a and the other Devas could never perceive Her by any of their senses, yet everyone at Vrind\=avan saw Her very easily. She is the Gem amongst women. And when She is seen on the breast of Kri\d{s}\d{n}a, it seems that lightnings flash in the blue mass of clouds in the sky. In days gone by, Brahm\=a practised several austerities for sixty thousand years to purify Himself by seeing the nails of Her toe; but far from seeing that, He could not have that even in his dreams. At last He succeeded in seeing Her at Vrind\=avana and became blessed. O Child N\=arada! This is the fifth Prakriti and She is denominated as R\=adh\=a. Every female in every Universe is sprung from a part of \'Sr\={\i} R\=adh\=a or part of a part. O N\=arada! Thus I have described to you the five Highest Prakritis, Durg\=a and others. Now I am going to describe those that are parts of these Prakritis. Hear. The Ganges, Gang\=a has sprung from the lotus feet of Vi\d{s}\d{n}u; Her form is fluid-like; She is eternal. And She is the veritable burning fire to burn away the sins of the sinners. She is sweet to touch in taking baths and in drinking; She gives final liberation to the J\={\i}vas, and leads easily to the Goloka abode. She is the holiest amongst the places of pilgrimage and is the first of the running rivers. She is the rows of pearls in the clotted hairs of Mah\=adeva's head and She is the Tapasy\=a (asceticism) incarnate of the Tapasv\={\i}s (ascetics) of the Bh\=arata Var\d{s}a. This Ganges purifies the three worlds and is the part of M\=ul\=a Prakriti; She shines like the Full Moon, is white like white lotus and like milk; She is pure \'Suddha Sattva, clear, free from any Ahamk\=ara, chaste and

beloved of N\=ar\=ayana. The Tulas\={\i} Dev\={\i} is the consort of Vi\d{s}\d{n}u. She is the ornament of N\=ar\=aya\d{n}a, and dwells always at the lotus feet of N\=ar\=ayana. By Her are performed all the acts of worship, all austerities, and all Sankalpas (resolves). She is the chief of all the flowers, holy and able to give merits (Pu\d{n}yam) to others. At Her sight and touch, Nirv\=a\d{n}a can be obtained; and, were it not for Her, there could be no other fire in this Kali Yuga to burn the sins. She Herself is of the nature of Fire and at the touch of Whose lotus-feet, the earth is purified; all the T\={\i}rthas desire to have Her sight and touch for purification and without Her all acts in this world become fruitless. She bestows Mok\d{s}a (liberation) to those who want final liberation, grants all sorts of desires to several people, Who Herself is like a Kalpa Vrik\d{s}a, Who is the Presiding Deity of all the trees in Bh\=arata and Who has come here to grant satisfaction to the ladies of Bh\=arata Var\d{s}a and She is considered very superior throughout all parts of India. This Tulas\={\i} Dev\={\i} is the chief factor of M\=ul\=a Prakriti.

71-95. Then comes the Manas\=a Dev\={\i}, the daughter of Ka\'syapa. She is the dear disciple of \'Sankara and is therefore very learned in matters of \'S\=astras. She is the daughter of Ananta Deva, the Lord of Snakes and is very much respected by all the N\=agas. She Herself is very beautiful, the Lady of the N\=agas, the mother of the N\=agas and is carried by them. She is decorated with ornaments of the Snakes; She is respected by the N\=agendras and She sleeps on the bed of Snakes. She is Siddha Yogin\={\i}, the devotee of Vi\d{s}\d{n}u and always ready in the worship of Vi\d{s}\d{n}u; She is the Tapas and the bestower of the fruits of Tapas. Herself an ascetic, She spent three lakh years (according to the Deva measure) and has become the foremost of the ascetics in Bh\=aratvarsa. She is the Presiding Deity of all the mantras; Her whole body shines with Brahm\=ateja (the Holy Fire of Brahm\=a). Herself of the nature of Brahm\=a, She again meditates on Brahm\=an. She is sprung from a part of \'Sr\={\i} Kri\d{s}\d{n}a and the chaste wife of Jarat K\=aru Muni, the mother of \=Astika, the great Muni; She is the part of M\=ul\=a Prakriti. O Child N\=arada! Now comes the \'Sasth\={\i} Dev\={\i}, the Mother of Devasen\=a. She is the most superior amongst the Gaur\={\i} and the sixteen M\=atrik\=as. This chaste woman is the giver of sons and grandsons in the three worlds and the nurse, the foster mother of all. She is the sixth part of M\=ul\=a Prakriti and is hence known by the name of \'Sasth\={\i}. She lives near to every child as an aged Yogin\={\i}. Her worship is everywhere prevalent in the twelve months Vai\'s\=akha, etc. When the child gets born, on the sixth day of Her worship is done in the lying-in-chamber and again on the 21st day (after twenty days have passed away) the most auspicious worshipful ceremony of Her is performed. The Munis bow down to Her with reverence and want to visit Her daily.

She protects all children always with a mother's affectionate heart. This \'Sasth\={\i} Dev\={\i} is again the part of M\=ul\=a Prakriti. Then appears the Dev\={\i} Mangala Chandik\=a. She goes from one house to another, on land or through water or in air, doing great good to them; She has come out of the face of the Prakriti Dev\={\i} and is doing always all sorts of good to this world. Her name is Mangala Chand\={\i} because She is all auspicious at the time of creation and assumes very furious angry appearance at the time of destruction. So the Pundits say. On every Tuesday in all the worlds Her worship is done; and She, when pleased, gives to women sons, grandsons, wealth, prosperity, fame and good of all sorts and grants all desires. This Mangala Chandi is again the part of M\=ul\=a Prakriti. Now comes the lotus-eyed M\=ahe\'svar\={\i} K\=al\={\i} who when angry can destroy all this universe in a moment, who sprang from the forehead of the M\=ul\=a Prakriti, D\=urg\=a to slay the two demons \'Sumbha and Ni\'sumbha. She is the half-portion of D\=urg\=a and qualified like Her, fiery and energetic. The beauty and splendour of whose body make one think as if the millions of suns have arisen simultaneously. Who is the foremost of all the \'Saktis and is more powerful than any of them, Who grants success to all the persons, Who is superior to all and is of Yogic nature, Who is exceedingly devoted to Kri\d{s}\d{n}a and like Him fiery, well-qualified, and valorous, Whose body has become black by the constant meditation of \'Sr\={\i} Kri\d{s}\d{n}a, Who can destroy in one breath this whole Brahm\=anda, Who was engaged in fighting with the Daityas simply for sport and instruction to the people and Who, when pleased in worship can grant the four fruits Dharma, Artha, K\=ama and Mok\d{s}a. This K\=al\={\i} is also the part of Prakriti. The Dev\={\i} Basundhar\=a (Earth) is again the part of M\=ul\=a Prakriti. Brahm\=a and the other Devas, all the Muni mandalams (the spheres of Munis), fourteen Manus and all men sing hymns to Her. She is the support of all and filled with all sorts of grains. She is the source of all gems and jewels, She bears in Her womb all the precious metals. All sorts of best things issue from Her. She is the Refuge of all. The subjects and kings worship Her always and chant hymns to Her. All the J\={\i}vas live through Her and She bestows all sorts of wealth and prosperity. Without Her, all this, moving or non-moving, become void of any substratum. Where to rest on!

96-143. O Child N\=arada! Now hear about them who are issued again from the parts of M\=ul\=a Prakriti as well as the names of their wives. I will now narrate duly. The Dev\={\i} ``Sv\=ah\=a'' is the wife of Agni (Fire), and the whole Universe worships Her. Without her, the Dev\={\i} can never take any oblations. Dak\d{s}i\d{n}\=a and Diks\=a are both the wives of Yaj\~na (Sacrifice). They are honoured everywhere. So much so that without Dak\d{s}i\d{n}\=a (the fees given at the end of the Sacrifice) no sacrificial ceremonies

can be complete and fructifying. The Dev\={\i} ``Svadh\=a'' is the wife of the Pitris. All worship this Dev\={\i} ``Svadh\=a'' whether they are Munis, Manus, or men. If this mantra ``Svadh\=a'' be not uttered while making an offering to the Pitris, all turn out useless. The Dev\={\i} ``Svasti'' is the wife of the V\=ayu Deva; She is honoured everywhere in the Universe. Without this ``Svasti'' Dev\={\i}, no giving nor taking nor any action can be fructifying and useful. ``Pust\={\i}'' (nourishment) is the wife of Ga\d{n}apat\={\i}. All in this world worship this Pust\={\i} Dev\={\i}. Without this ``Pusti'', women or men alike all become weaker and weaker. Tust\={\i} (satisfaction, contentment) is the wife of Ananta Deva. She is praised and worshipped everywhere in this world. Without Her no one anywhere in the world can be happy. ``Sampatt\={\i}'' is the wife of \=Is\=ana Deva. The Suras, the men all alike worship Her. Were it not for Her, all in this world would be oppressed with dire poverty. The Dev\={\i} ``Dhrit\={\i}'' is the wife of Kapila Deva. She is honoured equally in all places. Were it not for Her, all the people in this world would have become impatient. The ``Sat\={\i}'' Dev\={\i} is the wife of Satya Deva (Truth). She is endearing to the whole world. The liberated ones worship Her always. Were it not for the truth loving Sat\={\i}, the whole world would have lost the treasure in friendship. ``Day\=a'' (Mercy) endearing to the whole world is the chaste wife of ``Moh\=a Deva''. She is liked by all. Were it not for Her, all the world would have become hopeless. The Dev\={\i} ``Pratisth\=a'' (fame, celebrity) is the wife of Pu\d{n}ya Deva (merit).She gives merits to persons according as they worship Her. Were it not for Her, all the persons would remain dead while living. The Dev\={\i} ``K\={\i}rti'' (fame) is the wife of Sukarma (good works). Herself a Siddha (one who has acquired the result of one's success), all the blessed people honour Her with great reverence. Were it not for Her, all the persons in this world would have been dead, devoid of any fame. Kriy\=a (work-efforts, action, doing) is the wife of ``Udyoga'' (enthusiasm). All honour Her greatly. O Muni N\=ar\=ada! Were it not for Her, the whole people would be void of any rules and regulations. Falsehood is the wife of Adharma (unrighteousness). She is honoured greatly by all the cheats that are extant in this world. Were she not liked by them, then all the cheats would become extinct. She did not fall in the sight of anybody in the Satya Yuga. Her subtle form became visible in the Tret\=a Yuga. When the Dv\=apara Yuga came, She became half developed. And at last when the Kali Yuga has come, She is fully developed and there is no second to Her whether in bold confidence and shamelessness or in talking much and pervading everywhere. With her brother Deceitfulness, She roams from one house to another. Peace and modesty and shame are both the wives of good behaviour. Were they not existent, all in this

world would have turned out deluded and mad. Intelligence, genius and fortitude, these three are the wives of J\~n\=ana (knowledge). Had they not lived, every one would become stupid and insane. M\=urti is the wife of Dharma Dev\={\i}. She is of the nature of Beauty to all and very charming. Were it not for Her, Param\=atm\=an would not get any resting place; and the whole universe would have become Nir\=alamba (without anything to rest). This chaste M\=urti Dev\={\i} is of the nature of splendour, loveliness and Lak\d{s}m\={\i}. She is everywhere respected, worshipped and reverenced. ``Sleep'', the Siddha Yogin\={\i}, is the wife of Rudra Deva, who is of the nature of K\=al\=agni (the universal conflagration at the break-up of the world). All the J\={\i}vas spend their nights with Her. The twilights, night and day are the wives of K\=ala (Time). If they were not, the Creator even would not be able to reckon time. Hunger and thirst are the wives of Lobha (covetousness). They are thanked, respected and worshipped by the whole world. Had they not lived, the whole world would have merged ever in an ocean of anxieties. Splendour and burning capacity are the wives of Tejas (fire). Without these, the Lord of the world could never have created and established order in this universe. Death and old age are the daughters of the K\=ala, and the dear wives of Jvar\=a (the disease). Without these, all the creation would come to an end. The Tandr\=a (drowsiness, lassitude) and Pr\={\i}ti (satisfaction) are the daughters of Nidr\=a (sleep). And they are the dear wives of Sukha (pleasure). They are present everywhere in this world. O Best of Munis! \'Sraddh\=a (faith) and Bhakti (devotion) are the wives of Vair\=agyam (dispassion). For then all the persons can become liberated while living (J\={\i}vanmuktas). Besides these there is Aditi, the Mother of the Gods, Surabhi, mother of cows; Diti, the mother of the Daityas; Kadru, the mother of the N\=agas (serpents); Vinat\=a, the mother of Garuda, the prince of birds; and Danu, the mother of the D\=anavas. All are very useful for the purpose of creation. But these all are parts of M\=ul\=a Prakriti. Now I will mention some of the other parts of Prakriti. Hear. Rohi\d{n}\={\i}, the wife of the Moon, Sanj\~n\=a, the wife of the Sun; \'Satar\=up\=a, the wife of Manu; \'Sach\={\i}, the wife of Indra; T\=ar\=a, the wife of Brihaspati; Arundhat\={\i}, the wife of Va\'sistha; Anas\=uy\=a, the wife of Atri; Devah\=ut\={\i}, the wife of Kardama; Pras\=uti, the wife of Dak\d{s}a; Menak\=a, the mind born daughter of the Pitris and the mother of Ambik\=a, Lop\=amudr\=a, Kunt\={\i}, the wife of Kuvera, the wife of Varu\d{n}a, Bindhy\=aval\={\i}, the wife of the King Bali; Damayant\={\i}, Ya\'sod\=a, Devak\={\i}, G\=andh\=ar\={\i}, Draupad\={\i}, \'Saivy\=a, Satyavat\={\i}, the chaste and noble wife of Bri\d{s}abh\=anu and the mother of R\=adh\=a; Mandidar\={\i}; Kau\'saly\=a, Kaurav\={\i},; Subhadr\=a; Revat\={\i}, Satyabh\=am\=a, K\=alind\={\i}, Laksman\=a; J\=ambavat\={\i}; N\=agnajiti, Mitrabind\=a,

Lak\d{s}a\d{n}\=a, Rukmi\d{n}\={\i}, S\={\i}t\=a, the Laksm\={\i} incarnate; K\=al\={\i}, Yojana Gandh\=a, the chaste mother of Vy\=asa, \=U\d{s}\=a, the daughter of V\=a\d{n}a, her companion Chitralekh\=a; Prabh\=avati, Bh\=anumat\={\i}, the Sat\={\i} M\=ay\=avat\={\i}, Re\d{n}uk\=a, the mother of Para\'sur\=ama; Rohi\d{n}\={\i}, the mother of Balar\=ama, Ekanand\=a and the sister of \'Sr\={\i} Kri\d{s}\d{n}a, Sat\={\i} Durg\=a and many other ladies are the parts of Prakriti and all the female sex, everywhere in the Universe are all come from the parts of Prakriti. So to insult any woman is to insult the Prakriti. If one worships a chaste Brahmi\d{n} woman, who has her husband and son living, with clothings, ornaments, and sandal paste, etc., one worships, as it were, Prakriti. If any Vipra worships a virgin girl, eight years old, with clothings, ornaments and sandal paste, know that he has worshipped the Prakriti Dev\={\i}. The best, middling, and worst are all sprung from Prakriti. Those women that are sprung from Sattva Gu\d{n}a are all very good natured and chaste; those that are sprung from Rajo Gu\d{n}a are middling and very much attached to worldly enjoyments and do their selfish ends and those that are sprung from Tamo Gu\d{n}a are recognised as worst and belonging to the unknown families. They are very scurrilous, cheats, ruining their families, fond of their own free ways, quarrelsome and no seconds are found equal to them. Such women become prostitutes in this world and Apsar\=as in the Heavens. The Hermaphrodites are parts of Prakriti but they are of the nature of Tamo Gu\d{n}as.

144-159. Thus I have described to you the nature of Prakriti. So in this Punyabh\=umi Bh\=arata Var\d{s}a, to worship the Dev\={\i} is by all means desirable. In days past by, the King Suratha worshipped the M\=ul\=a Prakriti Durg\=a, the Destructrix of all evils. Then again \'Sr\={\i} R\=ama Chandra worshipped Her when he wanted to kill R\=ava\d{n}a. Since then Her worship is extant in the three worlds. She was first born as the honourable daughter of Dak\d{s}a. She destroyed the whole hosts of Daityas and D\=anavas. It was She who, hearing the abusive words uttered against Her husband at the Yaj\~na by Dak\d{s}a, Her father, gave up Her body and took up again Her birth. She took Her birth in the womb of Menak\=a and got again Pa\'supati as Her husband. And of the two sons, K\=artika and Ga\d{n}e\'sa, born to Her, K\=artika was the A\d{n}sa (part) of N\=ar\=ayana and Ga\d{n}apati was \'Sr\={\i} Kri\d{s}\d{n}a Himself, the Lord of R\=adh\=a. O Devar\d{s}i! After the two sons, Laksm\={\i} Dev\={\i} came out of Durg\=a. Mangala R\=aja, the King Mars first worshipped Her. Since then, all in the three worlds began to worship Her, whether they are Devas or men. The King A\'svapati first worshipped S\=avitr\={\i} Dev\={\i}; and since then the Devas, Munis, all began to worship Her. When the Dev\={\i} Saravast\={\i} was born, the Bhagav\=an Brahm\=a first worshipped Her; next the greatest Munis, Devas all began

to worship Her. On the full moon night of the month of K\=artik, it was Bhagav\=an \'Sr\={\i} Kri\d{s}\d{n}a, The Highest Spirit, that worshipped, first of all, the Dev\={\i} R\=adh\=a within the R\=asa Mandalam, the enclosure, within which the R\=asa L\={\i}l\=a was performed (the circular dance) in the region Goloka. Then under the command of \'Sr\={\i} Krisna, all the Gopas (cow-herds), Gop\={\i}s, all the boys, girls, Surabh\={\i}, the queen of the race of the cows, and the other cows worshipped Her. So since Her worship by the inhabitants of Goloka, by Brahm\=a and the other Devas and the Munis, all began to worship ever \'Sr\={\i} R\=adh\=a with devotion and incense, light and various other offerings. On earth She was first worshipped by Suyaj\~na, in the sacred field of Bh\=aratvar\d{s}a, under the direction of Bhagav\=an Mah\=adeva. Subsequently, under the command of the Bhagav\=an \'Sr\={\i} Kri\d{s}\d{n}a, the Highest Spirit, the inhabitants of the three worlds began to worship Her. The Munis with great devotion, with incense, flowers and various other offerings worship always the Dev\={\i} R\=adh\=a. O Child N\=arada! Besides these, all the other Dev\={\i}s that have issued from Prakriti Dev\={\i} are all worshipped. So much so that in the villages, the village Deities, in the forests, the forest Deities and in the cities, the city Deities are worshipped. Thus I have described to you all according to the \'S\=astras the glorious lives of the Dev\={\i} Prakriti and Her parts. What more do you want to hear?

Here ends the First Chapter on the Description of Prakriti in the Ninth Book of the \'Sr\={\i} Mad Dev\={\i} Bh\=agavatam of 18,000 verses by Mahar\d{s}i Veda Vy\=asa.



