\chapter{On the greatness of Bhasma}

1-20. N\=ar\=aya\d{n}a said :-- O Best of Munis! What shall I describe to you the effects of using the Bhasma! Only applying the ashes takes away the Mah\=ap\=atakas (great sins) as well as other minor sins of the devotee. I speak this truly, very truly unto you. Now hear the fruits of using simply the ashes. By using Bhasma, the knowledge of Brahm\=a comes to the Yatis; the desires of enjoyments are eradicated; the improvement

is felt in all the virtuous actions of the householders and the studies of the Vedas and other \'S\=astras of the Brahmach\=aris get their increase. The \'S\=udras get merits in using Bhasmas and the sins of others are destroyed. To besmear the body with ashes and to apply the curved Tripundras is the source of good to all beings. The \'Sruti says so. That this implies the performance of sacrifices by all, is also asserted in the \'Srutis. To apply ashes to the whole of the body and to use Tripundra is common to all the religions; it has nothing, in principle, contradictory to others. So the \'Sruti says. This Tripundra and the besmearing with ashes is the special mark of the devotees of \'Siva; this again is asserted in the \'Sruti. This Bhasma and the Tripundra are the special marks by which one is characterised; it is said so in the Vaidik \'Sruti. \'Siva, Vi\d{s}\d{n}u, Brahm\=a, Indra, Hira\d{n}yagarbha, and their Avataras, Varu\d{n}a and the whole host of the Devas all gladly used this Tripundra and ashes. Durg\=a, Lak\d{s}m\={\i}, and Sarasvat\={\i}, etc., all the wives of the gods daily anoint their bodies with ashes and use the Tripundras. So even the Yak\d{s}as, R\=ak\d{s}asas, Gandharbhas, Sidhas, Vidy\=adharas, and the Munis have applied Bhasma and Tripundra. This holding on of ashes is not prohibited to anybody; the Br\=ahma\d{n}as, K\d{s}attriyas, Vai\'syas, \'S\=udras, mixed castes, and the vile classes all can use this Bhasma and Tripundra. O N\=arada! In my opinion they only are the Sadhus (saints) who use this Tripundra and besmear their bodies with ashes. In seducing this Lady Mukti (liberation is personified here as a lady) one is to have this gem of \'Siva Lingam, the five lettered Mantra Namah \'Siv\=ay\=a as the loving principle, and holding on the ashes as the charming medicine (as in seducing any ordinary woman, gems, jewels and ornaments, love and charming medicines are necessary). O N\=arada! Know the place where the person, who has besmeared the holy with ashes and who has used Tripundra takes his food as where \'Sankara and \'Sankar\={\i} have taken their food together. Even if anybody himself not using the Bhasma, follows another who has used the Bhasma, he will be soon honoured in the society even if he be a sinner. What more than this, if anybody himself not using the ashes, praises another who uses the Bhasma, he is freed from all his sins and gets soon honour and respect in the society. All the studies of the Vedas come to him though he has not studied the Vedas, all the fruits of hearing the \'Srutis and the Pur\=a\d{n}as come to him, though he has not heard them, all the fruits of practised Dharma come to him though he has not practised any, if he always uses this Tripundra on his forehead and gives food to a beggar who uses Tripundra on his forehead. Even in courtries as Bihar (K\={\i}kata, etc., that have got a bad name) if there be a single man in the whole country whose body is besmeared with ashes and who uses this Tripundra, that is considered then as K\=a\'s\={\i} (Benares

city). Anybody, of a bad or of a good character, be he a Yogi or a sinner, using Bhasma, is worshipped like my son, Brahm\=a. O N\=arada! Even if an hypocrite uses Bhasma, he will have a good future, which cannot be attained even by performing hundreds of sacrifices. If anybody uses Bhasma daily either through good companion or through neglect, he will be entitled, like me, to the highest worship. O N\=arada! Brahm\=a, Vi\d{s}\d{n}u, Mahe\'svara, P\=arvat\={\i}, Lak\d{s}m\={\i}, Sarasvat\={\i} and all the other Devas become satisfied with simply holding on this Bhasma. The merits that are obtained by using only the Tripundra, cannot be obtained by gifts, sacrifices, severe austerities, and going to sacred places of pilgrimages. They cannot give one-sixteenth part of the result that accrues from holding the Tripundra. As a King recognises a person as his own, whom he has given some object of recognition, so Bhagav\=an \'Sankara knows the man who uses Tripundras as His own person. They that hold Tripundras with devotion can have Bhol\=a N\=atha under their control; no distinction is made here between the Br\=ahma\d{n}as and Ch\=and\=alas. Even if anybody be fallen from the state of observing all the \=Ach\=aras or rules of conduct proper to his \=A\'srama and if he be faulty in not attending to all his duties, he will be Mukta (freed) if he has used even once this Bhasma Tripundra. Never bother yourself with the caste or the family of the holder of the Tripundras. Only see whether the sign Tripundra exists in his forehead. If so, consider him entitled to respect. O N\=arada! There is no mantra higher than this \'Siva Mantra; there is no Deity higher than \'Siva; there is no worship of greater merit-giving powers than the worship of \'Siva; so there is no T\={\i}rtha superior to this Bhasma. This Bhasma is not an ordinary thing; it is the excellent energy (semen virile) of fire of the nature of Rudra. All sorts of troubles vanish, all sorts of sins are destroyed by this Bhasma. The country where the lowest castes reside with their bodies besmeared with ashes, is inhabited always by Bhagav\=an \'Sankara, Bhagavat\={\i} Um\=a, the Pramathas (the attendants of \'Siva) and by all the T\={\i}rthas. Bhagav\=an \'Sankara, first of all, held this Bhasma as an ornament to his body by purifying it first with ``Sadyo J\=ata,'' etc., the five mantras. Therefore if anybody uses the Bhasma Tripundra according to rules on his forehead, the writings written at the time of his birth by Vidh\=at\=a Brahm\=a will all be cancelled, if they had been bad. There is no doubt in this.

Here ends the Thirteenth Chapter of the Eleventh Book on the greatness of Bhasma in the Mah\=apur\=a\d{n}am \'Sr\={\i} Mad Devi Bh\=agavatam of 18,000 verses by Mahar\d{s}i Veda Vy\=asa.



