\chapter{On the description of the sorrows of Hari\'schandra}

1-5. Vy\=asa said :-- O King! At this moment, the Muni Vi\'sv\=amitra, endowed with his power of tapas, came up there, very angry as if the God of Death, to ask of his wealth. Seeing him Hari\'schandra fallen thus senseless on the ground, Vi\'sv\=amitra, then, began to sprinkle water on his body. O King! The man who is involved in a debt his troubles increase day by day. So get up and pay your promised Dak\d{s}i\d{n}\=a. The King, thus sprinkled with water, cold as snow, regained his consciousness; but, seeing Vi\'sv\=amitra, he fainted again. At this, the Dv\={\i}ja Vi\'sv\=amitra consoled him and angrily spoke to him thus :--

6-10. O King! If you want to maintain your steadiness, give, then my Dak\d{s}i\d{n}\=a. Look! It is Truth that makes the Sun shine; It is the Truth that has stationed this Earth in its position; what to speak more, even the Svarga is established on Truth; so the greatest Dharma lies in Truth. If the fruit of the thousand A\'svamedhas be held in one pan and Truth be held on the other pan of the balance, then Truth outweighs the thousand horse sacrifices or what need I to speak all about this! O King! If you fail to give my Dak\d{s}i\d{n}\=a before the Sunset, I will, no doubt, curse you. Saying this, Vi\'sv\=amitra went away. The King also became very terrified. The wealthless King was pained by the words of the Muni; but he was more troubled with the thought how he would pay him and keep to Truth.

11-13. S\=uta said :-- O \d{R}i\d{s}is! At this time, a Br\=ahmi\d{n}, skilled in the Vedas, with many other Br\=ahmi\d{n}s, started out of his house, at that very place. The queen, then seeing the Br\=ahmi\d{n} ascetic close by, addressed the King in words reasonable and in accordance with the Dharma, O Lord! A Br\=ahmi\d{n} is considered the father of the other three Var\d{n}as (i.e., K\d{s}attriyas, Vai\'syas, and \'S\=udras) and a son can certainly take the father's things; so it is my intention that you beg your wealth from this Br\=ahmi\d{n}.

14-18. The King said :-- ``O One of thin waist! To beg suits the Br\=ahma\d{n}as; it is prohibited to the K\d{s}attriyas; I being a K\d{s}attriya do not wish to take anything as gift. The Br\=ahmi\d{n}s are the Gurus of all the Var\d{n}as. So they are always to be respected. It is not proper to beg from a Br\=ahmi\d{n}; especially the K\d{s}attriyas never ask anything from

the Br\=ahmi\d{n}s; it is totally prohibited. Offering oblations, study, gift and the governing of subjects and protecting those that take refuge is the Dharma of the K\d{s}attriyas but they would never, never, ask any other man `Give, give,' and utter these words indicative of humility O Dev\={\i}! The words `I am giving you' are impressed within my heart; so I will earn money from some other source and give that to the Muni.''

19-20. The Queen said :-- ``O King! Time keeps some men in one and the same state; again it throws others into troubles; Time it is that gives respect to one and again it is Time that gives disrespect to others. Time it is that makes one a donor and it is the same Time that makes another a beggar. So even the \d{R}i\d{s}i Vi\'sv\=amitra, learned and endowed with the strength of Tapas, becoming angry has deprived you of your kingdom and happiness and has thus done quite an irreligious act in the shape of tormenting others. You can now judge in this the wonderful workings of Time.''

21-22. The King said :-- ``I would rather out off my tongue into two pieces by a sharp sword than I would quit my K\d{s}attriya pride; and I would never be able to utter the words `Give, give.' O Fortunate One! I am a K\d{s}attriya; so I never ask anything of anyone. I always say that, by the strength of my arms, I will earn money and pay off my debt.''

23-27. The Queen said :-- ``O King! Indra and the other Devas have given me over duly to your hands. So I am your religious (legal) wife; especially I have got education and I ought to be protected. Therefore O Luminous One! If you do not like to beg then you can sell me and pay off your Dak\d{s}i\d{n}\=a.'' The King Hari\'schandra became grieved very much to hear these words and lamented, saying, ``O What a painful thing is this! What a painful thing is this!'' His wife again spoke :-- ``O King! Will we, afterwards, be burnt by the fire of curse from a Br\=ahmi\d{n} and thus lowered very much? So keep my word now. You are selling me, not because that you are infatuated with desire for gambling nor you are deprived of all knowledge by enjoyments in worldly things nor you are selling me owing to avert the danger of your kingdom. It is that you are selling me to pay off the debt to your Guru. So nothing sinful a fault will be incurred by you. So sell me and keep to Truth and the fruits thereof.''

Here ends the Twenty-First Chapter of the Seventh Book on the description of the sorrows of Hari\'schandra in the Mah\=a Pur\=a\d{n}am, \'Sr\={\i} Mad Dev\={\i} Bh\=agavatam of 18,000 verses by Mahar\d{s}i Veda Vy\=asa.



