\chapter{On the cause of the war between \=Adi and Baka}

1-2. The King said :-- ``O Best of Munis! Tell me the names of the holy places of pilgrimage on the surface of this earth, the holy K\d{s}etras and the holy rivers; what are the fruits acquired in bathing there and in making charitable gifts as well; also what are the rules of the journeys and acts there are to be conducted?''

3-34. Vy\=asa said :-- Hear; I am describing to you various T\={\i}rthas or places of pilgrimage as well as those that are highly extolled as the best places favourite to the Dev\={\i}. Amongst the rivers the following are reckoned as chief and holy :-- The Ganges, Jumn\=a, Sarasvat\={\i}, Narmadd\=a, Gandak\={\i}, Sindhu, Gomat\={\i}, Tamas\=a, Cavery, Chandrah\=ag\=a, Vetravat\={\i}, Charmanvat\={\i}, Saraju, T\=ap\={\i}, and S\=avramat\={\i}. Besides these, there are hundreds of rivers on the surface of this earth; of them, those that fall into the ocean, they are holier and those that have not reached the ocean are less holy. Of those rivers that fall into the ocean, those that always flow with great current, they are comparatively holier; but in the two months \'Sr\=avan and Bh\=adra (15th July - 15th September) all the rivers are considered as if they are during their menstruation periods; at this time also some rivers carry water of the rains just sufficient to supply the villagers with water. O King! The following are the famous places of pilgrimages calculated to bestow merits:-- Pu\d{s}kara, Kuruk\d{s}ettra, the holy Dharm\=ara\d{n}ya, Prav\=asa, Pray\=aga, Naimi\d{s}\=ara\d{n}ya, and Arbud\=ara\d{n}ya. O King! Of the mountains, the following are considered as sacred :-- \'Sr\={\i}\'saila, Sumeru, Gandham\=adana; of the lakes, the following are very holy and very famous :-- M\=anasarovara, Vi\d{n}dusarovara and Ak\d{s}oda; these are the chief lakes. To those Munis that meditate on their \=Atman, all the hermitages are sacred; still the hermitage of Badri is always considered very sacred and the most celebrated; here Nara and N\=ar\=aya\d{n}a, the two famous Munis, practised their asceticism. The V\=aman\=a\'srama and \'Satay\=up\=a\'srama are also well known; thus every hermitage is named after the Muni that practised asceticism there. Thus innumerable holy places on the surface of the earth are mentioned by the Munis as tending

to sanctify the hearts of the persons. At all these holy places, the Dev\={\i} is worshipped in special sites consecrated to Her. All the sins are destroyed by their mere sight. The devotees of the Dev\={\i} stay there, with rules obeyed. I will mention afterwards some of these places in the course of my narrations. O Best of kings! Going to these holy places, charity, vow, sacrifices, asceticism and good acts all depend on one another. The holy places of pilgrimages, asceticism, and observance of vows depend on the purity of the articles (Dravya \'Suddhi), on the purity and one pointedness of actions (Kriy\=a \'Suddhi) and on the purity of the mind and heart (Chitta \'Suddhi). Some may attain, at times, the Dravya \'Suddhi and Kriy\=a \'Suddhi; but every one finds it very difficult and, indeed, rarely get the Chitta \'Suddhi. O King! This mind always tries to seek shelter with various objects and is, therefore, always restless. How, then, can the purity of mind be effected, with ease, when it is occupied with all sorts of thoughts on various objects. Cupidity, anger, greed, pride, and egoism, these bring about all sorts of obstacles in the holy places of pilgrimages, in practising tapasy\=a and in observing vows. O King! Non-injury, truthfulness, non-stealing, chastity and purity, controlling of senses, and observing one's own religion, all these bring about the fruits of the labour in visiting all t\={\i}rthas. They bestow fruits that can be obtained by visiting all the t\={\i}rthas. During one's pilgrimage, one forsakes one's Nitya Karma (daily duties) and one has to come in contact with various persons. Hence one's journey becomes fruitless; rather it becomes a source of sin. The waters of the sacred places can only wash the outside dirts and the impurities of the physical bodies; they can never wash the impurities of their inner minds. Were it the fact that the waters of the t\={\i}rthas could purify their minds, why was it, then, that the Munis, residing on the banks of the Ganges, and devoted to God, ever indulged themselves with feelings of jealousy and enmity against each other. The humble Munis like Va\'sistha, and the \d{R}i\d{s}is like Vi\'sv\=amitra were always entangled in love and hatred and they were ever impatient with anger. Therefore it is evident that the internal purification, the purification of heart, the bathing in the G\~n\=an Gang\=a flowing within, no doubt removes more the dirt than the Ganges and other places of pilgrimages. O King! No doubt this fact must be admitted on all hands that one's impurity of mind is washed away if by the strange combination of Fate, one comes in intimate contact with a man possessed of the Divine Knowledge. O King! The Vedas or \'S\=astras, vows or austerities, sacrifices or gifts none can purify the heart. See! Va\'sistha, the son of Brahm\=a, though versed in the Vedas and residing on the banks of the Ganges, was under the control of love, hatred and other infirmities. Out of the enmity of Vi\'sv\=amitra and Va\'sistha, arose the great battle

named \=Adi Baka, astonishing even to the Gods. In this, the ascetic Vi\'sv\=amitra was cursed by Va\'sistha, on account of some curse in connection with the king Hari\'schandra and had to take his birth as a crane (Baka). The \d{R}i\d{s}i Va\'sistha was cursed also by Vi\'sv\=amitra and was born as a bird named \'Sar\=ari. Thus the two powerful \d{R}i\d{s}is were born as \=Adi Baka and lived on the banks of the M\=anasarovara and they fought for full ten thousand years (ajuta) terribly, out of anger, with their nails and beaks like two maddened lions.

35-36. The King asked :-- ``O Muni! Why were the two Maharsis, the two great ascetics and devoted to religion, involved in enmity with each other? Both of them were intelligent; how was it that they knowing the act of cursing to be a source of pain to men, cursed each other so painfully?''

37-48. Vy\=asa said :-- O King! In former times, there was born in the solar dynasty a king Hari\'schandra, the son of Tri\'sanku; he was the best of the kings and reigned before R\=amchandra. That King had no issue and therefore promised to Varu\d{n}a, ``O Lord of water and ocean! If I get a son born to me, I will perform a sacrifice, called Naramedha, where I will sacrifice my son for your propitiation.'' Varu\d{n}a was very pleased with the king when he made such a vow; and the exquisitely beautiful queen held the foetus in the womb. Seeing his wife in the family way, the king was very pleased and performed all the purificatory ceremonies pertaining to the foetus in the womb. O King! When the queen was delivered of a son endowed with all auspicious signs, the king Hari\'schandra was very glad and performed duly all the J\=ata Karma (natal) ceremonies and distributed as charity big sums of gold and many cows giving good quantities of milk. When the festivities on the birth of the child were celebrated in the palace on a grand scale, Varu\d{n}a, the Lord of Waters, assuming the form of a Br\=ahmi\d{n}, came up there. The King, too, honoured him duly with seat and worshipped him regularly and asked him about his purpose, when Varu\d{n}a spoke to him :-- ``O King! I am Varu\d{n}a, the Lord of Waters; you promised before that you would perform Naramedha sacrifice where you would sacrifice your son; now do those things and keep your words true.'' The King became very much confused and was very much pained at heart. He then checked his mental feelings of pain and spoke to the Deva Varu\d{n}a, with folded hands :-- ``O Lord! I will do the sacrifice duly and fulfil the promise that I made before you and keep my word. But, O Best of the Devas! My legal wife will be pure from her S\=utik\=a-\'Sauchak after one month, when I will perform the Naramedha sacrifice.''

49-53. Vy\=asa said :-- O King! Hearing thus the words of the king Hari\'schandra, Varu\d{n}a returned to his own abode; the King also became glad, but he was somewhat anxious for fear of the destruction of the child. When one month was complete, the sweet-speaking Varu\d{n}a, the holder of the noose, assuming the form of a very pure Br\=ahmi\d{n}, again came there to the palace of the king to examine him. The King worshipped him duly and gave him the seat to sit and spoke, with humility, the following reasonable words :-- ``O Lord! My son is not yet purified; how can he be tied to the sacrificial post for being immolated? Therefore I will perform that sacrifice when the boy becomes cleansed after a purificatory rite and becomes a K\d{s}attriya. O Deva! If you know me as your humble servant, have mercy on me; I will then consider myself as blessed. See! The children, not passed through purificatory rites, are not entitled to any act; therefore wait for some time longer.''

54-56. Varu\d{n}a said :-- ``O King! You are deceiving me and putting off the time longer and longer; I now see that you were issueless before and now that you have got a son, you are bound up in an indissoluble tie of affection for a son. Whatever it be, I now go back to my home at your pitiful request; I will wait for some time longer and I will come again. O child! Let you then be true to your words; if it be otherwise, I will surely curse you and thus give vent to my angry feelings.''

57. The King said :-- O Lord of the Waters! After the completion of the Sam\=avartan ceremony (a pupil's return home after finishing his holy study), I will duly sacrifice my son at the great Naramedha sacrifice; there is no doubt.

58-71. Vy\=asa said :-- Varu\d{n}a was very pleased at the King's words and quickly went back saying ``Let it be so.'' The king also became comforted. On the one hand, the king Hari\'schandra's son became widely known by the name of Rohita; and as he got older, he became gradually versed in all the sciences and became very clever and intelligent. That boy then came to know by degrees the cause of the sacrifice in full detail; and knowing that his death is quite certain, became very afraid and quickly fled away from the King and went and stayed in caves of mountains with a fearful heart. Then, when the proper time came, Varu\d{n}a came up there to the royal palace, desirous to have the sacrifice and spoke to the King thus :-- ``O King! Now the prescribed time has come; therefore perform the sacrifice that you have resolved to celebrate.'' The King was very much pained to hear this and spoke with a very sad appearance :-- ``O Best of the Devas! What can I do now?

My son has fled away out of the fear of his life; I do not know his whereabouts.'' Varu\d{n}a became very angry at these words and cursed him thus :-- ``O Liar! You are an hypocrite pundit; therefore you deceived me frequently. Let therefore the disease dropsy come and attack your body.'' Varu\d{n}a, the Holder of the noose, cursing thus, went back to his own abode. The King was attacked with that disease, remained in his own residence, afflicted with cares and anxieties. Rohita, the son of the king Hari\'schandra, heard about the severe illness of his father when he was very much tormented with that disease, as the curse of Varu\d{n}a. One day a traveller told him :-- ``O son of the King! Your father is very ill with dropsy, due to the curse, and is very sorry. Certainly your brain has turned wrong; vain is your coming in this world; you have passed your life to no purpose, for you are staying still in this mountain cave, abandoning your sorrowful father. Certainly you are a bad disobedient son; what use is there in your keeping up this body? What purpose will be served by your birth? When you have got this body, you have abandoned that father and are staying in this solitary cave. Know this as certain that to sacrifice one's life is the duty of a good and obedient son; therefore what more shall I say now than this that your father the king Hari\'schandra ailing from a severe illness is very sorry for you and is always weeping.''

72-74. Vy\=asa said :-- O King! Hearing from the passerby these good words, the prince Rohita wanted to go to his sorrowful father attacked with disease when Indra assuming a Br\=ahmi\d{n} form came up to him and began to speak to him when he was alone like one who was filled with mercy. O Son of a King! You are a fool; are you not positively acquainted with the fact that your father is in trouble; why then do you intend in vain to go there?

Here ends the Twelfth Chapter on the cause of the war between \=Adi and Baka in the Sixth Book of the Mah\=apur\=a\d{n}am of \'Sr\={\i} Mad Dev\={\i} Bh\=agavatam of 18,000 verses by Mahar\d{s}i Veda Vy\=asa.



