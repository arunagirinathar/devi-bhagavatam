\chapter{On the hymns to the Great Dev\={\i} by Vi\d{s}\d{n}u}

1-20. Brahm\=a said :-- Thus speaking, Bhagav\=an Jan\=ardana Vi\d{s}\d{n}u spoke to me again :-- ``Come, let us bow down to Her again and again and let us go to Her. We shall reach at Her feet fearlessly and we will chant hymns to Her; Mah\=a M\=ay\=a will be pleased with us and will grant us boons. If the guards at the entrance prevent us from going, we would stand at the gateway and we will chant hymns to the Dev\={\i} with one mind.''

Brahm\=a said :-- When Hari addressed us in the above way, we two became choked by intense feelings of joy; our voice became tremulous and

we waited there for some time; our hearts were elated with joy to go to Her. We then accepted Hari's word said ``Om'' and got down from our car and went with hastened steps and with fear to the gate. Seeing us standing at the gateway, the Dev\={\i} Bhag\=avat\={\i} smiled and within an instant transformed us three into females. We looked beautiful and youthful women, adorned with nice ornaments; thus we greatly wondered and went to Her. Seeing us standing at Her feet in feminine forms, the beautiful Dev\={\i} Bhag\=avat\={\i}, looked on us with eyes of affection. We then bowed to the great Dev\={\i}, looked at one another and stood before Her in that feminine dress. We three, then, began to see the pedestal of the great Dev\={\i}, shining with the lustre of ten million Suns and decorated with various gems and jewels. We next discerned that thousands and thousands of attendants are waiting on Her. Some of them are wearing red dress; some blue dress, some yellow dress; thus the Deva girls, variously dressed were serving Her and standing by Her side. They were dancing, singing on and playing with musical instruments and were gladly chanting hymns in praise of the Dev\={\i}. O N\=arada!We saw there another wonderful thing. Listen. We saw the whole universe, moving and non-moving within the nails of the lotus feet of the Dev\={\i}. We saw there myself, Vi\d{s}\d{n}u, Rudra, V\=ayu, Agni, Yama, Moon, Sun, Varu\d{n}a, Tvast\=a, Indra, Kuvera and other Devas, Apsar\=as, Gandarbhas, rivers, oceans, mountains, Visv\=avasus Chitraketu, Sveta, Chitr\=angada, N\=arada, Tumburu, H\=a H\=a H\=u H\=u and other Gandarbhas, the twin A\'svins, the eight Vasus, S\=adhyas, Siddhas, the Pitris, Ananta and other N\=agas, Kinnaras, Uragas, R\=aksasas, the abode of Vaikuntha, the abode of Brahm\=a, Kaila\'sa mountain, the best of all mountains; all were existing there. Within that nail of the toe were, reflected all the things of the Universe. The lotus whence I was born, the four faced Brahm\=a like myself on that lotus, Bhagav\=an Jagann\=ath lying on that bed of Ananta, the two Demons Madhu Kaitabha, all I saw there.

21-31. Seeing all these wonderful things within the nails of Her lotus feet, I became greatly surprised and thought timidly :-- ``What are all these!'' My companions Vi\d{s}\d{n}u and \'Sankara were struck with wonder. We three, then, made out that She was our Mother of the universe.

Thus full one hundred years passed away in seeing the various glories of the Dev\={\i} in the auspicious nectar-like Mani Dv\={\i}pa; as long we were there, Her attendants, the Deva girls adorned with various ornaments gladly considered us as Sakh\={\i}s. We, too, were greatly fascinated by their enchanting gestures and postures. For that reason, we saw always their beautiful movements with great gladness. Once, on an occasion, Bhagav\=an Vi\d{s}\d{n}u, while He was in that feminine form, chanted hymns in praise of the great Dev\={\i} \'Sr\={\i} Bhuvanes`var\={\i}.

\'Sr\={\i} Bhagav\=an said :-- Salutation to the Dev\={\i} Prakriti, the Creatrix; I bow down again and again to Thee. Thou art all-auspicious and grantest the desires of Thy devotees; Thou art of the nature of Siddhi (success) and Vriddhi (increase). I bow down again and again to Thee. I bow down to the World Mother, Who is of the nature of Everlasting Existence, Intelligence and Bliss. O Dev\={\i}! Thou createst, preservest and destroyest this Universe; Thou dost the Pralaya (the great Dissolution) and showest favour to the created beings. Thus Thou art the Authoress of the above five fold things that are done; so, O Bhuvane\'svar\={\i}, I bow down to Thee! Thou art the great efficient and material cause of the changeful. Thou art the Unchangeable, Immoveable Consciousness; Thou art the half letter (Ardham\=atr\=a), Hrillekh\=a (the consciousness that ever pervades both inside and outside the Universe); Thou art the Supreme Soul and the individual soul. Salutation again and again to Thee.

O Mother! I now realise fully well that this whole Universe rests on Thee; it rises from Thee and again melts away in Thee. The creation of this Universe shews Thy infinite force. Verily, Thou art become Thyself all these Lokas (regions). During the time of creation Thou createst the two formless elements ak\=as\=a and V\=ayu and the three elements with form, fire, water, and earth; then with these Thou createst the whole Universe and shewest this to the Enjoyer Puru\d{s}a, who is of the nature of consciousness, for His satisfaction. Thou again dost become the material cause of the twentythree (23) Tattvas, Mahat, etc., as enumerated in the S\=ankhya system and appearest to us like a mirage.

32. O Mother! Were it not for Thee, no object would be visible, Thou pervadest the whole Universe. It is for this reason that those persons that are wise declare that even the Highest Puru\d{s}a can do no work without Thy aid.

33-34. O Dev\={\i}! Thou createst and art giving satisfaction to the whole Universe by Thy power; again at the time of Pralaya Thou swallowest forcibly all these that are seen. So, O Dev\={\i}! Who can fathom Thy powers? O Mother! Thou didst save us from the hands of Madhu and Kaitabha. Then Thou hast brought us to this Mani Dv\={\i}pa and shewed us Thy own form, all the extended regions and immense powers and given us exquisite delight and joy. This is the highest place of happiness.

35-37. O Mother! When I Myself, \'Sankara and Brahm\=a or any one of us is unable to fathom Thy inconceivable glory, who else can then ascertain? O Bhav\=an\={\i}! Who knows, how many more than the several regions that we saw reflected in thy nails of Thy feet, exist in Thy creation. O One endowed with infinitely great powers! O Dev\={\i}! we saw another Vi\d{s}\d{n}u, another Hara, another Brahm\=a, all of great celebrity in the Universe exhibited by Thee; who knows how many other such Brahm\=as,

etc., exist in Thy other Universes! Thy glory is infinite. O Mother! I bow down again and again to Thy lotus feet and pray to Thee that may Thy this form exist always in my mind. May my mouth always utter Thy name and may my two eyes see always Thy lotus feet.

38-43. O Revered One! May I remember Thee as my Goddess and may'st Thou constantly look on myself as Thy humble servant. O Mother! What more shall I say than this :-- May this relation as mother and son always exist between Thee and me. O World-Mother! There is nothing in this world that is not known to Thee for Thou art omniscient. So O Bhav\=an\={\i}! What more shall my humble self declare to Thee! Now dost Thou do whatever Thou desirest. O Dev\={\i}! The rumour goes that Brahm\=a is the Creator, Vi\d{s}\d{n}u is the Preserver, and Mahe\'svara is the Destroyer! Is this true? O Eternal One! It is through Thy Will power, through Thy force, that we create, preserve and destroy. O Daughter of the Himalaya mountain! The earth is supporting this Universe; it is Thy endless might that is holding all this made of five elements. O Grantress of boons! It is through Thy power and lustre that the Sun is lustrous and becomes visible. Though Thou art the attributeless Self, yet by Thy M\=ayic power Thou appearest in the form of this Prapancha Universe. When Brahm\=a, Mahe\'sa, and I myself take birth by Thy power and are not eternal, what more can be said of Indra and other Devas than this that they are mere temporary things and created. It is only Thou that art Eternal, Ancient Prakriti and the Mother of this Universe. O Bhav\=an\={\i}! Now I realise from my remaining with Thee, that it is Thou that dost impart, out of mercy, the Brahm\=a vidy\=a to the ancient Puru\d{s}a; and thus He can realise His eternal nature. Otherwise He will remain always under delusion that He is the Lord, He is the Puru\d{s}a without beginning, that He is good and the Universal Soul, and thus suffers under various forms of egoism (Ahamk\=ara).

Thou art the Vidy\=a of the intelligent persons and the \'Sakti of the beings endowed with force; Thou art K\={\i}rt\={\i} (fame), K\=anti (lustre), Kamal\=a (wealth) and the spotless Tusti (peace, happiness). Amongst men, Thou art the dispassion, leading to Mukti (complete freedom from bondage). Thou art the G\=ayatri, the mother of the Vedas; and Thou art Svah\=a, Svadh\=a, etc. Thou art the Bh\=agavat\={\i}, of the nature of the three Gu\d{n}as; Thou art the half m\=atr\=a (half the upper stroke of a letter), the fourth state, transcending the Gu\d{n}as. It is Thou that givest always the \'S\=astras for the preservation of the Devas and the Br\=ahma\d{n}as. It is Thou that hast expanded and manifested this whole phenomenon of the visible Universe for the liberation of the embodied souls (J\={\i}vas), the parts of the pure holy Br\=ahma\d{n}, the Full, the Beginningless, the Deathless, forming the waves of

the lnfinite expanse of ocean. When the J\={\i}va comes to know internally and becomes thoroughly conscious that all this is Thy work, Thou createst and destroyest, that all this is Thy M\=ayic pastime, false, like the parts of an actor in a theatrical play, then and then only he desists for ever from his part in this Theatre of world. O Mother! O Destroyer of the greatest difficulties! I always take refuge unto Thee. Thou dost save me from this ocean of Sams\=ara, full of Moha (delusion). Let Thou be my Saviour when my end will come, from these infinitely troublesome and unreal pains arising from love and hatred. Obeisance to Thee! O Dev\={\i}! O Mah\=a vidy\=a! I fall prostrate at Thy feet. O Thou, the Giver of all desires! O Auspicious One! Dost Thou give the knowledge that is All-Light to Me.

Thus ends the fourth chapter of the Third Skandha on the hymns to the Great Dev\={\i} by Vi\d{s}\d{n}u in the Mah\=apur\=a\d{n}a \'Sr\={\i}mad Dev\={\i} Bh\=agavatam of 18,000 verses by Mahar\d{s}i Veda Vy\=as.



