\chapter{On the killing of \'Sankhach\=uda}

1-6. N\=ar\=aya\d{n}a said : -- \'Siva, versed in the knowledge of the Higher Reality, hearing all this, went himself with His whole host to the battle. Seeing Him, \'Sankhach\=uda alighted from his chariot and fell prostrate before him. With great force he got up and, quickly putting on his armour he took up his huge and heavy bow case. Then a great fight ensued between

\'Siva and \'Sankhach\=uda for full one hundred years but there was no defeat nor victory on either side. The result was stalemate. Both of them, Bhagav\=an and the D\=anava quitted their weapons. \'Sankhach\=uda, remained on his chariot and Mah\=adeva role on His Bull. Hundreds and hundreds of D\=anavas were slain. But extraordinarily endowed with divine power, \'Sambhu restored to life all those of His party that were slain.

7-30. In the meanwhile, an aged Br\=ahma\d{n}a, very distressed in his appearance, came to the battlefield and asked \'Sankhach\=uda, the King of D\=anavas :-- ``O King! Grant me what I beg of you; you give away in charity all sorts of wealth and riches; give me also what I desire; give me, a Br\=ahmi\d{n}, something also. I am a quiet peaceable aged Br\=ahmi\d{n}, very very thirsty. Make your Promise first and then I will speak to you what I desire.'' (Note :-- The Br\=ahmi\d{n}s only are fit for receiving frauds and cheatings.)

The King \'Sankhach\=uda, with a gracious countenance and pleasing eyes swore before him that He would give him what he would desire. Then the Br\=ahmi\d{n} spoke to the King with great affection and M\=ay\=a :-- ``I am desirous of your Kavacha (amulet).'' The King, then, gave him the Kavacha (the amulet, mantra written on a Bhurja bark and located in a golden cup). Bhagav\=an Hari (in the form of that Br\=ahmi\d{n}) took that Kavacha and, assuming the form of \'Sankhach\=uda came to Tulas\={\i}. Coming there, He made His M\=ay\=a (magic) manifest and held sexual intercourse with her. At this time Mah\=adeva took up the Hari's trident-aiming at the king of the D\=anavas. The trident looked like the Mid-day Sun of summer, flaming like a Pralaya fire. It looked irresistible and invincible as if quite powerful to kill the enemies. In brilliance it equalled the Sudar\'san Chakra (disc) and it was the chief of all the weapons. No other body than \'Siva and Ke\'sava could wield such a weapon. And everybody feared that but \'Siva and Ke\'sava. In length it was one thousand Dhanus and in width it was one hundred hands. It seemed lively, of the nature of Brahm\=a, eternal and not capable to be noticed, whence and how it proceeded. The weapon could destroy, by its own free L\={\i}l\=a (Will) all the worlds. When \'Siva held it aloft and aiming at \'Sankhach\=uda, He hurled it on him, the King of the Demons quitted his bows and arrows and with mind collected in a yoga posture, began to meditate on the lotus-feet of \'Sr\={\i} Kri\d{s}\d{n}a with great devotion. At that moment, the trident, whirling round fell on \'Sankhach\=uda and easily burnt him and his chariot to ashes. He, then assuming the form of a two-armed Gopa, full of youth, divine, ornamented with jewels, holding flute, mounted on a Divine Chariot, surrounded by kotis

and kotis of Gopas who came there from the region of Goloka, whose bodies were built up of excellent jewels, and \'Sankhach\=uda then went up to the Heavens (Goloka, where \'Sr\={\i} Brind\=abana is located in the middle). He went to Vrind\=aban, full of R\=asas (sentiments) and bowed down at the lotus feet of R\=adh\=a Kri\d{s}\d{n}a with devotion. Both of them were filled with love when they saw Sud\=am\=a, and, with a gracious countenance and joyful eyes, they took him on their laps. On the other hand the \'S\=ula weapon came with force and gladness back again to Kri\d{s}\d{n}a. The bones of \'Sankhach\=uda, O Narada! were transformed into conch-shells. These conch-shells are always considered very sacred and auspicious in the worship of the Devas. The water in the conch-shell is also very holy and pleasing to the Devas. What more than this, that the water in the conch-shell is as holy as the water of any T\={\i}rtha. This water can be offered to all the Gods but not to \'Siva. Wherever the conch-shell is blown, there Lak\d{s}m\={\i} abides with great pleasure. If bathing be done with conch-shell water, it is equivalent to taking bath in all the T\={\i}rthas. Bhagav\=an Hari resides direct in the conch-shell. Where \'Sankha is placed, there Hari resides. Lak\d{s}m\={\i} also resides there and all inauspicious things fly away from there. Where the females and \'S\=udras blow the \'Sankhas, Lak\d{s}m\={\i} then gets vexed and, out of terror, She goes away to other places. O N\=arada! Mah\=adeva, after killing the D\=anava, went to His own abode. When He gladly went away on His Vehicle, on the Bull's back, with His whole host, all the other Devas went to their respective places with great gladness. Celestial drums were sounded in the Heavens. The Gandharbas and the Kinnaras began to sing songs. And showers of flowers were strewn on \'Siva's head. All the Munis and Devas and their chiefs began to chant hymns to Him.

Here ends the Twenty-Third Chapter of the Ninth Book on the killing of \'Sankhach\=uda in \'Sr\={\i} Mad Dev\={\i} Bh\=agavatam of 18,000 verses by Mahar\d{s}i Veda Vy\=asa.



