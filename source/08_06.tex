\chapter{On the rivers and the mountains Sumeru and others}

1-32. N\=ar\=aya\d{n}a said :-- O N\=arada! This Aru\d{n}od\=a river that I mentioned to you rises from the Mandara mountain and flows by the east of Il\=avar\d{s}a. The Pavana Deva (the God of wind) takes up the nice smell from the bodies of the wives of the Yak\d{s}as and Gandharbas, etc., and the attendants of the Dev\={\i} Bhav\=an\={\i} and keeps the surroundings of the earth there filled with nice smell for ten Yoyanas around. Again the rose-apples with their nuts, of the size of an elephant, fall down upon the earth from the high peaks of the mountain Mandara and break into pieces; the sweet scented juices flow down as a river. This is called the Jamb\=u river and this flows by the south of Il\=avar\d{s}a. The Dev\={\i} Bhagavat\={\i} there is pleased with the Juice of that rose-apple (Jamb\=u) and is known by the name of Jamb\=adin\={\i}. The Devas, N\=agas, and \d{R}i\d{s}is all always worship with great devotion, the lotus-feet of the merciful Dev\={\i}, wishing the welfare of all the J\={\i}vas. The mere remembering of the name of the Dev\={\i} destroys all the disease, and all the sins of the sinner. Therefore the Devas always worship and chant the names of the Dev\={\i}, the Remover of all obstacles. She is installed on both the banks of the Jamb\=u river. If men recite Her names Kokil\=ak\d{s}\={\i}, Karu\d{n}\=a, K\=amap\=uj\={\i}t\=a, Kathoravigrah\=a, Devap\=ujy\=a, Dhany\=a, Gavastin\={\i} and worship, so they get their welfare both in this world and in the next. With the juice of the Jamb\=u fruit aided by the combination of the wind and the rays of the Sun, is created the gold. Out of this are made the ornaments for the wives of the Immortals and the Vidy\=adharas. This gold, created by the Daiva, is known by the name of the J\=amb\=unada gold. The love-stricken Devas make their crowns, waist bands and armlets out of this gold for their sweet-hearts. There is a big Kadamba tree on the mountain Supar\'sva; the five streams of honey called Madhu Dh\=ar\=a get

out from its cavities and running by the west of Il\=avrita Var\d{s}a, flow over the land. The Devas drink its waters; and their mouths become filled with the sweet fragrance. The air carries this sweet fragrant smell to a distance of even one hundred Yoyanas. The Dh\=are\'svar\={\i} Mah\=a Dev\={\i} dwells there, the Fulfiller of the desires of the Bhaktas, highly energetic, of the nature of K\=ala (the Time, the Destroyer), and having large face (Mah\=anan\=a), faces everywhere, worshipped by the Devas and is the presiding Deity of the woods and forests all around. The Dev\={\i}, the Lady of the Devas, is to be worshipped by the names ``Kar\=ala Deh\=a,'' ``K\=al\=amg\={\i}'', ``K\=amakotipravartin\={\i}''. The great Banyan tree named \'Satabala is situated on the top of the Kumud mountain. From its trunk many big rivers take their origin. These rivers possess such influences as to give to the holy persons there, the milk, curd, honey, clarified butter, raw sugar, rice, clothing, ornaments, seats, and beddings, etc., whatever they desire. Therefore these rivers are called K\=amadugh. They come gradually down the earth and flow by the north of Il\=avar\d{s}a. The Bhagavat\={\i} M\={\i}n\=ak\d{s}i dwell there and is worshipped by the Suras and the Asuras alike. That Deity clothed blue, of fearful countenance, and ornamented with hairs of blue colour, always fulfil the desires of the Devas dwelling in the Heavens. Those that worship Her, remember Her or praise Her by the names Atim\=any\=a, Atip\=ujy\=a, Mattam\=ata\d{n}ga G\=amin\={\i}, Madanonm\=adin\={\i}, M\=anapriy\=a, M\=anapriyatar\=a, M\=arabegadhar\=a, Marap\=ujit\=a, M\=aram\=adin\={\i}, May\=uravara\'sobh\=adhy\=a, \'Sikhiv\=ahanagarbhabh\=u, etc., are honoured by the Deity M\={\i}nalochan\=a Ek\=angar\=upi\d{n}\={\i} and the Parame\'svara and get all sorts of happiness. Those drink the clear waters of these rivers become free from old age or decay, worry, perspiration, bad smell, from any disease, or premature death. They do not suffer anything from error, from cold, heat, or rains, or from any paleness in their colour. They enjoy extreme happiness as long as they live and no dangers come to them. O Child! Now hear the names of the other twenty mountains that encircle the Golden Sumeru mountain at its base, as if they were the filaments round the pericarp of a flower. The first is Kura\d{n}ga; they come in order Kuraga, Ku\'sumbha, Vikankata, Trik\=uta, \'Si\'sira, Patanga, Ruchaka, Ni\d{s}adha, \'Sit\={\i}v\=asa, Kapila, \'Samkha, Vaid\=urya, Ch\=arudhi, Hamsa, Ri\d{s}abha, Naga, K\=alanjara and lastly N\=arada. The central one is the twentieth.

Here ends the Sixth Chapter of the Eighth Book on the rivers and the mountains Sumeru and others in the Mah\=apura\d{n}am \'Sr\={\i} Mad Dev\={\i} Bh\=agavatam, of 18,000 verses, by Mahar\d{s}i Veda Vy\=asa.



