\chapter{On the birth of Vritr\=asura}

1-11. Vy\=asa said :-- The extremely covetous Indra, then, mounted on his Air\=avata elephant and determined to kill the Muni. He went to him and saw him immersed in deep Sam\=adhi, firmly seated in his posture and with his speech controlled. At that time, a halo of light emanated from

his body and he looked like a second Sun and a blazing fire. Indra became very sad and dejected when he saw that. Indra then thought within himself thus :-- ``Oh! Can I slay this Muni, free from any vicious inclinations, and endowed with the power of Tapas, blazing like a fire! This is quite against the Dharma. But, Alas! He wants to usurp my position; how can I, then, neglect such an enemy?'' Thus cogitating, Indra hurled at the Muni his swift going, infallible thunderbolt, the Muni remaining engaged in his penance and shining like the Sun and Moon. The ascetic, struck thus, fell on the ground and died, like a mountain peak struck by thunder falling on the ground and presenting a wondrous sight. Indra became very glad when he killed the Muni; but the other Munis then cried aloud :-- ``Oh! We are killed! Alas! What a crime has Indra committed today! Oh! The vicious Indra has killed today this jewel amongst the Munis without any offence! Let, then, this sinner reap the fruits of his sinful act without any delay.'' Indra, then, went back soon to his own abode; on the other hand, the high-souled Muni, though killed, looked as it were, living by the lustre of his own body. Indra, then, seeing him lying like a living man thought that the Muni might get alive and so became very sad. While he was thus arguing in his mind, he saw before him a wood cutter named Tak\d{s}a and began to speak to him for his own selfish ends thus ``O Artisan! Cut all the heads of this Muni and keep my word; this highly lustrous Muni is looking as it were alive; therefore, if you sever his heads, he cannot be alive.'' Tak\d{s}a then cursed him and spoke thus.

12-14. ``O King of the Devas! The neck of this Muni is very big and therefore cannot be severed; my axe is not at all fit for this work. Specially I cannot do such a blameable act. You have done a very heinous crime, quite against the law of the good persons; I fear sin; I will not be able to cut the heads of a dead man. This Muni is lying dead; what use is there in severing his head again? O P\=aka\'s\=asana! The killer of the demon P\=aka! Why do you fear in this?''

15. Indra said :-- ``O Artisan! This Muni is my dire enemy. Life seems to be still lingering in his body; his body is still lustrous, I fear if the Muni be alive again!''

16. Tak\d{s}a told :-- ``Do you not feel shame in doing this heinous crime, when you know everything? Do you not fear God for the crime of killing a Br\=ahmi\d{n}?''

17. Indra said :-- I will make Pr\=aya\'schitta (penance) afterwards for the washing away of my sins; but my duty at present is to kill my enemy.

O Fortunate One! The wise men, clever in polity, say that enemies must be killed by any excuse whatsoever.

18. Tak\d{s}\=a then replied :-- ``O Maghavan! You are doing this sinful deed out of your avarice; but, O Lord! I have no cause whatsoever; how then without any cause, can I engage myself in such a vicious act?''

19-20. Indra said :-- ``O Tak\d{s}an! I will allot a share to you wherever there will be a sacrifice. The human beings will invariably offer to you the head of the animal killed at any sacrifice. Now cut his head according to this rule.''

21-42. Vy\=asa said :-- O King! That Tak\d{s}\=a became very glad when he heard thus from Indra and struck off the heads of the Muni with his very strong axe. O powerful King! When the three heads, thus severed, fell to the ground, thousands and thousands of birds came out of those heads in quick succession. The three groups of birds Kalavinkas, Tittiris and Kapinjalas came out very rapidly from the three heads in due succession. The Kapinjala birds came out of that mouth that used to chant the Vedas and used to drink Soma; the Tittiri birds came out of that mouth that used to see all the quarters as if it drank them; and the Kalavinka birds came out of that face that used to drink wine. Indra became very glad to see the birds thus coming out of his mouths and went back at once to his Heavens. O King! No sooner Indra went back, than Tak\d{s}\=a came back to his own house and felt himself very pleased to receive his share of sacrificial things. On returning to his home, Indra thought that he had done his duty in slaying his powerful enemy. It did not pass in his mind that he had committed the Brahmahatty\=a sin (i. e., that he had killed a Br\=ahmi\d{n}). When Vi\'svakarm\=a heard that his virtuous son had been killed, he became very angry (in his mind) and said that as Indra had killed his qualified son engaged in asceticism without any offence, he would create another son to kill Indra. Let the Devas see his strength and power of Tapasy\=a and let Indra, too, reap the far-reaching effects of his own Karma. Thus saying, Vi\'svakarm\=a distressed with anger, offered oblations in the sacrificial Fire, reciting Mantram from the Atharvan Vedas, with the object of producing a son. When Homa was performed for eight nights consecutively, a man quickly came out of that burning fire, as if he was the Incarnate of Fire itself. Seeing the lustrous son before him, come out of the fire and endowed with power and energy, Vi\'svakarm\=a said ``O Indra's enemy! Grow by my power of asceticism.'' When Vi\'svakarm\=a spoke these words, burning with anger, that brilliant fiery son began to grow, towering high above the Heavens. Within a moment that man looked a second God of Death and appeared like a

mountain and shone like the God Himself. Then he spoke to his own father Vi\'svakarm\=a, who was very distressed ``O Father! Put my name. Pray, what use can I be to you? Why do you look so aggrieved and anxious; please explain to me all the causes. I make a firm vow today that will remove the cause of your sorrow. Father ! Of what avail is that to his father when he is not able to remove his sorrows!'' O Father! Shall I drink the ocean or crumble the mountains to dust or shall I obstruct the passage of the rising Sun or shall I kill Indra, Yama, or the other host of Devas or shall I root out the earth and throw it with all beings into the ocean?''

43-53. O King! Hearing thus the sweet words of his son, Vi\'svakarm\=a gladly told his mountain-like son ``O my Son! You are capable to save me from troubles (Vrijina) hence you are named Vritra. O highly Fortunate One! Your brother, named Tri\'sir\=a, was a great ascetic; his three faces were all very strong. He was thoroughly conversant with the Vedas and the Vedang\=as and well versed in all the other knowledges. He remained always engaged in practising asceticism, surprising to the three worlds. Indra killed my qualified son with his thunderbolt; that wicked soul severed the three heads without any offence. Therefore, O Best of beings! Kill that vicious, shameless, deceitful, wicked Indra guilty of the sin of Brahmahatty\=a.'' O King! Thus saying, Vi\'svakarm\=a very much confounded with the breavement of his son, created various divine weapons. He prepared weapons specially suited to kill Indra, the best axes, tridents, clubs, \'Saktis, Tomaras and bows made of horns and arrows, Parighas, Patti\'sas, divine discus like the Sudar\'san Chakra, divine inexhaustible arrow cases with arrows, nice Kavacha, very substantial air-like swift-going chariot looking like a cloud and capable to carry great loads; all these he created and gave over to his son. O King! Vi\'svakarm\=a, the best of architects, excited by anger, made ready all the equipments necessary for war and gave them to his son Vritr\=asura and sent him to kill Indra.

Here ends the Second Chapter of the Sixth Book on the birth of Vritr\=asura in the Mah\=a Pur\=anam \'Sr\={\i} Mad Dev\={\i} Bh\=agavatam of 18,000 verses by Mahar\d{s}i Veda Vy\=asa.



