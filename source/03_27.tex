\chapter{On the virgins fit to be worshipped and the Glory of the Dev\={\i}}

1. Vy\=asa said :-- O king! Those Kum\=ar\={\i}s, who are defective in limbs, who are lepers, who are filled with sores and ulcers over their bodies, whose bodies emit offensive smell or whose bodies are polluted, or those who are

of a bad family are never to be accepted for worship in the Navar\=atra ceremony festival.

2-3. Those who are born blind, who are squint-eyed, who are blind of one eye, of disgraceful appearance, whose bodies are overgrown with hairs, or who are diseased or who are in their menstruation or in any other signs, indicating thus their passionate youthful tendencies, or those who are very lean and thin, or born of widows, or of women unmarried are always to be avoided in this P\=uj\=a.

4. O king! It is only the healthy, graceful, beautiful, without any ulcers, and who are not bastards, those virgins are to be selected for the Kum\=ar\={\i} P\=uj\=a.

5. In all the cases, the Kum\=ar\={\i}s, born of the Br\=ahmi\d{n} families, can be taken; when victory is desired, the Kum\=ar\={\i}s of the K\d{s}hattriya families are preferred; when profit is wanted, the Vai\'sya Kum\=ar\={\i}s and, when general welfare is wanted, the \'S\=udra Kum\=ar\={\i}s are to be taken.

6-7. O king! In the Navar\=atri P\=uj\=a, the Br\=ahmi\d{n}s should select for worship the Br\=ahmi\d{n} Kum\=ar\={\i}s; K\d{s}hattriyas, Br\=ahmi\d{n} or K\d{s}hattriya; the Vai\'sya worshippers can select for worship Br\=ahmi\d{n}, K\d{s}hattriya, or Vai\'sya Kum\=ar\={\i}s. And the \'S\=udra worshippers can select, for worship, any of the four classes. But artists and artisans should select for worship the Kum\=ar\={\i}s from their own families and tribes respectively.

8. If persons become unable to worship on all the days, then it is advised that they should perform the special worship on the eighth day (Astam\={\i} tithi).

9-10. In ancient times, on the eighth day, Bhadra K\=ali Goddess, the destroyer of the sacrifice, started by Dak\d{s}a, appeared on that day in hideous forms, surrounded by hundreds and lakhs of Yogin\={\i}s (one of a class of sixty goddesses or female attendants on K\=ali). Therefore one should worship in particular on the eighth day with scents, garlands, and pastes and various offerings.

11. On this day, P\=ayasa (a food prepared of rice, milk and sugar), and fresh fish are to be specially offered to the Deity. The Homa ceremonies, feasting of the Br\=ahma\d{n}s, and the worship of the Mother Goddess are done with various offerings, the fruits and flowers, and in good quantities.

12. O king! Those who are unable to observe the fasting in this Navar\=atra P\=uj\=a, will reap the same fruits, if they observe fasting for the three days only the Saptam\={\i}, the Astam\={\i}, and the Navam\={\i} tithis.

13. On the seventh, eighth, and ninth days, in these three tithis (lunar days) if one worships with devotion, one will acquire all the merits.

14. When the Dev\={\i}'s worship, Homa, Kum\=ar\={\i} worship and the feasting of the Br\=ahma\d{n}as, all these are done, know that the Navar\=atri P\=uj\=a is completed.

15. O Janamejaya! No worship or vow or charitable gifts extant in this world, can be compared, as regards their meritorious effects, with this Navar\=atra P\=uj\=a.

16. On observing this Navar\=atram Vrata, one gets riches, crops, sons and grandsons, prosperity and happiness, longevity, health and heaven and even the final beatitude.

17. Those who are desirous of learning, riches, or sons will get them all if they perform this most auspicious Navar\=atra ceremony, able to confer fortunes on the devotees.

18. On the performance of this sacrifice, those who want learning get all the learning; and he, who is deprived of his kingdom will get back all his kingdoms.

19. Those who did not, in their previous births, perform this meritorious vow, they become diseased, poor and devoid of sons in their present births.

20. Those women that are barren, or widows or devoid of sons, infer that they never, in their previous births, performed this sacrifice.

21. Those who have not performed the Navar\=atra ceremony, how can they acquire riches in this world and acquire happiness and peace in the next?

22. He who has worshipped the Goddess Bhagavat\={\i} Bhav\=an\={\i} Dev\={\i} with young leaves of the Bel tree, besmeared with red sandal paste, it is he that will undoubtedly become the king in this world.

23. That man who has failed to worship the Goddess of the whole universe, Who fructifies all the pursuits of human life, Who destroys all the troubles, pains and miseries, Who is all suspicious Bhagavat\={\i} Bhav\=an\={\i}, that fellow is sure to pass his days in this world, wretched, impoverished, and surrounded by his enemies on all sides.

24. When Hari, Hara, Brahm\=a, Indra, Fire, Varu\d{n}a, Kuvera, and the Sun when all these possessing all the wealth and powers and filled with the highest felicities, when they meditate constantly the Goddess of the universe, Who is All Existence Intelligence, and Bliss, then what to speak of the human beings! How is it that persons do not worship that Chandik\=a Dev\={\i}, the One that leads all human pursuits to success!

25-26. Why should not the people worship the Goddess Bhav\=an\={\i}, the bestower of all happiness, whose other names are Svah\=a and Svadh\=a, the mantrams under whose intrinsic energies the Devas and the Pitris always get satisfied, and which are recited by all the Munis when they chant in every sacrifice the Vedic mantrams? Under Whose Will power Brahm\=a the Creator, creates all this Universe? Under Whose energy, the

Vi\d{s}\d{n}u Jan\=aradan, the Deva of the Devas, incarnates in this earth in various forms and preserves this world, and under Whose power, \'Sankara destroys this whole Universe?

27. No body, in this whole universe, can have his existence without having recourse to that Prakriti Dev\={\i}, the \'Sakti incarnate; be he a Dev\={\i}, a human being or a bird, or a serpent, Gandharva, R\=akhsasa, Pi\'s\=acha, a mountain or a tree, he cannot move even of his own accord, without the help of this Force.

28. Therefore, why should not anybody worship that Chandik\=a Dev\={\i}, the Awarder of all desires and wealth? And how is it, that a man desiring one of the 4 objects of human pursuits, Dharma, wealth, desires, and the final beatitude, observes not the vow regarding that Deity.

29. So much so, that even a man who has committed a heinous offence, five such are enumerated, viz. (1) killing a Br\=ahma\d{n}, (2) drinking liquor, (3) stealing gold, (4) adultery with the wife of a spiritual guide (5) associating with any such person, if he performs the Navar\=atra vow, he will be absolved entirely from all such sins; there is no doubt in this.

30. O king! Once on a time there lived in the country of Kosala, a trader, poor and miserable, having under him many relations and dependants in his family, whose provisions he had to provide.

31. He had many sons and daughters; when they were very hungry and distressed, then they used to get a little food and that in the evening, only once in twenty-four hours.

32. That trader, too, worked under another, the whole day; and when it was evening, he used also to take his meals. Thus, being very much anxious and distressed, he maintained somehow or other his family members (that are to be maintained).

33-34. This trader was of a quiet temper, of a good conduct, truthful, always ready to act religiously, devoid of anger, steady and contented, void of vanity and jealousy; daily he used to worship the Devas, Pitris, and the guests and used to take his meals after all his family members had taken their meals.

35-36. Thus many days passed away when that good trader, named Su\'s\={\i}la, being very much perplexed with poverty and hunger, asked a quiet tempered Br\=ahmi\d{n} ``O Bh\=udeva! (deva incarnate on the earth) kindly tell me positively how this state of poverty can be got rid off!

37. O holy minded! Kindly advise me such as preserves my honour; I do not want wealth, nor do I like to be a rich man; O Br\=ahmi\d{n}! I want just enough to meet with the expenses, incurred in maintaining my family; please advise so that I may be able to earn this much only.

38. I have many sons; I have not got any food, sufficient enough to give them even a handful of rice.

39. Alas! My youngest son was crying today for food; I have driven him out of the house by chastising him. O Br\=ahmi\d{n}! What am I do? I have got no wealth; my heart is burning with grief and sorrow; my baby has gone out of the house, weeping and hungry.

40. My daughter has come to a marriageable age; I have no money. Her age has exceeded ten years; the marriageable age limit has been exceeded. Alas! What am I to do?

41-42. O Br\=ahmi\d{n}! I am expressing my sorrow for all that. You are merciful, and all-knowing; tell me any means, be it asceticism, gifts, vow, or the reciting of any mantrams by which I can maintain my family; I want wealth just sufficient for that purpose and nothing more.

43. O high minded one! Kindly devise and tell me some means by which my family members become happy in this world.''

44-46. Vy\=asa said :-- The Br\=ahmi\d{n} that used to practice vows when thus asked by the trader told him gladly ``O trader! Do now the Navar\=atri vow, the most auspicious, and worship the Bhagavat\={\i}, perform Homa, and feast the Br\=ahmi\d{n}s. Have the Vedas and Pur\=a\d{n}as recited and recite then slowly the \'Sakti mantram and try, as much as you can, to do other concomitant ceremonies; and your desires will thus be undoubtedly fulfilled.

47. There is no other vow superior to this in this world; this vow is very holy and will bring unto you happiness.

48. This vow leads to wisdom and liberation; destroys enemies and increases posterity and prosperity.

49. In former days, \'Sr\={\i} R\=ama Chandra suffered very much owing to his being deprived of his kingdom; and, then on account of his wife being stolen away. Subsequently he performed this Navar\=atra vow in Kiskindhy\=a, his heart being heavily laden with grief.

50. Though troubled very much, on account of the bereavement of S\={\i}t\=a, still R\=ama Chandra observed the Vow of Navar\=atra and worshipped the Goddess according to the prescribed rules and rites.

51-52. As a fruit of this worship he was able to bridge the great ocean and kill the giant Kumbha Kar\d{n}a, Meghan\=ada, the R\=ava\d{n}a's son, and R\=ava\d{n}a, the king of Lanka; and subsequently he was able to recover his S\={\i}t\=a. He installed Vibh\={\i}\d{s}a\d{n}a on the throne of Lanka (Ceylon) and at last returned to Ayodhy\=a and reigned there without any enemies.

53. O best of the Vai\'syas! R\=ama Chandra, of incomparable prowess, was able to obtain happiness in this world on account of the influence of this Navar\=atra ceremony.

54-55. Vy\=asa said :-- O king! That Vai\'sya, hearing thus the Br\=ahmi\d{n}'s words, made him his Guru, was initiated by him in the seed mantra of M\=ay\=a and ceaselessly, without any laziness, recited slowly the mantram for nine nights and worshipped the Dev\={\i}, with great caution and with various offerings. Thus for nine consecutive years he devoted himself to the Japam (reciting slowly) of the seed mantra of M\=ay\=a till, at last, when the ninth year was completed, the Great Goddess appeared distinctly before his eyes on the night of the great Astam\={\i} tithi (the eighth day of the bright half) and gave him various boons and delivered the Vai\'sya from poverty and bestowed on him wealth and his other desired things.

Here ends the 27th Chapter on the virgins fit to be worshipped and the Glory of the Dev\={\i} in the Mah\=a Pur\=a\d{n}am \'Sr\={\i}mad Dev\={\i} Bh\=agavatam by Mahar\d{s}i Veda Vy\=asa in the Third Adhy\=aya.



