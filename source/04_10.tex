\chapter{On the curse on Vi\d{s}\d{n}u by Bhrigu}

1-4. Janamejaya said :-- O Son of Par\=a\'sara! There has arisen a great doubt in my mind on hearing just now your words. These Nara N\=ar\=aya\d{n}a are the two sons of Dharma; they are ascetics, calm and quiet, the Am\'sas of Vi\d{s}\d{n}u; they reside in a holy place of pilgrimage! They are filled with the Sattvic qualities, subsisting always on roots and fruits of the forest, the highsouled hermits and truthful. How were they addicted to such warfare? Why had they left their invaluable asceticism? And with what object were they fighting for full one thousand Deva years with Prahl\=ada.

5. What was the end, O Muni, of their fight with Prahl\=ada? Kindly explain to me in detail the cause of this warfare.

6. Women, wealth or any other worldly object can be the cause of any quarrel or fight amongst any persons; but, in this case, the two ascetics had none of these; how then this idea of fight sprung within their minds.

7-8. And why did they practise such severe austerities? Was it that they had to overpower others, or enjoy pleasures themselves or to reach Heaven

that they practised tapasy\=a? What fruits did they eventually obtain from such penances?

9. They became very lean and thin through their asceticism; still how could they fight full one thousand Deva years without getting fatigued.

10. They were not entangled in this fight for kingdom, or wealth or for women or for any other worldly object; then why did they fight with the high souled Prahl\=ada?

11. Having no attachment for any worldly object nor any desire to gain any thing therefrom, why did they engage themselves so thoroughly, in such pains giving battle?

12. Intelligent persons always do works leading to bliss; they never do painful works; this is the long standing rule of the world.

13. The two sons of Dharma were the Am\'sas of Hari, all knowing and adorned with all qualities; why did they fight, subversive of religion?

14. O Mah\=ar\d{s}i! Even the dull and stupid persons in the world won't go to these deadly battles leaving asceticism and sam\=adhi, leading to the purification of all desires.

15. I have heard that Yay\=ati, the Lord of the earth was dropped from Heaven to this world, owing to his Ahamk\=ara, though he was a virtuous king devoted to charities and sacrifices.

16-17. No sooner Yay\=ati, said the king :-- did A\'svamedha sacrifice, etc., with Ahamk\=ara, egoism, he was dropped by Indra with thunderbolt in his hands. So one can see that, without Ahamk\=ara no fight can occur. The ascetics had no bodily strength; therefore if they had to fight, it is through the waste of their Tapasy\=a that they could do so.

18. Vy\=asa said :-- O king! The all knowing sages that have realised the truth or Dharma declare the threefold Ahamk\=ara arising out the Sattvic, Rajasic and Tamasic qualities respectively to be the causes of this world.

19. How, then, can these two Munis being embodied forego their Ahamk\=aras? Without any cause, no actions follow; this is quite certain.

20. Tapas, charities, sacrifices all originate from the Sattvic qualities. And quarrels arise from the Rajasic or Tamasic qualities.

21. All arise from Ahamk\=ara, whether good or bad; this is quite certain.

22. There is no other thing that enchains a soul than this Ahamk\=ara. It is out of Ahamk\=ara that this Universe is created: how can it be then free from it?

23. O King! Brahma, Vi\d{s}\d{n}u, Mahe\'sha, even these are with Ahamk\=aras. Then how can you expect other ordinary Munis to be free from it?

24. Encased with Ahamk\=ara, this Universe is rolling. Births and death occur repectively through this Karma.

25. O Lord of the earth. The Devas, birds and men are revolving in this world like the wheel of a chariot.

26. In this wide world who can count how many Avat\=aras Vi\d{s}\d{n}u had to take in all sorts of wombs, good or low.

27. Ordained by the Lord of the Universe, N\=aray\=a\d{n}a Himself had to take the Fish, Tortoise, Boar, Man Lion and the Dwarf incarnations.

28. Vasudeva Jan\=ardana the Lord, had to undertake countless Avat\=aras births in this world.

29. In the Vaivasvata manvantara, the Avat\=aras of Bhagav\=an Hari are being mentioned to you. Hear!

30. The all pervading Lord of the world, the God of the Gods, had to take several incarnations in this world, owing to the curses inflicted by Bhrigu.

31. The king said :-- There has now again arisen another fresh doubt, why was Vi\d{s}\d{n}u cursed by Bhrigu Muni?

32. O Muni! What injury did Hari commit to that Muni, and whereof the Muni Bhrigu cursed him.

33-34. Vy\=asa said :-- Hear, O king! the cause of the curse; I will narrate to you. In days of yore, the king Hira\d{n}yakasipu, the son of Kasyapa often quarrelled with the Devas; owing to this incessant warfare, the whole universe was much alarmed and perplexed.

35. And when Hiranyakasipu was slain by the Man-Lion incarnation, Prahl\=ada, the tormentor of the foes, continued his enmity towards the Devas and began to annoy them.

36. Thus one hundred years dreadful battle occurred between the Devas and Prahl\=ada, to the astonishment of all.

37-38. O king! The Devas fought very hard and were victorious. Prahl\=ada was defeated and was sorely grieved. Hearing that the Eternal Religion is the best, he handed his kingdom over to his son Bali and went to the Gandham\=adan hill to practise tapasy\=a.

39-40. The prosperous Bali, too, on gaining his kingdom, began to quarrel with the Devas and the war thus went on. Ultimately the powerful Indra and the Devas defeated the Asuras.

41-42. O king! Indra, of unequalled prowess, with the aid of Vi\d{s}\d{n}u, deprived the Daityas of their kingdom. The defeated Daityas took refuge of their family spiritual guide \'Sukr\=ach\=arya and addressed him thus, O Br\=ahmana! You are endowed with your fiery strength of Tapasy\=a and you are now

powerful; why are you not lending your helping hands to your Daitya followers. O foremost of the councillors. If you do not help us and save us, we will not be able to stay in this earth and will soon have to go down to P\=at\=ala.

43-44. Vy\=asa said :-- Thus addressed by the Daityas, the very kind hearted \'Sukr\=ach\=arya said, O Daityas! Do not be afraid; I will protect you by my fire of strength and vigour; and help you with sound counsels and medicines. Be brave and energetic and cast aside your mental agony and sorrow.

45-47. Vy\=asa said :-- O king! The Daityas became fearless under the patronage of \'Sukr\=ach\=arya. The Devas had their spies and knew all about these. They held councils with Indra and settled that before the Daityas had time to dislodge us from our Heaven with the mantra of \'Sukr\=ach\=arya, we will speedily go and attack them. Thus attacked all on a sudden, they will all be slain by us and we will drive them down to the P\=at\=ala.

48. Thus forming their resolves, with fully equipped arms and weapons, they went out of rage to fight with the Daityas and orderd by Indra and aided by Vi\d{s}\d{n}u, they began to kill the Demons.

49. When the Devas were thus slaying the Demons, they got very much terrified and exclaimed ``O Lord! Protect us! Protect us!'' and took the refuge of Sukra.

50. \'Sukr\=ach\=arya, seeing the Daityas very much perplexed and distracted, at once cried aloud out of the influence of his Mantra ``No fear, no fear,'' Then the Devas on seeing \'Sukr\=ach\=arya left the Daityas and fled away to their own places.

Here ends the Tenth Chapter of the Fourth Book of the Mah\=a Pur\=a\d{n}am, \'Sr\={\i} Mad Dev\={\i} Bh\=agavatam, of 18000 verses by Mahar\d{s}\={\i} Veda Vy\=asa on the curse on Vi\d{s}\d{n}u by Bhrigu.



