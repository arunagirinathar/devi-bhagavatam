\chapter{On the narrative of hells}

1-9. N\=ar\=aya\d{n}a said :-- O Devar\d{s}i! San\=atana, the son of Brahm\=a, recites thus in the assembly of the Devas, the glories of the Bhagav\=an Ananta Deva, and worships Him, thus :-- How can one of ordinary sight and understanding grasp the real nature of Brahm\=a, Whose mere Glance enables the Prakriti work Her Gu\d{n}as in the Creation, Preservation and Destruction of this Universe! Him Whose nature has no beginning nor end; Who though One, has created all this Prapa\~ncha (the universe of five elements) as a covering to the \=Atman (the True Self). He has made the Sat and Asat, out of his infinite compassion, this universe, full of cause and effect, visible in His One and only one \'Suddha Sattva nature where even the very powerful lion is imitating his Leel\=a (Pastime), void of all defects, to bring under His control the minds of His own kinsmen. (Note :-- This Ananta Deva is the Ruling Principle in the Fourth Dimensional Space.) To Whom else, then, the persons, desirous of Mok\d{s}a, will take refuge, the mere hearing or reciting Whose Name, in a fallen or a distressed condition, or merely in jest, takes away instantly all the sins! He is upholding the earth with the mountains, oceans, rivers and all the beings as if an atom on his thousand heads. He is infinite. His power knows no decrease in any time. No one can describe his actions even if one had thousand tongues to speak. He is of an infinite strength, of the endless high qualities and of unlimited understanding. Thus staying at the bottom of the earth, the Bhagav\=an Ananta Deva is upholding with ease this earth for her protection, unaided and independent. O Muni! The people get the fruits of their actions and desires as they

want and as they have followed the paths laid down in the \'S\=astras and become accordingly kings, men, deer or birds or other creatures in other states. O N\=arada! This I have described, as you questioned me before, the various and dissimilar fruits of various actions, done according to the dictates of the Dharma and the \'S\=astras.

10. N\=arada said :-- ``O Bhagav\=an! Kindly describe to me now why has the Bhagav\=an created so many diversities, when the Karmas, done by the J\={\i}vas, are the same.''

11-28. N\=araya\d{n}a said :-- O N\=arada! So many different states arise because the \'Sraddh\=as of the doers are so very different. The fruits differ because the \'Sraddh\=as vary, some being S\=attvik, some R\=ajasik and some T\=amasik. If the \'Sraddh\=a be S\=attvik, happiness comes always; if it be R\=ajasik, incessant pain and misery is the result; if it be T\=amasik, misery comes and the loss of the knowledge of good or bad is the result. Thus the fruits differ as the \'Sraddh\=a varies. O Best of Dv\={\i}jas! Thousands and thousands of states occur to a man as the result of their Karmas, done under the influence of the beginningless Avidy\=a (Nescience). O Dv\={\i}jottama! I will now deal in detail with their varieties; hear. Behind this Triloki, below this earth and over the Atala, the Pitris named Agni\d{s}v\=attas and other forefathers live. Those Pitris stay there, and, practising deep Sam\=adhis, they offer always, to their best blessings to their own Gotra (families) respectively. There Yama, the God of the Pitris gives punishment to the dead brought there by His messengers according to their Karmas and faults. By the command of the Bhagav\=an, the Yama, surrounded by his own Ga\d{n}as (persons), judges and does full justice according to the Karmas that they had done and the sins they had committed. He sends always those of his messengers who obey his order and know the Tattva of Dharma, and who are posted to their respective duties to carry out what He commands. The writers of the \'S\=astras describe twenty-one Narakas or hells; others say there are twenty-eight hells. Now hear their names :-- T\=amisra, Andha T\=amisra, Raurava, Mah\=araurava, Kumbh\={\i}p\=aka, K\=alas\=utra, Asipatrak\=anana, \'S\=ukaramukha, Andhak\=upa, Krimibhojana, Taptam\=urti, Samdam\'sa, Vajrakantaka, \'S\=almal\={\i}, Vaitara\d{n}\={\i}, P\=uyoda, Pr\=a\d{n}arodha, Vi\'sasana, L\=al\=abhak\d{s}a, S\=aramey\=adana, Av\={\i}chi, Apahp\=ana, K\d{s}\=arakardama, Rak\d{s}oga\d{n}a, Sambhoja, \'S\=ulaprota, Danda\'s\=uka, Avat\=arodha, Pary\=avartanaka, and S\=uchimukha. These are the twenty-eight Narakas or hells. (N.B. These are 29).

These hells are very tormenting. O Son of Brahm\=a! The embodied beings (j\={\i}vas) suffer these according to their own Karmas respectively.

Here ends the Twenty-first Chapter of the Eighth Book on the narrative of hells in the Mah\=a Pur\=a\d{n}am, \'Sr\={\i} Mad Dev\={\i} Bh\=agavatam, of 18,000 verses, by Mahar\d{s}i Veda Vy\=asa.



