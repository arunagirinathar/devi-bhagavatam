\chapter{On the glory of Tulas\={\i}}

1. N\=arada said :-- How did N\=ar\=aya\d{n}a impregnate Tulas\={\i}? Kindly describe all that in detail.

2-11. N\=ar\=aya\d{n}a said :-- For accomplishing the ends of the Devas, Bhagav\=an Hari assumed the Vai\d{s}\d{n}av\={\i} M\=ay\=a, took the Kavacha from \'Sankhach\=uda and assuming his form, went to the house of Tulas\={\i}. Dundubhis (celestial drums) were sounded at Her door, shouts of

victory were proclaimed and Tulas\={\i} was informed. The chaste Tulas\={\i}, hearing that sound very gladly looked out on the royal road from the window. Then for auspicious observances, She offered riches to the Br\=ahmi\d{n}s; then She gave wealth to the panegyrists (or bards attached to the courts of princes), to the beggars, and the other chanters of hymns. That time Bhagav\=an N\=ar\=aya\d{n}a alighted from His chariot and went to the house of the Dev\={\i} Tulas\={\i}, built of invaluable gems, looking exceedingly artistic and beautiful. Seeing her dear husband before her, She became very glad and washed his feet and shed tears of joy and bowed down to Him. Then She, impelled by love, made him take his seat on the beautiful jewel throne and giving him sweet scented betels with camphor, began to say :-- ``Today my life has been crowned with success. For I am seeing again my lord returned from the battle.'' Then she cast smiling glances askance at him and with her body filled with rapturous joy lovingly asked him the news of the war in sweet words :--

12-13. O Thou, the Ocean of mercy! Now tell me of your heroic valour, how you have come out victorious in war with Mah\=adeva who destroys countless universes. Hearing Tulas\={\i}'s word, the Lord of Lak\d{s}m\={\i}, in the guise of \'Sankhach\=uda, spoke these nectar-like words with a smiling countenance.

14-17. O Dear! Full one Samvatsara the war lasted betwixt us. All the Daityas were killed. Then Brahm\=a Himself came and mediated. Peace, then, was brought about and by the command of Brahm\=a, I gave over to the Devas their rights. When I returned to my home, \'Siva went back to His \'Sivaloka. Thus saying, Hari, the Lord of the world, slept and then engaged in sexual intercourse with her. But the chaste Tulas\={\i}, finding this time her experience quite different from what She used to enjoy before, argued all the time within herself and at last questioned him :--

18-22. Who are you? O Magician! By spreading your magic, you have enjoyed me. As you have taken my chastity, I will curse you. Bhagav\=an N\=ar\=aya\d{n}a, hearing Tulas\={\i}'s words and being afraid of the curse, assumed His real beautiful figure. The Dev\={\i} then saw the Eternal Lord of the Devas before her. He was of a deep blue colour like fresh rain-clouds and with eyes like autumnal lotuses and with playful L\={\i}l\=as equivalent to tens and tens of millions of Love personified and adorned with jewels and ornaments. His face was smiling and gracious; and he wore his yellow-coloured robe. The love-stricken Tulas\={\i}, seeing That Lovely Form

of V\=asudeva, immediately fell senseless; and at the next moment, regaining consciousness, she began to speak.

23-27. O Lord! Thou art like a stone. Thou hast no mercy. By hypocrisy Thou hast destroyed my chastity, my virtue and for that reason didst kill my husband. O Lord! Thou had no mercy; Thy heart is like a stone. So Let Thee be turned into a stone. Those who declare Thee as a saint, are no doubt mistaken. Why didst Thou for the sake of others, kill without any fault, another Bhakta of Thine. Thus speaking, Tulas\={\i} overpowered with grief and sorrow, cried aloud and repeatedly gave vent to lamentations. Seeing her thus very distressed, N\=ar\=aya\d{n}a, the Ocean of Mercy, spoke to her to cheer her up according to the rules of Dharma.

28-102. O Honoured One! For a long time you performed tapasy\=a in this Bh\=arata, to get Me. \'Sankhach\=uda, too, performed tapasy\=a for a long time to get you. By that tapas, \'Sankhach\=uda got you as wife. Now it is highly incumbent to award you also with the fruit that you asked for. Therefore I have done this. Now quit your this terestrial body and assume a Divine Body and marry Me. O R\=ame! Be like Lak\d{s}m\={\i}. This body of yours will be known by the name of Gandak\={\i}, a very virtuous, pure and pellucid stream in this holy land of Bh\=arata. Your hairs will be turned into sacred trees and as they will be born of you, you will be known by the name of Tulas\={\i}. All the three worlds will perform their P\=uj\=as with the leaves and flowers of this Tulas\={\i}. Therefore, O Fair-faced One! This Tulas\={\i} will be reckoned as the chief amongst all flowers and leaves. In Heavens, earth, and the nether regions, and before Me, O Fair One, you will reign as the chief amongst trees and flowers. In the region of Goloka, on the banks of the river Viraj\=a, in the R\=asa circle (the celestial ball dance,) where all amorous sentiments are played in Vrind\=arana forest, in Bh\=a\d{n}d\={\i}ra forest, in Champaka forest, in the beautiful Chandana (Sandal Forests and in the groves of M\=adhav\={\i}, Ketak\={\i}, Kunda, Mallik\=a, and M\=alat\={\i}, in the sacred places you will live and bestow the highest religious merits. All the T\={\i}rthas will reside at the bottom of the Tulas\={\i} tree and so religious merits will accrue to all. O Fair-faced One! There I and all the Devas will wait in expectation of the falling of a Tulas\={\i} leaf. Anybody who will be initiated and installed with the Tulas\={\i} leaves water, will get all the fruits of being initiated in all the sacrifices. Whatever pleasure Hari gets when thousands and thousands of jars filled with water are offered to him, the same pleasure He will get when one Tulas\={\i} leaf will be offered to Him. Whatever fruits are acquired by giving Ayuta cows as presents, those will be also acquired by giving Tulas\={\i} leaves. Especially

if one gives Tulas\={\i} leaves in the month of K\=artik, one gets the fruits same as above mentioned. If one drinks or gets the Tulas\={\i} leaf water at the momentous Time of Death, one becomes freed of all sins and is worshipped in the Vi\d{s}\d{n}u Loka. He who drinks daily the Tulas\={\i} leaf water certainly gets the fruit of one lakh horse sacrifices. He who plucks or culls the Tulas\={\i} leaf by his own hand and holding it on his body, quits his life in a T\={\i}rath, goes to Vi\d{s}\d{n}u Loka. Whoever holds in his neck the garland made up of Tulas\={\i} wood, gets certainly the fruit of horse sacrifices at every step. He who does not keep his word, holding the Tulas\={\i} leaf in his hand, goes to the K\=alas\=utra Hell as long as the Sun and Moon last. He who gives false evidence in the presence of the Tulas\={\i} leaf, goes to the Kumbh\={\i}p\=aka Hell for the life-periods of fourteen Indras. He who drinks or gets a bit of the Tulas\={\i} leaf water at the time of death, certainly goes to Vaikuntha, ascending on a car made up of jewels. Those who pluck or cull the Tulas\={\i} leaves in the Full Moon night, on the twelfth lunar day, on the passing of the sun from one sign to another, the mid-day, or on the twilights, on the night, while applying oil on their bodies, on the impurity periods, and while putting on night dresses, verily eat off the N\=ar\=aya\d{n}a's head. O Chaste One! The Tulas\={\i} leaf kept in the night, is considered sacred. It is considered good in \'Sr\=addha, vow ceremony, in the making over of any gift, in the installation of any image or in worshipping any Deva. Again, the Tulas\={\i} leaf fallen on the ground or fallen in water or offered to Vi\d{s}\d{n}u, if washed out can be used in holy and other purposes. Thus, O Good One! You will remain as tree in this earth and will remain in Goloka as the Presiding Deity thereof and will enjoy daily the sport with Kri\d{s}\d{n}a. And also you will be the Presiding Deity of the river Gandak\={\i} and thus bestow religious merits in Bh\=arata; you will be the wife of the Salt Ocean, which is My part. You are very chaste; in Vaikuntha you will enjoy me as R\=ama lives with Me. And as for Me, I will be turned into stone by your curse; I will remain in India close to the bank of the river Gandak\={\i}. Millions and millions of insects with their sharp teeth will make rings, (the convolutions in the \'S\=alagr\=ama or sacred stones), on the cavities of the mountains there, representing Me. Of these stones, those that have one door (entrance hole), four convolutions, adorned by the garland of wild flowers (having a mark like this) and which look like fresh rain-cloud, are called Lak\d{s}m\={\i} N\=ar\=aya\d{n}a M\=urtis (forms). And those that have one door, four convolutions and look like fresh rain-clouds but no garlands are called Lak\d{s}m\={\i} J\=anardana Chakras (discus). Those that have two doors, four convolutions, and decked with mark like cow's hoof and void of the garland mark are called Raghun\=atha chakras. Those that are very small in size, with two Chakras and look like fresh rain

clouds and void of the garland marks are named V\=amana Chakras. Those that that are very small in size, with two Chakras and the garland mark added, know then to be the \'Sr\={\i}dhara Chakras. These always bring in prosperity to the household. Those that are big, circular, void of garland mark, with two circular Chakras, are known as D\=amodara forms. Those that are mediocre in size, with two Chakras and marked as if struck by an arrow, having marks of arrows and bow-cases are known as Ra\d{n}a-R\=amas. Those that are middling, with seven Chakras, having marks of an umbrella and ornaments, are called R\=ajar\=aje\'svaras. They bestow the royal Lak\d{s}m\={\i} to persons. Those that have twice seven chakras, and are big, looking like fresh rain-clouds are named Anantas. They bestow four fold fruits (Dharma, wealth, desire and liberation). Those that are in their forms like a ring, with two chakras, beautiful, looking like rain-clouds, having cow-hoof marks and of mediocre size are named Madhus\=udanas. Those that have one Chakra are called Sudar\'sanas. Those that have their Chakras hidden are called Gad\=adharas. Those that have two Chakras, looking horse-faced, are known as Hayagr\={\i}vas. O Chaste One! Those that have their mouths very wide and extended, with two Chakras, and very terrible, are known as Narasimhas. They excite Vair\=agyas to all who serve them. Those that have two Chakras, mouths extended and with garland marks (elliptical marks) are called Lak\d{s}m\={\i} Nrisinghas. They always bless the householders who worship them. Those that have two Chakras near their doors (faces), that look even and beautiful, and with marks manifested are known as V\=asudevas. They yield all sorts of fruits. Those that have their Chakras fine and their forms like fresh rain-clouds and have many fine hole marks within their wide gaping facets are called Pradyumnas. They yield happiness to every householder. Those that have their faces of two Chakras stuck together and their backs capacious, are known as Sa\d{n}kar\d{s}a\d{n}as. They always bring in happiness to the householders. Those that look yellow, round and very beautiful are Anirudhas. The sages say, they give happiness to the householder. Where there is the \'S\=alagr\=ama stone there exists \'Sr\={\i} Hari Himself; and where there is Hari, Lak\d{s}m\={\i} and all the T\={\i}rthas dwell there. Worshipping \'S\=alagr\=am \'Sil\=a, destroys the Brahmahaty\=a (killing a Br\=ahmi\d{n}) and any other sin whatsoever. In worshipping the \'S\=alagr\=ama stone looking like an umbrella, kingdoms are obtained; in worshipping circular \'Sil\=as,
great prosperity is obtained; in worshipping cart-shaped stones, miseries arise; and in worshipping stones, whose ends look like spears (\'S\=ulas), death inevitably follows. Those whose facets are distorted, bring in poverty; and yellow stones bring in various evils and afflictions. Those whose Chakras look broken, bring in diseases; and those whose Chakras

are rent asunder bring in death certainly. Observing vows, making gifts, installing images, doing \'Sr\=addhas, worshipping the Devas, all these become highly exalted, if done before the \'S\=alagr\=ama \'Sil\=a. One acquires the merits of bathing in all the T\={\i}rthas and in being initiated in all the sacrifices, if one worships the S\=alagr\=ama \'Sil\=a. What more than this, that the merits acquired by all the sacrifices, all the T\={\i}rthas, all vows, all austerities and reading all the Vedas are all acquired by duly worshipping by the holy \'S\=alagr\=ama \'Sil\=a. He who performs his Abhi\d{s}'eka ceremony always with \'S\=alagr\=ama water (being sprinkled with \'S\=alagr\=ama water at the initiation and Installation ceremonies), acquires the religious merits of performing all sorts of gifts and circumambulating the whole earth. All the Devas are, no doubt, pleased with him who thus worships daily the \'S\=alagr\=ama. What more than this, that all the T\={\i}rthas want to have his touch. He becomes a J\={\i}vanmukta (liberated while living) and becomes very holy; ultimately he goes to the region of \'Sr\={\i} Hari and remains in Hari's service there and dwells with him for countless Prakritic dissolutions. Every sin, like Brahm\=a Haty\=a, flies away from him as serpents do at the sight of Garuda. The Dev\={\i} Vasundhar\=a (the Earth) becomes purified by the touch of the dust of his feat. At his birth, all his predecessors (a lakh in number), are saved. He who gets the \'S\=alagr\=ama \'Sil\=a water during the time of his death, he is freed of all his sins and goes to the Vi\d{s}\d{n}u Loka and gets Nirv\=a\d{n}a; he becomes freed entirely from the effects of Karma and he gets, no doubt, dissolved and diluted for ever in (the feet of) Vi\d{s}\d{n}u. He who tells lies, holding \'S\=alagr\=ama in his hands, goes to the Kumbh\={\i}p\=aka Hell for the life-period of Brahm\=a. If one does not keep his word, uttered with the \'S\=alagr\=ama stone in his hand, one goes to the Asipatra Hell for one lakh manvantaras. He who worships the \'S\=alagr\=ama stone without offering Tulas\={\i} leaves on it or separates the Tulas\={\i} leaves from the stone, will have to suffer separation from his wife in his next birth. So if one does not offer the Tulas\={\i} leaves in the conchshell, for seven births he remains without his wife and he becomes diseased. He who preserves the \'S\=alagr\=ama stone, the Tulas\={\i} and the conchshell, in one place, becomes very learned and becomes dear to N\=ar\=aya\d{n}a. Look! He who casts his semen once in his wife, suffers intense pain, no doubt, at each other's separation. So you become dear to \'Sankhach\=uda for one Manvantara. Now, what wonder! That you will suffer pain, at his bereavement. O N\=arada! Thus saying, \'Sr\={\i} Hari desisted. Tulas\={\i} quitted her mortal coil and assumed a divine form, began to remain in the breast of \'Sr\={\i} Hari like \'Sr\={\i} Lak\d{s}m\={\i} Dev\={\i}. Hari also went with her to Vaikuntha. Thus Lak\d{s}m\={\i}, Sarasvat\={\i}, Gang\=a, and Tulas\={\i}, all the four came so

very dear to Hari and are recognised as \=I\'svar\={\i}s. On the other hand, the mortal coil of Tulas\={\i}, no sooner quitted by Tulas\={\i}, became transformed into the river Gandak\={\i}. Bhagav\=an Hari, too, became also converted into a holy mountain, on the banks thereof, yielding religious merits to the people. The insects cut and fashion many pieces out of that mountain. Of them, those that fall into the river, yield fruits undoubtedly. And those pieces that fall on the ground become yellow coloured; they are not at all fit for worship. O N\=arada! Thus I have spoken to you everything. What more do you want to hear now? Say.

Here ends the Twenty-fourth Chapter of the Ninth Book on the glory of Tulas\={\i} in the Mah\=apur\=a\d{n}am \'Sr\={\i} Mad Dev\={\i} Bh\=agavatam of 18,000 verses by Mahar\d{s}i Veda Vy\=asa.



