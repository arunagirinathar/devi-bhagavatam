\chapter{On the earnestness of Hari\'schandra to pay off the Dak\d{s}i\d{n}\=a}

1-4. Hari\'schandra said :-- ``O Muni! I will not take my food until I pay you your Dak\d{s}i\d{n}\=a in gold; know this to be my resolve; therefore O One of good vows! Discard all your anxieties for Dak\d{s}i\d{n}\=a. I am the King of the Solar dynasty; especially since the time I have completed my R\=ajas\=uya Sacrifice, I give to everyman whatever he desires. So, O Lord! How can it be possible that I will not give what I have voluntarily promised myself? O Best of Dv\={\i}jas! I will certainly pay off your debt. I must give you the gold as you desire; be calm and patient; but you will have to wait one month; and on getting the money I will pay it off to you.''

5-8. Vi\'sv\=amitra said :-- ``O King! Kingdom, treasury and strength are the three sources of income; but you are now deprived of all these. Whence, now, do you expect to get gold? O King! Vain are your hopes to get money; what am I to do now? You are now wealthless and how can I, out of greed, give you trouble? O King! Better say `I will not be able to give you Dak\d{s}i\d{n}\=a,' and I will then quit my strong expectation and go away as I like. And you, too, can think that you have no gold, so how can you give money and so you can go wherever you like with your wife and son.''

9-20. Vy\=asa said :-- O King! Hearing these words of the Muni, at his time of departure, the King said :-- ``O Br\=ahma\d{n}a! Be patient and I will certainly give you your Dak\d{s}i\d{n}\=a. O Dv\={\i}ja! My wife, son and I myself are all healthy; so selling these, I will give you the money; there is no doubt in this. O Lord! Kindly enquire whether there is anyone who can purchase us and I will agree to become the slave with my wife and son. O Muni! You can sell all of us and the price you get, you can take two and a half loads of gold out of that and be pleased.'' Thus saying, the King went to Benares where \'Sa\d{n}kara was staying with his dear consort Um\=a. The King saw the beautiful city, the sight of which makes one's heart dance with more joy and he said that he had become blessed. Then he went to the banks of the Bh\=agirath\={\i} and bathed

in the Ganges and offered peace-offerings (Tarpa\d{n}) to the Devas and the Pitris and completing the worship of his I\d{s}ta Deva (his own Deity) looked around where he would go. The King entering into the beautiful city of Benares began to think that no human being is protecting this city but \'Siva Himself is protecting it. So if he lives there, he would not be living in a city which has been given away by him to Vi\'sv\=amitra. The King, then, distressed much with pain and trouble and being very much bewildered, began to journey on foot with his wife and son and entered into the city and placed his confidence. At this moment he saw the Muni Vi\'sv\=amitra, wanting Dak\d{s}i\d{n}\=a and humbly bowed down and spoke with folded palms :-- ``O Muni! My dear wife, my son and I myself are living here; you can take any of us and have your work done; or say what other work we will have to do for you.''

21. Vi\'sv\=amitra said :-- ``You promised that you would pay Dak\d{s}i\d{n}\=a at the end of one month; and today that one month is completed; if you remember, then give me the Dak\d{s}i\d{n}\=a.''

22. The King said :-- ``O Br\=ahma\d{n}a! You are wise and are endowed with the power of tapas (asceticism); as yet one month is not complete; still half a day is remaining; wait till then; and no longer.''

23-27. Vi\'sv\=amitra said :-- ``O King! Let it be. I will come again and if you do not give me then, I will curse you. Thus saying Vi\'sv\=amitra went away. The King then thought within himself how be would pay him back what he had promised. There is no influential friend of mine in this Benares city who can help me with money; where then can I get the requisite money. I am a K\d{s}attriya. Pratigraha (begging or accepting any gift) is forbidden to me and how can I beg or accept any gift! According to the code of Dharma, the offering of sacrifices (on one's own behalf), studying, and giving are the three duties ordained to a King. And if I die not paying a Br\=ahmi\d{n}'s Dak\d{s}i\d{n}\=a, I will be polluted with the sin of stealing a Br\=ahmi\d{n}'s property and I will then be born a worm or will became a Preta. So to sell myself (and pay off the debts) is better than this.''

28-33. S\=uta said :-- O \d{R}i\d{s}is! When the King was thus thinking humbly with his face bent downwards, and in a distracted state of mind, his wife spoke to him with tears in her eyes and in a voice, choked with feelings :-- ``O King! Discard all cares and keep your own Dharma, Truth. He who is divorced from Truth is forsaken like a Preta. O Best of all men! To keep one's Truth is one's Dharma; there is no other Dharma superior to it; so the sages declare. He whose

words turn out false, his Agnihotra, study, and gifts and all action, become fruitless. Truth is very much praised in the Dharma \'S\=astra and this Truth raises up and saves the virtuous souls. Similarly falsehood, no doubt, drags a vicious man to hell. The King Yay\=ati performed the Horse sacrifice, and the R\=ajas\=uya sacrifice and went to Heavens but once he spoke falsely and so he was dislodged from the Heavens.''

34. The King said :-- ``O Thou, going like an elephant! I have my son who will multiply my line; speak out what Thou wishest to say.''

35. The Queen said :-- ``O King! The wives are meant for sons (your having me has been fulfilled as there is your son). So sell me for the money value and give the Dak\d{s}i\d{n}\=a to the Br\=ahmi\d{n}. Let you not deviate from the Truth.''

36-45. Vy\=asa spoke :-- Hearing this, the King fainted. Afterward regaining consciousness, he wept with a grievous heart. O gentle One! What you have uttered just now has caused me much pain; am I such a Sinner as to forget entirely all your conversations and your sweet smiles! Alas! O Sweet-smiling One! You ought not to speak such words. O Fair One! How have you been able to utter these harsh words not fit to be spoken! Speaking thus, the King became impatient at the idea of selling his wife and fainted and fell to the ground. Seeing him fainted and lying flat on the ground, the Queen became grievously hurt and spoke with great compassion. O King! Whose evil have you done that you have fallen into this calamity? Alas! He who is accustomed to sleep in a room adorned with carpets is today like a humble man, sleeping on the ground! The King who gave crores and crores of golden mohurs to the Br\=ahmi\d{n}s, that same King, my husband is lying now on the ground! Alas! What a painful thing! O Fate! What has this King done to you that You have thrown this Indra and Upendra like King in this dire calamity! Thus saying, the beautiful queen (of good hips) very much grieved by the sight of her husband's pain fell down unconscious on the ground. Then the boy prince, seeing father and mother both senseless, lying on the ground, became very much troubled, and, becoming hungry, cried, ``O Father! O Father! I am very hungry; give me food to eat; O Mother! O Mother! My tongue is being parched; give me food to eat,'' and the boy began to weep repeatedly.

Here ends the Twentieth Chapter of the Seventh Book on the earnestness of Hari\'schandra to pay off the Dak\d{s}i\d{n}\=a in the Mah\=apur\=a\d{n}am \'Sr\={\i} Mad Dev\={\i} Bh\=agavatam of 18,000 verses, by Mahar\d{s}i Veda Vy\=asa.



