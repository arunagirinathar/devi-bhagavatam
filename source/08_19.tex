\chapter{On the narrative of the Atala, etc.}

1-32. N\=ar\=aya\d{n}a said :-- O Vipra! In the first beautiful region Atala, the exceedingly haughty son of the D\=anava M\=ay\=a, named Bala, is living. He has created the ninety-six M\=ay\=as. All the requisites of the inhabitants are obtained by them. The other M\=ay\=avis know one or

two of these. None of them are capable to know all, as they are exceedingly difficult to be carried out. When this powerful Bala yawned, the three classes of women were produced, fascinating to all the Lokas. They were named Pum\'schal\={\i} (or unchaste woman) Svairi\d{n}\={\i}, (an adultress) and K\=amin\={\i} (a lovely women). When any man, beautiful and lovely to them, enters into their Atala region, they, with the help of the H\=ataka sentiment (of love), generate in him, while in solitude, the power to enjoy (copulate) and with their sweet smiles and amorous lovely looks and with great caution embrace him thoroughly and begin to converse with him and with amorous gestures and postures, and thus please him well. When the people enjoy this H\=atakarasa, they think often and often, that they themselves have become gods, they have become Siddhas and powerful like Ayuta elephants; being blind with vanity and finding them endowed with powers and prosperity, they think themselves so repeatedly and constantly. O N\=arada! Thus the position in Atala has been described. Now hear, the description of the second region Vitala. Vitala is situated below the earth. There the Bhagav\=an Bhava, worshipped by all the Devas, has assumed the name of H\=atake\'svara and is staying there coupled with Bhav\=an\={\i}, surrounded by His attendants specially for the increase of the creation of Brahm\=a. The river H\=ataki flows there and has Her origin from the essences (Semen virile) of them both. Fire, augmented by the help of the wind, begins to drink it. When the Fire leaves that, making a Phutk\=ara noise (i.e., blowing out air through the mouth), the gold, named H\=ataka, is created. This gold is very much liked by the Daityas. The Daitya women use this gold always for their ornaments. Below Vitala is Sutala. It is reckoned as of some special importance. O Muni! The highly meritorious Bali, the son of Virochana lives here. The Bhagav\=an V\=asudeva, brought down this Bali into Sutala, for the welfare of Indra. He assumed the body of Trivikrama and gave to Bali all the wealth of the three Lokas, all the Lak\d{s}m\={\i} went to him and installed him in the position of the Lord of the Daityas. What more can be said than this, that what prosperity, wealth and riches that Indra could not obtain, that \'Sr\={\i} Lak\d{s}m\={\i} Dev\={\i} Herself has followed Bali. Bali, as the Lord of Sutala, has become entirely fearless, remains here upto this day and is worshipping V\=asudeva. O N\=arada! It is said by the high-minded persons that when V\=asudeva Himself, the Controller of all, appeared as a beggar, Bali gave him land, and, therefore, on account of making gift to a good person, he acquired so much prosperity. But this cannot be reasonable. For, it is not at all reasonable to cast the effects of making this gift on N\=ar\=aya\d{n}a, O N\=arada! Who is Self-manifest by His own Extraordinary

Glory and Who is Himself filled with all Ai\'svarya (prosperity) and Who can bestow the Highest Goal of life and other requirements of men. This N\=ar\=aya\d{n}a is the Deva of the Devas; if anybody takes His name, when in the greatest distress, he gets himself immediately freed from the Gu\d{n}as, the cause of bondage due to his Karma. All persons perform many Yogas and follow the paths advised by the S\=amkhya method, with their minds directed to the All-Controller Bhagav\=an, to abandon all sorts of troubles and miseries. O N\=arada! Know that the Bhagav\=an does not shew us His Favour when he gives us greatest wealth and prosperity. For the wealth and riches are the offspring of M\=ay\=a and the source of all worries, miseries and mental troubles; and one is liable to forget the Bhagav\=an when one gets such a wealth. The Bhagav\=an is pervading all this universe and is full of wisdom; and He is seeing always all the ways and means; He took away, in the way of begging, rather cheated all that Bali had, leaving only his body; and at last, finding no other means, fastened him by the Varu\d{n}a P\=a\'sa (noose), threw him in the middle of the mountain cleft (cave) and then has stationed Himself at his door as a Door-keeper. Once, out of his extreme devotion, Bali did not care at all for his difficulties, troubles, or miseries. Rather he gave out that Indra, whose minister is Brihaspati had acted very foolishly. For when the Bhagav\=an becomes very graciously pleased, he wanted from Him ordinary wealth. But what will the wealth of the Trilokas avail? It is a quite insignificant thing. Surely, He is an illiterate and stupid brute who, for mere wealth, leaves the Bhagav\=an, Who is the Fountain of all Good Wishes to the Humanity. My grandfather Prahl\=ada, who was highly fortunate, who was devoted to the God and who was always ready to do good to others, he did not ask for any other thing than the servantship of God (the D\=asya Bh\=ava). When his powerful father died, the Bhagav\=an wanted to give him unbounded wealth; but the Bh\=agavata (devoted) Prahl\=ada did not want that. None of us, who are marked with so many deficiencies can know the nature of the Bhagav\=an V\=asudeva, Whose omnipotence cannot be compared and all these manifested worlds are but His Up\=adhis (adjuncts, limitations). O Devar\d{s}i! Thus Bali, the Lord of Daityas, the highly respected and renowned in all the Lokas, is reigning in Sutala. Hari Himself is his Door-keeper. Once the King R\=ava\d{n}a, the source of torment to all the people, went out to conquer the whole world; and when he entered Sutala, that Hari, ever ready to show Grace to His devoted, threw him at a distance of one Ayuta Yoyanas by the toe of His foot. Thus by the grace of the Devadeva V\=asudeva, Bali is reigning in Sutala, and enjoying all sorts of pleasures, without any equal anywhere.

Here ends the Nineteenth Chapter of the Eighth Book on the narrative of the Atala, etc., the P\=at\=alas in \'Sr\={\i} Mad Dev\={\i} Bh\=agavatam, the Mah\=a Pur\=a\d{n}am, of 18,000 verses, by Mahar\d{s}i Veda Vy\=asa.



