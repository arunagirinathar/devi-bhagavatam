\chapter{On the birth of Gaur\={\i}, the seats of the Deity, and the distraction of \'Siva}

1-12. Vy\=asa said :--O King! They went to the forest and fixed their seats on the slope of the Him\=alay\=an mountain and engaged them-

selves in repeating silently the seed Mantra of Mah\=a M\=ay\=a and thus practised their austerities. O King! One hundred thousand years passed in the meditation of the Par\=a \'Sakti. The Dev\={\i}, pleased, became visible to them. Her form was three-eyed, and of the form of Existence, Intelligence and Bliss (Sachhid\=ananda); She was filled with mercy. In Her one hand there was the noose, in another hand, goad; in another hand there was the sign bidding her devotees discard all fear, and in the other hand She was ready to offer boons. The good-natured Munis, seeing this Form of the World Mother began to praise Her. ``O Dev\={\i}! Thou art existing separately in every gross body; we bow down to Thee. Thou art existing wholly (cosmically) in all the gross bodies; we bow down to Thee. O Parame\'svar\={\i}! Thou art existing separately in every subtle body; we bow down to Thee; Thou art existing universally in all the subtle bodies; we bow down to Thee, Thou art existing separately in all the causal bodies wherein all the Linga Dehas (subtle bodies) are interwoven; we bow down to Thee. Thou art existing universally in all the causal bodies; we bow down to Thee. Thou art of the nature of the unchangeable Brahm\=a, the receptacle of all the J\={\i}vas and thus residest in all the bodies; so we bow down to Thee. Thou art of the nature of \=Atman, the Goal of all the beings; we bow again and spin to Thee.'' Thus the pure-natured Dak\d{s}a and the other Munis praised Her with voice, choked with feelings of intense devotion and bowed down to Her feet. Then the Dev\={\i}, pleased, spoke to them in a cuckoo voice. ``O Highly Fortunate Ones! I am ever ready to grant boons; so ask what you desire.'' O King! Hearing thus, they asked that Hari and Hara both regain their former natural states and be united respectively with their \'Saktis, Lak\d{s}m\={\i} and Gaur\={\i}. Dak\d{s}a again asked :-- ``O Dev\={\i}! Let your birth be in my family. O Mother! I will, no doubt, consider myself as having then realised the fulfilment of my life. So, O Parame\'svar\={\i}! Speak by Thy own mouth how Thy worship, Japam, meditation will be conducted as well the various fit places where they would be performed.''

13-16. The Dev\={\i} said :-- ``The insult shown towards my \'Saktis has led to this calamitous state of Hari and Hara. So they should not repeat such crime. Now, by My favour, they will regain their health and, of the two \'Saktis, one will be born in your family and the other will take Her birth in the K\d{s}iroda S\=agara, the ocean of milk. Hari and Hara will get back their \'Saktis, when I will send them the chief Mantra. The chief Mantra of Mine is the said Mantra of M\=ay\=a; this is always sweet to Me; so worship this Mantra and make Japam of this. The Form that you are seeing before you, this is My Bhuvane\'svar\={\i} form (that of the Goddess

of the Universe), or worship My Vir\=at (cosmic) form; or Sachchid\=ananda form. The whole world is my place of worship; so you can meditate on Me and worship Me always and in all places.''

17-23. Vy\=asa said :-- When the Bhuvane\'svar\={\i} Dev\={\i} living in the Ma\d{n}i Dv\={\i}pa thus giving Her reply, went away, Dak\d{s}a and other Munis all went to Brahm\=a and informed him with great earnestness of everything that happened. O King! Thus Hari and Hara both became devoid of their haughtiness and got back their previous natures by the Grace of the Supreme Deity and were thus enabled to perform their functions as before. Then, on a certain time, the Dev\={\i} Bhagavat\={\i}, the Fiery Nature of the Par\=a \'Sakti, took Her birth in the house of the Praj\=apati Dak\d{s}a. O King! Everywhere in the Trilokas, great festivities were held. All the Devas became glad and showered flowers. The Dundubhis of the Devas were sounded by the hands and made very grave sounds. The pure-minded saints were gladdened; the Sun's rays looked purer and cleaner; the rivers were elated with joy and began to flow in their channels. When the World-auspicious Dev\={\i}, the Destroyer of the birth and death of the J\={\i}vas took Her birth, everything looked propitious. The wise Munis named Her ``Sat\={\i}'' as She was of the nature of Par\=a Brahm\=a and Truth Herself. The Praj\=apati Dak\d{s}a handed over the Dev\={\i}, who was before the \'Sakti of Mah\=adeva, to that Deva of the Devas, Mah\=adeva. Due to the misfortune of Dak\d{s}a, the daughter of Dak\d{s}a burnt Herself in a blazing fire.

24-25. Janamejaya said :-- ``O Munis! You have made me now hear a very inauspicious word. How can such a great thing of the nature of the Highest Intelligence come to be burnt up in a fire! The mere recollecting of Whose Name dispels the terrible danger of the burning up by the fire of Sams\=ara, how can She be burnt up by fire, I am extremely eager to hear; kindly describe that to me in detail.''

26-37. Vy\=asa said :-- O King! Hear. I am describing to you the ancient history of the burning of Sat\={\i}. Once on a time, the famous \d{R}i\d{s}i Durv\=as\=a went to the bank of the river Jamb\=u and saw the Dev\={\i} there. There he remained with his senses controlled and began to repeat silently the root Mantra of M\=ay\=a. Then the Goddess of the Immortals, the Bhagavat\={\i} was pleased and gave the Muni a beautiful garland as Her Pras\=ada that was on Her neck, that emitted the sweet fragrance of Makaranda (juice of flowers; Jasamine). Whereon the bees were about to cluster. The Mahar\d{s}i took it quickly and placed it on his head. He then hurriedly went to see the Mother to the place where Sat\={\i}'s Father,

the Praj\=apati Dak\d{s}a was staying and bowed down to the feet of the Sat\={\i}. The Praj\=apati then asked him :-- ``O Lord! Whose extraordinary garland is this? How have you got this enchanting garland, rare to the mortals on this earth!'' The eloquent Mahar\d{s}i Durv\=as\=a then spoke to him with tears of love flowing from his eyes :-- ``O Praj\=apati! I have got this beautiful garland that has no equal, as the Pras\=ada (favour) of the Dev\={\i}.'' The Praj\=apati asked that garland then from him. He, too, thinking that there was nothing in the three worlds that cannot be given to the devotee of the \'Sakti, gave that garland to the Praj\=apati. He took that on his head; then placed it on the nice bed that was prepared in the bed-room of the couple. Being excited by the sweet fragrant smell of that garland in the night, the Praj\=apati engaged in a sexual intercourse! O King! Due to that animal action, the bitter enmity arose in his mind towards \'Sankara and His Sat\={\i}. He then began to abuse \'Siva. O King! For that offence, the Sat\={\i} resolved to quit her body that was born of Dak\d{s}a, to preserve the prestige of the San\=atan Darma of devotion to Her Husband and burnt Her body by the fire arising out of Yoga.

38. Janamejaya said :-- ``O Muni! What did Mah\=a Deva do, thus pained by the bereavement from His consort dearer than His life, when the Sat\={\i}'s body was thus consumed.''

39-50. Vy\=asa said :-- O King! I am unable to describe what happened afterwards. O King! Out of the fire of anger of \'Siva, the Pralaya seemed to threaten the three worlds. V\={\i}rabhadra came into existence with hosts of Bhadra K\=al\={\i}s, ready to destroy the three worlds. Brahm\=a and the other Devas took refuge to \'Sankara. Though Mah\=adeva lost everything on Sat\={\i}'s departure, He, the Ocean of Mercy, destroyed the sacrifice of Dak\d{s}a, cut off his head and instead placed the head of a goat, brought him back to life and thus made the Gods free from all fears. He, the Deva of the Devas, then became very much distressed and going to the place of sacrifice, began to weep in great sorrow. He saw that the body of the Intelligent Sat\={\i} was being burnt in the fire of the Chit\=a. He cried aloud :-- Oh my Sat\={\i}! Oh My Sat\={\i}! And taking Her body on His neck, began to roam in different countries, like a mad man. Seeing that, Brahm\=a and the other Devas became very anxious and Bhagav\=an Vi\d{s}\d{n}u cut off the body to pieces by His arrows. Wherever the parts fell, \'Sankara remained there in so many different forms. He then said to the Devas :-- Whoever will worship, with deep devotion in these places, the Bhagavat\={\i}, will have nothing left unattained. The Highest Mother will remain close to them there. The persons that will make Pura\'schara\d{n}a (the repetition) of the Mantrams, especially the M\=ay\=a V\={\i}ja (the root Mantra of M\=ay\=a), their Mantrams will become, no

doubt, fructified and become incarnate. O King! Thus saying, the Mah\=adeva, being very much distressed for Sat\={\i}'s departure, passed His time in those places, making Japam, Dhy\=anam and taking to Sam\=adhi.

51-52. Janamejaya said :-- Where, in what places the several parts of the Sat\={\i} fell? What are the names of those Siddhap\={\i}thas? And what is their number? Kindly describe these in detail, O Great Muni! No doubt I will highly consider myself blessed by hearing these words from your blessed mouth.

53-102. Vy\=asa said :-- O King! I will now describe those P\={\i}thas (Sacred places), the mere hearing of which destroys all the sins of men. Hear. I describe duly those places where the persons desiring to get lordly powers and to attain success ought to worship and meditate on the Dev\={\i}. O Mah\=ar\=aja! The face of Gaur\={\i} fell in K\=a\'s\={\i}; She is well known there by the name Vi\'s\=al\=aks\={\i}; that which fell in Naimi\d{s}\=ara\d{n}ya became known by the name of Linga Dh\=ari\d{n}\={\i}. This Mah\=a M\=ay\=a is known in Pray\=ag (Allahabad) by the name of Lalit\=a Dev\={\i}; in Gandha M\=adan, by the name of K\=amuk\={\i}; in the southern M\=anasa, by Kumud\=a; in the northern M\=anasa, by Visvak\=am\=a, the Yielder of all desires; in Gomanta, by Gomat\={\i} and in the mountain of Mandara, She became known by the name of K\=amach\=arin\={\i}. The Dev\={\i} is known in Chaitraratha, by the name of of Madotkat\=a; in Hastin\=apura, by Jayant\={\i}; in K\=anyakubja by the name of Gaur\={\i}; in the Malaya Mountain, by Rambh\=a; in the Ek\=amrap\={\i}tha, by K\={\i}rtimat\={\i}, in Vi\'sve, by the name of Vi\'sve\'svar\={\i}; in Pu\d{s}kara, by the name of Puruh\=ut\=a. She is known as Sanm\=arga D\=ayin\={\i} in the Ked\=ara P\={\i}tha; as Mand\=a, in the top of the Him\=alay\=as; and as Bhadrakar\d{n}ik\=a in Gokar\d{n}a. She is known as Bhav\=an\={\i} in Sthane\'svara, as Vilvapatrik\=a in Vilvake; as M\=adhavi in \'Sr\={\i}\'saila; as Bhadr\=a in Bhadre\'svara. She is known as Jar\=a in Var\=aha \'Saila; as Kamal\=a in Kamal\=alaya; as Rudra\d{n}\={\i} in Rudra Kot\={\i}; as K\=al\={\i} in K\=alanjara; She is known as Mah\=a Dev\={\i} in \'S\=alagr\=ama, as Jalapriy\=a in \'Sivalingam; as Kapil\=a in Mah\=ali\d{n}gam, as Mukute\'svar\={\i} in M\=akota. As Kumar\={\i} in M\=ay\=apur\={\i}, as Lalit\=ambik\=a in Sant\=an\=a; as Mangal\=a in Gay\=a K\d{s}etra, as Vimal\=a in Puru\d{s}ottama. As Utpal\=ak\d{s}\={\i} in Sahasr\=ak\d{s}a; as Mahotpal\=a in Hirany\=ak\d{s}a; as Amogh\=aks\={\i} in the Vip\=as\=a river; as P\=atal\=a in Pundra Vardhana. As N\=ar\=aya\d{n}\={\i} in Sup\=ar\'sva, as Rudra Sundar\={\i} in Trik\=uta; as Vipul\=a Dev\={\i} in Vipul\=a; as Kaly\=an\={\i} in Malay\=achala. As Ekav\={\i}r\=a, in Sahy\=adri; as Chandrik\=a in Hari\'schandra; as Rama\d{n}\=a in R\=ama T\={\i}rtha; as Mrig\=avat\={\i} in the Yamun\=a. As Kotiv\={\i} in

Kotat\={\i}rtha; as Sugandh\=a in M\=adhavavana; as Trisandhy\=a in the God\=avar\={\i}; as Ratipriy\=a in Gang\=adv\=ara. As \'Subh\=anand\=a in \'Siva Ku\d{n}dam, as Nandin\={\i} in Dev\={\i}k\=atata; as Rukmi\d{n}\={\i} in Dv\=aravat\={\i}; as R\=adh\=a in Brind\=avana. As Devak\={\i} in Mathur\=a; as Parame\'svar\={\i} in P\=at\=ala; as S\={\i}t\=a in Chitrakuta; as Vindhy\=adhiv\=asin\={\i} in the Vindhy\=a range. O King! As Mah\=alak\d{s}m\={\i} in the sacred place of Karav\={\i}ra, as Um\=a Dev\={\i} in Vin\=ayaka; as \=Arogy\=a in Vaidy\=an\=atha; as Mahe\'svar\={\i} in Mah\=ak\=ala. As Abhay\=a in all the U\d{s}\d{n}a t\={\i}rthas, as Nitamb\=a in the Vindhy\=a mountain; as M\=andav\={\i} in M\=andavya; as Sv\=ah\=a in M\=ahe\'svar\={\i}p\=ura. As Prachand\=a in Chhagalanda, as Chandik\=a in Amarakantaka; as Var\=aroh\=a in Some\'svara; as Puskar\=avat\={\i} in Prabh\=asa. As Devam\=at\=a in Sarasvat\={\i}; as Par\=av\=ar\=a in Samudrtata; as Mah\=abh\=ag\=a in Mah\=alay\=a, as Pingale\'svar\={\i} in Payo\d{s}\d{n}\={\i}. As Simhik\=a in Krita\'saucha; as Ati\'s\=ank\=ar\={\i} in K\=artika; as Lol\=a in Utpal\=avartaka; as Subhadr\=a in \'So\d{n}a Sangam. As the Mother Lak\d{s}m\={\i} in Siddhavana; as Anang\=a in Bh\=arat\=a\'srama; as Vi\'svamukh\={\i} in J\=alandhara; as T\=ar\=a in the Ki\d{s}kindhya mountain. As Pust\={\i} in Devad\=aru Vana; as Medh\=a in K\=a\'sm\={\i}ramandalam; as Bh\={\i}m\=a in Him\=adri; as Tust\={\i}i in Vi\'sve\'svara K\d{s}etra. As \'Suddh\={\i} in Kap\=alamochana; as M\=at\=a in K\=ay\=avaroha\d{n}a; as Dhar\=a in \'Sankhoddh\=ara; as Dhrit\={\i} in Pind\=araka; as Kal\=a in Chandrabh\=ag\=a river; as \'Sivadh\=ari\d{n}\={\i} in Achchoda; as Amrit\=a in Ven\=a; as Urva\'s\={\i} in Vadar\={\i}. As medicines in Uttara Kuru; as Ku\'sodak\=a in Ku\'sadv\={\i}pa; as Manmath\=a in Hemak\=uta; as Satyav\=adin\={\i} in Kumuda. As Vandan\={\i}y\=a in A\'svattha; as Nidhi in the Vai\'srava\d{n}\=alaya; as G\=ayatr\={\i} in the mouth of the Vedas; as P\=arvat\={\i} near to \'Siva. As Indr\=a\d{n}i in the Devalokas; as Sarasvat\={\i} in the face of Brahm\=a; as Prabh\=a (lustre) in the Solar disc; as Vai\d{s}\d{n}av\={\i} with the M\=atrik\=as. She is celebrated as Aru\d{n}dhat\={\i} amongst the Sat\={\i}s, the chaste women and as Tilottam\=a in the midst of the R\=am\=as. Again this Mah\=adev\={\i} of the nature of the Great Intelligence (Samvid) is always existent in the form of \'Sakti named Brahmakal\=a in the hearts of all the embodied beings. O Janamejaya! Thus I have mentioned to you the one hundred and eight p\={\i}thas (sacred places or seats of the Deity) and as many Dev\={\i}s. Thus are mentioned all the seats of the Dev\={\i}s and along with that, the chief places in India (the world). He who hears these excellent one hundred and eight names of the Dev\={\i} as well as Her seats, gets himself freed from all sins and goes to the Loka of the Dev\={\i}. O Janamejaya! His heart gets purified and is rendered blessed, no doubt, who duly makes j\=atr\=a (sojourn) to all these seats of the Deity, performs \'Sr\=addhas, offers peace-offerings to the Pitris and worships with the highest devotion the Goddess and asks frequently the pardon of the World Mother. O King! After worship, one should

feed the Br\=ahma\d{n}as, well dressed virgins (Kum\=ar\={\i}s) and Vatukas with good eatables. All the tribes whether they be Ch\=and\=alas, know them all to be of the nature of the Dev\={\i} and therefore they should be worshipped. Never one is to accept any donation or gifts (Pratigrahas) in these seats of the Dev\={\i}. The saintly persons should make Pura\d{s}chara\d{n}as (repeat the names of their own deities, attended with burnt offerings, oblations, etc.) of their own Mantrams with all their might in all these places and should never be miserly in their expenses on this account. He who starts to these sacred places, with devoted hearts filled with love, finds his Pitris in the higher and greater Brahm\=a Loka for one thousand Kalpas and he gets the highest knowledge, crosses the ocean of the world and becomes free. Many a people have attained success by repeating these one hundred and eight names of the Deity. Any place wherein are kept those names, embodied in a book, becomes free from such dangers as plague, cholera or any misapprehensions from planetary Deities and so forth. Nothing remains to be attained by these persons who repeat these one hundred and eight names. That man, devoted to the Dev\={\i}, certainly attains blessedness. That saintly person becomes of the nature of the Dev\={\i}. The Devas bow down and worship him when they behold him! What then need be said that the saints would worship him! The Pitris become pleased and get their good ends when these one hundred and eight names are read with devotion. These places are, as it were, Intelligence personified (Chinmaya) and places ready to yield freedom from bondage. Therefore, O King! Intelligent men should take their shelter in these places. O King! Whatever secrets and other deeper secrets about the Great Goddess you asked to know from me, I described to you. What more do you want to hear. Say.

Here ends the Thirtieth Chapter of the Seventh Book on the birth of Gaur\={\i}, the seats of the Deity, and the distraction of \'Siva in the Mah\=apur\=a\d{n}am \'Sr\={\i} Mad Dev\={\i} Bh\=agavatam, of 18,000 verses, by Mahar\d{s}i Veda Vy\=asa.

Note :-- The number one hundred and eight is a holy number, got by taking the half of 216,000, the number of breaths inhaled by a child in the womb who promises to take the name of God at his every breath or by taking one-eighth of 864,000, the number of seconds in a day. The two zeros are then dropped. Thus the number signifies the one who fulfils one's promise.



