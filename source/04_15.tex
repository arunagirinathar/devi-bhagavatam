\chapter{On the truce between the Daityas and the Devas}

1. Vy\=asa said :-- O king Janamejaya! Prahl\=ada was glad to hear the above words of the high souled Bh\=argava.

2. Knowing Fate to be the strongest, he addressed the Daityas :-- Never, in this battle will victory be ours.

3-5. Then the victorious Demons, infatuated with pride, told Prahl\=ada :-- What is Fate? We do not recognise it. We ought to fight. O Lord of us :-- Fate reigns over those that are idle, not energetic. Has Fate any shape? Who has created it? Has anybody seen Fate? However let us gather strength and fight. You are very intelligent and all knowing; It is proper that you should take our lead in the fight.

6. O king! When the Daityas spoke thus, Prahl\=ada, the great Destroyer of enemies, became the general and challenged the Devas to fight.

7. On seeing the Asuras in the battle field, the Devas, dressed with arms and weapons, began to fight with them.

8. For full one hundred years the dreadful battle was fought between Indra and Prahl\=ada; on seeing this, the Munis were astonished.

9. O king! In this fearful battle, the Daityas with their general Prahl\=ada, the followers of \'Sukr\=ach\=arya, got the victory.

10. Then Indra, advised by their Guru Brihaspat\={\i}, began to remember the Goddess of the Universe, the Most High, the Giver of welfare, the Destroyer of all sorrows and calamities, and the Bestower of freedom, worship Her, and sing hymns to Her with great devotion.

11-12. Indra said :-- Victory be to the name of the Goddess Mah\=am\=ay\=a, the Eternal Mother, the Holder of the trident! Holder of conchshell, disc, club, and lotus, the Giver of ``no fear.'' Salutation to Thee, the Goddess of the Universe; Thou art the Supreme Heroine in everything that relates to force, that is described in the \'Sakti Dar\'sana \'S\=astras. Thou art the Ten Tattvas, Thou art the Mother, Thou art the Mah\=avidy\=a (the Supreme Knowledge).

13. There are many Tattvas; here the ten tattvas are according to the \'Sakti Dar\'sana. There are many Dharma \'S\=astras. Here \'Sakti \'S\=astra is meant.

The Tattvas are those ultimate substances into which these gross manifestations resolve. The tattvas are Mah\=avindu, N\=ada \'Sakti, Mi\'sra Vindu, etc.

O World Mother! Thou art the Mah\=a Kundalin\={\i} (the great Serpent Fire); thou art the Everlasting Existence, Intelligence and Bliss; Thou art the Deity of the vital Fire (Pr\=a\d{n}a); Thou art the Deity of the Agnihotra (maintenance of the Sacred Fire and an oblation to It); Thou art the Holy Flame, burning always in the ethereal space in the Heart; Salutation to Thee!

14. Thou dwellest within the five Sheaths (the Annam\=ay\=a, the Pr\=a\d{n}am\=ay\=a, the Manom\=ay\=a, the Vij\~n\=anam\=ay\=a and the \=Anandam\=ay\=a sheaths are the five sheaths here referred to). Thou art the Indweller of the \=Ananda m\=ay\=a kosa, Thou art of the nature of Puchchha Brahm\=a, the end of Brahm\=a. Thou art the Deity of all, the \=Ananda (bliss) unblown, O Mother! Thou art the Deity of all the Upani\d{s}adas.

15. O Mother! Be pleased unto us; we have become powerless; protect us, O Mother! we are defeated by the Daityas; O Goddess! Thou art endowed with all the powers, Thou art our Sole Refuge in this Universe, in times of distress, and Thou art the Only One, strong and capable to remove all our dangers.

16. O Goddess! Those who incessantly meditate on Thee are really happy and those that do not meditate Thee, their fear, and sorrows are never removed; those that want ultimate freedom from bondage and who meditate on Thee always; those pure souls, being free from Ahamk\=ara, and free from attachment go, no doubt, beyond this ocean of world.

17. O World Mother! Thy prowess is ever manifested whenever protection is demanded; You always come forward and relieve the distressed; You are the great destroyer! Thou art the Time Incarnate of all these worlds; O Mother! We are fools; how can we appreciate your qualities.

18. Brahm\=a, Vi\d{s}\d{n}u, Mahe\'sa, I myself, Sun, Yama, Varuna, Fire, Air, the high minded munis, \=Agama, Nigama, the Tantras and the Vedas, are quite unable to realise Your unequalled prowess; Salutation to Thy Feet.

19. Those are blessed that are devoted to Thee; They are the great souls; they always dive in the Ocean of Bliss, being always free from the fangs of this Sams\=ara. Those that are not Your devotees, cannot cross this Ocean of Sams\=ara, where the Birth and Death are the billows.

20. O Goddess! Those that are always fanned by the white ch\=amaras and those that travel always in cars, they in their previous births worshipped Thee with various things; therefore they have acquired the effects of their meritorious deeds; this is my opinion.

21. Those that are always worshipped amongst the human beings, those that go on nice elephants, those that are surrounded by pleasures and enjoy the lovely companions of beautiful coquettish women, those that go surrounded by soldiers, O Goddess! I consider they worshipped Thee in their previous births, and they are now enjoying fruits of their past deeds.

22-23. Vy\=asa said :-- Thus praised by Indra, the Goddess of the Universe with four arms hurriedly appeared there mounted on a lion. Conchshell, disc, club, and lotuses were held by the beautiful eyed Goddess in Her four hands respectively, wearing a red apparel and ornamented with divine garlands.

24. The Goddess being pleased addressed the Devas with sweet words, ``Cast off your fear. O Devas! I will see presently all about your welfare.''

25. Addressing the Devas thus, the Divine Mother mounted on a lion, went hurriedly to the place where the demons were waiting, infatuated with pride.

26. All the Daityas with their general Prahl\=ada saw the Goddess before them and were terrified and began to address each other ``What are we to do now?''

27-28. This Chandik\=a Goddess has come here to protect the Devas. She destroyed Mahi\d{s}\=asura and Chanda Munda; it was She that killed, in days of yore, Madhukaitava with evil look.

29-30. Seeing the demons thus full of sorrowful thoughts, Prahl\=ada addressed the Daityas :-- ``It is better not to fight but let us fly away all together.'' Then the Daitya Namuchi told the Daityas ready to fly away ``If you fly away, this World Mother will instantly kill you all with weapons in Her hands.

31. Do that by which we can protect us. Let us worship the Goddess of the Universe, and, getting Her permission, we will go this very day to the P\=at\=ala.''

32. Prahl\=ada said ``I will worship the Goddess Mah\=am\=ay\=a, the Creatrix, Preservrix and Destructrix of the Universe, the World Mother, and the Assurer of safety to Her devotees.''

33. Vy\=asa said :-- Thus saying, the knower of the highest knowledge, Prahl\=ada, the devotee of Vi\d{s}\d{n}u, began to sing hymns with folded hands in praise of the Goddess, the Upholdress of the Universe.

34. I bow down to Thee, the incarnate of the mantra ``Hr\={\i}m'' the Refuge of all, and within Whom this whole Universe, moving and unmoving, is appearing untruly as a snake is mistaken for a garland of flowers.

35. O Goddess! All these Universes, moving and unmoving, have sprung from Thee; Brahm\=a, Vi\d{s}\d{n}u and others are Creators, Preservers in name only; Thou hast created them all.

36. O Mah\=am\=ay\=a! You are the Divine Mother of all! When You have created the Asuras and the Suras, how can you then see any difference between the Devas and the Daityas?

37. As a Mother makes no distinction between her good sons and bad sons, so You are not to make any difference between us and the Devas; this is our prayer to you.

38. O Goddess! You have been sung in all the Pur\=a\d{n}as as the World Mother; therefore, O Mother! We are your sons just as the Devas are.

39. O Mother! As they have got their interests, so we too have got our interests; therefore there is no difference between he Daityas and the Devas. Therefore if anyone makes any difference, it is due to the subtle error.

40. O Goddess! As we are attached to wealth, wives, and other pleasures of the senses, so the gods are; O Goddess! How then can any difference exist between them and us.

41. O Mother! They are the sons of Mahar\d{s}i Ka\'syapa; we also are his sons; Therefore you cannot have partiality for them before us.

42. O World Mother! In You no such difference is visible anywhere. Therefore do You here preserve equality amongst us both.

43. The Suras and Asuras all have sprung from the permutations and combinations of the 3 qualities! Then how the Devas being embodied can possess more qualities than us.

44. Every embodied soul possesses always cupidity, anger, covetousness; how then can one expect to remain without any quarrels with others.

45. We think that it is all sport with You to see our opinions different, rather contradictory, and it is You who got us involved in quarrels with each other and it is Your pleasure to witness how we fight against each other.

46. Sinless one! O Ch\=amunde! Were You not so fond to see our fight, how then, we being brothers are at war against each other. Certainly it is Your Divine Sport.

47. O Goddess! I know what is religion, I know who is Indra. It is the very idea to enjoy these sensual pleasures that is the only cause of our incessant quarrels.

48. O Mother! You are the Sole Ruler of this Sams\=ara; no sensible man can carry out the words of a man who yearns for something. (i.e., O Mother, You are the only one that is desireless; so we can obey your words).

49. O Mother! Once the Devas and the Asuras conjointly churned the ocean. At that time Vi\d{s}\d{n}u, on the plea of distributing the jewel, and the ambrosial nectar, incurred quarrels amongst them.

50. O Mother! You have made him the Preserver and Controller of the Universe and the Spiritual Guide of the world. And it was He who took away the Goddess Lak\d{s}m\={\i}, the beautiful lady amongst the Deva women.

51. Indra, the Lord of the Gods, took the elephant named Air\=avat, the flower P\=arij\=at, the Heavenly Cow yielding all desires, and the horse Uchchai\'srav\=a. Thus, through the desires and devices of Vi\d{s}\d{n}u, they got the excellent things.

52. O! What a wonder is this that the Devas were considered holy persons, after they had committed such unholy acts; no doubt the Devas had done a very heinous crime. O Goddess! You can judge Yourself what is the just and unjust thing in this case.

53. What is Religion? And where is Religion? And what are the acts done by a religious man? What is uprightness, justice, and purity? You better examine which party has observed virtue? Who has shown uprightness, justice and parity? To whom victory and defeat are due? You are the only one capable to judge all these things.

54-55. Alas! Whom to tell all the conclusions arrived at in the Mim\=amsakas. If any one considers, one will find the world is the field of dissensions and quarrels; the argumentators look to the logical reasoning only; followers of the Vedas look to the rules and regulations only; these so called men of gross ideas they acknowledge that this world is created and preserved by the One only, and yet they quarrel amongst each other.

56-57. If there be One and only One Lord of this wide infinite Sams\=ara, then why would there be differences and quarrels amongst each other? Why is there not seen any agreement in opinion and why do the \'S\=astras differ and why are there so many differences in the opinions held by the knowers of the Vedas.

58. O Goddess! This whole Universe, moving and unmoving is selfish; hence arise so many differences between several opinions. There was no one unselfish in this world and there would be no unselfish persons born hereafter.

59-64. Look! The Moon stole away perforce knowingly the wife of Brihaspat\={\i}; Indra, knowing what is religion stole away the wife of Gautama; Brihaspat\={\i} enjoyed forcibly the wife of his younger; and also he outraged his elder brother's wife in her pregnant state and cursed the boy in the womb and
made him blind. What more to say than Vi\d{s}\d{n}u, all full of S\=attvic qualities, severed perforce the head of R\=ahu. O Mother! Look to the case of my grandson Bali who used to pay due respects to all, who was the foremost amongst the virtuous, observer of rigorous truth, performer of sacrifices, liberal, peaceful, all-knowing. The pretender Hari, taking the form of a dwarf in his V\=amana incarnation, deceived Bali and took away all his kingdoms. Alas! Still the intelligent good persons reckon the Deva Vi\d{s}\d{n}u as the preserver of Religion. What a wonder! Those who are flatterers become victorious in this world; and defeat come to those that speak of Dharma.

65. O Goddess! You are the Mother of all the worlds; do whatever You like. But You should know that the Demons are all under Your protection; kill or save them as You like.

66. The Dev\={\i} said :-- O Demons! Leave you all the anger arising from this warfare and go without any fear to P\=at\=ala and live there at your ease and happiness.

67. Better now wait on Time; whether you will get auspicious or inauspicious fruits for your deeds. Know whoever is desireless and unattached, to him happiness is always and everywhere.

68. Whose mind is avaricious, He does not get peace and happiness, even if he acquires the Trilok\={\i}. Even, in the golden age, avaricious persons did not get happiness, though they acquired the fruits of their actions.

69. Therefore you get yourselves freed of your sins and obey My order and leave the earth and go down to the P\=at\=ala.

70. Vy\=asa said :-- On hearing the Dev\={\i}'s words, the Demons obeyed and bowing at Her feet and preserved by Her, went to P\=at\=ala.

71. Then the Dev\={\i} disappeared; and the Devas went away to their own homes. Thus the Devas and the Daityas, abandoning their feelings of enmity towards each other, lived in peace.

O King! He who hears this fact, gets himself freed from all sorts of calamities and reaches the Highest Peace.

Here ends the Fifteenth Chapter in the Fourth Book of \'Sr\={\i} Mad Dev\={\i} Bh\=agavatam, the Mah\=a Pur\=a\d{n}am of 18,000 verses, on the truce between the Daityas and Devas and on their departures with peace, by Mahar\d{s}i Veda Vy\=asa.



