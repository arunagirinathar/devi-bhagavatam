\chapter{On Sudar\'sana's marriage}

1. Vy\=asa said :-- O King! Then, on hearing his daughter's words, that high souled king of Benares, Sub\=ahu, came to the spot where the kings were staying and said :-- ``O kings! Now you can go to your own camps; tomorrow I will perform my daughter's marriage ceremony.

2. Let you all be pleased with me and graciously accept the food and drink, given by me. Tomorrow let you all come here and perform my daughter's marriage ceremony.

3. O Kings! My daughter is not coming today to this hall of Svayamvara; what can I do now; I will console her and bring her here tomorrow. Therefore do you all go now to your own camps respectively.

4. Intelligent persons should not quarrel with the members of their own family. But they should always shew kindness towards their own sons and daughters who are under their protection. However, I will make my daughter understand and bring her tomorrow morning. You may all go now to your places as you desire.

5. Tomorrow morning we will settle about the pledge, whether by choice or by fulfilling a promise, that requires strength, and have the marriage celebrated; or better you all together would decide what mode of Svayamvara is to be adopted.''

6. The kings heard Sub\=ahu and trusted him. Then seeing that the city is well guarded on all sides, they went to their own camps and performed their mid-day duties.

7-8. The king Sub\=ahu on this side began to perform all the duties regarding the marriage of his daughter, after duly consulting with all the chief members of the family. At the appointed time of marriage he brought his daughter in a well concealed and guarded chamber, had the bathing ceremony of the bridegroom elect performed by the priests, versed in the Vedas, and had him well dressed and did other requisite things. Then he brought the bridegroom in the house, made him seat on a Ved\={\i} (platform) and duly worshipped him.

9. Then the large-hearted king gave to the bridegroom seat, \=Achaman\={\i}ya (water for rinsing the mouth and such articles of food as require rinsing one's mouth after eating them), Arghya (articles for worshipping deservedly, p\=adyam, e.g., water for washing the feet with an offer of green grass, rice, etc., made in worshipping a God or a Br\=ahma\d{n}), the two silken cloths and sheet, cows, and two ear-rings and then wanted to give Sudar\'sana his daughter.

10. The high minded Sudar\'sana accepted all the offerings given by the king. Seeing this, Manoram\=a was relieved of her anxiety. Manoram\=a began to think that beautiful and well adorned daughter as if the daughter of Kuvera (the God of wealth); and thanked herself and thought as if all her duties were over.

11. Then the royal ministers carried gladly and fearlessly the beautiful Sudar\'sana, worshipped with ornaments and clothings, in a good nice carriage to the centre of the amusement court.

12. On the other hand, the elderly female members, who knew all about the prescribed rules, performed the dressing of the princess in a befitting manner and placing her in a beautiful conveyance took her before the bridegroom elect, in the marriage hall, where there was the platform regularly built.

13-14. The Sacred Fire was then lit, the royal priest began to perform the Homa ceremony duly; when the amusement ceremony of the bridegroom and bride united in love was duly performed, the priest called them there. After this the bridegroom and bride performed duly the L\=aj\=a Homa ceremony and circumambulated the Sacred Fire. Thus all the ceremonies, befitting the gotra and family, were all fully performed according to the prescribed rules.

15-17. Then the king Sub\=ahu, excited by feelings of love, in the marriage time, gave to the prince Sudar\'sana the following presents: well adorned two hundred chariots, with horses and the arrow cases filled with arrows, one hundred and twenty five elephants, dressed with golden ornaments, looking like so many mountains, one hundred beautiful female elephants and one hundred maid servants, all dressed in golden ornaments.

18-20. The king gave the bridegroom also one thousand servants well adorned, bearing the complete set of all sorts of weapons, many gems and jewels, clothings, nice variegated woolen clothes, beautiful capacious rooms to live in, and two thousand excellent horses born in the Sindhu country, three hundred good camels able to carry sufficient loads, and two hundred carriages, filled with grains, etc.

21. Then the king bowed to the king's daughter Manoram\=a and with clasped hands, said :-- ``O royal daughter! I am now become your servant; now kindly say what is your desire?''

22. Hearing these beautiful words of the king, Manoram\=a said :-- ``O king! all good to you and let your family increase in sons and grandsons. You have increased my honour by giving in marriage your daughter (jewel) to my son. I have no other desire than to see your welfare constant and the increase in your family, posterity and prosperity.

23. O king! Your are the chief amongst the kings. Your have made my son great and strong like the Sumeru mountain by giving him your daughter in marriage. You are high and my related. I am not the daughter of a panegyrist or a bard; how can I then praise you for this noble act of yours.

24-25. O king! Your character is wonderful and pure. What more shall I say to you than this that you all, in the face of many other kings, have given your daughter to my son in marriage, who is banished from his kingdom, is deprived of his father and is living in the forest, penniless, armyless, subsisting himself on roots and fruits only.

26. In these cases the kings as a rule make relations with those only, who are their equals in rank and position, of noble families of equal grade, having forces and wealth equal to each other. No other king would have offered his beautiful well-qualified daughter in marriage to my prince who is without any wealth.

27. O king! On your this act, all the other kings, holding great influence and possessing armies, have turned out your enemies. I, being a woman am unable to describe the amount of patience in you.''

28. The king Sub\=ahu of Benares, hearing the sweet words of Manoram\=a was highly pleased and, with folded hands, began to say, ``O Dev\={\i}, you better take my this celebrated kingdom; I will become the commander of your forces and will try my best to guard this city.

29. Or you can take half of my kingdom and remain here with your son. It is not my desire that you leave this Benares and go and live in the forest.

30-31. The kings have become very offended; I will first try to appease them; if they be not satisfied, I will adopt the means of ``gift'' or sowing dissensions amongst them; and even, if, in that, I fail, I will ultimately take to war. O Dev\={\i}! Victory or defeat is under the hands of the Destiny; still victory comes to those who are in the right path and defeat to those who are in the wrong path. How then can the victor arise to those sinful kings?''

32. Hearing the king's words, pregnant with meaning, Manoram\=a felt herself highly respected; and, with a cheerful heart, said the following good words.

33. ``O king! let all good come on you! you better discard all fear and reign with your sons here; my son Sudar\'sana, too, will become the king of Ayodhya by the Grace of \'Sr\={\i} Bhagavat\={\i} Bhuvane\'svar\={\i}, the Supreme Cause of the innumerable worlds, and will roam in this world; there is no doubt in this.

34. May Bhagavat\={\i} Bhavan\={\i} bring all good unto you; now kindly permit us to depart to our homes, O king! I always contemplate the Highest Goddess Ambik\=a; and I have no time to indulge in other thoughts.''

35. Thus, on various subjects, Manoram\=a and the king Sub\=ahu began to talk with each other, causing satisfaction to both like nectar, when the morning broke out.

36. The kings, knowing early in the morning, that the princess had been given away in marriage, became very much enraged and went out of the city and began to discuss with one another.

``We will kill today the king Sub\=ahu, the disgrace amongst the kings as well that boy Sudar\'sana, totally unfit to marry the princess, and take away the kingdom and the princess \'Sa\'sikal\=a. How can we return to our homes, with this severe disgrace, stamped on our heads.

37. Hear, O kings! the sound of the drums, mridangas, other instruments; the sounds of the conchshells have even been overpowered. Hark! The various musical sounds and the chanting of the Vedas. It is then certain that the King Sub\=ahu has finished the marriage ceremony of his daughter \'Sa\'sikal\=a with Sudar\'sana.

38. Oh! This king has deceived us with his words and performed the marriage ceremony, according to ordinary religious rules.

39. Now O kings! decide unanimously what to do and come to a definite conclusion.''

When the kings were thus discussing, the king of Benares, of indomitable prowess, the king Sub\=ahu, after finishing his daughter's marriage, came there with his famous friends to invite them.

40. Seeing the King of Benares present, all the other kings did not utter a single word, but they remained silent, beaming with anger.

41. Sub\=ahu then approached to the kings, bowed down, and, with folded hands, said :-- ``Be kind enough to come to my house for dinner.

42. O kings! My daughter \'Sa\'sikal\=a after all has selected Sudar\'sana; I could not help in this. You are all kind and noble; therefore you all be peaceful and let the matter drop.''

43. The kings hearing him were filled with rage and said, ``We have all taken food; our desires have been fulfilled; you better now go back to your own home.

44-45. Your behaviour with us is all right and proper; now do your other duties and let the kings go back to their homes.'' Hearing these words of the kings, the king of Benares was very much terrified and returned home, thinking that the kings were all filled with rage and might do serious harm to him. Thus he began to pass away his time in dire anxiety.

46. Then the king Sub\=ahu disappeared; the kings united made this resolve that they would block the passage of Sudar\'sana, kill him, and take the girl away.

47. Some of these kings rather said :-- ``What is the use in killing the king's son. We will all go willingly to see the fun.''

48. Thus the kings went and remained blocking the path of Sudar\'sana; and the king Sub\=ahu, on returning home, began to make arrangements for the departure of the bridegroom and the bride.

Thus ends the 22nd Chapter on Sudar\'sana's marriage in \'Sr\={\i} Mad Dev\={\i} Bh\=agavatam of 18,000 verses by Mahar\d{s}i Veda Vy\=asa.



