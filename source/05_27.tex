\chapter{On the description of the war of Raktab\={\i}ja}

1-14. Vy\=asa said :-- O King! Seeing the two D\=anavas killed in the battle, the remnant soldiers all fled away back to \'Sumbha. Some of them were cut and wounded in many places by arrows, some had their arms severed, some were bleeding; thus they entered crying into the sky. On reaching the lord of the Daityas, they began to make frequently the noise indicative of danger and exclaimed, ``O King! Save us, Save us; K\=alik\=a is devouring everything today. The two great warriors Chanda and Munda, the tormentors of the Devas, were slain by Her; all the soldiers were devoured by Her; we have fled away panic-stricken. O Lord! K\=alik\=a has rendered the battlefield horrible by the dead bodies of elephants, horses, camels, warriors, and foot soldiers. A river of blood is flowing there of which the flesh of the soldiers is sufficient mud, their hairs are like aquatic plants, the broken chariot wheels are like whirlpools, the severed arms and feet are like fishes and their heads look like Tumbi fruits (long gourds). O King! Save your line; go quickly to P\=at\=ala. The Dev\={\i} has become angry and will, no doubt, destroy our

race. Even the lion is eating away the D\=anavas; and the K\=alik\=a Dev\={\i} is killing innumerable D\=anavas by Her arrows. Therefore, O King! What intentions are you cherishing in your mind? Is it that you have desired to be merely slain with your younger brother Ni\'sumbha! And what good purpose will this cruel woman, destroying your race, serve, for Whose sake, you have desired to kill all your friends? O King! Victory or defeat in this world are under the Da\={\i}va. The wise never risk to meet with great difficulties for the gratification of an ordinary whim. O Lord! Look at the wonderful deeds of that Great Creator! What more wonder can there be than this that a woman alone killed all the D\=anavas. O King! You have conquered by the help of your army all the Lokap\=alas (guardians of the quarters of the sky); but now that Lady, though alone and unsupported by anybody, is challenging you to fight.

15-24. O King! In ancient times, in the holy pilgrimage of Pu\d{s}kara, the sacred place of the Devas, you performed austerities when Brahm\=a, the Grandsire of all the worlds, came to you to grant a boon. Then you asked the boon and wanted to become to be immortal. But when Brahm\=a refused to grant it you wanted from him and were granted that you would not be killed by any male being, be he a Deva, D\=anava, a man, N\=aga, Kinnara, Yak\d{s}a, or any other person. O Lord! For that very reason we think now that this Lady has timely come to kill you. Think over it seriously and cease fighting. O King! This Dev\={\i} is the great Mah\=a-M\=ay\=a, the Highest Prakriti; It is She that devours everything at the end of a Kalpa. This auspicious Dev\={\i} is the Creatrix of all the worlds and the Devas. She is the embodiment of the three qualities, endowed with all the powers. She is T\=amasi, i.e., is the Destructrix of the whole world. This Dev\={\i} can never be conquered, Imperishable, Eternal, She is the Sandhy\=a and the Refuge of the Devas. She is G\=ayatr\={\i}, the Mother of the Vedas. She is All-knowing and always manifested. This Undecaying Lady is void of any Prakritic attributes, though She at times possesses attributes. She is Success Incarnate and bestows success to all; She is Bliss Herself and gives bliss to all. This Gaur\={\i} bids all the Devas discard all their fears. She is \'Suddha Sattva. Thus knowing, O King! Quit thy inimical feelings to Her; seek refuge unto Her; the Dev\={\i} would then certainly protect you. Be obedient to Her and save your race. Then the remnant D\=anavas will be able to live for a very long time.''

25. Vy\=asa said :-- O King! Hearing thus, \'Sumbha, the conqueror of the Devas, told them truly in words becoming of a hero.

26-42. \'Sumbha said :-- ``O Fools! Hold your tongue.You have fled because your desire to live is very strong. So you better go to

P\=at\=ala without any delay. This world is under the control of Fate; so I need not think about Victory. I am under this Fate just as Brahm\=a and other Devas are under it. Brahm\=a, Vi\d{s}\d{n}u, Rudra, Yama, Ag\d{n}i, Varu\d{n}a, S\=urya, Chandra, and Indra are all under the sway of this Destiny. O Fools! Whatever is inevitable will certainly come to pass. What need I think over it then? The effort also comes to be of such a nature as will lead to that ordained by Fate. Thus thinking, the wise never grieve; especially the wise ones never leave their own Dharma for fear of death. The happiness, pain, longevity, birth and death of all the embodied souls are all determined by Fate when their proper time arrives. See! When the time is over, Brahm\=a, Vi\d{s}\d{n}u and Mah\=adeva, the lord of P\=arvat\={\i} die away; on the expiration of their terms of lives, Indra and other Devas go to destruction. Similarly I am also completely under the sway of time; so what doubt is there that I, too, will go to destruction when I have observed my own Dharma! This Lady is challenging me to fight of Her own will; how can I fly away and live hundreds of years. I will fight today. Let the result come whatever it may. I will gladly take the victory or defeat whatever the case may be. The learned approving of the cause of effort declare Fate as fictitious; those who realise their sayings know that they are full of reason. Without exertion no end can be achieved; weak persons depend on the destiny. Foolish persons say that Fate is strong; but the wise do not say so. There is no proof whether Fate exists or not; in fact what is called Fate is invisible; how can it then be seen? Has anybody seen Fate? It is simply a scare for the illiterate; remedy only to console one's mind in times of distress. Simply proximity to a grindmill without any man's effort cannot grind a material. Therefore if exertion be made in proportion to the gravity of the work, success is sure to ensue; if exertion be made less in proportion, the work does not come to a successful issue. If time, place and one's enemie\'s forces be correctly taken into account and then if the proper attempts be made, success follows; thus Brihaspat\={\i} has said.''

43-44. Vy\=asa said :-- O King! Thus making a firm resolve to send the powerful Raktab\={\i}ja to the battle with a vast army; \'Sumbha said :-- ``O Raktab\={\i}ja! You are a very powerful hero; therefore do you go to the battle. O Fortunate One! Fight as you are the strength of your forces.''

45-46. Raktab\={\i}ja said :-- ``O King! You need not be a bit anxious for this work.Certainly I will either slay Her or I will bring Her under your control. Please see my skill in this warfare; that Lady, favourite

of the gods, is worth nothing; I will just now conquer Her and make Her your slave.''

47-50. Vy\=asa said :-- O Best of Kurus! Thus saying, the powerful Raktab\={\i}ja mounted on his chariot and went to the battle accompanied by his forces. The battalion consisted of cavalry, infantry, chariots and elephants. Thus surrounded he departed from the city for that Dev\={\i}, seated on a mountain top. Then the Dev\={\i}, seeing him coming, blew Her conchshell; the D\=anavas were terrified at that sound and the joy of the Devas increased. Hearing that sound Raktab\={\i}ja came very hurriedly to Ch\=amund\=a and began to speak to Her sweetly.

51-62. O Girl! Do you think me weak and thus want to terrify me with the sound of a conchshell? O Lean One! Have you taken me to be a Dh\=umralochana? O Sweet speaking one! My name is Raktab\={\i}ja; I have come here for Thy sake. If Thou desirest to fight, be prepared; I am not a bit afraid of that. O Dear! You saw those who were weak; I do not belong to that class. Therefore dost Thou fight as Thou likest and then Thou wilt be able to ascertain my strength. O Beautiful! If Thou didst serve the old persons before, if Thou hadst heard the science of politics and morals, if Thou hadst studied the political economy, joined the assemblage of the Pundits or if Thou hast been well versed in literature and Tantras, then hear this my good counsel which will serve as a medicinal diet to Thee. Out of the nine sentiments, the \'Sring\=ara (Amorous love sentiments) and \'S\=anti (Peace) are considered as the chief by the assemblage of the Pundits. Again out of these two, the love sentiment is the king. Drenched with this sentiment, Vi\d{s}\d{n}u lives with Kamal\=a; Brahm\=a, the four-faced, lives with S\=avitri; Indra with \'Sach\={\i} and \'Sankara resides with his wife Um\=a. The tree stands with creepers surrounding it, the deer lives with his female deer, the pigeon lives with the female pigeon; thus all the beings are very attached to this sentiment of remaining in couples. Those who cannot enjoy owing to certain disease or illness, they are deprived by Fate of such enjoyments. Those who are ignorant of this love sentiment in couples, they are deprived of it by the sweet jugglery of words of the cheat and yet attached to the Peace sentiment. When delusion, the destroyer of Buddhi, the common sense, occurs, when the violent indomitable anger, greed, and lust arise, where, then, is the place for knowledge and dispassion? Therefore, O Auspicious One! Dost Thou marry the beautiful \'Sumbha or the powerful Ni\'sumbha.

63. Vy\=asa said :-- O King! When Raktab\={\i}ja spoke all these words, standing before the Dev\={\i} K\=alik\=a, Ambik\=a and Ch\=amund\=a began to laugh.

Here ends the Twenty-seventh Chapter of the Fifth Book on the description of the war of Raktab\={\i}ja in \'Sr\={\i} Mad Dev\={\i} Bh\=agavatam, the Mah\=a Pur\=a\d{n}am, of 18,000 verses by Mahar\d{s}i Veda Vy\=asa.



