\chapter{On the Nimi's getting of another body and the beginning of the story of Haihayas}

1. Janamejaya said :-- ``The getting back of another similar body by Va\'sistha is certainly described by you. Now tell me how the King Nimi got another body.''

2-31. Vy\=asa said :-- O King! The \d{R}i\d{s}i Va\'sistha only got back his body; but the King Nimi did not get back his body what had been cursed by Va\'sistha. The priests engaged at the sacrifice by Nimi began to consider, when the \d{R}i\d{s}i Va\'sistha cursed him, in the following way :-- Oh! What a wonderful thing is this? Before the sacrifice is complete, the King Nimi has been cursed; this is against what we had expected; What can we do? What is inevitable must come to pass; how can we thwart it? By various Mantrams, they kept alive the body of the King in which breathing was still going on a little; and they prevented the body from decaying by worshipping the body with various Mantra \'Saktis and kept it in a stationary state. When the sacrificial ceremony was completed, the \d{R}i\d{s}is began to praise the gods with hymns whereon the Devas became pleased and came to that spot. When the Munis informed the Devas fully of the condition of the King's body, the Devas spoke to the sorrowful King thus :-- ``O Performer of good vows! We are all pleased with your sacrifice; now ask boons from us. O King! You ought to get an excellent birth as the fruit of performing this sacrifice. So ask what body, the body of a Deva or of a man, you desire? Or you can ask, if you like, for another similar body, that your priest Brihaspati has got quitting his first body whereby he has become proud and is now staying in the Loka of Yama.'' O King! At these words the King Nimi was very glad and spoke to them thus :-- O Devas! I have no aspiration for the body that is always liable to destruction; I therefore want to reside on the top of the eyelids of all the beings. Therefore I ask this boon that I be able to move in the shape of V\=ayu (air) on the top of the eyes of all the beings. Thus said, the Devas spoke to the soul of Nimi :-- ``O King! Pray to the most auspicious Deity, the Dev\={\i}, the Highest Goddess. She has been pleased with this sacrifice; therefore your prayer will certainly be granted.'' Hearing thus, the King began to pray with various hymns with intense devotion, in tremulous voice, the Dev\={\i}. The Dev\={\i} became pleased and

appeared before him. Seeing Her shining like a crore of suns and looking exceedingly lovely and beautiful, all the persons there became very happy. They began to think themselves as very blessed and as having done all what they had to do. Knowing the Dev\={\i} Bhagavat\={\i} pleased, the King asked this boon from her :-- ``O Dev\={\i}! Give me that knowledge, pure and simple, whereby final liberation is obtained. Also, I may be able to reside on the top of the eyes of all the beings.'' The Dev\={\i}, the Lord of the Devas, the Mother of the World being highly pleased, said thus :-- ``O King! At the expiry of this your Pr\=arabdha Karma, you will acquire pure knowledge and you will reside on the tops of the eyes in the shape of V\=ayu, and through your residing there the beings will twinkle, i.e., open and close their eyes. The men, beasts, and birds will twinkle due to your residing there; but the Immortals will always remain with a steadfast gaze; they will not twinkle.'' Thus granting him the boon, and addressing all the Munis the Bhagavat\={\i}, the Highest Deity, disappeared. When the Dev\={\i} disappeared from their sight, the Munis then thought much and they took the body of the King Nimi to burn it duly. For the sake of getting a son from Nimi, the high-souled Munis performed Homa ceremony (oblations to the fire) and placing the piece of wood Arani on his body began to utter Mantrams and burned his body. When the woods were thus burned, a son, endowed with all auspicious signs, looking like a second Nimi, was born to them. As this son was born due to the burning of the Aranis, the boy was named Mithi, and as it came out of the body of Janaka, the boy was named Janaka. O King! As the King Nimi lost his body, i.e., became Videha through the curse of Va\'sistha, all his descendants were known as Videha. Thus the son of Nimi was well known as the King Janaka. He built a beautiful city on the banks of the Ganges; the city became famous also by his name (Janakapuri). The King Janaka beautified this city with many forts, arcades, markets and many nice buildings and palaces; and his city was full of wealth and grains. O King! All the Kings of this line became famous by the name of Janaka and all were endowed with the Supreme knowledge and known as Videha. O King! I have now described to you the story of the King Nimi who got disembodiedness (Videhatva) out of the curse.

32-35. The King said :-- ``O Bhagav\=an! You have described the cause why the King Nimi was cursed; my mind has grown very doubtful and restless on hearing it. The \d{R}i\d{s}i Va\'sistha was the son of Brahm\=a and the best of the Br\=ahmi\d{n}s; especially he was the royal priest; how was it, then, that he was cursed by the King! Why did not the King Nimi forgive him as he was the Guru and a Br\=ahmi\d{n}? Why he became angry, when he performed such a great, auspicious sacrifice? He

was born of the family of Ik\d{s}\=aku and he knew well the truths of the religion; then how was it that he became subject to anger and cursed his own Guru Br\=ahmi\d{n}.''

36-46. Vy\=asa said :-- O King! It is very hard and rare for the persons not possessed of self-restraint to forgive; especially when one is fully capable, it is very rare to find one in the three worlds, who can forgive. He who has forsaken all attachments and has conquered hunger and sleep and is always engaged in the Yoga practices, even that ascetic Muni is not capable to conquer completely lust, anger and greed and Ahamk\=ara, etc., the passions raging in the mortal coil. None existed before in this whole world who conquered his passions! None exists now and none will be born ever-after. Hardly will be seen any in this earth, or the Heavens, or the Loka of Brahm\=a or in Vaikuntha, even in Kail\=asa, that has conquered completely his passions! What can be said in regard to the ordinary mortals of this earth when the sons of Brahm\=a, the Mahar\d{s}is, ascetics, \d{R}i\d{s}is are all pierced by the S\=attva, R\=ajas, or T\=amo Gu\d{n}as. Behold! The Kapila was the Knower of \'Sankhy\=a and always engaged in his Yoga practices and he was a pure and holy soul; yet, by strange combinations of Fate, he became angry and burnt to ashes the sons of the King Sagara. O King! Out of Ahamk\=ara, these three worlds are created; therefore this world and Ahamk\=ara are related to each other as effect and cause; how then the J\={\i}vas that are born of this Sams\=ara can extricate themselves from this Ahamk\=ara? Brahm\=a, Vi\d{s}\d{n}u and Mahe\'sa are also pierced by those three Gu\d{n}as; different feelings are seen in their different bodies. Therefore it need hardly be said that the manifestation of the pure S\=attva Gu\d{n}a alone is not to be seen in any of the human beings; for the three Gu\d{n}as reside in a mixed way in all persons. Sometimes the S\=attva preponderates; sometimes the R\=ajas and sometimes the T\=amas preponderates. Sometimes they reside together, the three balancing one other.

47-63. O King! Only that Eternal Highest Puru\d{s}a is undecaying and untainted and can hardly be measured or seen by all the beings. That Highest Soul, the Highest of the High, is Nirgu\d{n}a (void of the three Gu\d{n}as); and She who resides in all the beings and is hardly knowable by the small intellectual persons, that Highest \'Sakti, the Incarnate of Brahm\=a, is also Nirgu\d{n}a (void of attributes). Param\=atm\=a (the Highest Soul) and the Highest Force are also One; their Forms are not different. When such a knowledge arises, then the J\={\i}vas can be free from all sins and faults and blemishes. From that knowledge comes the liberation, this is sounded in the Ved\=anta \'S\=astra like Dindima \'Sabda (thousands of small drums). He, who comes to know That, is freed from the endless cycle of birth and death composed of the three Gu\d{n}as; there is no doubt in this.

O King! Knowledge is of two kinds :-- The first is considered as coming from sound; this comes out of the knowledge of the meaning of the Vedas by the help of intellect. But this is full of fancies, agreements and doubts some of which are bad and some are good. The beings are led into errors by these discussions; errors cause destruction of intellect; and when the intellect is gone, the knowledge also goes away with it. Whereas the second kind of knowledge comes from intention or feeling within the depths of heart and brain and it is called Aparok\d{s}a J\~n\=ana. This knowledge is very rare to the beings. When one comes in contact with a Sad-Guru (a good teacher), then one gets this Aparok\d{s}a J\~n\=ana. From the sound knowledge, no successful results can issue; and, therefore it cannot give Aparok\d{s}a J\~n\=ana. Hence great effort is to be made for getting this Aparok\d{s}a J\~n\=ana. O King! As darkness cannot be destroyed merely by talking of light, without lighting any lamp, so the knowledge of sound merely cannot destroy the darkness of the inside. That Karma (action) is called True Karma which does not lead to bondage, and that Knowledge is the True Knowledge which leads to liberation. Other actions are only meant for one's own selfish enjoyments and other knowledges are merely the skill in arts. Good behaviour, doing good to others, having no anger, forgiveness, patience, and contentment are the best brilliant fruits of True Knowledge. O King! Without knowledge, without asceticism, and without the Yoga practices, the lust and other passions can never be destroyed. The minds of the J\={\i}vas are naturally restless and without control; all the beings are completely under the sway of their minds; thus they roam on the surface of the earth as good, middling and bad. Lust, anger, etc., orginate from this mind; and when mind is conquered, then those feelings can no more arise. O King! Therefore it was that Yay\=ati forgave when \'Sukr\=ach\=arya did wrong before. The King Nimi could not forgive Va\'sistha in the same way. Yay\=ati; the best of kings, though cursed by \'Sukr\=ach\=arya, the son of Bhrigu, did not curse in return but he took upon himself the old age. O King! Some kings are naturally peaceful, whereas some other kings are wicked by their nature. Therefore, in this matter, whose fault is this, how can we ascertain? See! In ancient times the Haihayas, out of their greed of wealth, and being thus insensible destroyed completely, out of anger, the Br\=ahmi\d{n} priests of the family of Bhrigu. What more than this that those K\d{s}attriyas did not consider the sin Brahmahatty\=a; rather out of their dire anger, they cut to pieces the sons of those Br\=ahma\d{n}as, that were in embryos in their mother's wombs.

Here ends the Fifteenth Chapter in the Sixth Book on the Nimi's

getting of another body and the beginning of the story of Haihayas, in the Mah\=apur\=a\d{n}am \'Sr\={\i} Mad Dev\={\i} Bh\=agavatam of 18,000 verses by Mahar\d{s}i Veda Vy\=asa.



