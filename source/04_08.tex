\chapter{On going to the T\={\i}rthas}

1. S\=uta said :-- Thus asked by the son of Par\={\i}k\d{s}it, the king Janamejaya, the best of the Br\=ahma\d{n}as, the son of Satyavat\={\i}, Vy\=asa spoke, in detail, the following :--

2. The virtuous Janamejaya became very much sorry and despondent when he heard in detail the improper acts of his own father Par\={\i}k\d{s}it, the son of Uttar\=a.

3. Owing to insulting the Br\=ahmi\d{n} boy, his father had to go to hell; and he was constantly thinking how to release his father.

4. The son is called ``Puttra'' for he releases his father from the hell, named ``Put''. He is the true son that can do so.

5-6. The fortunate son of Par\={\i}k\d{s}it became very much tormented and bewildered with fear when he heard what was the fate of his father, who died bitten by a serpent on the top of a palace, due to the curse of a Br\=ahmi\d{n}, void of any bathing, charities, and the purificatory acts.

7. When Vy\=asa returned home, Janamejaya asked him, the whole course of events of Nara N\=ar\=aya\d{n}a.

8. Vy\=asa said :-- O King! When the terrible Hira\d{n}ya Ka\'sipoo was slain, his son Prahl\=ada was installed on his throne.

9. During the government of Prahl\=ada, the chief of the Daityas, the worshipper of the Br\=ahma\d{n}as and Devas, the kings on earth began with faith to do many sacrifices for the satisfaction of the Devas.

10. The Br\=ahma\d{n}as were engaged in their Tapasy\=a, Dharma, and in frequenting the places of pilgrimages; the Vai\'syas, in their trade; and the \'S\=udras, in serving the other three classes.

11. The incarnation of Hari, the Nri Simha (Man-Lion) made Prahl\=ada, the king of the Daityas in the P\=at\=ala (Nether regions); and Prahl\=ada, engaged there, spent his time in the preservation and welfare of his subjects.

12. Once, on a time, the great ascetics Chyavana Muni, the son of Bhrigu went on his way to bathing in the river Narmada, at the place of pilgrimage, called Vy\=arhit\={\i}\'svara.

13. There he saw the great river Rev\=a and, while he was descending in the river, a dreadful snake caught hold of him and carried him to the P\=at\=ala. The Muni was greatly terrified and began to think of the Deva of the Devas, Jan\=ardana Vi\d{s}\d{n}u.

14-15. On remembering the lotus eyed Vi\d{s}\d{n}u, the serpent lost his poison, and Chyavana Muni did not find any trouble, though carried to the P\=at\=ala.

16. Then the serpent, coming to know of the power of the Muni, left him for fear that the Muni might curse him; the snake afterwards repented very much.

17. Chyavana, the best of the Munis, worshipped by the daughters of the serpents, roamed there and entered once into a beautiful palace of the N\=agas and the D\=anavas.

18. While he was walking, he was seen by the religious king of the Daityas, the Prahl\=ada.

19. The lord of the Daityas on seeing him worshipped him and enquired of him the cause of his coming there.

20. Are you sent here by Indra? Speak truly, O best of the Br\=ahma\d{n}as. Is it to pry into my kingdom out of the enmity between the Devas and the Daityas?

21. Chyavana said :-- What have I to do with Indra? That I might be sent by him, as his spy, messenger, to your city!

22. O Chief of the Daityas! Know me as Chyavana, the son of Bhrigu, prompt in religious duties and whose eye is illumined by knowledge. Do not fear that I am sent here by Indra.

23. When I went to bathe in a place of pilgrimage, in the holy Narmad\=a, and dropped into the river, a poisonous snake caught hold of me (and carried me here).

24. I took the name of Vi\d{s}\d{n}u; and the serpent, hearing the Vi\d{s}\d{n}u's name, became void of poison, and left me here as you see.

25. O king! Coming here, I see you. You are a devotee of Vi\d{s}\d{n}u; know me, too, a devotee of the same Vi\d{s}\d{n}u.

26. Vy\=asa said :-- O king! Prahl\=ada, the son of Hira\d{n}ya Ka\'sipu, on hearing his sweet words, gladly asked him about the various places of pilgrimages.

27. Prahl\=ada said :-- O Best of Munis! Kindly describe to me, in detail, which are the places of pilgrimages on the earth, P\=at\=ala, and in the Heavens, that verily lead to holiness.

28. Chyavana said :-- O King! He whose body, words, and mind have grown pure, to him, his every footstep is a place of pilgrimage; he, whose heart is impure and defiled, to him the holy Ganges even is a thing more hated and worse than the K\={\i}kata country (the name of Beh\=ar).

29. Every holy place will impart holiness to him whose mind is first pure and deprived of sin.

30-31. O Best of the Daityas! On the banks of the Ganges, are situated good many cities, towns, villages, places to assemble, mines, small villages, the living places of the aborigines, the ch\=and\=alas, and kaivartas, the H\=u\d{n}as, Bangas, the Khasas and the other Mlechchas.

32. The inhabitants of the above places drink at their will the holy Ganges water, equivalent to Br\=ahma, and bathe therein and do other works.

33. O King! There not even a single soul becomes pure. What use is a holy place to him, whose heart becomes attached to the sensual objects and who can therefore be called the lost souls.

34. Know, O king! the mind as the principal factor in any religious act or in any holy place. He who wants purity, let him first make his own mind pure.

35. The residents in any holy place deceive others and thus incur great sins. The sins committed in a place of pilgrimage can never be removed; they become unending and inexhaustible.

36. As the fruit, Indrav\=aru\d{n}\=a, is never sweet though fully ripe, so whose heart is defiled, he can never be pure though he bathes hundreds and thousands of times in the T\={\i}rtha water.

37. He who wants welfare of his own and others, he should first make his mind pure; when his mind becomes pure, then, the purity of material things and the purity of conduct can have any effect; then and then only resorting to places of pilgrimages becomes efficacious.

38-39. Always avoid company with the lowest class of persons in the holy places; it is far better to shew one's good will and compassion to all the souls (j\={\i}vas) by one's intellect and by one's acts. You have asked me about the holy places of pilgrimages; I will now tell you those that are the best.

40. O king! The holy Naimi\'s\=ara\d{n}ya is the first, next Chakrat\={\i}rtha; next Puskarat\={\i}rtha; there are many others besides these that cannot be counted. O Best of kings! There are lots of other holy places in this world.

41-42. Vy\=asa said :-- O king! Prahl\=ada, the king of the Daityas, on hearing the Muni's words, became ready to go Naimi\'sra\d{n}ya and, with very much gladness, exclaimed to his followers, the Daityas :-- O Good Ones! Get up; today we will go to Naimi\'s\=ara\d{n}ya and we will see the lotus eyed, yellow robed \'Sr\={\i} Achyutam, the Vi\d{s}\d{n}u.

43. Vy\=asa said :-- O King! When thus addressed by Prahl\=ada, the Demons were exceedingly glad; and they all marched out of P\=at\=ala.

44. The Daityas, and Demons all united went to Naimi\'s\=ara\d{n}yam and filled with much pleasure, they all bathed on reaching that holy place.

45. There, accompanied by the Daityas, Prahl\=ada roamed about the sacred places and saw the holy Sarasvat\={\i} river and Her pure clean water.

46. The highsouled Prahl\=ada bathed in the Sarasvat\={\i} river and his mind was satisfied.

47. The king of the Daityas was very much pleased and he perform ablutions and charities according to due rites in that most auspicious sacred place of pilgrimage.

Here ends the Eighth Chapter in the Fourth Book of \'Sr\={\i} Mad Dev\={\i} Bh\=agavatam, the Mah\=a Pur\=a\d{n}am of 18,000 verses by Mahar\d{s}\={\i} Veda Vy\=asa on going to the T\={\i}rthas.



