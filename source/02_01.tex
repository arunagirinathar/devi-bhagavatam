\chapter{On the birth of Matsyagandh\=a}

1-5. The \d{R}i\d{s}is said :-- ``O S\=uta! Your words sound wonderful to us indeed! but you have not as yet definitely spoken to us the original events in detail; so a great doubt has arisen in our minds. We know that the king \'Santanu married Vy\=asa's mother, Satyavat\={\i}. Now say, in detail in how Vy\=asa became her son? How such a chaste woman Satyavat\={\i}, remaining in her own house, came to be married again by \'Santanu? and how the two sons came to be born of \'Santanu's sperm and Satyavat\={\i}'\'s ovum? Now O highly fortunate Suvrata? Kindly describe in detail this highly sanctifying historical fact. These \d{R}i\d{s}is, who are observing vows, are desirous to hear of the birth of Veda Vy\=asa and Satyavat\={\i}.''

6-23. S\=uta said :-- I bow down with devotion to the Highest Primordial Force, the bestower of the four fold aims of existence of human life, who grants to all, their desires when so prayed by the help of the V\=agbhava V\={\i}jamantra with their heart and soul, for the success of all their desires. The above v\={\i}ja is so potent in its effect that even pronounced very lightly, even under a pretext, it grants all siddhis. So the Dev\={\i} should be remembered by all means; and now saluting Her, I begin my narration of the auspicious Pur\=a\d{n}ic events. In days gone by there reigned a king, named Uparichara; he ruled over the Chedi country and respected the Br\=ahmins; he was truthful and very religious. Indra, the lord of the Devas, became very pleased by his asceticism and presented him an auspicious celestial car (going in the air) made of pearls, and crystals, helping him in doing what he liked best. Mounting on that divine chariot, that religious king used to go everywhere; he never remained on earth; he used to remain always in the atmosphere and therefore be had his name as ``Uparichara Vasu'' (moving in the upper regions). He had a very beautiful wife, named Girik\=a; and five powerful sons, of indomitable vigour, were born to him.

The king give separate kingdoms to each of his sons and made them kings. Once on an occasion, Girik\=a, the wife of the Uparichara Vasu, after her bath after the menstruation and becoming pure came to the

king and informed him of her desire to get a son; but that very day his Pitris (ancestors) requested him also to kill deer, etc., for their Sr\=addha (solemn obsequies performed in honour of the manes of deceased ancestors). Hearing the Pitris, the king of Chedi became somewhat anxious for his menstruous wife; but thinking his Pitris words more powerful and more worthy to be obeyed, went out on an hunting expedition to kill deer and other animals, with the thought of his wife Girik\=a in his breast. Then while he was in the forest, he remembered his Girik\=a, who was equal in her beauty and loveliness to Kamal\=a, and the emission of semen virile took place. He kept this semen on the leaf of a banyan tree and thought ``How the above semen be not futile; my semen cannot remain unfruitful; my wife has just now passed her menstruous condition; I will send this semen to my dear wife.'' Thus thinking the time ripe, he closed the semen under the leaves of the banyan tree and charging it with the mantra power (some power) addressed a falcon close by thus :-- ``O highly fortunate one! Take this my semen virile and go to my palace. O Beautiful one! Do this my work: take this semen virile and go quick to my palace and hand it over to my wife Girik\=a for to-day is her menstruation period.''

24. S\=uta said :-- ``O \d{R}i\d{s}is! Thus saying, the king gave that leaf with the virile therein to the falcon, who is capable of going quick in the air, took it and immediately rose high up in the air.

25-26. Another falcon, seeing this one flying in the air with leaf in his beak, considered it to be some piece of flesh and fell upon him. Immediately a gallant fighting ensued between the two birds with their beaks.

27. While the fighting was going on, that leaf with semen virile fell down from their beaks on the waters of the Jumn\=a river. Then the two faIcons flew away as they liked.

28-39. O \d{R}i\d{s}is! While the two falcons were fighting with each other, one Apsar\=a (celestial nymph) named Adrik\=a came to a Br\=ahmin, who was performing his Sandhy\=a Bandanam on the banks of the Jumn\=a. That beautiful woman began to bathe in the waters and took a plunge for playing sports and caught hold of the feet of the Br\=ahma\d{n}a. The Dvija, engaged in Pr\=a\d{n}\=ay\=ama (deep breathing exercise), saw that the woman had amorous intentions, and cursed her, saying :-- ``As you have interrupted me in my meditation, so be a fish.''

Adrik\=a, one of the best Apsar\=as, thus cursed, assumed the form of a fish Safari and spent her days in the Jumn\=a waters. When the semen virile of Uparichara Vasu fell from the beak of the falcon, that fish Adrik\=a came

quickly and ate that and became pregnant. When ten months passed, a fisherman came there and caught in a net that fish Adrik\=a. When the fish's belly was torn asunder, two human beings instantly came out the the womb. One was a lovely boy and the other a beautiful girl. The fisherman was greatly astonished to see this. He went and informed the king of that place who was Uparichara Vasu that the boy and the girl were born of the womb of a fish. The king also was greatly surprised and accepted the boy who seemed auspicious. This Vasu's son was highly energetic and powerful, truthful and religious like his father and became famous by the name of the king Matsyar\=aj. Uparichara Vasu gave away the girl to the fisherman. This girl was named K\=ali and she became famous by the name of Matsyodar\={\i}. The smell of the fish came out of her body and she was named also Matsyagandh\=a. Thus the auspicious Vasu's daughter remained and grew in that fisherman's house.

The \d{R}i\d{s}is said :-- The beautiful Apsar\=a, cursed by the Muni, turned into fish; she was afterwards cut asunder and eaten up by the fisherman. Very well! What happened afterwards to that Apsar\=a? How was she freed of that curse? and how did she go back to the Heavens?

Thus questioned by the \d{R}i\d{s}is, S\=uta spoke as follows :-- When the Apsar\=a was first cursed by the Muni, she was greatly astonished; she began to weep and cry like one greatly distressed and afterwards began to praise him. The Br\=ahmi\d{n}, seeing her weeping, took pity on her and said :-- ``O good one! Don't weep; I am telling you how your curse will expire. As an effect of having incurred my wrath, you will be born as a fish and when you will give birth to two human children, you will be freed of your curse.''

The Br\=ahmin having spoken thus, Adrik\=a got a fish-body in the waters of the Jumn\=a. Afterwards she gave birth to two human children and became freed of the curse when she, quitting the fish form assumed the divine form and went up to the Heavens. O \d{R}i\d{s}is! The beautiful girl Matsyagandh\=a thus took her birth and was nourished in the fisherman's house and grew up there. When the extraordinarily lovely girl of Vasu, Matsyagandh\=a attained her youth, she continued to do all the household duties of the fisherman and remained there.

Thus ends the first chapter of the Second Skandha on the birth of Matsyagandh\=a in the Mah\=apur\=a\d{n}a \'Sr\={\i} Mad Dev\={\i} Bh\=agavatam of 18,000 verses by Mahar\d{s}i Veda Vy\=asa.



