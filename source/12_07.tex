\chapter{On the D\={\i}k\d{s}\=a vidhi or on the rules of Initiation}

1-3. N\=arada said :-- I have heard the one thousand names or n\=ama stotras equivalent in its fruits to \'Sr\={\i} G\=ayatr\={\i}, highly potent in making a good fortune and tending to a splendid increase of the wealth and prosperity. Now I want to hear about how initiations in Mantrams are performed, without which nobody, be he a Br\=ahmi\d{n}, a K\d{s}attriya, a Vai\'sya or a \'S\=udra, is entitled to have the Dev\={\i} Mantra. O Lord! Kindly describe the ordinary (S\=am\=anya) and the special (vi\'se\d{s}a) rules thereof.

4-41. N\=ar\=aya\d{n}a said :-- O N\=arada! Listen. I am now telling you about the rules of initiation (D\={\i}k\d{s}\=a) of the disciples, pure in heart. When they are initiated, they become entitled then and not before that, to worship the Devas, the Fire and the Guru. That method of instruction, and religious act and ceremony is called the D\={\i}k\d{s}\=a (initiation) by which the Divine Knowledge is imparted; and at once flashes in the heart and mind of the initiated that Knowledge and all his sins are then destroyed. So the Pundits of the Vedas and the Tantras, say. (The Divine Knowledge is like lightning, fire, arising and permeating the body, mind, and spirit.) This D\={\i}k\d{s}\=a ought to be taken by all means. This gives excellent merits and pure results. Both the Guru and the \'Si\d{s}ya (disciple) ought to be very pure and true. (This is the first essential requisite. Then the results are instantaneous). First of all, the Guru is to perform all the morning duties, he is to take his bath and perform his Sandhy\=a Vandanams. He is to return home from the banks of the river with his Kamandalu and observe maunam (silence). Then, in order to give D\={\i}k\d{s}\=a, he is to enter into the assigned room (Y\=aga Mandapa) and take his seat on an \=Asana that is excellent and calculated to please all. He is to perform \=Achanara and do Pr\=a\d{n}\=ay\=ama. Then he should take water in his Arghya vessel and putting scents and flowers in that, charge that water with Phatk\=ara mantra (that is, condense electricity Spirit in that). Then uttering the Phat mantra, he is to sprinkle the water on the doorways of the worshiproom and begin his P\=uj\=a. Firstly, on the top of the door at one end invoke the Deity Ga\d{n}an\=ath\=a by His mantra, at the other end invoke Sarasvat\={\i} by Her mantra and at the middle, invoke Lak\d{s}m\={\i} Dev\={\i} by Her mantra duly and worship them, with flowers. Then, on the right side worship Gang\=a and Bigh\d{n}e\'sa; and on the left side worship K\d{s}ettrap\=ala and Yamun\=a, the daughter of the Sun. Similarly, on the bottom of the door, worship the Astra Devat\=a by the

Phat mantra. Then consider the whole Mandapa as inspired with the presence of the Dev\={\i}, and see the whole place as pervaded by Her through and through. Then, repeat the Phat mantra and destroy the Celestial obstacles as well as those from the middle space (Antar\={\i}k\d{s}a); strike the ground thrice with the left heel and thus destroy the Terrene obstacles. Then touching the left branch on the left side of the chouk\=at, put the right foot forward and enter into the Mandapa. Then instal the \'S\=anti Kumbha (the peace jar) and offer the ordinary Arghya (S\=am\=any\=argha). Next worship the V\=astun\=atha and Padmayoni with flowers and \=Atapa rice and the Arghya water, on the south-west and then purify the Pa\d{n}cha Gavya. Next sprinkle all the Mandapa and the entrance gate with that Arghya water. And, while sprinkling with Arghya water, consider the whole space right through as inspired with the presence of the Dev\={\i} and repeat the M\=ula Mantra with devotion and sprinkle with Phat mantra. The Kart\=a, then, uttering the mantra ``Phat,'' is to drive away all the evils from the Mandapa and uttering the mantra ``H\=um'' sprinkle water, all around, thus pacifying the atmosphere and bringing peace into the hearts of all present.

Then burn the Dh\=upa incense inside and scatter Vikira (water, sandal-paste, yava, ashes, Durba grass with roots, and \=Atapa rice). Then collect all these rice, etc., again with a broom made of Ku\'sa grass to the north east corner of the Mandapa; making the Sankalpa and uttering Svasti v\=achana (invocation of good), distribute and satisfy the poor and orphans with feeding, clothing and money. Then he should bow down to his own Guru and take his seat humbly on the soft \=Asanam allotted to him with his face eastwards and meditate on the Deity (\=Ista Deva) of the mantra that is to be imparted to the disciple. After meditating thus, he is to do the Bh\=uta \'s\=uddhi (purification of elements) and perform Ny\=asa, etc., of the Deya mantra (the mantra that is to be imparted to the disciple) according to the rules stated below, i.e., the \d{R}i\d{s}i on the head; the chhandas in the mouth, the \=Ista Devat\=a in the heart, B\={\i}ja on the anus and \'Sakti Ny\=asa on the two legs. Then he is to make sound thrice by the clap of his palm and thus thwart off all the evils of the earth and the middle space and then make digbandhan (tieing up the quarters) by the mudr\=a chhotik\=a three times (snapping the thumb and forefinger together ). Then perform the Pr\=a\d{n}\=ay\=ama with the M\=ula mantra of the would-be-\=Ista-Devat\=a and do the M\=atrik\=a Ny\=asa in one's own body, thus :-- Om Am namah \'sirasi, Om \=Am namah on the face, Om Im namah on the right eye, Om \=Im namah on the left eye, and so on, assign all the letters duly to their respective places. Then perform the Kar\=anga Ny\=asa on the fingers and the Sada\d{n}ga Ny\=asa thus :-- Speak: Om Hriday\=aya namah, touching on the heart, utter Om \'Sirase sv\=ah\=a, touching the head; Om

\'Sikh\=ayai Va\d{s}at, touching the tuft; Om Kavach\=aya H\=um, touching on the Kavacha, ``Om netratray\=aya Vau\d{s}at,'' touching the eye, and ``Om Astr\=aya Phat'' touching both the sides of the hand, the palm and its back. Then finish the Nay\=asa by doing the Var\d{n}any\=asa of the M\=ula mantra in those places that are said in the cognate kalpas (i.e., throat, heart, arms, legs, etc.).

O N\=arada! Next consider within your body the seat of an auspicious \=Asana (a seat) and make the Ny\=asa of Dharma on the right side, J\~n\=anam on the left side, of Vair\=agyam (dispassion) on the left thigh, prosperity and wealth on the right thigh, of non-Dharma in the mouth and of Non-J\~n\=anam on the left side, Avair\=agyam (passion) on the navel, and poverty on the right side. Then think of the feet of the \=Asana (the body) as Dharma, etc., and all the limbs as Adharma (non-Dharma). In the middle of the \=Asana (body), i.e., in the heart consider Ananta Deva as a gentle bed and on that a pure lotus representing this universe of five elements. Then make Ny\=asa of the Sun, Moon, and Fire on this lotus and think the Sun as composed of twelve Kal\=as (digits) the Moon composed of sixteen Kal\=as (digits) and the Fire as composed of ten Kal\=as. Over this make Ny\=asa of S\=attva, R\=aja and T\=amo Gu\d{n}as, \=Atm\=a, Antar\=atm\=a, Param\=atm\=a and J\~n\=an\=atm\=a and then think of this as his \=Ista's altar where the devotee is to meditate on his \=Ista Devat\=a, the Highest Mother. Ny\=asa-assignment of the various parts of the body to different deities which is usually accompanied with prayers and corresponding gesticulations. Next the devotee is to perform the mental worship of the Deya Mantra Devat\=a according to the rules of his own Kalpa; next he is to show all the Mudr\=as, stated in the Kalpa for the satisfaction of the Deva. The Devas become very pleased when all these Mudr\=as are shown to them.

42-46. O N\=arada! Now, on one's left side, erect an hectagon; inside it a circular figure; inside this again a square and then draw within that square a triangle and over it show the \'Sankha Mudr\=a. After finishing the P\=uj\=a of the Six Deities at the six corners of the hectagon, Fire, etc., take the tripod of the \'Sankha (conch-shell) and sprinkling it with Phat mantra, place it within the triangle. Utter, then, the Mantra ``Mam Vahniman dal\=aya Da\'sa Kal\=atmane Amuka Devy\=a Arghyap\=atrasth\=an\=aya namah'' and thus worshipping the \'S\=ankhya vessel place it within the mandala. Then worship in the \'Sankha p\=atra, the ten Kal\=as of Fire, beginning from the East, then south-east and so on. Sprinkle the \'Sankha, conchshell, with the M\=ula Mantra and meditating on it, place the \'Sankha (conch shell) on the tripod. Repeating the mantra ``Am S\=urya mandal\=aya Dvada\'sakal\=atmane Amukodevy\=a

Arghyap\=atr\=aya namah'' worship in the Arghyap\=atra \'Sankha, sprinkle water in the \'Sankha with the Mantra ``Sam \'Sankhya namah.'' Worship in due order the twelve Kal\=as of the Sun Tapin\={\i}, T\=apin\={\i}, Dh\=umr\=a, etc., utter the fifty syllables of the M\=atrik\=a in an inverse order (i.e., beginning, see the S\=arad\=a Tilaka, with K\d{s}am, Ham, Sam, Sam, \'Sam, etc.,) and repeating the M\=ula Mantra also in an inverse order, fill the \'Sankha, three-fourths, with water. Next perform in it the Ny\=asa of Chandrakal\=a and uttering the Mantra ``Um Soma mandal\=aya Soda\'sakal\=atmane Amukademt\=ay\=a Arghy\=a-mrit\=aya namah,'' worship in this conchshell. Next with Anku\'sa mudr\=a, invoke all the t\={\i}rthas there, repeating the Mantra ``Gange Cha Yamune chaiva, etc.,'' and repeat eight times the M\=ula Mantra (the basic Mantra). Then perform the \'Sadamga Ny\=asa in the water and with the Mantra ``Hrid\=a namah, etc.,'' worship and, repeating eight times the M\=ula Mantra, cover it with Matsyamundr\=a. Next place on the right side of the \'Sankha, the Prok\d{s}a\d{n}\={\i} P\=atra (the Ko\'s\=a vessel from which water is taken for sprinkling) and put a little water in it. By this water sprinkle and purify all the articles of worship as well as one's own body and consider one's \=Atman as pure and holy.

47-81. After doing works thus far the until Vi\'se\d{s}\=arghya is placed, the devotee should erect Sarvato bhadra mandala within the altar and put the \'S\=ali rice within its pericarp. Next spread Ku\'sa grass on that Mandala and put on one Kurcha, looking well and auspicious within it, made of twenty-seven Ku\'sa grass knotted with Venyagra granthi. Worship here the \=Adh\=ara \'Sakti, Prakriti, K\=urma, \'Se\d{s}a, K\d{s}am\=a, Sudh\=asindhu, Ma\d{n}imandala, Kalpa vrik\d{s}a and \=Ista devat\=a and the P\={\i}tha. (Durg\=a Dev\={\i} yoga p\={\i}th\=aya namah). Then have an entire kumbha (waterjar) having no defect, wash it inside with Phat mantra, and encircle it with the red thread thrice as symbolising the three Gu\d{n}as. Place within this jar the Nava ratna (nine jewels) with Kurcha and worshipping it with scents and flowers put them in the jar repeating the Pra\d{n}ava, and place that on the P\={\i}tha (seat). Next consider the P\={\i}tha and Kumbha (waterjar) as one and the same and pour waters from the T\={\i}rthas, repeating in an inverse order the M\=atrik\=a Var\d{n}as (from K\d{s}a to Ka) and fill it, thinking of the \=Ista Deva and repeating the basic mantra, put the new and fresh twigs (Pallavas) of A\'svattha, Panasa and mango trees, etc., in the jar and cover its mouth and place over it fruits, rice, and cha\d{s}aka (honey) and wrap it with two red cloths. Then perform the Pr\=a\d{n}a-Pratisth\=a and invoke the Spirit of the Dev\={\i} by the Pr\=a\d{n}asth\=apana Mantra and show the Mudr\=as, \=Av\=ahana, etc., and thus satisfy the Dev\={\i}. Then do the Soda\'sopach\=ara P\=uj\=a of the Dev\={\i} after me-

ditating on the Parame\'svar\={\i} according to the rules of the Kalpa.

First offer ``welcome'' in front of the Dev\={\i} and then duly offer the P\=adya, Arghya, \=Achaman\={\i}ya water, Madhuparka, and oils, etc., for the bath. Then offer nice red silken clothes and various jewels, ornaments; repeating the M\=atrik\=a syllables electrified with the Deya Mantra, worship the whole body of the Dev\={\i} with scents and flowers. Next offer to the Dev\={\i} the sweet scent of Kal\=aguru mixed with camphor and the K\=a\'sm\={\i}ri sandalpaste mixed with Kast\=ur\={\i} and various nice scented flowers, for example, the Kunda flowers, etc. Then offer the Dh\=upa prepared from Aguru, Guggula, U\'s\={\i}ra, sandalpaste, sugar, and honey and know that the Dh\=upa is very pleasing to the Dev\={\i}. Next offer various lights and offerings of fruits, vegetables and fooding. Be particular to sprinkle everything with the water of the Kos\=a, thus purifying, before it is offered to the Dev\={\i}. Then complete the A\d{n}ga P\=uj\=a, and the \=Avara\d{n}a-P\=uj\=a of the Dev\={\i}, then perform the duty of Vai\'svadeva. On the right side of the Dev\={\i} erect an altar (sthandila) six feet square and instal Ag\d{n}i (Fire) there. Invoke there the Deity, thinking of Her Form and worship Her with scents and flowers. Then with the Vy\=arhiti Mantra with Sv\=ah\=a prefixed and M\=ula (Deya) Mantra perform the Homa ceremony with oblations, charu and ghee, twenty five times. Next perform Homa again with Vy\=arhiti. Next worship the Dev\={\i} with scents, etc., and consider the Dev\={\i} and P\={\i}tha Devat\=a as one and the same. Then take leave of (visarjana) the Ag\d{n}i (Fire). Offer valis (sacrifices) all round to the P\=ar\'svadas of the Dev\={\i} with the remnant charu of the Homa.

Now again worship the Dev\={\i} with five offerings and offer betel, umbrella, ch\=amara and others and repeat the M\=ula mantra thousand times. After finishing the Japam, place Karkar\={\i} (a water-jar with small holes at the bottom, as in a sieve) on the rice in the north-eastern corner and invoke the Dev\={\i} there and worship Her. Uttering the mantra ``Rak\d{s}a Rak\d{s}a'' moisten the place with water coming out of Karkar\={\i}, and repeat the Phat mantra. After re-worshipping the Dev\={\i}, place Karkar\={\i} in due position. Thus the Guru finishes the Adhiv\=asa (foregoing) ceremony and takes his meals with the disciple and sleeps that night on that altar.

82-106. O N\=arada! Now I am describing briefly about the Homa Kunda (a round hole in the ground consecrated to the Deity) and the Samsk\=ara ceremony of the Sthandila (the sacrificial altar). Uttering, first, the M\=ula Mantra, see, fix your gaze on the Kunda; then sprinkle it with water and the Phat mantra and drive away the evil-spirits from there. Then with mantra ``H\=um'' again sprinkle it with water.

Then draw within it three lines Pr\=agagra and Udagagra (on the eastern and northern sides). Sprinkling it with water and the Pra\d{n}ava, worship within the P\={\i}tha, uttering the mantras from \=Adh\=ara \'Saktaye namah to Amuka Dev\={\i} Yoga P\={\i}th\=aya namah. Invoke, in that P\={\i}tha, the Highest One, Who is \'Siva \'Siv\=a with all one-ness of heart and worship Her with scents and offerings. Then think for a moment the Dev\={\i} as having taken bath and as one with \'Sankara. Bring then fire in a vessel and taking a flaming piece thereof throw that in the south-west corner. Then purifying it by the gaze and quitting the portions of Kravy\=adah, impart the Chaitanya by ``Ram,'' the Vah\d{n}iv\={\i}ja repeat ``Om'' over it seven times. Shew, then, the Dhenumudr\=a and protect it by Phat K\=ara and cover, veil, it with the mantra ``H\=um.'' Then turn the fire, thus worshipped with sandalpaste, etc., thrice over the Kunda and with both the knees on the ground and repeating the Pra\d{n}ava, consider the Ag\d{n}i as the V\={\i}rya of \'Siva and throw it on the yoni of the Dev\={\i} in the P\={\i}tha. Then offer \=Achamana, etc., to the Deva and the Dev\={\i} and worship. Then light the flame with the mantra ``Chit Pingala Hana Hana Daha Daha Pacha Pacha Sarvaj\~n\=a J\~n\=apaya Sv\=ah\=a.'' Then utter the stotra to the Ag\d{n}i Deva with great love, repeating the mantra ``Agnim Prajvalitam vande J\=atavedam Hut\=a\'sanam suvar\d{n}a var\d{n}amamalam samiddham Visvatomukham.'' Then perform the Sadamgany\=asa to the Ag\d{n}i Deva ``Om Sahasr\=archchi\d{s}e namah, Om Svasti P\=ur\d{n}\=aya Sv\=ah\=a,'' ``Om Uttistha puru\d{s}\=aya va\d{s}at,'' ``Om Dh\=uma vy\=apine H\=um Om Sapta Jihv\=aya vau\d{s}at'' ``Om Dhanur dhar\=aya Phat.'' Repeating the above six mantras, perform the Ny\=asa on the heart, etc., the six places. Now meditate on the Ag\d{n}i as of a golden colour, three-eyed, seated on a lotus and holding in His four hands signs of granting boons, \'Sakti, Svastika and sign of ``no fear,'' also meditate on Ag\d{n}i, as the seat of the greatest auspiciousness. Then moisten the Kunda on the top of the belt (mekhal\=a) with water. Next spread the Ku\'sa grass all around and draw the Ag\d{n}i yantra over it, i.e., triangle, hectagon, circle, eight-petalled figure and Bh\=upura; rather have this drawing before the Ag\d{n}isth\=apan\=a. Now meditate this only. Then, within the Yantra, recite ``Vai\'sv\=anara J\=ataveda Lohit\=ak\d{s}a sarvakarm\=a\d{n}i S\=adhaya Sv\=ah\=a'' and worship Ag\d{n}i. Then worship in the centre and in the hectagon at the corners worship the Saptajihv\=a (seven tongues Hira\d{n}ya, Gagan\=a, Rakt\=a, Kri\d{s}\d{n}\=a, Suprabh\=a, Bahur\=up\=a, Atiraktik\=a) and next worship within the pericarp of the lotus the Anga Devat\=as. Then recite the following mantras within the eight petals :-- ``Om Agnaye J\=atavedase namah,'' `` Om Agnaye Saptajihv\=aya namah,'' `` Om Agnaye Havyav\=ahan\=aya

namah,'' ``Om Agnaye A\'svodaraj\=aya namah,'' ``Om Agnaye Vai\'sv\=anar\=aya namah,'' ``Om Agnaye Kaum\=ara tejase namah,'' ``Om Agnaye Vi\'svamukh\=aya namah,'' ``Om Agnaye Devamukh\=aya namah'' and considering the forms to hold \'Sakti and Svastik, worship them. Then consider Indra and the other Lokap\=alas (Regents of the several quarters) situated in the east, south-east, and so on together with their weapons, the thunderbolt and the other weapons, and thus worship them.

107-134. O N\=arada! Next purify the sacrificial ladles, etc., sruk, sruva, etc., and ghee; then, taking ghee by sruva, go on with the Homa ceremony. Divide the ghee of the \=Ajyasth\=al\={\i} (the vessel in which the ghee for the Homa purposes is kept) in three parts; take ghee from the right side and saying ``Om Agnaye Sv\=ah\=a'' offer oblations on the right eye of the Ag\d{n}i; take ghee from the left side and saying ``Om Som\=aye Sv\=ah\=a'' offer oblations on the left eye of the Ag\d{n}i; take ghee from the centre and saying, ``Om Agni\d{s}om\=abhy\=am Sv\=ah\=a,'' offer oblations on the central eye of the Agni. Take ghee again from the right side and saying ``Om Agnaye Svistakrite Sv\=ah\=a'' offer oblations to the mouth of the Ag\d{n}i. Then the devotee is to repeat ``Om Bhuh Sv\=ah\=a,'' ``Om Bhuvah Sv\=ah\=a,'' ``Om Svah Sv\=ah\=a'' and offer thrice the oblations; next he is to offer oblations thrice with the Ag\d{n}i mantra. After this, O Muni! for impregnation and each of the ten Samsk\=aras, natal-ceremony, tonsure, etc., he is to repeat the Pra\d{n}ava Mantra and offer the eight oblations of ghee on each occasion. Now hear of the tenfold Samsk\=aras :-- (1) Impregnation, (2) Pumsavan (a ceremony performed as soon as a woman perceives the foetus to be quick), (3) S\={\i}mantonnayana (a ceremony observed by women in the fourth, sixth or the eighth month of pregnancy), (4) J\=ata Karma (ceremony at the birth of a child), (5) N\=amakara\d{n}a, (naming the child), (6) Ni\d{s}kr\=ama\d{n}a (a ceremony performed when a new-born child is first taken out of the house into the open air usually in the fourth month), (7) Annapr\=a\'sana (when the rice is put in the mouth of the child), (8) Ch\=ud\=akara\d{n}a (the ceremony of the first tonsure), (9) Upanayana (holding the sacrificial thread; (10) God\=ana and Udv\=aha (gift of cows and marriage). These are stated in the Vedas. Next worship \'Siva P\=arvat\={\i}, the Father and the Mother of Ag\d{n}i and take leave of them. Next in the name of Ag\d{n}i, offer five Samidhas (fuel) soaked in ghee and offer one oblation of ghee to each of the \=Avara\d{n}a Devat\=as.

Then take the ghee by the \'Sruk and covering it with the \'Sruva, offer ten oblations to Ag\d{n}i, and Mah\=a Gane\'sa with mantras ending in Vau\d{s}at,

The Mah\=a Gane\'sa mantras run as follows :-- (1) Om, Om Sv\=ah\=a (2) Om \'Sr\={\i}m Sv\=ah\=a, (3) Om \'Sr\={\i}m Hr\={\i}m Sv\=ah\=a, (4) Om \'Sr\={\i}m Hr\={\i}m Kl\={\i}m Sv\=ah\=a, (5) Om \'Sr\={\i}m Hr\={\i}m Kl\={\i}m Glaum Sv\=ah\=a, (6) Om \'Sr\={\i}m Hr\={\i}m Kl\={\i}m Glaum Gam Sv\=ah\=a, (7) Om \'Sr\={\i}m Hr\={\i}m Kl\={\i}m Glaum ityantah Gam Ga\d{n}apataye Sv\=ah\=a, (8) Om Vara Varada ityantah Sv\=ah\=a, (9) Sarvajanam me Va\'sam ityanto Sv\=ah\=a and (10) \=Anaya Sv\=ah\=a ityantah.

Next perform in the Ag\d{n}i the P\={\i}tha P\=uj\=a and meditate on the Deya \=Istadeva and worship him. Next offer twenty-five oblations to his face, repeating the M\=ula Mantra. Then think of that and Ag\d{n}i Deva as one and the same, and then again as one with \=Atman. Then offer oblations to each of the Sadamga Devat\=as separately. Then search for the N\=adis (veins) of Vah\d{n}i and \=Ista Devat\=a and offer twenty one oblations. Then offer oblations to each of the two Devat\=as separately. Next offer one thousand and eight oblations to the \=Ista Deva with Til soaked in ghee or with the materials enumerated in the Kalpa. O Muni! Thus finishing the Homa ceremony, consider that the \=Ista Deva (the Dev\={\i}), Ag\d{n}i and the \=Avara\d{n}a Deities are all satisfied. Then, by the command of the Guru, the disciple is to take his bath and perform his Sandhy\=a, etc., and put on new clothes (cloth and ch\=adar) and golden ornaments. He is to come then, to the Kunda with Kamandalu in his hand and with a pure heart. He is to bow down to the elders and superiors seated in the assembly and take his seat in his \=Asana. \'Sr\={\i} Guru Deva then would look at the disciple with kind eyes and think the Chaitanya of the disciple within his own (the Guru's) body. Then the Guru Deva would perform the Homa and look at the disciple with a divine gaze, so that the disciple becomes pure-hearted and able to get the favours of the Devas. Thus the Guru must purify all the Adhvas (the passages) of the body of the disciple.

Then the Guru is to touch respectively the feet, generative organ, navel, heart, forehead, and the head of the disciple with K\=urcha (a bundle of Ku\'sa grass) and til soaked in ghee, in his left hand and offer at each touch eight oblations, repeating the mantra ``Om adya \'Si\d{s}yasya Kal\=adhv\=anam \'Sodhay\=ami Sv\=ah\=a, etc.'' Thus the Guru would purify Kal\=adhva (in the feet) Tattv\=adhva (in the generative organ), Bh\=uvan\=adhva (in the navel), Var\d{n}\=adhva (in the heart), Pad\=adhva (in the fore-head) and Mantr\=adhva (on the head), the six Adhv\=as and think these all to be dissolved in Brahm\=a (Brahmal\={\i}na).

135-155. Then, again, the Guru would think all these to be re-born from Brahm\=a and transfer the Chaitanya of the disciple that was in him to the disciple. Then the Guru must offer P\=ur\d{n}\=ahuti and consider

the \=Ista Devat\=a, placed in the fire by the visarjana mantra for the Homa purposes, as entered into the water-jar. He is to perform again the Vy\=arhiti Homa and offer all the Amg\=ahutis (oblations to all the limbs) of the fire and take leave of the fire withdrawing the Deity from the jar, into his own body. Uttering then the Vau\d{s}at Mantra he would tie the eyes of the disciple with a piece of cloth and would bring him from the Kunda to the mandala and make the disciple offer pu\d{s}p\=a\d{n}jali (flowers in his palm) to the \=Ista deva. Then he would take away the bandage or piece of cloth from his eyes and ask him to take his seat in the seat Kus\=asana. Thus the Guru, after having purified the elements of the body of the disciple and performed the Ny\=asa of the Deya Mantra, would make the disciple sit in another mandala. Then he would touch the head of the disciple with the twigs (Pallavas) of the Kunda and repeat the M\=atrik\=a Mantra and make him have his bath with the water of the jar which is considered as the seat of the \=Ista Deva. Then, for the protection of the disciple, he would sprinkle (abhi\d{s}eka) him with the water of the Vardhani vessel placed already in the north-east corner. Then the disciple would get up and put on the pair of new clothes and besmear his whole body with ashes and sit close by the Guru. When the merciful Guru would consider that the \'Siva \'Sakti has now passed out of his own body and that Divine Force, the Dev\={\i}, has entered into the body of the disciple, i.e., charged the disciple with the pass. Thinking now the disciple and the Devat\=a to be one and the same, the Guru would now worship the disciple with flowers and scents. The Guru would then place his right hand on the head of the disciple and repeat clearly in his right ear the Mah\=a Mantra of the Mah\=a Dev\={\i}. The disciple is to repeat also the Mah\=a Mantra one hundred and eight times and fall prostrate on the ground before the Guru and thus bow down to the Guru, whom the disciple now thinks as the incarnate of the Deva.

The disciple, the devotee of the Guru, would now give as a Dak\d{s}i\d{n}\=a all his wealth and property for his whole life to the Guru. Then he would give Dak\d{s}i\d{n}\=a to the priests and make charities to the virgins; the Br\=ahma\d{n}as, the poor and the destitute and the orphans. Here he is not to be miserly in any way in the expenditure. O N\=arada! Thus the disciple would consider himself blessed and he would daily remain engaged in repeating the Mah\=a Mantra. Thus I have described to you above
The most excellent D\={\i}k\d{s}\=a. Thinking all these, you are to remain ever engaged in worshipping the lotus feet of the Great Dev\={\i}. There is no Dharma higher than this in this world for the Br\=ahma\d{n}as. The followers of the Vedas would impart this Mantra according to the rules stated respectively in their own Grihya S\=utras; and the T\=antrikas

would also do the same according to their own Tantras. The Vaidiks should not follow the Tantra rules and the Tantriks are not to follow the Vaidik rules. Thus all the \'S\=astras say. And this is the San\=atan Creed. N\=ar\=aya\d{n}a said :-- O N\=arada! I have described all about the ordinary D\={\i}k\d{s}\=a that you questioned me. Now the essence in brief is this that you would remain always merged in worshipping the Par\=a \'Sakti, the Highest Force, the Mah\=a Dev\={\i}. What more shall I say than this that I have got the highest pleasure and the Nirv\=ana, the peace, that passeth all understanding, from my daily worshipping That Lotus Feet duly. Vedavy\=asa said :-- O Mah\=ar\=aja! O Janamejayan! After having said this D\={\i}k\d{s}\=atattva, the highest Yogi Bhagav\=an N\=ar\=aya\d{n}a, meditated by the Yogis, closed his eyes and remained merged in Sam\=adhi, in the meditation of the Lotus Feet of the Dev\={\i}.

Knowing this Highest Tattva, N\=arada, the chief of the \d{R}i\d{s}is, bowed down at the feet of the Great Guru N\=ar\=aya\d{n}a and went away immediately to perform the tapasy\=a so that he also might see the Mah\=a Dev\={\i}.

Here ends the Seventh Chapter of the Twelfth Book on the D\={\i}k\d{s}\=a vidhi or on the rules of Initiation in the Mah\=apur\=a\d{n}am \'Sr\={\i} Mad Dev\={\i} Bh\=agavatam of 18,000 verses by Mahar\d{s}i Veda Vy\=asa.



