\chapter{On the battle of Haihaya and K\=alaketu}

1. Vy\=asa said :-- O King! That powerful son of Lak\d{s}m\={\i}, Haihaya, became very glad to hear these words of Ya\'sovat\={\i} and said :--

2-14. O One of beautiful thighs! Hear in reply to your query :-- I am Haihaya, the son of Lak\d{s}m\={\i}, and I am known in this world by the name of Ekav\={\i}ra. Now you have made my mind dependent. What am I to do now? where to go? Thus distressed with bereavement from your dear companion, my mind is struck with Cupid's arrows and is confounded with her extraordinary beauty that you just now described. Next you described her qualifications and my mind is ravished. Again when you described before me what she uttered in the presence of the R\=ak\d{s}asa, I am struck with great wonder. Your dear companion Ek\=aval\={\i} said before the vicious D\=anava K\=alaketu, ``I have already selected the King Haihaya. I will not select any other than him, this is my firm resolve.'' These words have converted me into her slave. O sweet-haired One! Say now what service can I do to you both? I am not acquainted with that wicked demon's palace; never I went to his city. O Fair-eyed One! Say how I can go there; for you are the only one that can lead me there. Therefore take me quickly to that place where your beautiful clear companion is staying. Your dear companion, the daughter of the King is very much afflicted with sorrow; soon I will free her, by destroying that cruel R\=ak\d{s}asa. There is no doubt in this. O Auspicious One! I will rescue your dear companion and bring her to the city of yours and hand her over to the hands of her father. Then that King, the enemy destroyer, will perform the marriage ceremony of his daugther. I think this is the desire of your heart. O Sweet-speaking One! Know that that is also my desire. O Beautiful One! Now that desire will be fulfilled by your efforts. Show me quickly that place and see my prowess. O One with a face beautiful like the Moon! It seems that you will be able to do my work. Soon do such as I can kill that wicked demon, who steals other\'s wives. Now show me the way to the impassable city of that R\=ak\d{s}asa.

15-26. Vy\=asa said :--O King! Hearing the sweet words of the prince, Ya\'sovat\={\i} became very glad and gently began to speak out how he could go to the demon's city. O King! Take the success-giving Mantra of Bhagavat\={\i} and I would then be able to show you today the city guarded by the R\=ak\d{s}asas. O King! Better arrange to take your vast

army with you; for you will have to fight no sooner you go there. K\=alaketu is personally a great warrior surrounded by R\=ak\d{s}asas of great power and strength. Therefore be initiated in the Mantram of \'Sr\={\i} Bhagavat\={\i} and accompany me. So you will surely be successful. I will show you the way to the city of that Demon. Slay that vicious and vilest of the R\=ak\d{s}asas and rescue my dear companion. Hearing thus, Haihaya was duly initiated into the great Mantram of Yoge\'svar\={\i}, named Trilokitilaka Mantra (Hr\={\i}m Gaur\={\i} Rudradayite Yoge \'Svar\={\i} H\=um Phat Sv\=ah\=a is the Yoge\'svar\={\i} Mantra), by Mahar\d{s}i Datt\=atreya, accidentally come there (as if ordained by Fate), the chief of J\~n\=anins (the Gnostics), that is conducive to the welfare of the beings. Thus by the influence of the Mantram the King got the power of knowing all things and going everywhere with unobstructed speed. Then the King Haihaya quickly went with Ya\'sovat\={\i} to the impassable city of the R\=ak\d{s}asas, accompanied by a vast army. The city was surrounded by snakes and guarded by the terrible R\=ak\d{s}asas like the city of P\=at\=ala. The messengers of the R\=ak\d{s}asa, seeing the King coming, were struck with terror and crying aloud quickly went to K\=alaketu. K\=alaketu, struck with Cupid's arrows, was sitting beside Ek\=aval\={\i} and was speaking many modest words when the messenger went there suddenly and said :-- ``O King! The attendant of this lady Ya\'sovat\={\i} is coming here with a prince and an army.

27-29. O King! We cannot tell exactly whether the prince is the son of Indra, named Jayanta or K\=artikeya. After all, puffed up with the strength of his army, he is coming here. O King! The battle is imminent; now make your arrangements fully and carefully; fight with the son of a Deva or abandon this lotus-eyed Lady. O King! At a distance of three Yojanas from this place, be is staying with his army. Now equip yourself and quickly declare the war by blowing the war trumpets.''

30-36. Vy\=asa said :-- O King! Hearing the messenger's words, K\=alaketu, the King of the Demons, became overwhelmed with anger and at once sent many powerful R\=ak\d{s}asas, holding all sorts of weapons and spoke out to them :-- ``O R\=ak\d{s}asas! With weapons in your hands, go before them quickly.'' Ordering them thus, K\=alaketu asked in sweet words Ek\=aval\={\i} who was in front and very distressed. O Thin-bellied One! Who is coming here? Is he your father or any other man coming with his army to release you. Speak this to me truly. If your father comes here to take you back, being very much distressed with your bereavement, I will never fight with him, if I come to know this truly; rather I will bring him to my house and worship him with the excel-

lent horses, gems and jewels and clothings. Really I will show my full hospitality duly to him when he comes here. And if any other person comes, then I will take his life by the sharpened arrows; there is no doubt in this. Know this as certain whoever comes here for your rescue is brought by the hand of Death to me. Therefore, O Large-eyed One! Say who is this fool that is coming, not knowing me as the powerful and unconquerable K\=ala (Death).

37-38. Ek\=aval\={\i} said :-- ``O Highly Fortunate One! I do not know who is this body coming to this side with a violent speed. O King! How can I know that when I am in this state of confinement in your house. This man is not my father nor my brother. Some other powerful man is coming here. I do not know exactly what for he is coming.''

39-40. The Demon said :-- My messengers say that your comrade Ya\'sovat\={\i} has taken with her that warrior and is coming to this side with great energy. Where has your clever companion gone now? O Lotus-eyed! There is no enemy in the three worlds strong enough to fight against me.

41-66. Vy\=asa said :-- O King! Just then other messengers hurriedly came there terrified and spoke to K\=alaketu who had been staying in the house, thus :-- ``O King! The army has come quite close to the city and how are you staying in the house, calm and quiet? Better march out of the city with your vast army as early as possible.'' The powerful K\=alaketu, then, hearing their words, mounted on the chariot and quickly went out of his city. The King Haihaya, on the other hand, suffering from the bereavements of his dear lady, suddenly came there mounted on horseback. The terrible fight ensued then and there between the two and each one struck the other with sharpened weapons and the quarters all around blazed with their glitterings and clashings. When the terrible fight was going on, Haihaya, the son of Lak\d{s}m\={\i}, struck K\=alaketu, the King of the Daityas with a very powerful club (Gad\=a). Thus struck by the Gad\=a, the Lord of the Daityas fell on the ground like a mountain, struck by lightning, and died. All the R\=ak\d{s}asas fled away on all sides, struck with terror. Ya\'sovat\={\i} went then very hurriedly with a gladdened heart to Ek\=aval\={\i} and began to speak to her in terms of surprise and in sweet words :-- O Dear! O Dear! Come, Come; the great warrior, the prince Ekav\={\i}ra has killed the Lord of the Daityas in a dreadful battle. That King is now waiting, tired in the midst of his soldiers. He has already heard from me about your beauty and qualities; and now he is expecting to see you. O One Looking askance! Now satisfy your eyes and mind by seeing that King who is like the Cupid. When

I described to him before on the banks of the Ganges your beauty and qualifications, he got enamoured of you and now he is suffering from bereavements and wants to see you. Thus, hearing, Ek\=aval\={\i} determined to go to him and as she was yet unmarried, she became abashed and afraid. She thought how could she see the prince as she was unmarried. It might be that he being passionate would catch her by her arms. Thus, troubled with thought, that daughter of the King, with a sad look, and wearing poor clothes, Ek\=aval\={\i} went with Ya\'sovat\={\i} on a palanquin, carried on men's shoulders. Seeing that large-eyed daughter of the King coming there, the prince said :-- ``O Beautiful One! My two eyes are very thirsty to see you. Satisfy my eyes and mind by showing yourself to me.'' Seeing the prince passionate and the King's daughter very much abashed, Ya\'sovat\={\i}, who knew the rules of modesty, thus spoke to the prince :-- `` O Prince! The father of my dear companion expressed a desire to betroth her to your hands. She is also obedient to you. Therefore your meeting will certainly take place. O King! Wait; take her to her father; and he will perform duly the marriage ceremony and betroth her to your hands. Know this to be quite certain.'' The King took her words to be quite just and true and taking those two ladies went with his army to the house of the father of Ek\=aval\={\i}. Ek\=aval\={\i}'s father became very glad and cheerful to learn that his daughter was coming and, accompanied by his ministers, went hurriedly to her. After a long time the King saw his daughter in poor clothings and became highly pleased. Ya\'sovat\={\i} then described in detail all what happened before the King. The King then with his minister brought with great love, courtesy and gentleness Ekav\={\i}ra to his house and on an auspicious day performed the marriage ceremony of him with Ek\=aval\={\i}, in accordance with due ceremonies and rites. Then the King gave away many clothings, ornaments, jewels, and articles for fitting a house and many other things and worshipped duly and sent his daughter together with Ya\'sovat\={\i} away with the King Haihaya. Thus the marriage ceremony was performed and the son of Lak\d{s}m\={\i} gladly returned to his house and began to enjoy many pleasures with his wife. Then, in course of time, in the womb of Ek\=aval\={\i} the King Haihaya got a son named Kritav\={\i}rya. The son of this Kritav\={\i}rya is known as K\=artav\={\i}rya. O King! Thus I have narrated to you the origin of the Haihaya dynasty.

Here ends the Twenty-third Chapter in the Sixth Book on the battle of Haihaya and K\=alaketu in the Mah\=a Pur\=a\d{n}am \'Sr\={\i} Mad Dev\={\i} Bh\=agavatam by Mahar\d{s}i Veda Vy\=asa.



