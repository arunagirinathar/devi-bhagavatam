\chapter{On the description and Dh\=umralochana giving the news}

1-12. Vy\=asa said :-- O King! The messenger was thunderstruck with Dev\={\i}'s words and said :-- ``O Beautiful Lady! What art Thou speaking? It seems that Thou dost not think on this matter, owing to Thy feminine nature. O Dev\={\i}! Thou art boasting in vain; how canst Thou expect to conquer \'Sumbha in a battle when he has conquered Indra and other Devas and many other D\=anavas? Lotus-eyed One! There is no hero in the three worlds that can conquer \'Sumbha in battle; Thou seemest to be a mere trifle before that King of Demons in a face-to-face fight. O Fair One! Nowhere ought to be said any words without being thought over; one must weigh one's own and other's might and then speak accordingly. The King \'Sumbha, the Lord of the three worlds, enchanted by Thy fascinating beauty, is desirous of Thee; therefore dost Thou fulfill his desires and become his beloved wife. Thou better now abandonest Thy illiterate nature and worhippest \'Sumbha or Ni\'sumbha; I am speaking for Thy good; so keep my words. The amorous love sentiment is the best of nine prevailing sentiments. Therefore every intelligent being ought to cherish with gladness this amorous feeling. And if Thou, O Weak girl! dost not go to \'Sumbha, then that Lord of the Earth will become very angry and will order his servants to take Thee perforce before him. O Fair One! Those proud Demons will carry Thee by holding Thy hair before \'Sumbha; there is no doubt in this. O thin bodied One! Better forego Thy boldness in every way and keep Thy self-respect. Thou art the object of respect and admiration and so should go before him. What difference is there between the fight which makes one's body liable to cuts and wounds by sharp arrows and pleasures that arise out of sexual intercourse! These are like the two opposite poles; therefore judge what is useless and what is useful and keep my good advice. Thou shalt be exceedingly happy if Thou servest \'Sumbha or Ni\'sumbha.''

13-19. The Dev\={\i} spoke :-- ``O Messenger! You are fortunate; you are well trained to speak out truth; I know full well that \'Sumbha and Ni\'sumbha are strong. Still out of My childish nature, the promise that I made before I cannot undo it. Therefore speak to the powerful \'Sumbha or Ni\'sumbha that none can be my husband simply from his beauty without defeating Me in battle no one can marry Me. So conquer Me soon and marry Me as you like. Though of a weaker sex, I have come here

to fight; know this as certain. Therefore if you be capable, fight and do the duty of a warrior. And if you be terrified by seeing my trident or if you want your life, quit the Heavens and this earth and go down to P\=at\=ala without any delay. O Messenger! Go just now to your master and tell him sweetly these words. Then that powerful Lord of the D\=anavas will judge what ought to be done. O Knower of Dharma! To speak out truth before an enemy, before one's own master is certainly the duty of a messenger in this world; therefore go quickly and tell him what are verily true.''

20-21. Vy\=asa said :-- O King! The messenger was quite surprised to hear the bold words, though full of reason and morals, of the Dev\={\i}, puffed up by the vanity of Her strength and departed. Coming to the Lord of the Daityas, the messenger bowed down before his feet and told him sweet words, full of morals, in a very humble way after pondering over and over again what he would say.

22-29. The messenger said :-- O King! Words, true and the same time sweet, ought to be spoken before one's master; but these are very rare in this world. On the other hand, if unpleasant words be spoken, the King gets very angry. So I am very anxious now. O King! Whether that lady is weak or strong, whence has She come, whose wife is She, I have not been able to ascertain all these. How then can I say about Her conduct? But, on seeing that harsh speaking woman, I have come to understand this much that She is exceedingly haughty and has come to fight. O King! You are very intelligent; therefore judge what ought to be done after hearing only what that lady has told me to speak to you. That Lady said :-- ``In days of childhood, while I was playing, out of my childish nature, I promised before my comrades that I would marry that valiant warrior who would defeat me thoroughly in a battle and thus curb My pride. O best of kings! You are religious; so you ought to make My word turn out false. Bring Me under your control by defeating Me in a battle.'' O King! Hearing these words I have returned; now do whatever you like. That Lady is determined to fight and is waiting there, firmly mounted on a lion, and with various weapons in Her hands. Now judge and do what is best.

30. Vy\=asa said :-- O King! Hearing thus the words of Sugr\={\i}va, the king \'Sumbha asked his hero brother Ni\'sumbha who was close by.

31-32. O Brother! You are intelligent; speak out truly what ought to be done now? The lovely woman is challenging us to fight. Shall I go to fight or you would go with forces? I will do whatever you say.

33-34. Ni\'sumbha said :-- O King! It is not proper that you or I would go to the battlefield. Better send Dh\=umralochana to the field quickly. Let that hero go there and defeat that beautiful Lady and bring Her here. You can then marry Her.

35. Vy\=asa said :-- Hearing thus his younger brother's words, \'Sumbha filled with anger, instantly sent Dh\=umralochana who was close by to battle.

36-40. \'Sumbha said :-- ``O Dh\=umralochana! Take a vast army and go at once to the battlefield and bring that stupid Lady, vainly boasting of Her strength. If any Deva, D\=anava or any other powerful human being take Her side, kill him instantly. Slay Her companion the Goddess K\=al\={\i} and bring Her too. Do all these responsible duties and return quickly. That Chaste Lady is to be protected by all means. The body of that thin Lady is very delicate; so shoot arrows at Her very carefully and see that they are not sharp. But kill those that will help Her with weapons in their hands. Try your best to protect Her, never to kill Her.

41-60. Vy\=asa said :-- O King! No sooner ordered thus by the king, Dh\=umralochana bowed down to the king, and, accompanied by sixty thousand D\=anava forces, quickly went to the battlefield and saw there that the Lady was sitting in a beautiful garden. Seeing that deer-eyed Lady, Dh\=umralochana began to address Her with great humility and in sweet words full of reason and goodness. O Dev\={\i}! O highly Fortunate One! Hear! \'Sumbha is very much distressed owing to Thy absence. Lest there be any break in the love sentiments, that King, a wise statesman, sent a messenger with instructions to speak Thee in sweet and suitable terms; but, O fair One! That messenger, on arriving before the King had told all the contrary words. O Knower of love sentiments! Hearing thus the messenger's words, my lord \'Sumbha, sick with love, has become immersed in cares and anxieties. That messenger had not been able to realise the true meaning of Thy words. O honourable Lady! The sentence uttered by Thee, ``He who will conquer me in battle'' is full of deep meanings; he was stupid; hence he could not realise the meaning of the word ``battle'' intended by Thee. O Beautiful One! ``Battle'' means two different things according to persons for whom it is intended; it is of two kinds :-- One out of excitement and another out of sexual intercourse. With Thee, the sexual intercourse is intended; and with any other enemy, excitement in a real fight is meant. Out of these, the fight of sexual intercourse is full of sweetness and the fight with enemies is painful. O Beautiful One! I know Thy intentions fully. In Thy heart reigns

that fight of sexual intercourse. Knowing me as expert in these affairs, the king \'Sumbha has sent me today to Thee with a vast army. O highly Fortunate Lady! Thou art clever and shrewd; hear my gentle words; serve \'Sumbha, the lord of the three worlds, the destroyer of the Deva's pride. Thou wilt be the dearest queen-consort and enjoy the best pleasures. The powerful \'Sumbha knows the real meaning of the fight of sexual intercourse; so he will easily conquer Thee. When Thou wilt shew various amorous gestures, he will also show his feelings. And the K\=alik\=a Dev\={\i}, your companion will remain with Thee as a helping mate in your vital pleasures. The lord of the Daityas, expert in the science of love, will certainly conquer Thee engaged in amorous fight and will lay Thee stretched on a soft bedding and will make Thee tired; he will make Thy body covered with blood by striking with nails and he will bite Thy lips to pieces; then Thou wilt perspire profusely and wilt cease fighting. Thus Thy mental desire for fight - sexual intercourse - will be satisfied. O Beloved! At Thy mere sight \'Sumbha will be completely subject to Thee. Therefore dost Thou keep my sweet and beneficial words. Thou art an honourable Lady; and Thou wilt be highly honoured by all if Thou marryest \'Sumbha. Those are certainly very unfortunate who like fighting with weapons. O Beloved! The sexual intercourse is always favourite to Thee; therefore it is not worthy of Thee to fight with weapons. Therefore dost Thou make the king free of sorrows by pouring on him Thy mouth nectar and by making his heart bud forth by Thy kicking, as Bakula and Kurubaka trees blossom forth when drenched with mouth nectar and Asoka tree gets blossomed by the kicking of women.

Here ends the Twenty-fourth Chapter of the Fifth Book on the description and Dh\=umralochana giving the news in \'Sr\={\i} Mad Dev\={\i} Bh\=agavatam, the Mah\=a Pur\=a\d{n}am of 18,000 verses by Mahar\d{s}i Veda Vy\=asa.



