\chapter{On the appearance of the Highest \'Sakti}

1-8. Janamejaya spoke to Veda Vy\=asa :-- O Bhagav\=an! Thou art the knower of all the Dharmas and Thou art the chief, the crown of the Pundits, knowing all the \'S\=astras. Now I ask Thee how is it that the twice-born have ceased to worship the Highest \'Sakti, the G\=ayatr\={\i} and they now worship the other Devat\=as, on the face of the distinct command in the \'Srutis that the worship of the G\=ayatr\={\i} is nity\=a, that is, daily to be done at all times, especially during the three Sandhy\=a times, by all those that are twice-born?

In this world some are the devotees of Vi\d{s}\d{n}u, some, the followers of Ga\d{n}apat\={\i}, some are K\=ap\=alikas, some follow the doctrines prevalent in China; some are the followers of Buddha or Ch\=arv\=aka; some of them again wear the barks of trees and others roam naked. So various persons are seen having no trace of faith in the Vedas.

O Br\=ahma\d{n}a! What is the real cause underlying secretly here in this! Kindly mention this to me. Again there are seen many men, well versed in various metaphysics and logic, our B.A.s and M.A.s but then, again, they have no faith in the Vedas. How is this? Nobody wants anything ominous to him consciously. But how is it that these so-called learned men are

fully aware and yet they are wonderfully void of any trace of faith in the Vedas? Kindly mention the cause underlying this, O Thou! The foremost of the knowers of the Vedas.

There is, again, another question :-- Thou hadst described before the glories of Ma\d{n}idv\={\i}pa, the highest and the best place of the Dev\={\i}. Now I want to hear how is that Dv\={\i}pa greater than the great. Satisfy this servant of thine by describing these. If the Guru be pleased, he reveals even the greatest and the highest esoteric secret to his disciple.

9-10. S\=uta spoke :-- Hearing the words of the King Janamejaya, the Bhagav\=an Veda Vy\=asa began to answer the questions in due order. The hearing of this increases the faith of the twice-born in the Vedas.

11-30. Vy\=asa said :-- Well has this been asked by you, O King! in due time and in an appropriate moment. You are intelligent and it seems that you have got the faith in the Vedas. I now answer. Listen. In ancient days, the Asuras, maddened with pride, fought against the Devas for one hundred years. The war was very extraordinary and remarkable. In this great war various weapons were used, variegated with numerous M\=ay\=as or ingenious devices. It tended to destroy the whole world. By the mercy of the Highest and the Most Exalted \'Sakti, the Daityas were overcome by the Devas in that Great War. And they quitted the Heavens and the Earth and went to the nether regions, the P\=at\=ala. The Devas were all delighted and began to dwell on their own prowesses and became proud. They began to say :-- ``Why shall not victory be ours. Why are not our glories great? We are by far the best! Where are the Daityas? They are devils, powerless. We are the causes of creation, preservation and destruction. We all are glorious! Oh! What can be said before us in favour of the Asuras, the devils?'' Thus, not knowing the Highest \'Sakti, the Devas were deluded. At this moment, seeing this plight of the Devas, the World Mother took pity on the Devas and, to favour them, O King! She appeared before them in the form of the Most Worshipful, the Great Holy Light. It was resplendent like ten million Suns, and cool as well like ten million Moons. It was brilliant and dazzling like ten million lightning flashes, without hands and feet, and exceedingly beautiful! Never was this witnessed before! Seeing this Extraordinary Beautiful Lovely Light, the Devas were taken aback; they spoke amongst themselves, thus :-- ``What is this! What is this! Is this the work of the Daityas or some other great M\=ay\=a (Mayic) played by them or is it the work of another for creating the surprise of the Devas!'' O King! Then they all assembled together and decided to approach towards that Adorable

Light and to ask It what It was. They, then, would determine its strength and decide what to do afterwards. Thus, coming to this ultimate conclusion, Indra called Ag\d{n}i and said :-- ``O Ag\d{n}i! You are the mouth-piece of the Devas. Therefore do you go first and ascertain distinctly what this Light is.'' Hearing thus the words of Indra, Ag\d{n}i, elated by his own prowess, set out immediately from the place and went to that Light. Seeing Ag\d{n}i coming, the Light addressed him thus :-- ``Who are you? What is your strength? State this before Me.'' At this Ag\d{n}i replied :-- ``I am Ag\d{n}i. All the yaj\~n\=as, ordained in the Vedas are performed through me. The power of burning everything in this universe resides in me.'' Then that adorable Light took up a straw of grass and said :-- ``O Ag\d{n}i! If you can burn everything in this universe, then do you burn this trifling straw.'' Ag\d{n}i tried his best to burn the straw but he could not burn it. He got ashamed and fast went back to the Devas. Asked by the Devas, Ag\d{n}i told them everything and said :-- ``O Devas! Know verily that the pride cherished by us that we are supreme, is entirely false.''

31-50. Indra then asked V\=ayu (wind) and said :-- ``O V\=ayu! You are dwelling in this universe, through and through; by your efforts, all are moving; therefore you are the Pr\=a\d{n}a of all; it is possible that all forces are concentrated within you. Go and ascertain what is this Light? Verily I do not see any other person here than you who can ascertain this great adorable Light.'' Hearing these commendable words of Indra, V\=ayu felt himself elated and went at once to that place where was that Light. Seeing the V\=ayu, the Light, the Yak\d{s}a, the demi-god, the Spirit asked in a gentle language :-- ``Who are you? What strength is there in you? Speak out all these to me.'' At this, V\=ayu spoke arrogantly, ``I am M\=atarisvan, I am V\=ayu; about my strength, I can move anything and I hold everything. It is through the strength of mine, that this universe is, and is alive and brisk with movements and works.'' That Highest Mass of Light then replied :-- ``O V\=ayu! Move this straw that lies before you, and if you cannot, quit your pride and go back to Indra ashamed.'' At this V\=ayu tried all his might but, alas! He could not move the straw a bit from that place!

V\=ayu then gave up his pride and returned to the Devas and spoke to them all about the Yak\d{s}a (a sort of demi-god, a ghost). O Devas! Our pride is vain; in no way can we be able to ascertain the nature of that Light. It seems that that Holy Light, adorable by all, is extraordinary. Then all the Devas spoke with one voice to Indra :--

``When You are the King of the Devas, better go yourself and ascertain the reality of Its Nature.'' Indra, then, with great pride, went himself to the Light; the Light, too, began to disappear gradually from the place, and ultimately vanished from Indra's sight. When Indra found that he could not even speak to That Light, he became greatly ashamed and began to conceive of his own nothingness. He thought thus :-- ``I won't go back to the Devas. What shall I say to them? Never will I disclose to them my inferiority; one is better to die than do this. One's self-honour is the only treasure of the great and honourable. If honour is gone, what use, then, is there in living?'' O King! Then Indra, the Lord of Devas, quitted his pride and took refuge unto That Great Light which exhibited, ere long, such a glorious character. At this moment, a celestial voice was heard from the Heavens :-- ``O Indra! Go on now and do the japam, the reciting of the M\=ay\=a V\={\i}ja Mantra, the basic Mantra of M\=ay\=a. All your troubles will, then, be over.'' Hearing this celestial voice, Indra began to repeat the M\=ay\=a V\={\i}ja, the Seed Mantra of M\=ay\=a, with rapt concentration and without any food.

51-61. Then on the ninth lunar day of the month of Chaitra when the Sun entered the meridian, suddenly there appeared in that place a Great Mass of Light as was seen before. Indra saw, then, within that Mass of Light, a Virgin Form in full youth. The lustre from Her body was like that of ten million Rising Suns; and the colour was rosy red like a full-blown Jav\=a flower. On Her forehead was shining the digit of the Moon; Her breasts were full, and, though veiled under the cloth, they looked very beautiful. She was holding noose and a goad in Her two hands and Her other two hands indicated signs of favour and fearlessness.

Her body was decked with various ornaments and it looked auspicious and exceedingly lovely; nowhere can be seen a woman beautiful like Her. She was like a Kalpa Vrik\d{s}a (celestial tree yielding all desires); she was three eyed and Her braid of hair was encircled with M\=alat\={\i} garlands. She was praised on Her four sides by the Four Vedas, Incarnate, in their respective Forms. The brilliancy of Her teeth shed lustre on the ground as if ornamented with Padmar\=aga jewels. Her face looked smiling. Her clothing was red and Her body was covered with sandalpaste. She was the Cause of all causes. Oh! She was all Full of Mercy. O King Janamejaya! Thus Indra saw, then, the Um\=a Parvat\={\i} Mahe\'svar\={\i} Bhagavat\={\i} and the hairs of his body stood on ends with ecstasy. His eyes were filled with tears of love and deep devotion and he immediately fell prostrate before

the feet of the Dev\={\i}. Indra sang various hymns to Her and praised Her. He became very glad and asked Her, ``O Fair One! Art Thou that Great Mass of Light? If this be, kindly state the cause of Thy appearance.'' O King! Hearing this, the Bhagavat\={\i} replied.

62-83. This My Form is Brahm\=a, the Cause of all causes, the Seat of M\=ay\=a, the Witness of all, infallible and free from all defects or blemishes. What all the Vedas and Upani\d{s}adas try to establish, what ought to be obtained, as declared by all the rules of austerity, and for which the Br\=ahma\d{n}as practise Brahmacharyam, I am all that. I have told you about that Brahm\=a, of the nature of the Great Holy Light. The sages declare that That Br\=ahma\d{n} is revealed by ``Om'' and ``Hr\={\i}m'', the two V\={\i}jas (mystic syllables) that are My two first and foremost Mantras wherein I remain hidden. I create this universe with My two parts (in My two aspects); therefore My V\={\i}ja mantra is two. ``Om'' V\={\i}ja is denominated as Sachchid\=ananda (everlasting existence, intelligence and bliss) and ``Hr\={\i}m'' V\={\i}ja is M\=ay\=a Prakriti, the Undifferentiated Consciousness, made manifest. Know, then, That M\=ay\=a as the Highest \'Sakti and know Me as that Omnipotent Goddess at present revealed before your eyes. As moonlight is not different from the Moon, so this M\=ay\=a \'Sakti in the state of equilibrium is not different from Me. (The powerful man and the power he wields are not different. They are verily one and the same.) During Pralaya (the Great Latency period), this M\=ay\=a lies latent in Me, without there being any difference. Again at the time of creation, this M\=ay\=a appears as the fructification of the Karmas of the J\={\i}vas. When this M\=ay\=a is potential and exists latent in Me, when M\=ay\=a is Antarmukh\={\i}, it is called Unmanifested and when the M\=ay\=a becomes Kinetic, when the M\=ay\=a is Bahirmukh\={\i}, when She is in an active Kinetic state, it is said to be Manifested. There is no origin or beginning of this M\=ay\=a. M\=ay\=a is of the nature of Brahm\=a in a state of equilibrium. But, during the beginning of the creation, Her form consisting of the several Gu\d{n}as appears, when M\=ay\=a is Bahir Mukh\={\i}, She becomes T\=amas, in Her Unmanifested state. O Indra! For this reason Her state of abstraction, and becoming introspective, this is Her Antarmukh\={\i} state; it is known as M\=ay\=a and Her looking outward is Her Bahirmukh\={\i} state; it is denominated by T\=amas and the other gu\d{n}as. From this comes S\=attva and then R\=ajas and Brahm\=a, Vi\d{s}\d{n}u and Mahe\'sa are of the nature of the three gu\d{n}as. Brahm\=a has the R\=ajo gu\d{n}a in Him preponderating; in Vi\d{s}\d{n}u, the S\=attva gu\d{n}a preponderates and in Mahe\'sa, the Cause of all Causes, is said to reside the T\=amo gu\d{n}a. Brahm\=a is known as of the Gross Body; Vi\d{s}\d{n}u is known as of the Subtle Body; and Rudra is known as of the Causal Body and I am known as Tur\={\i}ya, transcending the Gu\d{n}as.

This Tur\={\i}ya Form of Mine is called the state of equilibrium of the Gu\d{n}as. It is the Inner Controller of all. Beyond this there is another state of Mine which is called the Formless Brahm\=a (Br\=ahma\d{n} having no Forms). Know, verily, that my Forms are two, as they are with or without attributes (Sagu\d{n}a or Nirgu\d{n}a). That which is beyond M\=ay\=a and the M\=ayic qualities is called Nirgu\d{n}a (without Pr\=akritic attributes) and that which is within M\=ay\=a is called Sagu\d{n}a. O Indra! After creating this universe, I enter within that as the Inner Controller of all and it is I that impel all the J\={\i}vas always to their due efforts and actions. Know, verily, that It is I that engage Brahm\=a, Vi\d{s}\d{n}u and Rudra, the causes of the several works of creation, preservation and destruction of this universe (they are performing their functions by My Command). Through the terror from Me the wind blows; through my terror, the Sun moves in the sky; through My terror, Indra, Ag\d{n}i, and Yama do their respective duties. I am the Best and Superior to all. All fear Me. Through My Grace you have obtained victory in the battle. Know, verily, that it is I that make you all dance like inert wooden dolls as My mere instruments. You are merely My functions. I am the Integral Whole. I give sometimes victory to you and sometimes victory to the Daityas; Yea, I do everything as I will, keeping My independence duly and, according to the Karmas, justly. Oh! You all, have forgotten me though your pride and sheer nonsense. You have been carried deep into dire delusion by your vain egoism. And know now that to favour you, this My Adorable Light has issued suddenly. Hence forth banish ever from your heart all your vain boastings and idle pratings. Take refuge wholly unto Me with all your head, heart and soul, unto My Sachchid\=ananda Form and be safe. (At times the Devas forget and so fall into troubles).

84-93. Vy\=asa said :-- Thus saying, the M\=ula Pakriti, the Great Dev\={\i}, the Goddess of the Universe, vanished from their sight. The Devas, on the other hand, began to praise Her then and there, with rapt devotion. Since that day, all the Devas quitted their pride and engaged themselves in worshipping the Dev\={\i} devotedly. They worshipped the G\=ayatr\={\i} Dev\={\i} daily during the three Sandhy\=a times and performed various Yaj\~n\=as and thus they worshipped Bhagavat\={\i} daily. Thus, in the Satya Yuga, everybody engaged themselves in repeating the Mantra G\=ayatr\={\i} and worshipped the Goddess indwelling in the Pra\d{n}ava and Hr\={\i}mk\=ara. So, See now for yourself, that the worship of Vi\d{s}\d{n}u or \'Siva or initiation in the Vi\d{s}\d{n}u Mantra or in the \'Siva Mantra are not mentioned anywhere in the Vedas as to be done always and for ever. They are done for a while and not required any more when the objects are fulfilled; only the worship of G\=ayatr\={\i} is always compul-

sory, to be done at all times, as mentioned in the Vedas. O King! If a Br\=ahma\d{n}a does not worship the G\=ayatr\={\i}, know, then, for certain, that in every way, he is sure to go down lower and lower. There is no doubt in this. A Br\=ahmi\d{n} is not to wait, no never, to do any other thing; he will have all his desires fulfilled if he worships only the Dev\={\i} G\=ayatr\={\i}. Bhagav\=an Manu says that a Br\=ahmi\d{n}, whether he does any other thing or not, can be saved if he worships only the Divine Mother G\=ayatr\={\i}. (This worshipping the G\=ayatr\={\i} is the highest, greatest, and most difficult of all the works in this universe). If any devotee of \'Siva or Vi\d{s}\d{n}u or of any other Deity worships his desired Deity without repeating the G\=ayatr\={\i}, he is sure to suffer the torments of hell. (But this age of K\=al\={\i} deludes the people and draws away their minds from reciting this G\=ayatr\={\i} save a few of them.) O King! For this reason, in the Satya Yuga, all the Br\=ahma\d{n}as kept themselves fully engaged in worshipping the G\=ayatr\={\i} and the lotus feet of the Dev\={\i} Bhagavat\={\i}.

Here ends the Eighth Chapter in the Twelfth Book on the appearance of the Highest \'Sakti in the Mah\=apur\=a\d{n}am \'Sr\={\i} Mad Dev\={\i} Bh\=agavatam of 18,000 verses by Mahar\d{s}i Veda Vy\=asa.



