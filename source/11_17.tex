\chapter{On the description of Sandhy\=a and other daily practices}

1-5. N\=ar\=aya\d{n}a said :-- If one divides or separates the p\=adas while reciting or making Japam of the G\=ayatr\={\i}, one is freed from the Br\=ahmi\d{n}icide, the sin of Brahmahaty\=a. But if one does so without breaking the p\=adas, i.e., repeats at one breath, then one incurs the sin of Brahmahaty\=a. Those Br\=ahma\d{n}as who do the Japam of the G\=ayatr\={\i} without giving due pause to the p\=adas, suffer pains in hells with their heads downwards for one hundred Kalpas. O G\=ayatr\={\i}! Thou art of one foot, of two feet, of three feet and of four feet. Thou art without foot, because Thou art not obtained. Salutation to Thy Fourth Foot beautiful and which is above the Trilok\={\i} (R\=ajas). This cannot obtain that. Firstly, G\=ayatr\={\i} is of three kinds :-- ``Samput\=a,'' ``Ekomk\=ar\=a,'' and ``Sadomk\=ar\=a.'' There is also the G\=ayatr\={\i}, with five Pra\d{n}avas, according to the Dharma \'S\=astras and Pur\=a\d{n}as. There is something to be noted while muttering or making the japam of the G\=ayatr\={\i} :-- Note how many lettered G\=ayatr\={\i} you are going to repeat (make japam). When you have repeated one-eighth of that, repeat (make japam) the Tur\={\i}ya p\=ada of G\=ayatr\={\i} (i.e., the fourth P\=ada, the mantram parorajase S\=avadom\=a pr\=apat) etc., (see the daily practices, page 107) once and then complete repeating the G\=ayatr\={\i}. If the Br\=ahma\d{n}a makes the Japam (the silent muttering) in the above way he gets himself united with Brahm\=a. Other modes of making the Japam do not bear any fruit. Om G\=ayatryasye kapad\={\i} dv\={\i}pap\={\i} Tripad\={\i} chatus p\=adasi nahi padyase namaste T\=ury\=aya dar\'sat\=ayapad\=aya paro Rajase S\=abado m\=a pr\=apat. G\=ayatr\={\i} is one-footed in the form of Trilok\={\i}, two-footed, the Tray\={\i} Vidy\=a from thy second foot; tripad\={\i} (all Pr\=a\d{n}as are thy third foot, chatu\d{s}padi, as the Puru\d{s}a apadi without any foot, Parorajase above the R\=ajas, the dust; asau-that; adah this not pr\=apat may obtain. The Yogis who are \=Urdharet\=as (hold Brahm\=a charyam, continence) are to make Japam of the Samput\=a G\=ayatr\={\i} (i.e., with Om). G\=ayatr\={\i} with one pra\d{n}ava and as well the G\=ayatr\={\i} with six pra\d{n}avas. The householder Brahmach\=ar\={\i} or those who want mok\d{s}a are to make Japam of G\=ayatr\={\i} with Om prefixed.

6. Those householders who affix Om to the G\=ayatr\={\i} do not get the increase of their families.

7-8. The Tur\={\i}ya p\=ada (foot) of G\=ayatr\={\i} is the mantra ``Parorajase S\=avodom\=a pr\=apat.'' (Brihad. up. v.14.7). Salutation to Thy beautiful Fourth Foot which is above the Trilok\={\i} (R\=ajas). This cannot obtain that. The presiding deity of this mantra is Brahm\=a. I am now speaking of the full Dhy\=anam (meditation) of this Brahm\=a so that the full fruit of the Japam (recitation) may be obtained. There is a full blown lotus in the heart; its form is like the Moon, Sun, and the Spark of Fire; i.e., of the nature of pra\d{n}ava and nothing else. This is the seat of the inconceivable Brahm\=a. Think thus. Now on that seat is seated well the steady constant subtle Light, the essence of Ak\=a\'sa, the everlasting existence, intelligence and bliss, the Brahm\=a. May He increase my happiness (see page 107 the daily practice of the Hindus by R. B. Sris Chandra Basu, on the Invocation of the G\=ayatr\={\i}).

Note :-- Aum! G\=ayatryasyekapad\={\i} dv\={\i}pad\={\i}, tripad\={\i}, chatu\d{s}padasi, nahi padyase namaste tury\=aya dar\'s at\=aya pad\=aya parorajase, s\=avado m\=apr\=apat O G\=ayatr\={\i}! Thou art of one foot (in the farm of Trilok\={\i}), of two feet (the Tray\={\i} vidy\=a from Thy second foot) of three feet all Pr\=a\d{n}a, etc., are Thy third foot and of four feet (as the Puru\d{s}a). Thou art without foot because Thou art not obtained. Salutation to Thy beautiful fourth foot which is above the Trilok\={\i} (R\=ajas). This cannot obtain that.

9. Now I am speaking of the Mudr\=a of the Tur\={\i}y\=a G\=ayatr\={\i} :-- (1) Tri\'s\=ula, (2) Yoni, (3) Surabhi, (4) Ak\d{s}am\=al\=a, (5) Linga, (6) Padma and (7) Mah\=amudr\=a. These seven Mudr\=as are to be shewn.

10-14. What is Sandhy\=a, that is G\=ayatr\={\i}; there is no difference whatsoever between the two. The two are one and the same. Both are of the nature of Existence, Intelligence and Bliss. The Br\=ahma\d{n}as would daily worship Her and bow down before Her with greatest devotion and reverence. After the Dhy\=anam, first worship Her with five upach\=aras or offerings. Thus :--

Om lam prithivy\=atmane gandham, arpay\=ami namo namah. Om Ham \=ak\=a\'s\=atmane pu\d{s}pam arpay\=ami namo namah. Om ram Vahny\=atmane d\={\i}pam arpay\=ami namo namah. Aum vam amrit\=atmane naivedyam arpay\=ami namo namah. Om yam ram lam vam ham pu\d{s}p\=anjalim arpay\=ami namo namah. Thus worshipping with five upach\=aras, you must shew Mudr\=as to the Dev\={\i}.

15-16. Then meditate on the Form of the G\=ayatr\={\i} mentally and slowly repeat the G\=ayatr\={\i}. Do not shake head, neck and while making japam, do not show your teeth. According to due rules repeat the G\=ayatr\={\i} one hundred and eight times, or twenty-eight times. When unable, repeat ten times; not less than that.

17-20. Then raise the G\=ayatr\={\i} placed before on the heart (seat) by the mantra ``G\=ayatrasyai kapad\={\i} Dv\={\i}pad\={\i}, etc., and then bid farewell to Her after bowing down to Her and repeating the mantra ``Omuttame \'Sikhare Dev\={\i} bh\=umy\=am parvata m\=urdhani Br\=ahma\d{n}a ebhyobhya anuj\~n\=at\=a Gachcha Dev\={\i} yath\=asukham'' on the highest top of the mountain summit in earth (i.e., on the Meru mountain) dwells the goddess G\=ayatr\={\i}. Being pleased with Thy worshippers go back, O Dev\={\i}! to Thy abode as it pleaseth Thee.'' (See page 110, The Daily Practices of the Hindus.)

The wise men never mutter nor recite the G\=ayatr\={\i} mantra within the water. For the Mahar\d{s}is say that the G\=ayatr\={\i} is fire-faced (agnimukh\={\i}). After the farewell shew again the following mudr\=as :-- Surabhi J\~n\=an, S\=urpa, K\=urma, Yoni, Padma, Linga and Nirv\=ana Mudr\=as.

Then address thus :-- ``O Dev\={\i}! O Thou who speakest pleasant to Ka\'syapa! O G\=ayatr\={\i}! Whatever syllables I have missed to utter in making Japam, whatever vowels and consonants are incorrectly pronounced, I ask Thy pardon for all my above faults.'' O N\=arada! Next one ought to give peace offerings to the G\=ayatr\={\i} Dev\={\i}.

21-33. The Chchhanda of G\=ayatr\={\i} Tarpa\d{n}am (peace offerings to G\=ayatr\={\i}) is G\=ayatr\={\i}; the \d{R}i\d{s}i is Vi\'sv\=amitra; Savit\=a is the Devat\=a; its application (Niyoga) is in the peace offerings.

``Om Bh\=uhrigvedapuru\d{s}am tarpay\=ami.''
``Om Bhuvah Yajurvedapuru\d{s}am tarpay\=ami.''
``Om Svah S\=amaveda puru\d{s}am tarpay\=ami.''
``Om Mahah Atharvaveda puru\d{s}am tarpay\=ami.''
``Om Janah Itih\=asapur\=a\d{n}a puru\d{s}am tarpay\=ami.''
``Om Tapah Sarv\=agama puru\d{s}am tarpay\=ami.''
``Om Satyam Satyaloka puru\d{s}am tarpay\=ami.''
``Om Bh\=uh bh\=urloka puru\d{s}am tarpay\=ami.''
``Om Bhuvah bhuvoloka puru\d{s}am tarpay\=ami.''
``Om Svah svarloka puru\d{s}am tarpay\=ami.''
``Om Bh\=uh rekapad\=am G\=ayatr\={\i}m tarpay\=ami.''
``Om Bhuvo dv\={\i}t\={\i}yapad\=am G\=ayatr\={\i}m tarpay\=ami.''
``Om Svastripad\=am G\=ayatrim tarpay\=ami.''
``Om Bh\=urbh\=uvah Sva\'schatuspad\=am G\=ayatr\={\i}m tarpay\=ami.''

Pronouncing these, offer the Tarpa\d{n}ams. Next add the word Tarpay\=ami to each of the following words ``\=U\d{s}as\={\i}m, G\=ayatr\={\i}m, S\=avitr\={\i}m, Sarasvatim Vedam\=ataram, Prithv\={\i}m, Aj\=am, Kau\'s\={\i}k\={\i}m, S\=amkrit\={\i}m, Savajit\={\i}m, etc.,'' and offer Tarpa\d{n}ams. After the Tarpa\d{n}am is over, offer the peace-chantings, (\'S\=antiv\=ari) repeating the following mantras.

``Om J\=ata vedase sunav\=ama romam, etc.''
``Om M\=anastoka, etc.''
``Om Tryamvakam Yaj\=amahe, etc.''
``Om Tachchhamyoh, etc.''

Then touch all the parts of your bodies, repeating the two mantra, ``Om atodeva, etc.'' And reciting the mantram ``Svon\=a Prithiv\={\i},'' bow down to the earth, after repeating one's name, Gotra, etc.

34-45. O N\=arada! Thus the rules of the morning Sandhy\=a are prescribed. Doing works so far, bid farewell to the above-mentioned G\=ayatr\={\i}. Next finishing the Agnihotra Homa sacrifice, worship the five Devat\=as, \'Siv\=a, \'Siva, Gane\'sa, S\=urya and Vi\d{s}\d{n}u. Worship by the Puru\d{s}a S\=ukta mantra, or by Hr\={\i}m mantra, or by Vyahriti mantra or by \'Srischate Lak\d{s}m\={\i}\'scha, etc., place Bhavan\={\i} in the centre; Vi\d{s}\d{n}u in the north east corner, \'Siva in the south-east corner; Gane\'sa in the south-west corner, and the Sun in the north-west corner; and then worship them. While offering worship with the sixteen offerings, worship by repeating sixteen mantras. As there is no other act more merit-giving than the worship of the Dev\={\i}, so the Dev\={\i} should first of all be worshipped. Then worship in due order the five Devat\=as placed in five positions. As the worship of the Dev\={\i} is the chief object, so in the three Sandhy\=as, the worship of the Sandhy\=a Dev\={\i} is approved of by the \'Srutis. Never worship Vi\d{s}\d{n}u with rice; Gane\'sa with Tulas\={\i} leaves; the Dev\={\i} Durg\=a with Durba grass and \'Siva with Ketak\={\i} flower. The under-mentioned flowers are pleasing to the Dev\={\i} :-- Mallik\=a, J\=ati, Kutaja, Panasa, Pal\=asa, Vakula, Lodha, Karav\={\i}ra, \'Si\d{n}\'sapa, Apar\=ajit\=a, Bandh\=uka, Vaka, Madanta, Sindhuv\=ara, Pal\=a\'sa, Durbh\=a, \'Sallak\={\i}, M\=adhav\={\i}, Arka, Mand\=ara, Ketak\={\i}, Kar\d{n}ik\=ara, Kadamba, Lotus, Champaka, Y\=uthik\=a, Tagara, etc.

46-47. Offer incenses Guggul, Dh\=upa and the light of the Til oil and finish the worship. Then repeat the principal (male) mantra (make Japam). Thus finishing the work, study the Vedas in the second quarter of the next day; and in the third quarter of that day feed father, mother and other dependent relatives, with money earned by one's own self according to the traditions of one's family.

Here ends the Seventeenth Chapter of the Eleventh Book on the description of Sandhy\=a and other daily practices in the Mah\=apur\=a\d{n}am \'Sr\={\i} Mad Dev\={\i} Bh\=agavatam of 18,000 verses by Mahar\d{s}i Veda Vy\=asa.



