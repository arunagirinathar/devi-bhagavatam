\chapter{On the Amb\=a Yaj\~na rules}

1. The king spoke :-- O Lord! Kindly describe the rules and regulations as to how the Dev\={\i} Yaj\~na (sacrifice) is performed with its duly prescribed rites and ceremonies. Hearing it I will unwearied perform it, as far as it lies in my power, with as little delay as possible.

2. How the worship is done, what are the mantras, what are the articles required for oblations, how many Br\=ahmi\d{n}s are required and what Dak\d{s}in\=as are to be paid to them, describe in detail all these.

3-5. Vy\=asa said :-- O king! I am telling you duly how the Yaj\~na is performed, hear. The actions are always threefold according as the preparations are made and according as they are practised with regard to the observed rules. The threefold divisions are S\=attvik, R\=ajasik and T\=amasic. The Munis do the S\=attvik P\=uj\=a, the kings celebrate R\=ajasic and the Rakshasas do the T\=amasik P\=uj\=as. There is another P\=uj\=a which is devoid of qualities and which is performed by the liberated ones. I will describe to you all these in detail.

6-7. O king! The Yaj\~na is then called S\=attvik, when it is performed in a S\=attvik country, like Benares, etc., in S\=attvik time, e.g., in Uttar\=aya\d{n}a, when the materials collected are earned rightly, when the mantras are those of the Vedas, when the Br\=ahmi\d{n} is Srotriya, where there is S\=attvik faith, void of any attachment towards the sensual objects, when all these happen to coincide. O king! When all the above takes place and there is purification of materials, actions, and mantras, i.e., when the

materials are all right, when the actions are done as they ought to be, and where there is no error or omission, etc., in the mantras, etc., then and then only the Yaj\~na becomes perfect and no doubt yield full results; there would be nothing contrary to this.

8-9. If the Yaj\~na is performed with articles not rightly earned, then there is no fame either in this world nor there is any reward in the next world. Therefore it is necessary that the Yaj\~na should be performed with rightly earned materials; then there is fame in this world and better state in the next world; and happiness is also acquired; there is no doubt in this.

10. O king! It is before your eyes, as it were, that the P\=andavas performed the R\=ajas\=uya Yaj\~na, the king of sacrifices, and, on the completion whereof, the excellent Dak\d{s}in\=as were paid to the Br\=ahmi\d{n}s and others.

11. In that Yaj\~na the highly intelligent \'Sr\={\i} Kri\d{s}\d{n}a Himself, the Lord of the Y\=adavas was present, as well as many other Br\=ahma\d{n}as, like Bh\=aradv\=aja and other fully enightened souls.

12. But within three months after completing the sacrifice, the P\=andavas suffered extreme hardships and had to live, with extreme difficulty, as exiles in the forest.

13. Consider the insult shown towards Draupad\={\i}, the P\=andava's defeat in the play of gambling, their going away to dwell in the forest; these hardships were borne by the P\=andavas. What rewards did then the P\=andavas derive from the R\=ajas\=uya Yaj\~na?

14-15. All the high-souled P\=andavas had to work as slaves of Vir\=ata; and Draupad\={\i}, the best of women, was very much troubled and insulted by Kichaka. When all these occurred, any one can easily ask where were the ashirv\=adas of the pure souled Br\=ahma\d{n}as? Also what result did they derive from their unflinching devotion towards \'Sr\={\i} Kri\d{s}\d{n}a when they were involved in the above critical state?

16. No one protected Draupad\={\i}, the chaste and the best, the daughter of Drupada, when she was drawn by her hair on her head into the hall of assembly where gambling was being played?

17. O king! How could all these happen in a place where \'Sr\={\i} Bhagav\=an Kesava Himself and the high souled Yudhisth\={\i}ra were present? If one argues, one would conclude ``there must have been something wrong in that Yaj\~na.''

18. If you say that nothing wrong happened in the Yaj\~na, all these were caused by Fate; then it comes to this :-- that the Vedic mantras, \=Agamas and the other Vedic rites are all fruitless.

19. If it be argued that though the Vedic mantras are powerful enough to bear fruits, yet whatever is predestined to come to pass, will surely pass, then the proposition resolves into this :-- that all the means, expedients, and appliances lead to meaningless conclusions.

20. Then the \=Agamas, the Vedas merely recommend a vidhi or precept by stating the good arising from its proper observance and the evils arising from its omission and also by adducing historical instances as its support; in other words, they are powerless as far as bearing fruits is concerned; all the acts are meaningless, asceticism to attain Heaven comes as useless and the peculiar duties of caste are fruitless. O king! This view is exceedingly culpable; it is never fit for acceptance by the highsouled persons.

21. O King! If what is laid by God in the womb of futurity (a state of things preordained by God in which it is sure to take place in the fullness of time), be taken as the first-hand proof, then all the other proofs are rendered null and void. Therefore Fate and human exertion both are to be undoubtedly taken into account to ensure success.

22. Human exertions being applied, if the results come otherwise, the wise Pundits would infer that some defects, omissions or imperfections crept into the work.

23. All the Pundits, very learned and instituters of sacrifices have classed Karmas under different headings according as the agents, mantras, and articles employed in the worship vary.

24. Once on an occasion Vi\'svar\=upa, ordained as a Guru by Indra (in a Yaj\~na) (intentionally) did things contrary so as to benefit the Daityas, who belonged to his mother's side.

25. Vi\'svar\=upa uttered repeatedly the mantrams beneficial to the gods, while they were present; and, during their absence, prayed heartily for the welfare of the Daityas; and, in the long run, protected the Daityas.

26. On seeing the Asuras gaining strength, Indra, the Lord of the Devas, became very much enraged and instantly cut off Vi\'svar\=upa's head by his thunderbolt.

27. O King! This is then the instance where the contrary fruits were borne out by the agent employed in performing the Yaj\~na; there is no doubt in this. This is not possible in the other cases.

28. See, again, the king of P\=anch\=ala performed his sacrifice to get a son to kill Dro\d{n}a, the son of Bh\=aradv\=aja; and though he did this out of angry motives, still Dhristadyumna was born out of fire; and Draupad\={\i} sprang out of the altar.

29. Again, in days of yore, Da\'saratha, the king of Kosala, was sonless; and he instituted a sacrifice to get one son; and lo! be got four sons.

30. Therefore O King! If the Yaj\~na be performed according to proper rules and regulations, it yields fruits in all respects; again if it be done unrighteously, without any regard for the rules, etc., it yields results just the contrary; there is no doubt in this.

31-32. Therefore, there must have been some defects in the Yaj\~na of the P\=andavas; hence contrary effects ensued, and therefore the truthful king Yudhisth\={\i}ra and his powerful brothers and the chaste Draupad\={\i} were all defeated in the play at dice.

33. It might be that the materials were not of a good stamp; they were all earned by killing the kings, good many in number, and earned thus unrighteously; or it might happen that the P\=andavas did their Yaj\~na with too much egoism. However, this is certain that there had crept in some defects in their actions.

34. O King! The S\=attvik Yaj\~na is rare; it can be done only by the S\=attvik Munis who live in the 3rd order of the household life or who live as hermits.

35-36. The ascetics that eat daily the S\=attvik food, the roots and fruits, collected from forests and obtained rightly, that is good to the Munis and that is well cleaned and purified, are the only ones that can perform with full devotions the S\=attvik Yaj\~nas, where no animals are sacrificed (where there are no sacrificial posts to which the victim is fastened at the time of immolation) and where offerings of cakes of ground rice in vessels are given. These are the best of all the S\=attvik Yaj\~nas.

37. The K\d{s}attriyas and the Vai\d{s}yas perform the Yaj\~nas with Abhim\=an (self-conceit and egoism) where many presents are given, animals are sacrificed, and all things are well cleansed, purified and elaborately decorated. This Yaj\~na is called R\=ajasic.

38. That Yaj\~na is according to the sages, T\=amasik, where the D\=anavas, puffed up with arrogance, infatuated with anger, jealousy and wickedness perform their acts with the sole object of killing their enemies.

39. That Yaj\~na is called M\=anas Y\=ag or m\=anasic (mental) where the high-souled Munis, void of worldly desires, collect mentally all the necessary articles and perform the Yaj\~na with the sole object of liberation from the bondages of the world.

40. In all the other Yaj\~nas (than the M\=anas Y\=ag) some imperfections or other naturally arise, due to some defects in the materials, or want of faith, or in the performance or in the Br\=ahmi\d{n}s.

41. No other Yaj\~nas can be so complete as the M\=anasa Yaj\~na; the reason being that in the other Yaj\~nas some, imperfections come out due to time, place, and separate ingredients to be collected.

42-43. Now hear who are the persons fit to undertake this mental Yaj\~na in honour of the Great Goddess. First this mind is to be purified, by making it void of the Gu\d{n}as; the mind being pure, the body becomes also pure, there is no doubt. When the mind becomes completely pure, after it has abandoned all sensual objects, fit for enjoyment, then that man is entitled to perform the Mother's Yaj\~na.

44-45. There he should build mentally the big hall for sacrifice, many Yojanas wide, decorated with high polished pillars out of the materials brought for the purpose (e. g., fortitude, etc.). Within the hall he will imagine a wide and spacious altar and place the Holy Fire on it mentally according to due rules and regulations.

46-47. He is to select mentally the Br\=ahmi\d{n} priests and consecrate them as Brahm\=a, Adharyu, Hot\=a, Prastot\=a, Udg\=at\=a, Pratihatr\=a and other assistants. He is to worship mentally all these priests.

48. Then he will have to imagine the five V\=ayus, Pr\=a\d{n}a, Ap\=ana, Vy\=ana, Sam\=ana, and Ud\=ana, as the five fires and locate them duly on the altar.

49-50. Pr\=a\d{n}a V\=ayu stands for G\=arhapatya; Ap\=ana, for \=Ahavan\={\i}ya; Vy\=ana for Dak\d{s}in\=a; Sam\=ana for Avasathya; and Ud\=ana for Sabhya Agni. These fires are all very terrible; then one should place these carefully on the altar with great concentration of mind. He is to collect then all the other necessary materials and think that all are very pure and free from any defects.

51-57. In the M\=anasic Yaj\~na, mind is the offerer of oblations and mind the Yajam\=ana, the performer of the Sacrifice; and the Presiding Deity of the Sacrifice is the Nirgu\d{n}a Brahm\=a. The Great Goddess, the Nirgu\d{n}a Energy, who is always auspicious and gives the feeling of dispassion and indifference to worldly objects is the awarder of fruits in this Yaj\~na. She is the Brahm\=a Vidy\=a, She is the substratum of all and She is all pervading. The Br\=ahmi\d{n} is to take the Dev\={\i}'s name and offer oblations in the fire of Pr\=a\d{n}a, the necessary articles for the Dev\={\i}'s satisfaction. Then he is to make his Chitta and Pr\=a\d{n}a void of any worldly thought, or any worldly support and to offer oblations to the Eternal Brahm\=a through the mouth of Kundalin\={\i} (the Serpent Fire.) Next, within his Nirvikalpa mind, by means of Sam\=adhi, be should meditate own Self, the Mahe\'svar\={\i} Herself by his consciousness. Thus, when he will see his own self in all the beings and all the beings in his own self, then

the J\={\i}va will get the vision of the Goddess Mah\=avidy\=a, giving auspicious liberation (Mok\d{s}a). O King! After the high souled Munis have seen the Goddess, of everlasting intelligence and bliss, then he becomes the knower of Br\=ahma\d{n}. All the M\=ay\=a, the cause of this Universe becomes burnt up; only, as long as the body remains, the Pr\=arabdha Karma remains.

58. Then the J\={\i}vas become liberated, while living; and when the body dissolves, he attains to final liberation. Therefore, O Child! Whoever worships the Mother becomes crowned with success; there is no doubt in this.

59. Therefore follow the advice of the Guru, the Spiritual Teacher; and with all attention, hear, think and meditate on the Great Goddess of the World.

60. O King! Liberation is sure to ensue of this M\=anasa Yaj\~na. All the other Yaj\~nas are Sak\=ama (with some object in view) and therefore their effects are temporary.

61-62. He who wants enjoyments in Heaven, should perform the Agnistoma Yaj\~na, with due rites and ceremonies; such is the Vedic injunction. But when the acquired merit expires, the sacrificer will have to come again into this world of mortals. Therefore the M\=anasa Yaj\~na is eternal and best.

63-65. This M\=anasa Yaj\~na is not fit to be performed by kings intent on getting victory. The Yaj\~na that you performed, the serpent Yaj\~na, is T\=amasic, for you wanted to take vengeance on your enemy, the serpent Tak\d{s}aka; and millions of serpents were made to be burnt in that sacrifice.

O King! Hear now about the Dev\={\i} Yaj\~na, that was performed by Vi\d{s}\d{n}u in the beginning of the creation. You better now do that Dev\={\i} Yaj\~na with due rules.

66-67. I will tell you all about the rules; there are Br\=ahmi\d{n}s that know the rules and know best also the Vedas; they know also the seed mantrams of the Dev\={\i}, as well as the rules of their application; they are clever in all the mantrams. These will be your priests and you yourself will be the sacrificer.

68. O King! Do this sacrifice duly and deliver your father from hell by the merits that you will acquire thereby.

69. O Sinless One! The sin incurred on account of insulting a Br\=ahmi\d{n} is serious and leads the sinner to hell. Your father committed that sin and incurred the curse from a Br\=ahmi\d{n}. Therefore he has gone to the hell.

70. Your father died also out of a snake bite which is not a meritorious one. The death occurred also in a palace built high up in the air (on a pillar), instead of taking place on the ground on a bed of Ku\'sa grass.

71. O best of the Kurus! The death did not occur in any battle nor on the banks of the Ganges. Void of proper bathing and charities, etc., he died in a palace.

72. O best of Kings! All the ugly causes, leading to hell, were present in the case of your father. See, again, there is also one thing which done will lead to one's liberation; but that was absent too with your father.

73-76. That is this :-- Let a man remain, wherever he may, whenever he comes to learn that his end is approaching, even if he had not practised before any good practices or meritorious deeds, and even if he becomes senseless in the trial time of death, when dispassion comes to an individual whose mind gets, for the time being, clear and free from any worldly thoughts, then he should think thus :-- ``This my body, composed of five elements, will soon be destroyed; there is no cause whatsoever in having any remorse for it; let whatever come, that it may; I am free, void of qualities; and I am the Eternal Puru\d{s}a; death is not capable to do any harm to me. All the elements are liable to decay and destruction; what remorse can overtake me? I am not a man of the world, I am always free, Eternal Brahm\=a; I have got no connection with this body that is merely the outcome of actions.

77. Before I did meritorious or unmeritorious acts, leading to happiness and pain; therefore I have got this mortal coil and am enjoying the fruits of my past auspicious or inauspicious Karma.''

78. Whoever thinks thus and dies, even if he does not take proper purificatory bath or make any charity, he gets himself freed from the awful Sams\=ara and never comes to see himself again born in this world.

79. O King! This method of parting from one's body is rarely attained even by the Yogins; this is the acme, the highest height of all the human efforts towards liberation.

80. But your father, hearing even the curse from a Br\=ahmi\d{n}, retained his attachment towards his body; therefore he did not attain dispassion.

81. He thought thus :-- ``My body is now free from any disease; my kingdom is free from enemies or any other source of danger; how can I now get myself saved from this untimely death.'' Thinking thus, he ordered to call the Br\=ahma\d{n}s, who know the mantrams.

82. Then that king ascended to the palace, with medicines, many mantras and many other instruments.

83-84. He considered his fate to be the strongest and therefore did not take his bath in any holy place; he did not perform any charities, did not sleep on the ground or remember any mantram of the Dev\={\i}. Due to Kali entering into his body, he committed the sin of insulting an ascetic and plunged himself in the ocean of delusion and died bitten by the Tak\d{s}aka snake on the top of a palace.

85. The King has now fallen undoubtedly to the hell, on account of those vicious deeds. Therefore, O King! dost Thou deliver your father from the sin.

86. S\=uta said, O \d{R}i\d{s}is! Hearing these words from the fiery Vy\=asa, the king Janamejaya became very sad and tears came from his eyes and flowed down his cheeks and throat.

He then exclaimed in a suffocating voice ``Fie on me! my father is still in the hell. I will now do at once whatever leads my father to heaven.''

Thus ends the twelfth chapter on the Amb\=a Yaj\~na rules in the 3rd Adhy\=aya of \'Sr\={\i} Mad Dev\={\i} Bh\=agavatam, the Mah\=a Pur\=a\d{n}am of 18,000 verses composed by Mah\=ar\d{s}i Veda Vy\=asa.



