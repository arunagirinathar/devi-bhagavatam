\chapter{On the offences caused to the Earth and punishments thereof}

1-3. N\=arada said :-- I am now desirous to hear about the merits acquired by making gifts of land, the demerits in stealing away lands, digging wells in other's wells, in digging earth on the day of Ambuv\=ach\={\i}, in casting semen on earth, and in placing lamps and lights on the surface of the earth as well the sins when one acts wrongly in various other ways on the surface of the earth and the remedies thereof.

4-30. \'Sr\={\i} N\=ar\=ayana said :-- If one makes a gift of land in this Bh\=arata of the measure of a Vitasti (a long span measured by the extended thumb and little finger) to a Br\=ahma\d{n}a who performs Sandhy\=a three times a day and is thus purified, one goes and remains in \'Siva Loka (the abode of \'Siva). If one gives away in charity a land full of corn to a Br\=ahmi\d{n}, the giver goes and lives in Vi\d{s}\d{n}u Loka in the end for a period measured by the number of dust particles in the land. If one presents a village, a plot of land, or grains to a Br\=ahmi\d{n}, both the giver and the receiver, become freed of their sins and go to the Dev\={\i} Loka (the abode of the Dev\={\i}). Even if one be present when a proposal for a gift of land is being made and says ``This act is good,'' one goes to Vaikuntha with one's friends and relatives. He remains in the K\=alasutra hell as long as the Sun and Moon exist, who takes back or steals away the gift to a Br\=ahmi\d{n}, offered by himself or by any other body. Even his sons, grandsons, etc., become destitute of lands, prosperity, sons, and wealth and remain in a dreadful hell named Raurava. If one cultivates the grazing land for the cows and reaps therefrom a harvest of grains, one remains for one hundred divine years in the Kumbh\={\i}p\=aka hell. If one cultivates any enclosure for cows or tanks and grows grains on them, one remains in the Asipatra hell for a period equivalent to fourteen Indra's falls. One who bathes in another's tank without taking off five handfuls of earth from it, goes to hell and one's bath is quite ineffectual. If anybody, out of his amorous passion casts his semen privately on the suface of the ground, he will have to suffer the torments of hell for as many years as are the numbers of dust particles on that area. If anyone digs ground on the day of Ambuv\=ach\={\i}, one remains in hell for four Yugas. If, without the permission of the owner of a well or tank, a stupid man clears the old well or tank and digs

the slushy earth from the bottom, his labour goes in vain. The merit goes to the real owner. And the man who laboured so much goes to Tapta Kunda Naraka for fourteen Indra's life-periods. If any one takes out five handfuls of earth from another's tank, when he goes to bathe in it, he dwells in Brahm\=a Loka for a period of years amounting to the number of particles in those handfuls of earth. During one's father's or grandfather's \'Sr\=adha ceremony, if one offers pinda without offering any food (pinda) to the owner of the soil, the \'Sr\=adha performer goes certainly to hell. If one places a light (Prad\={\i}pa) directly on the earth without any holding piece at the bottom, one becomes blind for seven births; and so if one places a conch-shell on the ground (\'Sankha), one becomes attacked with leprosy in one's next birth. If any body places pearls, gems, diamonds, gold and jewels, the five precious things on the ground he becomes blind, if one places the phallic emblem of \'Siva, the image of \'Siv\=ani, the \'S\=alagr\=ama stone on the ground, he remains for one hundred Manvantaras to be eaten by worms. Conchshells, Yantras (diagrams for \'Sakti worshippers), the water after washing \'Silas (stones) i. e., Chara\d{n}\=amrita, flowers, Tulas\={\i} leaves, if placed on the ground, lead him who places these, verily to hells. The beads, garlands of flowers, Gorochana (a bright yellow pigment prepared from the urine or bile of a cow), and camphor, when placed on the ground, lead him who places so to suffer the torments of hell. The sandal wood, Rudr\=ak\d{s}a m\=al\=a, and the roots of Ku\'sa grass also, if placed on the ground, lead the doer to stay for one manvantara in the hell. Books, the sacred Upanayana threads, when placed on the ground make the doers unfit for Br\=ahmi\d{n} birth; rather he is involved in a sin equivalent to the murder of a Br\=ahmi\d{n}. The sacred Upanayana thread when knotted and rendered fit for holding, is worth being worshipped by all the castes. One ought to sprinkle the earth with curd, milk, etc., after one has completed one's sacrifices. If one fails to do this, one will have to remain for seven births in a hot ground with great torment. If one digs the earth when there is an earth quake or when there is an eclipse, that sinner becomes also devoid of some of his limbs in his next birth. O Muni! This earth is named Bh\=umi since She is the abode of all; she is named K\=a\'syap\={\i} since she is the daughter of Ka\'syapa; is named Vi\'svambhar\=a, since she supports the Universe; She is named Ananta, since she is endlessly wide; and She is named Prithiv\={\i} since she is the daughter of the King Prithu, or she is extensively wide.

Here ends the Tenth Chapter of the Ninth Book on the offences caused towards the surface of the earth and punishments thereof

in hells - in the Mah\=apur\=a\d{n}am \'Sr\={\i} Mad Dev\={\i} Bh\=agavatam of 18,000 verses by Mahar\d{s}i Veda Vy\=asa.



