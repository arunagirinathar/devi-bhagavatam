\chapter{On the birth of Va\'sistha from Mitr\=a Varu\d{n}a}

1-2. Janamejaya said :-- ``O Bhagav\=an! Mahar\d{s}i Va\'sistha was the mind-born son of Brahm\=a; how is it then that you have named him as Maitr\=a-V\=aru\d{n}\={\i}. Is it that he got this name by some action or by some Gu\d{n}as? Kindly tell me the origin of this name, O Best of speakers!''

3-4. Vy\=asa said :-- O King! It is quite true that the illustrious Va\'sistha was the noble son of Brahm\=a but he had to quit that body due to the curse of the King Nimi and he had to take a second body from Mitr\=a Varu\d{n}a; hence he is named in this world as Maitr\=a-V\=aru\d{n}\={\i}.

5-6. The King said :-- ``O Bhagav\=an! How was it that the religious Va\'sistha, the best of the Munis, the son of Brahm\=a was cursed by the King? Oh! The Munis have to suffer the dreadful curse of K\d{s}attriya kings! This seems very wonderful to me. O Knower of Dharma! Why did that king curse the innocent Muni? I am very curious to hear the cause of this; kindly tell me the cause of the curse.''

7-30. Vy\=asa said :-- O King! I told you already in particular all the causes of these. This Sams\=ara is pervaded by the three Gu\d{n}as of M\=ay\=a, S\=attva, R\=aja and T\=ama. Whether the kings practise their Dharma or the ascetics practise their tapas, all their actions are interpenetrated with these Gu\d{n}as; therefore they cannot shine so brightly. The Kings, Munis performed very severe penances and austerities under the influence of lust, anger, greed and Ahamk\=ara. O King! All, whether they be the K\d{s}attriyas or the Br\=ahma\d{n}as, who perform their sacrifices overpowered with this R\=ajo Gu\d{n}a, really, none of them performs these actions guided by S\=attva Gu\d{n}a. The King Nimi was cursed by the \d{R}i\d{s}i and the \d{R}i\d{s}i was cursed again by the King Nimi; thus they met with greater calamities and painful sufferings, the fruits from the hands of the powerful Fate. O King! In this world of the three Gu\d{n}as, it is very difficult for the beings to get the Dravya \'Suddhi, Kriy\=a \'Suddhi, and the pure effulgent Chitta \'Suddhi, O King! Know this as the influence wielded

by the Highest \'Sakti, the Mother of this Universe. Nobody is able to overstep it; but he, whom She favours, can cross in a moment this world, bounded by the three Gu\d{n}as. What more can be said than the fact that Hari, Hara, and Brahm\=a and the other Gods cannot free themselves without Her grace. Moreover, the sinners like Satyavrata and others become free when Her Grace comes upon them. Nobody in these three worlds can know what reigns in Her mind; again, this is also a certain fact that She gets Herself bound by Her own will to Her devotees. Therefore it is extremely desirable that one should have recourse to S\=attvik\={\i} devotion for the complete removal of faults and sins. And as the devotion with attachment and vanity is always injurious to men, therefore it is highly beneficial to quit it; there is no doubt in this. O King! There was a king named Nimi, born of the family of Ik\d{s}\=aku. He was beautiful, well qualified, virtuous, truthful, charitable, endearing to his people, a sacrificer, of pure conduct and manners, ready to govern his subjects, intelligent and endowed with knowledge. For the benefit of the Br\=ahma\d{n}as, that high-souled king established a city named Jayantupur in close vicinity to the hermitage of Gautama. Thus some time passed when this R\=ajasik idea arose in his mind that ``I will perform a sacrifice extending for a good many years when I will give exhorbitant Dak\d{s}i\d{n}\=as (remunerations to the priests and Br\=ahmi\d{n}s).'' Getting permission from his own father Ik\d{s}\=aku, he began to collect all the ingredients necessary for the sacrifice, as advised by the high-souled persons. He invited the all-knowing Munis and ascetics, versed in the Vedas and in conducting sacrifices, e.g., Bhrigu, Angir\=a, V\=amadeva, Gautama, Va\'sistha, Pulastya, Richika, Pulaha, Kratu and others, all well-versed in the Vedas. Then that religious King Nimi, collecting all the materials necessary for sacrifice worshipped his own Guru Va\'sistha and then spoke to him (the Guru) with great humility. O Best of Munis! I will perform a sacrifice; kindly perform this my sacrificial act; you are my Guru and therefore you know everything; so do this sacrifice for me. All the articles for this purpose are brought and cleansed. O Guru! Know that for five thousand years I mean to be engaged in this sacrifice, this is my Sankalpa (will). I will worship the Goddess Ambik\=a in this sacrifice and for Her satisfaction I am arranging for it according to the prescribed rules. Hearing the King Nimi's words, Va\'sistha said :-- ``O best of Kings! Indra, the King of the Devas, has already selected me for his sacrificial ceremony. Now Indra is ready to do the sacrifice for the propitiation of the Highest \'Sakti and I have initiated him for five hundred years. Therefore, O King! You will have to wait till I complete the Indra's Yaj\~n\=a. After fully completing all his works, I will come here. Therefore, O King! Wait till then.''

31-42. The King said :-- ``O Best of Munis! I have already invited other Munis for this sacrifice and have collected all the materials; how, then, can I wait for you? O Br\=ahma\d{n}a! You are the foremost of those versed in the Vedas and you are the family Guru of the noble Ik\d{s}\=aku. How is it, then, avoiding my work you are ready to go elsewhere O Best of Br\=ahmi\d{n}s! Under the uncontrollable greed of wealth you have lost all senses and you are ready to go away without doing my work. This does not behove you.'' O King! Though thus tempted by the King Nimi, the \d{R}i\d{s}i Va\'sistha went to the Indra's sacrifice. The King, too, became absent-minded and selected for the sacrifice the \d{R}i\d{s}i Gautama. He then commenced his sacrificial ceremony close to the ocean by the side of the Him\=alay\=an range and gave profusely the Dak\d{s}i\d{n}\=as. The King Nimi was engaged in this sacrificial act for five thousand years. In this the Rittviks (priests) were worshipped with sufficient wealth and cows; they were extremely glad. Then, when the five hundred years extending sacrifice of Indra was completed, the \d{R}i\d{s}i Va\'sistha came to see the King Nimi's sacrifice and waited there to see the King. The king was then asleep; so the servants did not awake him from his sleep; and the King did not come to the \d{R}i\d{s}i. Feeling insult at this, the Mahar\d{s}i Va\'sistha became infuriated with rage. Not seeing the King, he became very angry; and, subject to this, he cursed the King, When I am your lifelong Guru, especially when I prohibited you and you have forsaken me and selected another Guru and by your sheer force you are initiated, then be devoid of your body. Let your body fall off today.

43-50. Vy\=asa said :-- The King's attendants, hearing thus the curse given by Va\'sistha to the King, instantly awoke him from his sleep and informed him that the \d{R}i\d{s}i Va\'sistha not seeing you, became very angry. The King Nimi, quite sinless, went then to the angry Va\'sistha and humbly spoke to him the following reasonable words, pregnant with meaning. O Knower of Dharma! I am your Yajam\=ana; though I repeatedly requested you to perform my sacrifice, yet you quitted me out of the covetousness and went somewhere else. I cannot be charged with any fault. You are the foremost of Br\=ahmi\d{n}s; and knowing that contentment is the substance of your Dharma, you did not feel ashamed to do this blameable act. You are the son of Brahm\=a; and, being versed in the Vedas and Ved\=angas, you are yet unaware of the subtle and very difficult nature of the Br\=ahmi\d{n}ic religion. Now you want to cast your own fault on my shoulders and you are trying in vain to curse me. Anger is more to be blamed than Ch\=and\=ala! The wise men should overcome it by all means. When you, infuriated with rage, have been able to curse me for nothing, then I now curse you, ``Let your this body, inflamed with

anger, drop off.'' O King! Thus the King cursed the Muni and the Muni cursed the King; and both of them were, therefore, very sorry.

51-52. Va\'sistha then became troubled with cares and took shelter with Brahm\=a and informing him about the great curse given by the King Nimi said :-- Father! The King has cursed me saying, ``Let your body fall off today. Now the great trouble due to the falling off of the body has arisen. What am to do now?

53-69. O Father! Kindly tell me from whom shall I take my birth and take such means as I can get a body like what I have now. Also by Your unbounded power, do so that I can retain the knowledge in that body as I have at present; You are fully competent to do this.'' O King! Hearing thus the words of Va\'sistha, Brahm\=a spoke thus to his dear son :-- Go and enter into the Tejas (essence) of Mitr\=a Varu\d{n}a and remain contented; then you will get, in due time, a body not born of any womb and you will be again religious, truthful, knower of the Vedas, all-knowing and worshipped by all; there is no doubt in this. When Brahm\=a said this, the Mahar\d{s}i Va\'sistha bowed down to the Grand Sire, and circumambulating him, went to the abode of Varu\d{n}a. Then he quitted his excellent body; and, with his subtle body, the part of his J\={\i}va, entered into the body of Mitr\=a Varu\d{n}a. Then once on a time Urva\'s\={\i}, exquisitely beautiful and lovely, surrounded by her comrades, went wilfully into the abode of Varu\d{n}a. Mitr\=a-Varu\d{n}a, the two Devas became very passionate to see that Apsar\=a (the celestial nymph) endowed with youth and beauty and being enchanted with the arrows of cupid, and, being senseless, addressed to the Deva Kany\=a Urva\'s\={\i}, beautiful in all her parts, thus :-- ``O Lovely One! Seeing you, we are very much troubled with the arrows of cupid; O Beautiful One! Select us and remain and enjoy here at your pleasure.'' When they said thus, Urva\'s\={\i} became attached to them; and, under their control, began to stay in the house of Mitr\=a Varu\d{n}a. When Urva\'s\={\i} began to remain there, strongly attached to them, the semen of Mitr\=a Varu\d{n}a dropped in an uncovered jar. And the two beautiful sons of the \d{R}i\d{s}is were born out of that; Agasti was the first child and Va\'sistha the second. Thus, out of the semen of Mitr\=a Varu\d{n}a, the two ascetics were born. The first Agasti turned out a great ascetic in his childhood and resorted to forest; Ik\d{s}\=aku the best of Kings, selected Va\'sistha as his family priest. O King! Ik\d{s}\=aku, the best of Kings, nursed him for the welfare of his own line; the more so, because to know that he was the Muni Va\'sistha; and thus he was very pleased with him. Janamejaya! Thus I have described to you about the getting of another body by Va\'sistha, due to the curse of Nimi, and have also described his re-birth in Mitr\=a-Varu\d{n}a's family.

Here ends the Fourteenth Chapter of the Sixth Book on the birth of Va\'sistha from Mitr\=a Varu\d{n}a in \'Sr\={\i} Mad Devi Bh\=agavatam, the Mah\=a Pur\=a\d{n}am of 18,000 verses by Mahar\d{s}i Veda Vy\=asa.



