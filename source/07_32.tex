\chapter{1-50. The Dev\={\i} said :-- "Hear, Ye Immortals! My words with attention, that I am now going to speak to you, hearing which will enable the J\={\i}vas to realise My Essence. Before the creation, I, only I, existed; nothing else was existent then. My Real Self is known by the names Chit, Sambit (Intelligence), Para Brahma and others. My \=Atman is beyond mind, beyond thought, beyond any name or mark, without any parallel, and beyond birth, death or any other change or transformation. My Self has one inherent power called M\=ay\=a. This M\=ay\=a is not existent, nor non-existent, nor can it be called both. This unspeakable substance M\=ay\=a always exists (till the final emancipation or Mok\d{s}a).}
M\=ay\=a can be destroyed by Brahma J\~n\=ana; so it can not be called existent, again if M\=ay\=a does not exist, the practical world cannot exist. So it cannot be called non-existent. Of course it cannot be called both, for it would involve contradictions. This M\=ay\=a (without beginning but with end at the time of Mok\d{s}a) naturally arises as heat comes out of fire, as the rays come out of the Sun and as the cooling rays come out of the Moon. Just as all the Karmas of the J\={\i}vas dissolve in deep sleep (\'Su\d{s}upti), so at the time of Pralaya or the General Dissolution, the Karmas of the J\={\i}vas, the J\={\i}vas and Time all become merged, in one uniform mass in this great M\=ay\=a. United with My \'Sakt\={\i}, I am the Cause of this world; this \'Sakt\={\i} has this defect that it has the power of hiding Me, its Originator.
I am Nirgu\d{n}a. And when I am united with my \'Sakt\={\i}, M\=ay\=a, 1 become Saguna, the Great Cause of this world. This M\=ay\=a is divided into two, Vidy\=a and Avidy\=a. Avidy\=a M\=ay\=a hides Me; whereas Vidy\=a M\=ay\=a does not. Avidy\=a creates whereas Vidy\=a M\=ay\=a liberates.
M\=ay\=a united with Chaitanya (Intelligence), i. e., Chid\=abh\=asa is the efficient cause of this Universe; whereas M\=ay\=a reduced to and united with five original elements is the material Cause of the Universe. Some call this M\=ay\=a tapas; some call Her inert, material; some call Her knowledge; some call Her M\=ay\=a, Pradh\=ana, Prakriti, Aj\=a (unborn) and some others call Her \'Sakt\={\i}. The \'Saiva authors call Her Vimar\'sa and the other Ved\=antists call Her Avidy\=a; in short, this M\=ay\=a is in the heads of all the Pundits. This M\=ay\=a is called various in the Nigamas.

That which is seen is inert; for this reason M\=ay\=a is Jada (inert) and as the knowledge it conveys is destroyed, it is false. Chaitanya (Intelligence) is not seen; if It were seen, it would have been Jada. Chaitanya is self-luminous; not illumined by any other source. Were It so, Its Enlightener would have to be illumined by some other thing and so the fallacy of Anavasth\=a creeps in (an endless series of causes and effects). Again one thing cannot be the actor and the thing, acted upon (being contrary to each other); so Chaitanya cannot be illumined by itself. So It is Self-luminous; and it illumines Sun, Moon, etc., as a lamp is self-luminous and illumines other objects. So, O Mountain! This My Intelligence is established as eternal and everlasting. The waking, dreaming and deep sleep states do not remain constant but the sense of "I" remains the same, whether in waking, dreaming or deep sleep state; its anomaly is never felt. (The Bauddhas say that) The sense of intelligence, J\~n\=ana, is also not felt; there is the absence of it; so what is existent is also temporarily existent. But (it can then be argued that) then the Witness by which that absence is sensed, that Intelligence, in the shape of the Witness, is eternal. So the Pundits of all the reasonable \'S\=astras declare that Samvit (Intelligence) is Eternal and it is Blissful the fountain of all love. Never the J\={\i}vas or embodied souls feel "I am not"; but "I am" this feeling is deeply established in the soul as Love. Thus it is clearly evident that I am quite separate from anything else which are all false. Also I am one continuous (no interval or separation existing within Me). Again J\~n\=ana is not the Dharma (the natural quality) of \=Atman but it is of the very nature of \=Atman. If J\~n\=ana ware the Dharma of \=Atman, then J\~n\=ana would have been material; so J\~n\=ana is immaterial. If (for argument's sake) J\~n\=ana be denominated as material, that cannot be. For J\~n\=ana is of the nature of Intelligence and \=Atman is of the the nature of Intelligence. Intelligence has not the attribute of being Dharma. Here the thing Chit is not different from its quality (Chit). So \=Atman is always of the nature of J\~n\=ana and happiness; Its nature is Truth; It is always Full, unattached and void of duality. This \=Atman again, united with M\=ay\=a, composed of desires and Karmas, wants to create, due to the want of discrimination, the twenty-four tattvas, according to the previous Samsk\=aras (tendencies), time and Karma. O Mountain! The re-awakening after Pralaya Su\d{s}upti is not done with Buddhi (for then Buddhi is not at all manifested). So this creation is said to be effected without any Buddhi (proper intelligence). O Chief of the Immovables! The Tattva (Reality) that I have spoken to you is most excellent and it is my Extraordinary Form merely. In the Vedas it is known as Avy\=akrita (unmodified), Avyakta (unmanifested)

M\=ay\=a \'Sabala (divided into various parts) and so forth. In all the \'S\=astras, it is stated to be the Cause of all causes, the Primeval Tattva and Sachchid\=ananda Vigraha. Where all the Karmas are solidified and where Ichch\=a \'Sakt\={\i} (will), J\~n\=ana \'Sakt\={\i} (intelligence) and Kriy\=a \'Sakt\={\i} (action) all are melted in one, that is called the Mantra Hr\={\i}m, that is the first Tattva. From this comes out \=Ak\=asa, having the property of sound, thence V\=ayu (air) with "touch" property; then fire with form, then water having "Rasa" property; and lastly the earth having the quality "smell." The Pundits say that the "sound" is the only quality of \=Ak\=asa; air has two qualities viz., sound and touch, fire has three qualities sound, touch, form; water has four qualities sound, touch, form, taste; and the earth has five qualities sound, touch, form, taste and smell. Out of these five original elements, the allpervading, S\=utra (string or thread) arose. This S\=utr\=atman (soul) is called the "Linga Deha," comprising within itself all the Pr\=a\d{n}as; this is the subtle body of the Param\=atman. And what is said in the previous lines as Avyakta or Unmanifested and in which the Seed of the World is involved and whence the Linga Deha has sprung, that is called the Causal body (K\=ara\d{n}a body) of the Param\=atman. The five original elements (Apa\~nchikrita called the five Tan M\=atr\=as) being created, next by the Pa\~nch\={\i}kara\d{n}a process, the gross elements are created. The process is now being stated :-- O Girij\=a! Each of the five original elements is divided into two parts; one part of each of which is subdivided into four parts. This fourth part of each is united with the half of four other elements different from it and thus each gross element is formed. By these five gross elements, the Cosmic (Vir\=at) body is formed and this is called the Gross Body of the God. J\~n\=anendriyas (the organs of knowledge) arise from Sattva Gu\d{n}as of each of these five elements. Again the Sattva Gu\d{n}as of each of the J\~n\=anendriyas united become the Antah Kara\d{n}\=ani. This Antah kara\d{n}a is of four kinds, according as its functions vary. When it is engaged in forming Sankalpas, resolves, and Vikalpas (doubts) it is called "mind." When it is free from doubts and when it arrives at the decisive conclusion, it is called "Chitta"; and when it rests simply on itself in the shape of the feeling "I", it is called Ahamk\=ara. From the Rajo Gu\d{n}a of each of the five elements arises V\=ak (speech), P\=a\d{n}i (hands) P\=ada (feet), P\=ayu (Anus) and Upastha (organs of generation). Again their Rajo parts united give rise to the five Pr\=a\d{n}as (Pr\=a\d{n}a, Ap\=ana, Sam\=ana, Ud\=ana and Vy\=ana) the Pr\=a\d{n}a Vayu resides in the heart; Ap\=ana Vayu in the Arms; Sam\=ana Vayu resides in the Navel; Ud\=ana Vayu rasides in the Throat; and the Vy\=ana V\=ayu resides, pervading all over the body. My subtle body (Linga Deha) arises from the union of the five

J\~n\=anendriyas, the five Karmendriyas (organs of action), the five Pra\d{n}as and the mind and Buddhi, these seventeen elements. And the Prakriti that resides there is divided into two parts; one is pure (Suddha Sattva) M\=ay\=a and the other is the impure M\=ay\=a or Avidy\=a united with the Gu\d{n}as. By M\=ay\=a is meant She, who, without concealing Her refugees, protects them. When the Supreme Self is reflected on this \'Suddha Sattva, M\=ay\=a, He is called \=I\'svara. This Suddha M\=ay\=a does not conceal Brahma, its receptacle; therefore She knows the All-pervading Brahma and She is omniscient, omnipotent, the Lady of all and confers favours and blessings on all. When the Supreme Self is reflected on the Impure M\=ay\=a or Avidy\=a, He is called J\={\i}va. This Avidy\=a conceals Brahma, Whose nature is Happiness; therefore this J\={\i}va is the source of all miseries. Both \=I\'svara and J\={\i}va have, by the influence of Vidy\=a and Avidy\=a three bodies and three names. When the J\={\i}va lives in his causal body, he is named Pr\=aj\~na; when he lives in subtle body he is known as Taijasa; while he has the gross body, he is called Vi\'sva. So when \=I\'svara is in His causal body, he is denominated \=I\'sa; when He is in His subtle body, he is known as S\=utra; and when He is in His gross body, He is known as Vir\=at.
The J\={\i}va glories in having three (as above-mentioned) kinds of differentiated bodies and \=I\'svara glories in having three (as above-mentioned) kinds of cosmic bodies. Thus \=I\'svara is the Lord of all and though He feels Himself always happy and satisfied, yet to favour the J\={\i}vas and to give them liberation (Mok\d{s}a) He has created various sorts of worldly things for their Bhogas (enjoyments). This \=I\'svara creates all the Universe, impelled by My Brahma \'Sakt\={\i}. I am of the nature of Brahma; and \=I\'svara is conceived in Me as a snake is imagined in a rope. Therefore \=I\'svara has to remain dependent on My \'Sakti.
Here ends the Thirty-second Chapter of the Seventh Book on Self-realization, spoken by the World Mother in the Mah\=apur\=a\d{n}am \'Sr\={\i} Mad Dev\={\i} Bh\=agavatam, of 18,000 verses, by Mahar\d{s}i Veda Vy\=asa.



