\chapter{On the King Suratha's going to the forest}

1-4. Janamejaya said :-- O Best of Munis! The glory of Chandik\=a has been fully described by you. By whom was She worshipped in the ancient times after the reading and hearing of Her three glorious deeds (the killing of Madhu Kaitava, etc.)? Who was it that derived the best effects by worshipping the Dev\={\i}, the Bestower of all desires? When and with whom was She pleased and then offered boons? O Ocean of mercy! Kindly narrate fully all these things to me. O Br\=ahma\d{n}a! Describe to me also the rules how the meditation, worship and Homa of the Great Dev\={\i} are conducted. S\=uta said :-- ``O \d{R}i\d{s}is! Kri\d{s}\d{n}a Dvaip\=ayana, the son of Satyavat\={\i}, was very glad to hear these questions of Janamejaya and began to describe how the worship, etc., of the Mah\=a M\=ay\=a, the Dev\={\i} Bhagavat\={\i} are to be done.''

5-21. Vy\=asa said :-- O King! In days of yore in Sv\=arochi\d{s}a Manvantara there was a king, named Suratha, very liberal-minded and devoted to govern well his subjects. He was truthful, active and energetic, and devoted to his Guru; he always served the twice-born and he never used to hold any sexual intercourse except with his legal wife. He was generous, not liking to quarrel with anybody, and expert in the science of archery. While he was thus governing his kingdom, the Mlechchas, the hill tribes, turned out his enemies. They destroyed the city of Kol\=a, became very haughty and turbulent and desired to conquer the whole earth by their sheer force. Thus accompanied by the great four-fold army

elephants, chariots, cavalry and infantry they came to conquer the dominion of the King Suratha. A dreadful fight then ensued between the King and the dreadful Mlechchas. O King! The Mlechcha forces were not at all numerous whereas the armies of the king were large; still the Mlechchas were lucky to win the battle. The King, defeated, fled to his own city which was a strongly fortified place. The good King, wise in statesmanship when he saw that his ministers had gone over to the enemie\'s party, became very anxious and thought whether it was advisable for him to wait for a better opportunity, remaining within his own extensive city, well guarded by a strong wall and ditch or it would be better to fight on. The King thought also that it would not at all be advisable to consult with his ministers who were, then, under the control of his enemies; what then would he do under the circumstances? Those vicious ministers could at any time deliver him to the hands of his enemies; what would then happen to him! Those men, that are avaricious, can do anything in this world; therefore it would never be advisable to trust them. The people under the sway of greed commit injury to their fathers, brothers, friends, acquaintances, their Gurus and the adored Br\=ahma\d{n}as. When the ministers had joined with his enemies, they could well be classed with the vicious; no doubt in this. Never could they be trusted under the above circumstances. Thus pondering over the matter, the King became absent-minded, and, finding no remedy, went out of the city alone, mounted on a horse. The intelligent King, helpless, entered into a dense forest and thought where would he go now? Knowing, then, that there was, at a distance of three Yojanas from that place, a hermitage of the great ascetic the Sumedh\=a \d{R}i\d{s}i, the King went there. (N.B. :-- A Yojana is a distance measuring four Krosas or eight or nine miles.)

22-33. O King! That hermitage was more beautiful than even the Heavens; it was on the bank of a river; various kinds of trees were there; it was frequented with wild animals having no enmity with each other; the whole place was echoed with the sounds of cuckoos. The students were studying and reverberating the atmosphere with their Vedic chants; hundreds of herds of deers were running there; rice trees had grown there wildly at places and their harvests were collected at places; good flowery trees and others with delicious fruits were seen there; at places fragrant smells of oblations of ghee, etc., were coming; all these were delighting to any man who went or stayed there. The King Suratha was very glad to see that \=A\'srama; he became fearless and wanted to stay there in the hermitage of the Br\=ahmi\d{n}. Fastening his horse at the root of a tree, the King approached humbly to the \d{R}i\d{s}i, and saw that the

Muni was seated on a deer skin under the shade of dense S\=al trees. He was peaceful, lean and thin by tapasy\=a. His stature was straight; and he was teaching his disciples and explaining to them the meaning of the Veda \'S\=astras.

He was void of anger, greed, etc., beyond all the dualities, without any jealousy, always devoted to the contemplation of his Self, truthful and full of peace. Seeing him the King was filled with tears and prostrated before him and fell like a stick before him. The Muni, seeing him thus asked him to get up and enquired about his welfare. A disciple then at the sign of the Guru, gave him a Ku\'s\=asan, to take his seat. The King got up and at his permission took his seat on that ku\'s\=asan; the the Muni worshipped the King duly by offering to him water to wash his feet, and Arghya (an offer of green grass, rice, etc). Then the Muni asked him, ``Who are you? What for are you come here? Why are you so anxious? Tell frankly all these that are not yet known to me. What do you want? Speak out your mind. Even if that be impracticable, I will no doubt try my best to accomplish your desired ends.''

34-36. The King said :-- ``O Muni! I am the King Suratha; defeated by my enemy, I have left my kingdom, palace, and wife and have come to your refuge. O Br\=ahma\d{n}a! I am ready to do whatever you order me; on this surface of the earth there is no one but you who can protect me. Now I am very much terrified by my enemy; therefore I have come to you. O Muni! You protect those who come to seek your refuge; I have now come here to seek your shelter; so save me from this danger.''

37-38. The Mahar\d{s}i said :-- ``O King! Stay here without any fear; none of your enemies would be able to enter this hermitage by my power of Tapasy\=a, even if they be very powerful. O Best of Kings! You will not be allowed to kill any animals here; you will have to sustain yourself on this wild rice, roots and fruits, etc., as the rules of the forest living permit.''

39-48. Vy\=asa said :-- Thus hearing his words, the King began to live there, with all purity and without any fear, on roots and fruits. Once the King, while taking rest under the shade of a tree, while thinking of various things, thought of his own house thus :-- ``My enemies have, no doubt acquired my kingdom, but they are vicious and wicked, shameless Mlechchas and always addicted to sinful deeds; certainly they are tormenting my subjects. My elephants and horses are not regularly getting their food and have all become powerless; certainly they are suffering very much from my enemies. All the servants that were nourished by me before are now all suffering from troubles, having been subjected by my enemies. The wicked enemies are certainly squandering away my hoarded

wealth to bad immoral purposes, in gambling, drinking and in revelling with prostitutes. Those Mlechchas and my ministers are always intent on vicious acts; they do not know who are the proper persons to be given charities; so they will no doubt exhaust away my coffers in doing sinful acts.'' While the King was thus meditating, seated at the root of a tree, there came one man of the Vai\'sya caste looking very distressed. The King saw and instantly bade him take his seat beside him; then the King asked the Vai\'sya :-- ``O Noble One! Of what caste are you? Whence are you coming to this forest? What is your name? What for you look so pale and distressed? What calamity has befallen to you? O Good One! Two persons become friends whenever they speak seven words amongst them; according to this rule I am your friend; tell me, therefore, truly all these things.''

49. Vy\=asa said :-- The Vai\'sya, hearing these words from the King, took his seat and felt himself much relieved and thinking that he has met with a saint, began to speak thus :--

50-52. O my Friend! I belong to the Vai\'sya caste; my name is Sam\=adhi; I was rich, never I had any jealousy towards anybody; always I used to speak truth and was devoted to religious acts. My wife and sons are very greedy of money and are irreligious; so they cut off all their affections and connections with me, very difficult to cut though, and have driven me out of the house on the pretext that I am very miserly. Thus forsaken by my relatives, I have now come to this forest. You look to be a fortunate man; therefore kindly, O Dear One! give me now your introduction and oblige.

53-55. The King said :-- I am the King Suratha; lately I had a defeat from the dacoits; moreover my ministers deceived me; consequently I am deprived of my kingdom and have now come here. O Best of Vai\'syas! Fortunately you have come to me today as my friend. We two will repose here gladly in this beautiful forest covered with trees. O Intelligent One! Now quit your sorrow; be calm and quiet and rest with me, at your leisure, here happily.

56-58. The Vai\'sya said :-- O King! My friends and relatives must have been helpless, very sorrowful and they are distressed at my absence; they must have been troubled very much by diseases and misfortunes no doubt and have become very anxious. O King! I cannot remain quiet; my mind is being troubled with the thought how my wife and sons are spending their times now in pain or happiness? I am always thinking when I would see again my sons, wife, relatives, friends, acquaintances and my house? I cannot make me calm and quiet.

59-60. The King said :-- O Intelligent One! What pleasure can you expect to see your wicked sons and treacherous relatives who have driven you out of your house? Even the enemies are far better, provided they do good to us; what sorts of friends are they who impose on us afflictions and sorrows. Do you, therefore, make your mind calm and quiet and remain here in greatest peace and happiness.

61. The Vai\'sya said :-- O King! Even those that are wicked and cruel cannot quit their relatives. Today my mind is greatly agitated with the thought of my relatives; I cannot remain quiet.

62. The King said :-- My mind too, is incessantly troubled with the thought of my kingdom. Come; let both of us go to the Muni and ask him what is the medicine for the cure of these our mental agonies.

63-64. Vy\=asa said :-- O King! Thus making their determinations, they went humbly to the Muni to ask him what were the causes of their sorrows? The King then went close to him and bowing down before him, took his seat and began to ask calmly and quietly the Muni who was sitting calm and serene.

Here ends the Thirty-second Chapter of the Fifth Book on the King Suratha's going to the forest in the Mah\=apur\=a\d{n}am, \'Sr\={\i} Mad Dev\={\i} Bh\=agavatam of 18,000 verses by Mahar\d{s}i Veda Vy\=asa.



