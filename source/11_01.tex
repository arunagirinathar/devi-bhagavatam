\chapter{On what is to be thought of in the morning}

1-13. N\=arada said :-- O Bhagav\=an! O Thou, the Eternal One! O N\=ar\=aya\d{n}a! O Lord of the past and the future! Thou art the Creator and the Lord of all the beings that lived in the past and that will come into existence in the future. Thou hast described to me the highly wonderful and excellent anecdote of the Exalted Dev\={\i}. How She did assume the forms of Mah\=ak\=al\={\i}, Mah\=a Lak\d{s}m\={\i}, Mah\=a Sarasvat\={\i} and Bhr\=amar\={\i}, for the fulfilment of the Deva\'s purposes and how the Devas got back their possessions by the Grace of the Dev\={\i}. All you have described. O Lord! Now I want to hear the rules of Sad\=ach\=ara (right way of living), the due observance of which by the devotees pleases the World-Mother. Kindly describe them.

N\=ar\=aya\d{n}a said :-- O Knower of Truth! Now I am telling you those rules of the right way of living, which rightly observed, always please Bhagavat\={\i}. Listen first, I will talk of the Br\=ahmi\d{n}s, how their welfare is secured, what the Br\=ahma\d{n}as ought to do on getting up early in the morning from their bed. From the sunrise to the sunset the Br\=ahma\d{n}as should do all the daily and occasional duties (Nitya and Naimittik Karmas) and they are to perform the optional works for some particular object such as Puttresti Yaj\~na and other good works (not acts of black magic as killing, causing pain and inconveniences to others, etc.). It is the Self alone and not the Father, Mother, etc., nor any other body that, helps us on our way to that happiness in the next world. Father, Mother, wife, sons and others are helps merely to our happiness in this world. None of them are helpful to us in bettering our states in the next world.

Deliverance of one's Self depends verily on his own Self. Therefore one should always earn and store dharma (religion) and observe always there the right conduct to help one in the next world. If Dharma be on our side, this endless sea of troubles can be safely crossed. The rules of right living as ordained by Manu in \'Srutis and Manu Smritis are the principal Dharmas. The Br\=ahma\d{n}as should always be observant to their Dharma as ordained in the \'S\=astras, \'Sruti and Smriti. Follow the right conduct and then you will get life, posterity and increase of happiness easily here and hereafter. By right conduct, food is obtained and sins are easily destroyed; the right conduct is the auspicious principal

Dharma of men. Persons of right living enjoy happiness in this world as well as in the next. Those, who are veiled in darkness by Ignorance and thus wildly enchanted, can verily see their way to Mukti if they follow the Great Light revealed to them by Dharma and the right conduct. It is by Sad\=ach\=ara, that superiority is attained. Men of right conduct always do good deeds. From good deeds, knowledge comes. This is the advice of Manu.

14-24. Right way of living is the best of all the Dharmas and is great Tapasy\=a (asceticism). The knowledge comes from this Right Living. Everything is attained thereby. He who is devoid of Sad\=ach\=ara, is like a \'S\=udra, even if he comes of a Br\=ahmi\d{n} family. There is no distinction whatsoever betwen him and a \'S\=udra. Right conduct is of two kinds :-- (1) as dictated by the \'S\=astras, (2) as dictated by the popular custom (Laukika). Both these methods should be observed by him who wants welfare for his Self. He is not to forsake one of them. O Muni! The village Dharma, the Dharma of one's own caste, the Dharma of one's own family and the Dharma of one's own country all should be observed by men. Never, never he is to do anything otherwise. With great loving devotion that is to be preserved. Men who practise wrong ways of living, are censured by the public; they always suffer from diseases. Avoid wealth and desires that have no Dharma in them. Why? If in the name of dharma, painful acts (e. g. killing animals in sacrifices) are to be committed, those are blamed by the people; so never commit them. Avoid them by all means. N\=arada said :-- ``O Muni! The \'S\=astras are not one, they are many and they lay down different rules and contradictory opinions, How then Dharma is to be followed? And according what Dharma \'S\=astra?'' N\=ar\=aya\d{n}a said :-- \'Sruti and Smriti are the two eyes of God; the Pur\=a\d{n}am is His Heart. Whatever is stated in \'Sruti, the Smriti and the Pur\=a\d{n}ams is Dharma; whatever else is written in other \'S\=astras is not Dharma. Where you will find differences between \'Sruti, Smriti and Pur\=a\d{n}as, accept the words of the \'Srutis as final proofs. Wherever Smriti disagrees with the Pur\=a\d{n}as, know the Smritis more authoritative.

And where differences will crop up in the \'Srutis themselves, know that Dharma, too, is of two kinds. And where the differences will crop up in the Smritis themselves, consider, then, that different things are aimed at. In some Pur\=a\d{n}as, the Dharma of the Tantras is duly described; but of these, which go against the Vedas, they are not to be accepted any means.

25-37. Tantra is accepted as the authoritative proof then and then only when it contradicts not the Vedas. Whatever goes clearly against the Vedas can in no way be accepted as a proof. In matters concerning Dharma, the Vedas is the Sole Proof. Therefore that which is not against the Vedas can be taken as proof; otherwise not. Whoever acts Dharma according to other proofs than what is ordained in the Vedas, goes to the hell in the abode of Yama to get his lesson. So the Dharma that is by all means to be accepted as such, is what is stated in the Vedas. The Smritis, the Pur\=a\d{n}as, or the Tantra \'S\=astras can be taken also as authoritative when they are not conflicting to Vedas. Any other \'S\=astras can be taken as authoritative when it is fundamentally coincident with the Vedas. Else it can never be accepted.

Those who do injury to others even by the blade of a Ku\'sa grass used as a weapon, go to hell with their heads downwards and their feet upwards. Those that follow their own sweet free will, that take up any sort of dress (e. g. Bauddhas), those that follow the philosophical doctrines called P\=a\'supatas, and the other hermits and saints and persons that take up other vows contrary to the religions of the Vedas, for example, the Vaikh\=anasa followers, those who brand their bodies by the hot Mudr\=as, at the places of pilgrimages, e. g. Dv\=ark\=a, etc., they go to hell with their bodies scorched by red hot brands (Tapta Mudr\=as). So persons should act according to the excellent religions commanded by the Vedas. Everyday he should get up from his bed early in the morning and think thus :-- ``What good acts have I done, what have I given as charities? Or what I advised others to do charities what greater sins (Mah\=ap\=atakas) and what smaller sins have I committed?'' At the last quarter of the night he should think of Para Brahm\=a. He should place his right leg on his left thigh and his left on his right thigh crosswise keeping his head straight up and touching the breast with his chin, and closing his eyes, he should sit steadily so that the upper teeth should not touch the lower jaw.

He should join his tongue with his palate and he should sit quiet, restraining his senses. He should be \'Suddha Sattva. His seat should not be very low. First of all he should practice Pr\=a\d{n}\=ay\=ama twice or thrice; and within his heart he should meditate the Self of the shape of the Holy Flame or the Holy Light. (Om Mani Padma Hum.)

38-49. He should fix his heart for a certain time to that Luminous Self whose Eyes are everywhere. So the intelligent man should practise Dh\=ara\d{n}\=a. Pr\=a\d{n}\=ay\=ama is of six kinds :-- (1) Sadh\=uma (when the breaths are not steady), (2) Nirdh\=uma(better than the Sadh\=uma),

(3) Sagarbha (when united with one's mantra), (4) Agarbha (when the practice is without the thought of any mantra), (5) Salak\d{s}ya (when the heart is fixed on one's Deity) and (6) Alak\d{s}ya (when the heart is not fixed on one's Deity). No yoga can be compared with Pr\=a\d{n}\=ay\=ama. This is equal to itself. Nothing can be its equal. This Pr\=a\d{n}\=ay\=ama is of three kinds, called Rechaka, P\=uraka and Kumbhaka. The Pr\=an\=ay\=ama consists of three letters, A, U, M, i. e. of the nature of ``Om''. Or, in other words the letter A, of the Pra\d{n}ava Om indicates P\=uraka, the letter ``U'' denotes Kumhhaka and the letter ``M'' denotes Rechaka. By the \=Id\=a N\=adi (by the left nostril) inhale as long as you count ``A'' (Vi\d{s}\d{n}u) thirty-two times; then withold breath, i.e., do Kumbhaka as long as you count ``U'' (\'Siva) sixtyfour times and by the Pingal\=a N\=adi (the right nostril) do the Rechaka, i.e., exhale the breath as long as you count ``M" (Brahm\=a) for sixteen times. O Muni! Thus I have spoken to you of the Sadh\=uma Pr\=a\d{n}\=ay\=ama. After doing the Pr\=a\d{n}\=ay\=ama as stated above, pierce the Six Chakras (i. e., plexuses) (called Sathakra bheda) and carry the Kula Kundalin\={\i} to the Brahm\=a Randhra, the brain aperture, or to the thousand petalled lotus in the head and meditate in the heart the Self like a Steady Flame. (The N\=adis are not those which are known to the Vaidya or the Medical \'S\=astras. The latter are the gross physical nerves, The N\=adis here are the Yoga N\=adis, the subtle channels (Vivaras) along which the Pr\=a\d{n}ik currents flow. Now the process of piercing the six Chakras (or nerve centres or centres of moving Pr\=anik forces) is being described. Within this body, the six nerve centres called Padmas (Lotuses) exist. They are respectively situated at the (1) M\=ul\=adh\=ara (half way between Anus and Linga M\=ula), called the Sacral Plexus; (2) Linga M\=ula (the root of the genital organs); called postatic plexus; (this is also called Sv\=adhisth\=ana) (3) Navel, the Solar Plexus (4) Heart, the cardiac Plexus, (5) Throat (6) Forehead, between the eye brows there; the lotus in the forehead, called the cavernous plexus (\=Ajn\=a Chakra) has two petals; in these two petals, the two letters ``Ham'' ``K\d{s}am'' exist in the right hand direction (with the bands of the watch; going round from left to right keeping the right side towards one circumambulated as as a mark of respect). I bow down to these which are the two-lettered Brahm\=a. The lotus that exists in the throat laryngeal or pharyngeal plexus has sixteen petals (vi\'suddh\=a chakra); in these are in due order in right hand direction the sixteen letters (vowels) a, \=a, i, \={\i}, u, \=u, ri, r\={\i}, lri, lr\={\i}, e, ai, o, au, am, ah; I bow down to these which are the sixteen lettered Brahm\=a. The lotus that exists in the heart, the cardiac plexus (an\=ahata chakra), has twelve petals; wherein are the twelve letters k, kh, g, gh, n, ch, chh, j, jh, \~n, t, th; I bow to to these twelve lettered Brahm\=a. The Solar

plexus forms the Great Junction of the Right and Left sympathetic chains \=Id\=a and Pingal\=a with the Cerebro spinal Axis. The lotus that exists in the navel, called the Solar Plexus, or Epigastric plexus (Ma\d{n}ipura Chakra) has ten petals wherein are the ten letters d, dh, \d{n}, t, th, d, dh, n, p, ph, counting in the right hand direction (that is clockwise) (and the action of this clock is vertical in the plane of the spinal cord; also it may be horizontal). The lotus that exists at the root of the genital organ, the genital plexus or postatic plexus has six petals. The petals are the configurations made by the position of N\=adis at any particular centre. Sv\=adhisth\=ana chakra or Svayambhu Linga, wherein are situated the six letters, b, bh, m, y, r, l; I bow down to this six-lettered Brahm\=a. (These are the Laya Centres). The lotus that exists in the M\=ul\=adh\=ara, called the sacral or sacrococcygeal plexus has four petals, wherein are the four letters v, \'s, \d{s}, s. I bow down to these four-lettered Brahm\=a. In the above six nerve centres or Laya Centres, or lotuses, all the letters are situated in the right hand direction (clockwise). (Note :-- All the nerves of the body combine themselves in these six nerve centres or Laya Centres. Each of these centres is spheroidal and is of the Fourth Dimension. At each centre many transitions take place, many visions take place, many forces are perceived and wonderful varieties of knowledge are experienced. These are called the Laya Centres. For many things vanish into non-existence and many new Tattvas are experienced.) Thus meditating on the Six Chakras or plexuses, meditate on the Kula Kundalin\={\i}, the Serpent Fire. She resides on the four petalled lotus (Centre of \'Sakti) called M\=ul\=adh\=ara Chakra (Coccygeal plexus); She is of Rajo Gu\d{n}a; She is of a blood red colour, and She is expressed by the mantra ``Hr\={\i}m,'' which is the M\=ay\=av\={\i}ja; she is subtle as the thread of the fibrous stock of the water lily. The Sun is Her face; Fire is Her breasts; he attains J\={\i}van mukti (liberation while living) within whose heart such a Kula Kundalin\={\i} arises and awakens even once. Thus meditating on Kula Kundalin\={\i}, one should pray to Her :-- Her sitting, coming, going, remaining, the thought on Her, the realisation of Her and chanting hymns to Her, etc., all are Mine, Who is of the nature of all in all; I am that Bhagavat\={\i}; O Bhagavat\={\i}! All my acts are Thy worship; I am the Dev\={\i}; I am Brahm\=a, I am free from sorrow. I am of the nature of Everlasting Existence, Intelligence and Bliss. Thus one should meditate of one's own-self. I take refuge of that Kula Kundalin\={\i}, who appears like lightning and who holds the current thereof, when going to Brahm\=arandhra, in the brain, who appears like nectar when coming back from the brain to the M\=ul\=adh\=ara and who travels in the Su\d{s}umn\=a N\=adi in the spinal cord. Then one is to meditate on one's own Guru, who is thought of as one with God, as seated

in one's brain and then worship Him mentally. Then the S\=adhaka, controlling himself is to recite the following Mantra ``The Guru is Brahm\=a, the Guru is Vi\d{s}\d{n}u, it is the Guru again that is the Deva Mahe\'svara; it is Guru that is Para Brahm\=a. I bow down to that \'Sr\={\i} Guru.''

Here ends the First Chapter of the Eleventh Book on what is to be thought of in the morning in the Mah\=a Pur\=a\d{n}am \'Sr\={\i} Mad Dev\={\i} Bh\=agavatam of 18,000 verses by Mah\=ar\d{s}i Veda Vy\=asa.



