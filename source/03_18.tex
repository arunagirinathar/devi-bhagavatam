\chapter{The Svayambara of \'Sa\'sikal\=a}

1. Vy\=asa said :-- The King's lovely daughter was very glad on hearing the words of the Br\=ahma\d{n}a, and drowned herself in ecstacy of love. The Br\=ahmi\d{n} also departed, thinking of the whole affair.

2. The daughter was already attached to the prince, and now she became the more merged in love for him and became very anxious. Now, on the departure of the Br\=ahmi\d{n}, she felt herself struck by the arrows of love.

3-4. Then \'Sa\'sikal\=a, oppressed by love, addressed her dear companion, who followed her inclinations thus :-- ``O my companion! I have not as yet had any knowledge of the king's son; still the signs of love have sprung up in my body and mind, from the moment that I heard about him from the Br\=ahmi\d{n}. The love is giving me much trouble; tell, my companion what am I to do now? and whither shall I go?

5. O dear companion! I saw him like a second God of Love in my dreams; and, since then, my innocent mind is being troubled with his being away from me.

6. O fair one! The sandal paste on my body appears to me like a poison, this garland is like a serpent and the moon's rays seem like a fire.

7. O companion! My mind gets not rest anywhere, in palaces, gardens, in lakes, in hills, at any time, during the day or night; all the enjoyable things have assumed now contrary aspects and are paining me.

8. The bedding, betel leaves, music, singing, and dancing, all now fail to give me satisfaction and peace.

9. O companion! I would have gone to-day where is residing that deceiver; but I fear for my father as well for the honour of my family.

10. My father is not yet declaring the svayamvara for my marriage. What shall I do? Had he given me in marriage to that Sudar\'san, I would have allowed him embrace me and satisfy his passions to-day!

11. O friend! look at the strange ideas of the Creator! There are hundreds of kings today who are influential and I do not consider them beautiful; and that King's son is exiled from his kingdom and yet he has stolen away my heart.''

12. Vy\=asa said :-- Thus that King's son, Sudar\'sana, though helpless, and living in a forest on roots and fruits, deprived of wealth, power, and army, began to reign in the heart of that princess.

13-14. \'Sa\'sikal\=a, too, began to recite slowly the root mantra of Sarasvat\={\i} and therefore her love towards this prince held out signs of success.

Once engaged in meditation on that excellent root mantra on K\=ama, and, while repeatedly reciting it mentally, he got in a dream the vision of that Ever Full, the World Mother Ambik\=a, that cannot be expressed in words, the Vai\d{s}\d{n}av\={\i} \'Sakti and capable to bestow all wealth and property.

15-16. At this time the King of Ni\d{s}\=adas, the lord of \'Srimgaverpur came to the hermitage and presented an excellent chariot together with all other necessary things. This chariot was drawn by four horses, decorated with nice flags and was endowed with the prospect of getting victory everywhere; thinking it thus a befitting present to be given to the King, he gave it to Sudar\'sana.

17. Sudar\'sana, too, accepted the offer of a friend and worshipped him well in return, with roots and fruits of the forest.

18-19. The lord of the Ni\d{s}\=adas, thus worshipped as a guest, went away. The Munis and ascetics then began to address Sudar\'sana, with fondness ``O son of the king! Don't be anxious and restless; within a very short time, you will get your kingdom, no doubt, by your own good luck and prowess.

20. O one attached to one's vows! The Goddess that enchants the whole universe, the Giver of boons, \'Sr\={\i} Ambik\=a, has been pleased with you; assistance is also rendered well to you; therefore do not trouble yourself with contrary thoughts any more.''

21. The Munis, who have taken vows, addressed Manoram\=a also ``Your son will soon become the lord of the world; you need not care any more.''

22. Then the lean and thin Manoram\=a, hearing the Muni's words said ``O Br\=ahma\d{n}as! Let your \=asiss (words of benedictions) be justified with success. What wonder that a kingdom be obtained by the good will of the S\=adhus!

23. There is no force, no minister, no help, no property; how, under what combinations, can then my son get kingship?

24. You are the best of the knowers of mantrams; due to the influence of your good will, my son will surely be a king; there is no doubt in this.''

25. Vy\=asa said :-- Wherever that intelligent Sudar\'sana used to go on his chariot, there he seemed by his own prowess, as if he were surrounded by a whole army consisting of 109, 350 foot, 65,610 horse, 21,870 chariots and 21,870 elephants.

26. This is the influence of the seed mantra; it is not an ordinary acquisition. It is because Sudar\'sana, with gladness and one-pointedness of his mind, meditated on his seed mantra, that he acquired the above powers; there is no doubt in this.

27. Becoming pure and peaceful, whoever gets this wonderul seed mantra from a true spiritual guide and meditates on it incessantly, is destined certainly to attain all desires.

28. O best of kings! There is no such thing either in the heavens above or in the worlds below, that a man won't get, when the Supreme Goddess becomes pleased.

29. Those are certainly very unfortunate and of dull intellect, who cannot place their faith on the worship of this Goddess and consequently suffer incessantly all sorts of troubles.

30. O best of the Kurus! At the beginning of the creation, this Amb\=a Dev\={\i} was the Mother of all the Gods, and is therefore known as the First Mother.

31. She is practically seen in this world in the shape of Buddhi (iutellgence), K\={\i}rti (fame), Dhr\={\i}ti (fortitude), Lak\d{s}m\={\i} (Goddess of wealth), \'Sakti (the Force), \'Sraddh\=a (Faith), Mati (Intellect), Smriti (memory), etc.

32. It is only the deluded souls that do not realise the nature of the Dev\={\i}; it is only those, whose hearts are destroyed by the glare of false argument, that do not worship this All-auspicious Goddess of the Universe.

33-34. O king! Brahm\=a, Vi\d{s}\d{n}u, \'Sambhu, Indra, Varu\d{n}a, Yama, V\=ay\=u, Agni, Kuvera, Vi\'svakarm\=a, P\=u\d{s}\=a, Bhaga, the two Asvins, \=Adityas, Vasus, Rudras, Vi\'svedevas, Maruts, all worship the Supreme Deity of Creation, Preservation and Dissolution.

35. Who is there amongst tho wise that does not serve this Highest Energy? The real nature of that Auspicious Goddess, the Bestower all desires, Sudar\'sana came to know very well.

36. She is the Real Essence, Brahm\=a, very rarely realised; She is the Higher Vidy\=a and the Lower Vidy\=a (Avidy\=a) and She is the vital energy, the Mukhya Pr\=a\d{n}a, of the best of the Yogis, who are desirous of liberation.

37. O king! What individual is there that is able to realise the nature of Pram\=atm\=a (the Highest Universal Self) without having recourse to Her, Who is manifesting this universal consciousness, by bringing into existence these S\=atvik, R\=ajasik, and T\=amasik creations.

38. Sudar\'sana, though he dwelt in the forest, realised a greater happiness than that in obtaining the sovereignty of a kingdom, by constantly meditating on that Goddess.

39. \'Sa\'sikal\=a, too, being too much oppressed with the arrows of love, any how remained with her soul in her body, having had to be always cared for her health in various ways by her attendants.

40. Then the king Sub\=ahu, on coming to know that her daughter is desirous of getting her husband, made arrangements for her Svayambara (a marriage in which the girl chooses her husband from among a number of suitors assembled together) without any delay.

41-44. The Svayambara of the royal family, the Pundits say, is of three kinds :-- lst Ichchh\=a Svayambara (optional); 2nd Pa\d{n}ya Svayamvara by fulfilling a promise, e.g. R\=amachandra broke in two the bow of \'Siva and married S\={\i}t\=a; 3rd the Svayambara, preferring one who will prove the strongest hero by one's own prowess. Of these three kinds of Svayambaras, the king Sub\=ahu preferred Ichch\=a Svayamvara (according to the bride's free choice).

Accordingly the king employed many artisans, had platforms covered with beautiful carpets and big halls decorated beautifully in various ways.

45-47. Thus the assembly hall for Svayambara built and decorated and all the necessary articles and equipments brought thither, the fair eyed \'Sa\'sikal\=a, told her companions with sorrow ``Better go to my mother and say her privately that I have already selected mentally my husband the beautiful Sudar\'sana, the son of the king Dhruvasandhi in my mind; I won't marry any other prince than him; the Goddess Bhagavat\={\i} has settled him for my husband.''

48-50. Vy\=asa said, the companion of \'Sa\'sikal\=a hearing thus, went quickly to her mother Vaidarbh\={\i} and addressed her sweetly in private ``O chaste one! Your daughter, with a sorrowful heart, has sent me to you to say the following; Please hear and do at your earliest convenience, what is good and beneficial.'' She said ``There is staying in the hermitage of Bh\=aradv\=aja, the son of the king Dhruvasandhi; I have mentally selected him as my husband; I won't select any other prince.''

51. Vy\=asa said :-- The queen, hearing her words, told to her husband, when he returned to the palace, all her daughter's words as she had heard them.

52-53. Hearing this, the king Sub\=ahu was astonished and then laughed frequently and then began to say to his wife, the daughter of the king of Vidarbha the following true words :-- ``O fair one! That king's son Sudar\'sana is a minor, he has been exiled to the forest; now he is helpless and is residing with his mother in a dense forest.

54. For his sake, the king V\={\i}rasena was slain in battle by the king Yudh\=ajit. O fair eyed! how can that helpless exiled poor boy become her husband.

55. Do say therefore to \'Sa\'sikal\=a that, in the assembly hall for her Svayamvara, many kings commanding honour and respect would be present. She would then choose whomever she likes. She need not repeat such words any more.''

Thus ends the Eighteenth Chapter of the 3rd Skandha about the Svayambara of \'Sa\'sikal\=a, the daughter of the king K\=asir\=aja in the Mah\=a Pur\=a\d{n}am \'Sr\={\i} Mad Dev\={\i} Bh\=agavatam of 18,000 verses by Mahar\d{s}i Veda Vy\=asa.



