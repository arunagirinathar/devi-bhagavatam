\chapter{On praising the Dev\={\i}}

1-5. Vy\=asa said :-- O King! N\=ar\=ayana, the Lord of Lak\d{s}m\={\i}, and Knower of the essences of all subjects, seeing the Devas extremely attached to him and anxious, spoke to them thus :-- O Suras! Why have you kept silent? Tell me why you have all come, let it be good or bad, tell me; I will try to remove your miseries. The Devas said :-- O Lord! Is there anything unknown to you in this Triloki; You know everything; why then art Thou asking us again and again? In ancient times You in your Dwarf incarnation overspread the three worlds by Your three feet and thus bound the King Vali in his own premises and gave over the sovereignty over the Devas to Indra. O All Pervading One! It is You who deluded the Daityas and procured nectar for the Devas, and it is You who sent them to the house of Death. Therefore, O Lord! You are the one and only one that is capable in warding off all the evils that befall on the Devas.

6-31. Thus hearing the Deva\'s words, Vi\d{s}\d{n}u said :-- ``O Suras! You need not fear; I know one remedy, approved by all, by which that Daitya might be killed so that you would be happy. I am now giving out to you. Your welfare, your benefit must be looked at by me whether by the exercise of my intelligence or by using my prowess, by wealth pretext or by any other means whatsoever. Four means, viz., conciliation, gifts, sowing discord, or punishment are mentioned by the wise statesmen to be applied to friends and specially to the enemies. Brahm\=a was worshipped by Vritra with severe austerities and He granted boons and it is due to the influence of that favour that this Asura has become indomitable. The more so that Vi\'svakarm\=a created him from the sacrificial fire; it is through all these causes that the Demon Vritr\=asura, conqueror of the enemie\'s stronghold, has grown up so very powerful that he can hardly be conquered by any being. O Suras! First peace must be negotiated with him; then deceive him, otherwise the enemy will be very difficult to conquer. First entice him and bring him under control; then kill him. Now take the \d{R}i\d{s}is and Gandharbas with you and go where the powerful enemy Vritr\=asura is residing and make a treaty with him; thus he will be conquered. Swear on oath and accede to the terms he proposes and thus create faith in him; then cultivate friendship with him; lastly, when time will come, kill that powerful enemy. O Suras! I will also enter, unseen by anybody, into the excellent weapon of Indra, his thunderbolt and will help him in due time. Wait till the period of

his longevity expires; otherwise his death will never take place. Now go to that Asura, with Gandharbas and \d{R}i\d{s}is and cunningly cultivate friendship between him and Indra, by conciliatory words; when he begins thus to put his confidence, then deceive him. I will enter hiddenly into the strong well covered thunderbolt. When Indra will come to know that the Demon has put complete faith in him, he will hurl his thunderbolt against him and thus the enemy will be killed and not otherwise. O Lord of the Devas! Do not consider for the present the act of treachery that you will commit: take my help and kill that wicked Demon with thunderbolt. To practise hypocrisy with an hypocrite is not considered a sin; specially no powerful enemy can be killed only by the well known rules applicable to warriors, without any deceit. I also deceived, before, Vali, with my dwarf body and again I deceived all the Demons by showing myself as a beautiful woman; therefore to practise deceit with a strong deceitful enemy is never considered a sin. Know this. O Devas! Now you all conjointly worship the Dev\={\i} Bhagavat\={\i} with Mantras and prayers and take Her shelter; the Yoga M\=ay\=a, then, will help you. We, too, worship that Dev\={\i}, the Highest Prakriti, the Incarnate of pure Sattva Gu\d{n}a, Who grants success, bestows us all our desires, Who is Herself the object of desires, and Who is never realised by any except by those Yogis, self-controlled pure men. Indra, too, will certainly be able to kill his enemy in battle if he worships Her; for the Mah\=a M\=ay\=a, the Creatrix of Delusion, will, when worshipped, delude that Demon. Thus deluded by Her M\=ay\=a, Vritr\=asura will easily be killed by him; there is no doubt in this, what more do you want than this that everything will be successfully accomplished when the Dev\={\i} Ambik\=a is propitiated and gets well pleased. She regulates the hearts of all and is the Cause of all causes. Without Her worship no one's desires can be expected to be accomplished. Therefore, O Best of Suras! Worship the Universal Mother, the Prakriti Dev\={\i} with greatest devotion and with greatest purity for the destruction of your enemy. See! In days of yore, I fought for five thousand years, dreadfully with the two Demons Madhu and Kaitabha and then killed them. I worshipped, then, the Mah\=a M\=ay\=a, the Highest Prakriti; She was thus pleased and deluded the two Asuras; thus the two powerful Asuras puffed up with vanity were deluded and thus I could kill those terrible Daityas under a pretext. Therefore, O Suras! You, too, worship that Highest Prakriti with the greatest devotion; She will then surely fulfill your desires.

32-49. O King! When the intelligent Vi\d{s}\d{n}u enlightened thus the Devas, they went to the top of the Mount Sumeru, adorned with the Mand\=ara

trees, and, remaining at a secluded place, recited slowly Her Mantrams and thus engaged in asceticism and meditation, began to chant hymns and praise that Universal Mother, the Holder of the world, the Remover of all world ailings, and the Creatrix, Preservatrix and Destructrix of the world and the Bestower of all desires to Her devotees. The Devas said :-- ``O Dev\={\i}! Be graciously pleased unto us! O Thou, the Destructrix of the afflictions of the distressed! We have taken refuge unto Thy lotus-feet. We have been defeated by Vritr\=asura in the battle, we are very much oppressed and afflicted. O Thou, the Highest Reality! O Thou, the Mother of the whole Universe! Protect us as a Mother protects her child; we are fallen into this difficulty arising from our enemies. O Mother! Nothing is hidden from Thy knowledge in the three worlds. Why art Thou taking no notice of us, that are being tormented by the Asuras! O Mother! Thou createst, preservest, and destroyest the three worlds; Brahm\=a, Vi\d{s}\d{n}u and Mahe\'sa are created by Thy mere will and are doing all Thy works. Mother! They are not independent; by the contraction of Thy eye-brows, they are directed and enjoy all the pleasures. The Mother protects her sons afflicted with various difficulties and dangers, even when they are found guilty of various offences. It is Thou that hast made this rule; then why, O Merciful! Art Thou not protecting us who are quite innocent and whom Thou dost know as having taken refuge unto Thy lotus-feet. O Dev\={\i}! If Thou thinkest that we forget Thee, being too much attracted by the enjoyments that Thou hast been pleased to confer unto us and therefore we are proper not to be looked upon with Thy merciful eyes, we would say that this is quite true; but, O Mother ! Nowhere is seen a feeling of a Mother to Her child; we are no doubt, objects of Thy mercy and favour always. Besides there is no fault of us in this matter, O Mother! that we do not worship Thee and become immersed in sensual enjoyments; for Thy creation, the Moha (delusion) is very powerful and deludes us. O Mother! Thou art naturally Merciful! Knowing these, why art Thou not showing mercy unto us. O Dev\={\i}! Thou hadst killed before in battle, for our sake, the powerful Daitya Chief Mahi\'s\=asura, very terrible to all the beings. Then why art not Thou, O Mother ! killing this dreadful Vritr\=asura? O Mother! Thou hadst killed the two brother Daityas, \'Sumbha and Ni\'sumbha, extraordinarily powerful, and the other Daityas that followed them; O Thou, the embodiment of mercy! Similarly destroy now this deceitful strong Vritr\=asura. O Mother! Delude this proud Asura so that he could not manifest, in the least, his power. We are very much troubled by the Asuras and overwhelmed with terror from them; Thou savest us; for there is no other in the three worlds that can by his own force remove the sorrows and sufferings of the Devas. O

Mother! Though Thou hast shown favour towards Vritra, now dost kill him soon, whose nature is cruel and tormenting to others. O Bhav\=an\={\i}! Better dost Thou save him from sin by Thy holy arrows. Otherwise that vicious Asura will surely enter into the hideous Hell. It is for his welfare that Thou oughtest to kill him. Those that had been before enemies of the Gods, Thou didst purify them by weapons in the battle-field and hadst sent them to the Nandana Garden in the Heavens. O Thou, the Mercy personified! Was it not that Thou didst not save them from hell? Then why art not Thou killing this Vritr\=asura! We know this for certain that the Asura is Thy enemy, not Thy servant; for that mischievous soul is giving us trouble. O Mother! How can he be Thy servant and devotee who torments the Devas that are always engaged in worshipping Thy lotus feet. O Mother! How can we perform Thy worship? The flowers and other articles used in worship all are created by Thee; especially we and the Mantras, in fact, everything is the manifestation of Thy power. Therefore, O Bhav\=an\={\i}! We worship Thee by laying ourselves prostrate on Thy feet. Be'st Thou pleased. Those men are blessed that worship with devotion Thy lotus feet for crossing this ocean of world. O Dev\={\i}! Those Yogins that want final liberation and forsake therefore all attachments, vik\=aras and delusions, even they attain success then only when they meditate Thy lotus feet. Those that are great Sacrificers and know best the essence of the Vedas, even they when they offer oblations to the sacrifice, utter ``Sv\=ah\=a'' that is cheering to the Devas and ``Svadh\=a'' very consoling to the Pitris; thus they always think of Thee (for Sv\=ah\=a and Svadh\=a are Thy names only). O Mother! Thou art the retentive power and memory. Thou art the beauty, Thou art the peace, Thou art the Buddhi (intellect) well known to clarify men's minds; and Thou art the prosperity and wealth of all these three worlds. O Dev\={\i}! Those that worship Thee, Thou givest them, out of mercy, those wealth in some way or other.

50-57. Vy\=asa said :-- O King! Thus worshipped by the Devas, the Dev\={\i} Bhagavat\={\i} appeared before them in a very beautiful form, thin, adorned with all ornaments. Her two hands holding a noose, and goad, and the other two hands making signs to discard all fear and ready to grant boons; Her loins very beautiful, girdled with a gold band with small bells pending and making sweet tinkling sounds; Her feet with anklets (ornaments) making sweet sonorous sounds with tiny tinkling bells. Her voice was exceedingly sweet and lovely, Her forehead was adorned with the crescent of the Moon and on Her head was glittering a diadem of jewels, Her lotus-face adorned with sweet soft smiles and with Her three beauteous lotus-eyes looking like Ind\={\i}baras. Her

body was of a red colour like the P\=arij\=ata flowers and Her limbs were marked with red sandal-paste. She was dressed in a red attire. The Dev\={\i} looked well pleased, like an ocean of infinite mercy, wearing complete dress suited to happy interviews, the Creatrix of all this Cosmos, the Highest, the Knower of all, the Directrix of all, and the Great Upholder of all. She looked like an embodiment of the Truth of all Ved\=antas and the Incarnate of ever Existence, Intelligence, Bliss, the Mah\=a Dev\={\i} Bhagavat\={\i} Bhuvane\'svar\={\i}. The Devas all bowed down before Her standing in front of them. The Mother then spoke :-- ``What business have you got here? Speak to Me.''

58-59. The Devas said :-- ``O Bhagavat\={\i}! Vritr\=asura is tormenting much the Devas; Bewitch him. O Dev\={\i}! Do such as he can trust the Devas; and impart then strength on our weapons such as he can be killed.'' Vy\=asa said :-- ``O King! That will be done.'' Saying thus, the Dev\={\i} departed then and there. The Devas became very glad and returned respectively to their abodes.

Here ends the Fifth Chapter of the Sixth Book on the praising of the Dev\={\i} by the Devas in \'Sr\={\i} Mad Dev\={\i} Bh\=agavatam of 18,000 verses by Mahar\d{s}i Veda Vy\=asa.



