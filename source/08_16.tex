\chapter{On the motion of the planets}

\'Sr\={\i} N\=ar\=aya\d{n}a said :-- O N\=arada! Now hear the wonderful movements of the planets and their positions. The auspicious and inauspicious events of the mankind, are due to the different movements of these planets. As in a potter's wheel going round and round, the motion of the insects crawling on the wheel, appears in a contrary direction, so the motion of the Sun and other planets moving on the Zodiac composed of the R\=as\={\i}s (12 constellations) which again always moves round the Meru as an axle, appears different. Their motion from one star to

another and from one constellation to another appears so likewise. These two motions therefore are not contradictory but are consistent; so it is settled everywhere by the learned Pundits (as being subservient to the Zodiac). O N\=arada! He, Who is the Origin of all, Who is the \=Adi Puru\d{s}a, from Whom all these have sprung, Who is endowed with six extraordinary powers, in Whom all this Prapa\~ncha, this material world composed of the five elements remains, that N\=ar\=aya\d{n}a, roaming about, has divided the Tray\={\i} \=Atm\=a into twelve parts for the perfect happiness of all and for Karma \'Suddhis (the purification of Karma, acts). The sages furnished with J\~n\=an and Vij\~n\=ana have thus argued on the point, following the path as laid out in the Vedas. The S\=urya N\=ar\=aya\d{n}a, moving on in the six seasons, spring, etc., has established, cold, heat, etc., as the Dharma of the seasons, duly for the fructification of the Karmas of the individual beings. Those persons that worship this \=Adipuru\d{s}a, with devotion, according to the knowledge of the Vedas the customs and usages of Var\d{n}a (castes) and \=A\'srama (Brahmacharya, etc.), and with various performances of Yogas, get their fruits respectively according to their desires. This Sun is the \=Atman of all the Lokas and resting on the Zodiac between the Heavens and the Earth, enjoys the twelve months in the twelve constellations, Aries, etc. These months are the limbs of the year. Two fortnights make one month. The 21 asterisms go to form one month according to the Solar measure, of the day and night.

The period that the Sun takes to travel over the two constellations is called Ritu or the Season (i.e., two months). The Scientists say that this season is the limb of one Samvatsara. The path that the Sun describes within the three seasons or half the year in the Zodiac is called one Ayanam. The time taken by the Sun with earth and heavens to make a circuit of the Zodiac is called one Vatsara or year. This year is reckoned into five divisions as :-- Samvatsara, Parivatsara, Id\=avatsara, A\d{n}uvatsara, and Idvatsara. These are functioned by the \'S\={\i}ghra, Manda, and uniform motions of the Sun. So the Munis say. Thus far the motion of the Sun has been described. Now hear that of the Moon. The Moon is situated one lakh Yoyanas higher than the Sun and shares with the motion of the Sun for one year; and She enjoys as well every month with the Sun in the shape of the dark and bright fortnights. The Moon, the Lord of Night and of the medicinal plants also enjoys the day and night by the help of one constellation or 2\sfrac{1}{4} Nak\d{s}attras. Thus, by Her \'S\={\i}ghragati, the Moon enjoys the Nak\d{s}attras. During the bright fortnight, the Moon becomes more and more visible and gives pleasure to the Immortals by Her increasing phases; and, during the

dark fortnight by Her waning phases, She delights the Pitris. She performs revolution in the day and night by Her both the phases of the bright and dark fortnights. Thus She becomes the Life and Soul of
all the living beings. The Moon, endowed with the highest prosperity, travels one Nak\d{s}attra in thirty Muh\=urtas. She is Full and the Soul without any beginning. She fructifies the desires (Sankalpas) and
resolves of all; hence She is called Manomaya. She is the Lord of all the medicinal plants (O\d{s}adhis); hence She is called Annamaya. She is filled with nectar; hence She is called the Abode of Immortality and She gives Nirv\=a\d{n}a (the final liberation) to all. Hence She is called Sudh\=akara. She nourishes and satisfies the Devas, Pitris, men, reptiles and trees; hence She is called ``Sarvamaya.'' By Her influence the asterisms travel over the three lakh Yoyanas. The God Himself has made the Nak\d{s}attra Abhijit to revolve round the Meru, along with the other Nak\d{s}attras in the Zodiac; so this is reckoned as the twenty-eighth Nak\d{s}attra. The planet Venus (\'Sukra) is situated above the Moon two lakh Yoyanas high. He sometimes goes before the Sun, sometimes behind and sometimes along with Him. He is very powerful. His motion is of three kinds :-- (1) \'S\={\i}ghra, (2) Manda, and (3) uniform. He is generally favourable to all the persons and does for them many auspicious things. So it is stated in the \'S\=astras. O Muni! \'Sukra, the illustrious scion of Bhrigu, removes the obstacles to the rains. Next to \'Sukra, the planet Mercury (Budha) is situated two lakh Yoyanas high. Like \'Sukra, he, too, goes sometimes in front sometimes behind and sometimes along with the Sun. And his motion too, is of three kinds :-- \'S\={\i}ghra, Manda, and uniform. When Mercury the Son of Moon, is away from the Sun, then Ativ\=ata (strong winds, hurricanes), Abhrap\=ata (the falling of meteors from the clouds) and draught and other fears arise. The planet Mars, the son of the Earth is situated two lakh Yoyanas higher. Within three fortnights (45 days) he travels one R\=a\'s\={\i}. This occurs when his motion is not retrograde. This Mars causes all sorts of mischief, evils, and miseries to mankind. The planet Jupiter is situated two lakh Yoyanas higher. He passes through one R\=a\'s\={\i} in one year. When his motion is not retrograde, he is always in favour with the Brahm\=a V\=adis. Next to Brihaspati, come the planet Saturn, the son of the Sun, two lakh Yoyanas higher. He takes thirty months to pass over one R\=a\'s\={\i}. This planet causes all sorts of unrest and miseries to all. Therefore He is called a Manda Graha (a malefic planet). Next to it, is situated the Saptar\d{s}i mandala, the Great Bear, eleven lakh Yoyanas higher up. O Muni! The seven planets always do special favours to all. These circumambulate the Vi\d{s}\d{n}upada, the Polar Star.

Here ends the Sixteenth Chapter in the Eighth Book on the motion of the planets in the Mah\=a Pur\=a\d{n}am \'Sr\={\i} Mad Dev\={\i} Bh\=agavatam of 18,000 verses by Mahar\d{s}i Veda Vy\=asa.



