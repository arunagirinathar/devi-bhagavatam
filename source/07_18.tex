\chapter{On the origin of the quarrel between Hari\'schandra and Vi\'sv\=amitra}

1-6. Vy\=asa said :-- O King! Once on a time Hari\'schandra went out to the forest on an hunting excursion; and, while roaming to and fro, he saw that a very beautiful lady was crying. The King, seeing this, took pity on her and asked :-- ``O Fair One! Why are you in this forest crying alone? O Large eyed One! Has someone pained you? What is the cause of your sorrow? Express this quickly before me. Why have you come here in this dreadful lonely forest? What are the names of your husband and your father? O Beautiful One! In My kingdom, no demon can give any trouble to another's lady; I will immediately kill him who has given you this trouble. O thin-bellied One! Be comfortable; do not weep; tell me why you are in this sorrowful state; know that no sinner can remain within my territory.'' Hearing the words of the King, the lady wiped out her tears by her hand and began to say :--

7-8. O King! I am Siddharupi\d{n}\={\i}, of the nature of success; to get me, Vi\'sv\=amitra is practising terrible austerities. So these troubles have arisen from him, the son of Kau\'sika. O King! For this reason I am sorry in Your kingdom. O One of good vows! I am a gentle lovely Lady; still that Muni is giving me so much trouble.

9-16. The King said :-- ``O Large-eyed One! No longer you will have to suffer any more pains. Be patient. I will go and make the Muni desist from his tapasy\=a.'' Thus comforting the lady, the King went hurriedly to the Muni Vi\'sv\=amitra and, bowing down to him said with clasped palms :-- O Mahar\d{s}i! Why are you ailing your body by this terrible severe austerity! O Highly intelligent One! For what great noble cause are you practising this hard tapasy\=a; speak truly to me. O Son of G\=adhi! I will fulfil your desires; there is no need of your practising this severe penance; please get out of it immediately. O Mahar\d{s}i! You know everything; so what shall I say anything further? See! It ought not anyone to practise this extremely dreadful tapasy\=a, causing troubles to the people within my territory. Thus prohibited by the King Hari\'schandra, the Muni became very angry at his heart and went towards his own hermitage. The King, too, went back to his palace. The Muni on his arrival at his hermitage, began to cogitate in his mind, ``Why has the King unjustly desisted me from

my tapasy\=a'' and also the discussions that took place between him and Va\'sistha. Vi\'sv\=amitra became very angry at his heart and ready to take the vengeance of this. He thought over on many points and created a terrible demon of a dreadful appearance in the form of a boar and sent it to the territory of the King Hari\'schandra.

17-28. That terrible boar, of huge body, entered into the kingdom, raising a dreadful sound. The guards became afraid at his terrible noise. Entering into the forest, that boar began to whirl round and round and destroy the M\=alati forest, at another place the Kadamba forest, and at others the Y\=uthik\=a forest. At other places he began to dig up the earth by his tusks and root out the Champaka, Ketak\={\i}, Mallik\=a and various other trees. At other places again, he rooted out nice gentle U\'s\={\i}ra, Karav\={\i}ra, Muchukunda, A\'soka, Vakula, Tilaka and other trees and so massacred the nice gardens and forests. The forest guards, then, taking their weapons, rushed forward on that boar. Those that were making garlands and the florists became very distressed and uttered uproars of consternation. That boar, as if an incarnate of Death, though routed out with flights of arrows, could not be terrified; rather when he began to harass the guards very much, they became very much afraid and being very distressed took the refuge of the King and, trembling, said :-- ``O King! Protect us. Protect us.'' And they cried piteously. Seeing the guards terrified and distressed, the King asked them :-- Whom do you fear so much and why you are so distressed? Speak truly before me. O Guards! I do not fear the Deva nor the Demons; so tell me who has created this panic amongst you. I, no doubt, will send that vicious cheat unto the door of Death by this arrow, who has come against me in this world. What sort of enemy is that? What is his form? What is his power and where is he residing now; speak this quickly to me. Be that enemy a Deva or a D\=anava, I will slay him immediately by the multitude of arrows.

29-31. The M\=al\=ak\=aras said :-- The enemy is not a Deva, nor a D\=anava, Yak\d{s}a nor a Kinnara; it is a boar of a huge body that has entered into the forest. Very powerful, he is uprooting by his teeth all the beautiful flower trees; in fact, he is ruining all the gardens and forests. O King! We shot arrows on him, struck him with cudgels and hurled stones at him so much; yet he did not get a bit afraid; rather he turned back to kill us.

32-51. Vy\=asa said :-- O King! Hearing these words, the King's fury knew no bounds and, immediately getting on horseback, he went towards the garden and forest. Then the horsemen, elephant drivers, charioteers and infantry, all followed him. When the King went there,

he saw the terrible boar, of a huge body, whirling round and round and making the peculiar sound in the forest; and he witnessed also the destroyed condition of the forest and became very angry. He then drew his bow and arrows and fell down on him to take away his life. Seeing the King coming angrily towards him with bow and arrows in his hands, the boar began to sound more terribly and ran forward before him. The King saw the boar coming towards him with his mouth wide opened and began to shower arrows upon him to kill him. The boar immediately made those arrows useless, and very violently and quickly jumped and passed away, over the King. When the boar passed away, the King angrily drew his bow with great care and shot sharpened arrows at him. One moment the boar came in the King's sight; and at another moment he vanished away; thus the boar began to flee, uttering all sorts of sounds. The King Hari\'schandra then became very angry and drawing his bow pursued him, mounting on a horse, swift like the wind. The soldiers then entered the forest and scattered hither and thither; the King alone pursued the boar. The sun entered unto the meridian; and the King came to be alone in a lonely forest. His horse was fatigued, and he, too, was tired of hunger and thirst. The boar went away out of sight. The King also missed his way in that dense jungle and became greatly absorbed with intense cares and anxieties. He then began to think, ``Where shall I now go? There is none to help me in this dense jungle. Especially I don't know the right path.'' While he was thus thinking, he saw, all on a sudden, a river with clear water in that lonely forest. He became much delighted to see the flowing river and, alighting from horseback, he drank that water and made the horse also drink it. He became much relieved by drinking; and though he was much bewildered not to find the right track, he wanted now to go to his own city. At this moment Vi\'sv\=amitra came up there in an old Br\=ahmi\d{n} form; the King also looking at him bowed down to the Br\=ahmi\d{n} garbed Vi\'sv\=amitra, who then spoke to the King :-- ``O King! Welfare be unto you! What for have you come here? O King! What object have you got in view in this lonely forest? Be calm and quiet and speak everything before me.''

52-58. The King said :-- ``O Br\=ahmi\d{n}! One powerful boar of a huge body entered into my garden and spoilt altogether all the gentle flower trees there. To desist that boar, I pursued him with bow in hand and went out of the city. That powerful boar, very swift and, as it were, a magician, has escaped my sight and gone away where I do not know. I pursued him and have come now to this place and I do not know where my soldiers have gone. O Muni! Now I am deprived of my men,

I am hungry and thirsty. I do not know which is the road to my city; nor do I know where my soldiers have gone. O Dear Lord! It is to my great fortune that you have come in this lonely forest. Now I want to return to my home; kindly shew me the way. I have completed my R\=ajas\=uya sacrifice. I always give everyone whatever he wants. This is known to everybody. O Dv\={\i}ja! If you want money for your sacrifice, then come with me to Ayodhy\=a and I will give you abundance of wealth. I am Hari\'schandra, the famous King of Ayodhy\=a.''

Here ends the Eighteenth Chapter of the Seventh Book on the origin of the quarrel between Hari\'schandra and Vi\'sv\=amitra in the Mah\=apur\=a\d{n}am \'Sr\={\i} Mad Dev\={\i} Bh\=agavatam, of 18,000 verses, by Mahar\d{s}i Veda Vy\=asa.



