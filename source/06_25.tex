\chapter{On the cause of Moha of Vy\=asa Deva asked before N\=arada}

1-10. Vy\=asa said :-- OKing! The mother became astonished to hear me. Becoming very anxious for a son, she began to speak to me. O Child! The wife of your brother, the daughter Amb\=alik\=a of K\=a\'s\={\i}r\=aj, is a widow; she is very sorrowful; she is endowed with all auspicious signs and endowed with all good qualities; better cohabit with that beautiful young wife and get a child according to the tradition of the \'Sistas. Persons born blind are not entitled to kingdoms. Therefore take my word and procreate a beautiful son and thus keep my honour. O Muni! Hearing the mother's words, I began to wait in Hastin\=apura till Amb\=alik\=a, the daughter of K\=a\'s\={\i}r\=aj, finished her ablutions after menstruation. That King's daughter, of curling hairs, came to me alone at her mother-in-law's order, and became very much abashed. Seeing me an ascetic with matted hairs on my head and void of every love sentiment, perspiration came on her face; her body turned pale and her

mind void of any love towards me. When I saw that lady trembling and pale beside me, I angrily spoke :-- ``O One of beautiful waist! When you have turned out pale, considering your own beauty, let your son be of a pale colour.'' Thus saying I spent there that night with Amb\=alik\=a. After enjoying her I took farewell from my mother and went to my place.

11-21. In due course, the two daughters of the King gave birth to two sons respectively, one blind and the other pale. The son of Ambik\=a was named Dh\d{r}tar\=a\d{s}\d{t}ra; and the son of Amb\=alik\=a was named P\=a\d{n}\d{d}u, as his colour was p\=andu (pale). Mother became absent-minded when she saw the two sons in those states. After one year she again called me and said :-- ``O Dvaip\=ayana! These two persons are not so fit to become kings; therefore beget one more son beautiful and according to my liking.'' When I consented, she became very glad and, in due course, asked Ambik\=a to embrace me and give birth to a son, endowed with extraordinary qualities, and fit to preserve the line worthy of the Kuru dynasty. The bride did not then say anything on account of her bashfulness. But when I went in the night time according to my mother's order, to the sleeping room, Ambik\=a sent to me a maid-servant of Vichitrav\={\i}rya, full of youth and beauty, and adorned with various ornaments and clothings. That maid-servant of beautiful hairs and of a swan-like gait adorned with garlands and red sandal-paste, came to me with many enchanting gestures and making me take my seat on the cot, became herself merged in love sentiments. O Muni! I became pleased with her gestures and amorous sports and passed the night, full of love towards her and played and cohabited with her. At last I gladly gave her the boon, ``O Fortunate One! Your child, begotten by me, will be endowed with all good qualities, will be of good form, will be conversant with all the essences of Dharma, calm and quiet and truthful.''

22-34. In due course, a child named Vidura was born to her. Thus I had three sons; and in my mind grew up M\=ay\=a and affection that these were my sons. When I saw again those three sons, heroic and full of manliness, the only cause of my sorrow due to the bereavement of my son \'Suka vanished away from my mind. O Lord of Dv\={\i}jas! M\=ay\=a is very powerful and extremely hard to be abandoned by those who are not masters of their senses; She enchants even the wise, though She does not possess any form nor any substratum nor any support. I could not find any peace, even in the forest, as my mind was attached to my mother and children. O Muni! My mind then began to oscillate like a pendulum and I remained sometime in Hastin\=apura and sometime on the

banks of the Sarasvat\={\i}. I could not stay in a certain fixed place. By discrimination, the knowledge sometimes flashed in my mind :-- Whose sons are these? The attachment is nothing but merely a delusion. On my death they would not be entitled to perform my \'Sr\=addha ceremony. These sons are begotten by ways and manners not sanctioned by Dharma; what happiness can they bring to me? O Muni! The powerful M\=ay\=a has caused this delusion in me. What! Knowing this Sams\=ara to be unreal, Alas! I have fallen into this well of the Darkness of delusion. Thus I repented when I thought over the matter deeply and when I was alone in a solitary place. When, subsequently, through the mediation of Bh\={\i}\d{s}ma, the powerful P\=a\d{n}\d{d}u got the kingdom, I became pleased to see the prosperity of my son. O Muni! This is also the creation of M\=ay\=a. The daughter of the King \'S\=urasena, named Kunt\={\i}, and the daughter of the King of Madra, named M\=adr\={\i} became the two beautiful wives of P\=a\d{n}\d{d}u. P\=a\d{n}\d{d}u was cursed by a Br\=ahma\d{n}a that he would die if he cohabited with any woman; he therefore became dispassionate and quitting his kingdom, went to the forest with his two wives. Hearing P\=a\d{n}\d{d}u gone to the forest I felt pain and went to my son who was staying with his wives and consoling him, came to Hastin\=apura, where I held a conversation with Dh\d{r}tar\=a\d{s}\d{t}ra and then came back to the banks of the river Sarasvat\={\i}.

35-50. P\=a\d{n}\d{d}u in his forest life, got five sons out of his wives by the Devas Dharma, V\=ayu, Indra, and the twin A\'svins. Dharma, V\=ayu, and Indra begat respectively of Kunt\={\i} the three sons Yudhisthira, Bh\={\i}masena and Arju\d{n}a; and the two A\'svins begat of M\=adr\={\i} the two sons Nakulu and Sahadeva. Once M\=adr\={\i}, full of youth and beauty, was staying alone in a solitary place and P\=a\d{n}\d{d}u seeing her embraced her and due to the curse, died. When the funeral pyre was ablaze, the chaste M\=adr\={\i} entered into the fire and died a Sat\={\i}. Kunt\={\i} was prevented from doing so, as she was to nurse and look after her young children. The Munis then took the sorrowful Kunt\={\i}, the daughter of \'S\=urasena, bereft of her husband to Hastin\=apur and handed her over to the high-souled Bh\={\i}\d{s}ma and Vidura. When I came to hear this, my mind was greatly agitated to see the pain and pleasure that other people suffered. Bh\={\i}\d{s}ma, Vidura, and Dh\d{r}tar\=a\d{s}\d{t}ra began to nourish and support Yudhisthira and others as they considered them the sons of their dearest P\=a\d{n}\d{d}u. The cruel and wicked sons of Dh\d{r}tar\=a\d{s}\d{t}ra, Duryodhana and others united with each other and began to quarrel horribly with the sons of P\=a\d{n}\d{d}u. Dro\d{n}\=ach\=arya came there accidentally and Bh\={\i}\d{s}ma treated him with great respect and requested him to stay in Hastin\=apura and educate the sons of Kuru. Kar\d{n}a was the the son of

Kunt\={\i}, when she was young and unmarried; and he was quitted by her no sooner he was born. The charioteer S\=uta (or carpenter) Adhiratha found him in a river and nourished him. Kar\d{n}a was the foremost of the heroes and therefore the great favourite of Duryodhana. The enmity between Bh\={\i}ma and Duryodhana, etc., began to grow greater day by day. Dh\d{r}tar\=a\d{s}\d{t}ra, thinking the difficult situation of his children, fixed the residence of the sons of P\=a\d{n}\d{d}u at the V\=ara\d{n}\=avata city so that the quarrels might die away. Out of enmity, Duryodhana ordered his dear friend Purochana to build there a house of lac for the P\=andavas. O Muni! When I heard that Kunt\={\i} and her five sons were burnt in the lac-house, I became merged in the ocean of sorrows and thought that they were my grandsons. I was overwhelmed with sorrow and began to search after them in deep forests day and night till at last I found them in Ekachakr\=a city, lean and thin and very much distressed with sorrow.

51-63. I became very glad to see them and sent them soon to the city of the King Drupada. Wearing the deer's skin, they went there dejected with sorrow in the Br\=ahmi\d{n}'s dress and stayed in the royal court. The victorious Arju\d{n}a shewed prowess and pierced the mark (the eye of the fish) and obtained Kri\d{s}\d{n}\=a, the daughter of the King Drupada. By the order of the mother Kunt\={\i}, the five brothers married her. O Muni! I became very glad to see that they were all married. The P\=andavas, then, accompanied by P\=anch\=al\={\i}, soon went to Hastin\=apura. Dh\d{r}tar\=a\d{s}\d{t}ra then fixed Kh\=andavaprastha as the residence of the P\=andavas. Vi\d{s}\d{n}u, the son of V\=asudeva, then performed the Yaj\~n\=a with the victorious Arju\d{n}a and satisfied the Great Fire. The P\=andavas next performed the R\=ajas\=uya sacrifice and that made me very glad. Seeing the affluence and prosperity of the P\=andavas and the great assembly hall beautiful and exquisitely artistic, Duryodhana was burnt up, as it were, with malice and made arrangements for play in dice, very injurious in its consequences. \'Sakuni was expert in playing deceitfully and Yudhisthira the son of Dharma, was not expert in this play. So Duryodhana made \'Sakuni play for him and stole away all that Yudhisthira had and insulted, at last, in the royal assembly, the daughter of Drupada, Yaj\~n\=asen\={\i} and gave her much trouble. The P\=andavas then went with P\=anch\=al\={\i} in an exile in the forest for twelve years. And I was very much grieved to hear this O Muni! Though I know all about the San\=atan Dharma, yet I was deluded and merged in these worlds of pains and pleasures. Who am I? To whom do these sons belong? My mind roams day and night on the thought of all these. O Muni! What shall I do? And whither shall I go? I don't find happiness anywhere; my mind is, as it were, floating in a

rocking machine and it is never being fixed. O Best of Munis! You are all-knowing; solve my doubts so that my mental fever may be quietened and I may be happy.

Here ends the Twenty-fifth Chapter on the cause of Moha of Vy\=asa Deva asked before N\=arada in \'Sr\={\i} Mad Devi Bh\=agavatam of 18,000 verses by Mahar\d{s}i Veda Vy\=asa.



