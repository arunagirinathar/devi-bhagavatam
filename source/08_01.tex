\chapter{On the description of the worlds}

1-4. Janamejaya asked :-- ``O Lord! I have heard all that you have described about the sweet nectar-like characters of the Kings of the Solar and the Lunar dynasties. Now kindly describe the real Tattva of the Vir\=at Form of the Great Dev\={\i} and how She was worshipped in every Manvantara by the Regent of that Manvantara and the Kings thereof. In what part of the year and in which place, under what circumstances and in what form and with what Mantras was the Dev\={\i} worshipped? I am very anxious to hear all this. O Guru! In fact describe the gross forms of the \=Ady\=a \'Sakti, the Dev\={\i} Bhagavat\={\i} by concentrating attention to Which, I can have the power to understand the subtle forms of the Dev\={\i} and I can get the highest good in this world.''

5-7. Vy\=asa said :-- O King! Now hear. I am describing to you in detail about the worship of the Dev\={\i} Bhagavat\={\i} that leads to the welfare of the Whole World; the hearing of which or the practice of which enables one to get the highest good. In days of yore, the Devar\d{s}i N\=arada asked N\=ar\=aya\d{n}a about this very point; I will now tell you what the Bhagav\=an, the Promulgator of the Yoga Tattva, advised N\=arada. Once on a time the all powerful Devar\d{s}i N\=arada entitled with all the Yogic powers, and born from the body of Brahm\=a was travelling all over this earth and came to the hermitage of the \d{R}i\d{s}i N\=ar\=aya\d{n}a. Resting a while, and the troubles of the journey over, he bowed down to the Yogi N\=ar\=aya\d{n}a and asked Him what you ask me now. N\=arada said :-- O Deva Deva Mah\=adeva! O Thou, the Ancient Puru\d{s}a, the Excellent One!

8-9. O Omniscient ! O Thou, the Holder of the Universe! O Thou Who art the repository of the good qualities and Who art praised by all!

10-12. O Deva! Now tell me what is the ultimate cause of this Universe: whence has this Universe its origin? And how does it rest? To whom does it take refuge? Where does it dissolve in the time of Pralaya? Where do all the Karmas of these beings go to? And what Object is that whose knowledge destroys forever the M\=ay\=a, the Cause

of all this Moha (illusion)? Whose worship, what Japam, and Whose meditation in the lotus of heart are to he made, by which, O Deva! the knowledge of Param\=atman rises in the heart, as the darkness of the night vanishes by the rising of the Sun.

13. O Deva! Kindly reply to these my questions in such a clear manner as the ignorant people in this Sams\=ara can understand and get themselves across this ocean of Sams\=ara.

14-15. Vy\=asa said :-- Thus asked by the Devar\d{s}i, the ancient N\=ar\=aya\d{n}a, the Best of the Munis, the great Yogi gladly spoke :-- O Devar\d{s}i! Hear I will now speak to you all the Tattvas of this world, knowing which the mortal never falls into the illusion of this world.

16. O Child! The original cause of this Universe is the Dev\={\i} Mah\=a M\=ay\=a (the image of the Supreme Chaitanya Para Brahm\=a); this is the opinion of the \d{R}i\d{s}is, the Devas, Gandharvas, and other intelligent persons.

17-23. It is written in the Vedas and other \'S\=astras that the Dev\={\i} Bhagavat\={\i}, worshipped by all in the Universe, creates, preserves and destroys the Universe by the influence of Her three Gu\d{n}as. I now describe to you the nature of the Dev\={\i}, worshipped by the Siddhas, Gandharbas and \d{R}i\d{s}is, the mere remembering of Whom destroys all sins, and gives final liberation Mok\d{s}a (and Dharma, Artha, and Kama also). The powerful Sv\=ayambhuva Manu, the First, the husband of \'Satar\=up\=a, the prosperous and the Ruler of all the Manvantaras worshipped the sinless Praj\=apati Brahm\=a, his Father with due devotion and satisfied Him when the Grandsire of the Lokas, the Hira\d{n}yagarbha spoke to his son :-- The excellent worship of the Dev\={\i} should be done by you. By Her Grace, O Son, your work of creating worlds will be successful. Thus spoken by Brahm\=a, the Bibhu Sv\=ayambhuva Manu, the Vir\=at incarnate, worshipped the World Mother with great austerities. And with his concentrated devotion, he satisfied the Dev\={\i} Deve\'s\={\i} and began to chant hymns to Her, the First-born, the M\=ay\=a, the \'Sakti of all, and the Cause of all causes.

24-36. Manu said :-- Thou art Brahm\=a, the ocean of the Vedas, Kri\d{s}\d{n}a, the abode of Lak\d{s}m\={\i}, Purandara. I bow down again and again to Thee, the Deve\'s\={\i}, the Cause of M\=ay\=a, the Cause of this Universe. Thou holdest \'sankha (the conchshell), chakra, gad\=a, etc., in Thy hands and Thou residest in the heart of N\=ar\=aya\d{n}a; Thou art the Vedas incarnate, the World Mother, the Auspicious One, bowed down by all the Devas, and the Knower of the Three Vedas. O Thou, endowed with all

powers and glory! O Mah\=am\=aye! Mah\=abh\=age! Mahodaye! (the Self-manifested). Thou residest as the better half of Mah\=a Deva, and Thou dost all what are dear to Him. Thou art the most beloved of Nanda, the Cow-herd (in the form of Mah\=a M\=ay\=a, the daughter who concealed Kri\d{s}\d{n}a and slipped from the hands of Kamsa and got up in the air and remained as Vindhy\=av\=asin\={\i}; also in the form of \'Sr\={\i} Kri\d{s}\d{n}a). Thou gavest much pleasure and wert the cause of all the festivities; Thou takest away the fear due to plague, etc.; Thou art worshipped by the Devas. O Thou, the auspicious Bhagavat\={\i}! Thou art the welfare of all incarnate; Thou fructifiest the desires of all to success! Thou art the One to Whom all take refuge and Thou removest their all the dangers; O Thou the three-eyed! Gaur\={\i}! N\=ar\=ayan\={\i}! Obeisance to Thee. I bow down to that ocean of all brightness and splendour, without beginning or end, the One Consciousness, wherein this endless Universe rises and remains interwoven therein. I bow down to the Dev\={\i}, whose Gracious Glance enables Brahm\=a, Vi\d{s}\d{n}u, and Mahe\'svara to do their respective works of creating, preserving, and destroying the Universe. O Dev\={\i}! Thou art the Only One, whom all can bow, since the lotus-born Brahm\=a, terrified by the horrible Daityas, was freed by Thy prowess only. O Bhagavat\={\i}! Thou art modesty, fame, memory, lustre; Thou art Lak\d{s}m\={\i}, Girij\=a, the daughter of Him\=alay\=a, Thou art Sat\={\i}, the Dak\d{s}a's daughter; Thou art the S\=avitr\={\i} the Mother of the Vedas, Thou art the intelligence of all and Thou art the cause of fearlessness. So I now engage myself reciting Thy Japam, Thy hymns and Thy worship. I meditate on Thee and see Thy form within my heart and hear Thy praises. Be graciously pleased on me, O Dev\={\i}! It is by Thy Grace that Brahm\=a is the Revealer of the Four Vedas, Vi\d{s}\d{n}u is the Lord of Lak\d{s}m\={\i}, Indra is the Lord of the Devas and of the three worlds; Varu\d{n}a is the Lord of waters, Kuvera is the Lord of wealth, Yama is the Lord of the dead, Nairrita is the Lord of the R\=ak\d{s}asas, and Soma is the Lord of the water element and praised by the three worlds. Therefore, O Auspicious World Mother! I bow down again and again to Thee.

37-45. N\=ar\=aya\d{n}a said :-- O Child! When Svayambhuva Manu, the son of Brahm\=a, chanted thus the hymns to the \=Ady\=a \'Sakti Bhagavat\={\i} N\=ar\=ayan\={\i}, She became pleased and spoke to him thus :-- The Dev\={\i} said :-- ``O King, the Brahm\=a's son! I am pleased with your devoted worship and hymns; so ask boon from Me that you desire.'' Manu said :-- ``O Dev\={\i}! If Thou art graciously pleased, grant that my creation be finished without any hitch.'' The Dev\={\i} said :-- ``O King of Kings! By My blessing, your work of creation will be completed without any obstruction. And by your pu\d{n}ya (merits) they will no doubt multiply on and on. He who reads

with devotion this hymn (stotra) composed by you, will get sons, fame and beauty in the world and, in the end, he will be entitled to get the Highest Place. The people will have powers unopposed by anybody, will get wealth and grains, will get victory everywhere and happiness; and his enemies will be ruined.'' N\=ar\=aya\d{n}a said :-- ``O Child! The Dev\={\i} Bhagavat\={\i} \=Ady\=a \'Sakti granted thus the desired boon to Sv\=ayambhuva Manu and vanished away at once from his sight.'' Then the powerful Manu, obtaining thus the boon, spoke to his father :-- O Father! Now give me a solitary place where I can worship the Dev\={\i} with sacrifices and do my work of creating a good number of people.

46-48. Hearing thus the words of the son, the Praj\=apati, the Lord thought over the matter for a long time :-- ``How this work would be done? Alas! I have spent an endless time in this work of creation; but as yet nothing has been done. For the Earth, the receptacle of all the Jiv\=as is submerged in water and has gone down to the Ras\=atala. What is to be done now? There is only one hope and that is this :-- If the Bhagav\=an, the Primeval Person, under Whose Command I am engaged in this work of creation, helps me in this work of mine, no doubt it will be accomplished then and then only.''

Here ends the First Chapter of the Eighth Book on the description of the worlds in the Mah\=apur\=a\d{n}am, \'Sr\={\i} Mad Dev\={\i} Bh\=agavatam, of 18,000 verses, by Mahar\d{s}i Veda Vy\=asa.



