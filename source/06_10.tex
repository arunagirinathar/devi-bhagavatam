\chapter{On the phase of Karma}

1-5. Janamejaya said :-- ``O Br\=ahma\d{n}a! You have described in detail the wonderful character of Indra, his displacement from his Heavens, and his suffering many hardships and at the same time, you have described very widely the greatness of the Highest Goddess of the world. But one doubt has arisen in my mind that Indra was very powerful and when he got the lordship over the Devas, which means in other words that no trouble would pain him, how was it that he had to feel pain and agony? He got the Lordship of the Devas and his highest position by performing one hundred Horse Sacrifices; how was it, then that he was again displaced from that position? O Ocean of mercy! Kindly explain to me the causes of all these. You know everything; you are the best of the Munis and the maker of the Pur\=a\d{n}as; I am your devoted disciple; therefore nothing there can be that cannot be mentioned to me. So, O highly fortunate One! Kindly remove my doubts.'' S\=uta said :-- Thus asked by Janamejaya, Vy\=asa the son of Satyavat\={\i} gladly spoke, in due order, the following words :--

6-29. Vy\=asa said :-- O King! Hear, then the causes that are certainly very wonderful. The seers say that Karma is of three kinds :-- Sa\~nchita (accumulated), Vartam\=ana (present) and Pr\=arabdha (commenced). Each of these is again subdivided into three, S\=attvik, R\=ajasik, and T\=amasik. The accumulated effects of Karmas done in many past lives is called Sa\~nchita, O King! The effects of this S\=anchita Karma, be it auspicious or inauspicious, be it for a long or for a short time, must have to be enjoyed by the beings whether they be good or bad. This Sa\~nchita Karma done by the embodied beings in several previous births, can never be totally exhausted even in hundred Koti Kalpas without their being enjoyed. The Karma that is being done by a J\={\i}va and that has not as yet been completed, that is called Vartam\=ana Karma. The J\={\i}vas do this Vartam\=ana Karma, auspicious or inauspicious, in their present embodiments. At the time of birth, a part of the Sa\~nchita Karma, the soul takes up for fructification. This part of Sa\~nchita Karma is called Pr\=arabdha Karma. This exhausts only when its effects have been fully borne out by the embodied soul. The beings cannot but bear the effects of this Pr\=arabdha Karma. O King! Know this for certain that the effects of merits or demerits done previously must be borne by anybody, be he a Deva, or a man, or an Asura, or a Yak\d{s}a or a Gandharba. The acts done previously go to form the new births of all beings. When the Karma gets exhausted, then no more birth takes place. There is no doubt in this. Brahm\=a, Vi\d{s}\d{n}u, Rudra, Indra and the other Devas, the D\=anavas, Yak\d{s}as, Gandharbas, all are under the control of this Karma. O King! Were it not so, how could they get bodies that are the causes of the enjoyments of pains and pleasures of all the beings. Therefore, O King! Out of the Sa\~nchita Karmas done in many previous births, some Karmas get ripe in due time and they manifest themselves; those manifested Sa\~nchita Karmas are called Pr\=arabdha Karmas (those that are being enjoyed by an individual in the present birth). Impelled by this Pr\=arabdha Karma, the Devas and the human beings, all do meritorious acts as well as sinful acts. Thus Indra out of his past meritorious acts attained his Indraship, and, out of his past sinful acts, committed the sin Brahmahatty\=a and so he was dislodged from his Indraship. What doubt can exist here? O King! So Nara and N\=ar\=ayana, the sons of Dharma, had to take births out of their previous Karmas; again Arju\d{n}a and Kri\d{s}\d{n}a were born out of their Karmic effects as part incarnations of this Nara and N\=ar\=ayana. The Munis describe this Karma as the basis of the Pur\=a\d{n}as. Know that he is born of a Deva who is very wealthy and prosperous; he who is not born of the part of a Muni, never writes any spiritual treatise on J\~nana or Knowledge; he who is not born of Rudra, never worships Rudra; who is not born of a Deva never distributes rice in charity; he who is not born

of \'Sr\={\i} Vi\d{s}\d{n}u, never becomes the king and lord of the earth. O King! The embodied souls derive their bodies certainly from Indra, Agni, Yama, Vi\d{s}\d{n}u, and Kuvera. Indra presides over lordship, Agni presides over energy, Yama presides over anger, and Vi\d{s}\d{n}u presides over strength. He who is powerful, fortunate, enjoying many enjoyments, learned, charitable, is said to be born of a Dev\=am\'sa. O Lord of the earth! Similarly the P\=andavas and V\=asudeva who was as glorious as N\=ar\=ayana were born of Dev\=am\'sas. O King! Know this as quite certain that the bodies of the J\={\i}vas are the receptacles of pains and pleasures; and the embodied souls (J\={\i}vas) experience alternately pleasure and pain. No J\={\i}va is independent; he is always under the Great Fate. He experiences birth, death, pleasure and pain, not out of his self will, but compelled and guided, as it were, by the unseen Fate.

30-41. O King! How very strong is that Fate can easily be judged by the following. The P\=andavas were born in forest; then they went their own homes. They performed the Great R\=ajas\=uya Sacrifice by virtue of their own strength. After this they had to suffer their exiles in forest a much greater and more terrible hardship indeed! Next Arju\d{n}a performed a very hard asceticism when the Devas, not self-controlled, became pleased and granted him an auspicious boon. Still he could not extricate himself from the hands of the terrible hardship; nowhere could be found the fruits of the merits acquired in the past when he was afterwards remaining in exile in his human body in the forest! The severe tapasy\=a that he did in the Vadarik\=asrama in his past incarnation as Nara, the son of Dharma, did not bear any fruit in his Arju\d{n}a birth. Mysterious and inexplicable are the ways and means of Karma with which the bodies of the several beings are concerned. How could men get an idea of it when the Devas themselves are at a loss to solve it. Bhagav\=an V\=asudeva had to take birth in the prison, a very critical and dangerous place; he was then carried by Vasudeva to the milkman Nanda's abode at Gokula; he remained there eleven years and thence came back to Mathur\=a where he killed by force Kamsa, the son of \=Ugrasena. Then he released his sorrowful father and mother from the bonds of prison and made \=Ugrasena, the King of Mathur\=a. Afterwards he went to Dv\=ark\=a city, out of the fear of K\=ala Yavana, the King of the Mlechchas; thus Jan\=ardana Kri\d{s}\d{n}a performed many great and heroic deeds, being impelled by Fate. Then he left his mortal coil at Prabh\=asa, a place of pilgrimage, along with his relatives and acquaintances and then ascended to his Vaikuntha abode. All the Y\=adavas, sons, grandsons, friends, brothers, sisters and ladies of the houses all died under the curse of a Br\=ahmi\d{n}. O King! I have thus described to you the inexplicable ways of Karma.

What more shall I say than the fact that V\=asudeva was killed by the arrows of a hunter!

Here ends the Tenth Chapter of the Sixth Book on the phase of Karma in the Mahapur\=a\d{n}am \'Sr\={\i} Mad Dev\={\i} Bh\=agavatam of 18,000 verses by Mahar\d{s}i Veda Vy\=asa.



