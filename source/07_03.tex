\chapter{On the bestowing of the daughter of the King \'Sary\=ati to Chyavana Muni}

1-11. Vy\=asa said :-- O King! Thus the King, troubled with cares asked his soldiers, in an angry mood. Next he asked his friends in sweet words. The princess, seeing his father and his soldiers sorrowful, thought

of her piercing the two eyes of the Muni with a thorn and thus spoke to the King :-- O Father! While sporting in that forest, I came to see a very hard anthill covered with creepers and shrubs wherein I found two holes, O King! Through those small openings, I saw the two shining things as if they were fireflies and thinking them so I pierced them with thorns. At this time a faint voice I heard coming from that anthill. ``Oh! I am killed!'' I then took out my thorns and found them wet with water. ``What is this!'' I asked myself and was thunderstruck with fear; but I could not know what I pierced in that anthill. Hearing these gentle words of her daughter, the King \'Sary\=ati thought that that act had no doubt insulted the Muni and went at once to the anthill. He broke the anthill that covered the Muni and saw the suffering Chyavana aged in practising Tapasy\=a, very much in pain. The King prostrated flat before the Muni and then with folded hands, praised him with hymns and humbly said to him :-- ``O Intelligent One! My daughter has done this wrong act while sporting; Therefore O high-souled One! What she had done unknowingly, do you forgive out of your own high-hearted-ness and liberality. I have heard that the ascetics are always void of anger; therefore now you have to forgive this daughter of the offence and thus shew your kindness.''

12-16. Vy\=asa said :-- The Mahar\d{s}i Chyavana, hearing thus the King's words and specially seeing his humble and distressed nature, said :-- O King! I never was angry a bit; your daughter had pained me; yet I am not angry and have not cast on her any curse; you better see, that I am innocent; much pain is felt by me due to my eyes being pricked. O King! It seems that you are sorry and troubled for that sin. Who can acquire happiness in this world who has committed a great offence to a Bhakta of the Dev\={\i}, in spite he gets even \'Siva as his Protector. O King! On the one hand, I am now worn out by old age, and then, on the other hand, I am deprived of my eyes; what shall be now my means? Please say, who will take care of the blind man?

17. The King said :-- O Muni! The anger of the ascetics is transient; you are in practice of your tapasy\=a; so your anger is a thing of impossibility. So kindly forgive the offence of the daughter. I have got many persons who will incessantly take care of you.

18-22. Chyavana said :-- ``O King! There is none of my relations with me; then I am now made blind; how shall I go on with my tapasy\=a? I do not think that your servants will take care of me. O King! If you think it your duty to please me, then do my word,

give me your lotus-eyed daughter to serve me and take care of me. O King! I will be very glad if I acquire your daughter; she will serve me when I will be engaged in my tapasy\=a. O King! This, if observed, will satisfy me and all the troubles that are now with you and your army will no doubt disappear. O King! Think and grant me your daughter; I am an ascetic observing vows and if you give over your daughter to me, you will not incur any sin nor any fault.''

23-31. Vy\=asa said :-- O Bharata! Hearing thus the Muni's words, the King \'Sary\=ati was immersed in cares and could not say anything whether he would or would not give over his daughter to him. The King thought, ``My daughter is very fair like a Devakany\=a and this Muni is aged, ugly and specially he is blind; how then can I be happy if I give over my daughter to him. Who is there so stupid and vicious that knowing his good and bad, he for his own selfish happiness wants to deprive his beautiful daughter of the enjoyments of her married life. How will that fair eyebrowed daughter of mine pass her days happily in the company of this aged Muni when she will be overpowered by passion. The more so when the young beautiful ladies are not able to conquer their passions though possessed of husbands of their own standard and liking, how then can my daughter conquer her passion on getting this old blind husband! The exquisitely beautiful Ahaly\=a married Gautama; but, seeing the youthful beauty of that lovely lady, Indra deceived her and took away her chastity. Till at last, her husband Gautama finding that action contrary to Dharma, cursed him. Now through the severe curse of that Br\=ahma\d{n}a many troubles may arise; so I cannot in any case give my daughter Sukany\=a over to him.'' Thus thinking and absent-minded the King went back to his home and, being very distressed, called his ministers to form a council. O Ministers! What am I to do now? Is it advisable to give over my daughter to the Muni? Or is it better to suffer these pains? Judge and say what is the best course for me.

32. The Ministers said :-- ``O King! What shall we say in this critical juncture? How can you bestow your exceedingly beautiful daughter to that ugly unfortunate ascetic?''

33-45. Dvaip\=ayana said :-- At this moment, seeing her father and ministers troubled very much with cares, Sukany\=a understood at once everything by signs and hints; she then smilingly said to her dear father :-- ``O Father! Why are you looking so sad with cares? Perhaps you are very much troubled and sad for me. O Father! I have pained that Muni; so I will go and console him; what more than this that I will

give up myself at his feet and please him.'' Hearing these words of Sukany\=a, the King spoke to her very gladly before all the ministers. O Daughter! The Chyavana Muni is blind, aged and of a worn figure, especially of a very irritable temper; and you are a mere girl; how will you be able to serve him in that dreadful forest? You are like Rati in beauty and loveliness; how can I bestow my daughter to that aged worn out, blind Muni for my own pleasure! The father marries the daughter to him who has got relations, who is of a proper age, strong, who has got unequalled grains and wealth, gems and jewels; never to a man void of wealth. O broad-eyed One! You are exquisitely handsome; and that ascetic is very old; see what an amount of difference lies between you two. The Muni, moreover, has past his marriageable age; so how can I give over my daughter. O Lotus-eyed! You always dwell in beautiful places; how can I now make you dwell for ever in thatched huts? O Cuckoo avoiced one! Rather will I and my soldiers die than to bestow you to that blind husband. Let whatever come it may, I will never lose patience; therefore, O One of good hips! Be quiet. I will never give you to that blind man. O my Daughter! I don't care a straw whether my kingdom and my body live or die, but I will never be able to give you over to the ascetic. Hearing thus the father's words, Sukany\=a began to speak with a glad face the following sweet and gentle words :--

46-48. O Father! Do not trouble yourself for nothing with cares on my account. Give me over now to that best of Munis; then all the persons will be happy, no doubt. I will derive my intense pleasure there in that dense forest by serving with intense devotion my old husband, who is very holy. I have got not the least inclination towards these worldly enjoyments which are the sources of all troubles for nothing, My heart is now quiet. Therefore, O Father! I will become a chaste wife to him and act according to his liking.

49-54. Vy\=asa said :-- O King! The Ministers were greatly amazed on hearing these words and the King also became greatly pleased and took her to the presence of the Muni. Going before him, he bowed down to the Muni and said :-- ``O Lord! Please accept duly this daughter for your Sev\=a.'' Thus saying, the King betrothed his daughter to him according to rules. Chyavana Muni also became very glad to receive her. The Muni took the daughter willingly for his Sev\=a but refused other dowries that the King presented. Thus the Muni became pleased; immediately the soldiers began to evacuate and were very glad. Seeing this, the King's heart became filled with joy. When the King, thus finishing

the betrothal ceremony of his daughter wanted to return home, the thin bodied princess then told her father :--

55-64. Sukany\=a said :-- ``O Father! Take away all my ornaments and clothings and give me for my use an excellent deer skin and one bark. O Father! I will dress myself like the wives of Munis and serve my husband in such a way as will bring to you the unparallelled undying fame in Heaven, Earth and the Nether regions; also I will serve my husband's feet so that I can derive the highest happiness in the next world. I am now full of youth, especially beautiful; do not think a bit that as I am wedded to an aged ascetic, that my character will be spoilt. As Va\'sistha's wife Arundhati has attained celebrity in this world, so I will also attain success; there is no doubt in this. As the chaste wife Anas\=uy\=a of Mahar\d{s}i Artri has become widely known in this world so will I be known also and establish your fame.'' The exceedingly religious King hearing all these words of Sukany\=a gave her deer skin and all other articles wanted. The King could not help weeping, when he saw that his daughter had dressed herself like the daughter of a Muni. He stood fixed, very sad, on that very spot. All the queens were exceedingly filled with sorrow to see the daughter dressed in bark and deer-skin. Their hearts quivered and they began to weep. O King! Then the King \'Sary\=ati bade good bye to the Muni, leaving there his daughter. He went with a grievous heart and returned to his own city, accompanied by the ministers.

Here ends the Third Chapter of the Seventh Book on the bestowing of the daughter of the King \'Sary\=ati to the Chyavana Muni in \'Sr\={\i} Mad Dev\={\i} Bh\=agavatam, the Mah\=a Pur\=a\d{n}am of 18,000 verses, by Mahar\d{s}i Veda Vy\=asa.



