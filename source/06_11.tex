\chapter{On the ascertainment of Dharma}

1-10. Janamejaya said :-- ``O King of the Br\=ahma\d{n}as! You said that R\=ama and Kri\d{s}\d{n}a took their incarnations to relieve the burden of earth. One great doubt arises in my mind on this point. At the end of the Dv\=apara Yuga, the Earth, burdened and oppressed very much, assumed, in anguish, the form of a cow and took refuge under Brahm\=a. Brahm\=a, then, went with the Earth to Vi\d{s}\d{n}u, the Lord of Lak\d{s}m\={\i}, and thus prayed, ``O Bibhu! Let You, with all the other gods, incarnate soon on earth at the house of V\=asudeva to relieve the Earth of Her load, as well as to protect the righteous.'' When Brahm\=a thus prayed, the Bhagav\=an Vi\d{s}\d{n}u incarnated as the son of Devak\={\i}, along with Balar\=ama to lessen the burden of the Earth. And, in fact, he relieved, to a certain extent, the Earth by killing many vicious persons and many wicked and irreligious Kings. But, along with that, Bh\={\i}\d{s}ma, Dro\d{n}a, Vir\=ata, Drupada, Som\=adatta, and Kar\d{n}a, the son of the Sun were killed. But, See! that those who plundered afterwards His riches, and stole away the wives of Hari, those crores of \=Abh\={\i}ras, \'Sakas, Mlechchas, and Ni\d{s}\=adas and other vicious people remained alive; and how could it, then, be said that the Earth was relieved when Kri\d{s}\d{n}a did not kill those people! O Fortunate One! When I see all the people in this K\=al\={\i} Yuga addicted to sinful acts, this great doubt is not going out of my mind (how the Earth had been relieved of Her load).

11-14. Vy\=asa said :-- O King! As the Yuga changes, so the people changes in course of time. Nothing can alter its course, for this is caused by the Yuga Dharma (the Dharma peculiar to each Yuga). Therefore if all the subjects that are considered wicked and vicious according to the law of the Yuga Dharma, then this creation would be destroyed; hence Kri\d{s}\d{n}a killed only those D\=anavas and vicious K\d{s}attriyas that were really the burden of Earth. O King! The persons that are devoted to religion take their births in the Satya Yuga; those that are fond of religion and wealth they become manifest in the Tret\=a Yuga; those that like Dharma (religion), Artha (wealth) and

Kama (desires), they are born in the Dv\=apara Yuga, and those that dote on wealth and lust, they are seen in the K\=al\={\i} Yuga. O King! Know this as certain that these characteristics, peculiar to each Yuga, never vary; and know this too, that Time, the Lord of Dharma and Adharma, is always present.

15-18. The King said :-- ``O Intelligent One! Where are those pious persons now that were born as high-souled religious persons in the Satya Yuga; where are those Munis now who were devoted to charity in the Tret\=a or Dv\=apara Yuga? Again where will go these shameless and merciless persons, that are being seen now in this K\=al\={\i} Yuga, these vicious creatures that revile their own Gurus? O Highly Intelligent One! I am very eager to know how these religious matters are brought to a decision and settlement; kindly describe to me in detail all these secret truths.''

19-30. Vy\=asa said :-- O King! Persons, born in the Satya Yuga, that perform acts of merit, go to the Deva Loka. O King! The Br\=ahmi\d{n}s, K\d{s}attriyas, Vai\'syas and \'S\=udras, if they remain in their own spheres and if they be devoted to religious acts, go to their respective spheres, earned by their meritorious deeds. By virtue of truth, mercy, charity, going to one's own wives, not injuring animals, and having no jealousy and showing mercy equally towards all, by practising these universal forms of religion, even the lowest castes, e.g., washermen and others all go to the Paradise. So in the Tret\=a and Dv\=apara Yugas men go to Heaven by virtue of their merits, earned in practising their own Dharma; but in this K\=al\={\i} Yuga persons addicted to vicious acts go to terrible hells and remain there till the end of the K\=al\={\i} Yuga when they will be again born in this earth. O King! When the Satya Yuga begins and the K\=al\={\i} Yuga ends, at this junction time, the virtuous highsouled persons descend from Heaven and are born on this earth; and when the K\=al\={\i} begins and the Dv\=apara ends, the vicious souls come on the earth again from their hells. O King! Know this as the course of Time; it never becomes otherwise. See, then, that the K\=al\={\i} Yuga tends to do vicious things and the people, therefore, become vicious. At times, the birth of beings takes place otherwise than the laws of Yugas, out of the strange combinations of Fate (i.e., good persons are seen in the K\=al\={\i} and vicious persons are seen in the Satya). For this reason those that do meritorious acts in the K\=al\={\i} Yuga are born as men in the Dv\=apara; so the Dv\=apara good persons take their births as men in the Tret\=a; and the Tret\=a good persons are born as men in the Satya Yuga. Again those who are vicious in the Satya Yuga become persons of the K\=al\={\i} Yuga. The J\={\i}vas suffer miseries on account of their own bad Karmas; they again suffer more miseries

by doing over and over again those bad Karmas by virtue of the Yuga Dharma.

31. Janamejaya said :-- ``O Bhagav\=an! Describe particularly the details of the Yuga Dharma. I am now very desirous to hear which Dharma is for which Yuga?''

32-54. Vy\=asa said :-- O King! I will now show to you by example the influence of the religion peculiar to each Yuga; hear it attentively. O King! The hearts even of saints are quite disturbed by the Yuga Dharma. See! Your father was a religious and high-souled monarch; still the wicked K\=al\={\i} defiled his mind and prompted him to do an act very insulting to a Br\=ahma\d{n}a. Otherwise why would he, being a renowned prince amongst the K\d{s}attriyas and a descendant of Yay\=ati, thus go and encircle a snake round the throat of an ascetic Br\=ahmi\d{n}? Therefore, O King! All actions are being influenced by the Yuga Dharma. The Pundits, also recognise this. If you try your best to perform any religious act, even then the Yuga Dharma would prevail, yet you would be able to perform to a certain extent, a part of your intention. O King! In the Satya Yuga, the Br\=ahmi\d{n}s were versed in the Vedas, always devoted to worship the Highest Force, with an ardent desire to see the Dev\={\i}; they were devoted to G\=ayatr\={\i} with Pra\d{n}ava, devoted to the meditation of G\=ayatr\={\i}, always reciting silently G\=ayatr\={\i}, and the M\=ay\=av\={\i}ja Mantram, the chief mantram. In every village, the Br\=ahmi\d{n}s were very eager to erect temples of the Dev\={\i} Mah\=a M\=ay\=a Ambik\=a and were truthful, merciful and pure and devoted to their own respective works. The K\d{s}attriyas, skilled in the science of the highest knowledge, were ever engaged in doing things ordained by the Vedas and were always intent in protecting well their subjects. The Vai\'syas did their cultivation and trade and the \'S\=udras always served the other three castes. Thus, in the Satya Yuga, all the Var\d{n}as (castes) were devoted to the worship of the Dev\={\i} Ambik\=a, the Highest \'Sakti; but in the Tret\=a Yuga, the observance of the religion declined a little and in the Dv\=apara, it declined very much. O Ornament of Indra! Those who were R\=ak\d{s}asas before, they become the Br\=ahmi\d{n}s of the K\=al\={\i} Yuga; they are the flowers of atheists, deceptors of men, untruthful, without any Vedas, devoid of the Vedic practices, arrogant, cunning, egoistic, and capable only to serve the \'S\=udras. Some of them try to find fault with the San\=atan Dharma and are the promulgators of various other creeds, wicked, fallen from their religion and given to much talking. O King! As K\=al\={\i} gets stronger, so the true religion declines and ultimately dies; and, in that proportion, the K\d{s}attriyas, Vai\'syas and \'S\=udras are also devoid of their religion. When K\=al\={\i} will be in full swing, the K\d{s}attriyas, Vai\'syas and \'S\=udras will all be untruthful, vicious; the Br\=ahmi\d{n}s will act like \'S\=udras and will accept other's gifts. O King! The women in the

K\=al\={\i} Yuga would be very passionate, avaricious and ignorant. They would be very powerful and insolent, wilful, vicious and untruthful and so would be a source of pain to the society. They would think themselves vainly religious and learned and would be always ready to impart religious instructions and deceive their own husbands and be exceedingly vicious. O King! Our minds are purified by the food that we take; when our minds are pure, the Light of Dharma shines clearly. The customs and practices of Var\d{n}a and \=A\'srama Dharmas get intermixed with each other and so arises the fault of Dharma \'Samkara (i.e., mixture of the several parts of religion with each other). When the Dharma \'Samkara creeps in, the Var\d{n}a \'Sankara is seen (i.e., purity in blood and other matters of birth are lost). Thus, in the K\=al\={\i} Yuga, all the Dharmas will gradually die out and ultimately nothing will be heard about one's own religion. O King! In this Yuga even the religious high-souled persons will be found to do irreligious acts! The nature of K\=al\={\i} is so; nobody will be able to quit it. O King! Thus, in this age, men naturally commit vicious things; with ordinary means, therefore, no one becomes able to extricate from the worst vicious habits.

55-56. Janamejaya said :-- ``O Bhagav\=an! You know all and you are versed in all the \'S\=astras; what will be the fate of so many persons in this K\=al\={\i} Yuga? If there be any path, kindly describe it to me.''

57-65. Vy\=asa said :-- O King! There is only one path and none other which can save a man from the sin of this K\=al\={\i}; and that is this :-- The J\={\i}vas must meditate on the lotus-feet of the Highest Dev\={\i} for the purification of all their faults and sins. O King! There is so much strength in Her sin-destroying Name, that the amount of sin in this world falls much less in proportion to that. Where, then, is the cause of fear? Her Name, uttered at random, even in an unconscious state, bestows so much unspeakable results that even Hari, Hara and others have not the capacity to know that. O King! The mere remembrance of the name of \'Sr\={\i} Dev\={\i} is an atonement for a multitude of sins; then it behoves that every man, afraid of the K\=al\={\i} Yuga, residing in a place of pilgrimage, ought to remember incessantly the Name of the Highest Deity. Even if anybody cuts, pierces, and kills all the beings in this whole world, he won't be touched with the sins, if he bows down, with devotion, before the Dev\={\i}. O King! I have narrated to you all the secret truths of all the \'S\=astras. Consider all these fully and always worship the lotus-feet of the Dev\={\i}. All men are reciting silently the Japam called the Ajap\=a G\=ayatr\={\i}; still they do not know the glory of it; such is the powerful influence of M\=ay\=a. All the Br\=ahma\d{n}as are reciting in the depth of their hearts the G\=ayatr\={\i} Mantram, yet they do not know the glory of it (otherwise they would have

been liberated); such is the great influence of M\=ay\=a. O King! I have described to you all that you asked me about the Yuga Dharmas; what more do you want to hear?

Here ends the Eleventh Chapter of the Sixth Book on the ascertainment of Dharma in the Mah\=a Pur\=a\d{n}am, \'Sr\={\i} Mad Dev\={\i} Bh\=agavatam, of 18,000 verses by Mahar\d{s}i Veda Vy\=asa.



