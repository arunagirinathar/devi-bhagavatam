\chapter{On the questions put by Janamejaya}

1-10. Janamejaya said :-- ``O Bhagav\=an! What is that great Yaj\~na (sacrifice) named Amb\=a Yaj\~na about which you referred just now? Who is the Amb\=a? Where was She born? From whom and what for did Her birth take place? What are Her qualities? What is Her form and nature? O Ocean of mercy! You are all-knowing; kindly describe everything duly. Along with this, describe in detail the origin of Brahm\=anda. O Brahm\=a\d{n}a! You know every thing of this whole Universe. I heard that Brahm\=a, Vi\d{s}\d{n}u and Rudra are the three Devat\=as, who are successively originated to create, preserve, and destroy this Universe. Are these three highsouled entities independent? or Do they do their respective duties, being subservient to another Person? Now I am very eager to know all these. So P\=ar\=a\'sara's son! Describe all these to me. Are these highly powerful Brahm\=a, Vi\d{s}\d{n}u and Mahe\'svara subject to Death like ordinary beings? Or are they of the nature of everlasting Existence, Intelligence and Bliss? Are they subject to the three fold pains arising from their own selves from elements and from those arising from gods? Are they subject Time? How and wherefrom were they originated? Do they feel the influence of pleasure, pain, sleep or laziness? O Muni! Do their bodies consist of seven Dh\=atus? (blood, etc.) or are they of some other kinds? A great doubt has arisen in me on all these points. If these bodies be not made up of five elements, then of what substance are they built of? And of what gu\d{n}as are their senses built also? How do they enjoy objects of enjoyments? How long is their longevity? O Br\=ahma\d{n}a! where do they, Brahm\=a, Vi\d{s}\d{n}u, and Mahe\'svara, the best of the gods live? And of what nature are their powers and prosperities? I like very much to hear all these. So describe all these in detail to me.''

11-24. Vy\=asa said :-- ``O highly intelligent king! The questions that you have asked me today whence and how Brahm\=a, etc., were born ? etc., are

very difficult. In ancient days, once, on an occasion, I asked many questions like you to the Muni N\=arada. At first he was greatly suprised to hear my queries, afterwards he gave due replies to them. O king! I will answer to you in the same way; listen. Once I saw that the all-knowing, peaceful N\=arada, the knower of the Vedas was sitting on the banks of the Ganges. I became very glad and fell at his feet. By his order I took one excellent seat. Hearing, then, of his welfare and seeing him sitting on the sands I asked him :-- ``O highly intelligent One! Who is the Supreme Architect of this widely extended Universe? Whence is this Brahm\=anda born? Is it eternal or temporary? When it is an effect, then it is natural that it cannot be created without a cause. Now when the cause, the creator, is certain, is he one or many? O sage! as regards this wide Sams\=ara, I have expressed my doubt; now answer me what is the Real and True, and thus remove my doubts. Many believe Mah\=a Deva, the Lord of all the other Devas as the Supreme God, the Cause of all. He is the source of deliverance to all the J\={\i}vas; devoid of birth and death; always auspicious; peaceful in Himself and the controller of the three gu\d{n}as. He is the one and only cause of creation, preservation and destruction. Some Pundits believe Vi\d{s}\d{n}u as the God of all and praise Him as such. It is Vi\d{s}\d{n}u that is the powerful Supreme Self, the Lord of all and the First Person \=Adipuru\d{s}a. It is He that has no birth nor death, the Deliverer of the whole J\={\i}vas, Omnipresent; His faces are everywhere; He is the Granter of enjoyments and liberation to the devotees. Some others call again Brahm\=a, the Cause of all. It is He that is omniscient and the Stimulator of all beings.

The four-faced Brahm\=a, the best of all the Devas is born from the navel lotus of some One of endless force. He resides in Satyaloka; He is the Creator of all and the Lord of all the Devas. Again some other Pundits call the Sun, S\=urya as God. In the morning and in the evening they chant His hymns, without any lack of slackness and laziness. Again there are some others, who say that Indra is the lord of all the J\={\i}vas; He is thousand-eyed; it is Indra, the husband of \'Sach\={\i}, that is the God of all. Those who perform Yaj\~nas (sacrifices) worship V\=asava, the king of the Devas. He drinks Soma juice Himself and those who drink Soma are his beloved. He is the one and only Lord of Sacrifices. Thus all men worship, according to their respective wishes, Varu\d{n}a, Soma, Agni, Pavana (wind), Yama (the god of Death), Kuvera, the lord of wealth; there are some again who worship the elephant-faced Ga\d{n}apati, the Fructifier of all actions, the Granter of desires of all the devotees, and the Giver of success to all in all enterprises, no sooner He is remembered. Some \=Ach\=aryas (professors) say again that the All auspicious the \=Adi M\=ay\=a, the Great \'Sakti Bhav\=an\={\i}, the Giver of everything, Who is the nature of with and without attributes

Who is not different from Brahm\=a, who is both Puru\d{s}a and Prakriti, the Creatrix, the Preservatrix and the Destructrix of all, the Mother of all the gods, beings and lokas, is the Great Goddess of this Brahm\=a\d{n}da. She is without beginning and end, full, present in all the beings and everywhere. It is this Bhavan\={\i} that assumes the various endless forms such as Vaisnav\={\i} \'S\=ankar\={\i}, Br\=ahm\={\i}, V\=asav\={\i}, V\=aru\d{n}\={\i}, V\=ar\=ah\={\i}, Nara Simh\={\i}, Mah\=a Lak\d{s}m\={\i} the one and secondless Vedam\=at\=a, and others. It is this Vidy\=a nature that is the One and the only Root of this tree of Sams\=ara (universe).

The mere act of remembering Her destroys heaps of afflictions of the devotees and fulfills all their desires. She gives Mok\d{s}a to those who are desirous of liberation and gives rewards to those who want such. She is beyond the three Gu\d{n}as and still She emanates them. Therefore the Yogis that want rewards meditate Her, Who is of the nature of Vidy\=a and Who is devoid of attributes. The best Munis, the knowers of the truths of Vedanta meditate on Her as formless, immutable, stainless, omnipresent Brahm\=a devoid of all Dharma. She is described in some Vedas and Upanishads as full of Light (Tejas). Some intelligent persons describe God as of infinite hands, infinite ears, infinite legs, infinite faces, peaceful, Vir\=at Puru\d{s}a and describe sky as the Pada (place) of Vi\d{s}\d{n}u. Other knowers of the Pur\=a\d{n}as describe Him as Puru\d{s}ottama. There are some others again who declare that this creation cannot be done by a single individual. Some atheists say that this inconceivable infinite Universe can never be created by one God. So there is no such definite God that can be called its Creator. Though without any creator, this Brahm\=a\d{n}da is sprung from the Nature and conducted by Her. The followers of the S\=amkhya system say that Puru\d{s}a is not the creator of this Universe; they declare that Prakriti is the Mistress of this Universe O Muni! Thus I have expressed to you what the Muni Kapila, the Ach\=arya of the S\=ankhyas and the other philosophers declare as their opinions; various doubts, thus, reign always in my breast. Owing to these doubts my mind is so confused that I cannot arrive at any definite conclusion. My mind is very much unsettled as to what is Dharma and what is Adharma. What are the characteristics of Dharma? I cannot make out them. For the Devas are all sprung from the Sattva Gu\d{n}a and are always attached to the true Dharma; yet they are frequently troubled by the sinful D\=anavas. How, then, can I place my confidence on the permanence of the Dharma? My forefathers, the P\=andavas were always endowed with good behaviours and good actions and they remained always in the path of the Dharma; yet they suffered a good deal of troubles and sufferings. In these cases it is very difficult understand the greatness of Dharma. So, O Father! Seeing all these, my mind is thrown into a sea of doubts and troubles. O Great Muni!

There is nothing impracticable with you; so remove my doubts. O Muni! I am always plunged and raised and plunged again in this sea of delusion. So save me by lifting me on a boat of wisdom and carry me across this ocean of sams\=ara (this world).

Thus ends the first chapter on the third Skandha on the questions put by Janamejaya in the Mah\=apur\=a\d{n}a \'Sr\={\i}mad Dev\={\i} Bhag\=avatam of 18,000 verses by Mahar\d{s}i Veda Vy\=as.



