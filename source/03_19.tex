\chapter{On the going to the Svayamvara assembly of Sudar\'sana}

1-2. Vy\=asa said :-- After the king Sub\=ahu had spoken thus, the Queen made her daughter \'Sa\'sikal\=a, who had always sweet smiles on her lips sit on her lap and after consoling her with sweet words, began to say ``O fair eyed! You always practise vows and other religious performances; why are you, then, speaking these unpleasant words? The King has heard all what you wanted to say and has been very sorry.

3-5. That Sudar\'sana is very unfortunate, deprived of his kingdom, helpless, void of wealth and army, abandoned by his friends, exiled with his mother in the forest, subsisting on roots and fruits, lean and thin. Thus he is not worthy of becoming the husband of yours. There are many learned, beautiful, approved of all, qualified with all royal marks, princes fit to become your husband. They all will come in this Svayamvara.

6. There is one brother of this Sudar\'sana, who is endowed with all kingly qualifications, beautiful, and qualified in various other ways. He is the king of the Kosala country.

7. There is another point worth consideration; please hear it. The King Yudh\=ajit is trying his best to kill Sudar\'sana on a befitting opportunity.

8. He already counselled with his ministers and killed in a desperate fight the king V\={\i}rasena and installed his daughter's son on the throne.

9. Even he came up so far as the hermitage of Bh\=aradv\=aja to kill Sudar\'sana; afterwards he was prevented by the Munis from doing so then he returned home.''

10-11. \'Sa\'sikal\=a replied :-- ``Mother! That prince, though staying in the forest, is approved of by me; under the advice \'Sary\=ati, the chaste Sukany\=a married Chyavana Muni and served her husband all along; so I will marry this king's son and will always be engaged in serving him. The women are able to attain heaven and emancipation, if they serve their husbands; therefore if we be sincere in serving our husbands, we will no doubt be happy.

12. I have seen in my dream that the Goddess Bhagavat\={\i} has ordained him to be my husband; how can I now accept any other body as my husband than him?

13. The Dev\={\i} Bhuvane\'svar\={\i} has pictured his frame firmly in my heart; I will never be able to leave my dearest beautiful husband and to contract marriage with any other person.''

14. Vy\=asa said :-- Thus the mother, the daughter of the King of Videha, found many signs and at last desisted. She then reported to the King all the words of \'Sa\'sikal\=a.

15-16. When \'Sa\'sikal\=a, on the day before the marriage day, became very anxious and, sent in a great hurry, one trustworthy Br\=ahmi\d{n}, versed in the Vedas to the hermitage of Bh\=aradv\=aja with this message ``O Br\=ahma\d{n}, go in such a way to Sudar\'sana, as my father be not able to know about it and tell Sudar\'sana all my words.

17-18. My father has called in for my marriage a Svayambara ceremony; many powerful kings will attend with their armies; O Deva! The Goddess Bhagavat\={\i} has ordered me in dream and accordingly I, with full gladness of my heart, have become yours already in my heart.

19. Rather I will take poison or I will jump in a blazing fire, than I can obey my father's and mother's words and marry another.

20. By my mind, word, and deed, I have selected you my husband; and pleasure and happiness is sure to attend on us by the blessings of the Bhagavat\={\i}.

21. Please depend unto Her, at Whose command this whole universe, moving and unmoving is resting, unto that Great Destiny and come to this place without fail.

22. What the Goddess, whose commands \'Sankara all the other Devas obey, has ordered, can never turn out false.

23. O Br\=ahmi\d{n}! You the foremost amongst the virtuous; do therefore call on that King's son in privacy and speak out all there to him. What shall I say more to you. Do all that my object may be fullfilled.''

24. Thus saying, she gave the Br\=ahmi\d{n} his Dak\d{s}in\=a and sent him to Sudar\'sana. He went there and reported all the matter duly to him and quickly returned back.

25. On coming to know all this, Sudar\'sana determined to start; and the Mah\=ar\d{s}i Bh\=aradv\=aja, with gladness, sent him.

26. Vy\=asa said :-- Seeing her son ready to start, the mother Manoram\=a became very sorry and, trembling and shedding tears, thus spoke to her son.

27-28. ``Sudar\'sana! Where are you going now? How do you dare to go there in the Svayamvara alone, where are present kings and all your terrible enemies. O Son! You are as yet a boy. The King Yudh\=ajit will certainly go there with the object of killing you; there will then is no other body to help you. So you should never go to that place.

29. You are my only son; I am very poor and helpless; I have no other to lean upon than you; therefore you ought not to throw me in despair at this moment.

30. See Sudar\'sana! The King Yudh\=ajit who had slain my father, that uncontrollable king will come there; if you go there alone, he will certainly kill you.''

31. Sudar\'sana replied ``Mother! What is inevitable will certainly come to pass; there is no need to discuss further on the subject. I will go at the command of the World Mother to that assembly hall
Svayamvara.

32. O Auspicious one! Do not give vent to sorrow; I do fear none by the grace of the Bhagavat\={\i}.''

33. Vy\=asa said :-- Thus saying, Sudar\'sana mounted on his chariot and was ready to start. Seeing this Manoram\=a began to bless him and so cheer him.

34-37. O Son! Let Ambik\=a Dev\={\i} protect your front; Padmalochan\=a protect your back; P\=arvat\={\i}, your two sides; \'Siv\=a Dev\={\i}, all around you; V\=ar\=ah\={\i}, in dreadful paths; Durg\=a, in royal forts, K\=alik\=a, in terrible fights; Parame\'svar\={\i}, in the platform hall; M\=atamg\={\i}, in the Svayamvara hall; Bhavan\={\i}, the Avertress of world, amidst the kings; Girij\=a, in mountain passes; Chamund\=a, in the sacrificial ground, and let the eternal K\=amag\=a, protect you in the forests.

38. O Descendant of Raghu family! Let the Vai\d{s}nav\={\i} force protect you in quarrels; let Bhairav\={\i} protect you in battles and amongst your enemies.

39. O Son! Let the Mah\=a M\=ay\=a Jagaddh\=atr\={\i} Bhuvane\'svar\={\i} protect you everywhere and at all times.

40. Vy\=asa said :-- Then Manoram\=a, speaking thus to him, trembled with fear and again said :-- ``O Sudar\'sana, I will also accompany you; there will not be otherwise.

41. I will never be able to remain anywhere without you and even for the twinkling of an eye. O Son, carrry me thither where you are desiring to go.''

42. Thus saying, his mother with her attendants was ready to start. The Br\=ahma\d{n}as pronounced their blessings. All then went out.

43. Sudar\'san, the descendant of the Raghu family, mounted then alone on his chariot and reached Benares. There the King Sub\=ahu, hearing that he had come, welcomed him and worshipped him with various presents.

44. He gave him, the house for his residence, and made arrangements for his food and drink and other necessary requirements and gave order to his servants to wait on the prince.

45. Then, from various quarters, the kings assembled together; and Yudh\=ajit, too, came there accompanied by his daughter's son, \'Satrujit.

46-48. The King of Kar\=u\d{s}a, the King of Madra, the King of Sindhu, the King of M\=ah\={\i}smat\={\i}, the valiant warriors, the King of P\=anch\=ala, the kings of the mountainous tract, the King of Karnat, the powerful King of K\=amar\=upa, the King of Chola, and the very powerful King of Vidarbhas with 180 Ak\d{s}auhin\={\i} soldiers all arrived and assembled there. Benares was then crowded all over with soldiers and soldiers.

49. Many other kings came there on their beautiful elephants to witness the Svayamvara ceremony.

50. Then the princes began to talk amongst them ``The King's son Sudar\'sana, too, had come there and is staying unconfused and calm.

51. Is it that the high minded Sudar\'sana, born of the K\=akutstha family, had come there on a chariot, helpless, to marry?

52. Can it be that the princess will overlook these Kings with soldiers and weapons, and select the long armed Sudar\'sana?''

53. Then the King Yudh\=ajit addressed all the other kings ``I will slay Sudar\'sana for the sake of the daughter; there is no doubt in this.''

54-55. Hearing Yudh\=ajit's words, the king of Keral, the foremost of those who know morals, began to say :-- ``O king! In this Ichchh\=a svayamvara, it is not proper to fight. Here there will be no marriage for the prowess; there is no arrangement fixed to steal away the bride elect by force; here the bride will select of her own free choice; what cause can then there crop up here for quarrels?

56. Before, you had driven him out of his kingdom; and though you are the superior king, you have taken his kingdom by force and installed your daughter's son on the throne.

57. O King! This Sudar\'san is born of the K\=akutstha family and the son of the King of Kosala. Why would you kill this innocent boy?

58. O Long lived! Better be sure that there is some God of this Universe; He is governing all; and if you commit anything wrongful know that you will get the fruit of that due to you; there is no doubt it
this.

59. O King! There is victory everywhere of the Truth and Dharma, always you find Adharma and Falsehood defeated. Therefore dost thou forsake your evil and mean intentions and pacify your vile mind.

60. Your daughter's son is also present here; he is beautiful and prosperous and is reigning a kingdom. Why will not that bride elect him as bridegroom?

61-62. Consider again that there are many other powerful princes and kings in this Svayamvara; the princess may select them also. Therefore let all the kings assembled here say that if the selection of the bridegroom be performed in that way, what cause of a quarrel can there crop up? Knowing all these, you ought not to quarrel here.''

Thus ends the Nineteenth Chapter on the going to the Svayamvara assembly of Sudar\'sana and the other kings in the Mah\=a Pur\=a\d{n}am \'Sr\={\i} Mad Dev\={\i} Bh\=agavatam of 18,000 verses, by Mah\=ar\d{s}i Veda Vy\=asa.



