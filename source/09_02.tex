\chapter{On the origin of Prakriti and Puru\d{s}a}

1-4. N\=arada said :-- O Lord! I have heard all that you said in brief about the Prakriti Dev\={\i}. Now describe in detail. Why the M\=ul\=a Prakriti \=Adya \'Sakti (the Prime Force) was created at the beginning before the creation of this world of five elements. How did She, being of the nature of the three Gu\d{n}as, come to be divided into five parts? I desire to hear all this in detail. Now kindly describe their auspicious births, methods of worship, their meditation, their stotras (praises), Kavachas (the mystic syllables considered as a preservation like armour), glory and power in detail.

5-26. N\=ar\=ayana spoke :-- ``O Devar\d{s}i! The M\=ul\=a Prakriti, of the nature of M\=ay\=a of Para Brahman is an eternal entity (the nabho mandal); Time (K\=ala), the ten quarters, the Universe Egg, the Goloka and, lower than this, the Vaikuntha Dh\=ama all are eternal things. \=Atman and Prakriti are in inseparable union with each other as Fire and its burning capacity, the Moon and her beauty, the lotus and its splendour, the Sun and his rays are inseparably united with each other. As the goldsmith cannot prepare golden orna-

ments without gold and as the potter cannot make earthen pots without earth, so the \=Atman cannot do any work without the help of this omnipotent Prakriti. The letter ``Sa'' indicates ``Ai\'syaryam'' prosperity, the divine powers; and ``Kti'', denotes strength; and in as much as She is the bestower of the above two, the M\=ul\=a Prakriti is named ``\'Sakti''. ``Bhaga'' is indicative of knowledge, prosperity, wealth, fame; and in as much as M\=ul\=a Prakriti has all these powers, She is also called ``Bhagavat\={\i}''. And \=Atman ``is always in union with this Bhagavat\={\i} Who is all powers, so He is called ``Bhagav\=an'. The Bhagav\=an is therefore sometimes with form; and sometimes He is without form. (Note :-- When Prakriti becomes latent, God is without form; with Prakriti manifest, God is with form.) The Yogis always think of the Luminous Form of the Formless Bhagav\=an and declare Him to be all blissful Para Brahma, the God. Though He is invisible, the Witness of all, Omniscient, the Cause of all, the Giver of everything and of every form, yet the Vai\d{s}\d{n}avas do not say so. The Vai\d{s}\d{n}avas declare how can fire, strength and energy come when there is no fiery, strong, energetic Person behind it? Therefore He who shines in the centre of this fiery sphere is the Para Brahma; He is the Fiery Person; He is higher than the Highest. He is All Will; He is All Form, the Cause of all causes and His Form is Very Beautiful. He is Young; He looks very peaceful and loved by all. He is the Highest; and His Blue Body shines like new rain-clouds. His two eyes defy the beauty of the autumn lotuses in the mid-day; His exquisitely nice rows of teeth put all the series of pearls in the dark back-ground. The peacock's feather is seen on His crown; the garland of M\=alat\={\i} flowers is suspended from His neck.; His nose is exceedingly beautiful; the sweet smile is always seen on His lips. There is no second like Him in showing favour to the Bhaktas. He wears yellow clothings, as if the burning fire is emanating all round; the flute is seen on both His hands, reaching His knees. His body is decorated all over with jewels. He is the Sole Refuge of this Universe; the Lord of all, omnipotent and omnipresent. No trace of deficiency can be seen in Him; He is Himself a Siddha (perfect) Puru\d{s}a; and the foremost of all Siddha Puru\d{s}as; bestows Siddhis to all. The Vai\d{s}\d{n}avas meditate always That Eternal \'Sr\={\i} Kri\d{s}\d{n}a, the Deva of the Devas. He takes away fully all the fears of birth, death, old age, and ills and sorrows. The age of Brahm\=a is the twinkling of His eye. That Highest Self, the Para Brahma is denominated as Kri\d{s}\d{n}a. The word ``Kri\d{s}'' denotes Bhakti to \'Sr\={\i} Kri\d{s}\d{n}a and the letter ``\d{n}a'' signifies devotion to His service. So He is the Bestower of Bhakti and devotion to His service. Again ``Kri\d{s}'' denotes all; everything; and

``\d{n}a'' signifies the root. So He Who is the Root and Creator of all, is \'Sr\={\i} Kri\d{s}\d{n}a. When He desired, in the very beginning, to create this Universe, there was nothing then except \'Sr\={\i} Kri\d{s}\d{n}a; and at last, impelled by K\=ala (His Own Creation), He became ready, in His part, to do the work of creation.

27-61. The Lord, who is All Will, willed and divided Himself into two parts, His Left part becoming female and His Right part becoming male. Then that Eternal One, Who is greatly loving, looked at the female, His left part, the Sole Receptacle to hold all the contents of love, very lovely to the eyes, and looking like the beautiful lotus. The loins of this woman defy the Moon; Her thighs put the plaintain trees quite in the background; Her breasts are mistaken for the beautiful Bel fruits; flowers are scattered as Her Hairs on the head; Her middle part is very slender, very beautiful to behold! Exceedingly lovely; appearance very calm; sweet smile reigning in Her lips; side long glances with Her; Her clothing is purified with fire; all over Her body decorated with gems. Her eyes, also, like the Chakora bird (Greek partridge) began to drink incessantly with joy the moon beams from the face of \'Sr\={\i} Kri\d{s}\d{n}a, defying, as it were, the ten millions of moons. On Her forehead there was the dot of vermillion (red-lead); over that, the dot of white sandal paste and over that was placed the musk. The fillets or braids of hair on Her head are slightly curved; this was decorated with M\=alat\={\i} garlands; on Her neck was suspended the necklace of gems and jewels and She is always very amorous towards Her husband. On looking at Her face, it seems that ten millions of moons have arisen at once; when She walks, Her gait puts (humiliates) those of ganders and elephants in the shade. O Muni! \'Sr\={\i} Kri\d{s}\d{n}a, the Lord of the R\=asa Dance, and the Person of Taste in the R\=asa sport, looked askance at Her for a while and then catching hold of Her by Her hand went to the R\=asamandalam and began to play the R\=asa sport (the amorous pastime). It seemed then the Lord of amorous pastimes had become incarnate there and had been enjoying the various pleasures of amorous passions and desires. So much, that Brahma's one day passed away in that sport. The Father of the Universe, then becoming tired, impregnated in an auspicious moment in Her womb who was born of His left portion. The Prakriti Dev\={\i} was also tired of the embraces of \'Sr\={\i} Kri\d{s}\d{n}a; so after the intercourse, she began to perspire and breathe frequently. Her perspiration turned into water and deluged the whole universe, with water; and Her breath turned into air and became the life of all beings. The female that sprung from the left side of V\=ayu became his wife and out of their contact originated Pr\=a\d{n}a, Ap\=ana,

Sam\=ana, Ud\=ana and Vy\=ana, the five sons. These are the five vital V\=ayus of all the beings. Besides these from the womb of the V\=ayu's wife came out N\=aga and the other four lower V\=ayus. The water that came out from perspiration, Vara\d{n}a Deva became the Presiding Deity of that; and the female, sprung out of the left side of Varu\d{n}a Deva, became the wife of Varu\d{n}a, called Varu\d{n}\=an\={\i}. On the other hand, the \'Sakti, of the nature of knowledge of \'Sr\={\i} Kri\d{s}\d{n}a, remained pregnant for one hundred manvantaras. Her body became effulgent with Brahma-teja (the fire of Brahma). Kri\d{s}\d{n}a was Her life and She again was dearer to Kri\d{s}\d{n}a than his life even. She remained always with \'Sr\={\i} Krisna; so much so that She constantly rested on His breast. When one hundred manvantaras passed away, that Beautiful One gave birth to a Golden Egg. That egg was the repository of the whole universe. The Beloved of Kri\d{s}\d{n}a became very sorry to see the egg and out of anger, threw that within the water collected in the centre of the Universe. Seeing this, \'Sr\={\i} Kri\d{s}\d{n}a raised a great cry and immediately cursed Her duly and said :-- ``O Angry One! O Cruel One! When you have forsaken out of anger this son just born of you, I say then that you become from today bereft of any issue. Besides, let all those godly women that will spring out of your parts, they also be deprived of having any issue or sons and they will remain ever constant in their youth. O Muni! While \'Sr\={\i} Kri\d{s}\d{n}a was thus cursing, suddenly came out from the tongue of the beloved of Kri\d{s}\d{n}a, a beautiful daughter, of a white colour. Her clothings were all white, in her hands there were lute and book and all Her body was decorated with ornaments made of gems and jewels. She was the Presiding Deity of all the \'S\=astras. Some time later the M\=ula Prakriti, the Beloved of Kri\d{s}\d{n}a divided into two parts. Out of Her left portion came Kamal\=a and out of Her right portion came R\=adhik\=a. In the meantime \'Sr\={\i} Kri\d{s}\d{n}a divided Himself into two parts. From his right side appeared a form two-handed; and from the left side appeared a form four-handed. Then \'Sr\={\i} Krisna addressed the Goddess Speech, holding flute in Her hand, ``O Dev\={\i}! You follow this four-handed Person as his wife'' and then spoke to R\=adh\=a :-- ``O R\=adhe! You are a sensitive, proud lady; let you be My wife; so it will do you good''. \'Sr\={\i} Kri\d{s}\d{n}a also told Lak\d{s}m\={\i} gladly to become the wife of the four-handed N\=ar\=ayana. Then N\=ar\=ayana, the Lord of the world, took both Lak\d{s}m\={\i} and Sarasvat\={\i} to the abode Vaikuntha. O Muni! Both Lak\d{s}m\={\i} and Sarasvat\={\i} became issueless, being born of R\=adh\=a. From the body of N\=ar\=aya\d{n}a arose his attendants, all four-handed. They were all equal to Him in appearance, in qualifications; in spirit and in age. On the other hand, from the body of Kamal\=a arose millions of female attendants all equal to Her in form and qualifications. Then

arose innumerable Gopas (cow-herds) from the pores of \'Sr\={\i} Kri\d{s}\d{n}a. They were all equal to the Lord of Goloka in form, Gu\d{n}as, power and age; they were all dear to Him as if they were His life.

62-88. From the pores of R\=adhik\=a came out the Gopa Kany\=as (cow-herdesses). They were all equal to R\=adh\=a and all were Her attendants and were sweet-speaking. Their bodies were all decorated with ornaments of jewels, and their youth was constant, they were all issueless as \'Sr\={\i} Kri\d{s}\d{n}a had cursed them thus. O Best of Br\=ahma\d{n}as! On the other hand, suddenly arose Durg\=a, the M\=ay\=a of Vi\d{s}\d{n}u (The Highest Self) eternal and whose Deity was Kri\d{s}\d{n}a. (N. B. Durg\=a was the Avat\=ara of M\=ula Prakriti not the Avat\=ara of R\=adh\=a as Lak\d{s}m\={\i} and Sarasvat\={\i} were.) She is N\=ar\=ayan\={\i}; She is \=I\'s\=an\={\i}; She is the \'Sakti of all and She is the Presiding Deity of the intelligence of \'Sr\={\i} Kri\d{s}\d{n}a. From Her have come out many other Dev\={\i}s; She is M\=ula Prakriti and She is \=I\'svar\={\i}; no failings or insufficiencies are seen in Her. She is the Tejas (of the nature of Fire) and She is of the nature of the three Gu\d{n}as. Her colour is bright like the molten gold; Her lustre looks as if ten millions of Suns have simultaneously arisen. She looks gracious always with sweet smile on Her lips, Her hands are one thousand in number. Various weapons are in all Her hands. The clothings of the three-eyed one are bright and purified by Fire. She is decorated with ornaments all of jewels. All the women who are the jewels are sprung from Her parts and parts of parts and by the power of Her M\=ay\=a, all the people of the world are enchanted. She bestows all the wealth that a householder wants; She bestows on Kri\d{s}\d{n}a's devotees, the devotion towards Kri\d{s}\d{n}a; nay, She is the Vai\d{s}\d{n}av\={\i} \'Sakti of the Vai\d{s}\d{n}avas. She gives final liberation to those that want such and gives happiness to those that want happiness. She is the Laksm\={\i} of the Heavens; as well She is the Laksm\={\i} of every household. She is the Tapas of the ascetics, the beauty of the kingdoms of the kings, the burning power of fire, the brilliancy of the Sun, the tender beauty of the Moon, the lovely beauty of the lotus and the \'Sakti of \'Sr\={\i} Kri\d{s}\d{n}a the Highest Self. The Self, the world all are powerful by Her \'Sakti; without Her everything would be a dreary dead mass. O N\=arada! She is the seed of this Tree of World; She is eternal; She is the Stay, She is Intelligence, fruits, hunger, thirst, mercy, sleep, drowsiness, forgiveness, fortitude, peace, bashfulness, nourishment, contentment and lustre. The M\=ula Prakriti praising \'Sr\={\i} Kri\d{s}\d{n}a stood before Him. The Lord of R\=adhik\=a then gave Her a throne to sit. O Great Muni! At this moment sprang from the navel lotus the four-faced Brahm\=a, with his wife S\=avitr\={\i}, an exceedingly beautiful woman. No sooner the four-faced Brahm\=a,

the foremost of the J\~n\=anins, fond of asceticism and holding Kamandalu in His hand came into being than He began to praise \'Sr\={\i} Kri\d{s}\d{n}a by His four mouths. On the other hand the Dev\={\i} S\=avitr\={\i}, with a beauty of one hundred moons, born with great ease, wearing apparel purified by fire and decorated with various ornaments praised Kri\d{s}\d{n}a, the One and Only Cause of the Universe and then took Her seat gladly with Her husband in the throne made of jewels. At that time Kri\d{s}\d{n}a divided Himself into two parts; His left side turned into the forn of Mah\=adeva; and His right side turned into the Lord of Gopik\=as (cow-herdesses). The colour and splendour of the body of Mah\=adeva is pure white like white crystal; as if one hundred suns have arisen simultaneously. In His hands there are the trident (Tri\'sul) and sharp-edged spear (Patti\d{s}a); He is wearing a tiger skin; on His head is matted hair (Jat\=a) of a tawny hue like molten gold; His body was besmeared all over with ashes, smile reigning in His face and on His forehead, the semi-moon. He has no clothing on His loins; so He is called Digambara (the quarters of the Sky being His clothing); His neck is of a blue colour; the serpent being the ornaments on His body and on His right hand the nice bead of jewels well purified. Who is always repeating with His five faces the Eternal Light of Brahm\=a, and Who has conquered Death by praising \'Sr\={\i} Kri\d{s}\d{n}a, Who is of the nature of Truth, the Highest Self, the God Incarnate, the material cause of all things and the All auspicious of all that is good and favourable, and the Destroyer of the fear of birth, death, old age, and disease and Who has been named Mrityunjaya (the conqueror of Death). This Mah\=adeva took His seat on a throne made of jewels (diamonds, emeralds, etc.).

Here ends the Second Chapter of the Ninth Book on the origin of Prakriti and Puru\d{s}a in the Mah\=apur\=a\d{n}am \'Sr\={\i} Mad Dev\={\i} Bh\=agavatam of 18,000 verses by Mahar\d{s}i Veda Vy\=asa.



