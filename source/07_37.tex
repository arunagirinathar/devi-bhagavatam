\chapter{1. The Him\=alay\=as said :-- "O Mother! Now describe your Bhaki Yoga, by which ordinary men who have no dispassion get the knowledge of Brahma easily.}
2-10. The Dev\={\i} said:--"O Chief of Mountains! There are three paths, widely known, leading to the final liberation (Mok\d{s}a). These are Karma Yoga, J\~n\=ana Yoga and Bhakti Yoga. Of these three, Bhakti Yoga is the easiest in all respects; people can do it very well without incurring any suffering to the body, and bringing the mind to a perfect concentration. This Bhakti (devotion) again is of three kinds as the Gu\d{n}as are three. His Bhakti is T\=amas\={\i} who worships Me, to pain others, being filled with vanity and jealousy and anger. That Bhakti is R\=aj\=asic, when one worships Me for one's own welfare and does not intend to do harm to others. He has got some desire or end in view, some fame or to attain some objects of enjoyments and ignorantly, and thinking himself different from Me, worships Me with greatest devotion. Again that Bhakti is S\=attvik\={\i} when anybody worships Me to purify his sins, and offers to Me the result of all his Karmas, thinking that J\={\i}va and \=I\'svara are separate and knowing that this action of his is authorized in the Vedas and therefore must be observed. This S\=attvik\={\i} Bhakti is different from the Supreme Bhakti as the worshippers

think Me separate; but it leads to the Supreme Bhakti. The other two Bhaktis do not lead to Par\=a Bhakti (the Supreme Bhakti or the Highest unselfish Love.)
11-20. Now hear attentively about the Par\=a Bhakti that I am now describing to you. He who hears always My Glories and recites My Name and whose mind dwells always, like the incessant flow of oil, in Me Who is the receptacle of all auspicious qualities and Gu\d{n}as. But he has not the least trace of any desire to get the fruits of his Karma; yea he does not want S\=am\={\i}pya, S\=arsti, S\=ayujya, and S\=alokya and other forms of liberations! He becomes filled with devotion for Me alone, worships Me only; knows nothing higher than to serve Me and he does not want final liberation even. He does not like to forsake this idea of Sevya (to he served) and Sevaka (servant who serves). He always meditates on Me with constant vigilance and actuated by a feeling of Supreme Devotion; he does not think himself separate from Me but rather thinks himself "that I am the Bhagavat\={\i}." He considers all the J\={\i}vas as Myself and loves Me as he loves himself. He does not make any difference between the J\={\i}vas and myself as he finds the same Chaitanya everywhere and mainfested in all. He does not quarrel with anybody as he has abandoned all ideas about separateness; he bows down, and worships the Ch\=and\=alas and all the J\={\i}vas. He who becomes filled with devotion to Me whenever he sees My place, My devotees, and hears the S\=astras, describing My deeds, and whenever he meditates on My Mantras, he becomes filled with the highest love and his hairs stand on their ends out of love to Me and tears of love flow incessantly from both his eyes; he recites My name and My deeds in a voice, choked with feelings of love for Me. [N. B.--The Par\=a Prema Bhakti is like the maddening rush of a river to the Ocean; thence in the shape of vapour to the highest; Him\=alay\=an Mountain peaks to be congealed into snow where various plays of bright colours take place.]
21-30. O Lord of the mountains! He worships Me with intense feeling as the Mother of this Universe and the Cause of all causes. He performs the daily and occasional duties and all My vows and sacrifices without showing any miserly feeling in his expenditure of money. He naturally longs to perform My festivities and to visit places where My Utsabs are held. He sings My name loudly and dances, being intoxicated with My love, and has no idea of egoism and is devoid of his body-idea, thinking that the body is not his. He thinks that whatever is Pr\=arabdha (done in his previoas lives) must come to pass and therefore does not become agitated as to the preservation of his body and soul. This sort of Bhakti is called the Par\=a Bhakti or the Highest Devotion. Here the

predominent idea is the idea of the Dev\={\i} and no other idea takes its place. O Mountain! He gets immediately dissolved in My Nature of Consciousness whose heart is really filled with such Par\=a Bhakti or All Love. The sages call the limiting stage of this devotion and dispassion as J\~n\=ana (knowledge). When this J\~n\=ana arises, Bhakti and dispassion get their ends satisfied. Yea! He goes then to the Ma\d{n}i Dv\={\i}pa, when his Ahamk\=ara does not crop up by his Pr\=arabdba Karma, though he did not fail to give up his life in devotion. O Mountain! That man enjoys there all the objects of enjoyments, though unwilling and at the end of the period, gets the knowledge of My Consciousness. By that he attains the Final Liberation for ever. Without this J\~n\=ana, the Final Liberation is impossible.
31-33. He realises Para Brahma who gets in this body of his the above J\~n\=ana of the Pratyak \=Atm\=a in his heart; when his Pr\=a\d{n}a leaves his body, he does not get re-birth. The \'Sruti says :-- "He, who knows Brahma, becomes Brahma." In the logic of Kantha, Ch\=am\={\i}kara, (gold on the neck) the ignorance vanishes. When this ignorance is destroyed by knowledge, he attains all his knowledge the object to be attained, when he recoginises the gold on his neck.
34-37. O Best of Mountains! This My consciousness is different from the perceived pots, etc., and unperceived M\=ay\=a. The image of this Param\=atm\=a is seen in bodies other than the \=Atm\=a as the image falls in a mirror; as the image falls in water, so this Param\=atm\=a is seen in the Pitrilokas. As the shadow and light are quite distinct, so in My Ma\d{n}idv\={\i}pa, the knowledge of oneness without a second arises. That man resides in the Brahma Loka for the period of a Kalpa who leaves his body without attaining J\~n\=ana, though he had his Va\={\i}r\=agyam. Then he takes his birth in the family of a pure prosperous family and practising again his Yoga habits, gets My Consciousness.
38-45. O King of Mountains! This J\~n\=ana arises after many births it does not come in one birth; so one should try one's best to get this J\~n\=ana. If, attaining this rare human birth, one does not attain this J\~n\=ana, know that a great calamity has befallen to him. For this human birth is very hard to attain; and then the birth in a Br\=ahmi\d{n} family is rarer; moreover amongst the Br\=ahmi\d{n}s, the knowledge of the Veda (the Consciousness is exceedingly rare.) The attaining of the six qualities (which are considered as six wealth), restraint of passions, etc.; the success in Yoga and the acquisition of a pure real Guru, all these are very hard to be attained in this life. O Mountain! The maturity and the activities of the organs of the senses, and the purification of the body according to the Vedic rites are all very difficult to attain. Know this again that to get a desire for final liberation is acquired by the merits acquired in many births. That man's birth is entirely futile, who attaining all the above qualifications does not try his best to attain this J\~n\=ana, So one should

try one's best to acquire the J\~n\=ana. Then, at every moment, he gets the fruits of the A\'svamedha sacrifice. There is no doubt in this. As ghee (clarified butter) resides potentially in milk, so the Vij\~n\=ana Brahma resides in every body. So make the mind the churning rod and always churn with it. Then, by slow degrees, the knowledge of Brahma will be attained.
Man attains blessedness when he gets this J\~n\=ana; so the Ved\=anta says: Thus I have described to you in brief, O King of Mountains! all that you wanted to hear. Now what more do you want?
Here ends the Thirty-seventh Chapter of the Seventh Book on the glories of Bhakti in the Mah\=a Pur\=a\d{n}am, \'Sri Mad Dev\={\i} Bh\=agavatam, of 18,000 Verses, by Mahar\d{s}i Veda Vy\=asa.



