\chapter{On the description of Brahm\=a Yaj\~n\=a, Sandhy\=as, etc.}

1-25. The twice born (Br\=ahma\d{n}a) is firstly to sip three times (make \=Achamana); then to make the m\=arjana (sprinkle water) twice; he is to touch the water by the right hand and sprinkle water on his two feet. Next, he is to sprinkle with water his head, eyes, nose, ears, heart, and head thoroughly. Then speaking out the De\'sa and K\=ala (place and time) he should commence the Brahm\=a Yaj\~n\=a. Next for the destruction of all the sins and for getting liberation, he should have the Darbha (sacrificial grass, and the Ku\'sa grasses), two on his right hand, three on his left hand, one grass each on his seat, sacrificial thread, his tuft, and his heels. No sin can now remain in his body.

``I am performing this Brahm\=a Yaj\~n\=a for the satisfaction of the Devat\=a according to the S\=utra,'' thus thinking he is to repeat the G\=ayatr\={\i} thrice. Then he is to recite the following mantras :-- ``Agnim\={\i}le purohitam, etc.,'' ``Yadamgeti'' ``Agnirvai,'' ``Mah\=avratanchaiva panth\=a,'' ``Ath\=atah \'Samhit\=ay\=a\'scha vid\=amaghavat,'' ``Mah\=avratasya,'' ``\=I\d{s}etvorjetv\=a,'' ``Agna \=ay\=ahi'' ``\'Sanno Dev\={\i} rabb\={\i}staye,'' ``Tasya Sam\=amn\=ayo'' ``Briddair\=adaich'' ``\'Sik\d{s}\=am pravak\d{s}y\=ami,'' ``Pa\d{n}cha Samvatsareti,'' ``Mayarasatajabhetyeva,'' ``Gaurgm\=a,'' also he is to recite the two following S\=utras :-- ``Ath\=ato Dharma Jij\~n\=as\=a,'' ``Ath\=ato Brahm\=a Jij\~n\=asa.'' Next he is to recite the mantra ``Tachhamyoh'' and also the mantra ``Namo Br\=ahma\d{n}e namo stvagnaye namah prithivyai nama Osadh\={\i}bhyoh namah''. (These mantras are the famous mantras of the Rig Veda). Next perform the Deva-tarpa\d{n}am, thus :-- ``Om Praj\=apati stripyatu'', ``Om Brahm\=a stripyatu'', ``Om Ved\=as tripyantu,'' ``Om Ri\d{s}ayastri pyantu'', `` Om Dev\=astripantu,'' ``Om Sarva\d{n}i chhand\=amsi tripyantu'', ``Om Om K\=ara stripyatu'', ``Om Va\d{s}at K\=ara stripyatu'', ``Om Vy\=arhitayas tripyantu'', Om S\=avitr\={\i} tripyatu'', ``Om G\=ayatr\={\i} tripyatu'', ``Om Yaj\~n\=a stripyantu,'' Om Dy\=av\=a prithivyau tripyat\=am. Om antar\={\i}k\d{s}am tripyatu, Om Ahor\=atr\=ani tripyantu, Om S\=amkky\=a stripyantu, Om Siddh\=a stripyantu, Om Samudr\=a stripyantu, Om Nady\=as tripyantu, Om girayas tripyantu, Om Ksettrau\d{s} adhivana spati gandharv\=a Psarasas tripyantu, Om n\=ag\=a vay\=amsi g\=avascha s\=adhy\=a vipr\=asta thaiva cha, yak\d{s}\=a rak\d{s}\=a\d{n}si bhutan\={\i} tyeva mant\=ani tripyantu. Next, suspending the sacrificial thread from the neck, perform the \d{R}i\d{s}i tarpa\d{n}am, thus :-- Om \'Satarchinas tripyantu, Om m\=adhyam\=as tripyantu,

Om Gritsamada stripyatu, Om Vi\'sv\=amitra stripyantu, Om V\=amadeva stripyantu, Om Atri stripyatu, Om Bharadv\=ajastripyatu, Om Va\'sisthastripyatu, Om Prag\=athastripyatu, P\=avam\=anyastripyantu. Next, holding the sacrificial thread over the right shoulder and under the left arm, perform the Tarpa\d{n}am, thus :--

Om K\d{s}udras\=ukt\=a stripyantu.
Om Mah\=as\=ukt\=astripyantu.
Om Sanaka stripyatu.
Om Sananda stripyatu.
Om San\=atana stripyatu.
Om Sanat Kum\=ara stripyatu.
Om Kapila stripyatu.
Om \=Asuristripyatu,
Om Vohalistripyatu.
Om Pa\~ncha\'sikha stripyatu.
Om Sumantu Jaimini Vai\'samp\=ayana Paila S\=utra Bh\=a\d{s}ya bh\=arata Mah\=a Bh\=arata Dharm\=ach\=aryah stripyantu.
Om J\=anant\={\i}v\=aha vig\=argya Gautama \'S\=akalya v\=abhravya M\=andavya M\=and\=ukey\=a stripyantu.
Om G\=arg\={\i} V\=achak\d{n}av\={\i} tripyatu.
Om Vadav\=a pr\=atithey\={\i} tripyatu.
Om Sulabh\=a maitrey\={\i} tripyatu.
Om Kahola stripyatu.
Om Kau\d{s}\={\i}taka stripyatu.
Om Mah\=a Kau\d{s}itaka stripyatu.
Om Bh\=aradv\=aja stripyatu,
Om Paimga stripyatu.
Om Mah\=apaimga stripyatu.
Om Sujaj\~n\=a stripyatu.
Om S\=amkhy\=ayana stripyatu.
Om Aitareya stripyatu.
Om Mahaitareya stripyatu.
Om V\=askala stripyatu.
Om S\=akala stripyatu.
Om Suj\=ata vaktra stripyatu.
Om Audav\=ahi stripyatu.
Om Sauj\=ami stripyatu,
Om \'Saunaka stripyatu,
Om \=A\'sval\=ayana stripyatu.

26-54. Let all the other \=Ach\=aryas be satisfied. ``Om Ye Ke ch\=asmat kule J\=at\=a aputr\=a gotri\d{n}o mrit\=ah. te grih\d{n}antu may\=a dattam vastrani\d{s}p\={\i}di to dakam.'' Saying thus offer water squeezed out of a cloth. O N\=arada! Thus I have spoken to you of the rules of Brahm\=a Yaj\~n\=a. Whoever performs thus the Brahm\=a Yaj\~n\=a gets the fruits of studying all the Vedas. Then performing, in due order, the Vai\'sva deva, Homa, \'Sr\=addha, serving the guests, and feeding the cows, the devotee is to take his meals during the fifth part of the day along with the other Br\=ahma\d{n}as. Then the sixth and the seventh parts of the day he is to spend in reading histories and the Pur\=a\d{n}as. Then the eighth part of the day he is to devote in seeing the relatives, talking with them and receiving visits from other persons; then he will be prepared to perform the evening Sandhy\=a. O N\=arada! I am now talking of the evening Sandhy\=a. Listen. \'Sr\={\i} Bhagavat\={\i} is pleased very quickly with him who performs the evening Sandhy\=a. First make the \=Achaman and make the V\=ayu (air) in the body steady. With heart tranquilled and with the seat Baddha Padm\=asana, be calm and quiet while engaged in performing the Sandhy\=a. At the commencement of all actions prescribed in the \'Srutis and Smritis, first perform the Sagarbha Pr\=a\d{n}\=ayama. In other words recite the mantra mentally for the due number of moments and make the Pr\=an\=ayama. Simply meditating is called Agarbha Pr\=a\d{n}\=ayama. Here no mantra is necessary to be recited. Then have the Bhuta\'suddhi (have the purifications of the elements) and make the Sankalpa. First of all, the purification of elements, etc., are to be done first; one becomes, then, entitled to do other actions. While doing P\=uraka (inhaling), Kumbhaka (retaining) and Rechaka (exhaling) in Pra\d{n}\=ay\=ama, meditate on the Deity stated duly. In the evening time meditate on the Bhagavat\={\i} Sandhy\=a Dev\={\i} thus :-- The name of the then G\=ayatr\={\i} Dev\={\i} is Sarasvat\={\i}. She is old, of black colour, wearing ordinary clothes; in her hands are seen conch shell, disc, club and lotus. On Her feet the anklets are making sweet tinkling sounds; on Her loins there is the golden thread; decked with various ornaments. She is sitting on Garuda. On Her head the invaluable jewel crown is seen; on Her neck, the necklaces of stars; Her forehead is shining with a brilliant lustre emitting from the pearl and jewel T\=atamka ornaments. She has put on yellow clothes; Her nature is eternal knowledge and ever-bliss. She is uttering S\=ama Veda. She resides in the Heavens and daily She goes in the path of the Sun. I invoke the Dev\={\i} from the Solar Orb. O N\=arada! Meditate on the Dev\={\i} thus and perform the Sandhy\=a. Then perform the M\=arjanam by the mantra ``\=Apohisth\=a'' and next by the mantra

``Agni\'scha m\=a manyu\'scha.'' The remaining actions are the same as before. Next, repeat the G\=ayatr\={\i} and offer, with a pure heart, the offering of Arghya to the Sun for the satisfaction of N\=ar\=aya\d{n}a. While offering this Arghya, keep the two legs level and similar and take water in folded palms and meditating on the Devat\=a within the Solar Orb, throw it towards Him. The fool that offers S\=ury\=arghya in the water, out of ignorance, disregarding the injunctions of the \'Srutis, will have to perform Pr\=aya\'schitta for that sin. Next, worship the Sun by the S\=urya mantra. Then taking one's seat, meditate on the Dev\={\i} and repeat the G\=ayatr\={\i}. One thousand times or five hundred times the G\=ayatr\={\i} is to be repeated. The worship, etc., in the evening is the same as in the morning. Now I am speaking of the Tarpa\d{n}am in the Evening Sandhy\=a. Hear. Va\'sistha is the \d{R}i\d{s}i of the aforesaid Sarasvat\={\i}. Vi\d{s}\d{n}u in the form of Sarasvat\={\i} is the Devat\=a; G\=ayatr\={\i} is the Chhanda; its application is in the Evening Sandhy\=a Tarpa\d{n}am. Now the Tarpa\d{n}am of the Sandhy\=anga (the adjunct of Sandhy\=a) runs as follows :--

``Om Svah Puru\d{s}am Tarpay\=ami.''
``Om S\=amavedam Tarpay\=ami.''
``Om S\=uryamandalam tarpay\=ami.''
``Om Hira\d{n}yagarbham tarpay\=ami.''
``Om Param\=atm\=anam tarpay\=ami.''
``Om Sarasvat\={\i}m tarpay\=ami.''
``Om Devam\=ataram tarpay\=ami.''
``Om Samkritim tarpay\=ami.''
``Om Vriddh\=am Sandhy\=am tarpay\=ami
``Om Vi\d{s}\d{n}u r\=upin\={\i}m Usas\={\i}m tarpay\=ami.''
``Om Nirmrij\={\i}m tarpay\=ami.''
``Om Sarvasiddhi k\=ari\d{n}\={\i}m tarpay\=ami.''
``Om Sarvamantr\=a dhipatik\=am tarpayami.''
``Om Bhurbhuvah Svah Puru\d{s}am tarpay\=ami.''

Thus perform the Vaidik Tarpa\d{n}am. O N\=arada! Thus have been described the rules of the sin destroying evening Sandhy\=a. By this evening Sandhy\=a, all sorts of pains and afflictions and diseases are removed. And ultimately the Mok\d{s}a is obtained. What more than this that you should know this Sandhy\=a Banda\d{n}am as the principal thing amongst the good conduct and right ways of living. Therefore \'Sr\={\i} Bhagavat\={\i} fructifies all the desires of the Bhaktas who perform this Sandhy\=a Vanda\d{n}am.

Here ends the Twentieth Chapter of the Eleventh Book on the description of Brahm\=a Yaj\~n\=a, Sandhy\=as, etc., in the Mah\=apur\=a\d{n}am \'Sr\={\i} Mad Dev\={\i} Bh\=agavatam of 18,000 verses by Mahar\d{s}i Veda Vy\=asa.



