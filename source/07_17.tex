\chapter{On the freeing of \'Sunah\'sepha and the curing of Hari\'schandra}

1-6. Vy\=asa said :-- O King! When Vi\'sv\=amitra saw that the boy was crying very pitifully, he went to him with a merciful heart and said :-- ``O Child! I am giving you the Varu\d{n}a Mantra; recollect this within your mind and if you go on repeating that Mantra silently, you will certainly fare well.'' The sorrowful \'Sunah\'sepha, hearing thus from Vi\'sv\=amitra, began to repeat silently in his mind the above Mantra, clearly pronouncing each letter. O King! No sooner \'Sunah\'sepha repeated that Mantra than the kind-hearted Varu\d{n}a came suddenly before the boy, greatly pleased with him. Everyone in the assembly became thoroughly surprised to see Varu\d{n}a Deva come there and they all became glad and chanted hymns in honour of him. The diseased Hari\'schandra was also thoroughly surprised, fell to his feet, and with folded palms began to sing hymns to Varu\d{n}a, standing before him.

7-14. Hari\'schandra said :-- ``O Deva of the Devas! I am very vicious; my intellect is much defiled; I am a sinner before you; O Merciful One! Now show your mercy and sanctify this humble self. I was very much troubled on not having a son; so I had disregarded your words; now show your mercy on me; what offence can cling to him whose intellect is already out of order? A beggar does not see his own faults; I am also in want of a son; so I could not see my defects. O Lord! Being afraid of the terrors of hell, I have deceived

you. Those, who are sonless, cannot find rest anywhere. Especially he is barred from the Heavens. Being terrified by this dictate of \'S\=astra, I have shown disregard to your words. O Lord! You are wise and I am ignorant; especially I am extremely afflicted by this terrible disease; I am also deprived of my son; so you ought not to take any notice of my faults. O Lord! I do not know where my son has gone; O merciful One! Perhaps he, being afraid of his life, has fled away to some forest. For your satisfaction, I have now commenced your sacrifice with this purchased boy; I gave an equivalent value and I have purchased this boy. O Deva of the Devas! Your sight only has taken away my infinite troubles; now if you be pleased, I can be free of my this disease dropsy and my troubles will all be over.'' Vy\=asa said :-- O King! Hearing thus the words of that diseased King, Varu\d{n}a, the Deva of the Devas, took pity on him and thus spoke.

15-22. Varu\d{n}a said :-- ``O King! \'Sunah\'sepha is uttering hymns of praise to me; he has become very distressed; so quit him. Your sacrifice, too, is now completed; now let you be free from your present disease.'' Thus saying, Varu\d{n}a freed the King of his disease in the presence of all his councillors; the King became possessed of a beautiful body and got himself completely cured and shone bright before the assembly. Shouts of victory arose from the midst of the sacrificial ground when the Br\=ahmi\d{n} boy was freed of his bonds of rope, by the mercy of the high-souled Deva Varu\d{n}a. The King became very glad on his being recovered immediately from his disease and \'Sunah\'sepha, too, became free from his anxiety and pacified when he got himself liberated from his being immolated on the sacrificial post. Then the King Hari\'schandra completed his sacrifice with great modesty. Afterwards \'Sunah\'sepha addressed the councillors with folded palms and said :-- O Councillors! You know well the Dharma; O Speakers of truth! Kindly specify according to the dictates of the Vedas. O Omniscient ones! Whose son am I now? Who is my most respectful father? Please deliver your judgment and I will take his refuge.

23-34. When \'Sunah\'sepha spoke thus, the members of the assembly began to speak to each other, ``The boy must be of Ajigarta; whose else can he be? This boy is born of the limbs of Ajigarta; and he has nursed him according to his might. So he must be his son; whose else can he be?'' V\=ama Deva then told the people of the assembly, ``The father of the boy sold his son for money; the King purchased him. So he can be said as the son of the King; or he may be called the son of Varu\d{n}a, in as much as he freed him from his rope bondage. For, he

who nourishes another with food, who saves one from one's fear, who protects one by giving money, who bestows learning to anybody and he who gives birth to any of the above five classes of persons can be called his father.'' O King! Thus some one turned out to be in favour of Ajigarta, some other in favour of the King; but nobody came to any definite conclusion. When matters stood in this doubtful condition, the omniscient all-respected Va\'sistha Deva addressed the disputing members thus :-- ``O high-souled Ones! Kindly hear what the \'Srutis say on this point. When the father has cut off his filial attachment and has sold his son, his fatherly connection has ceased then. No doubt this boy was purchased by the King Hari\'schandra. But when the King fastened him to the sacrificial post, he cannot be called as the father. Again when this boy singing hymns in honour of Varu\d{n}a, he being glad freed him of his bondage, so Varu\d{n}a cannot be called his father. For whoever praises a god by the great Mantra, that Deva becomes pleased with him and gives him wealth, life, cattle, kingdom and even final emancipation. Rather Vi\'sv\=amitra saved the boy by giving him in his critical moment the powerful great Mantra of Varu\d{n}a; hence the boy can be called as the son of Vi\'sv\=amitra and of none else.''

35-40. Vy\=asa said :-- O King! Hearing the words of Va\'sistha, all the members of the assembly gave their unanimous consent and Vi\'sv\=amitra with his heart filled with love, exclaimed, ``O Son! Come to my house.'' And caught hold of his right hand. \'Sunah\'sepha, too, accompanied him and went away. Varu\d{n}a also went to his own abode with a gladdened heart. The councillors, too, departed. Freed from his disease, the King gladly began to govern his subjects. At this time his son Rohit\=a heard all about Varu\d{n}a and became very glad and leaving the impassable forest passes and mountains, returned home. The messengers informed the King of the arrival of the prince; the King heard and his heart overflowed with love and he gladly came there with no delay.

41-48. Seeing the father coming, Rohit\=a\'sva became filled with love and overpowered with sorrow for long separation began to shed tears and fell prostrate at his feet. The King raised him up and embraced him gladly and smelling his head enquired of his welfare. When the King was thus asking his son, taking him on his lap, the hot tears of joy flowed from his eyes and fell on the head of the prince. The King and the prince then began to govern together his kingdom. The King described in detail all the events of the sacrifice where human victims are immolated. He started next the R\=ajas\=uya sacrifice, the best of all sacrifices, and duly worshipping the Muni Va\'sistha, made him the

Hot\=a in that sacrifice. When this grand sacrifice was finished, the King respected the Muni Va\'sistha with abundant wealth. Once, on a time, the Muni Va\'sistha went gladly to the romantic Heaven of Indra; and Vi\'sv\=amitra, too, went there also and both the Munis then met with each other. The two Mahar\d{s}is took their seats in that Heaven. But Vi\'sv\=amitra was astonished to see Va\'sistha greatly respected in Indra's hall of assembly and asked him, thus :--

49. ``O Muni! Where have you received this great honour and worship? O Highly Fortunate One! Who has worshipped you thus? Speak out truly.''

50-53. Va\'sistha said :-- ``O Muni! There is a King named Hari\'schandra; he is very powerful and my client; that King performed the great R\=ajas\=uya sacrifice with abundant Dak\d{s}i\d{n}\=as. There is no other King truthful like him; he is virtuous, charitable, and ever ready in governing his subjects. O Son of Kau\'sika! I have got my worship and honour in his sacrifice. O best of Dv\={\i}jas! Are you telling me to speak truly? Again I speak truly to you that there never was a King truthful, heroic, charitable, and very religious like him nor there will be such a one.''

54. Vy\=asa said :-- O King! Hearing such words, the Vi\'sv\=amitra, of a very angry temper, spoke to him with his reddened eyes :--

55-59. ``O Va\'sistha! Hari\'schandra obtained a boon from Varu\d{n}a when he made a certain promise; then he cheated Varu\d{n}a with deceitful words. So he is a liar and cheat. Why are you praising then that King? O Intelligent One! Let us now stake all our virtues that we have earned since our birth by our asceticism and studies. You have praised exceedingly that King who is a great cheat; but if I cannot prove him to be a liar of the first order, I will lose all my virtues from my birth; but if it be otherwise, then all your virtues will be destroyed.'' Thus the two Munis quarrelled with each other and making this stake, departed from the Heavens and went to their respective \=A\'sramas.

Here ends the Seventeenth Chapter in the Seventh Book on the freeing of \'Sunah\'sepha and the curing of Hari\'schandra in the Mah\=apur\=a\d{n}am \'Sr\={\i} Mad Dev\={\i} Bh\=agavatam, of 18,000 verses by Mahar\d{s}i Veda Vy\=asa.



