\chapter{On Sarasvat\={\i} stotra by Y\=aj\~navalkya}

1-5 N\=ar\=aya\d{n}a said :-- O N\=arada! I now describe the Stotra (hymn) of Sarasvat\={\i} Dev\={\i}, yielding, all desires that Y\=aj\~navalkya, the best of the \d{R}i\d{s}is recited in days of yore to Her. The Muni Yaj\~navalkya forgot all the Vedas out of the curse of Guru and with a very sad heart went to the Sun, the great merit-giving place. There he practised austerities for a time when the Lol\=akhya Sun became visible to him, when, being overpowered by great sorrow, he began to cry repeatedly; and then he sang hymns to him. Then Bhagav\=an S\=urya Deva became pleased and taught him all the Vedas with their Amgas (limbs) and said :-- ``O Child! Now sing hymns to Sarasvat\={\i} Dev\={\i} that you get back your memory.'' Thus saying, the Sun disappeared. The Muni Y\=aj\~navalkya finished his bath and with his heart full of devotion began to sing hymns to the V\=ag Dev\={\i}, the Goddess of Speech.

6-32. Y\=aj\~navalkya said :-- ``Mother! Have mercy on me. By Guru's curse, my memory is lost; I am now void of learning and have become powerless; my sorrow knows no bounds. Give me knowledge, learning, memory, power to impart knowledge to disciples, power to compose books, and also good disciples endowed with genius and Pratibh\=a (ready wit). So that in the council of good and learned men my intelligence and power of argument and judgment be fully known. Whatever I lost by my bad luck, let all that come back to my heart and be renewed as if the sprouts come again out of the heaps of ashes. O Mother! Thou art of the nature of Brahm\=a, superior to all; Thou art of the nature of Light, Eternal; Thou art the presiding Deity of all the branches of learning. So I bow down again and again to Thee. O Mother! The letters Anusv\=ara, Vi\d{s}arga and Chandravindu that are affixed, Thou art those letters. So obeisance to Thee! O Mother! Thou art the exposition (Vy\=akhy\=a) of the \'S\=astras; Thou art the

presiding Deity of all the expositions and annotations. Without Thee no mathematician can count anything. So Thou art the numbers to count time; Thou art the \'Sakti by which Siddh\=antas (definite conclusions) are arrived at; Thus Thou dost remove the errors of men. So again and again obeisance to Thee. O Mother! Thou art the \'Sakti, memory, knowledge, intelligence, Pratibh\=a, and imagination (Kalpan\=a). So I bow down again and again to Thee. Sanatkum\=ara fell into error and asked Brahm\=a for solution. He became unable to solve the difficulties and remained speechless like a dumb person. Then \'Sr\={\i} Kri\d{s}\d{n}a, the Highest Self arriving there, said :-- O Praj\=apati! Better praise and sing hymns to the Goddess of speech; then your desires will be fulfilled. Then the four-faced Brahm\=a advised by the Lord, praised the Dev\={\i} Sarasvat\={\i}; and, by Her grace, arrived at a very nice Siddh\=anta (conclusion). One day the goddess Earth questioned one doubt of Her to Ananta Deva, when He being unable to answer, remained silent like a dumb person. At last He became afraid; and advised by Ka\'syapa, praised Thee when He resolved the doubt and came to a definite conclusion. Veda Vy\=asa once went to V\=alm\={\i}ki and asked him about some S\=utras of the Pur\=a\d{n}as when the Muni V\=alm\={\i}ki got confounded and remembered Thee, the Mother of the world. When by Thy Grace, the Light flashed within him and his error vanished. Thereby he became able to solve the question. Then Vy\=asadeva, born of the parts of \'Sr\={\i} Kri\d{s}\d{n}a, heard about the Pur\=a\d{n}a S\=utras from V\=alm\={\i}ki's mouth and came to know about Thy glory. He then went to Pu\d{s}kara T\={\i}rtha and became engaged in worshipping Thee, the Giver of Peace, for one hundred years. Then Thou didst become pleased and grant him the boon when he ascended to the rank of the Kav\={\i}ndra (Indra amongst the poets). He then made the classification of the Vedas and composed the eighteen Pur\=a\d{n}as. When Sad\=a \'Siva was questioned on some spiritual knowledge by Mahendra, He thought of Thee for a moment and then answered. Once Indra asked Brihaspati, the Guru of the Devas, about \'Sabda \'S\=astra (Scriptures on sound). He became unable to give any answer. So he went to Pu\d{s}kara T\={\i}rtha and worshipped Thee for a thousand years according to the Deva Measure and he became afterwards able to give instructions on \'Sabda \'S\=astra for one thousand divine years to Mahendra. O Sure\'svar\={\i}! Those Munis that give education to their disciples or those that commence their own studies remember Thee before they commence their works respectively. The Mun\={\i}ndras, Manus, men, Daityendras, and Immortals, Brahm\=a, Vi\d{s}\d{n}u and Mahe\'sa all worship Thee and Sing hymns to Thee. Vi\d{s}\d{n}u ultimately becomes inert when He goes on praising Thee by His thousand mouths. So Mah\=a Deva becomes when

he praises by His five mouths; and so Brahm\=a by His four mouths. When great personages so desist, then what to speak of me, who is an ordinary mortal having one mouth only!'' Thus saying, the Mahar\d{s}i Y\=aj\~navalkya, who had observed fasting, bowed down to the Dev\={\i} Sarasvat\={\i} with great devotion and began to cry frequently. Then the Mah\=am\=ay\=a Sarasvat\={\i}, of the nature of Light could not hide Herself away. She became visible to him and said ``O Child! You be good Kav\={\i}ndra (Indra of the poets).'' Granting him this boon, She went to Vaikuntha. He becomes a good poet, eloquent, and intelligent like Brihaspati who reads this stotra of Sarasvat\={\i} by Y\=aj\~navalkya. Even if a great illiterate reads this Sarasvat\={\i} stotra for one year, he becomes easily a good Pundit, intelligent, and a good poet.

Here ends the Fifth Chapter of the Ninth Book on Sarasvat\={\i} stotra by Y\=aj\~navalkya in \'Sr\={\i} Mad Dev\={\i} Bh\=agavatam of 18,000 verses by Mahar\d{s}i Veda Vy\=asa.



