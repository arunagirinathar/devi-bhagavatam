\chapter{On the incidents preliminary to the Haihaya and Bh\=argava affairs}

1-5. Janamejaya said :-- In whose family were born those K\d{s}attriya Haihayas that killed in ancient times the Bh\=argavas, disregardless of the sin incurred in killing a Br\=ahmi\d{n}? O Grandsire! Never do the good persons become angry without a serious cause; therefore kindly state why they got angry. How was the enmity caused between them and the priests? As far as I can think, the cause is not so simple a one as led to this enmity between the K\d{s}attriyas and the priests. Otherwise why then would they slay the offenseless Br\=ahmi\d{n}s, fit to be worshipped; and how was it that the K\d{s}attriyas, though they were so very powerful, did not fear to commit a sin. O Muni! Can any K\d{s}attriya Chief kill a Br\=ahmi\d{n}, worthy of the highest respect, merely on a trifling cause! Describe to me, then, how this happened. A great doubt has thus arisen in my mind.

6. S\=uta said :-- O \d{R}i\d{s}is! Vy\=asa, the son of Satyavat\={\i}, became very pleased when he was asked this question by Janamejaya, and, recollecting the whole course of events regarding the Haihayas, began to narrate it.

7-22. Vy\=asa said :-- O son of Pariksit! I will now narrate that wonderful story of old that I know fully; now hear this very attentively. In ancient times there was a King named K\=artav\={\i}ry\=arjuna of the family of Haihaya. He was of thousand hands, powerful, and always ready to observe religious duties. He was the incarnation of Hari, and the disciple of Mahar\d{s}i Datt\=atreya and the worshipper of the Supreme Force (\=Ady\=a \'Sakti). He was well known as a perfect adept in the Yoga practices and of a very charitable disposition. But this King was the client of the Br\=ahmi\d{n}s of the Bh\=argava clan. He was always devoted to performing sacrifices, exceedingly religious, and always engaged in making gifts. So many a time did he perform the great sacrifices and gave a profuse quantity of wealth to the Bh\=argavas. Due to the gifts and presents of K\=arta V\={\i}rya, the Bh\=argava priests became possessed of many horses, and gems and jewels and so became wealthy and prosperous on the surface of this earth. O King! When K\=artav\={\i}ry\=arjuna, the best of Kings, left the mortal coil and got up to Heavens, his descendants became entirely void of any wealth

by the indomitable influence of Time. Now, on a certain occasion, the Haihayas had to perform certain actions which necessitated a vast sum of money; they came to the Bh\=argavas and humbly prayed for a very large amount of wealth. But the Br\=ahmi\d{n}s, out of their greed of money, replied they had no money and thus they did not give any money whatsoever. Rather the Bh\=argavas thought that the Haihayas would perforce take their wealth, and, fearing thus, some of them buried all their valuables underneath the ground; and others gave as charities to the Br\=ahma\d{n}as. The greedy Bh\=argavas, bewildered with fear, thus transferred all their properties elsewhere, quitted their homes and fled away to mountains and other places. The greedy Br\=ahmi\d{n}s did not give any wealth to their Yajam\=anas (their clients) though they saw them very much distressed; but they fled away out of fear to mountains and fastnesses where they found shelter. At last the Haihayas, the best of the K\d{s}attriyas, became very sorry till, at last, for the sake of their good actions, they went to the Bh\=argava\'s houses for the sake of money and found they had quitted their homes and fled away; their homes were all vacant. Then they began to dig underneath their houses for money and some got the money thus. Then the K\d{s}attriyas began to labour hard and got hordes of money from underneath the ground. Next they raided upon other Br\=ahma\d{n}a\'s houses and dug and excavated and searched for more money. The Br\=ahmi\d{n}s were helpless and, crying, all took their refuge, out of fear, under the Bh\=argavas.

23-42. The K\d{s}attriyas made an exhaustive search of the Br\=ahma\d{n}a\'s houses and got lots of money. They then charged the Br\=ahma\d{n}as as having had spoken falsehood and they became very angry, and killed the Br\=ahma\d{n}as with arrows who took their refuge. O King! The Haihayas were so very angry at that time that they went wherever the Bh\=argavas took their shelter and cut asunder the foetus in the wombs of their Bh\=argava\'s wives and thus they roamed all over on the surface of the earth. Wherever they saw any Bh\=argava, be he a minor, or a youth or a old man, at once they killed him with sharp arrows, disregarding the sin Brahmahatty\=a. When the Bh\=argavas were thus all killed, then they caught hold of their wives that were pregnant and destroyed their wombs. When the vicious K\d{s}attriyas thus destroyed the lives in their wombs, the helpless women began to cry like the awe-stricken ewe. Then the other Munis, the inhabitants of the sacred places of pilgrimages, seeing the Haihaya K\d{s}attriyas inflamed with anger, said :-- ``O K\d{s}attriyas! Quit your terrible anger towards the Br\=ahmi\d{n}s. Being the best of the K\d{s}attriyas, you are killing the foetus in the wombs of the pregnant Br\=ahma\d{n}a ladies! You are doing, no doubt, a very vicious and unjustifiable act! You should

know that an act, very bad or very good, bears fruit in this life; therefore those that seek their welfare should entirely omit this exceedingly hateful and vicious act.'' Then the exceedingly angry Haihayas told the merciful ascetics :-- You all are saints; therefore you do not know the real import of what are called vicious acts. Those Bh\=argavas, thoroughly dexterous in cunning pursuits, deceived our largehearted forefathers and stole away all their gold and jewels, as thieves do with a passerby on a road. These Bh\=argavas are cheats, vain persons and their persuasions are like herons. A great act had to be done by us and we wanted money at 25 per cent interest with all the becoming humility; yet they did not give us the money; rather seeing on their face their clients distressed and sorrowful they spoke that they had no money, no money and then they remained silent. True, they got all their money from K\=artav\={\i}rya; but it may be questioned why they stored it? Why did not they perform sacrifices with that? Why did not they give sufficient money to the other priests (Y\=ayakas) that did the sacrifices. Never should any Br\=ahmi\d{n} hoard his money; he should give that in charity and enjoy at his pleasure. O Twice-born! In amassing wealth, there exist three fears :-- Fear from the thieves and robbers, fear from the King, fear from dreadful fire accidents, and especially great terrible fear from the cheats. This is the nature of wealth; it leaves its preserver. See, moreover, when a hoarder of money dies, he certainly has to quit it. If a wealthy man, before dying, performs sacrifices and other good pious acts by his earned money, then he gets certainly good states in future; otherwise, he quits his wealth, to no purpose and earns a bad state in his future life; there is no doubt in this. We humbly wanted to pay a quarter interest and asked money for the performance of a great act; yet they, the greedy ones, were doubtful about our promise; and though our priests, they did not give us the money. O Mahar\d{s}is! Gift, enjoyment and destruction, these are the three courses which any wealth has to pass through; those persons that have done good deeds, enjoy their wealth and give as charities and thus they make a good and real use of their money; and of those that are vicious, their wealth goes away in ruin and to no purpose. He who does not enjoy nor give in charities but is only clever in hoarding and who is a miser, the Kings punish him by all means, that man who cheats himself and who suffers only pains and miseries. For that reason, we are now ready to kill those Br\=ahmi\d{n}s, the vilest of men, the cheats, though they are our Gurus. O Mahar\d{s}is! You are great persons; therefore you do not be angry after you have come to know all these.

43-51. Vy\=asa said :-- Thus consoling the Munis, with reasonable words, the Haihayas began to roam about, in search of the wives of the

Bh\=argavas. The Bh\=argava wives were very much distressed with fear and became very lean and thin. They fled away to the Him\=alay\=an Mountain weeping, and crying, and trembling with fear. Thus the Bh\=argavas were being killed by those vicious greedy Haihayas, infuriated with anger, and as they liked. O King! This greed is the greatest enemy of a man, residing in his own body; this greed is the root of all evils, of all sins. Life is in danger due to this covetousness. It is due to this greed that quarrels ensue amongst the several castes, the Br\=ahmi\d{n}s, etc., and that the human beings are very much troubled with thirst after worldly enjoyments. This greed makes a man forsake all his religious rites and long existing customs and observances of his family; and it is due to this avarice of gold that men kill their fathers, mothers, brothers, friends, Gurus, sons, acquaintances, sisters, and sisters-in-law and others. Really when a man is bent on avarice, nothing heinous remains to him that cannot be done by him. This greed is a more powerful enemy than anger, lust and egoism. O King! Men abandon their lives for their greed; what more can be said than this? So one should be always alert on this. O King! Your forefathers, the P\=andavas and Kauravas, were all religious and they followed the path of virtue and goodness. Yet they all were ruined simply for this greed. See! The dreadful fight and separation amongst the relatives took place where there were the high-souled persons like Bh\={\i}\d{s}ma, Dro\d{n}a, Krip\=ach\=arya, Kar\d{n}a, Vahlika, Bh\={\i}masena, Yudhisthira, Arju\d{n}a, and Ke\'sava, only through the avaricious feelings. In this battle Bh\={\i}\d{s}ma, Dro\d{n}a and the sons of P\=andavas were all slain; the brothers and fathers were all slain in battle. Thus what improper acts and mischiefs can there be that cannot be committed when the human minds are overpowered by this greed? O King! The vicious Haihayas slew the Bh\=argavas all through this avarice.

Here ends the Sixteenth Chapter in the Sixth Book on the incidents preliminary to the Haihaya and Bh\=argava affairs in the Mah\=apur\=a\d{n}am \'Sr\={\i} Mad Dev\={\i} Bh\=agavatam by Mahar\d{s}i Veda Vy\=asa.



