\chapter{On Indra's getting the fruits of Brahmahatty\=a and on the downfall of king Nahu\d{s}a}

1-2. Vy\=asa said :-- Indra was quite surprised to see in this state of solitude his dear wife \'Sach\={\i}, large-eyed and overwhelmed with much sorrow and spoke thus :-- ``O Beloved! I am remaining here alone this desolate place unnoticed by all the J\={\i}vas; O Auspicious faced One! How have you come to know this! And how is it that you have come here!''

3-5. \'Sach\={\i} said :-- O Lord of the Devas! I have been able to know this place where you are staying by the grace of Bhagavat\={\i}'s Feet and I will get you back by Her grace. The Devas and Munis all united have installed the King Nahu\d{s}a in your throne. That fellow says ``O fair One! I am now made the King Indra; therefore you worship me as your husband.'' And thus oppresses me always. O Destroyer of other's strength! That vicious one speaks to me thus; I am weak; What can I do to him?

6. Indra said :-- ``O Beautiful One! I am now here waiting for the proper opportunity; O auspicious One! You should also make your mind calm and remain there, and wait for the proper time.''

7-12. Vy\=asa said : -- O King! After Indra had spoken thus, \'Sach\={\i} Dev\={\i} became sorrowful, drew a deep sigh and, trembling, said :-- ``O Fortunate One! How can I stay there? That vicious man, puffed up with vanity and proud of his position will forcibly bring me under his control. The Devas and Munis say this to me out of his fear :-- O Beautiful One! The Lord of the Devas is now very much distressed with the arrows of the Cupid; therefore go and worship him. O Tormentor of foes! How can the Br\=ahmi\d{n} Brihaspat\={\i} protect me, being himself powerless and under the control of the Devas. O Lord! This is now my grave anxiety; I am a weak woman, having none to protect me and therefore always under the guidance of a man. Fate is now against me; how can I keep myself religious? I am a chaste woman, devoted to my husband; I have got no shelter there; who will protect me when I fall into misery!

13-21. Indra said :-- ``O Beautiful One! I will now tell you one means which, if you adopt, will no doubt preserve your character in times of crisis. Women cannot preserve their chastity when they are protected by others by thousand and one means; for lust penetrates into their restless minds and carries them to impure ways. It is the good and pure character that preserves a woman from a vicious course; therefore O Smiling One! You adopt this good conduct and character and remain steady in your place. In case that deceitful wicked King Nahu\d{s}a shows his violence upon you, then take time and secretly cheat him, O Mad\=alase! Go to him when there will be no other body present and say :--``O Lord of the world! Please come to me on a conveyance carried by the \d{R}i\d{s}is (great ascetics), I will then be very delighted and gladly yield myself to you; this is certainly my vow.'' O Beautiful One! When you will say thus, that King, blinded by passion, will engage the Munis for the carriers of his conveyance. The ascetics, then, will be angry and curse him; the

Munis will certainly burn him by the fire of their wrath; and the Divine Mother will no doubt help you. He who remembers the lotus-feet of the Ambik\=a Dev\={\i} never meets with any difficulties; and if there arises any difficulty, know certainly that it is for his immeasurable benefit. Therefore worship, with your whole heart, the Mother of the Universe, Who resides in the jewelled island (Mani Dv\={\i}pa) according to the words of the Guru Brihaspat\={\i}.

22-25. Vy\=asa said :-- O King! Hearing thus the Indra's words, \'Sach\={\i} Dev\={\i} said ``Let it be so'' and went to Nahu\d{s}a, filled with confidence and inspiration to carry on the future work. Nahu\d{s}a was very glad to see \'Sach\={\i} Dev\={\i} and spoke thus :-- ``O Sweet-speaking One! Are you all right? I am now completely yours; you have fulfilled my word; therefore I say truly that I am your servant. O Gentle-speaking One! When you have come to me, know that I am very glad. O Smiling One! Do not feel any shame before me. I am now your devotee; worship me. O large eyed One! Speak out what is that dear thing that I can do for you? I will carry that out at once.''

26-27. \'Sach\={\i} said :-- ``O Lord V\=asava! You have done all the works; now I have got one desire to ask from you, kindly fulfil this and then I will be yours. O One full of auspiciousness! Now fulfil my desire; I am speaking this to you.''

28. Nahu\d{s}a said :-- ``O Thou, having a face sweet like the Moon! Speak out your desire; I will carry it out. O Beautiful One with nice eye-brows! Even if that be unattainable, I will give that to you.''

29-31. \'Sach\={\i} said :-- ``O King of Kings! I cannot trust you; Swear on oath that you will fulfill my desire. O King! A truthful King very rare on this earth; I will speak out my desire when I will be convinced that you are bound by truth. O King! When you will fulfil my desire, I will always remain under your control; this I speak truly to you.''

32. Nahu\d{s}a said :-- ``O Beautiful One! On all the sacrifices and gifts that I have ever made, I swear, on all my merits, that I will certainly carry out your word.''

33-37. \'Sach\={\i} said :-- ``Indra has got for his vehicles the horse Uchchai\'srav\=a, the Air\=avata elephant and the chariot; V\=asudeva has got vehicle Garuda; Yama has got his buffalo; \'Sankara his Bull; Brahm\=a his Swan; Kartika has got his peacock and Gane\'sa has got his mouse. But now, O Lord of the Devas! I want to see your vehicle, never witnessed before: I want to see the Munis and the great ascetics, observing vows, to be your vehicle; this is not Vi\d{s}\d{n}u's, Rudra's nor of the Devas, and R\=ak\'sasas. O King! Let the Munis carry your conveyance, this is my ever burning desire. O King of this earth! I know you the highest of all Devas; let your glory and splendour increase ever and ever; this is the intense desire reigning in my heart.''

38-56. Vyasa said :-- O King! Hearing \'Sach\={\i}'s words, Nahu\d{s}a, weak in intellect, laughed and beguiled by the M\=ay\=a of Bhagavat\={\i} began to praise her and said at once :-- ``O Beautiful One! Truly you have made a nice suggestion of my vehicle. O One having luxuriant hairs! I will soon carry out your words. O Sweet-smiling One! Whoever is effeminate and of weak virility, he is never able to engage the Munis as his carriers; no doubt, my unbounded strength will be rendered manifest when I come to you on a vehicle carried by the Munis. What wonder is there that the seven \d{R}i\d{s}is (the seven stars of the constellation Great Bear) and all the Devar\d{s}is would carry me, knowing me as the most capable and superior in all the three worlds by virtue of my sheer asceticism?'' Vy\=asa said :-- O King! The King Nahu\d{s}a became very pleased and dismissed \'Sach\={\i} Dev\={\i}. He, then, with a heart flamed with passion, called the Munis and said: --``O Munis! I am now become Indra and endowed with all powers thereof; therefore you all do my work without being at all surprised. I have got the seat of Indra but Indr\=an\={\i} is not coming to me. I called her to my presence and when I informed her of my desire, She had spoken to me with affection the following words:-- O Indra of the Devas! O Giver of one's honour! Better come to me on a vehicle carried by the Munis and do thus the one thing for me that I like. O Mahar\d{s}is! To carry out this task is, indeed, difficult for me; therefore do you all unite and, out of mercy, do this for me in all its completeness. My heart is being always burnt, as I am very much attached to the wife of Indra; so I take refuge unto you to do this wonderful work for me.'' Though this request was very indecent and greatly humiliating yet the Munis agreed to it, out of pity, and also impelled, as it were, by the great Fate. When the Great Seers, the Munis consented to this proposal, the King, whose heart was very much attached to the daughter of Pulom\=a, became very glad and getting on the beautiful vehicle carried by the Munis, told them, move on quickly (Sarpa, Sarpa). Then the King Nahu\d{s}a, getting very much impassioned, touched with his feet the heads of the Munis, and, being as it were smitten by the arrows of cupid, began to whip frequently the \d{R}i\d{s}i Agastya, the best of the ascetics, who devoured the R\=ak\d{s}asa V\=at\=api, who was the husband of Lop\=amudr\=a and who drank out the ocean, saying move on, move on (Sarpa, Sarpa another meaning of which is Snake). The Muni, then, became very angry, on being thus whipped, and cursed him saying :-- ``'O Wicked One! As you are whipping me, saying Sarpa, Sarpa, so go and remain in the dense forest as a huge snake. When many years will elapse and when you will crawl on your own limb and suffer intense troubles, after that you will again come to heaven. You will be free from the curse when you will see the King Yudhisthira and hear from him the answers to several questions.''

57-67. Vy\=asa said :-- O King! Thus cursed, the King Nahu\d{s}a began to chant hymns to that best of the Munis, and, while praising, fell from the Heavens and instantly turned into a snake. Brihaspat\={\i}, then, quickly went to the M\=anasarovara Lake and informed Indra everything in detail. Indra became very glad on hearing in detail of the downfall of the King Nahu\d{s}a from Heaven and remained there gladly. When the Devas and Munis saw this downfall into the earth of Nahu\d{s}a, they all went to the Lake M\=anasarovara where Indra was staying. They then all encouraged Indra and honoured him by bringing him back to the Heavens. All the Devas and \d{R}i\d{s}is installed Indra on the throne and then performed the Inauguration ceremony of the all auspicious Dev\={\i}. On getting back his own throne. Indra, too, began to sport in the beautiful Nandana Garden with his dear consort \'Sach\={\i}, in the home of the Devas. Vy\=asa said :-- O King! Indra had to suffer such severe hardships on account of his slaying the Mahar\d{s}i Vi\'svar\=upa, the Lord of the Asuras. Subsequently through the grace of the Dev\={\i}, he got back his own seat. O King! Thus I have narrated before you to my best, this excellent story of the killing of Vritr\=asura and thus have answered your question. O Ornament of the Kuru family! The fruits will be exactly according to the Karma done. The effects of the Karma done must be borne whether they be auspicious or inauspicious. (So Indra had to suffer for his Karma, the killing of a Br\=ahma\d{n}a.)

Here ends the Ninth Chapter of the Sixth Book on Indra's getting the fruits of his killing a Br\=ahma\d{n}a and on the downfall of the King Nahu\d{s}a from the Heavens in the Mah\=apur\=a\d{n}am \'Sr\={\i} Mad Dev\={\i} Bh\=agavatam of 18,000 verses by Mahar\d{s}i Veda Vy\=asa.



