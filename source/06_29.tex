\chapter{On the N\=arada's getting again his male form}

1-11. N\=arada said :-- O Dvaip\=ayana! When the King T\=aladhvaja asked me thus, I thought over earnestly and said thus :-- ``I do not know whose daughter I am; nor do I know quite certainly where are my father and mother; one man placed me here on this tank and has gone away, whither I do not know. O King! I am now an helpless orphan; what shall I do now? Where to go? What to do by which I can have my welfare? I am all the while thinking on these. O King! The Destiny is powerful; I have not the least control over it; you know Dharma and you are a King. Do now as you like. O King! Do nourish me; I have no father, no mother, nor any acquaintances and friends; there is no place for me also to stand on; therefore I am now your dependent.'' When I spoke thus, the King looked at my face and became love-stricken for me; he then told his attendants to bring an excellent rectangular and spacious palanquin to be carried on four men's shoulders, gilt and adorned with jewels and pearls, where soft sheets were spread inside and covered all over with silken cloths. Instantly the servants went away and brought for me a beautiful palanquin. I got on it to serve the best wishes of the King. The King also gladly took me home. In an auspicious day and in an auspicious moment he married me in accordance with due rites and ceremonies in the presence of the Holy Fire.

12. I became dearer to him than even his own life and the King, with great fondness, kept my name as Saubh\=agya Sundar\={\i}.

13-20. The King then began to sport with me amorously according to the rules of the K\=ama \'S\=astra in various ways and with great enjoyments and pleasures. He then left all his kingly duties and state affairs and he began to remain day and night with me deeply immersed in amorous sports;

so much his mind was merged in me in these plays that he could not notice the long time that passed away in the interval. He used to drink the V\=aru\d{n}\={\i} wine and, forsaking all the state affairs, began to enjoy me in nice gardens, beautiful lakes, lovely palaces, beautified houses, excellent mountains and enviable forests and became completely subservient to me. O Dvaip\=ayana! Being incessantly engaged with the King in amorous sports and remaining obedient to him, my previous body, male ideas, or the birth of Muni, nothing whatsoever came in my memory. I remained always attached to him, being obedient to him with a view to be happy and I constantly thought over ``that this King is very much attached to me, I am his dearest wife to all others; always he thinks of me, I am his chief consort, capable to give him enjoyment.'' My mind became entirely his and I completely forgot the eternal Brahmaj\~n\=an and the knowledge of the Dharma \'S\=astras.

21-31. O Muni! Thus engaged in various amorous sports, twelve years passed away as if a moment and I could not perceive that. Then I became pregnant; and the King became very glad and performed all the ceremonies pertaining to my impregnation and holding of the child in my womb. In order to satisfy me, the King used to ask me always what things I liked; I used to be very much abashed; seeing this, the King used to be still more glad. Ten months thus passed away and in an auspicious Lagna and when the asterism was favourably strong, I gave birth to a son; the King became very glad and great festivities were held on the birth ceremony of the child. O Dvaip\=ayana! When the period of the birth-impurity was over, the King saw the face of the child and was greatly delighted; I then became the dearest wife of the King. Two years after again I became impregnated; the second auspicious son was born. The King gave the name Sudhanv\=a to the second son and on the authority of the Br\=ahmi\d{n}s, kept the name of the eldest son as V\={\i}ravarm\=a. Thus I gave birth to twelve sons, in due course of time, to the King's great liking; and I was engaged in rearing up those children and thus I remained enchanted. Again in due course, I gave birth to eight sons; thus my household was filled with happiness. The King performed the marriage ceremonies of all those children duly and befittingly; and our family became very large with sons and their wives.

32-52. Then I had some grandsons and they increased my attachment and the consequent delusion with their all sorts of playful sports. Sometimes I felt happy and prosperous and sometimes I felt pain and sorrow when my sons fell ill. Then my body and mind became very much troubled with sorrows. Again the quarrels amongst my sons and my daughters-in-

law, brought terrible pain and remorse in my mind. O Best of Munis! Thus I was greatly immersed in the terrible ocean of these imaginary thoughts, sometimes happy and sometimes painful, and I forgot my previous knowledge and the knowledge of the \'S\=astras. I was merged in the thought of myself being a woman and lost myself entirely in doing the household affairs. I began to think ``that I have so many daughters-in-law; so many powerful sons of mine are playing together in my house; Oh! I am fortunate and full of merits amongst women'' and thus my egoistic pride increased. Not for a moment even occurred the thought that I had been N\=arada; the Bhagav\=an had deceived me by His M\=ay\=a. O Kri\d{s}\d{n}a Dvaip\=ayana! I was deluded by M\=ay\=a and passed away my time in the thought ``that I am the king's wife, chaste and of good conduct following good \=Ach\=ara; I have so many sons and grandsons; I am blessed in this Sams\=ara and that I am so happy and prosperous.'' One powerful king of a distant country turned out an inveterate enemy of my husband and came to the city of Kanauj to fight with my husband, accompanied by chariots, and elephants and the fourfold army. That enemy besieged the city with his army; my sons and grandsons went out and fought valiantly with him but owing to the great Destiny, the enemies killed all my sons. The King retreated and returned to his palace. Next I heard that powerful King killed all my sons and grandsons and had gone back to his country with his army. I then hurriedly went to the battlefield, crying loudly. O Long-lived One! Seeing my sons and grandsons lying on the ground, in that horrible and distressed state, I became merged in the ocean of sorrows and lamented and wept loudly and wildly, ``O my Sons! Where have you gone leaving me thus? Alas! The pernicious Fate is very dominant, and very painsgiving and indomitable. It has killed me today.'' By this time, the Bhagav\=an Madhus\=udana came to me there in the garb of a beautiful aged Br\=ahmi\d{n}. His dress was sacred and lovely; it seemed he was versed in the Vedas. Seeing me weeping distressedly in the battlefield he said :-- ``O Dev\={\i}! O cuckoo-voiced One! It seems you are the mistress of a prosperous house and you have got husband and sons! O thin-bodied One! Why are you thus lamenting and feeling yourself distressed! All this is simply illusion caused by Moha; think; who are you? whose sons are these? Now think of your best hereafter; Don't weep, get up and be comfortable, O Good-eyed one!

53-54. O Dev\={\i}! To shew respect to your sons, etc., gone to the other worlds, offer them water and Til. The friends of the deceased ought to take their bath in a place of pilgrimage; never they should bathe in their houses. Know this as ordained by Dharma.

55-66. N\=arada said :-- O Dvaip\=ayan! When the old Br\=ahmi\d{n} thus addressed me, I and the King and other friends got up. The Bhagav\=an Madhus\=udana causing this creation, in the form of a Br\=ahma\d{n}a, led the way and I followed him quickly to that sacred place of pilgrimage. The Vi\d{s}\d{n}u Bhagav\=an, the Lord Jan\=ardana Hari, in the form of a Br\=ahmi\d{n}, kindly took me to the tank named Pumt\={\i}rtha (male t\={\i}rtha) and said :-- ``O One going like an elephant! Better take your bath in this tank; forego your sorrows that are of no use; now the time has arrived to offer water to your sons. Better think that you had millions of sons born to you in your previous births and for that your millions of sons and daughters lost their lives; you had millions of fathers, husbands, and brothers and you lost them again; O Dev\={\i}! Now tell me for whom you will now grieve? All these, then, are merely mental phenomena; this world is full of delusion, false like a mirage and dream-like; the embodied souls, simply get pains and sorrows and nothing else.'' N\=arada said :-- On hearing his words, I went to bathe in that Pumt\={\i}rtha, as ordered by him. Taking a dip, I found that, in an instant, I became a man; the Bhagav\=an Hari, in his own proper form, was standing on the edge with a lute in his hand. O Br\=ahmi\d{n}! When getting out of the water, I came to the bank and saw the lotus-eyed Kri\d{s}\d{n}a, pure consciousness then flashed in my heart. Then I thought ``that I am N\=arada; I have come to this place and being deluded by the M\=ay\=a of Hari, I got the female form.'' When I was thinking thus, Hari exclaimed, ``O N\=arada! Get up; what are you doing, standing in the water?'' I was astonished; and, recollecting my feminine nature, very severe indeed, began to think why I was again transformed into a male form.

Here ends the Twenty-ninth Chapter of the Sixth Book on the N\=arada's getting again his male form in the Mah\=apur\=a\d{n}am \'Sr\={\i} Mad Dev\={\i} Bh\=agavatam of 18,000 verses by Mahar\d{s}i Veda Vy\=asa.



