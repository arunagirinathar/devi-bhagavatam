\chapter{On the taking away of Hari\'schandra's Kingdom}

1-12. Vy\=asa said :-- O King! Hearing thus the words of the King Hari\'schandra, the Mahar\d{s}i Kau\'sika smilingly said :-- ``O King! This T\={\i}rath is very sacred; if one bathes here, one is cleansed of one's sins and virtue springs up. So, highly fortunate One! Bathe in this and do peace-offerings (tarpa\d{n}am) to your fathers. O King! This time is very auspicious and highly meritorious; so take a bath in this sacred Pu\d{n}ya T\={\i}rtha and make charities as far as it lies in your power. Sv\=ayambhuva Manu says :-- He, who arriving at a t\={\i}rtha capable to give high merits (Pu\d{n}ya), does not bathe and make charities, deceives himself; so he is the slayer of his soul, no doubt. So, O King! Do meritorious acts as best as you can in this excellent t\={\i}rtha. Then I will shew you the way and you will go to Ayodhy\=a. O K\=akutstha! Today I will be pleased with your gifts and I will accompany you to show you the way; this I have decided.'' Hearing the deceitful words of the Mahar\d{s}i, the King took off his upper garments and tying the horse on to a tree, went towards the river to bathe according to due rites. O King! The accidental combination, that was to have been so (sure to come), so enchanted the King by the Muni's words, that he got himself entirely under the control of the Muni. He duly completed his bath and offered peace offerings to the Devas and the Pitris and then spoke to Vi\'sv\=amitra. ``O Lord! I am now making gifts to you. O Fortunate One! Cows, lands, jewels, elephants, horses, chariots or horses, etc., anything that you like I will give you just now. There is nothing that I cannot give. When I performed previously the R\=ajas\=uya

sacrifice, I took, then, before all the Munis, this vow. So, O Muni! You are also present at this principal T\={\i}rtha (place of pilgrimage); so express what you desire; I will give you your desired object.''

13-15. Vi\'sv\=amitra said :-- ``O King! Your glory is spread far and wide in this world; especially I have already heard that there is no second man charitable like you. The Muni Va\'sistha has said :-- The
King of the solar dynasty, the Tri\'sanku's son, Hari\'schandra is foremost and first amongst the kings in this world and there is no one so liberal-minded as he is; such a king there never was nor ever there will be. So, O King! Now the marriage time of my son has arrived; so I pray before you today, that you give me wealth to celebrate this marriage.''

16. The King said :-- ``O Br\=ahmi\d{n}! Yes! Celebrate the marriage ceremony; I will give you your desired wealth. What more can be said than this that whatever wealth you would want, I will give that abundantly. There is no doubt in this.''

17-22. Vy\=asa said :-- O King! Hearing these words of the King, the Muni Kau\'sika became ready to deceive him and originating the G\=andharb\={\i} M\=ay\=a, created a beautiful youth and one daughter aged ten years and showing them to the King, said, ``The marriage of these two is to be celebrated today. O King! To marry the boys and the girls in the house-hold is to earn more merits than the R\=ajas\=uya sacrifice. So today you will get that desired fruit if you make charities for the marriage of this Br\=ahmi\d{n} Youth.'' The King was much enchanted by his M\=ay\=a; so no sooner he heard those words, he immediately promised :-- ``That will be done,'' he did not raise any objection whatsoever. Vi\'sv\=amitra then showed the way and the King went to his city. Vi\'sv\=amitra, too, thus deceiving the King, went back to his \=A\'srama. When the King was staying in Agni\'s\=al\=a (cook-room), Vi\'sv\=amitra Muni went to him and said :-- ``O King! The marriage rites have been finished; so today give me what I desire in this sacrificial hall.''

23-24. The King said :-- ``O Br\=ahmi\d{n}! Speak out what you want; now I like to get fame. So if there be anything in the world, that is not to be given by me, if you want, I will give that even to you, no doubt. The mortal, possessing all wealth, if he does not earn good name and fame, capable to give happiness to him in his next world, passes his life in vain.''

25. Vi\'sv\=amitra said :-- ``O King! Give to this bridegroom, while within this sacred sacrificial altar, your entire kingdom with the royal umbrella and Ch\=amara for fanning the king and elephants, horses, chariots, infantry and all the gems and jewels.''

26-33. Vy\=asa said :-- O King! The King Hari\'schandra was deluded by his M\=ay\=a; so no sooner he heard the Muni's words, he willingly said without the slightest consideration :-- ``O Muni! I give as you pray, my this vast dominion to you.'' The very cruel Vi\'sv\=amitra then said :-- ``O King! I have accepted your offer; but O Intelligent One! Give now the requisite Dak\d{s}i\d{n}\=a to complete your gift. Manu says gift without Dak\d{s}i\d{n}\=a is fruitless; so to get the fruit of your gift give Dak\d{s}i\d{n}\=a as duly fixed.'' The King was exceedingly surprised to hear this and said :-- ``O Lord! Kindly say what amount of wealth am I to give to you as Dak\d{s}i\d{n}\=a. O Saint! Say the value of your Dak\d{s}i\d{n}\=a. O Ascetic! Don't be impatient; I will give you the Dak\d{s}i\d{n}\=a to that amount, no doubt.'' Hearing this, Vi\'sv\=amitra told to the King :-- ``At present give me two and a half loads of gold as Dak\d{s}i\d{n}\=a.'' The King Hari\'schandra became greatly amazed and promised :-- ``I will give you that,'' he then anxiously mounted on his horseback and became ready to go quickly. At this time, his soldiers who lost their road in quest of their king, came to him. They were very glad to see him; but, seeing him anxious, they began to praise him in great haste.

34-47. Vy\=asa said :-- O King! Hearing their words, the King did not say anything, good or bad; but thinking on his own doing entered into the zenana. Oh! What have I promised to give? I have made a gift of all that I have; I am cheated in this matter by the Muni like one robbed by a thief in a wilderness. My whole dominion including my dress I have promised to give to him. Moreover I will have to pay besides two and a half loads of gold. My brain seems to have been completely destroyed. What to do now? I did not know the cunningness of the Muni. Therefore I am cheated by this deceitful Br\=ahmi\d{n}. It is next to impossible to understand the work of Daiva. Oh! My Fate! What will happen to me now? Very much bewildered the King entered in the interior of the palace. The queen seeing her husband immersed in cares, enquired into the cause, thus :-- ``O Lord Why have you become so absent-minded? Kindly say what you are thinking now? O King! The son has come back from the forest before you completed your R\=ajas\=uya sacrifice; why then are you in grief now? Kindly speak out the cause of your sorrow. Nowhere is your enemy, strong or weak; only Varu\d{n}a was angry with you; now he is also very satisfied. So there is nothing further for you to do to think. O King! Owing to cares, this body gets weaker and weaker day by day. So nothing is like cares to lead one to death.'' When his dear wife said so, the King expressed to her somewhat the cause of his

cares, good or bad. But the King was much absorbed with his cares so that he could not eat nor sleep though his bedding was perfectly white and clear. Early in the next morning, when, getting up from his bed, he was doing anxiously his morning duties, Vi\'sv\=amitra came up there. When the sentinel informed the King of the arrival of Vi\'sv\=amitra, he gave order for him to enter. Vi\'sv\=amitra, the Looter of his all and everything, came before him and told the King who repeatedly bowed down to him :-- ``O King! Now leave your kingdom and give me the gold that you promised as Dak\d{s}i\d{n}\=a and prove that you are truthful.''

48-63. Hari\'schandra said :-- ``O Lord! I have given you this vast dominion of mine; so my Kingdom has now become yours; I am leaving this Kingdom and going to somewhere else. O Kau\'sika! You need not think a bit for this. O Br\=ahma\d{n}a! You have taken my all according to the technical rule; so now I am unable to give you Dak\d{s}i\d{n}\=a. If, in time, wealth comes to me, I will at once give you your Dak\d{s}i\d{n}\=a.'' Saying him thus, the King told his wife \'Saivy\=a, and his son Rohita, ``In this Agnihotra room I say that I have given my vast dominion to the Muni Vi\'sv\=amitra. Elephants, horses, chariots, gold and jewels all I have given to him along with my kingdom. What more than this that save us three, everything else I have given to him. O Mahar\d{s}i! Take fully this prosperous dominion; we are going somewhere else to a forest or a mountain cave.'' The exceedingly virtuous Hari\'schandra spoke thus to his wife and son, and, paying respects to the Muni, went out from his house. Seeing the King going thus away, his wife and son, afflicted with cares, followed him with their sad faces. Seeing thus, all the inhabitants of Ayodhy\=a cried aloud, and great consternation and uproar arose in the city. O King! What is this act that you have done? How has this suffering come to you! O King! The great Fate, without any consideration, has certainly deceived you. The Br\=ahma\d{n}as, K\'sattriyas, Vai\'syas and \'S\=udras, all the four Var\d{n}as gave vent to their sorrows, when they saw the King going away with his wife and son. The Br\=ahmi\d{n}s and the other inhabitants of the city, all were afflicted with sorrows and began to abuse the vicious Br\=ahma\d{n}a saying that ``He is a cheat, etc.'' O King! Give the gold for Dak\d{s}i\d{n}\=a and then go; or say that you will not be able to give and I will then not take the Dak\d{s}i\d{n}\=a. Or if you entertain within yourself any greed, then take back all your Kingdom. O King! If you think that you have really made this gift, then give what you have promised. The son of G\=adhi was saying so, when the King Hari\'schandra very humbly bowed down to him with folded palms and said to him.

Here ends the Nineteenth Chapter of the Seventh Book on the taking away of Hari\'schandra's Kingdom in the Mah\=a Pur\=a\d{n}am \'Sr\={\i} Mad Dev\={\i} Bh\=agavatam of 18,000 verses, by Mahar\d{s}i Veda Vy\=asa.



