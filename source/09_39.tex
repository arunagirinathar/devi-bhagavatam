\chapter{On the story of Mah\=a Lak\d{s}m\={\i}}

1-3 N\=arada said :-- ``O Lord! I have heard in the discourse on S\=avitr\={\i} and Yama about the Formless Dev\={\i} M\=ula Prakriti and the glories of S\=avitr\={\i}, all true and leading to the endless good. Now I want to hear the story of the Dev\={\i} Lak\d{s}m\={\i}. O Thou, the Chief of the knowers of the Vedas! What is the nature of Lak\d{s}m\={\i}? By whom was She first worshipped? and by what Mantra? Kindly describe Her glories to me.''

4-33. N\=ar\=aya\d{n}a said :-- Of old, in the beginning of the Pr\=akritik Creation, from the left side of Kri\d{s}\d{n}a, the Supreme Spirit, appeared in the R\=asamandalam (the Figure Dance) a Dev\={\i}. She looked exceedingly handsome, of a dark blue colour, of spacious hips, of thin waist, and

with high breast, looking twelve years old, of steady youth, of a colour of white Champaka flower and very lovely. The beauty of Her face throws under shade millions and millions of autumnal full moons. Before Her wide expanded eyes, the midday lotus of the autumnal season becomes highly ashamed. By the Will of God, this Dev\={\i} suddenly divided Herself into two parts. The two looked equal in every respect; whether in beauty, qualities, age, loveliness, colour, body, spirit, dress, ornaments, smile, glance, love, or humanity, they were perfectly equal.

Now she who appeared from the right side is named R\=adh\=a and she who came from the left side is named Mah\=a Lak\d{s}m\={\i}. R\=adh\=a wanted first the two armed \'Sr\={\i} Kri\d{s}\d{n}a, Who was Higher than the highest; then Mah\=a Lak\d{s}m\={\i} wanted Him. R\=adh\=a came out of the right side and wanted first Kri\d{s}\d{n}a; so Kri\d{s}\d{n}a, too, divided himself at once into two parts. From His right side came out the two-armed and from his left side came out the four-armed. The two-armed person first made over to Mah\=a Lak\d{s}m\={\i} the four armed One; then the two armed Person Himself took R\=adh\=a. Lak\d{s}m\={\i} looks on the whole universe with a cooling eye; hence She is named Lak\d{s}m\={\i} and as She is great, She is called Mah\=a Lak\d{s}m\={\i}. And for that reason the Lord of R\=adh\=a is two-armed and the Lord of Lak\d{s}m\={\i} is four-armed. R\=adh\=a is pure Apr\=a kritic \'Suddha Sattva (of the nature of pure Sattva Gu\d{n}a, the illuminating attribute) and surrounded by the Gopas and Gop\={\i}s. The four-armed Puru\d{s}a, on the other hand, took Lak\d{s}m\={\i} (Padm\=a) to Vaikuntha. The two-armed person is Kri\d{s}\d{n}a; and the four-armed is N\=ar\=aya\d{n}a. They are equal in all respects. Mah\=a Lak\d{s}m\={\i} became many by Her Yogic powers (i.e., She remained in full in Vaikuntha and assumed many forms in parts). Mah\=a Lak\d{s}m\={\i} of Vaikuntha is full, of pure Sattva Gu\d{n}a, and endowed with all sorts of wealth and prosperity. She is the crest of woman-kind as far as loving one's husbands is concerned. She is the Svarga Lak\d{s}m\={\i} in the Heavens; the N\=aga Lak\d{s}m\={\i} of the serpents, the N\=agas, in the nether regions; the R\=aja Lak\d{s}m\={\i} of the kings and the Household Lak\d{s}m\={\i} of the householders. She resides in the houses of house-holders as prosperity and the most auspicious of all good things. She is the progenetrix, She is the Surabhi of cows and She is the Dak\d{s}i\d{n}\=a (the sacrificial fee) in sacrifices. She is the daughter of the milk ocean and she is Padmin\={\i}, the beauty of the spheres of the Moon and the Sun. She is the lustre and beauty of the ornaments, gems, fruits, water, kings, queens, heavenly women, of all the houses, grains, clothings, cleansed places, images, auspicious jars, pearls, jewels, crest of jewels, garlands, diamonds, milk, sandal, beautiful twigs, fresh rain cloud, or of all other colours. She was first worhipped in Vaikuntha by N\=ar\=aya\d{n}a. Next She was worshipped by Brahm\=a and then

by \'Sankara with devotion. She was worshipped by Vi\d{s}\d{n}u in the K\d{s}hirode Samudra. Then she was worshipped by Sv\=ayambhuva Manu, then by Indras amongst men, then by Munis, \d{R}i\d{s}is, good householders, by the Gandharbas, in the Gandharbaloka; by the N\=agas in the N\=agaloka. She was worshipped with devotion by Brahm\=a for one fortnight commencing from the bright eighth day in the month of Bh\=adra and ending on the eighth day of the dark fortnight in the three-worlds. She was worshipped by Vi\d{s}\d{n}u, with devotion in the three worlds on the meritorious Tuesday in the months of Pau\d{s}a, Chaitra, and Bh\=adra. Manu, also, worshipped Her on the Pau\d{s}a Sankr\=anti (the last day of the month of Pau\d{s}a when the Sun enters another sign) and on the auspicious Tuesday in the month of M\=agha. Thus the worship of Mah\=a Lak\d{s}m\={\i} is made prevalent in the three worlds. She was worshipped by Indra, the Lord of the Devas and by Mangala (Mars) on Tuesday. She was then worshipped by Ked\=ara, N\={\i}la, Subala, Dhruva, Utt\=anapada, \'Sakra, Bali, Ka\'syapa, Dak\d{s}a, Kardama, S\=urya, Priyavrata, Chandra, V\=ayu, Kuvera, Varu\d{n}a, Yama, Hut\=asana and others. Thus Her worship extended by and by to all the places. She is the Presiding Deity of all wealth; so She is the wealth of all.

Here ends the Thirty-ninth Chapter of the Ninth Book on the story of Mah\=a Lak\d{s}m\={\i} in the Mah\=a Pur\=a\d{n}am \'Sr\={\i} Mad Dev\={\i} Bh\=agavatam of 18,000 verses by Mahar\d{s}i Veda Vy\=asa.



