\chapter{On the description of Vik\d{s}epa \'Sakti}

1-5. The King Janamejaya said :-- ``O Bhagav\=an! I am not satiated with the drink of the divine sweet nectar-like words coming out of your lotus mouth. You have described to me in detail the wonderful and variegated story of the origin of the Haihaya dynasty; but, O Muni! There has arisen in my mind a curiosity to know something more on this subject. See the Bhagav\=an Vi\d{s}\d{n}u, the Lord of Lak\d{s}m\={\i}, the Deva of the Devas, the Ruler of this whole Universe and the Cause of the Creation, Preservation and Destruction; yet that Best of Puru\d{s}as \'Sr\={\i} Bhagav\=an had to assume a horse form. He is undecaying and independent, how then He came to be dependent? Clear this doubt of mine. O Best of Munis! You are omniscient; therefore satisfy my curiosity by describing this wonderful event.''

6-16. Vy\=asa said :-- O King! Hear what I heard of yore from N\=arada how this doubt was removed. The mind-born son of Brahm\=a, Mahar\d{s}i N\=arada got powers to go everywhere by virtue of his Tapas, could know everything, was of a calm and quiet nature, dear to all and he was a poet. On one occasion he went out on tour round the world, playing with his lute in time with Svar and T\=an. One day he came to my \=A\'srama, singing many things concerning Brihat Rathantara S\=ama Veda and the sweet nectar-like G\=ayatr\={\i}, the Giver of Liberation. O King! There was a very sacred place of hermitage, beaming as it were with happiness and self-knowledge, named \'Samy\=apr\=asa, on the banks of the river Sarasvat\={\i}. There was situated my hermitage. Seeing the lustrous N\=arada the son of the Grand Sire Brahm\=a, coming, I got up and offered him duly P\=adya (water to wash his feet) and Argha (offerings of worship), etc., and worshipped him. When that Muni of indomitable lustre took his seat on the \=Asana, I sat beside him. When I found N\=arada, the Giver of Knowledge, at rest and quiet, I duly asked him the very same question that you have asked me just now. O Best of Munis! What happiness is there on the beings taking their birth in this world. I never found it in any place or in any concern, this I can say positively. Still why do the high minded persons do Karma, fascinated by the enchantments of the world. Look! I was born in an island. Just after my birth, my mother forsook me. Helpless, I grew in the forest as my Karma allowed. Next I performed a very severe tapasy\=a before Mah\=adeva, the Deva of the Devas, on the mountain with a desire to have a son.

17-38. As a fruit of that I got \'Suka as my son, the foremost of the Gnostics, and taught him completely the essence of the Vedas from the beginning to the end. O Devar\d{s}\={\i}! When my son got wisdom from you, he left this world even when I became very distressed on his bereavement and wept aloud and he went away to the next world. Very much afflicted for the parting of my son, I abandoned the great Mountain Meru. I got very lean due to the absence of my dear son whom I loved very much; and becoming very distressed and knowing this whole world to be an illusion, I remembered my mother and went to the Kuru J\=a\d{n}gala district, as if bound up and controlled by the snares of M\=ay\=a. When I heard that the King \'S\=antanu had married my mother, I built my hermitage on the holy banks of the Sarasvat\={\i} and remained there. When the King \'S\=antanu went to the next world, my chaste mother remained with two sons. At that time Bh\={\i}\d{s}ma looked after their sustenance and maintained them. The intelligent Gang\=a's son Bh\={\i}\d{s}ma Deva installed Chitr\=angada on the throne. A short while after this, Chitr\=angada, too, looking like a second Cupid and extremely lovely, went to the jaws of death. The mother Satyavat\={\i} was drowned in the sorrows for his son Chitr\=angada and began to weep for him. O King! Knowing my mother in that sorrowful condition, I went to her. Bh\={\i}\d{s}ma and I then consoled her with hopeful words. Bh\={\i}\d{s}ma Deva was averse to marrying and then becoming a King; and, therefore, he installed again the younger brother, the powerful Vichitrav\={\i}rya on the throne. O King! Bh\={\i}\d{s}ma defeated by his own prowess the kings and brought the two daughters of the King K\=a\'s\={\i}r\=aj and handed them over to Satyavat\={\i}, so that she might give them over to Vichitrav\={\i}rya. Then, on an auspicious day, and in an auspicious Lagna (moment) when the marriage ceremony of my brother Vichitrav\={\i}rya was performed, I became glad. My brother, a good archer, was shortly afterwards attacked with consumption and thus he died without any issue. At this my mother became very sad and dejected. Seeing the husband dead, the two daughters of K\=a\'s\={\i}r\=aja became ready to preserve their religion of chastity and said to their mother-in-law, sorrowful and weeping :-- We two shall accompany our husbands and become Sat\={\i} (i.e., be burnt up with our husbands). O Dev\={\i}! We will go to the Heavens with your son. We, the two sisters united, will enjoy with him in the Nandana Garden. The mother was very much attached to them and with the permission of Bh\={\i}\d{s}ma Deva, very affectionately made them desist from this great attempt. When all the funeral obsequies of Vichitrav\={\i}rya were over, my mother consulted with Bh\={\i}\d{s}ma and remembered me in Hastin\=anagara. As soon as she remembered me, immediately I knew her mental feelings and hurriedly came to Hastin\=anagara

and, with my head bowed, fell prostrate before her feet, and with folded hands addressed my mother who was very much inflamed with the fire of sorrow for the death of her son, thus :-- O Mother! Why have you called me here mentally? I see you are very much dejected; I am your servant; order me what I can do for you. O Mother! You are my greatest place of pilgrimage and you are my highest deity; I am very anxious since I have come here; say what you desire.

39-44. Vy\=asa said :-- O Best of Munis! When I said thus and waited before her, then she looked at Bh\={\i}\d{s}ma standing close by and said :-- ``O Child! Your brother died of consumption; therefore I am very sorrowful, lest the family becomes extinct. O Intelligent One! For the continuance of the line, then, with the permission of the Gang\=a's son, I have called you here today by the Sam\=adhi Yoga. O son of Par\=a\'sara! You re-establish the name of \'S\=antanu that is going now to be well nigh extinct. O Vy\=asa Deva! Relieve me soon from this sorrow of mine, lest this line be extinct. There are the two daughters of K\=a\'s\={\i}r\=aja, honest and good and endowed with youth and beauty. O Highly Intelligent One! Better you cohabit with them and save the family of Bh\=arata by begetting sons. You will not be touched with any sin.''

45-55. Vy\=asa said :-- O Devar\d{s}\={\i}! Hearing the mother's words, I became very anxious and humbly told her with great shame :-- ``O Mother! To touch another's wife is a very sinful act; knowing well the path of Dharma, how can I willingly and intentionally violate that? So also, the Mahar\d{s}is say :-- That the wife of a younger brother is like a daughter. Studying all the Vedas, how can I do this blame-worthy and adulterous act? To preserve a line of family by illegal ways is never to be done; for then the fathers of the sinners can never cross this ocean of world. How can he, who is the spiritual preceptor of all, and the writer of all the Pur\=a\d{n}as, do this act knowingly which is awfully strange and very bad and nasty in its nature.'' My mother was very much plunged into the sea of sorrows for the bereavement of her son; so to preserve the family, She came again to me, weeping and said :-- ``O son of Par\=a\'sara! If you follow my word, you won't incur any sin. O Child! If the reasonable words of the Gurus be even faulty, one should obey them according to the tradition of the \'Sistas. Therefore, O Child! Keep my word and preserve my honour; no sin will touch you. O Child! Think very well. Your mother is very sorry and is immersed in the ocean of afflictions; therefore it is your paramount duty to make her happy by begetting child for the continuance of the family.'' Hearing my mother speaking to me thus, Bh\={\i}\d{s}ma, the Gang\=a's son, the expert

in finding out truth in fine points with regard to Dharma, said to me :-- O Dvaip\=ayana! You are wholly sinless; you ought not therefore to argue on this point; obey your mother as she says and be happy.

56-61. Vy\=asa said :-- O King! Hearing his words and my mother's request, I decided to do this very hateful act with a fearless heart without any suspicion. When Ambik\=a finished her ablutions after menstruation, I gladly cohabited with her in the night; but that young lady seeing my ugly ascetic form, was not attached to me; I then cursed that beautiful woman thus :-- As you closed your eyes at the first cohabitation with me, your son will be born blind. O Muni! On the second day my mother enquired me when I was alone :-- O Dvaip\=ayana! Will there be born a son of the daughter of K\=a\'s\={\i}r\=aj? I then bowed my head with shame, and told, ``Mother! The son will be born blind, through my curse.'' O Muni! The mother then rebuked me harshly, ``O Child! Why did you curse that the son of Ambik\=a would be born blind?''

Here ends the Twenty-fourth Chapter in the Sixth Book on the description of Vik\d{s}epa \'Sakti in the discourse between Vy\=asa and N\=arada in the Mah\=apur\=a\d{n}am \'Sr\={\i} Mad Dev\={\i} Bh\=agavatam of 18,000 verses by Mahar\d{s}i Veda Vy\=asa.



