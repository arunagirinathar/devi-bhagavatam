\chapter{On the narration of the division of the continents}

1-2. N\=ar\=aya\d{n}a said :-- In Harivar\d{s}a, the Bhagav\=an Hari is shining splendid as a Yogi in the form of Narasimha. The Mah\=a Bh\=agavat (most devoted) Prahl\=ada, who knows the attributes of God worships and chants hymns with his whole hearted devotion, seeing that beautiful form, gladdening to all the people.

3-11. Prahl\=ada spoke :-- I bow down to Thee, the Bhagav\=an Nrisi\d{n}gha Deva. Thou art the Light of all Lights. Thy big teeth are like thunderbolts. Let Thee manifest in Thy most terrible form. Let Thee destroy the desires of the people to do Karma and let Thou devour the great Ignorance (Aj\~n\=ana) the Moha (delusion) of the people. Thou art the receptacle of the Sattva, Raja and Tamo Gu\d{n}as. Let myself be always free from any fear by Thy Grace. ``Om Khraum!'' Let this whole world rest completely in peace and happiness. Let the cheats quit their guiles and be pure and simple. Let all the people quit completely their animosities towards each other and think of their welfare. Let all the people be free from making injuries to others and be peaceful; and let them have their control over their passions. Let our mind be completely free from desires and rest entirely and devotedly to Thy lotus-feet. Let us not be attached to sons, wives, wealth, house or to any other worldly objects. If there be any attachment, let it be to the objects dear to the Bhagav\=an. He who barely sustains his body and soul and controls himself completely, success is very near to him; not so to the persons that are attached to the senses. The dirt of the mind, that is not washed away by bathing in the Ganges or by taking recourse to the T\={\i}rthas, etc., is removed by the company of the devotees to the God and by their influence, hearing, thinking, and meditating on the attributes of the Bhagav\=an. So who is there that does not serve the Bhagav\=an! He who has got Nisk\=ama Bhakti (devotion without regard to any fruits thereof) to the Bhagav\=an, to him come always the Devat\=a, Dharma and J\~n\=ana and other higher qualities. But he who indulges in various mental phantasms, without any Bhakti to the Bhagav\=an, he follows the worldly happiness that is certainly to be hated and never he gets Vair\=agyan and other higher qualities. As water is life to the fish, so the Bhagav\=an Hari is the self of all embodied beings and so He is to be specially prayed for. So if a high-souled person be attached to household happiness, without thinking of God, then his greatness dwindles into a trifling insignificance like the ordinary pleasures of man and woman when they are full of youth. So leave, at once, the home that is the source of Birth and Death and leave Tri\d{s}\d{n}\=a (thirst, desire), clinging to life, low-spiritedness, name, and fame, egoism, shame, fear, poverty and loss of one's honour and worship the Lotus-Feet of the Bhagav\=an Nrisi\d{n}gha Deva and be entirely fearless. Thus Prahl\=ada, the Lord of the Daityas, daily worships devotedly the Bhagav\=an Nrisi\d{n}gha, resplendent in his lotus heart, the death blow, the lion to all the elephant sins. In the Ketum\=ala Var\d{s}a, the Bhagav\=an N\=ar\=aya\d{n}a is reigning in the form of the K\=ama Deva, the God of Love. The people there always worship Him. The daughter of Ocean, the Indir\=a Dev\={\i}, who confers honour and glory to the Mahatmas,

is the presiding Deity of the Var\d{s}a. She always worships the Kama Deva with the following verses :--

12-18. The Lak\d{s}m\={\i} Dev\={\i} spoke :-- ``Om, Hr\=am, Hr\={\i}m, Hr\=um, Om namo Bhagavate Hri\d{s}ike\d{s}\=aya! Thou art the Bhagav\=an of the nature of Om. Thou are the Director, the Lord of the senses: Thy \=Atman is the Highest and the Receptacle of all the good things. All the Karma Vrittis, all the J\~n\=ana Vrittis, and effort and resolution and other faculties of the mind, act in their respective channels by Thy looking and by their being constantly practised in Thee. And the elements over which they get their masteries are subservient to Thy Laws. The mind and the other eleven Indriyas, and touch, taste and other five senses are but Thy parts. All the rites and ceremonies observed in the Vedas are found in Thee. Thou art the infinite store of all the foodings of the J\={\i}vas. From Thee flows the Param\=ananda, the Highest Bliss. Thou art All, Thou art the Substance, Purity incarnate; Thou art the Energy, the Strength manifest in all. Thou art the Finish of all the happiness and Thou art the Only One Substance, that can be desired by the people. So obeisance to Thee! This Thy Lordship is not dependent on any other body. The women that know Thee, the Lord of all and worship other bodies for their husbands, those husbands can never save them, their lives, their wealth, progeny or other dear things as those are controlled by K\=ala (Time) and Karma. So they cannot be termed husbands at all; Thou art the Real Husband; and no other. For Thou art naturally fearless and Thou protectest in every way the persons that become afraid. Thou art the Lord of all wealth; so no other is superior to Thee. How can then, they be independent whose happiness depends on others! The lady that desires to worship Thy lotus feet only and becomes subservient to no others, she attains all the desires. Again the lady who, desiring other desires than to get Thee, does not worship Thy lotus-feet, Thou fulfillest the desires of her too. But, O Bhagav\=an! When the period of enjoyment of these things ceases and when the objects of these enjoyments are destroyed, then she repents much due to the loss of those things. Brahm\=a, Mah\=adeva, the Suras and the Asuras practise hard Tapasy\=as to get me, impelled by their desires to attain the objects of sense enjoyments; but he only really gets me who worships and takes refuge of Thy lotus feet only, for my heart is entirely attached to Thee. So, O Achyuta! Kindly shew Thy Grace and put, on my head, Thy lotus palm, praised by the universal people that Thou placest on Thy Bhaktas. O Bhagav\=an! That Thou takest me in Thy Bosom is a sign of Thy Grace. No one can fathom the deeds of Thee, the Only Controller of all. Thus the Praj\=apati and the Lords of that Var\d{s}a, worship the Bhagav\=an,

the Friend of all, with a view to attain their respective desires and Siddhis. In Ramyak Var\d{s}a, the Matsya form of the Bhagav\=an is set up and consecrated. The Suras and the Asuras worship Him. The highly intelligent Manu always chant hymns to that Excellent Form thus :-- ``Obeisance to Him who is the Life of all, the Essence and Strength of all, to that Great Fish Form, the Body Incarnate of Sattva Gu\d{n}a, who is of the nature of Om and Bliss.''

19-23. Thou art the Lord of all the Lokap\=alas and of the form of the Vedas. Thou art within and without all this universe, moving and non-moving; still all the beings are unable to see Thy form. As the people bring under their control the wooden dolls, so Thou controllest the universe by the rules and prohibitions under the names of the Br\=ahma\d{n}as, etc. Thou art the God. The Lokap\=alas, being overcome by the fever of jealousy and pride, become quite unable, either individually or collectively, to quit their jealousies and to protect the tripeds, quadrupeds, reptiles and snakes; so Thou art the God. Thou hadst upheld this earth along with me and with the medicinal plants and creepers; and Thou shewedest the highest luminous light in the great ocean, at the time of Pralaya, tossed with surging waves and didst roam there. Thou art the Self of all the beings in the universe. So we bow down to Thee. Thus the Manu, the best of the mortal beings, used to praise the Bhagav\=an, who took His incarnation in the shape of the Fish, the Remover of all doubts. Manu, the foremost of the Bh\=agavatas (the devotees) is reigning there in the service of the Fish Incarnation of the Bhagav\=an, with intense meditation and expurging all sins and with great devotion.

Here ends the Ninth Chapter of the Eighth Book on the narration of the division of the continents in the Mah\=a Pur\=a\d{n}am \'Sr\={\i} Mad Dev\={\i} Bh\=agavatam of 18,000 verses by Mahar\d{s}i Veda Vy\=asa.



