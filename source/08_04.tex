\chapter{On the narration of the family of Priyavrata}

1-28. The \d{R}i\d{s}i N\=ar\=aya\d{n}a said :-- The eldest son of Sv\=ayambhuva, Priyavrata served always his father and was very truthful. He married the daughter of the Praj\=apati Vi\'sva Karm\=a, the exceedingly lovely and beautiful Barhi\d{s}mat\={\i}, resembling like him, adorned with modesty, good nature and various other qualifications. He begat ten sons, very spiritual and well qualified and one daughter named \=Urja\d{s}vat\={\i}. This daughter was the youngest of all. The names of the ten sons are respec-

tively :-- (1) \=Agn\={\i}dhra, (2) Idhmajibha, (3) Jaj\~nab\=ahu, (4) Mah\=av\={\i}ra, (5) Rukma\'sukra (Hira\d{n}yaret\=a), (6) Ghritapristha, (7) Savana, (8) Medh\=atithi, (9) V\={\i}tihotra and (10) Kavi. The name ``Agni'' was attached to each of the above names. Out of these ten, the three sons named Kavi, Savana, and Mah\=av\={\i}ra were indifferent and dispassionate to the world. In due time, these became extremely free from all desires and they were proficient in \=Atmavidy\=a (Self-Knowledge). They were all \=Urdharet\=a (of perpetual chastity; who has subdued all their passions) and took gladly to the Paramahamsa Dharma. Priyavrata had by his other wife three sons, named Uttama, T\=amasa, and Raivata. These were all widely known; each of them in due time became endowed with great prowess and splendour and became the Lord of one Manvatara. Priyavrata, the son of Sv\=ayambhuva, the King-Emperor enjoyed with his sons and relations, this earth for eleven Arvuda years; the wonder was this, that he lived so long and there was seen no decay in his strength as regards his body or his senses. Once on an occasion, the King observed that when the sun appeared on the horizon and got up, one part of the earth was illumined and the remaining part was enveloped in darkness. Seeing this discrepancy, he thought over for a long time and exclaimed, ``What! Will the Darkness be seen in my kingdom, while I am reigning? This can never be. I will stop this by my Yogic powers.'' Thus thinking, the King Priyavrata mounted on a luminous chariot, as big as the Sun, to illumine the whole world and circumambulated round the earth seven times. Whatever portion of the earth was trodden by the wheel on each occasion, became an ocean. Thus the seven oceans had their origins. And the portion of the earth, that was included within the ruts, became the seven islands (Dv\={\i}pas). 0 Child! Now hear about the seven Dv\={\i}pas and the seven Oceans :-- The first is the Jambu Dv\={\i}pa; the second is Plak\d{s}a, the third is \'Salmal\={\i}; the fourth is the Ku\'sa Dv\={\i}pa; the fifth is Krauncha; the sixth is the \'Saka Dv\={\i}pa; and the seventh is the Pu\d{s}kara Dv\={\i}pa. The second Dv\={\i}pa Plak\d{s}a is twice the first Jambu Dv\={\i}pa and so on; each succeeding Dv\={\i}pa is twice as large as its previous one. Now hear the names of the oceans. The first ocean is named Ks\=aroda (the salt water ocean); the second is Ik\d{s}urasa (the sugarcane ocean); the third is Sur\=a (the wine ocean), the fourth is Ghritoda (the clarified butter ocean) the fifth is K\d{s}\={\i}roda (the ocean of milk); the sixth is Dadhi Manda (the ocean of curds); and the seventh is that of the ordinary water. The Jambu Dv\={\i}pa is surrounded by K\d{s}\={\i}ra Samudra. The King Priyavrata made his son \=Agn\={\i}dhra, the lord of this Dv\={\i}pa. He gave to his Idhmajibha, the Plak\d{s}a Dv\={\i}pa surrounded by Ik\d{s}u S\=agara; so he gave to Jaj\~nab\=ahu the \'S\=almal\={\i} Dv\={\i}pa surrounded by Sur\=a S\=agara and he

gave the lordship of Ku\'sa Dv\={\i}pa to Hira\d{n}yaret\=a. Then he gave to his powerful son Ghritapristha the Krauncha Dv\={\i}pa surrounded by K\d{s}\={\i}ra Samudra and to his son Medh\=atithi the \'S\=aka Dv\={\i}pa surrounded by Dadhimanda S\=agara. Finally he gave to his V\={\i}tihotra, the Pu\d{s}kara Dv\={\i}pa surrounded by the ordinary water. Thus distributing duly amongst his sons, the separate divisions of the earth, he married his daughter, the youngest \=Urjasvat\={\i} to the Bhagav\=an U\'san\=a. In the womb of \=Urjasvat\={\i} the Bhagav\=an \'Sukr\=ach\=arya had his famous daughter Devay\=an\={\i}. O Child! Thus giving the charge of each Dv\={\i}pa to each of his sons and marrying his daughters to the worthy hands, he took to Viveka (discrimination) and adopted the path of Yoga.

Here ends the Fourth Chapter of the Eighth Book on the narration of the family of Priyavrata in the Mah\=a Pur\=a\d{n}am, \'Sr\={\i} Mad Dev\={\i} Bh\=agavatam of 18,000 verses by Mahar\d{s}i Veda Vy\=asa.



