\chapter{On the Yama's giving \'Sakti Mantra to S\=avitr\={\i}}

1-2. N\=ar\=aya\d{n}a said :-- O N\=arada! Hearing thus the supreme nature of M\=ula Prakriti from Dharmar\=aja Yama, the two eyes of S\=avitr\={\i} were filled with tears of joy and her whole body was filled with a thrill of rapture, joy and ecstacy. She again addressed Yama :-- ``O Dharmar\=aja! To sing the glories of M\=ula Prakriti is the only means of saving all. This takes away the old age and death of both the speaker and the hearer.

3-12. This is the Supreme Place of the D\=anavas, the Siddhas, the ascetics. This is the Yoga of the Yogins and this is studying the Vedas of the Vaidiks. Nothing can compare even to one sixteenth of the sixteenth parts of the (full) merits of those who are in \'Sakti's Service; call it Mukti, immortality, or attaining endless Siddhis, nothing can come to it. O Thou, the foremost of the Knowers of the Vedas! I have heard by and by everything from Thee. Now describe to me how to worship M\=ula Prakriti and what are the ends of karmas, auspicious and inauspicious.'' Thus saying, the chaste S\=avitr\={\i} bowed down her head and began to praise Yama in stotras according to the Vedas. She said :-- ``O Dharmar\=ajan! The Sun practised of yore very hard austerities at Pu\d{s}kara and worshipped Dharma. On this, Dharma Himself became born of S\=urya as his son. And Thou art that son of S\=urya, the incarnation of Dharma. So I bow down to Thee. Thou art the Witness of all the J\={\i}vas; Thou seest them equally; hence Thy name is Samana. I bow down to Thee. Sometimes Thou by Thy own will takest away the lives of beings. Hence Thy name is Krit\=anta. Obeisance to Thee! Thou holdest the rod to distribute justice and pronounce sentence on them and to destroy the sins of the J\={\i}vas; hence Thy name is Dandadhara; so I bow down to Thee. (Note :-- Any J\={\i}va, in course of his travelling towards Mukti, can expect to pass through the stage Yamaship; and if he pleases, he can become a Yama.) At all times Thou destroyest the universe. None can resist Thee. Hence Thou art named K\=ala; so obeisance to Thee! Thou art an ascetic, devoted to Brahm\=a, self-controlled, and the distributor of the fruits of Karmas to the J\={\i}vas; Thou restrainest Thy senses. Hence Thou art called Yama. Therefore I bow down to Thee.

13-17. Thou art delighted with Thy Own Self; Thou art omniscient; Thou art the Tormentor of the sinners and the Friend of the Virtuous. Hence Thy name is Pu\d{n}ya Mitra; so I bow down to Thee. Thou art born as a part of Brahm\=a; the fire of Brahm\=a is shining through Thy body. Thou dost meditate on Para Brahm\=a, Thou art the Lord. Obeisance to Thee!'' O Muni! Thus praising Yama, She bowed down at the feet of Him. Yama gave her the mantra of M\=ula Prakriti. How to worship Her and He began to recite the fruition of good Karmas. O N\=arada! He who recites these eight hymns to Yama early in the morning, getting up from his bed, is freed of the fear of death. Rather he becomes freed of all his sins. So much so, that even if he be a veritable awful sinner and if he recites daily with devotion this Yam\=astakam, Yama purifies him thoroughly.

Here ends the Thirty-first Chapter of the Ninth Book on the Yama's giving \'Sakti Mantra to S\=avitr\={\i} in the Mah\=apur\=a\d{n}am \'Sr\={\i} Mad Dev\={\i} Bh\=agavatam of 18,000 verses by Mahar\d{s}i Veda Vy\=asa.



