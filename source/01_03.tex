\chapter{On praising the Pur\=a\d{n}as and on each Vy\=asa of every Dv\=apara Yuga}

1-11. S\=uta said :-- ``O best of the Munis! I am now telling you the names of the Pur\=a\d{n}as, etc., exactly as 1 have heard from Veda Vy\=asa, the son of Satyavati; listen.

The Pur\=a\d{n}a beginning with "ma" are two in number; those beginning with ``bha'' are two; those beginning with ``bra" are three; those beginning with "va'' are four; those beginning respectively with ``A'', ``na'', ``pa'', ``Ling'', ``ga'', ``k\=u'' and ``Ska'' are one each and ``ma'' means Matsya Pur\=ana, M\=arkandeya Pur\=ana; ``Bha'' signifies Bhavi\d{s}ya, Bh\=agavat Pur\=a\d{n}as; ``Bra'' signifies Brahm\=a, Brahm\=anda and Brahm\=avaivarta Pur\=a\d{n}as; ``va'' signifies V\=aman, Vayu, Vi\d{s}\d{n}u and Varaha Pur\=a\d{n}as; ``A'' signifies Agni Pur\=a\d{n}a; ``Na'' signifies Narada Pur\=ana; ``Pa'' signifies Padma Pur\=ana; ``Ling'' signifies Linga Pur\=anam; ``Ga'' signifies Govinda Pur\=a\d{n}am; K\=u signifies Kurma Pur\=ana and ``Ska'' signifies Skanda Pur\=anam. These are the eighteen Pur\=a\d{n}as. O Saunaka! In the Matsya Pur\=a\d{n}a there are fourteen thousand slokas; in the wonderfully varied Markandeya Pur\=anam there are nine thousand slokas. In the Bhavisya Pur\=a\d{n}a fourteen thousand and five hundred slokas are counted by the Munis, the seers of truth. In the holy Bh\=agavata there are eighteen thousand \'Slokas; in the Brahm\=a Pur\=a\d{n}a there are Ajuta (ten thousand) \'Slokas. In the Brahm\=anda Pur\=a\d{n}a there are twelve thousand one hundred \'Slokas; in the Brahm\=a Vaivarta Pur\=a\d{n}am there are eighteen thousand \'Slokas. In the Vaman Pur\=a\d{n}a there are Ajuta (ten thousand) \'Slokas; in the Vayu Pur\=a\d{n}am there are twenty-four thousand and six hundred \'Slokas; in the greatly wonderful Vi\d{s}\d{n}u Pur\=ana there are twenty-three thousand \'Slokas; in the Agni Pur\=a\d{n}am there are sixteen thousand \'Slokas; in the Brihat Narada Pur\=a\d{n}am, there are twenty-five thousand \'Slokas, in the big Padma Pur\=a\d{n}a there are fifty-five thousand \'slokas; in the voluminous Linga Pur\=a\d{n}a eleven thousand \'slokas exist; in the Garuda Pur\=a\d{n}am spoken by Hari nineteen thousand \'slokas exist; iu the Kurma Pur\=a\d{n}a, seventeen thousand \'slokas exist and in the greatly wonderful Skanda Pur\=a\d{n}a there are eighty-one thousand \'slokas, O sinless \d{R}i\d{s}is! Thus I have described

to you the names of all the Pur\=a\d{n}as and the number of verses contained in them. Now hear about the Upa Pur\=a\d{n}as.

12-17. The first is the Upapur\=a\d{n}a narrated by Sanat Kum\=ara; next comes Narasimha Pur\=a\d{n}a; then Naradiya Pur\=a\d{n}a, \'Siva Pur\=a\d{n}a, Pur\=a\d{n}a narrated by Durvasa, Kapila Pur\=a\d{n}a, Manava Pur\=a\d{n}a, Au\'sanasa Pur\=a\d{n}a, Varu\d{n}a Pur\=a\d{n}a. Kalika Pur\=a\d{n}a, Samva Pur\=a\d{n}a, Nandi Ke\'swara Pur\=a\d{n}a, Saura Pur\=a\d{n}a, Pur\=a\d{n}a spoken by Par\=a\'sara, \=Aditya Pur\=a\d{n}a, Mahesvara Pur\=a\d{n}a, Bh\=agavata and Vasistha Pur\=a\d{n}a. These Upa Pur\=a\d{n}as are described by the Mahatmas.

After compiling the eighteen Pur\=a\d{n}as, Veda Vy\=asa, the son of Satyavati composed Mahabharata, that has no rival, out of these Pur\=a\d{n}as.

18-24. At every Manvantara, in each Dv\=apara Yuga, Veda Vy\=asa expounds the Pur\=a\d{n}as duly to preserve the religion. Veda Vy\=asa is no other person than Vi\d{s}\d{n}u Himself; He, in the form of Veda Vy\=asa, divides the (one) Veda into four parts, in every Dv\=apara Yuga, for the good of the world. The Brahm\=a\d{n}as of the Kali age are shortlived and their intellect (Buddhi) is not sharp; they cannot realise the meaning after studying the Vedas; knowing this in every Dv\=apara Yuga Bhagav\=an expounds the holy Pur\=a\d{n}a Samhitas. The more so because women, \'Sudras, and the lower Dvijas are not entitled to hear the Vedas; for their good, the Pur\=a\d{n}as have been composed. Tne present auspicious Manvantara is Vaivasvata; it is the seventh in due order; and the son of Satyavati, the best of the knowers of Dharma, is the Veda Vy\=asa of the 28th Dv\=apara Yuga of this seventh Manvantara. He is my Guru; in the next Dv\=apara, Yuga Asvatthama, the son of Drona will be the Veda Vy\=asa. Twenty-seven Veda Vy\=asas had expired and they duly compiled each their own Pur\=ana Samhitas in their own Dv\=apara Yugas.

25-35. The \d{R}i\d{s}is said :-- ``O highly fortunate S\=uta! kindly describe to us the names of the previous Veda Vy\=asas, the reciters of the Pur\=a\d{n}as in the Dv\=apara Yugas.

S\=uta said :-- In the first Dv\=apara, Brahm\=a Himself divided the Vedas; in the second Dv\=apara, the first Prajapati Vy\=asa did the same; so \'Sakra, in the third, Brihaspati, in the fourth, Surya in the fifth; Yama, in the sixth, Indra, in the seventh, Vasistha, in the eighth; Sarasvata \d{R}i\d{s}i in the ninth, Tridhama, in the tenth; Trivri\d{s}a, in the eleventh, Bharadv\=aja, in the twelfth; Antariksa, in the thirteenth; Dharma, in the fourteenth; Evaruni in the fifteenth; Dhananjaya, in the sixteenth; Medhatithi in tba seventeenth; Vrati, in the eighteenth; Atri, in the nineteenth; Gautama in the twentieth, Uttama, whose soul was fixed on Hari, in the twenty-first, V\=ajasrav\=a Vena, in the twenty second; his family descendant Soma

iu the twenty-third; Trinavi\d{n}du, in the twenty-fourth; Bh\=argava, in the twenty-fifth; Sakti, in the twenty-sixth, J\=at\=ukar\d{n}ya in the twenty-seventh and Kri\d{s}\d{n}a Dvaip\=ayana became the twenty-eighth Veda Vy\=as in the Dv\=apara Yugas. Thus I have spoken of the 28 Veda Vy\=asas, as I heard. 1 have heard the holy \'Srimad Bh\=agavat from the month of Kri\d{s}\d{n}a Dvaipayana. This removes all troubles, yields all desires, and gives Moksa and is full of the meanings of the Vedas. This treatise contains the essence of all the \'Sastras and is dear always to the Mamuksas (those who want Moksa or liberation).

36-43. O best Munis! Thus, compiling the Pur\=a\d{n}as Veda Vy\=asa thought this Pur\=a\d{n}a to be the best; so (without teaching it to other persons) he settled that his own son the high-sould \'Suka Deva born of the dry woods used for kindling fire (excited by attrition), having no passion for the worldly things, would be the fit student to be taught this Pur\=a\d{n}a and therefore taught him; at that time I was a fellow student along with \'Saka Deva and I heard every thing from the mouth of Vy\=asa Deva and realised the secret meanings thereof. This has happened through the grace of the merciful Guru Veda Vy\=asa.

Here ends the Third Chapter of \'Srimad Devi Bh\=agavatam on praising the Pur\=a\d{n}as and on each Vy\=asa of every Dv\=apara Yuga.



