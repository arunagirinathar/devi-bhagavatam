\chapter{On the story of K\=akutstha and the origin of M\=andh\=at\=a}

1-11. Vy\=asa said :-- O King! Once on a time, the time for Astaka \'Sr\=addha (the funeral ceremony in honour of the departed) arrived. Seeing this, the King Ik\d{s}\=aku ordered his son Vikuk\d{s}i :-- ``O Child! Go immediately to the forest and bring carefully pure sanctified meat for the

\'Sr\=addha purposes; see, that there be no neglect of duty.'' Thus ordered, Vikuk\d{s}i instantly went to the forest equipped with arms. He hunted in the forest lots of boars, pigs, deer, and hare. But he was so very tired with his journey in the forest and got so hungry that he forgot everything about the Astaka \'Sr\=addha and ate one hare there in the forest. The remaining excellent meat he brought and handed over to his father. When that meat was brought to be sprinkled for purification, the family priest Va\'sistha, on seeing it, at once came to know that some portion had already been eaten and it was the remaining part. The leavings of food are not fit for the sprinkling purposes; this is the \'S\=astric rule. Va\'sistha informed the King of this defect in the food. In accordance with the Guru's advice, the King coming to know thus the violation of the rule by his son, became very angry and banished his son from his kingdom. The prince became known from that time as Sa's\=ada; he did not become the least sorry for his father's anger; he went to the forest and gladly remained there. He gladly passed his time absorbed in religion and sustained his life on forest fruits and roots. After sometime when his father died, he inherited his kingdom. On becoming the King of Ayodhy\=a, \'Sa\'s\=ada had only one son; he became famous in the three lokas by the name of Kakutstha. He was known also by other names Indrav\=aha and Puranjaya.

12. Janamejaya said :-- ``O Holy One! How and why was the prince named Kakutstha. Why was he known by the two other names? Speak all this to me.''

13-14. Vy\=asa said :-- O King! When \'Sa\'s\=ada went to the Heavens Kakutstha became king. That religious king then began to govern the country of his father and grandfather with an authority supported by a powerful arm. At this time the Devas suffered a defeat from the D\=anavas and took refuge to Vi\d{s}\d{n}u, the Infallible and the Lord of the three worlds. The eternal great Vi\d{s}\d{n}u full of intelligence and bliss then addressed the Devas :--

15-16. Vi\d{s}\d{n}u said :-- ``O Devas! Go and pray to the King \'Sa\'s\=ada. He will be your ally and kill all the Demons. That King is religious; especially he is a worshipper of the Highest \'Sakti. He is a good archer and will come to help you. His strength is immense.''

17-18. Vy\=asa said :-- O King! Indra and the other Devas hearing the nectar like words of Hari went to Ayodhy\=a, to Kakutstha, the son of \'Sa\'s\=ada. Seeing the Devas at his palace, the king worshipped them duly and with great care and he asked them why they had come there.

19-20. The King said :-- ``O Devas! When you have favoured me by your presence here, I am blessed and sanctified; my life is crowned

with success. Say what I can do for you; I will carry it out even if it be very hard for me to perform.''

21-22. The Devas said :-- ``O Prince! Please help and back us and defeat the Daityas, invincible by the Devas and form an alliance with Indra. O King! By the grace of the Highest \'Sakti, you have nothing unattained anywhere; so we have come to you by the order of Vi\d{s}\d{n}u.''

23-41. The King said :-- ``O Devas! I can back you and become your ally if Indra carries me on his back in the time of war. I will fight now with the Daityas for the Devas; but I will go to the battle-field on Indra's back; this I speak to you truly.'' Vy\=asa said :-- O King! The Devas then spoke to Indra :-- ``O Lord of \'Sachi! It is now your bounden duty to do this; so quitting shame, be a carrier to this King.'' Indra got ashamed very much, but being requested frequently by Hari, at last assumed the appearance of a bull like the great Bull of \'Siva. The King mounted on that bull to go to the war; he fought while taking his seat on the hump on the shoulders of the bull (Kakud); therefore he was named Kakutstha. The King was carried by Indra on his back hence he was named Indrav\=aha; he conquered the D\=anavas in battle; hence he was called Puranjaya. The powerful King defeated the D\=anavas and gave away all their wealth to the Devas. He bade farewell to the Devas and returned to his own kingdom. Thus the alliance was formed with Indra. O King! Kakutstha became very celebrated on this earth; his descendants became kings and were known as Kakutsthas and were all very famous here on this Earth. Kakutstha had one powerful son, named K\=akutstha by his legal wife; K\=akutstha had the son Prithu, of mighty prowess. Prithu was the part incarnation personified of Vi\d{s}\d{n}u, and worshipper of the feet of the Supreme \'Sakti. His son was Visvarandhi; he became king and governed the kingdom. His son was Chandra; he came to be king, governed his subjects and multiplied very much his issues. Yuvan\=a\'sva was one of his sons; he was very powerful and spirited. \'Savanta was the son of Yuvan\=a\'sva; he was very religious. He built a nice city named \'S\=avant\={\i} like the Paradise of Indra. Brihada\'sva was the son of the high-souled \'S\=avanta; he had a son Kuvalay\=a\'sva. He became the Lord of the earth by the power of his arms. He killed Dhundu D\=anava; so he was very much celebrated by the name of Dhundum\=ara. His son was Dridh\=a\'sva; he governed the earth; His son was \'Sr\={\i}m\=an Harya\'sva. His son was Nikumbha; he became the King. Nikumbha had his son Varha\d{n}\=a\'sva. Kri\'s\=a\'sva was his son. His son was the powerful Parasenajit; his son's prowess knew no bounds. Parasenajit had the fortunate son Yauvan\=a\'sva. O fortunate One! The son of Yauvana\'sva was \'Sr\={\i}m\=an M\=andh\=at\=a; he became the Lord of the Earth and for the

satisfaction of the Dev\={\i} Bhagavat\={\i} had one thousand and eight palaces built in Benares and in the other chief places of pilgrimages. M\=audh\=at\=a was not born of his mother's womb but was born in the belly of his father. Then the ministers tore asunder the belly of his father and got him out.

42-43. Janamejaya said :-- O fortunate One! What you said was never seen nor heard ever before since. This sort of birth is highly improbable. How was that beautiful son born in the belly of his father? Describe this in detail and satisfy my curiosity.

44-49. Vy\=asa said :-- O King! The King Yauvan\=a\'sva had one hundred queens; yet he had no issues. The King always thought much about his son. Once the King, sorry and desirous of a son, went to the holy retreats of the \d{R}i\d{s}is. On arriving there, he began frequently to respire heavily before the ascetics. The \d{R}i\d{s}is became filled with pity on seeing his sorrowful condition. O King! The Br\=ahmi\d{n}s that said to him :-- O King! Why are you thus sorrowful and distressed? What is your sorrow that is troubling your heart? Speak truly. We will surely redress your grievance.

50-54. Yauvan\=a\'sva said :-- ``O Munis! I have got the kingdom wealth, excellent horses, one hundred illustrious chaste wives. I have no enemies in the three worlds; no one is stronger than me. All the Kings and ministers are obedient to my call. But O Ascetics! I have no son; this my sonless state is the only cause of my pain and sorrow. It has marred all my happiness. See! The persons that have no son cannot in any way go to Heavens. Therefore I am always being pained for this. You all are ascetics; you have taken great pains to learn the essence of the Veda \'S\=astras. So kindly order me what sacrifice is fit for me to have a son. O Ascetics! If you feel any pity for me, kindly perform this good work for me.''

55-65. Vy\=asa said :-- O King! Hearing the words of the King they were all filled with pity; and, with fulness of mind, made him to perform the sacrifice whose presiding deity was Indra. For the sake of the King, that he may get a son born to him, they had a jar filled with water by the Br\=ahmi\d{n}s and purified and charged that jar with the Vedic Mantrams. The King got thirsty in the night and entered in the sacrificial ground; seeing the Br\=ahmi\d{n}s asleep, the King himself drank that water, surcharged with the Mantram. The Br\=ahmi\d{n}s consecrated and kept that water apart, according to due rules, surcharging with Mantrams, for the wife of the King; but the King, getting thirsty, himself drank that water unconsciously. Next morning the Br\=ahmi\d{n}s

seeing the jar of water empty, were startled very much with fear; the Br\=ahmi\d{n}s then asked the King :-- Who drank the water? When they came to know that the King himself drank the water, the Munis thought this to he an act of Daiva (Fate) and completing the sacrifice returned to their abodes. Then the King became pregnant by the power of the sacrificial Mantrams. After some time, the son became fully developed. Then the King's ministers, cutting his right bowel, got the son out. Out of the God's favour, the King did not die. When the ministers were troubled with the thought whose milk the child will suck, then Indra spoke out the child would drink (M\=an-Dh\=at\=a) my forefinger and gave his finger into the child's mouth. For that reason his name was M\=andh\=at\=a. Thus I have described in detail the origin of M\=andh\=at\=a.

Here ends the Ninth Chapter of the Seventh Book on the story of Kakutstha and the origin of M\=andh\=at\=a in \'Sr\={\i} Mad Dev\={\i} Bh\=agavatam the Mah\=a Pur\=a\d{n}am, of 18,000 verses, by Mahar\d{s}i Veda Vy\=asa.



