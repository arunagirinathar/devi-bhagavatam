\chapter{On the battle between Yudh\=ajit and V\={\i}rasena}

1. Vy\=asa said :-- O king! When the war was declared, the two kings, excited by greed and anger, took up arms; and a dreadful encounter ensued.

2. On one side the king Yudh\=ajit of long arms, surrounded by his own army, with bows and arrows came ready to fight.

3. On the other hand, the fiery V\={\i}rasena, the second God of the Devas appeared in the battle, following a true K\d{s}attriya custom, on behalf his daughter's son.

4. Then that truthful king V\={\i}rasena, seeing Yudh\=ajit in battle, became very angry and hurled arrows on him, as a cloud rains on the mountain tops.

5. On being covered, as it were, by the sharp and swift arrows, sharpened on a stone slab hurled at him by V\={\i}rasena, Yudh\=ajit, too, quickly, shot arrows at V\={\i}rasena and cut off all his arrows.

6. O King! A dreadful fight then ensued between the cavalries, the warriors on the elephants; and the Devas, men, and Munis began to witness this terrible battle with wonder and astonishment. Birds, vultures and crows, desirous to eat the flesh of the dead soldiers, flew in the air.

7. Blood of elephants, horses and warriors, the bodies that lay dead, flowed in torrents awfully like rivers in that deadly battle ground. The torrent of blood excited fear amongst those who came to see it, as the river Vaitara\d{n}i on the way to hell (the Lord of Death) is very fearful to the sinners.

8. The human skulls were driven ashore by the current and they look like so many hollow shells of gourds scattered there for the play of the boys on the banks of the Jumn\=a.

9. When any warrior lay dead on the field, the vultures began to fly about in the air for devouring his flesh. It seemed then that the soul of the warrior beholding his beautiful body tried to reenter into his body, though he thought that it had become very inaccessible to him.

10. Some warriors on being slain in the battle instantly arose in a celestial car to the heavens and was seen addressing the celestial nymph, who came already within his embrace, thus ``O one of beautiful thighs. Behold! how my beautiful body is lying on the earth below!''

11. Another warrrior thus slain got up in the heavens on a celestial car, came in possession of a celestial nymph and when he was sitting with her in the car, his former wife in the earth made herself a sati and burnt herself up in the funeral pyre, thus got a celestial body, came up to the heavens; and that chaste virtuous woman drew away perforce her own husband away from that celestial nymph.

12. Two warriors, went up, slew each other and lay down dead at the same time. They went up in the heavens at the same time and there began quarrel with each other and fight with their weapons for one and the same celestial nymph.

13. Some hero got in the heavens a nymph more lovely and beautiful than himself and he thus became very much attached and devoted her. He began to describe his own heroic qualities and also to copy dotedly the qualities of his lover so that she might remain faithfully attached to him.

14. The dust, arising from the dreadful encounter of the soldiers in battle field, rose up in the air and covered the sun. It appeared night. After a while that dust became absorbed in the blood below, and the sun appeared very red, reflected with the colour of the blood.

15. Some Brahmach\=ar\={\i} fought in the battle and was slain. He went up to the heavens; instantly a lovely eyed Devakany\=a, a celestial nymph desired to select him as bridegroom with great devotion. But that clever man did not accept the offer, thinking that his vow of Brahmacharya would be thus broken.

16-17. O King! Thus when the battle was deadly, the king Yudh\=ajit shot a sharp, dreadful arrow at V\={\i}rasena and severed his head from body. V\={\i}rasena lay dead on the battle field and his army was routed. The soldiers fled away from the battle.

18-19. Hearing that her father was slain in the battle, Manoram\=a became very terrified and anxious. She then began to think that the vicious wicked king Yudh\=ajit will surely slay her son, for kingdom's sake and to satisfy his enmity with her father.

20. What shall I do now? My father is slain in the battle. My husband is no more. My child is a minor to-day. Where shall I go?

21. Greed is very sinful; who is there that cannot be bought over by love of gold? and what vicious act can there be, that cannot be done when actuated by covetousness?

22. A greedy man does not hesitate to kill his father, mother, spiritual guide, friends and others. There is no doubt in this.

23. It is the inordinate love of worldly things that makes a man eat what is held unclean in society, that makes a man approach a woman who is unapproachable, and it is greed that makes a man discard his own religion and become an apostate.

24. In this city I find none so powerful as I can remain there under his shelter and be able to rear up my child.

25. What can I do if the king Yudh\=ajit slay my son? There is none in this world who can save me, and, counting on whose shelter, I can stay here without any anxiety.

26. And this my co-rival wife L\={\i}l\=avat\={\i} will always practise enmity with me. She will never shew mercy on my son.

27. When Yudh\=ajit will arrive in this city, I will never be able to go out of it and he will to-day put my son in the prison on the pretext that he is a minor.

28-29. I heard that, in days of yore, Indra entered into the womb of his pregnant step mother with a small thunderbolt in his hand and divided the foetus into seven parts with that weapon, again each of these seven into seven parts again, thus the forty nine Maruts were born in the Heavens.

30-31. I heard also that in ancient times one queen gave poison to destroy the foetus in the womb of her rival wife. When the child came out of the womb, he was celebrated by the name of Sagara (with poison) in this earth.

32. The husband was alive, and still his queen Kaikeyi banished the eldest son of his king, \'Sr\={\i} R\=amchandra to the forest; and the king Da\'saratha sacrificed his life for that very reason.

33. The ministers no doubt wanted before to install my son as the king; but now they are not independent; they have now yielded themselves to the king Yudh\=ajit.

34. There is no brother of mine powerful enough to release me from my bondage; I see I have fallen into a great difficulty by the combination of unforeseen circumstances.

35. Though the success depends on Fate, still one should make an earnest effort. If one does not make any effort, fate also remains asleep. I will therefore soon make out a plan to save my son.

36-38. O King! Thinking thus, that woman Manoram\=a called in private the best and very respectable minister Vidalla, who was intelligent and expert in everything, and holding the hands of her son and weeping, said humbly in a depressed spirit ``O Minister! My father is slain in the battle field, this my son is a minor, and Yudh\=ajit is a powerful king; consider all these and tell me what I should do now?''

39-40. The venerable minister Vidalla then said to the queen Manoram\=a ``It is never advisable for us to stay here. Soon we will go into the forests of Benares. There I have got my powerful uncle Sub\=ahu. He is prosperous and has got a strong army. He will protect us.''

41. ``I will make the pretext that I am become very anxious for the king and therefore I am going out to see the king Yudh\=ajit and will go out of the city in my chariot. There is no doubt in this.''

42-43. Hearing, thus, the Vidalla's words, the queen Manoram\=a went to L\={\i}l\=avat\={\i} and said ``O faireyed! To-day I am going to see the father Yudh\=ajit.'' Thus saying, she went out of the city in a chariot, accompanied by her son, attendants and Vidalla.

44-45. Grieved at the loss of her father, fearful, distressed, and fatigued, Manoram\=a saw Yudh\=ajit and performed the cremation of her father V\={\i}rasena; and, trembling with fear, got to the banks of the Ganges after two day\'s swift journey.

46-48. There the robbers, the Nis\=adas plundered all their riches and took the chariot and went away. Manoram\=a had only her clothings, that she wore, left to her. She began to weep, and, holding the hands of her attendant, went to the Ganges shore, and being afraid crossed the river on a raft and went to the Chitrak\=uta mountain.

49. That terrified Dev\={\i} went to the hermitage of Bh\=aradv\=aja as early as possible. There she saw the ascetics and was relieved of her fear.

50. Bh\=aradv\=aja asked, ``O lotus eyed! Who are you and whose wife are you? Why have you taken so much trouble to come here? Answer all these truly.''

51. ``O beautiful one! are you a Dev\={\i} or a human being? your son is a very minor. Why have you come in this dense forest? It seems, as if you are deprived of your kingdom.''

52. Thus asked by the best of the Munis, the beautiful Manoram\=a became very much afflicted with grief and began to weep; she could not speak anything herself and ordered Vidalla to inform the Muni all what had happened.

53-54. Vidalla then said :-- There was a king of Kosala, named Dhruvasandhi. She is the legal wife of that king. Her name is Manoram\=a. That powerful king of the Solar Dynasty was killed by a lion in a forest. This boy Sudar\'sana is his son.

55. The father of this Manoram\=a was very religious. He died fighting for the cause of his daughter's son. Now the present queen has become much afraid and has therefore come to this wild forest.

56. The son of this woman is now a minor; he is now taking your refuge. O best of the Munis! Protect them.

57. To give protection to any distressed person is to acquire merits higher than performing a sacrifice. Therefore to protect one who is very much afflicted with fear and who is helpless will have still higher merits.

58. Bh\=aradv\=aja said :-- ``O beautiful one! Remain in this hermitage without any fear; rear up your son here. O auspicious one! There is no cause of fear here from your enemies.

59. Better nourish and support your child. Your son will surely be a king and if you remain in this hermitage, no sorrow or grief will overtake you.''

60. Vy\=asa said :-- When the great Muni Bh\=aradv\=aja said thus, the queen Manoram\=a became peaceful. The Muni gave them a cottage to live in and there they dwelt without any sorrow.

61. Thus Manoram\=a dwelt obediently with her maid servant, liked by all. Vidalla also remained there and Manoram\=a began to nourish her child.

Here ends the Fifteenth chapter on the Dev\={\i} M\=ah\=atmya and the battle between Yudh\=ajit and V\={\i}rasena and the going away of Manoram\=a to the forest in the 3rd Adhy\=aya of \'Sr\={\i} Mad Dev\={\i} Bh\=agavatam by Mahar\d{s}i Veda Vy\=asa.



