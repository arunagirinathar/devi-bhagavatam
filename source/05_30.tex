\chapter{On the killing of Ni\'sumbha}

1-10. Vy\=asa said :-- O King! Thus making a firm resolve that there would be either victory or death, the great warrior Ni\'sumbha went to fight with the Dev\={\i}, with great excitement and with all his forces. \'Sumbha, too the Lord of the Daityas, accompanied by his forces, went after Ni\'sumbha; \'Sumbha knew full well the rules of warfare; therefore he remained a witness there. Indra and the other Devas and Yak\d{s}as, all stationed themselves in the celestial space, eager to see that fight, covered with clouds. Ni\'sumbha came to the field, and, taking the strong bow made of horns, began to shoot arrows after arrows at the Divine Mother with the object of frightening Her. Seeing Ni\'sumbha with his excellent bow, shooting arrows, Chandik\=a began to laugh frequently. With a soft slow voice She spoke to K\=alik\=a :-- ``O K\=al\={\i}! See their foolishness! They have come before me, courting death. They are so much deluded by My M\=ay\=a, that they yet expect victory when they have already witnessed the death of Raktab\={\i}ja and many D\=anavas. Hope is so very strong that it never quits a man. How wonderful is this that some of their armies are destroyed, some are wounded, some are rendered senseless, some made powerless, some have fled; seeing all these, yet, they have come to fight, as it were, fastened by the cord of hope of victory. O K\=al\={\i}! Today I will certainly slay Ni\'sumbha and \'Sumbha. Their death is nigh; deluded by the Daiv\={\i} M\=ay\=a, they have come to Me. Therefore, in the face of all the Devas, I will kill them today.''

11-24. Vy\=asa said :-- O King! Thus saying, and suddenly drawing Her bow, Chand\={\i} covered Ni\'sumbha, in front, all over with a multitude of arrows. Ni\'sumbha, too, cut off those arrows into pieces by his sharp arrows; thus the fight became more and more dreadful between them. At this time, the lion of Bhagavat\={\i}, came down upon the forces, quivering his manes, like a powerful elephant going down into a lake. By his nails and teeth, he tore asunder the bodies of the D\=anavas that fell before him and devoured them, as if they were infatuated elephants. That lion thus crushing down the soldiers, Ni\'sumbha came forward hurriedly, drawing his excellent bow. Hundreds of other generals of the D\=anavas came up there to kill the Dev\={\i}, biting their lips and with their eyes reddened with anger. In the meantime \'Sumbha killed K\=alik\=a and came very hurriedly there with his forces to capture the Divine Mother. Coming to the battlefield \'Sumbha saw that the Divine Mother was

standing before him; though She was looking very beautiful, fit for love sentiment, yet She was filled also with the sentiment of fiery wrath. At that time the large eyes of Bhagavat\={\i}, the Beautiful in the three worlds, though naturally red, looked more red due to wrath. When \'Sumbha saw Her lovely features, the desire to marry Her and the hope of victory all vanished away from his mind; and he stood there with bow in his hand, firmly holding in his mind that he would die. Seeing the D\=anava in that state, She smiled and began to say, so that all the D\=anavas could hear. O Wretched Fools! If you all want to live, quit all your weapons here, go to the P\=at\=ala or to the middle of the ocean. Or be slain in the battle by My arrows and go to heavens and enjoy there without any fear all the enjoyments and sports there. Weakness and heroism, both cannot be expected at one and the same time and at the same individual; therefore I am ordering you to dispel your fears. Now go wherever you find your ease and happiness.

25-35. Vy\=asa said :-- O King! Hearing these words of the Dev\={\i}, that haughty Ni\'sumbha ran forward, holding in his hand the sharpened axe and shield adorned with eight Chandras (embossed) and firstly struck with sword violently on the proud lion; then, whirling with great force that sword, hurled it upon the Divine Mother. The Dev\={\i}, then, thwarted off the blow of the sword by Her club and struck at his arm by Para\'su. The warrior Ni\'sumbha, thus struck at his arm, patiently bore that hurt and struck at Chandik\=a by his axe. The Dev\={\i} then made such a terrible noise of bells that all the Daityas were struck with terror. Then She, desiring to kill Ni\'sumbha, began to drink nectar frequently. O King! Thus the terrible fight went on between the Devas and D\=anavas both trying to defeat the other party. Then began to dance in the battlefield, the cruel voracious dogs, jackals, vultures, herons, crows and other birds, very much gladdened. The battlefield was drenched with blood and the dead carcasses of innumerable D\=anavas, elephants, and horses. Ni\'sumbha, then, seeing the D\=anavas dead on the field, became very angry and ran forward with his terrible club before the Dev\={\i}. That proud Asura struck first at the head of the lion with that club and laughed again and again and struck the Dev\={\i} with that same club. The Dev\={\i}, too, got very angry seeing Ni\'sumbha before Her and striking
at Her. She then spoke thus :--

36. O You Stupid! Wait till I sever your head from your body by this axe. Soon you will be sent unto death with your head severed off your body.

37-64. Vy\=asa said :-- O King! Thus saying the Chandik\=a Dev\={\i} instantly cut off the head of Ni\'sumbha by Her axe with great caution.

The head thus severed from the body by the blow of the Dev\={\i}, the headless Demon began to roam there with great force with club in his hand. The Devas then got very much frightened; The Dev\={\i}, then, cut off the hands and feet of that headless Demon with sharpened arrows. That vicious wretch fell down lifeless, on the ground like a mountain. The powerful Daitya Ni\'sumbha being thus killed, a great uproar arose amidst his panic stricken forces. The soldiers, covered all over their bodies with blood, left all their weapons in the field, began to make Boomb\=a sound (a piteous cry with mouth and hands as sign of danger) and fled away to the King \'Sumbha. He, the tormentor of the foes, then asked them coming :-- ``Where is Ni\'sumbha now? Why have you fled away from the field?'' Thus hearing the King's words, they bowed down and said :-- O King! Your brother Ni\'sumbha is lying dead on the battlefield. O King! The Dev\={\i} killed all the D\=anava warriors that attended your brother; only we are left and have come here to give you the information. O King! Ni\'sumbha has been killed by the weapons of the Dev\={\i}. So we think you ought not to go to the battle. Know this as certain that the Lady, the Highest Cause of this Universe has come here to destroy the D\=anavas, the object being to serve the cause of the Gods. This Lady is not an ordinary woman; She is the Supreme Force; Her doings are inconceivable; what more can be said than the fact that the Devas never can know Her! This Dev\={\i} can assume various forms; She is the origin of M\=ay\=a; She is very clever; She is adorned with various ornaments and is holding various weapons in Her hands. Her doings are incomprehensible; She is like a Second Night of Dissolution (at the end of the world); She is Perfect, endowed with all auspicious signs, capable to go beyond the insurmountable. This wonderful Dev\={\i} is serving the cause of the gods and the Devas from the sky are singing hymns to Her. O King! It is now your paramount duty to fly away and save your life; if you live, you may have the chance for gaining the victory when time will turn out favourable; there is no doubt in this. It is Time that makes a strong man weak; and it is that very Time that makes that weak man strong again and stimulates him for victory. Time makes a generous donor a beggar and it is Time that makes the same beggar again a generous donor. Brahm\=a, Vi\d{s}\d{n}u, Mahe\'sa, Indra and other Devas are all under the sway of this Time; so Time is the Sovereign of all. Therefore, O King! Wait for this Time. Now Time is favourable to the Gods and inimical to you. Therefore Time is destroying now the Daityas. But the course of Time is not the same throughout. O King! The actions of Time are various no doubt. Time creates men and Time destroys them. The time of

creation is different from the time of destruction, this is evident to you before your eyes. See! When Time was favourable to you, you subject Indra and all other Devas and made them pay taxes to you; and now Time is unfavourable to you; so an ordinary weak woman is killing the powerful D\=anavas; Time, therefore, is doing favourable things and also unfavourable things. The host of Devas or the woman K\=al\={\i} is not the cause thereof. O King! The present Time is not favourable to you and the Daityas; knowing this, do as you like. See! Indra, Vi\d{s}\d{n}u, Varu\d{n}a, Yama and other prominent Devas all fled before in battle, quitting the weapons. So, knowing this world as subject to the control of Time, you can now fly away and go quickly to the P\=at\=ala. For if you live, you will get in future all the pleasures; and if you be killed, your enemies will all be very glad and roam everywhere fearlessly, singing propitious songs.

Here ends the Thirtieth Chapter of the Fifth Book on the killing of Ni\'sumbha in \'Sr\={\i} Mad Dev\={\i} Bh\=agavatam, the Mah\=a Pur\=a\d{n}am, of 18,000 verses by Mahar\d{s}i Veda Vy\=asa.



