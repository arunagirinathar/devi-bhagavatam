\chapter{On the story of Satyavrata}

1-11. Vyasa said :-- O King! That King M\=andh\=at\=a, true to his promise, conquered one after another the whole world and became the paramount sovereign of all the other emperors and got the title ``S\=arvabhauma'' (Sovereign of all the earth). O King! What more to speak of M\=andh\=at\=a's influence at that time than this that all the robbers, struck with his terror, all fled to the mountain caves. For this reason, Indra gave him the title ``Trasadasyu.'' He married Bindumat\={\i}, the daughter of \'Sa\'savindu. Her limbs were proportioned and perfect and so she was very beautiful. M\=andh\=at\=a had by that wife two sons :-- (1) the famous Purukutstha and (2) Muchukunda. Purukutstha had his son Anara\d{n}ya; this prince was celebrated by the name of Brihada\'sva. He was very religious and deeply devoted to his father. His son was Harya\'sva; he was religious and knew the Highest Reality. His son was Tridhanv\=a; his son was Aru\d{n}a. Aru\d{n}a's son was Satyavrata; he was very avaricious, lustful, wicked and wilful. Once on an occasion that vicious prince, overpowered by lust, stole away the wife of one Br\=ahmi\d{n} and so created an hindrance in his marriage. O King! The Br\=ahmi\d{n}s, united in a body, came to the King Aru\d{n}a, bewailing and lamenting and uttered repeatedly :-- Alas! We are ruined! The King addressed to the grieved subjects, the Br\=ahmi\d{n}s :-- ``O Br\=ahmi\d{n}s! What harm has been done to you by my son.''

Hearing thus the good words of the King, the Dvijas, versed in the Vedas, repeatedly blessed him and said :--

O King! You are the foremost of the powerful. So your son is like you. Today he has forcibly stolen away during the marriage ceremony a Br\=ahmi\d{n} daughter already given over in marriage.

12-36. Vyasa said :-- O King! The highly religious King hearing the words of the Br\=ahmi\d{n}s, took them to be true and said to his son :-- ``O One of evil understanding! You have rendered to-day your name useless by perpetrating this evil act. O Vicious One! Get away from my house! O Sinner! You will never be able to live in my territory!'' Seeing his father angry, Satyavrata repeatedly said :-- Father! Where shall I go? He said :-- ``Live with the Chandalas. You have stolen a Br\=ahmi\d{n}'s wife and so has acted like a Ch\=andala. Go and live with them happily. O Disgrace to your family! I don't like to get issues through you: you have obliterated this family's name. So, O Sinner! go wherever you like.'' Hearing the the words from his angry father, Satyavrata instantly quitted the house and went to the Ch\=and\=alas. The prince, wearing his coat of armour and holding bows and arrows, began to spend away his time with the Ch\=and\=alas; but he could not get out of his breast his feeling of sympathy and mercy. When he was banished by his liberal minded angry father the Guru Va\'sistha instigated the King to the above purpose. Satyavrata was therefore angry with Va\'sistha, inasmuch as he, versed the Dharma \'S\=astras, did not dissuade the father from banishing his son. His father, then, owing to some inexplicable cause, quitted the city and, for the sake of his son, went to the forest to practise austerities. O King! Owing to that sinful act, Indra did not rain at all in his kingdom for twelve years. O King! Just then Vi\'sv\=amitra, too, keeping his wife and children in that kingdom, began to practise severe austerities on the banks of the river Kau\'sik\={\i}. The beautiful wife of Ku\'sika then fell into great trouble how she could maintain the family. All the children, pained with hunger, began to cry, begging for Nib\=ar rice food. The chaste wife of Kau\'sika became very much troubled seeing all this. She thought, seeing the children hungry, ``Where am I to go now and from whom to beg, and what to do, inasmuch as the King was not then staying in the Kingdom. The husband is not also near; so who would protect my children? The boys are incessantly crying. Fie therefore to my life!'' She thought also thus :-- ``My husband left me in this penniless state; we are suffering for want of money. He does not know these, though he is quite able. Save my husband, who else will support my sons? They will all die now of starvation. I might sell one of my sons, whatever I get out of that, I can support the others; this is now my highest duty. I ought not to do otherwise

and kill all my children; so I will now sell one of my sons to support the others.'' Thus hardening her mind, she went out, tying the child by a rope round his neck. The Muni's wife, for the sake of the other children, fastened the middle son by a cord and got out of her house. The prince Satyavrata saw her distressed with pain and sorrow and asked :-- ``O Beautiful One! What are you now going to do? Who are you? This boy is crying; Why have you tied him by a rope round his neck? O Fair One! Speak out truly to me the cause of all this.''

37-38. The wife said :-- ``O Prince! I am the wife of Vi\'sv\=amitra. These are my sons. I am now going, for want of food, to sell one of these out of my own accord. O King! My husband has gone away to practise tapasy\=a; I do not know where he has gone. There is no food in the house; so I will sell one to support the other sons.''

39-56. Satyavrata said :-- ``O Chaste One! Save your children. I will bring to you your articles of food from the forest till your husband does not come here. Daily I will fasten some food on a tree close by your \=A\'srama. This I speak truly.'' The wife of Vi\'sv\=amitra, hearing these words of the prince, freed the child of the fastening and took him to her \=A\'srama. The child was named afterwards as G\=alaba, due to his being fastened by the neck. He became a great \d{R}i\d{s}i afterwards. The Vi\'sv\=amitra's wife then felt great pleasure in her home, surrounded by her children. Filled with regard, and mercy, Satyavrata duly performed his task and provided daily the family of Vi\'sv\=amitra with their food. He used to hunt wild boars, deer, buffaloes, etc., and used to take their flesh to the place where used to dwell the wife of Vi\'sv\=amitra and the children and tie that up to an adjoining tree. The \d{R}i\d{s}i's wife used to give those to her children. Thus getting excellent food, she felt very happy. Now when the King Aru\d{n}a went for tapasy\=a to the forest, the Muni Va\'sistha carefully guarded the Ayodhy\=a city, and the palace and the household. Satyavrata, too, used to sustain his livelihood daily by hunting, accordig to his father's order; and abiding by Dharma, lived in the forest outside the city. Satyavrata cherished always in his heart, for some cause, a feeling of anger towards Va\'sistha. When his father banished his religious son, Va\'sistha did not prevent his father. This is the cause of Satyavrata's anger. Marriage does not become valid until seven footsteps are trodden (a ceremony); so the stealing away of a girl within that period is not equivalent to stealing away a Br\=ahmi\d{n}'s wife. The virtuous Va\'sistha knew that; yet he did not prevent the King. One day the prince did not find anything for hunting; he saw in the

forest the cow of Va\'sistha giving milk. Very much distressed by hunger, the King killed the cow like a dacoit, partly out of anger and partly out of delusion. He fastened part of the flesh to that tree for the wife of Vi\'sv\=amitra and the remainder he ate himself. O One of good vows! The Vi\'sv\=amitra's wife did not know that to be beef and thought it to be deer's and so fed her sons with that. Now when Va\'sistha came to know that his cow had been killed, he was inflamed with anger and spoke to Satyavrata ``O Vicious One! What a heinous crime have you committed, like a Pi\'s\=acha, by killing the cow? For the killing of the cow, the stealing of a Br\=ahmi\d{n}'s wife and the fiery anger of your father, for these three crimes, let there come out on your head three \'Sankus or three marks of leprosy as the signs for your crimes. From this day you will be widely known by the name of Tri\'sanku and you will show your Pi\'s\=acha form to all the beings.''

57. Vy\=asa said : -- O King! The prince Satyavrata thus cursed by Vai\'sistha remained in that retreat and practised severe tapasy\=a.

58. But he got from a Muni's son the excellent Mantram of the Highest auspicious Dev\={\i} Bhagavat\={\i} and became merged in the contemplation of that.

Here ends the Tenth Chapter of the Seventh Book on the story of Satyavrata in the Mah\=apur\=a\d{n}am \'Sr\={\i} Mad Dev\={\i} Bh\=agavatam of 18,000 verses, by Mahar\d{s}i Veda Vy\=asa.



