\chapter{On the going to Heavens of Tri\'sanku and the commencement of Hari\'schandra's narrative}

1-8. Vy\=asa said :-- O King! Settling in his mind what to do, the great ascetic Vi\'sv\=amitra collected all the materials necessary for the sacrifice and invited all the Munis. Thus invited by Vi\'sv\=amitra, the Munis became informed all about the Sacrifice; but, owing to the

fact that the Muni Va\'sistha prevented them, none of them went to the sacrifice. When Vi\'sv\=amitra, the son of G\=adhi, came to know this, he became very anxious and very sad and came to the King Tri\'sanku and sat. The Maharsi Kau\'sika then became angry and said :-- ``O King! Va\'sistha preventing the Br\=ahmi\d{n}s have all refused to come to the sacrifice. But, O King! See my power of tapasy\=a; I will immediately fulfil your desires; I will instantly send you to the Heavens, the abode of the Gods.'' Thus saying, that Muni took water in his hand and repeated the G\=ayatr\={\i} Mantram. He gave to the King all the Pu\d{n}yams (merits) that he collected for himself up to then. Giving him thus all the Pu\d{n}yams, he spoke to the King :-- ``O King! Throw away all idleness and go to the abode of the Gods you wanted to go. O King of Kings! Gladly go to the Heavens by the power of all the merits collected by me for a long time and let you fare well there.''

9-20. Vy\=asa spoke :-- O King! When the King of the Vipras, Vi\'sv\=amitra, spoke thus, the King Tri\'sanku, by virtue of the Muni's Tapas, got high up in the air without any delay like a quick flying bird. Thus getting up and up, when the King reached the abode of Indra, the Devas, seeing the terrible Ch\=and\=ala-like appearance of Tri\'sanku, spoke out to Indra :-- ``Who is this person coming like a Deva with a violent speed in the air? Why does he look like a Ch\=and\=ala and is so fierce-looking?'' Hearing thus, Indra got up at once and saw that one, the meanest of the human beings and knowing him to be Tri\'sanku, reproachingly said to him :-- You are a Ch\=and\=ala, quite unfit for the Devaloka; so where are you going? You ought not to remain here; so go immediately back to the earth. O Destroyer of the enemies! Indra speaking thus, the King dropped from the Heavens and, like a Deva whose merits had been exhausted, fell down immediately. Tri\'sanku then cried out frequently ``O Vi\'sv\=amitra! O Vi\'sv\=amitra! Being displaced from the Heavens I am now falling very violently; so save me from this trouble.'' O King! Hearing his cry and seeing him getting down, Vi\'sv\=amitra said :-- ``Wait, wait.'' Though displaced from Heaven, the King by virtue of the Muni's Tapas, remained stationed at that place in the middle of the air. Vi\'sv\=amitra then began to do \=Achaman (sip water) and commenced his great Sacrifice to create another new creation and a second Svargaloka (Heaven). Seeing his resolve, the Lord of \'Sach\={\i} became very anxious and eagerly came to the son of G\=adhi without the least delay and said :-- ``O Br\=ahma\d{n}a! What are you going to do? O Saint! Why are you so very angry? O Muni!

There is no necessity to create another new creation. Order now what I am to do.''

21. Vi\'sv\=amitra said :-- ``O Lord of the Devas! The King Tri\'sanku has become very miserable to have a fall from the Heavens. Therefore this is now my intention that you gladly take him to your own abode.''

22-31. Vy\=asa said :-- O King! Indra was thoroughly aware of his determined resolve and very powerful asceticism; so he accepted to do according to his word, out of terror. The Lord Indra then gave the King a bright and divine body and made him take his seat in an excellent car and taking leave of Kau\'sika went with the King to his own abode. Vi\'sv\=amitra became glad to see Tri\'sanku go to the Heavens with Indra and remained happy in his own \=A\'srama. The King Hari\'schandra now hearing that his father has gone to Heaven by virtue of his Tapas, began to govern his kingdom with a gladdened heart. The King of Ayodhy\=a began then to live constantly with his clever wife full of youth and beauty. Thus time passed away; but the beautiful wife did not become pregnant. The King became very sorry and thoughtful. He then went to the holy hermitage of Va\'sistha and bowing down informed him of his mental agony due to his getting no son. O Knower of Dharma! You are skilled in the Science of Mantrams. Especially you know everything of Daiva (Fate). So, O Giver of honour! Do for me so that I get a son. O Best of Br\=ahmi\d{n}s! There is no salvation for one who has not got any son; you are well aware of this. Then why do you overlook my case when you can remove my sorrow. Even these sparrows are blessed who nourish their offsprings. And I am so very unfortunate that, day and night, I am immersed in cares and anxieties, due to my not having any son.

32. Vy\=asa said :-- O King! Hearing these pitiful utterances of the King, Va\'sistha thought over in his mind and spoke to him everything in particular.

33-41. Va\'sistha said :-- ``O King! True you have spoken that in this world there is no other sorrow more painsgiving than the state of not having any issue. Therefore, O King! you worship with great care the water-god Varu\d{n}a. He will crown your efforts with success. There is no other god than Varu\d{n}a to grant sons. So, O Virtuous One! Worship Him and you will get success. Both Fate and Self-exertion are to be respected by men; how can success come unless efforts are made. O King! Men who realise the Highest Truth should make efforts, guided by just rules; success comes to those who work; else never one is to

expect success.'' Hearing these words of the Guru, of unbounded energy, the King made a fixed resolve and bowing himself down, went away to practise tapasy\=a. On the banks of the Ganges, in a sacred place, seated on Padm\=asan, the King became merged in the meditation of the God Varu\d{n}a with noose in his hand and thus practised severe asceticism. O King! When he was doing this, the god Varu\d{n}a took pity on him and gladly came before his sight. Varu\d{n}a, then, spoke to the King Hari\'schandra :-- ``O Knower of Dharma! I am glad at your tapasy\=a. So ask boons from me.''

42-43. The King said :-- ``O God! I am without any son; give me a son, who will give me happiness and will free me from the three debts that I owe to the Devas, the Pitris and the \d{R}i\d{s}is. Know that with that object I am doing this Tapasy\=a.'' Then the God Varu\d{n}a, hearing these humble words of the sorrowful King, smiled and said.

44-45. O King! If you get your desired well-qualified son, what will you do for me to my satisfaction? O King! If you perform a sacrifice in honour of me and fearlessly sacrifice your son there like an animal, I will then grant you your desired boon.

46-47. The King :-- ``O Deva! Free me from this state of sonless-ness; O Water God! When my son will be born, I will do your sacrifice with my son as an animal in that. This I speak truly to you. O Giver of honour! There is no suffering more unbearable than this one, not to have any son; so grant me a good son so that all my sorrows be vanished.''

48. Varu\d{n}a said :-- ``O King! You will get a son as you desire; go home; but see what you have spoken before be fulfilled and turned true.''

49-55. Vy\=asa said :-- Hearing these words from Varu\d{n}a, Hari\'schandra went back and told everything about his getting the boon to his wife. The King had one hundred exquisitely beautiful wives of whom, \'Saivy\=a was the lawful wife and queen and was very chaste. After some time, that wife became pregnant and the King became very glad to hear this and her longings in that state. The King performed all her purificatory ceremonies, and when ten months were completed, and on an auspicious Nak\d{s}atra and on an auspicious day, she gave birth to a son, like that of a Deva son. On the birth of his son, the King, surrounded by the Br\=ahmi\d{n}s, performed his ablutions and first of all performed the natal ceremonies and distributed innumerable jewels and much

wealth; and the King's joy knew no bounds at that time. The liberal King gave away, in special charities, wealth, grains, and various jewels and lands and had the performance of music, dancing and other things.

Here ends the Fourteenth Chapter of the Seventh Book on the going to Heavens of Tri\'sanku and the commencement of Hari\'schandra's narrative in the Mah\=a Pur\=a\d{n}am \'Sr\={\i} Mad Dev\={\i} Bh\=agavatam of 18,000 verses, by Mahar\d{s}i Veda Vy\=asa.



