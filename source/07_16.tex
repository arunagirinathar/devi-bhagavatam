\chapter{On the story of \'Sunah\'sepha}

1-4. Vy\=asa said :-- O King! When Varu\d{n}a went away, the King was very much laid down with that dropsy and daily his pains began to increase and he began to suffer extreme pains. O King! The prince, on the other hand, heard, in the forest, of the illness of his father and filled with affection, wanted to go to his father. A year had passed away and the prince desired very gladly to go to his father and see him. Knowing this, Indra came there. He came instantly in the form of a Br\=ahmi\d{n} and with favourable arguments desisted the prince, who was about to go to his father.

5-31. Indra said :-- ``O Prince! It seems you are silly; you know nothing of the difficult state policies. Therefore it is that you are ready to go, out of sheer ignorance, to your father. O Fortunate One! If you go there, your father will get his sacrifice, where a human victim is to be offered, performed by the Vedic Br\=ahma\d{n}as and your flesh will be offered are oblations to the blazing Fire. O Child! The souls of all the beings are very dear; it is for that reason, for the sake of soul, that sons, wife, wealth and jewels are all dear. Therefore, though you are his dear son, like his son, yet he will certainly have you killed and get Homas offered, to free himself from the disease. O Prince! You ought not to go home now; rather when your father dies, you would certainly go there and inherit your Kingdom.'' O King! Thus hindered by V\=asava, the prince remained in that forest for one year more. But when the prince again heard of the severe illness of his father, he wanted again to go to his father, resolved to court the death of his ownself. Indra also came there in the form of a Br\=ahma\d{n} and, with reasonable words, repeatedly advised him not to go there. Here, on the other hand, the King Hari\'schandra became very much distressed and troubled by the disease and asked his family priest Va\'sistha Deva :-- ``O Br\=ahma\d{n}a! What is the sure remedy for the cure of the disease?'' Va\'sistha, the Brahm\=a's son, said :-- ``O King! Purchase one son by giving his value; then perform the sacrifice with that purchased son and you will be free from the curse. O King! The Br\=ahmi\d{n}s, versed in the Vedas, say that sons are of ten kinds, of whom the son, purchased by paying its proper value, is one of them. So buy one son. There will very probably be within your kingdom a Br\=ahmi\d{n} who might sell out of avarice, his son. In that case Varu\d{n}a Deva will certainly be

pleased and grant your happiness.'' Hearing these words of the high-souled Va\'sistha, the King became glad and ordered his minister to look after such a son. There lived in that King's dominion one Br\=ahmi\d{n}, named Ajigarta, very poor; he had three sons. The minister spoke to him to purchase his son :-- ``I will give you one hundred cows; give one son of yours for the sacrifice. You have three sons named respectively \'Sunahpuchcha, \'Sunah\'sepha and \'Sunolangula. Give me out of them one son and I will give you one hundred cows as his value.'' Ajigarta was very much distressed for want of food; so when he heard the proposal, he expressed his desire to sell his son. He thought that his eldest son was the rightful person to perform funeral obsequies and offer Pinda and he therefore did not spare him. The youngest son, too, he did not spare also, as he considered that his own. At last, he sold his second son for the price of one hundred cows. The King then bought him and made him the victim for the sacrifice. When that boy was fastened to the sacrificial post, he began to tremble and very much distressed with sorrow began to cry. Seeing this, the Munis cried out in a very pitiful tone. When the King gave permission for the immolation of that boy, the slaughterer did not take weapons to slaughter him. He told that he would never be able to kill the boy, since he is crying in a very pitiful tone. When he thus withdrew himself from his work, the King then asked his councillors :-- O Devas! What ought to be done now? \'Sunah\'sepha then began to cry in a very pitiful voice; the people present there began to discuss and there arose a great noise on the affair. Then Ajigarta stood up in the midst of the assembly and spoke :-- ``O King! Be patient; I will fulfil your desire. I am desirous of wealth and if you give me double the amount, I will slay immediately the victim; and you can complete early your sacrifice.'' O King! He who is hankering after money, can always entertain
feelings of enmity even towards his own son. There is no doubt in this.

32-35. Vy\=asa said :-- O King! Hearing those words of Ajigarta, Hari\'schandra gladly spoke to him :-- ``I will immediately give you another hundred excellent cows.'' Hearing thus, the son's father, avaricious of wealth, immediately resolved and became ready to slay his son. All the councillors seeing the father ready to slay his son, were struck with sorrow and began to lament exclaiming ``Alas! This wretch, a disgrace to his family, is now ready to kill his own son. Oh! We never saw before such a cruel vicious person. This Br\=ahmi\d{n} must be a Demon in a Br\=ahmi\d{n} body!

36-38. Fie on you! O Ch\=and\=ala! What a vicious work are you now going to do? What happiness do you derive by slaying the son, the jewel of jewels, only to get some wealth? O Sinner! It is stated in the Vedas that the soul takes its birth from one's body; so how are you going to slay your soul!'' When the hue and cry arose in the assembly, Vi\'sv\=amitra, the son of Kau\'sika, went to the King and, out of pity, said :--

39-56. O King! \'Sunah\'sepha is very piteously crying; so let him be free; and then your sacrifice will be complete and you will be free of your disease. There is no virtue like mercy and there is no vice like killing (Hims\=a). What is written about killing animals in the sacrifice, is only meant for the persons inclined to sensual objects and to give them a stimulus in that direction. O King! He who wants his own welfare and who wants to preserve his own body ought not to cut another's body. He who pities equally all the beings, gets contended with a trivial gain and subdues all his senses; God is soon pleased with him. O King! You should treat all the J\={\i}vas like yourself and thus always spend your life, so dear to all. You desire to preserve your body by taking away the life of this boy; similarly why would he not try to preserve his own body, the receptacle of happiness and pleasures. O King! You have desired to kill this innocent Br\=ahmi\d{n} boy; but he will never overlook this enmity of yours done in previous lives. If anybody kills another willingly, though he has got no enmity with him, then the one that is killed will certainly kill afterwards the slayer. His father, out of greed for money, is deprived of intellect and so has sold away his son. The Br\=ahmi\d{n} is certainly very cruel and sinful. There is no doubt in this. When one goes to Gay\=a or one performs an A\'svamedha sacrifice or when one offers a blue bull (Nila Vri\d{s}abha), one does so on the consideration that one would desire to have many sons. Moreover the King has to suffer for one-sixth of the sins committed by anyone in his Kingdom. There is no doubt in this. Therefore the King ought certainly to prohibit any man when he wants to do a sinful act. Why then did you not prevent this man when he desired to sell his son? O King! You are the son of Tri\'sanku; especially you are born in the Solar line of Kings. So how have you desired, being born an \=Ary\=a, to do an act becoming an An-\=Ary\=a (non-aryan). If you take my word and quickly free this Br\=ahmi\d{n} boy, you will certainly derive virtue in your body. Your father was converted into a Ch\=and\=ala by a curse but I sent him in his very body to the Heavens. And you are well acquainted with this fact. Therefore, O King! Keep my word out of your love for that. This boy is very pitifully crying; so free him. I pray this from you in this your R\=ajas\=uya sacrifice and if you do not keep my word, you will incur the

sin of not keeping my word. Do you not realise this? O King! You will have to give anything that a man wants from you in this sacrifice; but if you do otherwise, sin will attack you, no doubt.

57-59. Vy\=asa said :-- O King! Hearing these words of Kau\'sika, the King Hari\'schandra spoke thus :-- O son of G\=adhi! I am suffering very much from the dropsy; I will not be able therefore to free him. You can pray for some other thing. You ought not to throw obstacles in this my sacrifice. Vi\'sv\=amitra became very angry at this, and seeing the Br\=ahmi\d{n} boy very distressed, became sorrowful and mourned very much.

Here ends the Sixteenth Chapter in the Seventh Book on the story of \'Sunah\'sepha in the Mah\=apur\=a\d{n}am \'Sr\={\i} Mad Dev\={\i} Bh\=agavatam of 18,000 verses, by Mahar\d{s}i Veda Vy\=asa.



