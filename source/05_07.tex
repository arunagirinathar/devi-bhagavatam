\chapter{On the going of the Devas to Kail\=asa}

1-3. Vy\=asa said :-- O King! Mahi\d{s}a seeing the D\=anavas afflicted with grief, quitted his buffalo appearance, assumed a lion form and spreading this long manes began to roar aloud and fell amidst the Deva forces; then the Devas were terrified on seeing his sharp nails. That lion-form Mahi\d{s}a first attacked so severely the Garuda with his nails, that his whole body was besmeared with blood; then he attacked Vi\d{s}\d{n}u's arms with his nails.

4-11. Seeing the D\=anava, V\=asudeva Hari raised his discus in anger and attacked him with great force to kill him. Just when Hari struck the D\=anava violently with his Chakra, the powerful D\=anava quitted immediately his lion-form, assumed the buffalo form and struck Hari with his two horns. V\=asudeva, thus pierced in his breast with the horns, became confounded and fled away as best as he could till he reached his own abode, Vaikuntha. Seeing Hari thus fleeing away, \'Sankara, too, thought him invulnerable and fled to his Kail\=a\'sa mountain with fear. Brahm\=a, too, fled to his own abode with terror; but the powerful V\=asava took patience and remained steady in the battle. Varu\d{n}a taking his \'Sakti waited patiently for battle. Yama, too, with his staff remained there ready to fight. Kuvera, the Lord of the Yak\d{s}as, remained very busy in close fighting with the D\=anavas; Fire, taking \'Sakti, also waited. The Sun and Moon, the Lord of the stars, both remained in firm resolve to fight with Mahi\d{s}a, the lord of the D\=anavas.

12-22. O King! In the meanwhile, the D\=anava forces got angry and attacked them on all sides, shooting at the enemies a mass of dangerous serpent-like arrows. The Lord of the D\=anavas, Mahi\d{s}a, too, assuming the buffalo appearance, reigned supreme in the middle. At this moment fierce sounds of the warriors on both the sides were heard. During the

sharp contest of the Devas and D\=anavas, the sounds from the bowstrings and the clappings of the hands were heard like the roarings of thunder. The powerful D\=anava, then, swelled with pride, began to hurl the mountain tops with his horns, thus killing the Deva forces. Some by hoofs and some by the lashing of the tail, that angry Mahi\d{s}a, very wonderful to behold, sent to the region of Death. Then the Devas and Gandarbhas became very much frightened; so much so, that Indra fled away at once on the sight of Mahi\d{s}a. Indra thus retreating from the field, Yama, Kuvera, and Varu\d{n}a all quitted the battle-ground with fear. Indra fled away quitting his Air\=avata elephant and Uchchai\'srav\=a horse; so Mahi\d{s}a got the possession of the elephant and the horse, as well the heavenly cow of the Sun. So the D\=anavas considered themselves pre-eminently victorious and returned to their abodes. Next they wanted to go, as early as possible, to the Heavens, with all their forces. In no time Mahi\d{s}a went to the abode of Indra, deserted by all the terror-stricken Devas and got the possession thereof. Then taking his seat on the beautiful throne of Indra, he made the other D\=anavas occupy the several seats of the other Devas.

23-27. Thus fighting full one hundred years, the D\=anava Mahi\d{s}a, puffed up with pride, acquired the seat of Indra, his desired object. He banished the Devas from the Heavens; the Devas, thus tormented began to wander in the caves of hills and dales for a period of good many years. O King! The Devas, at last, were quite tired and took the four-faced Brahm\=a, the Creator's refuge. At that instant, the Lord of the world, the Rajas incarnate, the Originator of the Vedas, was seated on His lotus seat; surrounding Him were standing his mortal sons Mar\={\i}chi, etc., with their passions subdued, mind calm and beyond the sphere of the Vedas and Ved\=angas; there were there also Siddhas, Gandarbhas, Kinnaras, Ch\=aranas, Uragas, and Pannagas. The terrified Devas then began to praise and chant hymns to Brahm\=a, the Lord of the world.

28-33. The Devas said :-- ``O Creator! O Lotus-born! O Thou, the Remover of the pains and afflictions of all this world! How is it that you are not moved with pity towards the Devas, seeing that we are defeated by the lord of the D\=anavas and have been banished from our abode; what more shall we say, our troubles are now indescribable, as we are living in the caves of hills and dales. O Creator! A son may be a hundred times guilty of offence; is it, then, that the father, devoid of any feeling of covetousness, deserts his sons and gives them trouble! We are oppressed by the D\=anavas, we who are wholly devoted to your lotus-feet, why are you today showing signs of indifference towards us! That wicked D\=anava is thoroughly enjoying to-day the Heavens of the Devas, is forcibly taking their share of the oblations of clarified butter in the Yaj\~nas (sacrifices)

from the Br\=ahma\d{n}as; is enjoying the P\=arij\=ata tree and also the heavenly milching cow, the jewel of the ocean. What more shall we describe to you the strange doings of the Asuras; O Lord of the Devas! You are perfectly aware of all that they strive and execute; for, by your knowledge, you know everything of this world; therefore, O Lord! We lie prostrate at your feet. That vicious D\=anava, of wicked character and full of mischievous actions, gives us troubles in various ways wherever we go; O Lord of the Devas! Thou art our only Protector; therefore, O Lord! Do what is good to us. Thou art the Awarder of the desires of the Devas. Thou art the First Creator of the world, and Preserver; therefore if Thou dost not do us our good, to whom else shall we take refuge, when we are so severely oppressed as if we are burnt in a forest conflagration! Who else is more lustrous, more beneficent and more peace-giving Governor?

34-35. Vy\=asa said :-- O king! All the Devas, praising Him thus, bowed down to the Lord of creation with folded hands and saluted him, with their faces very heavy, overladen with deep sorrow. The Grand Sire of all the Lokas, seeing the plight of the Devas, consoled them with sweet words and made them happy.

36-43. O Suras! What shall I do? The D\=anava has become exceedingly haughty on account of his getting boons; he can be killed by females only; He is invulnerable by any male. What remedy is there now? Therefore, O Suras! Let us all go to Kail\=asa, the best of all the mountains; thence we will take \'Sankara, the expert in doing the works of Gods, and go to Vaikuntha, where Vi\d{s}\d{n}u, the Deva of the Devas resides. There we all will unite and hold a counsel and decide what is best to do, to serve the purpose of the gods. Thus making out the programme, Brahm\=a riding on his Hamsa went to Kail\=asa, accompanied by all the Devas. At the same time \'Siva came to know out of his introspection about the coming of Brahm\=a and the other Devas and soon came out of his dwelling abode. When they met each other, they saluted each other and felt very glad. The Devas then bowed down to them. Seats were given to the Devas; and when they sat respectively on their \=Asanas, the Lord of P\=arvat\={\i} also took his own seat. \'Siva asked the welfare of Brahm\=a and the Devas and asked the reasons of their coming to Kail\=asa.

44. O Brahm\=a! What has caused you to come here along with Indra and the other Devas? O highly fortunate one! Please mention it.

45-47. Brahm\=a said :-- O Deva of the Devas! The D\=anava Mahi\d{s}a is oppressing all the Devas in the Heavens; they therefore terrified are wandering hither and thither in the caves and hills with Indra. Mahi\d{s}a

and the other D\=anavas are now accepting their share of Yaj\~nas; the Lokop\=alas, being oppressed, have come to-day and are now taking shelter of Thee. O \'Sambhu! Considering the situation serious, I have taken them with me here; therefore, O Deva, do that which is reasonable and by which the purpose of the Devas can be carried out. O Bh\=uta Bh\=avana! (The creator of the world) The whole charge and responsibility of all the Devas devolves on Thee.

48. Vy\=asa said :-- O King! Hearing thus, \'Sankara smiled a little and spoke charming words to the Lotus-born in the following manner :--

49-55. O Bibhu! It is You that gave before this boon to Mahi\d{s}a; and therefore it is you that have wrought this mischief. The D\=anava has become so strong a hero that he has caused terror to all the Devas even. Now where can we get such a noble woman who becomes able to kill that D\=anava, elated with pride. My wife nor your wife ought to go to battle; even if they, the good ladies go, how will they be able to fight? The fortunate wife of Indra, too, is not expert in the art of warfare; where else there is another lady who can kill this demon, blinded with pride. I, therefore, propose this; let us all go today to Vi\d{s}\d{n}u and, praising him with hymns, engage him quickly to this cause of the gods. Vi\d{s}\d{n}u is foremost amongst the intelligent; therefore it is highly advisable to execute all actions after duly consulting with him. He, by dint of his high intelligence, will find out means and effect our purpose.

Vy\=asa said :-- O King! Brahm\=a and the other Devas heard Rudra and approved heartily and saying, ``Be it so'' instantly rose up. At the time, seeing all the auspicious signs concerning the success of the gods, they all became glad; and, riding on their respective vehicles, drove towards the abode of Vi\d{s}\d{n}u. Favourable fragrant winds, pleasant to touch, began to blow gently, birds began to chant hymns of praise and signs of success were seen all along their way. The sky was clear and the quarters became free; in short, everything showed favourable all along their way.

Here ends the Seventh Chapter on the going of the Devas to Kail\=asa in the Fifth Skandha of \'Sr\={\i} Mad Dev\={\i} Bh\=agavatam, the Mah\=a Pur\=a\d{n}am of 18,000 verses by Mahar\d{s}i Veda Vy\=asa.



