\chapter{On the going of Hari\'schandra to the Heavens}

1-7. S\=uta said :-- The King Hari\'schandra then prepared the funeral pile and placed his son on it. Next he and his wife with folded palms merged themselves in the meditation of the Parame\'svar\={\i}, the Lady of of the Universe. That Hundred-eyed is reigning within these five Ko\d{s}as (or sheaths) Annamaya, etc. She resides in the sacral plexus of the nature of Br\=ahma\d{n}, of the Puru\d{s}a composed of Anna and Rasa. And She is the Ocean of Mercy. Wearing the red robe, She is ever ready with various weapons in Her hands for the preservation of the Universe. When the King was engaged thus in meditating on Her, Indra and all the Devas with Dharma in their front came to the King Hari\'schandra with no delay. They all coming up said to the King :-- ``O King! Hear. I am the Grand Sire and here are present Dharma Himself, the Bhagav\=an Vi\d{s}\d{n}u, the S\=adhyas, Vi\'svadev\=as, Maruts, the Lokap\=alas, the Ch\=ara\d{n}as, the N\=agas, the Gandharbas, Siddhas, Rudras,

the twin A\'svins, and all the other Devas and Vi\'sv\=amitra himself. Vi\'sv\=amitra, who going ever the three worlds wishes to make friendship according to the law ordained by Dharma, is now himself desirous to grant you your desired objects.''

8. Dharma said :-- ``O King! Do not risk such an hazardous undertaking. I am Dharma; I am satisfied with your patience and forbearance, control of your senses, and the other S\=attvic qualities and have therefore come to you.''

9-10. Indra said :-- ``O Hari\'schandra! I have also come to you. So your good fortune knows no bounds today. You with your wife and son have conquered the Eternal World. O King! What is hardly attainable by any human being, you have conquered that, by dint of your own merits. So get up to the Heavens (vibrations of the Fourth-dimensional Space) with your wife and son.''

11-16. S\=uta said :-- Indra then sprinkled over the dead son on the funeral piles, the nectar, destroying the fatal effect produced by unnatural death. At that time big showers of flowers were thrown on him and Dundubhis were sounded. In the meanwhile, the prince got up from the funeral pile. He got back his former beautiful body and he looked peaceful, healthy, and greatly satisfied. Hari\'schandra embraced his son instantly in his bosom; the King and Queen also both regained their former beautiful appearance at that time and were decked with clothes and garlands. Their hearts were then deeply filled with great joy at their getting back their desired object and their health. Indra then said to the King :-- ``O Highly Fortunate One! Now ascend to the Heavens with your son and wife, by dint of your meritorious deeds and get the holy happy ends of your endeavours.''

17. Hari\'schandra said :-- ``O King of the Devas! The Ch\=and\=ala is my master; so until I get freedom from his bondage, I cannot go to the Heavens without his permission.''

18. Dharma said :-- I am myself that Ch\=and\=ala and had assumed that form and shewed you the city of the Ch\=and\=alas. Knowing that you will suffer.

19. What more than this, that I myself am that very Ch\=and\=ala, I am that very Br\=ahmi\d{n} and I am that very poisonous serpent who had smitten your boy. [Note: This is all the one and the same the Fourth Dimensional Space.] Indra said :-- Hari\'schandra! Now get up, by virtue of your own meritorious deeds to that place which is highly covetted by all the human beings that exist on earth.

20-24. Hari\'schandra said :-- ``O King of the Devas! I bow down to you. Kindly consider what I say now. All the inhabitants of the city Ko\'sala are in mourning, due to their being separated from me. How then, can I go to the Heavens leaving my sorrow-stricken subjects here. To abandon the Bhaktas, the devotees, is to incur the great sin due to the murder of a Br\=ahmi\d{n}, the killing of a woman, the drinking of liquors and the killing of a cow. O Indra! It is highly inadvisable to abandon a Bhakta who is always in service. How can one be happy when one abandons such devotees. So I will not go to the Heavens without them. You better go back to the Heavens. O Lord of the Devas! If my subjects can go with me, I am ready to go with them to the Heavens or to the Hell.''

25. Indra said :-- ``O King! Some of them are more sinful, some are more meritorious; different grades of people exist there. So, O King! How can you desire all to go simultaneously to the Heavens.''

26-29. Hari\'schandra said :-- ``O Indra! It is through the power of the citizens that the Kings enjoy their kingdoms, perform great many sacrifices, and do many engineering works (in excavating tanks, etc.) There is no doubt in this. So I, too, have done religious acts and sacrifices through my citizen\'s help. They gave me all the articles necessary for kings. So how can I now quit them so that I may get the Heavens. O Lord of the Devas! If my subjects have no such Pu\d{n}yams as to enable them to go up to the Heavens, then let the Pu\d{n}yams done by me in giving away charities, in the performance of sacrifices, and other meritorious works be divided amongst them equally. If I myself enjoy \'Svarga for a very long time; but, if by your favour, I can enjoy with them even one day's residence in \'Svarga for my merits, that is also superior to me.''

30-33. S\=uta said :-- ``Let that be;'' saying thus Indra, the Lord of the three worlds, Vi\'sv\=amitra, and Dharma who were very pleased went immediately to Ayodhy\=a from K\=a\d{s}'\={\i} by their yogic power. In an instant they reached Ayodhy\=a, filled with the Br\=ahma\d{n}as, K\d{s}attriyas, Vai\'syas, and \'S\=udras; and Indra exclaimed to them all :-- ``Let all the citizens come before Hari\'schandra, without any delay. Today they all will go to the Heavens by virtue of the Pu\d{n}yams of Hari\'schandra.'' Thus saying, they took all the men to Hari\'schandra. Then that religious King told his subjects, ``Let you all now ascend with me to the Heavens.''

34-40. S\=uta said :-- Hearing these words of Indra and their King, they all became very glad. Then those who were engaged in their worldly desires, they handed over the charge of their worldly concerns to their own

sons, gladly became ready to go up to the Heavens. The high-minded King Hari\'schandra then installed his son Rohit\=a\'sva on the royal throne and permitted him to go to the beautiful city Ayodhy\=a, filled with jolly and healthy inhabitants. Next addressing his son and friends, he took leave of them. Thus, by virtue of his own good deeds, the King Hari\'schandra attained great celebrity. He then got up and took his seat in the aerial car that has no equal and that goes at will. It was beautifully adorned, very rare even to the Devas and decked with bells emitting jingling Kinkini sounds. The high-souled \'Sukr\=ach\=arya, versed in the \'S\=astras and the Guru of the Daityas, seeing, Hari\'schandra in the Vim\=ana, spoke thus :--

41. Oh! What is the glorious result of forbearance (Titik\d{s}\=a)! What is the great fruit of charity! Oh! Due to whose influence, the King Hari\'schandra today has attained the same region with Mahendra!

42-43. S\=uta said :-- Thus I have described to you all the doings of Hari\'schandra. Any man, oppressed with sorrows and troubles, no doubt, attains constant happiness, if he hears it. What more than this, those who want \'Svarga get \'Svarga, those who want son get sons, those who want wife get wife, and those who want kingdoms get their kingdoms by hearing this incident.

Here ends the twenty-seventh Chapter of the Seventh Book on the going of Hari\'schandra to the Heavens, in the Mah\=apur\=a\d{n}am \'Sr\={\i} Mad Dev\={\i} Bh\=agavatam, of 18,000 verses, by Mahar\d{s}i Veda Vy\=asa.



