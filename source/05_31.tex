\chapter{On the death of \'Sumbha}

1. Vy\=asa said :-- O King! \'Sumbha, the Lord of the Daityas, hearing the words of the soldiers, began to say, then, with eyes rolling with anger.

2-15. \'Sumbha said :-- ``O Fools! What are you saying all this? How can I do this unspeakably mean act and then hope to live? How shall I be able to roam in this world when I have become the cause the slaying of my brothers and ministers? Time is the more powerful cause of all that takes place, good or bad; so when this formless Time is the Supreme Ruler, what use is there in my brooding over the result? Let whatever come that is inevitable, let whatever be done that is destined to take place; death or life, I do not think of either. The more so when Time is never able, even when worshipped, to thwart off death or life when their proper moment arrives. See! The God of rain gives us rain in the rainy season; but, it is seen that sometimes it does not rain in the month of \'Sr\=avan (the rainy season); whereas it rains sometimes in the month Agrah\=ayana, Pau\d{s}a, M\=agha, or Ph\=alguna (not the rainy season). Therefore it is evident that Time is not the chief factor. Fate is stronger than Time; Time is merely the instrumental cause. It is this Fate that has created all this universe; it cannot be rendered otherwise. I consider Fate Supreme; Fie on this one's own exertion! For, Lo! Ni\'sumbha, who had before conquered all the Devas, is slain today by an ordinary woman! Alas!

When Raktab\={\i}ja, too, had been slain, how can I desire to hold on to my life, foregoing all my name and fame! Even Brahm\=a, who has created all this universe, will not sooner come to an end than his longevity expires. Four thousand Yugas constitute one day of Brahm\=a; and in that one day fourteen Indras perished; so twice the life of Brahm\=a constitute the life of Vi\d{s}\d{n}u; similarly twice the life period of Vi\d{s}\d{n}u constitute the life period of Mahe\'sa; and when their longevities expire, they come to an end. This visible earth, mountains, sun and moon all will perish; so it has been specially ordained by the Destiny; therefore, O Fools! I do not care a bit for the death. When a being is born, he must die; and when anyone dies, he will be born again, there is no doubt in this. So one ought to preserve one's name and fame which is more permanent in this transitory body. Prepare my chariot; I will go today to the battlefield; let victory or defeat come what it may, as Fate has ordained. I will soon go to fight.''

16-33. Thus saying, \'Sumbha mounted on the chariot quickly and went where the Dev\={\i} Ambik\=a was staying. Then the four-fold army, cavalry, infantry, chariots, horses and elephants and innumerable soldiers, followed him with weapons in their hands. Going there to the Him\=alay\=a mountain, he saw the Divine Mother sitting on Her Lion. She appeared so very lovely as to enchant the three worlds. Her body was decorated with various ornaments, all the auspicious gems were manifest; the Devas, Gandarbhas, Yak\d{s}as and Kinnaras in the heavens were all worshipping Her with hymns and P\=arij\=ata flowers; and the Dev\={\i} was making beautiful sounds with bells and conches, indicative of Her victory. Seeing Her \'Sumbha was very much enchanted with passionate love and struck with, the five arrows of cupid, thought thus :-- How wonderful is Her lovely countenance! See! How wonderful and amazing is Her skilfulness! Delicacy and capability to endure the hardships of war, though quite contrary to each other, are both in Her. What a wonder is this! Her bodies are extremely delicate and limbs are lean and thin; besides She is lately blooming into womanhood; still She does not feel any passion; this is undoubtedly very wonderful! She is exquisitely beautiful that can be desired of in one's mind; and though She is endowed with all the auspicious signs, yet She has no inclinations for all the pleasures and allurements of the world and is now slaying the powerful Asuras; this is wonderful indeed! Now what steps are to be taken so that this Lady comes under my control? All the Mantrams also are not with me now to bring over this Swan-eyed Lady unto me. This proud lovely Lady is the incarnate of all Mantrams; how will She come under my control? This heroic Lady cannot be controlled by conciliatory words, allurements, dissensions; it is not advisable, too, to fly away from the battlefield and to

go to P\=at\=ala. What am I to do? Where shall I go in this critical moment? And if I die at the hands of this Lady, that death is not a glorious one; it will take away my fame. The death in a battlefield is conducive to one's well being, so the sages say, when both the parties are equally strong. The Devas have created this Lady stronger than even hundred strong men; She is a woman merely in name. This Lady is very powerful and has come here to destroy the D\=anavas; there is no doubt in this. What effect will conciliatory words now produce on Her; She has come to slay us; Will She be appeased with good words? Neither will allurements of precious things be of any avail, for She is decked with various arms and weapons; nor will it be of any use to sow dissensions between the Devas and Her. Further all the Devas are under Her control. Therefore it is far better to die than to fly; victory or death would come unto me today as Fate has ordained.

34-46. Vy\=asa said :-- O King! Thus thinking in his mind, \'Sumbha became ready to shew his strength; and firmly resolved to fight, he said to the Dev\={\i} before him :-- Dev\={\i}! Fight. But, O One of delicate limbs! Thy so much toil is in vain. Thou hast no sense at all; for Thou art doing contrary to the doings of woman-kind. The pair of eyes of women are their arrows; the eyebrows are their bows; their gestures and postures are their weapons and their hits are those persons who are skilled in amorous love sentiments. The dyes used in painting the bodies are their armours, their mental desires are their chariots, so sweet soft words and conversations are their trumpet sounds; women have no other things for their war preparations. Therefore, O Beloved! Any other weapons are mere mockeries and ridiculous; their modesty is their ornament; impudence can never grace them. An exquisitely beautiful woman, if engaged in a fight will look harsh; especially when Thou wilt draw Thy bow, how wilt Thou be able to hide Thy breasts? When Thou wilt run with Thy club, where will Thy gentle treadings go? O Beautiful! Thy councillors are this K\=alik\=a and the stupid Ch\=amund\=a. Chandik\=a is Thy adviser; her voice is very hoarse; how can then she be able to nurse Thee? Again this Lion, the terror of all the beings, is Thy carrier. Therefore, O Dear! Leave aside all these and come over unto me. O Beautiful One! That Thou art ringing Thy bells and dost not sound. Thy lute goes quite against Thy beauty and youth. O Sensitive One! If Thou likest to fight, better assume an ugly appearance, let Thy nature be ferocious and cruel; let Thy colour be black like a crow; lips elongated, legs long, nails ugly, teeth horrible, and let Thy eyes be ugly or yellow like those of a cat. O Dev\={\i}! Assume such an ugly appearance and stand firmly for the fight. O Deer-eyed One! Speak first harsh words

unto me; then I will fight with Thee; my hand does not get up to strike Thee with handsome teeth, in the battlefield, Who art like a second Rati.

47. Vy\=asa said :-- O Best of the descendants of Bh\=arata! When \'Sumbha said thus, the Divine Mother, seeing him passionate, smiled and said :--

48-50. O Stupid One! Why are you so much distressed with passion? O Fool! If your hand does not come forward to strike weapons at Me, then fight with this ugly K\=alik\=a or Ch\=amund\=a; they are your best compeers in the battlefield; they will fight with you; I will stand as a mere Witness. Thus saying, the Dev\={\i} Bhagavat\={\i} said to K\=alik\=a, in sweet words :-- ``O K\=alik\=a! Your nature is fierce; this \'Sumbha likes also the fierce; so kill him.''

51-69. Vy\=asa said :-- O King! That K\=alik\=a, the incarnate of Death, thus ordered, took up Her club immediately and became ready to fight, as if sent there direct by the God of Death. A dreadful fight then ensued between the two; and the highsouled Munis and the Devas were present there and witnessed the great event. \'Sumbha first struck at K\=alik\=a, raising his club. K\=alik\=a, then, struck \'Sumbha in return with her club violently. Instantly she made a dreadful sound, broke down his chariot, glittering like gold, into pieces, killed the horses of the chariot and slew the charioteer. Walking, then, on foot with a very heavy club in his hand, \'Sumbha struck with great anger on the breast of K\=alik\=a and began to laugh. K\=alik\=a, in the meanwhile, rendering his stroke useless, soon took up Her axe and cut off his left hand, pasted with sandal and decked with arms and weapons. His left hand thus out off, his whole body was drenched with torrents of blood; yet he came up with club in his hand and struck K\=alik\=a with it. K\=alik\=a, too, laughed and with Her scimitar cut off his right arm holding the club and ornamented with armlet. \'Sumbha became angry and came up violently to kick Her when K\=alik\=a quickly cut off his two legs. His arms and legs thus severed from his body, the Demon frightened K\=alik\=a and told Her, ``Wait, wait.'' And soon he came up before Her. Seeing the Demon coming, K\=alik\=a severed his neck from his body like a lotus; blood began to gush out in continuous streams. O King! The head of \'Sumbha, thus severed from his body, fell on the ground like a mountain. Immediately the life left the body. Seeing the D\=anava fall down lifeless, Indra and the other hosts of Devas began to worship the Dev\={\i} Bhagavat\={\i}, Ch\=amund\=a, and K\=alik\=a and chanted lovely hymns to them. The winds then began to blow pleasantly; all the quarters looked very clear and Fire in sacrificial altars, being circumambulated, became very propitious. On the other hand, those Daityas that remained alive quitted

their arms and weapons, bowed down to the Divine Mother, and fled away one and all to the P\=at\=ala. O King! I have now described in regular order to you how the Dev\={\i} protected the Devas and destroyed \'Sumbha and other Asuras. Those human beings on the surface of the earth that read this anecdote from the beginning to the very end or hear it constantly, get all their desires fulfilled; there is no doubt in this. O King! Verily he gets a son who has not got any son; he gets abundance of wealth who is without any wealth; the diseased become cured of their diseases; what more can be said than the fact that he who hears this glorious deed of the Dev\={\i} in its entirety, gets all that he desires. O King! That man who reads daily this holy anecdote or hears it, has never to fear from his enemies; in addition be gets liberation after leaving his this body.

Here ends the Thirty-first Chapter of the Fifth Book on the death of \'Sumbha in the Mah\=a Pur\=a\d{n}am \'Sr\={\i} Mad Dev\={\i} Bh\=agavatam of 18,000 verses by Mahar\d{s}i Veda Vy\=asa.



