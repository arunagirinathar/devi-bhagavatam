\chapter{On the conversation between the Dev\={\i} and Mahi\d{s}\=asura}

1-7. Vy\=asa said :-- O King! Hearing those words, the King Mahi\d{s}a in anger addressed the charioteer Dar\=uka :-- ``Bring over my chariot quickly. That chariot is drawn by one thousand excellent horses, is bedecked with banners, flags, and ensigns, is furnished with various arms and weapons, and is endowed with good wheels of a white colour, and beautiful poles in which the yoke is fixed.'' The charioteer brought the chariot instantly and duly informed the king, ``O King! I have got the chariot ready at your door, your beautiful chariot, bedecked with beautiful carpets and various arms and weapons.'' Hearing that the chariot had been brought, Mahi\d{s}a thought, the Dev\={\i} might not care him, seeing him ugly faced with a pair of horns and therefore decided to assume a human shape and then go to the battle. The beauty and cleverness are the delights

of women; therefore I will go before Her, with a beautiful body and with all the cleverness and dexterities. For I will never be delighted with anything but that woman looking at me with fondness and becoming passionately attached to me.

8-33. Thus thinking, the powerful King of the Demons quitted the buffalo appearance and assumed a beautiful human shape. He put on beautiful ornaments, armplates, etc., and wore divine cloths and had garlands on his neck and thus shone like a second Kandarpa, the god of Love. Taking, then, all the arrows and weapons, he mounted on the chariot, and attended by his army, went to the Dev\={\i}, elated with power and vanity. The Dev\={\i} blew Her conchshell when She saw Mahi\d{s}\=asura, the lord of the D\=anavas, come before Her with a handsome appearance, tending to captivate the minds of mistresses and surrounded by many powerful and valiant warriors. The King of the Demons heard the blow of the conchshell, wondrous to all, came up before the Dev\={\i} and smilingly spoke to Her thus :-- O Dev\={\i}! Whatever person there exists in this world, this wheel of Sams\=ara (the eternal round of births and deaths), be he or she a man or a woman, everyone always hankers after pleasure or happiness. And that pleasure is derived in this world by the combination of persons with each other; never is it seen where this combination is absent. Again this combination is of various kinds; I will mention them; Hear. Union is of various kinds according as it arises out of affection or out of natural consequences. Of these, I will now speak of unions coming out of affection, as far as my understanding goes. The union that comes between father, mother and their sons arises out of affection; it is therefore good. The union between brother and brother is middling, for mutual interests of give and take are there between the two. In fact, that union is considered as excellent which leads to happiness of the best sort and that union which leads to lesser happiness is known as mediocre. The union amongst the sailors, coming from distant lands, is known as natural. They come on various errands concerning their varied interests. This combination, because it offers the least amount of happiness, is considered as worst. The best union leads in this world to best happiness. O Beloved! The constant union of men and women of the same age is considered as par excellence; for it gives happiness of the very best sort. Both the parties, men and women, are elevated when they want to excel each other in their family connections, qualities, beauty; cleverness, dress, humility and propriety of conduct. Therefore, O Dear! If you establish with me that conjugal relation, you will get, no doubt, all the excellent happiness. Specially I will assume different forms at my mere will. All the Divine jewels and precious things that I have

acquired after defeating Indra and the other Devas in battle, and others are lying in my palace; you can enjoy all of them as my queen consort or you can make a charity of them as you like. O Beautiful One! I am your servant; consequently, at your word, I will no doubt quit my enmity with the Devas. In short, I will do anything that leads to your pleasure and happiness. O Sweet speaking One! O Large-eyed One! My heart is enchanted very much with your beauty; I will do, therefore, as you order me. O One having a broad hip! I am very much distressed; I now take refuge unto You. O One having beautiful thighs! I am very much struck with the arrows of Cupid, and I am very much discomforted; therefore, save me. To protect one who has come under one's refuge is the best of all virtues. O One of a somewhat whitish body! O One having a slender waist! I will spend the remaining portion of my life in serving you as your obedient servant. Never will I act contrary to your orders to the risk even of my life. Take this as literally true and do accordingly. I now throw aside all my weapons before Your feet; O Large eyed! I am very much distressed by the arrows of Cupid; dost Thou therefore show Thy mercy on me. O Beautiful One! Never I showed my weakness to Brahm\=a and the other Devas; but today I acknowledge that before You. I have defeated Brahm\=a and others; they are fully acquainted with my prowess in the battlefield. But, O Honoured Woman! Though I am so powerful, I now acknowledge myself as your servant. Better look at me and grant your mercy.

34. Vy\=asa said :-- O King! Mahi\d{s}a, the lord of the Daityas, having said so, that beautiful Bhagavat\={\i} laughed loudly and spoke smiling :--

35-45. The Dev\={\i} said :-- I do not desire any other body than the Supreme One! O Demon! I am His Will-power; I therefore create all these worlds. I am His \'Siv\=a (auspicious) Prakriti (Nature); That Universal Soul is seeing Me. It is owing to His proximity that I am appearing as the Eternal Consciousness, manifesting Itself as this Cosmos. As irons move owing to the proximity of magnets, I, too, though inert, owing to His proximity, work consciously. I do not desire to enjoy the ordinary pleasures; you are very dull and stupid; there is no doubt in this, when you desire sexual union. For women are considered as chains to hold men in bondage. Men bound up by iron chains can obtain freedom at any time, but when they are fastened by women, they can never obtain freedom. O Stupid! You now want to serve the source of urine, etc. Take refuge under Peace; peace will lead you to happiness. Great pain arises from connection with women; you know this; then why are you deluded? Better avoid your enmity with the Devas and

roam over the world anywhere you like. Or, if you desire to live, go to P\=at\=ala; or fight with Me. Know this for certain that I am stronger than you. O D\=anava! The Devas collected have sent Me here; I tell you this very truly; I am satisfied with you by your words of friendship; therefore dost thou fly away while you are living. See! When words are uttered seven times amongst each other, friendship is established between saints. That has been done so amongst us; so there is friendship now between you and me; I won't take away your life. O hero! If you desire to die, fight gladly; O powerful one! I will, no doubt, kill you.

46-65. Vy\=asa said :-- O King! Hearing the Bhagavat\={\i}'s words, the D\=anava, deluded by passion, began to speak in beautiful sweet words :-- O Beautiful One! Your body and the several parts thereof are very delicate and beautiful. A mere sight of such a lady makes one enchanted. Therefore, O Beautiful faced one! I fear very much to strike against your body. O Lotus-eyed One! I have subjugated Hari, Hara, the Lokap\=alas and the several other Devat\=as; I therefore ask whether it is proper for me to fight with you! O Fair one! If you like, you marry and worship me, or you can return to your desired place whence you have come. You have declared friendship with me; I therefore do not like to strike any weapons on you. I have now spoken for your good and welfare. You can gladly go away. O beautiful one! You are a fair woman with beautiful eyes; what fame shall I earn by killing you! O One of slender waist! Murdering a woman, a child, and a Br\=ahmi\d{n} certainly makes the murderer liable to suffer the consequences thereof. I will certainly carry you today to my place without killing you. If I use force to you, I will not get happiness; for, in such cases, the application of force leads to no happiness. O One having good hairs! I salute before you and speak that a man cannot be happy without the lotus face of a woman; similarly a woman cannot be happy without a man's lotus face. Where comes off the good combination between these two, then the highest pitch of happiness is conceived and pain arises on the disjunction thereof. True that you are well decked with ornaments all over your body but you seem wanting in cleverness; for you are not worshipping me. Who has advised you to renounce enjoyments? O Sweet speaking One! If this be true; then surely he is your enemy; he has deceived you. O Dear! Leave your this stubbornness and marry me; both of us shall then be happy. Vi\d{s}\d{n}u shines well with Kamal\=a, Brahm\=a looks splendid with Savitr\={\i}, Rudra is well associated with Parvat\={\i} and Indra with \'Sach\={\i}, so I will shine well with you; there is no doubt in this. No woman can ever be happy without any good husband. And why are you not then, ack-

nowledging me your husband even when you have got him. O Beloved! Where is now that Cupid of dull intellect? Why is he not troubling you with his maddening delicate five arrows? O Fair one! I think that Madana (the god of Love) out of his pity to you, seeing that you are very weak is not striking his arrows on you as he has done to me. O One looking askance! Or it may be that I have got some enmity with that Cupid; else why is he not shooting arrows at you? Or my enemies the Devas have advised the God of Love not to dart his arrows on you. O One of slender body! As Mandodar\={\i} had to marry afterwards, when she became passionate, a hypocrite, and so she had to repent thinking that she had not married before a beautiful auspicious king, so I think, O One, having eyes like the young of a deer! You, too, will have to repent like her if you decline to marry me now.

Here ends the Sixteenth Chapter of the Fifth Book on the conversation between the Dev\={\i} and Mahi\d{s}\=asura in \'Sr\={\i} Mad Dev\={\i} Bh\=agavatam, the Mah\=apur\=a\d{n}am, of 18,000 verses, by Mahar\d{s}i Veda Vy\=asa.



