\chapter{On the piercing of the eyes of Chyavana Muni}

1. Janamejaya said :-- ``O Highly Fortunate One! Kindly narrate in detail the spread of the families of those kings in the Solar line who were born and who were especially endowed with the knowledge of Dharma.''

2-8. Vy\=asa said :-- O Bharata! I now speak to you exactly what I heard of yore, from N\=arada, the best of the \d{R}i\d{s}is, how the Solar race spread. Once, on an occasion, the Muni \'Sr\={\i}m\=an N\=arada, on his tour, came at his will to my holy hermitage on the beautiful banks of the Sarasvat\={\i} river. On seeing him I bowed down at his feet and then remained standing before him. I then gave him a seat and worshipped him with great esteem. I then said to him :-- ``O Best of Munis! You are worshipped by the whole universe; my retreat is sanctified by your coming. O All-knowing One! Kindly narrate the histories of the Kings that were famous in the family of the seventh Manu; they were unequalled as far as their origin was concerned and their diameters as well were wonderful. Therefore I am very eager to know, in detail, the history of the Solar race. O Muni! Describe shortly or in detail as the circumstances may require.'' O King! When I made this question, N\=arada, the knower of the Highest Reality, gladly smiled, and, addressing me, began to describe the history of the Solar race.

9-26. N\=arada said :-- O son of Satyavat\={\i}! The history of the royal families is very holy and pleasant to hear; the more so when they are heard, one acquires Dharma and wisdom; therefore do you hear. In ancient times, Brahm\=a sprang from the navel-lotus of Vi\d{s}\d{n}u and created the

world. This is well known in every Pur\=a\d{n}a. That self born, all-powerful, all-knowing, the Doer of all, the Universal Soul practised Tapas in ancient times for Ajuta (ten thousand) years. By virtue of that Tapas, He got special powers to create the world. He meditated on the Auspicious Mother, and getting from Her the excellent powers, He created first the mind-born sons, all endowed with auspicious signs. Of them, Mar\={\i}chi became well known in this act of creation. His son Ka\'syapa was respected by all and he was of great celebrity. He had thirteen wives, all daughters of Dak\d{s}a Praj\=apati. The Devas, Daityas, Yak\d{s}as, Pannagas, beasts and birds all sprang from him. Therefore this creation is called the K\=a\'syap\={\i} creation. Amongst the Devas, the Sun is specially famous; his other name is Vivasv\=an. His son was named Vaivasvat Manu; he was a famous king. Besides, Manu had nine more sons. Ik\d{s}v\=aku was the eldest. Their names are: (1) N\=abh\=aga, (2) Dhrista, (3) \'Sary\=ati, (4)Nari\d{s}yanta, (5) Pr\=an\'su, (6) Nriga, (7) Dista, (8) Kar\=u\d{s}a, and (9) Ri\d{s}adhra. Ik\d{s}v\=aku, the son of Manu was born first. He had one hundred sons, and Vikuk\d{s}i was the wise and the eldest of these. I am now narrating how the nine sons, born afterwards of Manu, spread their families. Ambari\d{s}a was the son of N\=abh\=aga; he was very truthful, powerful, and religious. He always governed his subjects justly. Dh\=arstaka was the son of Dhrista; though he was a K\d{s}attriya, he attained to Br\=ahmanhood. He was naturally weak in fighting; always be was engaged in the works relating to the Br\=ahma\d{n}as. \=Anarta was the well known son and Sukany\=a was the beautiful daughter of \'Sary\=ati. The King \'Sary\=ati gave his beautiful daughter in marriage to the blind Chyavana \d{R}i\d{s}i; but the \d{R}i\d{s}i, though blind, got his beautiful eyes again by the good character of the daughter. We heard that the A\'svins, the Twins, the sons of the Sun, gave him back his eyesight.

27-29. Janamejaya said :-- ``O Brahm\=ans! How is it that the King \'Sary\=ati married his lovely-eyed daughter Sukany\=a to the blind Chyavana Muni? I have got a great doubt on this point. The King gives his daughter in marriage to a blind person, if she be deformed, ill-qualified or void of female signs. But the daughter, in this case, was beautiful. How then \'Sary\=ati, the Chief of Kings, gave over his daughter, knowing that the \d{R}i\d{s}i was blind? O Br\=ahma\d{n}a! I am always an object of favour to you; so explain to me the cause of it.''

30. S\=uta said :-- Glad to hear these words of Janamejaya, the Muni Dvaip\=ayana smilingly said :--

31-50. \'Sary\=ati, the son of Vaivasvata, had four thousand married wives. All of them were endowed with auspicious signs and beautiful

all of them were daughters of kings. They all were very obliging and dear to their husbands. But, out of all of these, the King had only one daughter exceptionally lovely and beautiful. The father and all the mothers loved exceedingly that sweet-smiling daughter. Not very far off the city, there was a beautiful lake of clear waters, like the M\=anasarovara lake. A Gh\=at way (steps) made of stones descended into the lake. Swans K\=arandavas, Chakrav\=akas, Datyu'has, S\=arasas and other birds used to play on its waters. Five varieties of lotuses were there in full bloom, bees were humming there all around. Various trees, \'S\=al, Tam\=ala, Sarala, Punn\=agas, A\'sokas, Banyans, Peepuls, Kadambas, rows of banana trees, Jamb\={\i}rs, Dates, Panasas, Betelnut trees, cocoanut trees, Ketakas, K\=anchanas, and other various beautiful trees encircled round the lake. Within these, the white Y\=uthik\=as, Mallik\=as, and other creepers and shrubs were seen beautifying the scenery. Especially there were, amongst them, Jack trees, Mango trees, tamarind trees, Karanjas, Kutakas, Pal\=a\'sas, Neem trees, Khadiras, Bel trees, and \=Amalaki trees; and peacocks were sounding their notes, cuckoos were cooing their beautiful voices. Close to that place, there was, in a sacred grove covered over by trees, staying Chyavana Muni, the Bhrigu's son, of a tranquilled mind, and the chief of the ascetics. Thinking the place lovely and free from any obstacles, the Muni took his firm seat there and, collecting all his thoughts within himself, took the vow of non-speaking and controlling his breath became engaged in practising tapasy\=a. Restraining his senses and foregoing eating and drinking, that Muni constantly meditated on Bhagavat\={\i} of the nature of Sat, Chit and \=Ananda, O King! While he was thus meditating, the anthill grew up round and covered his body and nice creepers covered that also all round. O King! Long intervals passed away and it was covered over with ants; so much so that that intelligent Muni was covered entirely and looked like a heap of earth. O King! Once the King \'Sary\=ati wanted to play in an artificial wilderness and came there to the lake with his wives. \'Sary\=ati became at once deeply engaged in playing on the clear waters of the lake, surrounded by the beautiful females. On the other hand, the quick beautiful daughter Sukany\=a, picking up flowers here and there with her companions also began to play. Dressed in ornaments, Sukany\=a, walked to and fro; her anklets making a beautiful tinkling sound, till she came to the ant-hill of Chyavana \d{R}i\d{s}i. She sportingly sat close to that anthill and instantly saw a shining substance inside through that, like fireflies. ``What is this?'' She thought and wishing to take it, took a thorn and became very eager to prick it up.

51-59. Slowly she went close to it and no sooner she got ready to prick it, than the Muni saw the beautiful, good-haired daughter as if to one's

liking. The ascetic Bh\=argava, seeing that auspicious nice lady with nice teeth, spoke out in a feeble voice :-- ``What are you doing? O thin-bellied One! I am an ascetic; better go away from here. You have got such big-eyes, yet you do not see me. I therefore forbid you in your this attempt; do not pierce the anthill with thorn.'' Though prevented, the daughter could not hear his words and asking ``What was that?'' pierced his two-eyes with thorns. Thus impelled by Fate, the princess sportingly pierced his eyes; but she suspected and thought ``What have I done?'' Thus becoming afraid she returned from that spot. His two eyes being pricked, the great Muni exceedingly pained, became very wrathful he incessantly gave vent to sorrows and remorse, being restless with pain. At that instant it happened that the king, ministers, soldiers, elephants, horses, camels, so much so that all the beings that were there, had all their evacuations (passing their urines and faeces) stopped. Seeing thus happened all on a sudden, the King \'Sary\=ati was very much pained and became very anxious. All the soldiers came to the King and informed him of the stoppage of their evacuations. The King thought over the cause why this had happened.

60-65. Cogitating thus, the King returned home. Becoming very much troubled with cares and anxieties, He asked his soldiers and kinsmen ``Who amongst you has done such an heinous act? On the west side of the lake the Mahar\d{s}i Chyavana is practising the great tapasy\=a in the midst of the forest; I think someone has done mischief to that king of ascetics, blazing like a fire; and therefore we are overcome with this disease. The highsouled aged son of Bhrigu has become specially proficient in his asceticism and has become supreme; I think someone must have injured him. Though I do not know who is that mischievous person that has shown him contempt or like that, this our state at present clearly shews that this is the fit punishment of that.'' Hearing this, the soldiers said :-- None of us has committed any mischief by word, mind or body; we know this very well.

Here ends the Second Chapter of the Seventh Book on the piercing of the eyes of Chyavana Muni in \'Sr\={\i} Mad Dev\={\i} Bh\=agavatam the Mah\=a Pur\=a\d{n}am, of 18,000 verses, Mahar\d{s}i Veda Vy\=asa.



