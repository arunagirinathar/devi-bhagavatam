\chapter{On the Dhy\=anam and Stotra of Mah\=a Lak\d{s}m\={\i}}

1-50. N\=arada said :-- O Bhagav\=an! I have heard about the glories of Hari, about the Tattvaj\~n\=anam (the True Knowledge) and the story of Lak\d{s}m\={\i}. Now tell me Her Dhy\=anam (meditation) and Stotram (recitation of hymns) of Her. N\=ar\=aya\d{n}a said :-- O N\=arada! Indra then, bathed first in the T\={\i}rath (holy place) and, wearing a cleansed cloth, installed, first of all, an earthen jar (ghata) on the beach of the K\d{s}iroda Ocean. Then he worshipped with devotion Gane\'sa, S\=urya, Fire, Vi\d{s}\d{n}u, \'Siva, and \'Siv\=a, the six deities with scents and flowers. Next Indra invoked Mah\=a Lak\d{s}m\={\i}, of the nature of the highest powers and greatest prosperity, and began to worship Her as Brahm\=a, who was acting as an officiating priest in the presence of the Munis, Br\=ahma\d{n}as, Brihaspati, Hari and the other Devas, had dictated him. He first smeared one P\=arij\=ata flower with sandal paste and reciting the meditation mantra of Mah\=a Lak\d{s}m\={\i} offered it to Her feet. The meditation mantra that was recited by Devendra, was what Bhagav\=an Hari first gave to Brahm\=a. I am now telling you that. Listen.

``O Mother! Thou residest on the thousand-petalled lotus. The beauty of Thy face excels the beauty of koti autumnal Full Moons. Thou art shining with Thy own splendour. Thou art very beautiful and lovely; Thy colour is like the burnished gold; Thou art with form, chaste, ornamented all over with jewel ornaments; Thou art wearing the yellow cloth and look! What beauty is coming out of it! Always a sweet smile reigns on Thy lips. Thy beauty is constant; Thou art the bestower of prosperity to all. O Mah\=a Lak\d{s}m\={\i}! I meditate on Thee.'' Thus meditating on Her endowed with various attributes with this mantra, Indra worshipped devotedly with sixteen upach\=aras (articles offered). Every upach\=ara (article) was offered with the repetition of mantra. All the things were very excellent, right and commendable. ``O Mah\=a Lak\d{s}m\={\i}!

Vi\'svakarm\=a has made this invaluable \=Asan (a carpet seat) wonderfully decked with jewels; I am offering this \=Asan to Thee. Accept. O Thou residing in the Lotus! This holy Ganges water is considered with great regard and desired by all. This is like the fire to burn the fuel in the shape of the sins of the sinners. O Thou! The Dweller in the Lotus! This D\=urbha grass, flowers, this Arghya (offering) of the Ganges water perfumed with sandalpaste, I am offering to Thee. Accept. O Beloved of Hari! This sweet scented flower oil and this sweet scented \=Amalaki fruit lead to the beauty of the body; therefore I present this to Thee. Accept. O Dev\={\i}! I am presenting this cloth made of silk to Thee; Accept. O Dev\={\i}! This excellent ornament made of gold and jewels, which increases the beauty, I am presenting to Thee. Accept. O Beloved of Kri\d{s}\d{n}a! I am presenting this sweet scented holy Dh\=upa prepared from various herbs and plants, exquisitely nice and the root of all beauty, to Thee. Accept. This sweet scented pleasant sandalpaste I offer to Thee, O Dev\={\i}! Accept. O Ruler of the Devas! I present this pleasing holy D\={\i}pa (lights) which is the eye of this world and by which all the darkness is vanished; accept. O Dev\={\i}! I present to Thee these very delicious offerings of fruits, etc., very juicy and of various kinds. Accept. O Deve\'s\={\i}! This Anna (food) is Brahm\=a and the chief means to preserve the life of living beings. By this the nourishment of the body and the mental satisfaction are effected. Therefore I am presenting this food to Thee. Accept. O Mah\=a Lak\d{s}m\={\i}! I am presenting this most delicious Param\=anna, which is prepared out of rice, milk and sugar, to Thee. Accept. O Dev\={\i}! I am presenting this most delicious and pleasant svastika prepared of sugar and clarified butter to Thee; accept. O Beloved of Achyuta! I am presenting to Thee various beautiful Pakk\=annas, ripe delicious fruits and clarified butter out of cow's milk; accept. O Dev\={\i}! The sugarcane juice, when heated, yields a syrup which again heated yields very delicious and nice thing called Gur. I am presenting this Gur to Thee; accept. O Dev\={\i}! I am presenting to Thee the sweetmeats prepared out of the flour of Yava and and wheat and Gur and clarified butter; accept. I am presenting with devotion the offering made of Svastika and the flour of other grains; accept. O Kamale! I am presenting to Thee this fan and white ch\=amara, which blows cool air and is very pleasant when this body gets hot; accept. O Dev\={\i}! I am presenting this betel scented with camphor by which the inertness of the tongue is removed; accept. O Dev\={\i}! I am presenting this scented cool water, which will allay the thirst and which is known as the life of this world; accept. O Dev\={\i}! I am presenting this cloth made of cotton and silk that increases the beauty and splendour of the body. Accept. O Dev\={\i}! I am presenting to Thee, the ornaments made of gold and jewels which are the source of beauty and loveliness. Ac-

cept. O Dev\={\i}! I am presenting to Thee these pure garlands of flowers which blossom in different seasons, which look very beautiful and which give satisfaction to the Devas and to the kings. Accept. O Dev\={\i}! I am presenting to Thee this nice scent, this very holy thing to Thee by which both the body and mind become pure, which is most auspicious and which is prepared of many fragrant herbs and plants; accept. O Beloved of the God Kri\d{s}\d{n}a! I am presenting this \=Achaman\={\i}ya water to Thee for rinsing the mouth, pure and holy, and brought from holy places of pilgrimages; accept. O Dev\={\i}! I am presenting to Thee, this bed made of excellent gems and jewels and flowers, sandalpaste, clothings and ornaments; accept. O Dev\={\i}! I am presenting to Thee all those things that are extraordinary, very rare in this earth and fit to be enjoyed by the Devas and worthy of their ornaments; accept.'' O Devar\d{s}i! Uttering those mantras, the Devendra offered those articles, with intense devotion according to the rules. He, then, made Japam of the M\=ula Mantra (the Radical Seed Mantra) ten lakhs of times. Thus his Mantra revealed the Deity thereof and thus came to a successful issue. The lotus born Brahm\=a gave this Mantra ``\'Sr\={\i}m Hr\={\i}m Kl\={\i}m Aim Kamal\=av\=asinyai Sv\=ah\=a'' to the Devendra. This is like a Kalpavrik\d{s}a (the tree in Indra's garden yielding whatever may be desired). This Vaidik mantra is the chief of the mantras. The word ``Sv\=ah\=a'' is at the end of the mantra. By virtue of this Mantra, Kuvera got his highest prosperity. By the power of this Mantra, the King-Emperor Dak\d{s}a S\=avar\d{n}i Manu and Mangala became the lords of the earth with seven islands. Priyavrata, Utt\=anap\=ada, and Ked\=arar\=aja all these became Siddhas (were fructified with success) and became King-Emperors. O N\=arada! When Indra attained success in this Mantra, there appeared before him Mah\=a Lak\d{s}m\={\i}, seated in the celestial car, decked with excellent gems and jewels. The Great Halo, coming out of Her body made manifest the earth with seven islands. Her colour was white like the white champaka flower and Her whole body was decked with ornaments. Her face was always gracious and cheerful with sweet smiles. She was ever ready to shew Her kindness to the Bhaktas. On Her neck there was a garland of jewels and gems, bright as ten million Moons. O Devar\d{s}i! No sooner did Indra see that World Mother Mah\=a Lak\d{s}m\={\i}, of a peaceful appearance, than his body was filled with joy and the hairs of the body stood on ends. His eyes were filled with tears; and, with folded palms, he began to recite stotras to Her, the Vaidik stotras, yielding all desires, that was communicated to him by Brahm\=a.

51-75. Indra said :-- ``O Thou, the Dweller in the lotus! O N\=ar\=aya\d{n}\={\i}! O Dear to Kri\d{s}\d{n}a! O Padm\=asane! O Mah\=a Lak\d{s}m\={\i}!

Obeisance to Thee! O Padmadalek\d{s}a\d{n}e! O Padmanibh\=anane! O Padm\=asane! O Padme! O Vai\d{s}\d{n}av\={\i}! Obeisance to Thee! Thou art the wealth of all; Thou art worshipped by all; Thou bestowest to all the bliss and devotion to \'Sr\={\i} Hari. I bow down to Thee. O Dev\={\i}! Thou always dwellest on the breast of Kri\d{s}\d{n}a and exercisest Thy powers over Him. Thou art the beauty of the Moon; Thou takest Thy seat on the beautiful Jewel Lotus. Obeisance to Thee! O Dev\={\i}! Thou art the Presiding Deity of the riches; Thou art the Great Dev\={\i}; Thou increasest always Thy gifts and Thou art the bestower of increments. So I bow down to Thee. O Dev\={\i}! Thou art the Mah\=a Lak\d{s}m\={\i} of Vaikuntha, the Lak\d{s}m\={\i} of the K\d{s}iroda Ocean; Thou art Indra's Heavenly Lak\d{s}m\={\i}; Thou art the R\=aja Lak\d{s}m\={\i} of the Kings; Thou art the Griha Lak\d{s}m\={\i} of the householders; Thou art the household Deity of them; Thou art the Surabh\={\i}, born of the Ocean; Thou art the Dak\d{s}i\d{n}\=a, the wife of the Sacrifices; Thou art Aditi, the Mother of the Devas; Thou art the Kamal\=a, always dwelling in the Lotus; Thou art the Sv\=ah\=a, in the offerings with clarified butter in the sacrificial ceremonies; Thou art the Svadh\=a Mantra in the K\=avyas (an offering of food to deceased ancestors). So obeisance to Thee! O Mother Thou art of the nature of Vi\d{s}\d{n}u; Thou art the Earth that supports all; Thou art of pure \'Suddha Sattva and Thou art devoted to N\=ar\=aya\d{n}a. Thou art void of anger, jealousy. Rather Thou grantest boons to all. Thou art the auspicious S\=arad\=a; Thou grantest the Highest Reality and the devotional service to Hari. Without Thee all the worlds are quite stale, to no purpose like ashes, always dead while existing. Thou art the Chief Mother, the Chief Friend of all; Thou art the source of Dharma, Artha, K\=ama and Mok\d{s}a! As a mother nourishes her infants with the milk of her breasts, so Thou nourishest all as their mother! A child that sucks the milk might be saved by the Daiva (Fate), when deprived of its mother; but men can never be saved, if they be bereft of Thee! O Mother! Thou art always gracious. Please be gracious unto me. O Eternal One! My possessions are now in the hands of the enemies. Be kind enough to restore my kingdoms to me from my enemie\'s hands. O Beloved of Hari! Since Thou hast forsaken me, I am wandering abroad, friendless, like a beggar, deprived of all prosperities. O Dev\={\i}! Give me J\~n\=anam, Dharma, my desired fortune, power, influence and my possessions.'' O N\=arada! Indra and all the other Devas bowed down frequently to Mah\=a Lak\d{s}m\={\i} with their eyes filled with tears. Brahm\=a, \'Sankara, Ananta Deva, Dharma and Ke\'sava all asked pardon again and again from Mah\=a Lak\d{s}m\={\i}. Lak\d{s}m\={\i} then granted boons to the Devas and before

the assembly gladly gave the garland of flowers on the neck of Ke'sava. The Devas, satisfied, went back to their own places. The Dev\={\i}, Lak\d{s}m\={\i}, too, becoming very glad went to \'Sr\={\i} Hari sleeping in the K\d{s}iroda Ocean. Brahm\=a and Mahe\'svara, both became very glad and, blessing the Devas, went respectively to their own abodes. Whoever recites this holy Stotra three times a day, becomes the King Emperor and gets prosperity and wealth like the God Kuvera. Siddhi (success) comes to him who recites this stotra five lakhs of times. If anybody reads regularly and always this Siddha Stotra for one month, he becomes very happy and he turns out a R\=ajar\=ajendra.

Here ends the Forty-second Chapter of the Ninth Book on the Dhy\=anam and Stotra of Mah\=a Lak\d{s}m\={\i} in the Mah\=a Pur\=a\d{n}am \'Sr\={\i} Mad Dev\={\i} Bh\=agavatam of 18,000 verses by Mahar\d{s}i Veda Vy\=asa.



