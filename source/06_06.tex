\chapter{On the slaying of Vritr\=asura}

1-3. Vy\=asa said :-- O King! Thus getting the boons from the Dev\={\i}, the Devas and the \d{R}i\d{s}is blazing with their asceticism, all united and consulted with each other; then they went to the excellent \=A\'srama of Vritra. There they saw Vritra in a sitting posture and with his own Tejas (fiery spirit in him) as if ready to burn the three worlds and to devour all the Devas. The \d{R}i\d{s}is, then, spoke to Vritra the sweet words full of sentiments for the serving of the Deva\'s ends, according to the principle of conciliation.

4-23. ``O highly fortunate Vritra! Terrific to all the Lokas! Yo have now established your dominion in all the places over this whole Universe; but your enmity with Indra is the only cause to interrupt you in your happiness; there is no doubt in this. This enmity has increased much the anxiety of you both and therefore has grown very painful. Neither you nor Indra can go to sleep peacefully, there is always that fear hanging on you both, on account of that enmity. And, see! A long long while has passed away since the last battle was fought between you two; yet all the Devas, Asuras, men and other subjects, are feeling a sense

of oppression and pain. In this world happiness is the only thing to be sought for and pain is to be avoided; this is the eternal state of things. Never does that man who practises enmity with another, get happiness; this has been ascertained by the wise. It is only those brave warriors, that found taste in warfare, that approve of battles; but the wise that are expert in amorous enjoyments do not like battle as destroying the sensual enjoyments; they do not like fighting with flowers even; what to say with sharpened arrows! In a battle, the victory is doubtful but the shooting of arrows is certain, This world is dependent on Fate (Daiva, i.e., dependent on the cosmic rulers or deities or Devas of the Universe), so is victory or defeat. So knowing this, one ought never to fight. Bathing in proper time, taking food and sleeping in fixed times and having a chaste serving wife, these are the means towards happiness in this world. While in warfares, shooting terrible arrows and striking with fierce axes take place; what happiness can there possibly exist? Rather the enemy finds pleasure there. There is a saying that death in battles leads one to Heaven, but this is merely an enticing statement, inciting one to war! Really it is fruitless. Supposing that happiness comes ultimately to those who pain their bodies by being shot with arrows and who allow their carcasses being devoured by the crows and jackals, then no man, even of dull understanding, will like this, what to speak of intelligent persons! Therefore, O Vritra! Let everlasting peace and friendship be established between you and Indra; both of you in that case will derive everlasting peace and happiness. Moreover if the enmity between you terminates from this instant, then we, the ascetics and Gandharbas will, no doubt, be able to remain in our own respective \=A\'srams with great comfort. O Powerful Hero! Owing to incessant wars between you and Indra, the Munis, Gandharbas, Kinnaras and beings are day and night, suffering very much. For the happiness of all peace-loving persons, we, the Munis, the residents of the forest, earnestly desire that there be formed friendship between you two. We desire that you, Indra and all the J\={\i}vas get happiness. O Vritra! We stand as mediators in this treaty between you and Indra; we will make each party swear on oath and thus make it conducive to the happiness of both. Indra will now swear on oath before you on the terms that you will dictate and thus will make your heart cheerful. Know this verily that this earth stands on Truth, the sun rises for the sake of Truth, the winds blow all along for Truth and the boundless ocean never oversteps its limit for Truth. Therefore let your friendship, be established on Truth. Thus tied together by bonds of friendship let you two sleep, play, make sports in water and sit together happily.''

24-28. Vy\=asa said :-- O King! Hearing the Mahar\d{s}i's words, the highly intelligent Vritra began to say :-- ``Risis! You are possessed with knowledge and many other qualifications and you are ascetics; you are therefore to be respected by me. You are the Munis and therefore you never speak anywhere falsehood; your conduct is good and you practise rite and ceremonies; you are calm; therefore you do not know the causes of pretexts. The intelligent should never cultivate friendship with a knave, licentious person who is void of understanding, an infamous, and a shameless person, specially if he be an enemy. This vicious Indra is shameless, deceitful, licentious, and the killer of a Br\=ahma\d{n}a; therefore no faith can ever be placed on such persons. You are saints and added with all good qualifications; therefore your minds never play in the mischievous thoughts of others; it is because your heart is calm and quiet that you cannot understand the minds of the deceitful and treacherous; therefore you ought never to stand as mediators between any two persons.''

29-32. The Munis said :-- ``O King! All the creatures certainly enjoy the fruits of their Karmas, whether good or bad; how then, can persons, of perverted intellect, obtain peace when they do mischief to others. The treacherous persons certainly go to hell and suffer miseries always. The slayers of Br\=ahma\d{n}as and the drunkards may get liberation; but never the faithless and those who go against their friends get off free; these will have to suffer undoubtedly in the hells. Therefore, O Knower of all things! Give out clearly what is going on exactly in your mind and the exact terms that you want; and the treaty will be made between you and Indra exactly according to those terms.''

33-34. Vritra said :-- ``O highly fortunate Munis! I can enter into a treaty of peace with Indra only on the condition that Indra with all the other Devas will not kill me in day or in night with any dry or liquid substance or with wood, stone, or thunderbolt and on no other terms.''

35-68. Vy\=asa said :-- O King! The \d{R}i\d{s}is then gladly accepted his word and brought Indra there and recited to him the terms of the treaty of peace. Indra, then, swore, an oath, before the Munis with Fire as the Witness that he would comply with the terms of the treaty, and was thus freed from his heavy thoughts and felt that he had been rid of a fever. Vritra, then, relied on lndra's words; became his friend, and began live, play and enjoy with him. They felt pleasure by their union and began to roam sometimes in the Nandana Garden, sometimes in the Gandha M\=adana, sometimes on the shores of oceans, Vritra was very much delighted when they were thus united in friendship; but Indra watched

him to find his faults; thus sometime passed. A few years passed away after the treaty had been concluded. And the straight-forward Vritra began to place very much confidence on Indra; but Indra meditated on the means how to kill him. One day Visvakarm\=a, knowing that his son Vritr\=asura placed implicit confidence on Indra, called his son and said :-- ``O my son Vritra! Hear my good words. See, it is never advisable to trust anybody with whom there has arisen once the enmity. Indra is your greatest enemy; he always intends evil to you; therefore do not trust him any more. Indra is never to be trusted, who is always covetous, inimical, rejoicing at others sufferings, licentious and addicted to other\'s wives; vicious, deceitful, finding faults with others, always jealous, a juggler, and puffed up with vanity. O Child! What more shall I say than this fact that that villain, without fearing sin, easily entered into the womb of his mother and cut the crying child in the womb into seven pieces and then each seventh part again into seven parts, thus altogether into forty-nine parts. Therefore O my son! He is never to be trusted on any account. He who is always addicted to vicious deeds never feels shame in perpetrating again another crime.'' Vy\=asa said :-- O King! Vritra's death time drew nigh; hence he could not take his father's words as auspicious, though he was warned by his father in words full of meaning. One day, in the evening time, at a very inauspicious dreadful moment, Indra saw Vritra on the shore of an ocean and began to think of the boon granted by Brahm\=a to the Asura thus :-- ``Now this is the terrible evening time; this cannot be called day nor can it be called night, and this demon is also here alone in this solitary place; it is advisable therefore to effect his death by force, there is no doubt in this.'' Thus arguing in his mind, Indra remembered the Undecaying Soul Hari. Bhagav\=an, the Best of Puru\d{s}as came there, unseen by anybody, and entered into the thunderbolt; Indra quickly collected himself to kill Vritr\=asura; but he thought how he could slay this Demon, unconquerable in the battle; and if he did not slay his enemy then by deceit, then his enemy would continue to live, and it would be impossible for him to get his own welfare. While he was thus thinking, he saw the foam of the waters of the ocean as big as a mountain; thinking that foam not to be dry nor wet and considering that foam not to be any weapon, he easily took that foam and instantly remembered with a heartful devotion the Highest Force Bhuvane\'svar\={\i}. On Her remembrance, the Bhagavat\={\i} infused Her part into that foam and the thunderbolt, instilled with the force of N\=ar\=ayana, was covered, too, by that foam. Indra, then, hurled the thunderbolt covered with foam on Vritra; and the Demon, thus struck, instantly fell down like a mountain. When Vritr\=asura was thus killed, Indra became very glad; the \d{R}i\d{s}is began to praise

him with various hymns. Indra, then, with all the other Devas worshipped the Dev\={\i}, through Whose Grace the enemy had been killed and they praised Her with various hymns. The image of the Bhagavat\={\i} the Supreme \'Sakti was built of ruby and installed in the Nandana Garden. O King! Since then all the Devas used to worship the Dev\={\i} thrice a day, morning, midday and evening and since then the \'Sr\={\i} Dev\={\i} became the tutelary deity of the Gods. Indra worshipped then Vi\d{s}\d{n}u also, the Highest of the Gods. When the terrible powerful Vritr\=asura was killed, the auspicious wind began to blow gently; the Devas, Gandharbas, R\=akhsasas, and Kinnaras began to roam about with great joy. Vritr\=asura was deluded by the M\=ay\=a of Bhagavat\={\i}, and Her force entered into the foam; hence Indra was capable to kill him suddenly and it is, for this reason, that the Dev\={\i}, the Goddess of the world, is known in the three worlds as ``Vritranihantr\={\i},'' the slayer of Vritra. But at the first sight Indra killed him by means of the foam; hence the people say that Vritra was killed by Indra.

Here ends the Sixth Chapter of the Sixth Book on the slaying of Vritr\=asura in the Mahapur\=a\d{n}am \'Sr\={\i} Mad Dev\={\i} Bh\=agavatam of 18,000 verses by Mahar\d{s}i Veda Vy\=asa.



