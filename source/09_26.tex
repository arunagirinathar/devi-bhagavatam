\chapter{On the narration of S\=avitr\={\i}}

1-2. N\=arada said :-- I have heard the anecdote of Tulas\={\i}. Now describe in detail the history of S\=avitr\={\i}. S\=avitr\={\i} is considered as the Mother of the Vedas. Why was She born, in days gone by? By whom was She first worshipped and subsequently also?

3-4. N\=ar\=aya\d{n}a said :-- O Muni! She was first worshipped by Brahm\=a. Next the Vedas worshipped her. Subsequently the learned men worshipped her. Next the King A\'svapati worshipped Her in India. Next the four Var\d{n}as (castes) worshipped Her.

5. N\=arada said :-- O Br\=ahma\d{n}! Who is that A\'svapati? What for did he worship? When the Dev\={\i} S\=avitr\={\i} became adorable by all, by which persons was She first worshipped and by which persons subsequently?

6-14. N\=ar\=aya\d{n}a said :-- O Muni! The King A\'svapati reigned in Bhadrade\'sa, rendering his enemies powerless and making his friends painless. He had a queen very religious; her name was M\=alat\={\i}; She was like a second Lak\d{s}m\={\i}. She was barren; and desirous of an issue, She under the instruction of Va\'sistha, duly worshipped S\=avitr\={\i} with devotion. But She did not receive any vision nor any command; therefore She returned home with a grievous heart. Seeing her sorry, the king consoled her with good words and himself accompanied her to Pu\d{s}kara with a view to perform Tapas to S\=avitr\={\i} with devotion and, being self-controlled, practised tapasy\=a for one hundred years. Still he could not see S\=avitr\={\i}, but voice came to him. An incorporeal, celestial voice reached his ears :-- ``Perform Japam (repeat) ten lakhs of G\=ayatr\={\i} Mantram.'' At this moment Par\=asara came up there. The king bowed down to him. The Muni said :-- O King! One japa of G\=ayatr\={\i}, destroys the days sins. Ten Japams of G\=ayatr\={\i} destroy day and night's sins.

15-40. One hundred G\=ayatr\={\i} Japams destroy one month's sins. One thousand Japams destroy one year's sins. One lakh G\=ayatr\={\i} Japams destroy the sins of the present birth and ten lakh G\=ayatr\={\i} Japams destroy the sins of other births. One hundred lakhs of Japams destroy the sins of all the births. If ten times that (i.e., 1,000 lakhs) be done, then liberation is obtained. (Now the method, how to make Japam). Make the palm of the (right) hand like a snake's hood; see that the fingers are all close, no holes are seen; and make the ends of the fingers bend downwards; then being calm and quiet and with one's face eastward, practise Japam. Then count from the middle of the ring (name-less) finger and go on counting right-handed (with the hands of the watch) till you come to the bottom of the index finger. This is the rule of counting by the hand. O King! The rosary is to be of the seed of white lotus or of the crystals; it should be consecrated and purified. Japam is to be done then in a sacred T\={\i}rtha or in a temple. Becoming self-controlled one should place the rosary on a banyan leaf or on a lotus leaf and smear it with cowdung; wash it, uttering G\=ayatr\={\i} Mantra and over it perform one hundred times G\=ayatr\={\i} Japam intently in accordance with

the rules. Or wash it with Pa\~nchagavya, milk, curds, clarified butter, cow urine and cowdung, and then consecrate it well. Then wash it with the Ganges water and perform best the consecrations. O R\=ajar\d{s}i! Then perform ten lakhs of Japam in due order. Thus the sins of your three births will be destroyed and then you will see the Dev\={\i} S\=avitr\={\i}. O King! Do this Japam, being pure, everyday in the morning, mid-day, and in the evening. If one be impure and devoid of Sandhy\=a, one has no right to do any action; and even if one performs an action, one does not get any fruit thereby. He who does not do the morning Sandhy\=a and the evening Sandhy\=a, is driven away from all the Br\=ahmi\d{n}ic Karmas and he becomes like \'S\=udras. He who does Sandhy\=a three times throughout his life, becomes like the Sun by his lustre and brilliance of tapas. What more than this, the earth is always purified by the dust of his feet. The Dv\={\i}ja who does his Sandhy\=a Bandanam and remains pure, becomes energetic and liberated while living. By his contact all the T\={\i}rthas become purified. All sins vanish away from him as snakes fly away at the sight of Garuda. The Dv\={\i}ja who becomes void of Sandhy\=a three times a day, the Devas do not accept his worship nor the Pitris accept his Pindas. He who has no Bhakti towards the M\=ula Prakriti, who does not worship the specific seed Mantra of M\=ay\=a and who does not hold festivities in honour of M\=ula Prakriti, know him verily to be an Ajagara snake without poison. Devoid of the Vi\d{s}\d{n}u mantra, devoid of the three Sandhy\=as and devoid of the fasting on the Ek\=ada\'si Tithi (the eleventh clay of the fortnight), the Br\=ahmi\d{n} becomes a snake devoid of poison. The vile Br\=ahmi\d{n} who does not like to take the offerings dedicated to Hari and who does the washerman's work and eats the food of \'S\=udra and drives the buffaloes, becomes a snake devoid of poison. The Br\=ahmi\d{n} who burns the dead bodies of the \'S\=udras, becomes like the man who is the husband of an unmarried girl. The Br\=ahmi\d{n} also who becomes a cook of a \'S\=udra, becomes a snake void of poison. The Br\=ahmi\d{n} who accepts the gifts of a \'S\=udra, who performs the sacrifice of a \'S\=udra, who lives as clerks and warriors becomes like a snake void of poison. The Br\=ahmi\d{n} who sells his daughter, who sells the name of Hari or eats the food of a woman who is without husband and son, as well as of one who has just bathed after her menstruation period, becomes like a serpent void of poison. The Br\=ahmi\d{n} who takes the profession of pimps and pampers and lives on the interest, is also like a serpent void of poison. The Br\=ahmi\d{n} who sleeps even when the Sun has risen, eats fish, and does not worship the Dev\={\i} is also like a poisonless serpent. Thus stating all the rules of worship in order, the best of the Munis told him the Dhy\=anam, etc., of the Dev\={\i}

S\=avitr\={\i}, what he wanted. Then he informed the King of all the mantras and went to his own \=A\'srama. The king, then worshipped accordingly and saw the Dev\={\i} S\=avitr\={\i} and got boons.

41-43. N\=arada said :-- What is the S\=avitr\={\i}'s Dhy\=an, what are the modes of her worship, what is stotra, mantra, that Par\=a\'sara gave to the King before he went away? And how did the King worship and what boon did he get? This great mystery, grand and well renowned in the \'Srutis, about S\=avitr\={\i}, I am desirous to hear in brief on all the points.

44-78. N\=ar\=aya\d{n}a said :-- On the thirteenth day (the trayoda\'si tithi) of the black fortnight in the month Jyaistha or on any other holy period, the fourteenth day (the chaturda\'si tithi) this vow is to be observed with great care and devotion. Fourteen fruits and fourteen plates with offerings of food on them, flowers and incense are to be offered and this vow is to be observed for fourteen years consecutively. Garments, holy threads and other articles are also offered and after the Vrata is over, the Br\=ahmi\d{n}s are to be fed. The lucky pot (mangal ghat) is to be located duly according to the rules of worship with branches and fruits. Ga\d{n}e\'sa, Ag\d{n}i, Vi\d{s}\d{n}u, \'Siva and \'Siv\=a are to be worshipped duly.

In that ghat S\=avitr\={\i} is to be next invoked and worshipped. Now hear the Dhy\=anam of S\=avitr\={\i}, as stated in the M\=adhyan Dina Sakh\=a, as well the stotra, the modes of worship, and the Mantra, the giver of all desires. I meditate and adore that S\=avitr\={\i}, the Mother of the Vedas, of the nature of Pra\d{n}ava (Om), whose colour is like the burnished gold, who is burning with Brahm\=a teja (the fire of Brahm\=a), effulgent with thousands and thousands of rays of the midday summer Sun, who is of a smiling countenance adorned with jewels and ornaments, wearing celestial garment (purified and uninflammable by fire), and ready to grant blessings to Her Bhaktas; who is the bestower of happiness and liberation, who is peaceful and the consort of the Creator of the world, who is all wealth and the giver of all riches and prosperity, who is the Presiding Deity of the Vedas and who is the Vedas incarnate, I meditate on Thee. Thus reciting the Dhy\=anam, mantra and meditating on Her, one is to offer Naivedyas (offerings of food) to Her and then place one's fingers on one's head; one is to meditate again, and then invoke the Dev\={\i} within the pot. One should next present fourteen things, uttering proper mantras according to the Vedas. Then one must perform special p\=uj\=a and chant hymns to the Dev\={\i} and worship Her. The fourteen articles of worship are as under :--

(1) Seat (\=Asan); (2) water for washing feet (P\=adya), (3) offering of rice and Durba grass (Arghya), (4) water for bath (Sn\=an\={\i}ya), (6) anointment with sandalpaste and other scents (Anulepana), (7) incense (Dh\=upa), (8) Lights (Dipa), (9) offerings of food (Naivedya), (10) Betels (Tamb\=ul), (11) Cool water, (12) garments, (13) ornaments, (14) garlands, scents, offering of water to sip, and beautiful bedding. While offering these articles, one is to utter the mantras, this beautiful wooden or golden seat, giving spiritual merits is being offered by me to Thee. This water from the T\={\i}rthas, this holy water for washing Thy feet, pleasant, highly meritorious, pure, and as an embodiment of P\=uj\=a is being offered by me to Thee. This holy Arghya with Durba grass and flowers and the pure water in the conch-shell is being offered by me to Thee (as a work of initial worship). This sweet scented oil and water being offered by me to Thee with devotion for Thy bathing purposes. Kindly accept these. O Mother! This sweet-scented water Divine-like, highly pure and prepared of Kunkuma and other scented things I offer to Thee. O Parame\'svar\={\i}! This all-auspicious, all good and highly meritorious, this beautiful Dh\=upa, kindly take, O World Mother! This is very pleasant and sweet scented; therefore I offer this to Thee. O Mother! This light, manifesting all this Universe and the seed, as it were, to destroy the Darkness is being offered by me to Thee. O Dev\={\i}! Kindly accept this delicious offering of food, highly meritorious, appeasing hunger, pleasant, nourishing end pleasure giving. This betel is scented with camphor, etc., nice, nourishing, and pleasure-giving; this is being offered by me to Thee. This water is nice and cool, appeasing the thirst and the Life of the World. So kindly accept this. O Dev\={\i}! Kindly accept this silken garment as well as the garment made of K\=arp\=asa Cotton, beautifying the body and enhancing the beauty. Kindly accept these golden ornaments decked with jewels, highly meritorious, joyous, beauteous and prosperous. Kindly accept these fruits yielding fruits of desires, obtained from various trees and of various kinds. Please have this garland, all auspicious and all good, made of various flowers, beauteous and generating happiness. O Dev\={\i}! Kindly accept this sweet scent, highly pleasing and meritorious. Please take this Sind\=ura, the best of the ornaments, beautifying the forehead, highly excellent and beautiful. Kindly accept this holy and meritorious threads an purefied by the Vedic mantrams and made of highly holy threads and knitted with highly pure knots. Uttering thus, offer the above articles that are to be offered to the Dev\={\i}, every time the specific seed mantra being uttered. Then the intelligent devotee should recite the stotras and subsequently of the Dak\d{s}i\d{n}\=as (presents) with devotion to the Br\=ahma\d{n}as. The Radical or the Specific Seed Mantra mantra is the eight lettered mantra Sr\={\i}m Hr\={\i}m Kl\={\i}m Sv\=aitrai Sv\=ah\=a; So the sages know. The Stotra, as stated in the M\=adhyand\={\i}na

\'S\=akh\=a, gives fruits of all desires. I am now speaking to you of that mantra, the Life of the Br\=ahma\d{n}as. Listen attentively. O N\=arada! S\=avitr\={\i} was given to Brahm\=a, in the ancient times of old in the region of Goloka by Kri\d{s}\d{n}a; but S\=avitr\={\i} did not come to Brahm\=a loka with Brahm\=a. Then by the command of Kri\d{s}\d{n}a, Brahm\=a praised the mother of the Vedas. And when She got pleased, She accepted Brahm\=a as Her husband.

79-87. Brahm\=a said :-- Thou art the everlasting existence intelligence and bliss; Thou art M\=ulaprakriti; thou art Hira\d{n}ya Garbha; Thou didst get pleased, O Fair one! Thou art of the nature of fire and Energy; Thou art the Highest; Thou art the Highest Bliss, and the caste of the twice-born. Dost thou get appeased, O Fair One! Thou art eternal, dear to the Eternal; thou art of the nature of the Everlasting Bliss. O Dev\={\i}, O Thou, the all auspicious One! O Fair One! Beest thou satisfied. Thou art the form of all (omnipresent)! Thou art the essence of all mantras of the Br\=ahma\d{n}as, higher than the highest! Thou art the bestower of happiness and the liberator O Dev\={\i}, O Fair One! Beest thou appeased. Thou art like the burning flame to the fuel of sins of the Br\=ahma\d{n}as! O Thou, the Bestower of Brahm\=a teja (the light of Brahm\=a) O Dev\={\i}! O Fair One! Beest appeased. By Thy mere remembrance, all the sins to me by body, mind and speech are burnt to ashes. Thus saying, the Creator of the world reached the assembly there. Then S\=avitr\={\i} came to the Brahmaloka with Brahm\=a. The King A\'svapati chanted this stotra to S\=avitr\={\i} and saw Her and got from Her the desired boons. Whosoever recites this highly sacred king of Stotras after Sandhy\=a Bandanam, quickly acquires the fruits of studying the Vedas.

Here ends the Twenty Sixth Chapter of the Ninth Book on the narration of S\=avitr\={\i} in \'Sr\={\i}mad Dev\={\i} Bh\=agavatam of 18,000 verses by Mahar\d{s}i Veda Vy\=asa.



