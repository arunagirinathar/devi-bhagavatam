\chapter{On the greatness in holding the Bibh\=uti}

1-17. N\=ar\=aya\d{n}a said :-- O N\=arada! Whatever is given as charities to any man besmeared with the holy ashes, takes away instantly all the sins of the donor. The \'Srutis, Smritis, and all the Pur\=a\d{n}as declare the greatness of this Bhasma. So the twice-born must accept this. Whoever holds this Tripundra, of this holy ashes at the three Sandhy\=a times, is freed from all his sins and goes to the region of \'Siva. The Yogi who takes a bath of ashes throughout his body during the three Sandhy\=as, gets his Yoga developed soon. By this bath of ashes, many generations are lifted up. O N\=arada! This ash bath is many times superior to the water bath. To take once a bath of ashes secures to one all the merits acquired by bathing in all the sacred places of pilgrimages. There is no doubt in this. By this bath of ashes, all the Mah\=ap\=atakas (great heinous sins) and other minor sins as well are instantly destroyed as heaps of wood are brought down to ashes in a moment by the fire. No bath is holier than this one. This is first mentioned by \'Siva and He took Himself this bath. Since then this bath of ashes has been taken with great care by Brahm\=a and the other Devas and the Munis for their own good in all the virtuous actions. This bath of ashes is termed the bath of fire. So he who applies ashes on his head, gets the state of Rudra while he is in this body of five elements. Those who are delighted to see persons with this ashes on their bodies are respected by the Devas, Asuras, and Munis. He who honours and gets up on seeing a man besmeared with ashes is respected even by Indra, the Lord of Heavens. Even if anybody eats any uneatables, then the sin incurred thereby won't touch him, if his body be then besmeared with ashes. He who first takes a water bath and then an ash-bath, be he a Brahmach\=ar\={\i} or an house-holder or an anchorite (V\=anaprasth\={\i}) is freed of all sins and gets in the end the highest state. Specially for the Yatis (ascetics), this ash bath is very necessary. This ash bath is superior to the water bath. For the bonds of Nature, this pleasure and pain, are cut asunder by this ash bath. The Munis know this Prakriti as moist and wet; and therefore Prakriti binds men. If anybody desires to cut asunder this bondage of the body, he will find no other remedy for this in the three worlds than this Holy Bath of ashes.

18-43. In ancient days the ashes were first offered to the Dev\={\i} gladly by the Devas for their protection, their good and purification, when they first saw the ashes. Therefore anybody who takes this bath of fire, gets all his sins destroyed and he goes to \'Siva Loka. He who daily uses this ashes has not to suffer from the oppression of the R\=ak\d{s}asas, Pi\'s\=achas, P\=utan\=as and the other Bh\=utas or from disease, leprosy, the chronic enlargement of spleen, all sorts of fistulae, from eighty sorts of rheumatism, sixty four kinds of bilious diseases, twenty two varieties of phlegmatic diseases and from tigers, thieves, and other vicious planetary influences. Rather he gets the power to suppress all these as a lion kills easily a mad elephant. Anybody who first mixes the ashes with pure cold water and then besmears his body with that and puts on the Tripundras, attains soon the Highest Brahm\=a. He who holds the Tripundra of ashes becomes sinless and goes to the Brahm\=a loka. He can even wipe off the ordnances of the fate on his forehead to go to the jaws of Death, if he uses, according to the \'S\=astras, the Tripundras on his forehead. If the ashes be used on the neck, then the sin, incurred through the neck, is completely destroyed. If the ashes be used on the neck, then the sin incurred by the neck, in eating uneatable things is entirely destroyed. If the ashes be held on the arms, then the sin incurred by the arms is destroyed. If it be held on the breast, the sin done mentally is destroyed. If it be held on the navel, the sin incurred by the generative organ is destroyed. If it be held on the anus, then the sin incurred by the anus is destroyed. And if it be held on the sides, then the sin incurred in embracing other's wives is destroyed. So, know fully, to use ashes is highly commendable. Everywhere three curved lines of ashes are to be used. Know these three lines as Brahm\=a, Vi\d{s}\d{n}u and Mahe\'sa; Dak\d{s}i\d{n}\=agni, G\=arhapatya fire and \=Ahavan\={\i}va fire; the S\=attva, R\=ajas and T\=amas qualities, Heaven, earth and P\=at\=ala (nether regions). If the wise Br\=ahmi\d{n} holds properly the ashes his Mah\=ap\=atakas are destroyed. He is not involved in any sin. Rather he, without any questionings, gets his liberation. All the sins, in the body besmeared with ashes, are burnt down by the ashes, which is of the nature of fire, into ashes. He is called Bhasmanistha (a devotee of Bhasma, i.e., ashes) who takes a bath of ashes, who besmears his body with ashes, who use the Tripundras of ashes, who sleeps in ashes. He is called also \=Atmanistha, a devotee of \=Atman (Self). At the approach of such a man, the Demons, Pi\'s\=achas, and very serious diseases run away to a distance. There is no doubt in this. In as much as these ashes reveal the knowledge of Brahm\=a, it is called Bhasita from Bhasma, to shine; because it eats up the sins, it is called Bhasma; because it increases the eight supernatural powers Anim\=a, etc., it is called

Bh\=uti; because it protects the man who uses it, it is called ``Rak\d{s}\=a.'' As the sins are all destroyed by the mere remembrance of Bhagav\=an Rudra, so seeing the person using the Tripundra, the demons, bad spirits and other vicious hosts of spirits fly away quickly, trembling with fear. As a fire burns a great forest by its own strength, so this bath of ashes burns the sins of those who are incessantly addicted to sins. Even if at the time of death one takes a bath of ashes, though he has committed an inordinate amount of vices, all his sins are soon destroyed. By this bath of ashes, the Self is purified, the anger is destroyed; the senses are calmed down. The man who uses even once this Bhasma comes to Me; he has not to take any more births in future. On Monday Am\=avasy\=a (also on the full moon day) if one sees the \'Siva Lingam, with his body besmeared all over with ashes, one's sins will all be destroyed. (All the sins are not seen; hence the tithi is called Am\=avas.) If people use Bhasma daily, all their desires will be fructified whether they want longevity, or prosperity or Mukti. The Tripundra that represents Brahm\=a, Vi\d{s}\d{n}u and \'Siva is very sacred. Seeing the man with Tripundra on, the fierce R\=ak\d{s}asas or mischievous creatures flee to a distance. There is no doubt in this. After doing the \'Saucha (necessary cleanliness) and other necessary things, one bathes in pure cold water and besmears his body with ashes from head to foot. By taking the water bath only, the outward unclean things are destroyed. But the ash bath not only cleanses the outer external uncleanliness but cleanses also all the internal uncleanliness. So even if one does not take the water bath, one ought to take this ash bath. There is to be no manner of doubt in this.

44-47. All the religious actions performed without this ash bath seem as if no actions are done at all. This ash bath is stated in the Vedas. Its another name is the Fire Bath. By this ash bath both outside and inside are purified. So a man who uses ashes gets the entire fruit of worshipping \'Siva. By the water Bath only the outside dirt is removed; but by this bath of ashes, outside dirts and inside dirts, both are fully removed. If this water bath be taken many times daily, still without an ash bath, one's heart is not purified. What more shall I speak of the greatness of ashes, the Vedas only appreciate its glories rightly! Yea, very rightly!

48-50. Or Mah\=a Deva, the Gem of all the Devas, knows the greatness of this Bhasma. Those who perform rites and works prescribed

by the Vedas, without taking this bath of ashes, do not get even a tithe of the fruits of their works done. Only that man will be entitled to the entire fruits of the Vedas who perform this bath of ashes duly. This is the opinion of the Vedas. This bath of ashes purifies more the things that are already pure; thus the \'Sruti says. That wretch who does not take the bath of ashes as aforesaid is a Great Sinner. There is no doubt in this. By this bath greater interminable merits accrue than what is obtained by innumerable baths taken by the Br\=ahma\d{n}as on the V\=aru\d{n}\={\i} momentous occasion. So take this bath carefully in the morning, midday and evening. This bath of ashes is ordained in the Vedas. So know those who are against this bath mentioned in the Vedas, are verily fallen! After evacuating oneself of one's urine and faeces, one ought to take this bath of ashes. Otherwise men will not be purified. Even if one performs duly the water bath and if one does not take this bath of ashes, that man will not be purified. So he cannot get any right to do any religious actions. After evacuating one's abdomen of the outgoing air, after yawning, after holding sexual intercourses, after spitting and sneezing, and after easing oneself of phlegm, one ought to take this bath of ashes. O N\=arada! Thus I have described to you here the greatness of \'Sr\={\i} Bhasma. I am again telling you more of it specially. Listen attentively.

Here ends the Fourteenth Chapter of the Eleventh Book on the greatness in holding the Bibh\=uti (ashes) in the Mah\=apur\=a\d{n}am \'Sr\={\i} Mad Dev\={\i} Bh\=agavatam of 18,000 verses by Mahar\d{s}i Veda Vy\=asa.



