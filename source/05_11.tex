\chapter{On the appearing of the D\=anava T\=amra before the Dev\={\i}}

1-3. Vy\=asa said :-- The King Mahi\d{s}\=asura, maddened with pride, heard the messenger's words and called the aged and experienced ministers and said thus :-- O Ministers! What am I to do now? Better judge you all well, and speak out definitely to me. Is it that this Dev\={\i} has been created by the Devas like the M\=ay\=a of Sambar\=asura and thus has appeared before us? You are all dexterous and know where to apply the four means of success, viz., conciliation, gift or bribery, sowing dissensions, and war; and therefore you would better tell me which one of the above four, I am to adopt now.

4-7. The ministers said :-- O King! One should always speak true and at the same time pleasant; the wise ones should then select only those which are beneficial and apply them. O King! As a medicine, though bitter, cures diseases, so true words, though appearing unpleasant, lead to beneficial results. Those that are simply pleasant, are generally injurious as to their effects. O Lord of the Earth! The bearers and approvers of truth both are very rare; truth speakers also are very difficult to be seen; laudatory sycophancy is found in a great measure in this world. O King! Nobody in the three worlds knows what will be good or what will lead to inauspicious results? How can we then definitely pronounce our judgment in this difficult matter?

8-9. The King said :-- Let each of you say separately, according to his own intellect, what is his opinion; I will hear them all and consider

for myself. Clever persons should hear the opinions of several persons, then judge for himself what is the best and then adopt that as what is to be done.

10. Vy\=asa said :-- Hearing his words, the powerful Vir\=up\=ak\d{s}a came out foremost of all and began to say pleasant words to the King.

11-16. O King! Please take for certain, what has been spoken by that ordinary woman, swelled with vanity, as words simply to scare you. The efforts and courage of a woman are known to all; who will be afraid therefore, to hear abusive language from a woman, praising her ownself in matters of warfare? O King! You have conquered the three worlds by your own heroic valour; now if you acknowledge your inferiority, out of fear to a woman, you would be subject to very much disgrace in this world. Therefore, O King! I will go alone to fight with Chandik\=a and I will kill Her. You can stay here now without any fear. O King! See my prowess now; I am just now going with my army and I will kill that violent Chandik\=a, maddened with pride, or I will tie Her down by a coil of snakes and bring Her before you; then that Lady, seeing Herself helpless, will become quite submissive to you; there is no doubt in this.

17-30. Vy\=asa said :-- Hearing these words of Vir\=up\=ak\d{s}a, Durdhara said :-- O King! Vir\=up\=ak\d{s}a is very intelligent; what he has said just now is all reasonable and true. O King! You are intelligent; hear my words full of truth also. As far as I think, I consider that woman with beautiful teeth as passionate. For that woman of broad hips has expressed a desire to bring you under control by making you fearful; the mistresses, proud of their beauty generally use such words when they become passionate. When they behave in this way, people call these amorous gestures. These crooked words of mistresses are the chief causes in attracting dear persons unto them. Those who are skilled in the art of love affair, some of them can know these things thoroughly well. O King! That woman has said, ``I will pierce and kill you by arrows, face to face, in the battlefield.'' The sense of this is different. The wise persons that are clever and experienced in the art of finding out the cause, declare that the above sentence is pregnant with deep and esoteric meaning. You can easily see that the handsome women have no other arrows with them; their side-glances are their arrows. And their words carry their hidden meanings, and, expressing their desires, are their flowers. O King! Brahm\=a, Vi\d{s}\d{n}u and Mahe\'sa even have no powers to shoot arrows at you; how can, then, that helpless woman, who appears so passionate, dart real arrows at you? O King! That lady said :-- ``O Stupid! I will kill your King by my arrow-like eye-sight.'' But the messenger was wanting in that power to appreciate; so he, no doubt, understood her words in their

contrary sense. The saying of that lady, ``I will lay your lord in the death-bed in the battle-field'' is to be taken in the light of inverted sexual intercourse, where woman is above the man. Her utterance, ``I will take away the vitality (life) of your lord'' is also significant. The semen virile is known as the vitality (life). Therefore the above expression means that she will make you devoid of your virility. There can be no other meaning. O King! Those women that are excellent shew by too much of their covert expressions (innuendos) that they select and like very much their beloved. The experts only in these amorous affairs will be able to appreciate these things. Knowing thus, dealings ought to be made with Her so that the harmony in amorous sentiments be not broken. O King! S\=ama (conciliation) and D\=ana (gifts) are the two means to be adopted; there is no other way. By these two, that Lady, whether she be proud or angry, is sure to have brought under control; I will go now and bring Her before you by such sweet words. O King! What is the use of my talking too much? I will make Her submissive to you like a slave girl.

31-44. Vy\=asa said :-- Hearing those words of Durdhara, the D\=anava T\=amra, who was very experienced in finding out the real nature, said :-- ``I am telling you what is sanctioned by virtue and is at the same time full of sweet amorous feelings, pregnant with deep meanings. Kindly hear; O Giver of honour! This intelligent woman is not at all passionate nor devoted to you; nor has that woman used any covert expressions to you. O Great Hero! This is strange indeed that a Lady, beautiful, handsome, and of strange features, at the same time alone and helpless, has come here to fight. A good-looking woman, powerful, and having eighteen hands is never heard of, nor ever seen by me in these three worlds. She is holding in each of Her hands powerful weapons. O king! All these seem to be the contrary actions of Time. O King! I saw ominous dreams during the night; and I conclude, therefore, that great dangers are over our heads. Early in the morning twilight, I saw in my dream that a woman, wearing a black raiment, was weeping in the inner courtyard; that some inauspicious events are forthcoming can be easily judged from the above. O King! The birds were screaming hoarsely in every house and various calamitous events were seen in various houses; at this time that woman, firmly resolved, was challenging you to fight; it, therefore seems to me that there is something very serious in this matter. O Lord! This woman is neither human, nor a Gandharv\={\i}, nor the wife of any Asura. Only to cause delusion to us, she, this wondrous M\=ay\=a has been created by the gods. O King! In no case, weakness is to be resorted; it is wise by all means to fight as best as possible; what is inevitable will come to pass; this is my opinion. No one is able to unriddle the doings

of the Devas, whether they would be auspicious or inauspicious. Therefore intelligent ones should weigh pros and cons carefully and remain patient and steady. O King! Life or death is at the hands of Destiny; Nobody, therefore, can do it otherwise.''

45-51. Hearing this, Mahi\d{s}\=asura said :-- ``O Highly fortunate T\=amra! Better, then, stand for fight, fully resolved and go to that Lady, beautiful, and conquer Her according to rules of justice and bring Her before me. In case She does not come under your control in fight, kill Her; but if She comes round, then show Her honour; do not kill Her. O All-knowing! You are a great hero and at the same thoroughly conversant with K\=ama \'S\=astra (science of love); therefore conquer that Fair One by any means you can. O valiant T\=amra, of mighty prowess! Go then with a mighty force and ponder over again and again and find out Her intention. Is She prompted by passion or by real inimical feeling or by any other motive? Try to find out whose M\=ay\=a is this? Know all these beforehand; then find out the remedy; next fight with Her according to your strength and prowess. Weakness should not be shown nor merciless behaviour is to be resorted; you should behave with Her according to the bent of Her mind.''

52. Vy\=asa said :-- O king! Thus hearing the King's words, T\=amra coming as if under the sway of Death, saluted the king Mahi\d{s}a and marched away with his army.

53-66. That wicked D\=anava, who, on his way, began to see all the fearful inauspicious signs, indicative of Death, became surprised and was caught with fear. When he arrived at the spot, he saw the Dev\={\i} standing on a lion, while She was decorated with all the weapons and instruments, and all the Devas were chanting hymns to Her. T\=amra, then bowed down before Her with humility and modesty and addressed Her with sweet words, according to the rules of the policy of conciliation. ``O Dev\={\i}! Mahi\d{s}a, the lord of the Daityas, has become enchanted on hearing Your beauty and qualifications and has become desirous to marry You. O Beautiful One! You would better be graciously pleased with that conqueror of the Immortals, the Mahi\d{s}\=asura; O Thou of delicate limbs! Make him your husband and enjoy all the exquisite pleasures of the Nandana garden as best as you can. The end and aim of attaining this human form, beautiful in every respect and the abode of all bliss, is to enjoy, in every way, all the pleasures of human existence and to avoid the sources of all troubles. This is the rule.

``O Thou of beautiful thighs like those of the young of an elephant! Your soft and delicate lotus-like hands are fit to play only with nice balls of

flowers; why then are You holding in Your hands all the weapons and arrows? What is the use of holding ordinary arrows, when those two eye-brows like bows, are existing with You? What need have you to take ordinary arrows when you are graced with those piercing eye sights, your arrows. The war is exceedingly painful in this world; those who know thus ought never to fight. It is only those human beings that are prompted by greed that fight with each other. What to speak of those sharpened arrows, one ought not to fight with flowers even; O Dev\={\i}! You can well say who is it that feels pleasure, when one's own body is pierced? Therefore, O Delicate One! Gladly you can worship Mahi\d{s}a, the lord of the world and the object of worship of the Devas and D\=anavas. Then he will satisfy all your desires. What more to say, you will no doubt be his queen-consort. O Dev\={\i}! If one tries one's best, it is doubtful whether one would be crowned with success; therefore keep my this request; you will surely get all the best pleasures. O Beautiful! You are well acquainted with all the politics; therefore you better enjoy thoroughly the pleasures of the kingdom for full many years. And if you marry Mahi\d{s}a you will have beautiful sons and those sons again will be kings; and enjoying the pleasures of your full grown womanhood, you will no doubt, be happy in your old age.''

Here ends the Eleventh Chapter of the Fifth Book on the appearing of the D\=anava T\=amra before the Dev\={\i} in \'Sr\={\i} Mad Dev\={\i} Bh\=agavatam, the Mah\=a Pur\=a\d{n}am, of 18,000 verses, by Mahar\d{s}i Veda Vy\=asa.



