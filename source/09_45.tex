\chapter{On the anecdote of Dak\d{s}i\d{n}\=a}

1-63. N\=ar\=aya\d{n}a said :-- The excellent, most sweet stories of Sv\=ah\=a and Svadh\=a are told; now I will tell you the story of Dak\d{s}i\d{n}\=a; hear attentively. In early days, in the region of Goloka, there was a good-natured Gop\={\i} named Su\'s\={\i}l\=a, beautiful, R\=adh\=a's companion and very dear to \'Sr\={\i} Hari. She was fortunate, respected, beautiful, lovely, prosperous, with good teeth, learned, well qualified and of exquisitely handsome form. Her whole body was tender and lovely like Kal\=avat\={\i} (one versed in 64 arts; moon). She was beautiful and her eyes were like water lilies. Her hips were good and spacious; Her breasts were full; she was Shy\=am\=a (a kind of women having colour like melted gold; body being hot in cold and cool in summer; of youthful beauty); as well She was of the Nyagrodha class of women (an excellent woman. Stanau Sukathinau Yasy\=a Nitambe cha Vi\'s\=alat\=a. Madhye Ks\={\i}\d{n}\=a bhavedy\=a S\=anyagrodha parimandal\=a). Always a smile sweetened Her face; and that looked always gracious. Her whole body was covered with jewel ornaments. Her colour was white like the white champakas. Her lips were red like the Bimba fruits; Her eyes were like those of a deer.

Su\'s\={\i}l\=a was very clever in amorous sciences. Her gait was like a swan. She was specially versed in what is called Prema Bhakti (love towards God). So She was the dearest lady of \'Sr\={\i} Kri\d{s}\d{n}a. And She was of intense emotional feelings. She knows all the sentiments of love; she was witty, humorous, and ardent for the love of \'Sr\={\i} Kri\d{s}\d{n}a, the Lord of the R\=asa circle. She sat by the left side of \'Sr\={\i} Kri\d{s}\d{n}a in the presence of R\=adh\=a. \'Sr\={\i} Kri\d{s}\d{n}a, then cast His glance on R\=adh\=a, the Chief of the Gop\={\i}s and hung down His head through fear. R\=adh\=a's face turned red; the two eyes looked like red lilies; all Her bodies began to quiver out of anger and Her lips began to shake. Seeing that state of R\=adh\=a, Bhagav\=an \'Sr\={\i} Kri\d{s}\d{n}a disappeared, fearing that a quarrel might ensue. Su\'s\={\i}l\=a and other Gop\={\i}s seeing that the peaceful Kri\d{s}\d{n}a of Sattv\=a Gu\d{n}a and of lovely form had disappeared, began to tremble with fear. Then one lakh Koti

Gop\={\i}s seeing Kri\d{s}\d{n}a absent and R\=adh\=a angry, became very much afraid and bowing their heads down with devotion and with folded palms began to say frequently, ``R\=adhe! Protect us, protect us,'' and they took shelter at Her feet. O N\=arada! Three lakh Gopas also including Sud\=am\=a and others took shelter at the lotus feet of \'Sr\={\i} R\=adh\=a out of fear. Seeing, then, Kri\d{s}\d{n}a absent and Her companion Su\'s\={\i}l\=a running away, R\=adh\=a cursed her thus :-- ``If Su\'s\={\i}l\=a comes again to this Goloka, she will be reduced to ashes.'' Thus cursing Her companion Su\'s\={\i}l\=a out of anger, R\=adh\=a, the Darling of the Deva of the Devas, and the Lady of the R\=asa circle went to the R\=asa circle and called on Kri\d{s}\d{n}a, the Lord of the same. Not being able to find out \'Sr\={\i} Kri\d{s}\d{n}a, a minute appeared a yuga to Her and she began to say :-- ``O Lord of Pr\=a\d{n}as! O Dearer than My life! O Presiding Deity of my life! O Kri\d{s}\d{n}a! My life seems to depart from Thy absence! Come quickly and show Thyself to me. O Lord! It is through the favour of one's husband that the pride of women gets increased day by day. Women's safeguards of happiness are their husbands. Therefore women, who are helpless creatures, ought always to serve their husbands according to Dharma. The husband is the wife's friend, presiding deity and the sole refuge and the chief wealth. It is through husbands that women derive their pleasures, enjoyments, Dharma, happiness, peace and contentment. If husbands are respected, wives are respected and if husbands are dishonoured, women are dishonoured too. The husband is the highest thing to a woman. He is the highest friend. There is no better friend than him. The husband is called Bhart\=a because he supports his wife; he is called Pati, because he preserves her; he is called \'Sv\=am\={\i}, because he is the master of her body; he is called K\=ant\=a because he bestows the desired things to her; he is called Bandhu, because he increases her happiness; he is called Priya, because he gives pleasure to her; he is called I\'sa, because he bestows prosperity on her; he is called Pr\=a\d{n}e\'svara, because he is the lord of her Pr\=a\d{n}a; and he is called Rama\d{n}a, because he gives enjoyment to her. There is no other thing dearer than husband. The son is born of the husband; hence the son is so dear. The husband is dearer to a family woman than one hundred sons. Those who are born in impure families, cannot know what substance a husband is made up of. Taking Baths in all the T\={\i}rthas, giving Dak\d{s}i\d{n}\=as in all the Yaj\~nas (sacrifice), circumambulating round the whole earth, performing all austerities, observing all vows, making all great gifts, holy fastings, all that are dictated in the \'S\=astras, serving the Guru, the Br\=ahma\d{n}as and the Devas all cannot compare to even one sixteenth part with serving faithfully the feet of the husband. The husband is the highest; higher than the Spiritual Teacher (Guru),

higher than the Br\=ahma\d{n}as, higher than all the Devas. As to man, the Spiritual Teacher who imparts the Spiritual Knowledge is the Best and Supreme, so to the women their husbands are the best of all. Oh! I am not able to realise the glory of my Dearest, by Whose favour I am the Sole Ruler of one lakh Koti Gop\={\i}s, one lakh Koti Gopas, innumerable Br\=ahmandas, and all the things thereof, and all the lokas (regions) from Bhu (earth) to Goloka. Oh! The womanly nature is insurmountable.'' Thus saying, R\=adhik\=a began to meditate with devotion on \'Sr\={\i} Kri\d{s}\d{n}a. Tears began to flow incessantly from Her eyes. She exclaimed, ``O Lord! O Lord! O Rama\d{n}a! Shew Thyself to me. I am very much weak and distressed from Thy bereavement.'' Now the Dak\d{s}i\d{n}\=a Dev\={\i}, driven out from Goloka; practised Tapasy\=a for a long time when She entered into the body of Kamal\=a. The Devas, on the other hand, performed a very difficult Yaj\~na; but they could not derive any fruit therefrom. So they went to Brahm\=a, becoming very sad. Hearing them, Brahm\=a meditated on Vi\d{s}\d{n}u for a long time with devotion. At last Vi\d{s}\d{n}u gave Him a reply. Vi\d{s}\d{n}u got out of the body of Mah\=a Lak\d{s}m\={\i} a Martya Lak\d{s}m\={\i} (Lak\d{s}m\={\i} of the earth) and gave Her Dak\d{s}i\d{n}\=a to Brahm\=a.Then with a view to yield to the Devas the as a fruits of their Karmas, Brahm\=a made over to the Yaj\~na Deva (the Deva presiding the sacrifice) the Dev\={\i} Dak\d{s}i\d{n}\=a, offered by N\=ar\=aya\d{n}a. Yaj\~na Deva, then, worshipped Her duly and recited hymns to Her with great joy. Her colour was like melted gold; her lustre equalled koti Moons; very lovely, beautiful, fascinating; face resembling water lilies, of a gentle body; with eyes like Padm\=a Pal\=asa, born of the body of Lak\d{s}m\={\i}, worshipped by Brahm\=a, wearing celestial silken garments, her lips resembling like Bimba fruits, chaste, handsome; her braid of hair surrounded by M\=alat\={\i} garlands; with a sweet smiling face, ornamented with jewel ornaments, well dressed, bathed, enchanting the minds of the Munis, below the hair of her forehead the dot of musk and Sind\=ura scented with sandalpaste, of spacious hips, with full breasts, smitten by the arrows of K\=ama Deva (the God of Love). Such was the Dak\d{s}i\d{n}\=a Dev\={\i}. Seeing Her, the Yaj\~na Deva fainted. At last he married her according to due rites and ceremonies. Taking her to a solitary place, he enjoyed her for full divine one hundred years with great joy like Lak\d{s}m\={\i} N\=ar\=aya\d{n}a. Gradually then Dak\d{s}i\d{n}\=a became pregnant. She remained so for twelve divine years. Then she duly delivered a nice son as the fruit of Karma. When any Karma becomes complete, this son delivers the fruits of that Karma. Yaj\~na Deva with His wife Dak\d{s}i\d{n}\=a and the above named Karmaphala, the bestower of the fruits of actions, gives the desired fruits to

all their sacrificial acts and Karmas. So the Pundits, the knowers of the Vedas, say. Really he, henceforth, began to give fruits to all the persons of their acts, with his wife Dak\d{s}i\d{n}\=a and son, the bestower of the fruits of the actions. The Devas were all satisfied at this and went away respectively to their own abodes. Therefore, the man who performs Karmas, generally known as Karma Kartas, should pay the Dak\d{s}i\d{n}\=a (the Sacrificial fee) and so he completes at once his actions. It is stated in the Vedas, that no sooner the Karma Karta pays the Dak\d{s}i\d{n}\=a, than he obtains the fruits of his Karmas at once. In case the Karma Karta, after he has completed his acts, does not pay either through bad luck or through ignorance, any Dak\d{s}i\d{n}\=a to the Br\=ahma\d{n}as, its amount is doubled if a Muh\=urta passes away and if one night elapses, its amount is increased, to one hundred times. If three nights pass away, and the Daksi\d{n}\=a not paid, the amount last brought forward, is increased again to hundred times; if a week passes, the last amount is doubled, and if one month passes away, the Dak\d{s}i\d{n}\=a is multiplied to one lakh times. If one year passes away, that is increased to ten millions of times and the Karma, also, bears no fruit. Such a Karma Karta is known as taking away unfairly a Br\=ahma\d{n}a's property and is regarded as impure. He has no right to any further actions. For that sin, he becomes a pauper and diseased. Lak\d{s}m\={\i} Dev\={\i} goes away from his house, leaves him, cursing him severely. So much so that the Pitris do not accept the \'Sr\=adh, Tarpa\d{n}am offered by that wretched fellow. So the Devas do not accept his worship, nor the Fire accepts the oblations poured by him. If the person that performs sacrifices does not pay the sacrificial fee that he resolves to pay and he who accepts the offer does not demand the sum, both of them go to hell. But if the performer of the sacrifices does not pay when the priests demand the fee, then the Yajam\=ana (the performer of the sacrifices) only falls down to hell as the jar, severed from the rope, falls down. The Yajam\=ana (pupil) is denominated as a Brahm\=asvapah\=ar\={\i} (one who robs a Br\=ahma\d{n}a's property); he goes ultimately to the Kumbh\={\i}p\=aka hell. There he remains for one lakh years punished and threatened by Yama's messengers. He is then reborn as a Ch\=and\=ala, poor and diseased. So much so that his seven generations above and his seven
generations below go to hell.

64-65. O N\=arada! Thus I have narrated to you the story of Dak\d{s}i\d{n}\=a. What more do you want to hear? Say. N\=arada said :-- ``O Best of Munis! Who bears the fruits of that Karma where no Dak\d{s}i\d{n}\=a is paid. Describe the method of worship that was offered to Dak\d{s}i\d{n}\=a by Yaj\~na Deva.'' N\=ar\=aya\d{n}a said :-- Where do you find the fruit of any sacrifice without Dak\d{s}i\d{n}\=a? (i.e., nowhere.) That Karma only gets

fruits where Dak\d{s}i\d{n}\=as are paid. And the fruits of the acts void of any Dak\d{s}i\d{n}\=a, Bali who lives in the P\=at\=ala only enjoys; and no one else.

66-71. For, in olden times, it was ordained by V\=amana Deva that those fruits would go to the king Vali. All those that pertain to \'Sr\=adh not sanctioned by the Vedas, the charities made without any regard or faith, the worship offered by a Br\=ahmi\d{n} who is the husband of a Vri\d{s}ala (an unmarried girl twelve years in whom menstruation has commenced), the fruits of sacrifices done by an impure Br\=ahma\d{n}a (a Br\=ahma\d{n}a who fails in his duties), the worship offered by impure persons, and the acts of a man devoid of any devotion to his Guru, all these are reserved for the king Bali. He enjoys the fruits of all these. O Child! I am now telling you the Dhy\=an Stotra, and the method of worship as per Ka\d{n}va \'S\=akh\=a of Dak\d{s}i\d{n}\=a Dev\={\i}. Hear. When Yaj\~na Deva, in ancient times got Dak\d{s}i\d{n}\=a, skillful in action, he was very much fascinated by her appearance and being love-stricken, began to praise her :-- ``O Beautiful One! You were before the chief of the Gop\={\i}s in Goloka. You were like R\=adh\=a; you were Her companion; and you were loved by Sr\={\i} R\=adh\=a, the beloved of \'Sr\={\i} Kri\d{s}\d{n}a.

72-97. In the R\=asa circle, on the Full Moon night in the month of K\=artik, in the great festival of R\=adh\=a, you appeared from the right shoulder of Lak\d{s}m\={\i}; hence you were named Dak\d{s}i\d{n}\=a. O Beautiful One! You were of good nature before; hence your name was Su\'s\={\i}l\=a. Next you turned due to R\=adh\=a's curse, into Dak\d{s}i\d{n}\=a. It is to my great good luck that you were dislodged from Goloka and have come here. O highly fortunate One! Now have mercy on me and accept me as your husband. O Dev\={\i}! You give to all the doers of actions, the fruits of their works. Without you, their Karmas bear no fruit. So much so, if you be not present in their action the works never shine forth in brilliant glory. Without Thee, neither Brahm\=a, nor Vi\d{s}\d{n}u nor Mahe\'sa nor the Regents of the quarters, the ten Dikp\=alas, can award the fruits of actions. Brahm\=a is the incarnate of Karma. Mahe\'svara is the incarnate of the fruits of Karmas; and I Vi\d{s}\d{n}u myself is the incarnate of Yaj\~nas. But Thou art the Essence of all. Thou art the Par\=a Prakriti, without any attributes, the Par\=a Brahm\=a incarnate, the bestower of the fruits of action. Bhagav\=an \'Sr\={\i} Kri\d{s}\d{n}a cannot award the fruits of actions without Thee. O Beloved! In every birth let Thou be my \'Sakti. O Thou with excellent face! Without Thee, I am unable to finish well any Karma.'' O N\=arada! Thus praising Dak\d{s}i\d{n}\=a Dev\={\i}, Yaj\~na Deva stood before Her. She, born from the shoulder of Lak\d{s}m\={\i}, became pleased with His Stotra and accepted Him for Her bridegroom. If anybody recites this Dak\d{s}i\d{n}\=a stotra during sacrifice, he gets all the results thereof.

If anybody recites this stotra in the R\=ajas\=uya sacrifice, V\=ajapaya, Gomedha (cow sacrifice) Naramedha (man sacrifice), A\'svamedha (horse sacrifice), L\=angala Sacrifice, Vi\d{s}\d{n}u Yaj\~na tending to increase one's fame, in the act of giving over wealth or pieces of lands, digging tanks or wells, or giving fruits, in Gaja medha (elephant sacrifice), in Loha Yaj\~na (iron sacrifice), Svarna Yaj\~na (gold sacrifice), Ratna Yaj\~na (making over jewels in sacrifices), T\=amra Yaj\~na (copper), \'Siva Yaj\~na, Rudra Yaj\~na, \'Sakra Yaj\~na, Bandhuka Yaj\~na, Varu\d{n}a Yaj\~na (for rains), Kandaka Yaj\~na, for crushing the enemies, \'Suchi Yaj\~na, Dharma Yaj\~na, P\=apa mochana Yaj\~na, Brahm\=a\d{n}\={\i} Karma Yaj\~na, the auspicious Prakriti Y\=aga, sacrifices, his work is achieved then without any hitch or obstacle. There is no doubt in this. The stotra, thus, is mentioned now; hear about the Dhy\=anam and the method of worship. First of all, one should worship in the \'S\=alagr\=ama stone, or in an earthen jar (Ghata) Dak\d{s}i\d{n}\=a Dev\={\i}. The Dhy\=anam runs thus :-- ``O Dak\d{s}i\d{n}\=a! Thou art sprung from the right shoulder of Lak\d{s}m\={\i}; Thou art a part of Kamal\=a; Thou art clever (Dak\d{s}a) in all the actions and Thou bestowest the fruits of all the actions. Thou art the \'Sakti of Vi\d{s}\d{n}u, Thou art revered, worshipped. Thou bestowest all that is auspicious; Thou art purity; Thou bestowest purity, Thou art good natured. So I meditate on Thee.'' Thus meditating, the intelligent one should worship Dak\d{s}i\d{n}\=a with the principal mantra. Then with the Vedic Mantras, p\=adyas, etc. (offerings of various sorts) are to be offered. Now the mantra as stated in the Vedas, runs thus :-- ``Om \'Sr\={\i}m Kl\={\i}m Hr\={\i}m Dak\d{s}i\d{n}\=ayai Sv\=ah\=a.'' With this mantra, all the offerings, such as p\=adyas, arghyas, etc., are to be given, and one should worship, as per rules, Dak\d{s}i\d{n}\=a Dev\={\i} with devotion. O N\=arada! Thus I have stated to you the anecdote of Daksin\=a. Happiness, pleasure, and the fruits of all karmas are obtained by this. Being engaged in sacrificial acts, in this Bh\=aratavar\d{s}a, if one hears attentively this Dhy\=anam of Dak\d{s}i\d{n}\=a, his sacrifice becomes defectless. So much so that the man who has got no sons gets undoubtedly good and qualified sons; if he has no wife, he gets a best wife, good natured, beautiful, of slender waist, capable to give many sons, sweet speaking, humble, chaste, pure, and Kul\={\i}na; if he be void of learning, he gets learning; if he be poor he gets wealth; if he be without any land, he gets land and if he has no attendants, he gets attendants. If a man hears for one month this stotra of Dak\d{s}i\d{n}\=a Dev\={\i}, he gets over all difficulties and dangers, bereavements from friends, troubles, inprisonments, and all other calamities.

Here ends the Forty-fifth Chapter of the Ninth Book on the anecdote of Dak\d{s}i\d{n}\=a in the Mah\=a Pur\=a\d{n}am \'Sr\={\i} Mad Dev\={\i} Bh\=agavatam of 18,000 verses by Mahar\d{s}i Veda Vy\=asa.



