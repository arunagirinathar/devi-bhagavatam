\chapter{On the story of Satyavrata}

1. Janamejaya said :-- O Mahar\d{s}i! Who was Satyavrata, the Br\=ahmi\d{n} whose name you have just taken? In what country was he born? Of what nature was he? Please describe all these to me and satisfy my curiosity?

2. How did he hear that sound ``Ai''; how did he repeat that word? How came out the success to him, that illiterate Br\=ahma\d{n}, at that very instant?

3. And how is it that that Great Goddess, who is omniscient and omnipresent, was pleased with him, kindly describe this interesting incident in detail.

4. S\=uta said :-- Vy\=asa, the son of Satyavat\={\i}, thus asked by the king, addressed in the following pure, sweet, and highly liberal words.

5. Vy\=asa said :-- Hear, O king! You are the best and foremost in the Kuru clan; what I before heard in the assembly of the Munis, I am now relating that ancient story, highly beneficial to you.

6. O best of the Kurus! Once in my peregrinations in the holy places of pilgrimages, I came to the Naimis\=aranya forest, that highly sacred place frequented by the Munis.

7-8. That time there were staying Sanaka, San\=atana and the other sons of Br\=ahma who were liberated while living. I went there and bowed down to the Munis and took my seat. Then the religious conversations ensued there in the assembly, when the great sage Mahar\d{s}i Jamadagni began to question the Munis in the following terms :--

9. O high-minded excellent ascetics and Munis! There has arisen a great doubt in my mind; I am desirous to have that doubt solved in this assembly of the Mahar\d{s}is.

10-12. O all-knowing Mahar\d{s}is that have fulfilled your vows! O Givers of one's honour! Now my question is this :-- Of the following Devas Br\=ahma, Vi\d{s}\d{n}u, Rudra, Indra, Varu\d{n}a, Fire, Kuvera, Wind, Visvakarm\=a, K\=artikeya, Ganesa, the Sun, the two A\'svins, Bhaga, P\=us\=a, Moon, and the other planets, who is the first and best to be worshipped, that can easily be served; who is very quickly satisfied and grants the desired boons; kindly tell me this as early as possible.

13. Thus questioned by the Muni Jamadagni, Mahar\d{s}i Loma\'sa, one in the assembly, spoke :-- O Jamadagni! Hear in reply to your question.

14-15. The Goddess of Energy is the best of the Devas, most excellent and highest to be worshipped. Those who want welfare, they ought to worship this Supreme Force. She is the Par\=a Prakriti, the Highest Nature, the Br\=ahma, conditioned by M\=ay\=a (Time, space, and causation). She grants all the desires, does good to all, pervades everywhere, and is the Mother of Br\=ahma and the other high souled Devas. She is the First Prakriti, and is the Root of this gigantic Tree of Universe.

16. If any one calls the Dev\={\i} in remembrance or distinctly utters Her Name, She fulfills all the desires of the human beings. If anybody worships Her, She is at once filled with mercy and becomes ready to grant boons.

17. O Munis! How, once on a time, at Br\=ahmi\d{n}, uttering one letter of Her mystical mantra, obtained Her Grace, I am now describing that most auspicious history before you. Be pleased to hear.

18. Once on a time, there lived in the country of Kosala,* a famous Br\=ahmi\d{n}, named Deva Datta. He had no issues and therefore started duly according to the prescribed rules a sacrifice called Puttresti for the sake of obtaining children.

*Kosala is a country situated, according to R\=am\=aya\d{n}a, along the banks of the Saray\=u (or Gogr\=a). It was divided into Uttara-Kosala and Dakshina Kosala. The former is also called Ganda and it must have therefore signified the country, north of Ayodhy\=a comprising Gonda and Bahraich. Aja and Dasaratha, etc., are said to have ruled over the province. At the time of R\=ama's death, his two sons Kusa and Lava reigned respectively at Kus\=avati in Southern Kosala in the defiles of the Vindhyas and at Sr\=avasti in northern Kosala.

19-20. On the banks of the Tamas\=a river, the Br\=ahmi\d{n} erected a temporary building (or an open shade) for performing the ceremony, and there built an altar and invited the Br\=ahmi\d{n}s, versed in the Vedas, and clever in performing sacrificial rites. There he placed the fire and began to perform according to the strict rules, the Puttresti sacrifice.

21-22. In that sacrifice, Suhotra, the best of the Munis acted the part of Br\=ahma (1); Y\=aj\~nyavalkya acted the part of Adhvaryu (2); Brihaspati, that of Hot\=a (3); Paila, that of Prastot\=a (4); Govila, that of Udg\=at\=a (6); and the other Munis acted as assistants. These all were duly paid their remunerations.

(1) One of the four priests employed at a Soma sacrifice as a superintendent.
(2) Any officiating priest technically distinguished from Hotri, Udg\=atri and Br\=ahma\d{n}. His duty was to measure the ground, build the altar, prepare sacrificial vessels, to fetch wood and water, light the fire, bring the animal and immolate it and while doing this to repeat the Yajurveda.
(3) A sacrificing priest who offers the oblations. Or one who recites the prayers of the Rigveda at a sacrifice.
(5) One of the four principal priests at a sacrifice, one who chants the hymns of the S\=amaveda.

23-24. The Hot\=a Govila, the excellent reciter of the S\=ama hymns, began to sing in accented tones called svarita (the accents are three Ud\=atta, Anud\=atta and Svarita) and the Rathantara S\=ama in 7 tunes.

Then he began to draw breath frequently; and consequently there was a break in time in the accent of Govila. Seeing this, Deva Datta was angry and immediately said to Govila.

25. Well, Govila, you are the foremost of the Munis and still you are doing your work like a quite illiterate man. I fear obstacles may arise in the getting of my son in this my sacrifice of Puttresti.

26. Govila then became much enraged and told Deva Datta ``your son will be illiterate, hypocrite, and dumb.''

27. Behold! Every being is subject to breathing and respiring; it is very hard to control them; there is no fault of mine in the accents of my songs being thus broken; it is strange that you, being intelligent, cannot understand this.

28. Being afraid to hear the curse from Govila, Deva Datta became very sorry and said ``O Muni! I have done no serious offence; why are you so offended without any cause. See! The Munis are void of anger and they always give delight to others.''

29-30. O best of Br\=ahmans! My offence is very trifling; why have you inflicted on me so severe a curse? I was already under the mental agony, since I had no issues; and now you have made me suffer move pain.

31. For the Vedic Pundits declare that it is better not to have any son than to have an illiterate stupid son; the more so, when a Br\=ahmi\d{n}'s son is illiterate, he is blamed by one and all.

32. An illiterate son is like a \'S\=udra or a beast; he is unfit for any action. O Br\=ahmin! What shall I do with an illiterate son?

33. An illiterate Br\=ahmi\d{n} is like a \'S\=udra; consequently not an object to be engaged in any act of worship or of gifts, he is not deserving to do any action.

34. A Br\=ahma\d{n}, bereft of the knowledge of the Vedas, living in a country is treated as a \'S\=udra by the king of the place and is liable to pay taxes.

35. Whoever wants to have any fruit in any action will never invite an illiterate Br\=ahmi\d{n} to take his seat in the ceremony relating to the Pitris or the Devas.

36. The king will consider an illiterate Br\=ahmin as if a \'S\=udra and will never engage him in any religious ceremony but will order him to do the work of a farmer in cultivating fields.

37. Rather to perform the funeral ceremonies by erecting a Ku\'sabata than to engage an illiterate Br\=ahmi\d{n} for the purpose.

38. One should give food to an illiterate Br\=ahmi\d{n} just sufficient to fill his belly and no more. If he does not do that, the giver and especially the receiver are subject to go down to hell.

39. Fie to a kingdom where honour is shown to the illiterate stupid Br\=ahma\d{n}as.

40. Where no difference is observed when seats, worship and gift are given to various persons, sages should draw their inference how the literate and illiterate persons are treated there.

41. When the illiterate fools become haughty, when they are paid honours and gifts, the literary persons should never dwell there.

42. The wealth of the wicked goes to the enjoyments of the bad persons; for the Nim trees, though abounding richly in fruits, are enjoyed only by crows.

43. Again, on the other hand, if the Br\=ahmi\d{n}s, versed in the Vedas, study the Vedas even after they have taken their food, still his father and forefathers are happy and play cheerfully in their heavens.

44. Therefore O Gov\={\i}la! You being the foremost of the Br\=ahmi\d{n} who are versed in the Vedas, what have you said just now? See in this world, death is rather to be preferred then to have an illiterate son. How is it, then, that you have cursed me that I would get an illiterate son, when you are the best one, highly qualified with knowledge.

45. O high minded one! You are capable to relieve the distressed; I am bowing down to your feet; shew your mercy and re-consider your curse.

46. Loma\'sa said :-- O Munis! Devadatta, saying these words, fell prostrate at his feet and began to eulogise him in very pitiful words, being very much grieved and with tears in his eyes.

47. Seeing him thus distressed, Govila was moved with pity. The persons that are noble have their anger satiated after a short while; the anger of the ignoble lasts for a long time.

48. The water is naturally cool; but it gets hot in contact with fire heat; and no sooner the heat is drawn away, water gets again cooled quickly.

49. The merciful Govila then addressed the distressed Devadatta ``your son though at first illiterate, will afterwards be very learned.''

50. The Br\=ahmi\d{n} Devadatta was very glad on getting this boon; then completing the sacrifice, rewarded the Br\=ahmi\d{n}s with their due dakshi\d{n}\=as and dismissed them.

51. In due course of time, his fair chaste wife Rohi\d{n}\={\i}, like the asterism Rohi\d{n}\={\i} became pregnant.

52. Devadatta performed the Garbh\=adh\=an (1) and Pumsavan (2) ceremonies and other purificatory rites duly.

53. He performed the S\={\i}mantonnayana ceremony according to rules and considered his Puttrvesti sacrifice successful and made various offerings to the Br\=ahmi\d{n}s.

N. B. -- (1) One of the Samsk\=aras, purificatory ceremonies, performed after menstruation to ensure or facilitate conception (this ceremony legalises in a religious sense the consummation of marriage).
(2) It is a ceremony performed on a woman's perceiving the first signs of a living conception, with a view to the birth of a son.
(3) ``Parting of the hair'' one of the twelve Samsk\=aras or purificatory rites observed by women in the fourth, sixth, or eighth month of their pregnancy.

54-55. In the auspicious lagna when Rohi\d{n}\={\i} asterism was present and in the auspicious day, his wife Rohi\d{n}\={\i} gave birth to a male child. Devadatta performed the nativities of the new born child and saw its face. Next that knower of the Pur\=a\d{n}as, Devadatta kept the name of the child as Utathya.

56. When the son was eight years old, Devadatta performed the Upanayana (thread) ceremony duly.

57-58. Next the child was made to accept the vow of Br\=ahmach\=ari; and Devadatta made him study the Vedas; but the child could not pronounce a single word and used to sit simply like a stupid boy. Though tried in various ways to read and write, that wicked boy never paid the slightest attention, simply sat idly. Seeing this, his father was very sorry and much grieved.

59. Thus twelve years passed. Yet the boy could not learn how to perform his Sandhy\=a Bandan\=a duly.

60. The rumour went abroad that Utathya, the son of Devadatta turned out very illiterate. All the Br\=ahma\d{n}as, ascetics, and other persons came to learn this fact.

61. Wherever Utathya used to go in any forest on hermitage, the people used to laugh at him, ridiculed his father and mother and began to chide that illiterate son.

62. Thus blamed by father, mother and all other persons, dispassion occupied the heart of Utathya.

63. Once when rebuked by his father and mother that it was better to have a blind and lame son instead of an illiterate brute, Utathya took recourse to renunciation and went to a dense forest.

64-65. On the banks of the Ganges in a beautiful spot free from obstacles, he built a beautiful hut and began to subsist on the roots and

fruits of the forest and with collected mind. Having made the excellent vow ``I will never speak untruth'' and holding the vow of celibacy, he lived in that beautiful hermitage.

Thus ends the 10th chapter in the 3rd Skandha of \'Sr\={\i} Mad Dev\={\i} Bh\=agavatam of 18,000 verses by Mahar\d{s}i Veda Vy\=asa relating to the story of Satyavrata.



