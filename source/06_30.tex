\chapter{On the glory of Mah\=a M\=ay\=a}

1-14. N\=arada said :-- O Best of Munis! The King was greatly astonished to see me dip in the tank in a female figure and get up from the tank in a male figure and thought, ``Where is my dearest wife? And how is this N\=arada Muni suddenly come here!'' The King, not seeing his wife, lamented very much and cried frequently, ``O my dear Wife! Where have you gone, leaving me here thus. Without you, O One of spacious hips! My life, palace and

kingdom, all, are quite useless. O Lotus-eyed one! What shall I do? O Smiling One! Why is not my life getting out of my body, suffering thus from thy separation? Without you, my sentiment of love has left me for ever. O Large-eyed One! Now I am lamenting for you. O Dear! Better give me your sweet reply; the love that you expressed at our first union, where has it gone now? O One with good eyebrows! Are you sunk in the water and have you given up your life? Or are you devoured by fishes or crocodiles? Or are you carried away by Varu\d{n}a, the Deva of the waters, to my great misfortune? O One of beautiful limbs! You are blessed, as you have gone away with your sons; O sweet-speaking One! Your affection for them was not artificial. Is it right for you to go up to the Heavens, attached by affection for your sons, leaving me your distressed husband alone, thus weeping for your separation? O Dear! I have lost both, you and my sons; yet death is not carrying me away; O! How hard is my lot! What to do? Where to go? R\=ama is not now in this world. He knew what was the pain caused by the separation from one's dearest wife. Oh! The cruel Fate has ordained very unwisely with great inconsistency the periods of parting from one another at different periods; when their minds and all other things are exactly the same in all circumstances of pleasure and pain. The practice of Sat\={\i} (burning with one's deceased husband), as ordained by the Munis, is certainly for the good of the chaste women; but it would have been good no doubt, were there such practices allowed for the men to burn themselves with their deceased wives.'' Bhagav\=an Hari then spoke to the lamenting King in reasonable words and consoled him thus :-- ``O King! Why are you thus troubling yourself with pain and sorrow? Where has gone your dearest wife? Have you not heard anything of \'S\=astras? or Have you not taken any shelter of any wise man!

15-27. Who was your wife? Who are you? Of what nature was your union and disunion and where did it take place? The union of wives and sons in this \'Sams\=ara is momentary like the meetings of persons on boats, while crossing a river. O King! Now go home, there is no use in your weeping thus in vain; the union and disunion of men are always under the control of Fate, the Daiva; therefore the wise should not lament for them. O King! Your union with the woman took place here; and now you have lost that beautiful, thin-bodied, large-eyed woman here also. Her father and mother you have not seen; you have got her like what is heard in the story of the crow and the T\=al fruit; as you got her wonderfully, so you have lost her wonderfully. O King! Do not grieve; Time cannot be ruled over; go home and

enjoy yourself subservient to Time. That beautiful woman has gone away in the manner she came to you; you ought to do your stately affairs in the way as you used to do before as the ruler of all. O King! Consider that if you weep day and night, that women will never return; why then are you giving vent to your sorrows in vain? Go now and have recourse to the path of the Yoga and thus while away your time. The enjoyable things come in course of time and they go away again in due course; therefore in this world of no gain whatsoever, the wise should never lament. Continuous pleasure or continuous pain does not always take place; pleasure and pain are never steady; they rotate always like a rotary instrument. Therefore, O King! Make your mind calm and quiet and rule happily your kingdom; or make over the charge of the kingdom to your sons and retire to the forest. This human body is seldom obtained; it is frail; therefore getting that body it is advisable to practise the realisation of the Supreme. O King! This organ of generation and this tongue reside also with the beasts, the peculiarity of human body is that knowledge can be realised in it; not in any other
inferior births. Therefore leave your home, leave your sorrows for your wife; all this is the M\=ay\=a of Bhagav\=an; by Her the world is deluded.''

28-37. N\=arada said :-- Bhagav\=an Hari speaking thus, the King bowed down to Him, the Deva of the Devas and finishing the bathing duties returned to his home. He then became possessed of dispassion and discrimination and making over the charge of his kingdom to his grandsons retired to the forest and realised the Supreme Knowledge. When the King went away, the Bhagav\=an began to laugh and laugh, seeing me again and again. I then told him, ``O Deva! You have deceived me. I now come to know how great is the power of M\=ay\=a. O Jan\=ardana! Now I remember all that I did in my feminine form. Tell me, O Hari! O Deva of the Devas! How I lost my previous consciousness, when I got down into the tank and bathed in it. O Lord of the world! Why was I enchanted, when I got the female form and when I got the King as my husband like \'Sach\={\i}'s getting Indra. The same mind I had; the old Jiv\=atm\=a was there and the previous subtle body was there; how, then, I lost their memories? O Lord! Give out the cause of it and clear my doubts; a great doubt has arisen in my mind. Many enjoyments I had in my female form, drinking liquor and other prohibited things I tasted; O Slayer of Madhu! What is the cause of all these? I could not know then that I was N\=arada, as I now recognise clearly what I was in and what I did in my female form. Say the Why of all these things.''

38-53. Vi\d{s}\d{n}u said :-- ``Know, O Intelligent N\=arada! That all this

is merely the Pastime of M\=ay\=a. There are many states going on in the bodies of all the living beings. The embodied beings have got their waking, dream, deep sleep and T\=uriya (beyond all the three above-mentioned) states; then why you doubt that when there is another body, there would be also the change in the states? When a man sleeps, he knows not anything, he does not hear anything; but when he gets awake, he again comes to know everything completely. The Chitta gets itself moved by sleep; then mind gets different states by dreams and there arises a variety of feelings. A mad elephant is coming to kill me, and I am not able to fly away. What to do? Where to go? There is no place where I can quickly go; thus, in dreams, there arise different mental states. Sometimes we see in dreams that our departed grandfathers are come in our houses. I am seeing them, talking with them and I am dining with them. Whatever pain and pleasure are felt in dreams, when they awake, they know of what happened in their dreams and can also describe in details, recollecting what had then happened. O N\=arada! Know the power of M\=ay\=a incomprehensible as the things seen in dreams cannot be certainly known that all those are false. O Muni! Neither I, nor \'Sambhu, nor Brahm\=a can measure the power wielded by M\=ay\=a and Her three Gu\d{n}as, very hard to fathom. How, then, can any ordinary mortal know them! Therefore, O N\=arada! None is able to fathom the M\=ay\=a. This world, moving and non-moving, is fashioned out of the triple Gu\d{n}as of the M\=ay\=a; nothing whatsoever can exist without them. The predominant Gu\d{n}a in Me is S\=attva; but R\=ajas and T\=amas exist in me; being the Lord of this world, I cannot override the three Gu\d{n}as. So your father, Brahm\=a, is predominant in R\=ajo Gu\d{n}a; but S\=attva and T\=amas never leave Him, Our Mah\=a Deva is predominant in T\=amo Gu\d{n}a, but S\=attva and R\=ajo are always with him. Therefore, no being can exist as separate from the three Gu\d{n}as; this point I have settled in \'Sruti. Therefore, O Lord of the Munis! Quit this endless Moha for the world, caused by M\=ay\=a, and very hard to get over and worship Bhagavat\={\i}, Who is of the nature of Br\=ahma\d{n}. O Intelligent One! Now you have seen the power of M\=ay\=a; and you have enjoyed many things produced by M\=ay\=a and you have realised the extremely wonderful nature of Her. Then why do you ask me further on this point?''

Here ends the Thirtieth Chapter of the Sixth Book on the glory of Mah\=a M\=ay\=a in the Mah\=apur\=a\d{n}am \'Sr\={\i} Mad Devi Bh\=agavatam of 18,000 verses by Mahar\d{s}i Veda Vy\=asa.



