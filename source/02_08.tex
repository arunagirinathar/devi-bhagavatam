\chapter{On the extinction of the family of Yadu and on the anecdote of Par\={\i}k\d{s}it}

1-23. S\=uta said :-- On the third day after the P\=andavas had returned to Hastin\=apur, the king Dh\d{r}tar\=a\d{s}\d{t}ra was burnt up together with G\=andhari and Kunti, by the conflagration of fire in the forest. Sanjaya went away at that time, leaving Dh\d{r}tar\=a\d{s}\d{t}ra in the forest, on a tour on pilgrimage. The king Yudhisthira heard all this from N\=arada and was very sorry. Now after thirty six years after the Kuru family had become extinct, all the descendants of Yadu in the Prabh\=as t\={\i}rtha were destroyed by the Br\=ahma\d{n}a's curse. The high-souled descendants of Yadu, intoxicated by drinking wine, fought against each other and were extirpated in the presence of Kri\d{s}\d{n}a and Balar\=am. Balar\=am then quitted his mortal coil; the lotus-eyed Bhagav\=an Kri\d{s}\d{n}a quitted his life, struck by the arrows of a hunter, to pay respect to a Br\=ahmi\d{n}'s curse. Vasudeva heard of Hari's quitting his mortal coil, and meditated the Goddess of the Universe within his heart and left his holy life. Arjuna became very sorry; he went to Prabh\=asa and performed the funeral obsequies of all duly. Seeing the dead body of Hari, Arjuna collected fuel and burnt his body together with his eight principal wives; he burnt also Balar\=am's body with that of his wife Revat\={\i}. Arjuna, then, went to the Dv\=arak\=a city and removed all the inhabitants of the city when the whole Dw\=ark\=a city of V\=asudeva was drowned in the waters of the ocean. While Arjuna was taking all the persons with him after getting out of Dw\=ark\=a, he felt himself very weak on the way; and therefore a band of robbers, known by the name of \=Abh\={\i}ras plundered all the wealth and all the wives of Kri\d{s}\d{n}a. Arjuna, of indomitable valour, after his arrival at Indraprastha made Vajra, Aniruddha's son, the king of the place.

Then the highly powerful Arjuna informed Vy\=asa of his powerlessness when Vy\=asa said :--``O highly intelligent one! When Hari and you will reincarnate in another Yuga, then your heroic strength will again be manifested. Hearing all these words, Prith\=a's son Arjuna returned to Hastin\=a with a sorrowful heart and informed everything to Yudhisthira, the Dharmar\=aja. Hearing the extinction of the Y\=adavas and Hari's quitting His mortal coil, Yudhisthira wanted to go to the Him\=alay\=as. He installed Par\={\i}ksit, Uttar\=a's son who was then thirty six years old on the throne and went out of his palace in company of his brothers, and Draupadi to the forests of the Him\=alay\=as. Thus the P\=andavas, Pr\={\i}tha's son, reigned for thirty six years in Hastin\=a and quitted their mortal coils in the Him\=alay\=as. Here the greatly religious sage-king Par\={\i}ksit governed with vigilance all his subjects for sixty years. After this, Par\={\i}ksit went once on an hunting expedition to a dense forest and shot a deer. He then searched for the deer and it became noon and he felt very thirsty, hungry, quite fatigued with his body, perspiring, when he saw a Muni merged in meditation; he asked the Muni ``Where can water be had?'' But the Muni held at that time the vow of silence; so he did not answer anything. Seeing this, the thirsty king, influenced by Kali, became angry and raised a dead serpent by the fore-end of his bow and coiled it round the Muni's neck. Even thus coiled with a snake round his neck the Muni remained as before motionless in his state of enlightenment and spoke nothing. The king also returned home.

24-49. Then the Muni's son, born from the cow's womb, \'Sring\={\i}, a great ascetic, a fiery devotee of Mah\=a\'sakti, heard of the above event, while he was playing in the forest. His friends spoke to him :-- ``O Muni! Some body has now enclosed a dead serpent around the neck of your father.'' Hearing their words, \'Sring\={\i} became very angry and taking water in his hands, cursed thus :-- ``He who has coiled to-day a dead serpent around my father's neck, let that villain be bitten by the serpent Tak\d{s}ak within one week from this day''. One disciple of the Muni then went to the king in his house and informed him of the Muni's curse. Abhimanyu's son Par\={\i}ksit heard of the curse pronounced by a Br\=ahmi\d{n}, and knowing infallible, spoke to the aged councillors :--

"O Ministers! Certainly it is through my fault that I have been cursed by a Br\=ahmi\d{n}'s son. Now find out and settle what is to be done though the persons versed in the Vedas say that death is inevitable under these circumstances; yet the wise ones should try their best to thwart this according to the \'S\=astras. Many sages who are the advocates of taking steps to redress any act, say that all the actions of wise persons are fructified by proper means; their solution does not remain unsolved.

Therefore I am saying that the powers of ma\d{n}is, mantrams and herbs (o\d{s}adhis) are indescribable; if applied duly, do you think that they will bear no fruit in this case? I heard that when a Muni's wife died out of snake-bite, the Muni gave away the half of his life to his wife Apsar\=a and made her alive again. It is not proper for the learned to depend on the maxim that what is inevitable must come to pass; one must try one's best to act for the living present. O Ministers! Have you seen any person in the Heavens or in the world who remains idle, depending on fate alone? The Sanny\=asins have renounced the world; but they must have to go to the houses of the house-holders, whether they be invited or not invited. See again. supposing that the food of a person is brought to him unasked and suppose it is thrown into the mouth by some one, can you conceive that food would go down into the belly, from the mouth without one's effort? Therefore one should exert one's own prowess from the very outset; though the intelligent ones should be satisfied with the thought ``What can be done? It is not ordained in my fate.'' When Par\={\i}ksit said thus, the ministers asked :-- ``Which Muni made his dead wife alive again, by giving her half his own life? And how did his wife die? Kindly describe all these in detail to me.'' The king said :-- Bhrigu Muni had a very beautiful wife Pulom\=a. In her womb the world renowned Chyavana Muni was born. Sukany\=a, the daughter of \'Sary\=ati was the wife of Chyavana. In her womb was born a beautiful son named Pramati; he was very famous. Pramati had his famous beautiful wife Prat\=ap\={\i}. In her womb was born the great ascetic son Ruru. At this time a person named Sth\=ulake\'sa, a religious truthful man of great name, was practising tapasy\=a. O Ministers! In the meanwhile, the chief Apsar\=a Menak\=a held sexual intercourse with Visv\=avasu Gandharva on the banks of a river and became pregnant. She went out from that place to the hermitage of Sth\=ulake\'sa on the river bank and gave birth to a very beautiful daughter. Seeing this girl quite an orphan and very beautiful, the Muni Sth\=ulake\'sa began to rear up her and named her Pramadvar\=a. This all-auspicious girl Pramadvar\=a attained youth in due course when the Muni Ruru saw her and became smitten with passion.

Thus ends the eighth chapter of the Second Skandha on the extinction of the family of Yadu and on the anecdote of Par\={\i}ksit in the Mah\=apur\=a\d{n}am \'Sr\={\i} Mad Dev\={\i} Bh\=ag\=avatam of 18,000 verses.



