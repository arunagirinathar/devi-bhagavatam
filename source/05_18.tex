\chapter{On the killing of the D\=anava Mahi\d{s}\=asura}

1-4. Mahi\d{s}a said :-- O Dev\={\i}! Mandodar\={\i} had a sister Indumat\={\i}; unmarried and endowed with all auspicious signs. She grew up in time to a marriageable age. The Svayambara assembly (a marriage in which the girl chooses her husband from among a number of visitors assembled together) was then called for the marriage of the maiden Indumat\={\i}. The Kings from various parts came there and the maiden Indumat\={\i} selected from among them a beautiful strong king, of noble lineage and endowed with all auspicious signs. At that time, by the undescribeable power of Destiny, Mandodar\={\i} seeing the deceitful, cunning, and hypocrite King of Madra, became passionate and desired to marry him.

5-17. That slender woman Mandodar\={\i} then addressed her father thus :-- ``O Father! Seeing the King of Madra in this assembly, I am desirous to marry him; so perform also my marriage ceremony now.'' When the king heard this request from her own daughter privately, he became very glad and began with promptness, to make preparations for the marriage. He invited the King of Madra to his own palace and gave him in marriage his own daughter Mandodar\={\i}, according to due rites and ceremonies with an abundance of dowry and wealth. The King of Madra Ch\=arude\d{s}\d{n}a became very glad to marry the beautiful Mandodar\={\i} and went back with her to his own abode. The King Ch\=arude\d{s}\d{n}a then enjoyed her for good many days; when one day a maid-servant found the king in sexual intercourse with another maid-servant in a lonely place and divulged this to Mandodar\={\i}; she finding the king in that state became angry and rebuked him with a slight smiling countenance. Again, on another occasion, Mandodar\={\i} saw the king willingly engaged in amusements and sports with an ordinary beautiful woman and became very sorry and thought thus :-- When I saw him in the Svayamvara, I could not recognise him as a cheat; I am deceived by this King; Oh! What a wrong act have I done through delusion. This King is a rogue and he is totally shameless and has no dislike for contemptible things; it is now too late to repent for him. How can I have any affection for this husband; fie on my living now! I forsake from this very day all the pleasures with my husband and all other worldly pleasures, and I take recourse now to contentment alone. I have committed a very wrong act that I ought never to have done; therefore it causes intense pain to me now. If I now commit suicide, then that sin will never forsake me, and I must have to enjoy the consequences thereof. And if I return to

my father's house, I will not be happy there, for my companions seeing me thus will, no doubt, ridicule me. Therefore, it is now advisable for me to avoid all the sensuous pleasures, become dispassionate and remain here patiently and abide by the strange combinations of Time.

18-20. Mahi\d{s}a said :-- Thus that women lamented and remorsed and began to remain there, very much sorrowful and distressed, renouncing thoroughly all the pleasures of the world. O Auspicious One! I am the king, yet you are showing your dislike for me; know, eventually, you, too, will be passionate and entertain afterwards an illiterate coward. Keep my word even now, it will be of great benefit and it will serve as a medicinal diet to you as to all women. In case you do not follow my advice, you will have to meet with extreme pain and misery, certainly.

21-25. Hearing the words of Mahi\d{s}\=asura, the Dev\={\i} said :-- O you fool! Go to the lower worlds or stand up for fight; I will send you and the other D\=anavas unto death and then go away at my pleasure. O Demon! I take up form to preserve the righteous, whenever they suffer pain in this earth. O Lord of the Daityas! Formless, birthless I am; yet, at times, I take up form and be born to save the Devas. Know this firmly. O wicked Mahi\d{s}a! The Devas prayed to Me for your destruction. Therefore I will not rest until I kill you. I speak all these truly to you. Therefore fight or go to P\=at\=ala, the abode of the Asuras; I speak truly to you again that I will destroy you wholly.

26-51. Vy\=asa said :-- O King! Hearing thus the Dev\={\i}'s words, the D\=anava took up his bow and came to the battle, fully stretching the string of his bow up to his ears, and began to shoot sharpened arrows with great force at the Dev\={\i}. The Dev\={\i}, too, hurled with anger, arrows tipped with iron and cut off the Asura's arrows to pieces. The fight between them rose to such a terrible pitch that it caused terror to both the Devas and the D\=anavas, trying hard to be victorious over each other. In the midst of the terrible encounter, the demon Durdhara came up to fight and made the Dev\={\i} angry and shot arrows, all terribly poisonous and sharpened on stones, at Her. The Bhagavat\={\i}, then, got very angry and hit him hard with sharp arrows. Durdhara, struck thus, fell down dead on the battlefield like a mountain top. The demon Trinetra, well skilled in the uses of arrows and weapons, seeing him killed, came up to fight and shot at the Great Goddess with seven arrows. Before these arrows came on Her, She cut them to pieces with Her sharp arrows and by Her trident killed Trinetra. Trinetra thus killed, Andhaka quickly came in the battlefield and struck violently on the head of the lion with his iron club. The lion killed that powerful Andhaka by striking the demon

with his nails and, out of anger, began to eat his flesh. Mahi\d{s}\=asura became greatly astonished at the death of these Asuras and began to shoot pointed arrows, sharpened on stone, at Her. The Dev\={\i} Ambik\=a cut his arrows into two before they came on Her and struck the Demon on his breast by Her club. That vile Mahi\d{s}\=asura, the tormentor of the Devas, fell in a swoon under the stoke of the club but patiently bore it and, at the next moment, came again and struck the lion on his head by his club. The lion, too, by his nails rent that great Asura to pieces. Mahi\d{s}\=asura, then, quitting the man-form took up the lion-form and by his claws cut the Dev\={\i}'s lion and wounded him very much by his nails. On Mahi\d{s}\=asura taking up this lion-form, the Dev\={\i} became very angry and began to shoot arrows after arrows at him all very terrible, sharp and like poisonous snakes. Then the Asura quitting the lion form assumed the appearance of a male elephant, oozing out juice from his temples and began to hurl the mountain tops by his trunk. Seeing the mountain peaks thus hurled on Her, She cut them off to pieces by Her sharp arrows and began to laugh. The Dev\={\i}'s lion on the other hand, sprang on the head of the elephant Mahi\d{s}a and by his claws rent him to pieces. To kill the Dev\={\i}'s lion, then, Mahi\d{s}a quitted his elephant-form and assumed the appearance of a Sarabha, more powerful and terrible than lion. The Dev\={\i} seeing that Sarabha became angry and struck on the head of that Sarabha with Her axe; the Sarabha, too, attacked the Dev\={\i}. Their fight became horrible; Mahi\d{s}\=asura, then, assumed the appearance of a buffalo and struck the Bhagavat\={\i} by his horns. That horrible Asura, of hideous appearance, swinging his tail, began to attack the thin bodied Dev\={\i}. That violent Asura caught hold of the mountain peaks by his tail and, whirling them round and round, hurled them on the Dev\={\i}. That vicious soul, then, maddened with his strength, laughed incessantly and addressed thus :-- ``O Dev\={\i}! Be steady in the battlefield. I will send you today unto death, and your youth and beauty too. You are an illiterate fellow as you have come maddened to fight with me. Really you are deluded in your pretensions that you are very strong; this idea of yours is absolutely false. I will kill you first and the hypocrite Devas after who want to vanquish me by standing up a woman in their front.''

52-53. The Dev\={\i} said :-- ``O Villain! Do not boast; keep yourself firm in the fight. Today I will kill you and make the Devas discard their fear. O Wretch! You are a Sinner; you torment the Devas and terrify the Munis. Let me have my drink of sweet decoction of grapes. And then I will slay you undoubtedly.''

54-70. Vy\=asa said :-- O King! Saying thus, the Dev\={\i}, wrathful and eager to kill Mahi\d{s}\=asura, took up the golden cup filled with wine and drank again and again. When the Dev\={\i} finished Her drink of the sweet grape juice, She pursued him with trident in Her hands, to the great joy of gladdening all the Devas. The Devas began to rain showers of flowers on the Dev\={\i} and praised Her and shouted victories to Her with Dundubhi (a Divine drum) Jai, J\={\i}va; victory, live. The \d{R}i\d{s}is, Siddhas, Gandarbhas, Pi\'s\=achas, Uragas, and Kinnaras witnessed the battle from the celestial space and became very much delighted. On the other hand, Mahi\d{s}\=asura, the hypocrite Pundit, began to assume various magic forms and struck the Dev\={\i} repeatedly. The Dev\={\i} Chandik\=a, then, infuriated and with eyes reddened, pierced violently the breast of that vicious Mahi\d{s}a with Her sharp trident. The Demon, then, struck by this trident, fell senseless on the ground; but got up in the next moment and kicked the Dev\={\i} forcibly. That Great Asura, thus kicking the Dev\={\i}, laughed repeatedly and bellowed so loudly that the Devas were all terrified with that noise. Then the Dev\={\i} held aloft the brilliant discus of good axle and of thousand spokes and loudly spoke to the Asura in front :-- O Stupid! Look! This Chakra will sever your throat today; wait a moment, I am sending you instantly unto death. Saying this, the Divine Mother hurled the Chakra. Instantly that weapon severed the D\=anava's head from his body. The hot streams of blood gushed out from his neck as the violent streams of water get out from mountains, coloured red with red sandstones. The headless body of that Asura moved, to and fro, for a moment and then dropped on the ground. The loud acclamations of ``Victory'' were sounded to the great joy of the Devas. The very powerful lion began to devour the soldiers that were flying away, as if he was very hungry. O King! The wicked Mahi\d{s}\=asura thus slain, the Demons that remained alive were terrified and fled away, very much frightened, to P\=at\=ala. The Devas, \d{R}i\d{s}is, human beings and the other saints on this earth were all extremely glad at the death of this wicked Demon. The Bhagavat\={\i} Chandik\=a quitted the battlefield and waited in a holy place. Then the Devas came there with a desire to praise and chant hymns to the Dev\={\i}, the Bestower of their happiness.

Here ends the Eighteenth Chapter of the Fifth Book on the killing of the D\=anava Mahi\d{s}\=asura in \'Sr\={\i} Mad Dev\={\i} Bh\=agavatam, the Mah\=a Pur\=a\d{n}am of 18,000 verses by Mahar\d{s}i Veda Vy\=asa.



