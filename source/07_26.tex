\chapter{On the narration of the sorrows of Hari\'schandra}

1-3. S\=uta said :-- The King Hari\'schandra with his face bent low thus said to the Queen :-- ``O Young One! I am a great sinner, otherwise why shall I be ready to do this heinous act! However now sit before me. If my hand be capable to kill you, then it will cut off your head.'' Thus saying, the King took the axe and moved forward to cut her. As the King did not recognise her as His Queen, so the Queen did not recognise him as Her husband, the King. So the Queen, being very much strained with sorrow, began to utter with a view to court her death.

4-16. O Ch\=and\=ala! If you like, I say something; hear my son is dead and is lying close to the outer skirts of the city. Wait till I bring my child before you and do his burning ceremonies. Next you can cut me off by your axe. The King said :-- ``Very well; let that be,'' and gave her permission to go to her dead son. Then the Queen, emaciated and pale, her body being covered all over with dust arrived at the burning ground and taking her dead son, bitten by a serpent, on her lap cried out loudly ``O Son! O my Child! O my young Son!'' and referring to her husband said :-- ``O King! See, today, the sad condition of your son, lying on the ground, as his bed. My son went to play with other boys and, bitten by a cruel poisonous serpent, left his life.'' Hearing the pitiful cry of that helpless woman, the King Hari\'schandra went to the dead and took off the cover of his face. Due to the long exile and the difficulties thereof, the Queen was changed altogether in her outer form, so the King could not recognise her weeping as his wife. On the other hand the King, too, had not the curled hair on his head as before; it has turned into matted hair and his skin especially has become like the bark of a dried tree; so the Queen could not make out the King also. The King then noticed all the King making auspicious signs on the several limbs of that dead boy, poisoned all over and lying on the ground and began to think thus :-- The face of the child is very beautiful like the Full Moon, nowhere there is any scar nor anything like this; the nose is high; the two cheeks are clean like a mirror and spacious; the hairs are blue, curling, similar, long and waving, the two eyes are widely expanded like a full blown lotus, the two lips are red like Bimba fruits; the chest is wide and spacious, the eyes are stretched up to the ears; the arms are extending up to the knees; the shoulders are

elevated; the legs are elongated, yet god-like like a lotus stem; the appearance is grave, the fingers are fine, yet strong enough to hold the world; the navel is deep and the region of the shoulders elevated. Certainly this boy was born in a royal family. Alas! What a pain is this! The cruel Death has reduced him to this state!

17-21. S\=uta said :-- Thus looking carefully that boy in the lap of his mother from his head to foot, the King Hari\'schandra got back to his ancient recollections. He recognised the boy to be his and wept aloud repeating the words Oh! Oh! The tears flowed from his eyes and he said :-- ``This is my boy that has been reduced to this state! Oh! The cruel Fate!'' Though the boy is dead, yet the King remained bewildered for a moment. The queen then spoke out of terrible pain :-- ``O Child! What sin is that which has caused this dire calamity, I cannot imagine!

22-27. O my Husband! O King! I am extremely worried of pains and troubles; leaving me thus, how is it and where you are passing away your time in a calm, quiet state! O Fortune! It is You that has brought about the loss of the R\=ajar\d{s}i Hari\'schandra's dominion, the separation from his friends and what more, you have caused his wife and son to be sold! Has he done so much mischief to you!'' Hearing her cries, the King's patience gave way and he came to recognise the Dev\={\i} and the son and exclaimed, ``She is my wife and the dead boy is my son. Oh! What a series of troubles, one coming after another.'' Being overpowered with extreme trouble and pain, the King fell unconscious on the ground; the Queen, too, looking at the King's state, fell motionless, and, void of senses, no sooner she recognised him as the King Hari\'schandra. Some time after, the King and Queen both got back at the same time their consciousness and, with great sorrow and agony, began to lament.

28-49. The King said :--``O Child! Why my heart does not rend to thousand pieces, seeing today your gentle face pale and lifeless, that was once beautiful with curls of hairs! O Rohit\=a! When will you come to me saying in a sweet voice, `Father! Father!' When shall I address you affectionately, `Oh my child! Oh my child!' embracing you within my breast! Whose tawny coloured dust on his knees will spoil my clothes, lap and my body! O Delightful Son! I have sold you as if an ordinary thing, though I am your father. As yet my pleasure of having a son is not satisfied. Owing to the mockery of the mean Fate, my unbounded kingdom, friends, and abundance of riches all have vanished away! Finally I had one son and that too is now in the jaws of death! Oh! With what an amount of terrible pain I am being burnt up today when I am seeing the lotus-face of my son, smitten by a serpent and

lying dead on the ground!'' Thus speaking in a voice choked with feelings and with tears in his eyes, as soon as he was going to take his boy in his lap, he fell senseless on the ground. Seeing the King lying on the ground, \'Saivy\=a thus thought :-- ``Such is His voice as makes me certain that He is the King Hari\'schandra, the best of men and the delighter of the learned men's hearts. His teeth are like those of the famous Hari\'schandra just like to Mukul and his nose is elevated and soft like the Tila flower. But if he be Hari\'schandra, how is it that he has come to this burning ground!'' Thus thinking, while she looked at the King, leaving for the moment the sorrow for his son, joy, pain and surprise attacked her heart simultaneously; and she, in that state, fell down unconscious on the ground. Then gradually regaining consciousness, she spoke in a pitiful voice :-- ``O Fortune! You have caused to the King who was once like an Immortal, the loss of his kingdom, friends, and even the sale of his wife and son. And now you have transformed him into a Ch\=and\=ala! You are merciless, religionless, void of any justice as to what is just and what is unjust. You are shameless. So fie on you! O King! Where are gone today that royal umbrella, that throne, that Ch\=amara, and that pair of fans on your both sides! Oh! What is this transformation caused by the Vidh\=at\=a (the Ordainer of Fate)! When the high-souled King used to travel, all the kings used to remove as His servants the dust of the roads by their clothings! Oh! Is He the same King of Kings, Hari\'schandra who is roaming in this unholy burning ground, burdened too much by his load of sufferings! Oh! Innumerable human skulls are lying here; the small earthen pots (brought for the purification of the bodies of the dead) are lying scattered close to each; the garlands of flowers for the dead, being intertwined with the hairs of the dead, are presenting a grim spectacle! The ashes, charcoals, half-burnt dead bodies, bones, and marrows all arranged one over another make the place more hideous. The marrows of the dead bodies have come out and are dried up by the sun. At places, vultures, and \'Sakun\={\i}s are crying hideously and the crows and other birds, eager to eat flesh, are roaming to and fro. All the quarters of the sky are looking blue with the smoke, arising out of the burning of the dead. The R\=ak\d{s}asas are constantly roaming hither and thither, gladly feasting on the human flesh. Is the King passing his days thus in this place? Alas! Oh! What a painful thing is this!'' The daughter of the King, \'Saivy\=a, was overpowered with an awful sorrow; and clasping the neck of the King, began to lament again, in a pitiful voice. O King! You have spoken that you are a Ch\=and\=ala. Is this a dream? Or a Reality? O King! If it be true that you are a slave of the Ch\=and\=ala, then say to me; my mind is being deluded very much! (i.e., I cannot

indulge this idea). O Knower of Dharma! You have shown your great zeal towards Dharma; and, for that reason, you are displaced from your royal throne! Now if such help comes out of worshipping the Br\=ahmi\d{n}s and the Devas, then Dharma cannot stand and, along with it, the truth, simplicity and harmlessness cannot exist.

50-55. S\=uta said :-- Hearing these words from the thin \'Saivy\=a, the King took a heavy sigh and then described to her in detail with tears flowing on his neck, how he got the Ch\=and\=ala state. The fearful Queen became very much pained to hear all this and heaving a deep sigh, described, as it was, how her son died. On hearing this, the King fainted and fell unconscious on the ground. Then regaining gradually his consciousness, he began to kiss, with his tongue, the face of his dead son. \'Saivy\=a then said in a choked voice :-- ``Now sever off my head and obey your master's word. O King! You will be saved then as having kept your truth; and your master's order would be carried out.'' Hearing this, the King fainted and fell down senseless. Getting up conscious in a moment, he began to weep bitterly.

56. The King said :-- ``O Beloved! How have you uttered such cruel words? How can I execute that which is hard even to utter!''

57-58. \'Saivy\=a said :-- ``O Lord! I have worshipped the Dev\={\i} Gaur\={\i} and other Devas and the Br\=ahmi\d{n}s; so, with their mercy, I will get you as my husband in my future birth.'' Hearing this, the King again fell down instantly on the ground; getting up immediately, he was overpowered with sorrow and began to kiss the face of the dead son.

59-71. The King said :-- ``O Dear! I won't be able to suffer an longer for a long time. But, O thin-bodied One! See, I am so very unfortunate that I have no command even over my heart. If I enter into the fire without the permission of the Ch\=and\=ala, then I will have to become again the slave of a Ch\=and\=ala in my future birth. Think it over. After that I will have to go to the hell and be tormented there. But this too I find beneficial to me. Rather I will go to the hell Mah\=a Raurava and there suffer for a long time the torments of the hell, yet I do not like to live a little longer when my boy, the continuer of my family, has left his life out of the queer fancies of the Great Time and I be merged in the sorrows for my son. My body is now at the command of the Ch\=and\=ala. How can I in this state quit my life without his permission. If I leave my body, I will be indebted to him and I will have to suffer in hell. Let this be so; still I will leave off my body, the receptacle of all these pains and troubles. Nowhere, in the Triloki, is any pain like

that felt in the demise of a son, not in crossing the Vaitara\d{n}\={\i} nor in the Asipatravanam! So I will now throw myself on the burning fire along with the dead body of my son. So, O Thin-bodied One! You should now excuse me (i.e., do not prevent me). O Sweet-smiling One! I now permit you to go back to the house of the Br\=ahmi\d{n}. If ever I have given in charity riches, offered oblation to the fire, and given satisfaction to my superiors then, in the other world, I will get you and my son. But there is no such chance now in this world. O Sweet-smiling One! If ever I had given you offence while conversing or making jokes with you, now at the time of my parting, excuse them all. O Auspicious One! Never despise the Br\=ahmi\d{n} out of your pride as a Queen. Look on your master as a Deva and try all your best to satisfy him.''

72-73. The Queen said :-- ``O R\=ajar\d{s}i! I will also throw myself on the burning fire. O Deva! I will not be able to carry on this burden, so I will accompany You. It is better for me to accompany you; so there will not be otherwise. O Giver of Honour! I will enjoy with You heaven or suffer with You in the hell.'' Hearing this, the King said :-- ``O Chaste One! Do as you please.''

Here ends the Twenty-sixth Chapter of the Seventh Book on the narration of the sorrows of Hari\'schandra in the Mah\=apur\=a\d{n}am \'Sr\={\i} Mad Dev\={\i} Bh\=agavatam, of 18,000 verses, by Mahar\d{s}i Veda Vy\=asa.



