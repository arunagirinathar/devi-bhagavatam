\chapter{On the story of Vi\'sv\=amitra}

1-3. Vy\=asa said :-- O King! Hearing thus the words of the Mahar\d{s}i Bh\=aradv\=aja and seeing that he made a firm resolve, the King Yudh\=ajit called his prime minister quickly and asked, ``O intelligent one! What am I to do now? I want to carry away by force this boy with his mother sweet Manoram\=a; no one desirous of one's welfare won't trifle away his enemy, be he even a very weak one; if he does so, that enemy

will get stronger day by day, as the disease consumption becomes stronger; and will ultimately become the cause of death.

4. There is no warrior, nor any soldiers here of the other party; no one will be able to resist me; I can take away, as I like, the enemy of my daughter's son and can kill him.

5. I will try to-day to carry him away by force, and Sudar\'sana being killed, my daughter's son will reign fearlessly without an enemy; there in no doubt in this.''

6. The prime minister said :-- No such hazardous courage need be shewn now; you have heard the Maharshi's words; he quoted you the example of Vi\'sv\=amitra.

7. O King! In days of yore, Vi\'sv\=amitra, the son of the King G\=adhi, was a celebrated monarch; one day while roaming, he accidentally reached the hermitage of Va\'sistha.

8. The powerful king Vi\'sv\=amitra bowed down before the Muni, and the Muni gave him a seat. The king took his seat there.

9. Then the high souled Va\'sistha invited the king to a dinner. Vi\'sv\=amitra, the king, went there with his whole army.

10-12. There was a cow, named Nandin\={\i}, of Va\'sistha. The Muni prepared all sorts of eatables from her milk and entertained them all. The king with his whole army was very much pleased; and, coming to know of the divine power of the cow, asked Va\'sistha to give his cow Nandin\={\i} over to him and said ``The udder of your cow Nandin\={\i} is like a big jar. I will give you thousand cows like that; I pray you to let me have your cow Nandin\={\i}.''

13. Va\'sistha said ``O King! This is my sacrificial cow; I cannot give you this cow in any way, let your thousand cows be yours.''

14. Vi\'sv\=amitra said :-- ``O Saint! I will give you cows lakhs or tens and hundreds of lakhs or any number you like. Please give me your cow; in case you be unwilling, I will carry her away perforce.''

15. Va\'sistha said :-- ``O King! As you like, better take it perforce; I will never be able to give you my cow Nandin\={\i} from my house.''

16. O King! Hearing thus the Va\'sistha's words, Vi\'sv\=amitra, the King, ordered at once his powerful followers to carry the cow Nandin\={\i} away by fastening a cord round her neck per sheer force.

17-19. The followers, obeying the order at once bound the cow with ropes and began to carry her away by force. At this Nandin\={\i}, trembling and with tears in her eyes, began to say to the Muni ``O One! whose wealth consists only in asceticism! Are you going to leave me? Otherwise

why these fellows are binding me with a cord and dragging me away?'' At this the Muni replied ``O Nandin\={\i}! I have never parted with you; I perform all my sacrifices through your milk. O auspicious one! I honoured this king, my guests, with eatables prepared from your food and for that reason he is carrying you away from me by sheer force. What can I do? O Nandin\={\i}! I have not the least desire to part with you.''

20. Hearing these words from the Muni, the cow became very angry and bellowed loudly and terribly.

21. At once came out from her body, on that very spot, the terrible demons wearing coats of armour, and holding various weapons; and they uttered aloud, ``Wait; you will soon meet with vengeance.''

22. They then destroyed all the forces of the king. And the king alone was left and he went away alone, much dejected and sorrowful.

23. Oh! That wicked king then cursed with great humility the K\d{s}attriya \'Sakti; and thinking the Br\=ahmanic power would be attained with great exertion, began to practise asceticism and penance.

24. Performing penance and tapasy\=a, very hard indeed, in the great forest, Vi\'sv\=amitra, the son of G\=adhi, succeeded at last in becoming a \d{R}i\d{s}i and then he renounced his K\d{s}attriya Dharma.

25. Therefore, O King! Dost Thou never quarrel with these ascetics and be involved in wars resulting in great enmity and causing the extinction of the race.

26. Better dost thou appease the Muni and now go back to your own kingdom. Let Sudar\'sana remain here at his pleasure.

27. O King! This minor boy has no wealth; what harm can he do to you? It is useless to show your enmity towards an orphan, a weak minor boy.

28. This world is under the control of Destiny; therefore one should shew mercy to all. O king! What use is there to shew one's jealousy? What is inevitable will surely come to pass.

29. O king! The thunderbolt comes sometimes like a blade of grass; a blade of grass acts sometimes like a thunderbolt.

30. O king! You are very intelligent; consider that by, combinations of circumstances, a hair can kill a powerful tiger and a gnat can kill an elephant. Therefore dost thou forsake this rashness and hear my beneficent advice.

31. Vy\=asa said :-- O king! The best of kings, Yudh\=ajit hearing the prime minister's advice bowed down humbly at the feet of the Muni and returned to his own city.

32. Manoram\=a, too, became free from anxiety, and, remaining peaceful in the hermitage, began to nourish and support her child, engaged in vows.

33. The lovely son of the king began to grow daily like the phases of the waxing moon and sport fearlessly with the boys of the Munis, altogether, wherever they liked, a sight very auspicious.

34. One day the minister Vidalla came there and the sons of the Munis seeing him began, in the presence of Sudar\'sana, to address him ``Klib,'' ``Klib.''

35. Sudar\'sana, too, hearing them pronounce ``Klib,'' ``Klib'' took up the one letter, ``Kli'' and uttered this only repeatedly, which is, in fact the prince of the root mantras of K\=ama, with anusv\=ara omitted.

36. Then the son of the king took that mantram and silently repeated this in his mind.

37. O King! Thus that boy Sudar\'sana was initiated in this root mantra of K\=ama (desire) spontaneously, out of his original Samsk\=ara (innate tendency) owing to the unavoidable destiny of Fate.

38-39. The son of the king, when he was five years old, got this most excellent mantra, though without its \d{R}i\d{s}i (seer), meditation, without its chhanda (metre) and without Ny\=asa (assignment of the various parts of the body to different deities, accompanied with prayers and corresponding gesticulations), and considered this as the quintessence of all, therefore meditated this always in his mind spontaneously and never forgot it.

40-41. When the king's son grew eleven years old, the Muni performed his Upanayana (sacred thread) ceremony and made him begin the study of the Vedas. The son, with that mantra power, soon mastered all the studies about archery, all the moral and political sciences in conformity with proper rules, within a very short time.

42-43. One day Sudar\'sana got a vision of the form of the Supreme Goddess, of a red colour, wearing red apparel and decorated with red ornaments, mounting on Garuda and with Her wonderful Vai\d{s}\d{n}av\={\i} powers and Her face, fully opened like the budding of a lotus flower.

44. Thus, expert in many branches of learning, Sudar\'sana served his Mother in that forest and began to wander on the banks of the Ganges.

45. One day the Mother of the Universe gave the bows, sharpened arrows, quiver and a mail coat of armour to that boy in that forest.

46-47. O King! At this time the extraordinary beautiful and lovely princess \'Sa\'sikal\=a, endowed with all auspicious qualities, the daughter of the king of K\=ashi, came to hear that a beautiful prince named Sudar\'sana, a second Kandarpa, full of heroism and endowed with all auspicious qualities is dwelling in a forest.

48. The princess, hearing this from a soothsayer, mentally loved and desired him and wanted finally to accept him as her legal husband.

49-50. Thus, on one occasion, at the end of a night (night-fall), the Goddess appeared in her dreams before her and consoled her and said ``O fair one! ask a boon from me; Sudar\'sana is my devotee; he will fulfill, at my word, all your desires.''

51. Thus seeing the beautiful figure of the Goddess in her dreams and hearing Her sweet words, the honoured \'Sa\'sikal\=a was drowned in the ocean of bliss.

52. When the princess awoke, her face beaming with gladness, her mother perceived her joy and inferred that her daughter must have been internally very glad, and asked her repeatedly, but \'Sa\'sikal\=a was too much abashed and did not give vent to the cause of her satisfaction.

53. The princess, remembering her dreams, began to laugh repeatedly on account of her excessive joy. At last she spoke out in detail all about her dreams to one of her lady friends, or companions.

54. On one occasion, that large eyed \'Sa\'sikal\=a went out for enjoyment to a nice garden beautified with champaka flowers, attended by her companion.

55. While the King's daughter seated under a champaka tree, was collecting flowers, she saw a Br\=ahmi\d{n}, coming towards her in great haste.

56. After bowing down before him, that beautiful princess, endowed with all auspicious qualifications, addressed him in sweet words ``O blessed one! whence are you coming?''

57. The Br\=ahma\d{n}a said :-- ``O girl! I am coming on an errand from the hermitage of Bh\=aradv\=aja Muni. Please mention what you are going to ask me?''

58. \'Sa\'sikal\=a replied ``O Noble one! What beautiful thing is therein that hermitage that is extraordinary and worth describing.''

59. The Br\=ahma\d{n}a said ``O fair one! There is staying the most lovely Sudar\'sana, the son of the King Dhruvasandhi. He is the loveliest of all men.

60. O fair one! He who has not seen him, I think, has his eyes given to him in vain.

61. O auspicious one! It appears as if the Creator, with a view to see how it looks, has invested him with all the qualities.

62. O beautiful one! what shall I say more to you, suffice to say that, that prince is fit to become your husband. I think that the Creator has, no doubt, settled already the union between you two, as a happy union of two congenial things (gold in union with Jewel).''

Thus ends the seventeenth chapter on the story of Vi\'sv\=amitra and on the getting of the root mantra of K\=ama by the son of the King in \'Sr\={\i} Mad Dev\={\i} Bh\=agavatam, of 18,000 verses by Mahar\d{s}i Veda Vy\=asa.



