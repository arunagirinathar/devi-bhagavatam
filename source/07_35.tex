\chapter{1. Him\=alay\=a said :-- "O Mahe\'svar\={\i}! Now tell me the Yoga with all its Amgas (limbs) giving the knowledge of the Supreme Consciousness so that, I may realise my Self, when I practise according to those instructions.}
2-10. \'Sr\={\i} Dev\={\i} said :-- "The Yoga does not exist in the Heavens; nor does it exist on earth or in the nether regions (P\=at\=ala). Those who are skilled in the Yogas say that the realisation of the identity between the Jiv\=atma and the Param\=atm\=a is "Yoga." O Sinless One! The enemies to this Yoga are six; and they are lust, anger, greed, ignorance, vanity and jealousy. The Yogis attain the Yoga when they become able to destroy these six enemies by practising the accompaniments to Yoga. Yama, Niyama, \=Asana, Pr\=a\d{n}\=ay\=ama, Praty\=ah\=ara, Dh\=ara\d{n}\=a, Dhy\=ana, and Sam\=adhi, these are the eight limbs of Yoga. Yama includes Ahims\=a (non-injuring; non- killing); truthfulness; Asteyam (non-stealing by mind or deed); Brahmacharya (continence); Day\=a (mercy to all beings); Uprightness; forgiveness, steadiness; eating frugally, restrictedly and cleanliness (external and internal). These are ten in number. Niyama includes also ten qualities :-- (1) Tapasy\=a (austerities and penances); (2) contentment; (3) \=Astikya (faith in the God and the Vedas, Devas, Dharma and Adharma); (4) Charity (in good causes); worship of God; hearing the Siddh\=antas (established sayings) of the Vedas; Hr\={\i} or modesty (not to do any irreligious or blameable acts); \'Sraddh\=a (faith to go do good works that are sanctioned); (9) Japam (uttering silently the mantrams, G\=ayatr\={\i}s or sayings of Pur\=anas) and (10) Homam (offering oblations daily to the Sacred Fire). There are five kinds of Asanas (Postures) that are commendable: Padm\=asan, Svastik\=asan, Bhadr\=asan, Vajr\=asan and V\={\i}r\=asan. Padm\=asan consists in crossing the legs and placing the feet on the opposite thighs (the right foot on the left thigh and the left foot on the right thigh) and catching by the right hand brought round the back, the toes of the right foot and catching by the left hand brought round the back the toes of the left foot; sitting then straight and with ease. This is recommended by the Yogis (and by this one can raise oneself in the air).
N. B. -- The hands, according to some, need not be carried round the back; both the hands are crossed and placed similarly on the thighs.
11-20. Place the soles of the feet completely under the thighs, keep

the body straight, and sit at ease. This is called the Svastik\=asan. Bhadr\=asan consists in placing well the two heels on the two sides of the two nerves of the testicle, near the anus and catching by the two hands the two heels at the lower part of the testicles and then sitting at ease. This is very much liked by the Yogis. Vajr\=asan (diamond seat) consists in placing the feet on the two thighs respectively and placing the fingers below the thighs with the hands also there, and then sitting at ease. V\={\i}rasan consists in sitting cross on the hams in placing the right foot under the right thigh and the left foot under the left thigh and sitting at ease with body straight.
Taking in the breath by the Id\=a (the left nostril) so long as we count "Om" sixteen, retaining it in the Su\d{s}umn\=a so long as we count "Om" sixty-four times and then exhaling it slowly by the Pingal\=a n\=adi (the right nostril) as long as we count "Om" thirty-two times. (The first process is called P\=uraka, the second is called Kumbhaka, and the third is called Rechaka). This is called one Pr\=a\d{n}\=ay\=ama by those versed in the Yogas. Thus one should go on again and again with his Pr\=a\d{n}\=ay\=ama. At the very beginning, try with the number twelve, i. e., as we count "Om" twelve times and then increase the number gradually to sixteen and so on. Pr\=a\d{n}\=ay\=ama is of two kinds :-- Sagarbha and Vigarbha. It is called Sagarbha when Pr\=a\d{n}\=ay\=ama is performed with repeating the Ista Mantra and Japam and meditation. It is called Vigarbha Pr\=a\d{n}\=ay\=ama when "Om" is simply counted and no other Mantram. When this Pr\=a\d{n}\=ay\=ama is practised repeatedly, perspiration comes first when it is called of the lowest order; when the body begins to tremble, it is called middling; and when one rises up in the air, leaving the ground, it is called the best Pr\=a\d{n}\=ay\=ama. (Therefore one who practises Pr\=a\d{n}\=ay\=ama ought to continue it till he becomes able to rise in the air).
21-30. Now comes Praty\=ah\=ara. The senses travel spontaneously towards their objects, as if they are without anyone to check. To curb them perforce and to make them turn backwards from those objects is called "Praty\=ah\=ara," To hold the Pr\=ana V\=ayu on toes, heels, knees, thighs, sacrum genital organs, navel, heart, neck, throat, the soft palate, nose, between the eyebrows, and on the top of the head, at these twelve places respectively is called the "Dh\=ara\d{n}\=a." Concentrate the mind on the consciousness inside and then meditate the Ista Devat\=a within the J\={\i}v\=atm\=a. This is the Dhy\=ana. Sam\=adhi is identifying always the J\={\i}v\=atm\=a and Param\=atm\=a. Thus the sages say. (Sam\=adhi is of two kinds (1) Sampraj\~n\=ata, or Savikalpak and (2) Nirvikalpak. When the ideas

the Knower, Knowledge and the Thing Known, rernain separate in the consciousness and yet the mind feels the one Akhanda Sachchid\=ananda Brahma and his heart remains, there, that is called Sampraj\~n\=ata Sam\=adhi; and when those three vanish away and the one Brahma remains, it is called Asampraj\~n\=ata Sam\=adhi). Thus I have described to you the Yoga with its eight limbs. O Mountain! This body composed of the five elements, and with J\={\i}va endowed with the essence of the Sun, the Moon, and the Fire and Brahma in it as one and the same, is denominated by the term "Vi\'sva." There are the 350,000 n\=adis in this body of man; of these, the principal are ten. Out of the ten again, the three are most prominent. The foremost and first of these three is Susumn\=a, of the nature of the Moon, Sun, and Fire, situated in the centre of the spinal cord (it extends from the sacral plexus below to the Brahmaradhra in the head at the top where it looks like a blown Dhust\=ura flower). On the left of this Su\d{s}umn\=a is the Id\=a N\=ad\={\i}, white and looking like Moon; this N\=ad\={\i} is of the nature of Force, nectar-like. On the right side of the Su\d{s}umn\=a is the Pingal\=a N\=ad\={\i} of the nature of a male; it represents the Sun. The Su\d{s}umn\=a comprises the nature of the all the Tejas (fires) and it represents Fire.
31-41. The inmost of Su\d{s}umn\=a is Vichtr\=a or Chitri\d{n}\={\i} Bh\=ulingam n\=ad\={\i} (of the form of a cobweb) in the middle of which resides the Ichch\=a (will), J\~n\=ana (knowledge) and Kriy\=a (action) \'Sakt\={\i}s, and resplendent like the Millions of Suns. Above Him is situated Hr\={\i}m, the M\=ay\=a V\={\i}ja Har\=atm\=a with "Ha" and Chandravindu repesenting the Sound (N\=ada). Above this is the Flame, Kula Ku\d{n}dalin\={\i} (the Serpent Fire) of a red colour, and as it were, intoxicated. Outside Her is the \=Adh\=ara Lotus of a yellow colour having a dimension of four digits and Comprising the four letters "va", "\'sa", "\d{s}a", and "sa". The Yogis meditate on this. In its centre is the hexagonal space (P\={\i}tham). This is called the M\=ul\=adh\=ara for it is the base and it supports all the six lotuses. Above it is the Sv\=adhisth\=ana Chakra, fiery and emitting lustre like diamond and with six petals representing the six letters "ba", "bha", "ma", "ya", "ra", "la". The word "Sva" means "Param Lingam" (superior Male Symbol). Therefore the sages call this "Sv\=adhisth\=an Chakram. Above it is situated the "Ma\d{n}ipura Chakram" of the colour of lightning in clouds and very fiery; it comprises the ten Petals, comprising the 10 letters da, dha, \d{n}a, ta, tha, da, dha, \d{n}a, pa, pha. The lotus resembles a full blown pearl; hence it is "Ma\d{n}ipadma." Vi\d{s}\d{n}u dwells here. Meditation here leads to the sight of Vi\d{s}\d{n}u, Above it is "An\=ahata" Padma with the twelve petals representing, the twelve letters Ka, Kha, Gha, ma, (cha), (chha), (Ja), (Jha,) \=Iya, ta, and tha. In the middle is B\=analingam, resplendent like

the Sun. This lotus emits the sound \'Sabda Brabma, without being struck; therefore it is called the An\=ahata Lotus. This is the source of joy. Here dwalls Rudra, the Highest Person."
42-43. Above it is situated the Vi\'suddha Chakra of the sixteen petals, comprising the sixteen letters a, \=a, i, \={\i}, u, \=u, ri, ri, li, lri, e, ai, o, ar, am, ah. This is of a smoky colour, highly lustrous, and is situated in the throat. The J\={\i}v\=atm\=a sees the Param\=atm\=a (the Highest Self) here and it is purified; hence it is called Vi\'suddha. This wonderful lotus is termed \=Ak\=a\'sa.
44-45. Above that is situated betwixt the eyebrows the exceedingly beautiful Aj\~n\=a Chakra with two petals comprising the two letters "Ha," and K\d{s}a. The Self resides in this lotus. When persons are stationed here, they can see everything and know of the present, past and future. There one gets the commands from the Highest Deity (e. g. now this is for you to do and so on); therefore it is called the Aj\~n\=a Chakra.
46-47. Above that is the Kail\=a\'sa Chakra; over it is the Rodhi\d{n}\={\i} Chikra. O One of good vows! Thus I have described to you all about the \=Adh\=ara Chakras. The prominent Yogis say that above that again, is the Vindu Sth\=an, the seat of the Supreme Deity with thousand petals. O Best of Mountains! Thus I declare the best of the paths leading to Yoga.
48. Now hear what is the next thing to do. First by the "P\=uraka", Pr\=a\d{n}\=ay\=ama, fix the mind on the Mul\=adh\=ara Lotus. Then contract and arouse the Kula Kundalin\={\i} \'Sakt\={\i} there, between the anus and the genital organs, by that V\=ayu.
49. Pierce, then, the Lingams (the lustrous Svayambhu \=Adi Lingam) in the several Chakras above-mentioned and transfer along with it the heart united with the \'Sakti to the Sahasr\=ara (the Thousand petalled Lotus). Then meditate the \'Sakt\={\i} united with \'Sambhu there.
50-51. There is produced in the Vindu Chakra, out of the intercourse of \'Siva and \'Sakt\={\i}, a kind of nectar-juice, resembling a sort of red-dye (lac). With that Nectar of Joy, the wise Yogis make the M\=ay\=a \'Sakt\={\i}, yielding successes in Yoga, drink; then pleasing all the Devas in the six Chakras with the offerings of that Nectar, the Yogi brings the \'Sakt\={\i} down again on the M\=ul\=adh\=ara Lotus.

52. Thus by daily practising this, all the above mantras will no doubt, be made to come to complete success.
53-54. And one will be free from this Sams\=ara, filled with old age and death, etc. O Lord of Mountains! I am the World Mother; My devotee will get all My qualities; there is no doubt in this. O Child! I have thus described to you the excellent Yoga, holding the V\=ayu (Pavana Dh\=ara\d{n}a Yoga).
55. Now hear from Me the Dh\=ar\=a\d{n}\=a Yoga. To fix thoroughly one's heart on the Supremely Lustrous Force of Mine, pervading all the quarters, countries, and all time leads soon to the union of the J\={\i}va and the Brahma.
56-58. If one does not quickly do this, owing to impurities of heart, then the Yogi ought to adopt what is called the "Avayava Yoga." O Chief of Mountains! The S\=adhaka should fix his heart on my gentle hands, feet and other limbs one by one and try to conquer each of these places. Thereby his heart would be purified. Then he should fix that purified heart on My Whole Body.
59-62. The practiser must practise with Japam and Homam the Mantram till his mind be not dissolved in Me, My Consciousness. By the practice of meditating on the Mantra, the thing to be known (Brahma) is transformed into knowledge. Know this as certain, that the Mantra is futile without Yoga and the Yoga is futile without the Mantra. The Mantra and the Yoga are the two infallible means to realise Brahma. As the jar in a dark room is visible by a lamp, so this J\={\i}v\=atm\=a, surrounded by M\=ay\=a is visible by means of Mantra to the Param\=atm\=a (the Highest Self). O Best of Mountains! Thus I have described to you the Yogas with their A\d{n}gas (limbs). You should receive instructions about them from the mouth of a Guru; else millions of \'S\=astras will never be able to give you a true realisation of the meanings of the yogas.
Here ends the Thirty-fifth Chapter of the Seventh Book on the Yoga and the Mantra Siddhi in the Mah\=a Pur\=a\d{n}am \'Sri Mad Dev\={\i} Bh\=agavatam of 18,000 verses, by Mahar\d{s}i Veda Vy\=asa.



