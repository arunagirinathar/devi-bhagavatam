\chapter{On the Dhruva Mandalam}

1-29. N\=ar\=aya\d{n}a said :-- Beyond the Saptar\d{s}i mandalam (the Great Bear), thirteen lakh Yojanas higher is situated, the Vi\d{s}\d{n}u's Paramam Padam (the highest place of Vi\d{s}\d{n}u). The Great Bh\=agavat (devotee of God), the most respectful, \'Sr\={\i}m\=an Dhruva, the son of Utt\=anap\=ada, is established there with Indra, Ag\d{n}i, Ka\'syapa and Dharma and the Nak\d{s}attras. The visitors pay to him always their respects. He is the patron of those who live till the end of a Kalpa. He is engaged in serving the lotus-feet of the Bhagav\=an. He has been made by God Himself the pillar round whom all the planets, stars, and the luminary bodies are revolving always and with great force in the Zodiac and in the celestial Heavens. The Devas also worship him. He, resplendent in his own glory, illumines and manifests all. As beasts tied to yoke go on tilling, so the planets and stars, fixed on the Zodiac, go quickly round and round this Dhruva, the Pole Star; some nearer, some further distant in spheres, propelled by V\=ayu. As the hawks hover round the sky, so the above-mentioned planets, go completely round and round under their own Karmas and controlled by the V\=ayu in the sky. Thus all the luminaries do not fall to the ground, as they are kept up in their respective positions by the favour of the union of Prakriti and Puru\d{s}a. Some say that this Jyoti\d{s}chakra, the celestial Heavens (the Zodiac) is \'Si\'sum\=ara. It is kept duly in its position for the purpose of holding things up by the power of the Bhagav\=an. Hence it does not fall. It is resting with its body coiled round and with its head lower down. O Muni! Dhruva, the son of Utt\=anap\=ada is staying at the tail end. And, in addition to him, also at the tail rests Brahm\=a, the Sinless Praj\=apati, worshipped by the Gods, Ag\d{n}i, Indra and Dharma. Thus the creation is at the tail and the Saptar\d{s}imandal is staying at his waist. Thus the celestial wheel (Jyoti\d{s}chakra) is resting with his coils turned in a right-hand direction. On his right side are found the Uttar\=aya\d{n}a Nak\d{s}attras, fourteen in number from Abhijit to Punarvasu and on his left side are found the other fourteen Dak\d{s}i\d{n}\=ayanam Nak\d{s}attras from Pu\d{s}y\=a to Uttar\=a\d{s}\=adh\=a. O Son of Brahm\=a! Thus the Nak\d{s}attras form the coil-shaped body of the \'Si\'sum\=ara, the Zodiac; half the Nak\d{s}attras

on the one side and the other half Nak\d{s}attras on the other. His back is on the Heavenly Ganges named Ajav\={\i}th\={\i}. Punarvasu and Pu\d{s}y\=a form the right and left side of the loins; \=Ardr\=a and A\'sle\'s\=a form the right and left feet (westward); Abhijit and Uttar\=a\d{s}\=adha form the right and left nostrils. O Devar\d{s}i! \'Srava\d{n}\=a and P\=urv\=a\d{s}\=adh\=a form the right and left eyes respectively; so say the persons that form the Kalpan\=as (fancies). Dhanisth\=a and M\=ul\=a form his right and left ears; Magh\=a, etc., the eight Dak\d{s}i\d{n}\=ayanam Nak\d{s}attras form the bones on the left side. O Muni! Mrigas\={\i}r\d{s}a, the Uttar\=aya\d{n}a Nak\d{s}attras form the bones on his right side, \'Satabhi\d{s}\=a and Jyesth\=a form the right and left shoulders. Agasti (the Canopus) forms the upper jaw and Yama, the lower jaw. The planet Mars forms his face; Saturn forms his organ of generation; Brihaspati forms the hump on the shoulders; the Sun, the Lord of the planets, forms his breast; N\=ar\=aya\d{n}a remains in the heart; and the Moon is in his mind. (Note :-- \'Si\'sum\=ara is also the constellation Dolphinus and is sometimes meant for the polar star.) O N\=arada! The two A\'svins form the nipples on his breast; U\'san\=a forms his navel; the Mercury is his Pr\=a\d{n}a and Ap\=ana; R\=ahu is his neck and Ketu is all over his body and the stars are reigning all over the hairs of his body. This Zodiac is the body composed of the Devas of that All Pervading Bhagav\=an. So every intelligent person should daily meditate this \'Si\'sum\=ara in the Sandhy\=a time, with perfect purity and keeping himself Mauna (silent), and with his whole heart. Then he should repeat the following mantras and get up and say :-- ``Thou art the Substratum of all the luminaries, we bow down to Thee; Thou createst and destroyest all. Thou art the Lord of all the celestials. Thou art the \=Adipuru\d{s}a, the foremost of all the Puru\d{s}as; we meditate fully on Thee. The planets, Nak\d{s}attras, and the stars are Thy body. The Daiva is established in Thee alone. Thou destroyest the sins of those that compose the Mantras. The sins are completely destroyed for the time being of him who bows down or remembers Thee in the morning, afternoon and evening.''

Here ends the Seventeenth Chapter of the Eighth Book on the Dhruva Mandalam in the Mah\=a Pur\=a\d{n}am \'Sr\={\i} Mad Dev\={\i} Bh\=agavatam of 18,000 verses, by Mahar\d{s}i Veda Vy\=asa.



