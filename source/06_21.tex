\chapter{On the installation of Ekav\={\i}ra and the birth of Ek\=aval\={\i}}

1-10. Vy\=asa said :-- O King! In the meanwhile the King Turvasu performed the J\=atakarma (a religious ceremony performed at the birth of a child) and other ceremonies of the child. The boy was nurtured duly and began to grow older day by day. The King began to enjoy his worldly life on getting this son and thought within himself that the boy had freed him from the three debts due to the Fathers, the \d{R}i\d{s}is and the Devas. Next, in the sixth month, the King performed the Annapr\=asana ceremony (putting the boiled rice in the mouth of the child) and in the third year performed regularly his Ch\=uda Kara\d{n}a (the ceremony of the first tonsure) ceremony. He distributed on those occasions various articles, wealth and cows to the Br\=ahma\d{n}as and other articles to various other mendicants and made them glad. In the eleventh year, he performed the boy's holy

thread (Upanayana) ceremony and tying the girdle made of a triple string of Munja grass and put the boy to learn archery. Next when the boy passed off proficiently in the study of the Vedas and in learning the kingly duties, the King desired to install him on the throne. The King Turvasu then collected with great care all the necessary articles for installation in an auspicious day, the combination of Pu\d{s}y\=a asterism and Arka Yoga. He called then the Br\=ahmi\d{n}s, well versed in the Vedas and in the \'S\=astras, and became ready, in accordance with due rites, to perform the installation ceremony of the prince. Waters were brought from various sacred places of pilgrimage and from the several oceans and on an auspicious day the King performed himself the installation of his son. When the ceremony was over the King gave away hoards of wealth to the Br\=ahmi\d{n}s and giving the charge of his kingdom to his son, he went to the forest with a desire to ascend to the Heavens.

11-22. Thus placing Ekav\={\i}ra on the throne, the King Turvasu shewed respects to his ministers, and, controlling his senses went to the forest accompanied by his wife. On the top of the Main\=aka mountain he took up the vow of V\=anaprastha and sustaining his life on leaves and fruits began to meditate P\=arvat\={\i}. Thus when his Pr\=arabdha Karma ended, he left his mortal coil with his wife and went by virtue of his good deeds to the Indraloka. Hearing that the King had ascended to Heavens, his son Ekav\={\i}ra Haihaya performed his funeral ceremonies according to the rules laid down by the Vedas. The King's son, the intelligent Haihaya, performed, one after another, all the ceremonies due and began to govern the kingdom which was free from enemies. The virtuous King Ekav\={\i}ra remained duly obedient to his ministers after he got possession of his kingdom and began to enjoy all the best things. The powerful King one day went on horseback to the banks of the Ganges with the minister's son. Roaming about, he found there the boughs of trees had assumed a very graceful appearance, with loads of fruits, echoed with the sweet voice of the cuckoos and with the humming of the bees. Close by were the hermitages of the Munis, where the bucks were skipping about and at other places the Vedas were being chanted. The smoke was seen rising from the altars, where oblations were being offered and appeared to form like a black canopy in the Heavens. Full ripe grains were enhancing the beauty of the fields and the cowherdesses were merrily watching the fields. Places of recreations adorned with full blown lotuses and beautiful groves were attracting the attention of the visitors. The various trees, Piy\=ala, Champaka, Panasa, Bakula, Tilaka, Kadamba and Mand\=ara, and others were adorned with fruits, stealing away the minds of the people. At other places, other trees Sal, Tam\=ala, Jack,

Mango, Kali Kadamba, etc., stood gracefully. Next when the King went to the Ganges water, he saw the gay beautiful full blown lotuses were spreading their fragrant scents all around.

23-31. On the right side of these lotuses, he saw a lotus-eyed girl. She was shining like the gold, her beautiful hairs were long and curling; her throat was like a Kambu, belly thin, lips like the Bimba fruits, several other limbs well built and graceful, breasts risen a little, nose beautiful and all her body was exquisitely lovely; that lady just blooming into youth was suffering bereavements from her comrades and was very distressed and seemed bewildered. She was crying like an ewe in a dense lonely forest. Seeing her, the King asked her what was the reason of her sorrows? O Cuckoo-voiced One! You are as yet a girl; who has left you alone in this state? O Sweet One! Tell me where is your husband now or where is your father? O One looking askance! What is your trouble; explain it to me. O thin-bellied One! I will, no doubt, remove all your sorrows and troubles. O fair-limbed One! In my dominion nobody ever gives trouble to any other body. O lovely One! There is no fear in my kingdom from thieves or R\=ak\d{s}asas; or any fear from any serious dangerous calamities on this earth, fear from lions, tigers or any other dangers while my sway is predominant.

32-41. O One of beautiful thighs! Why are you crying on this lonely bank of the Ganges? Tell me what is your pain? O Pure One! I can remove the pains and miseries, even of a serious nature, of men, whether they come from the Deva or human sources; and this is my principal vow. O Large-eyed One! Speak what is your inmost desire; I will carry it out instantaneously. When the king thus spoke, that beautiful woman spoke in gentle words :-- O King! Hear the cause of my sorrows. O King! Why will the people cry, to no purpose, unless calamities come before them? O Mighty-armed One! I now tell you why I am weeping. O King! There was a very religious King named Rabhya in another province that is not yours. At first he had no issue. He had a very beautiful wife named Rukmarekh\=a. She was clever, chaste and endowed with all auspicious qualities. But issueless as she was, she remained very sorry and, in a remorseful tone, she spoke to her husband Raibbya :-- O Lord! I am barren; I have no sons; I am therefore a very unhappy creature. My life is in vain; what use is there in my living? When the queen thus spoke very distressedly, the king called the Br\=ahma\d{n}as, versed in the Vedas, and began to perform an excellent sacrificial ceremony, in due accordance with the Vedic rules. With a desire to get a son, he made many presents in profuse quantities. When copious quantities

of ghee were offered as oblations, there arose, from the fire, a girl beautiful in all respects and endowed with all auspicious signs.

42-53. Her teeth were very nice, eyebrows very lovely, face enchanting like a Full Moon, the lustre of the body lovely and of a golden colour; her hairs were fine and curling; her lips like the Bimba flowers; her hands and face were of a red colour; her eyes were red like lotus and her limbs were soft and gentle. When the girl arose from the fire, the priest (Hot\=a) took that lean and thin lady of a nice waist by her arms and presented her to the King and said :-- O King! Accept this daughter, endowed with all auspicious signs. When Homa was being performed, the daughter came up like the garland Ek\=aval\={\i}; therefore this girl became famous in this world by the name Ek\=aval\={\i}. O Ruler of the earth! Take this girl, resembling a son and be happy.

O King! Vi\d{s}\d{n}u, the Deva of the Devas, has given you this Jewel, this daughter; so be contented. Hearing thus the words of the priest, the King saw this good-looking girl and with gladdened heart took the beautiful daughter from his hands. Thus with that lovely daughter he went to his wife Rukmarekh\=a and said :-- O Beautiful One! Take this daughter. The queen Rukmarekh\=a felt the pleasure of having a son when she got in her arms that lotus eyed beautiful daughter. The King next performed the natal and other ceremonies of the daughter and did all other acts as if she had been a son to him duly in accordance with the rules. The King performed his own sacrificial ceremonies and gave away lots of Dak\d{s}i\d{n}\=as to the Br\=ahmi\d{n}s and dismissed them and became very glad. That beautiful girl was nursed and cared after like a son and she grew older day by day. The Queen Rukmarekh\=a was very gladdened to get her. On that very day the birthday festival was performed as on the occasion of the birth of a son. And that daughter grew older, very affectionate and dear to all.

54-61. O Lovely One! You are a king and intelligent too; I will describe to you all the details; Hear. I am the daughter of the minister to that King. My name is Ya\'sovat\={\i}. That daughter and I look alike and of the same age. Therefore the king has made me her comrade. I spend my time day and night always with her as her constant dear companion. Ek\=aval\={\i} likes very much to remain and sport wherever she finds sweet-scented lotuses; at other places she does not find happiness. At the distant banks of the Ganges many lotuses grow; therefore Ek\=aval\={\i} goes there with great pleasure to that place with me and her other fellow mates. One day I told the King that Ek\=aval\={\i} used to

go daily to a distant solitary forest to see the lotus-lake. Then the King addressed her not to go and he got a lake built within the compounds of his palace and planted many lotus seeds therein. Gradually the lotuses began to blossom and the bees came there to drink honey. Still she used to go outside in search of lotuses. Then the King sent armed guards to accompany her. Thus that thin-bodied daughter of the King used to go daily to the banks of the Ganges for play, guarded by armed soldiers, accompanied by myself and other companions. Again when the sporting was over, she used to return to the palace.

Here ends the Twenty-First Chapter on the Sixth Book on the installation of Ekav\={\i}ra and the birth of Ek\=aval\={\i} in the Mah\=apur\=a\d{n}am \'Sr\={\i} Mad Devi Bh\=agavatam by Mahar\d{s}i Veda Vy\=asa.



