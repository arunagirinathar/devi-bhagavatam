\chapter{On Devak\={\i}'s marriage}

1-2. Vy\=asa said :-- O Bharata! I now narrate to you about the relief of the burden of the Earth, the destruction by the Yoga M\=ay\=a of the armies Kuruk\d{s}ettra and Prabh\=asa, the two sacred places, and about the birth, due to the curse of Bhrigu, of the Bhagav\=an Hari, of unparalleled prowess, under the influence of Mah\=am\=ay\=a, in the race of Yadu. Now hear.

3. Vi\d{s}\d{n}u's taking birth in the world was effected under the circumstances caused by Yoga M\=ay\=a, to relieve the burden of the Earth; this is my opinion.

N. B. :-- Prabh\=asa is a well-known place of pilgrimage near Dv\=ark\=a, in Gujerat.

4. O king! When the Goddess Mah\=am\=ay\=a, the Incarnate of the three qualities can make Brahm\=a, Vi\d{s}\d{n}u and the other Devas dance incessantly as their Internal Controller, then what wonder, that She would appear charming to the other J\={\i}vas and make them dance incessantly, as wooden dolls.

5. All the troubles incurred in remaining in the womb, amidst the urine, excreta and tissues, She had brought to bear finally on Vi\d{s}\d{n}u by Her ever famous Leel\=a (Divine Sport).

6. In days of yore, in R\=ama incarnation, She, That Supreme Goddess made the Gods become monkeys and you know very well already, what amount of trouble Bhagav\=an Vi\d{s}\d{n}u had to undergo by the hard iron chain of bondage, ``This is mine, this is I,'' etc., imposed by M\=ay\=a.

7. The Yogis who want final liberation and who have abandoned all their attachments and who want devotion, they worship the Supreme Goddess of the Universe, the Incarnate of Auspiciousness.

8. O king! Who will not serve Her? A trace of devotion towards Whom effects at once the salvation of the J\={\i}vas.

9. If any human being utters simply the name ``Bhuvane\'svar\={\i}'' (the Goddess of Universe) She gives him the three worlds; and if any one cries and utters for help ``Save me,'' then the Goddess of the Universe, being unable to find anything that She can repay him as a fit recompense for his utterance, becomes Herself indebted to that man.

10. O King! Know Vidy\=a (knowledge) and Avidy\=a (non-knowledge, spiritual ignorance, lower knowledge) Her two forms; Vidy\=a gives liberation; Avidy\=a causes bondage.

11. Brahm\=a, Vi\d{s}\d{n}u, Rudra, all these and their Avat\=aras are Her instruments and are under Her subjection, as if they are all fastened by a cord.

12-14. The Bhagav\=an Hari sometimes resides in Vaikuntha, sometimes resides in the sea of milk and enjoys pleasures, sometimes fights the powerful D\=anavas, sometimes performs extensive sacrificial ceremonies

sometimes performs severe asceticism and sometimes takes to deep sleep under the guidance of Yoga M\=ay\=a. Thus He never becomes free and independent.

15-16. O King! And like Vi\d{s}\d{n}u, Brahm\=a, Rudra, the other Gods Indra, Varu\d{n}a, Yama, Kuvera, Agni, the Sun, Moon and other celestial gods, the Sanaka and other Munis, Va\'sistha and other \d{R}i\d{s}is, all are incessantly controlled by the Supreme Goddess, as if they are the dolls in the hands of a playing magician.

17. All the Devas are controlled by the noose of Time, just as oxen are brought under control by men, by strings passed within their noses.

18. O King! Pleasure, pain, sleep, lassitude, idleness, and other passions and feelings are always found inherent in every embodied being.

19-23. The Devas are written down by authors in their books as not subject to death or decay; this statement is merely nominal and not real; for whoever is always subject to birth, growth, decay and death how can we call him immortal or beyond old age and death? Why do the Devas get into sorrows, and troubles? How can they be called gods? How can one enjoy when one is involved in a trouble? In this world, it is seen mosquitoes and other insects born from water die after a very short time; so, like these, the Devas at the expiry of their ordained life-period, die too. Then why not the Devas be treated like those insects? Why not shall we term them, ``Mortal''?

24-25. Some human beings live one year; some others live upto one hundred years, the Devas live longer than men; the life of the Pr\=aj\=apat\={\i} Brahm\=a exceeds those of the Devas; Rudra lives longer than Brahm\=a; and Vi\d{s}\d{n}u lives longer than Rudra. All these are thus subject by turns to birth, growth, and decay.

26. Those who are born, they die; those that die are again born. Thus O King! In this world all the J\={\i}vas, no doubt, move round and round like a wheel.

27. The J\={\i}vas are deluded by the network of Moha (charms) and thus deluded. They can never gain final release. So long as M\=ay\=a exists in them, their delusion is not destroyed.

28. O King! At the time of creation Brahm\=a and all other things came into existence, in due order, and these will duly dissolve at the time of the Great Dissolution (the Pralaya)?

29. Whatever is the cause of destruction to anybody here, that destroys the body in question. According to the Will Power of the Supreme Goddess, whatever is created by Brahm\=a, no none can undo that. Know this as perfectly certain.

30. Know this the predestined law that the birth, death, old age, diseases, pain or pleasure overtake all the J\={\i}vas according to the prescribed order of Nature; never these laws fail to operate in their actions.

31. See! The Devas that we see with our physical eyes, the Sun and Moon yield pleasure to all; still they suffer always troubles at the hands of their enemies (i.e., R\=ahu and Ketu, i.e., there always take place regularly solar and lunar eclipses, when they are in the ascending and descending modes.) This suffering is never removed.

32. The son of the Sun (Saturn) is always injurious to people; hence it is termed ``Manda'' (bad); the Moon was attacked with consumption and bears stain on his body (spots in the Moon disc). What to speak of ordinary men? The great men even are subject to the prescribed laws of Fate and Nature.

33. The Creator of the world, the four faced Brahm\=a is the author of the Vedas and awarder of Intelligence. He, too, on seeing Sarasvat\={\i}, his own daughter, was afflicted with passion.

34. When Sati, the wife of \'Siva, left off Her mortal coil, Mah\=adeva, though He could remove the sorrows of all, was very much moved with passion and greatly afflicted.

35. Then, being burnt very much as it were by the fire of passion, he threw himself down into the water of the river K\=alind\={\i}; and the water of that river became black-coloured, as if burnt by the burning fire of the sorrows of \'Siva.

36-37. O King! When Mah\=adeva, being infatuated with K\=ama, went into the forest of Bhrigu and becoming naked, began to copulate, the ascetic Bhrigu, seeing him in that state, exclaimed ``O You are very shameles\'s' and cursed Him thus :-- Let your penis drop off just now. Mah\=adeva, then to satisfy his thirst for passion, began to drink the water of the lake Amrita Vap\={\i}, dug by the D\=anavas.

38-39. Indra, too, the Lord of the Devas, turned into a bull and carried Vi\d{s}\d{n}u on his back on the face of the earth. What to speak where the omniscience and omnipotence disappeared of Bhagav\=an Vi\d{s}\d{n}u, Who is the First of all created beings and possessed of reason, and discrimination? Oh! What a great wonder, that He could not know about the golden deer?

40-41. Judge, O King! for yourself, the great power of M\=ay\=a, that even R\=ama Chandra was deluded by passion, and very much grieved for the

separation from his dear wife S\={\i}t\=a, and wept much for Her. Greatly deluded, he began to cry aloud and ask the trees ``Where has gone My S\={\i}t\=a, the daughter of Janaka? Is she devoured by the rapacious animals? or whether is she stolen by some mischievous person?

42-44. O Brother Lak\d{s}ma\d{n}a! I am being consumed by the fire of separation from my beloved; I will commit suicide now; and you too will die due to the separation from me; our mother, too, would die when they will hear of our deaths; Satrugh\d{n}a, too, will be very much afflicted at our death and will not hold his life. The mother Sumitr\=a, too, will destroy her life, being afflicted with her son's demise; and then Bharata's and his mother's desires will certainly be fulfilled.

45. O S\={\i}t\=a! I am very much moved by your separation; where have you gone, leaving me here! O deer-eyed, O one of thin waist! come; give life unto me!

46. What shall I do? Where shall I go? My life is entirely dependent on you, O daughter of Janaka! I am your darling! Now I am very much depressed owing to your separation. Please come and speak to me words of hope and courage.

47. Though R\=ama, of unequalled prowess, roamed about weeping from forest to forest, yet he could not find out S\={\i}t\=a.

48-49. He, who is the Refuge of all the worlds, the lotus-eyed R\=ama, got entangled into the delusion of M\=ay\=a and had to take refuge Himself under the monkeys, and with their help, constructed the bridge across the ocean, crossed the ocean and was thus able to kill the valiant warriors Kumbhakar\d{n}a and R\=ava\d{n}a.

50. Then R\=ama got back His S\={\i}t\=a before him but suspecting, since she had been stolen by the vicious R\=ava\d{n}a, made her take an oath, though it is to, be remembered that R\=ama was all-knowing.

51. O King! The power of Yoga M\=ay\=a is very great; what shall I speak of Her great power? This whole cosmos is always urged into activity by Her and thus goes rolling on and on incessantly.

52. Thus, in various incarnations, Bhagav\=an Vi\d{s}\d{n}u was always under the influence of previous curse and also under the control of Destiny and had to do various functions incessantly.

53. O King! Now I will speak to you about the birth of \'Sr\={\i} Kri\d{s}\d{n}a in the world for serving the purposes of gods, and will narrate His Leel\=a.

54. In days of yore, on the delightful banks of the river K\=alind\={\i}, there was a place, called Madhuban, where lived a powerful Daitya named Lavana, the son of Madhu.

55-56. That wicked Demon was exceedingly arrogant, on getting a boon, and he used to give an enormous amount of trouble to the Dvijas. Satrugh\d{n}a the younger of Lak\d{s}ma\d{n}, killed that uncontrollable Daitya and built a very beautiful city there and named it Mathur\=a.

57. The intelligent Satrugh\d{n}a, the destroyer of enemies, installed his two lotus-eyed sons in that kingdom and, when his end came, went to Heaven.

58. Afterwards on the decline of the Solar race, the Y\=adavas, born of the race of Yay\=ati, occupied that Mathur\=a city, giving salvation to all.

59. O King! There reigned in Mathur\=a city one Y\=adava king, valiant warrior, named \'S\=urasena; and he enjoyed all the pleasures Mathur\=a.

60. Under the curse of Varu\d{n}a, V\=asudeva took his birth as the son of the renowned \'S\=urasena, as the part incarnate of Ka\'syapa.

61. He took up the profession of a Vai\'sya and engaged himself in agriculture. And on the death of his father, the prosperous and wealthy Ugrasena became the King of Mathur\=a. The powerful Kamsa was the son Ugrasena.

62. On the other hand, the King Devaka had a daughter born to him named Devak\={\i}, the part incarnate of Aditi. She under the curse of Varu\d{n}a, followed Ka\'syapa.

63. The high souled King Devaka performed the marriage ceremony of his daughter Devak\={\i} with V\=asudeva.

64. When this marriage ceremony was over, a voice was heard from Heaven, saying :-- O fortunate Kamsa! The eighth son of this Devak\={\i} will take away your life.

65. The powerful Kamsa, hearing that voice from Heaven, was surprised and took it to be true and became very anxious.

66-67. Kamsa began to argue in his mind. Once he thought ``I would today destroy her; then my death won't take place; for I can't see any other way of escape from this difficulty,'' again he thought, ``She is my sister, daughter of my paternal uncle and therefore fit to be worshipped; how can I kill her!''

68. Lastly, he came to the final conclusion, thus ``She is the cause of my death, though she is my sister, fit to be worshipped; to kill her will not

lead me to sin; for it is enjoined by the wise :-- Do even a sin to avert one's own death.''

69. The sins can be remedied always by penances. Therefore to save one's life, by committing even a sinful act, ought to be done by the wise.

70-71. The vicious Kamsa thus arguing, holding the scabbard in his hand, drew from it the sword and dragged and caught hold of the newly married handsome woman by her hair to kill her before the presence of the public.

72. A cry of universal consternation and distress arose on all sides, seeing Kamsa thus ready to kill Devak\={\i}; then the warriors, under V\=asudeva, at once raised their bows and arrows, ready to fight.

73. These wonderfully valorous warriors loudly exclaimed to Kamsa, ``Leave Devak\={\i} at once'' ``Leave Devak\={\i} at once'' and then they were finally able out of their mercy to release the Devam\=at\=a Devak\={\i}, from the hold of the vicious Kamsa.

74. Deadly battles ensued then between the powerful Kamsa and those valorous warriors on V\=asudeva's side.

75-76. Seeing the exceedingly terrible battle, the old Y\=adavas asked Kamsa to desist from such a battle and advised him thus. This Devak\={\i} is your sister; you ought to pay her respects. Did you not consider even for a moment that she is as yet a girl. O Hero! You ought not to kill her at the time of this joyous marriage ceremony.

77. O Valiant Warrior! The murder of a woman is intolerable! Destroyer of fame, and most heinous crime! You should also consider that learned persons ought not to commit such dastardly acts as the murdering of females, depending simply on a voice from heaven, a very ordinary thing!

78. It may be that some of your V\=asudeva's enemy has uttered that harmful word, hiding himself from your sight. No reason can be shewn contrary to this.

79. We are of opinion that to ruin your name and to destroy the house of V\=asudeva, some magician, expert in black magic, your enemy has framed this voice from Heaven.

80. O king! You are a hero; why do you fear the words of a devil. We firmly believe, there is no doubt, that this is done by your malicious enemy to ruin your name.

81. O king! What is destined to take place, will take place; no one can stand against it otherwise. Therefore, at this marriage festivity, you ought never to kill this your respected sister.

82-83. O King Janamejaya! Though made to understand thus by the old wise Y\=adavas, the king Kamsa did not desist from his purpose; \'Sr\={\i} V\=asudeva, versed in morals, told him ``Kamsa! These three worlds are established on Truth. I say on Truth that I will hand over to you all my sons, born of the womb of Devak\={\i}, no sooner they are born.

84. And if I do not deliver to you all those sons, no sooner they are born then all my forefathers will fall down into the hell called Kumb\={\i}hp\=aka.''

85-86. The descendants of Puru, that were present there, hearing his truthful words, praised him repeatedly and told Kamsa ``V\=asudeva is a high minded personage; he is surely not to tell a lie. Therefore, O Thou, blessed one! Now leave Devak\={\i} and be free from committing the murder of woman.''

87. O king! Thus made to understand by the aged high minded Y\=adavas the king Kamsa accepted the truthful words of V\=asudeva and abandoned his anger.

88. Then the Dunduvis and other sounding instruments were sounded; and their sounds filled the place; and all repeatedly uttered jai, jai.

89. Then the famous V\=asudeva, the son of \'S\=urasena, thus pleased the king Kamsa and freed Devak\={\i}; and, surrounded by his relatives, he went quickly without any fear to his own house, accompanied by Devak\={\i}.

Here ends the 20th chapter in the 4th Adhy\=aya of \'Sr\={\i}mad Dev\={\i} Bh\=agavatam, the Mah\=a Pur\=a\d{n}am, of 18,000 verses, by Mahar\d{s}i Veda Vy\=asa, on Devak\={\i}'s marriage.



