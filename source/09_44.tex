\chapter{On the story of Svadh\=a Dev\={\i} in the discourse between N\=arada and N\=ar\=aya\d{n}a}

1-18. N\=ar\=aya\d{n}a said said :-- O N\=arada! I will tell you now the excellent anecdote of Svadh\=a, pleasing to the Pitris and enhancing the fruits of the \'Sr\=adh ceremony when foods are offered to the Pitris. Listen. Before the creation, the Creator created seven Pitris. Four out of them are with forms and the other three are of the nature of Teja (light).

Note :-- Kavyav\=ahoanalah Somo Yamaschaiv\=aryam\=a tath\=a, Agni\d{s}v\=att\=ah Barhi\d{s}adah Somap\=a Pitri Devat\=ah. These seven Pitris are according to the other Pur\=a\d{n}as. Seeing the beautiful and lovely forms of the Pitris, He made arrangements for their food in the form of \'Sr\=addhas and

Tarpa\d{n}as, etc. (funeral ceremony and peace-offerings), etc. (\'Sr\=adh, solemn obsequies performed in honour of the manes of deceased ancestors.)

Taking bath, performing \'Sr\=adh ceremony upto Tarpa\d{n}am (peace-offerings), worshipping the Devas and doing Sandhy\=a thrice a day; these are the daily duties of the Br\=ahma\d{n}as. If any Br\=ahma\d{n}a does not perform daily the Trisandhy\=as, \'Sr\=addha, Tarpa\d{n}am, worship and the reciting of the Vedas, he becomes devoid of fire like a snake without any poison. He who does not perform the devotional service of the Dev\={\i}, who eats food not offered to \'Sr\={\i} Hari, who remains impure till death, is not entitled to do any karma whatsoever. Thus, introducing the \'Sr\=addhas, etc., for the Pitris, Brahm\=a went to His own abode. The Br\=ahma\d{n}as went on doing the \'Sr\=addhas for the Pitris, but the Pitris could not enjoy them and so they remained without food and were not satisfied. They all, being hungry and sad, went to the Council of Brahm\=a and informed Him everything from beginning to end. Brahm\=a then created out of His mind one daughter very beautiful, full of youth and having a face lovely, as if equal to one hundred moons. That woman was best in all respects whether in form, beauty, qualities or in learning. Her colour was white like the white Champaka flower and her body was adorned all over with jewel ornaments. This form was very pure, ready to grant boons, auspicious and the part of Prakriti. Her face was beaming with smiles; her teeth were very beautiful and her body showed signs of Lak\d{s}m\={\i} (i.e., of wealth and prosperity). Her name was Svadh\=a. Her lotus-feet were situated on one hundred lotuses. She was the wife of the Pitris. Her face resembled that of a lotus and Her eyes looked like water lilies. She was born of the lotus born Brahm\=a. The Grand-father Brahm\=a made over that daughter of the nature of Tusti (Contentment) to the hands of the Pitris and they were satisfied. Brahm\=a advised the Br\=ahma\d{n}as privately that whenever they would offer anything to the Pitris, they should offer duly with the mantra Svadh\=a pronounced at the end. Since then the Br\=ahma\d{n}as are offering everything to the Pitris, with the Mantra Svadh\=a uttered in the end. Sv\=ah\=a is laudable, when offerings are presented to the Gods and Svadh\=a is commendable when offerings are made to the Pitris. But in both the cases, Dak\d{s}i\d{n}\=a is essential. Without Dak\d{s}i\d{n}\=a (sacrificial fee), all sacrifices are useless and worthless. The Pitris, Devat\=as, Br\=ahma\d{n}as, the Munis, the Manus worshipped the peaceful Svadh\=a and chanted hymns to Her with great love. The Devas, Pitris, Br\=ahma\d{n}as, all were pleased and felt their ends achieved when they got the boon from Svadh\=a Dev\={\i}. Thus I have told you everything about Svadh\=a. It is pleasing to all. What more do you want to hear? Say. I will answer all your questions.

19. N\=arada said :-- ``O Thou, the Best of the knowers of the Vedas! O Muni Sattama! I want now to hear the method of worship, the meditation and the hymns of Svadh\=a Dev\={\i}. Kindly tell me all about this.''

20-27. N\=ar\=aya\d{n}a said :-- You know everything about the all-auspicious Dhy\=an, Stotra, as stated in the Vedas; then why do you ask me again? However I will speak out this for the enhancement of knowledge. On the thirteenth day of the Dark Fortnight in autumn when the Magh\=a asterism is with the Moon and on the \'Sr\=addha day. One should worship with care Svadh\=a Dev\={\i}; then one should perform \'Sr\=addha. If, out of vanity, a Br\=ahmi\d{n} performs \'Sr\=adh without first worshipping Svadh\=a Dev\={\i} then he will never get the fruits of Tarpa\d{n}am or \'Sr\=adh. ``O Dev\={\i} Svadhe! Thou art the mind-born daughter of Brahm\=a, always young and worshipped by the Pitris. Thou bestowest the fruits of \'Sr\=addh. So I meditate on Thee.'' Thus meditating, the Br\=ahmi\d{n} is to pronounce the motto (m\=ula mantra) and offer the P\=adyam, etc., on the \'S\=alagr\=ama stone or on the auspicious earthen jar. This is the ruling of the Vedas. The motto is ``Om Hr\={\i}m, \'Sr\={\i}m, Kl\={\i}m, Svadh\=a Devyai Sv\=ah\=a.'' She should be worshipped with this Mantra. After reciting hymns to the Dev\={\i}, one is to bow down to the Svadh\=a Dev\={\i}. O Son of Brahm\=a! O Best of Munis! O Skilled in hearing! I now describe the stotra which Brahm\=a composed at the beginning for the bestowal of the desired fruits to mankind. Listen.

N\=ar\=aya\d{n}a said :-- The instant the Mantra Svadh\=a is pronounced, men get at once the fruits of bathing in the holy places of pilgrimages. No trace of sin exists in him at that time; rather the religious merits of performing the V\=ajapeya sacrifice accrue to him.

28-36. ``Svadh\=a,'' ``Svadh\=a,'' ``Svadh\=a,'' thrice this word if one calls to mind, one gets the fruits of \'Sr\=adh, Tarpa\d{n}am, and Bali (offering sacrifices). So much so, if one hears with devotion during the \'Sr\=adh time the recitation of the hymn to Svadh\=a, one gets, no doubt, the fruit of \'Sr\=adh. If one recites the Svadh\=a mantra thrice every time in the morning, midday and evening, every day, one gets an obedient, chaste wife begetting sons. The following is the hymn (Stotra) to Svadh\=a :-- ``O Dev\={\i} Svadhe! Thou art dear to the Pitris as their vital breaths and thou art the lives of the Br\=ahma\d{n}as. Thou art the Presiding Deity of the \'Sr\=adh ceremonies and bestowest the fruits thereof. O Thou of good vows! Thou art eternal, true, and of the nature of religious merits. Thou appearest in creation and disappearest in dissolution. And this appearing and disappearing go on forever. Thou art Om, thou art Svasti, Thou art Namas K\=ara (salutation); Thou art Svadh\=a, Thou art Dak\d{s}i\d{n}\=a, Thou art the various works as designated in the Vedas. These the Lord of the world has

created for the success of actions.'' No sooner Brahm\=a, seated in His assembly in the Brahm\=a Loka, reciting this stotra remained silent, than Svadh\=a Dev\={\i} appeared there all at once. When Brahm\=a handed over the lotus-faced Svadh\=a Dev\={\i} over to the hands of the Pitris, and they gladly took Her to their own abodes. He who hears with devotion and attention this stotra of Svadh\=a, gets all sorts of rich fruits that are desired and the fruits of bathing in all the T\={\i}rthas.

Here ends the Forty-fourth Chapter of the Ninth Book on the story of Svadh\=a Dev\={\i} in the discourse between N\=arada and N\=ar\=aya\d{n}a in the Mah\=a Pur\=a\d{n}am \'Sr\={\i} Mad Dev\={\i} Bh\=agavatam of 18,000 verses by Mahar\d{s}i Veda Vy\=asa.



