\chapter{On the description of the curse on Gang\=a, Mah\=abhi\d{s}a and Vasus}

1-8. The \d{R}i\d{s}is said :-- ``O Sinless S\=uta! You have described to us in detail the birth of Vy\=asa, of unrivalled fire, and of Satyavat\={\i}; but we have one great doubt in our minds though, O Knower of Dharma! which is not being removed by your words. O Sinless one! First, as regards the mother of Vy\=asa, the all auspicious Satyavat\={\i}, we have this doubt how she came to be united to the virtuous \'Santanu? The king \'Santanu, of the family of Puru is a greatly religious man; how could he have married Satyavat\={\i} knowing her to be a fisherman's daughter and born of a low family? Now say who was the first wife of \'Santanu and how Bh\={\i}sma, the intelligent son of \'Santanu came to be born of the parts of Vasu? O S\=uta! You have told before that Bh\={\i}sma, of indomitable valour, made the Satyavat\={\i}'s son, the brave Chitr\=angada, king; and subsequent to his death made his younger brother Vich\={\i}trav\={\i}rya king. But when the elder brother Bh\={\i}sma, the greatly religious and beautiful was present, how was it that Chitr\=angada and Vich\={\i}trav\={\i}rya having been installed by Bh\={\i}sma himself could have reigned.''

9-12. Again on the demise of Vich\={\i}trav\={\i}rya, Satyavat\={\i} became very much grieved and got two sons born of her two son's wives by Vedavy\=asa? How can we explain this fact? Why did she do this? Why did she not give to Bh\={\i}sma the kingdom? Why did Bh\={\i}sma not marry? And how was it that the elder brother Vy\=asa Deva, of indomitable valour, did such an irreligious act as to beget two (Goloka) sons from the wives of the brothers? Vy\=asa composed the Pur\=a\d{n}as and knew everything of religion; how then did he go to other's wives, especially, of his brother's wives?

13-14. O S\=uta! Why did Vy\=asa Deva do such a hateful act, in spite of his being a Muni? The actions of Vedas are inferred from their subsequent good conducts; how can this act of Vy\=asa be calculated as one amongst them? O Intelligent one! You are the disciple of Vy\=asa; therefore you are the best man to solve our doubts. We all of this Dharmakshettra Naimi\d{s}\=ara\d{n}ya are very eager to hear this.

15-39. At this S\=uta said :-- In ancient days there reigned a king named Mah\=abhi\d{s}a, in the family of Ik\d{s}\=aku endowed with all the qualities of a great king; he was the foremost of all the kings, truthful and religious. That highly intelligent king performed thousand horse-sacrifices (Asva

medhas) one hundred V\=ajapeya sacrifices and thereby satisfied Indra, the king of the Devas and went to Heavens. Once, on an occasion, that king went to the abode of Brahm\=a; the other gods also went there to serve Praj\=apati. The great river, Gang\=a Dev\={\i}, too, assuming the feminine form, went to Brahm\=a to serve him. Now, in the interval, violent winds arose and the clothing of Gang\=a Dev\={\i} went off; at this the Devas did not look at her; rather kept their faces downwards; but the king Mah\=abhi\d{s}a continued gazing at her. Gang\=a also came to know the king and that he had become attached to her. Brahm\=a, seeing that both of them are love-stricken and are shameless, became angry and cursed them immediately :-- O king! you better take your birth again in the human world and practise great meritorious deeds and come again to this Heaven. Thus saying, Brahm\=a looked at Gang\=a, who was attached to the king, and addressed her :-- ``You too better go to the human world and become his wife.'' Both of them, the king as well as Gang\=a, came out of Brahm\=a's abode, very much grieved in their hearts. The king Mah\=abhi\d{s}a thought of coming to this world and reflected on the kings thereof and settled to make the king Prat\={\i}pa of Puru's family his father. At this time the eight Vasus with their wives wandering in various places and enjoying as they liked came to the hermitage of Va\'sistha. Amongst the aforesaid eight Vasus Prithu and others, one Vasu Dyau's wife seeing Nandini, the sacrificial cow (K\=amadhenu) of Va\'sistha asked her husband :-- ``Whose is this excellent cow that I see? Vasu then replied as follows :-- `` O Beautiful one! This is Va\'sistha's cow. Whoever, be he a man or woman drinks her milk gets his longevity extended to ten thousand years and his youth never ends.'' Hearing this, the Vasu's wife said :-- ``There is a very beautiful comrade (Sakh\={\i}) of mine, the daughter of the Rajars\={\i}-U\'s\={\i}na in the world, of auspicious qualities. O Mah\=abh\=aga! Kindly bring to me from Va\'sistha's hermitage that auspicious sacrificial milch cow Nandini together with her calf that yields all desires; my Sakh\={\i} will then drink her milk and be thereby free from disease, old age and become the chief amongst all mankind. Hearing thus, his wife's word, the Vasu Dyau, though sinless, stole away together with Prithu and the other Vasus the cow Nandini in utter defiance to the self-controlled Muni Va\'sistha. When the cow Nandini had been stolen, the great ascetic Va\'sistha came quickly to the hermitage with abundance of fruits.

The ascetic Muni Va\'sistha, not finding, in his hermitage, his cow with her calf, searched for her in many forests and caves; but he, the son of Varu\d{n}a, could not find out his cow even after prolonged searches; he, then, took recourse to meditation and came to know that the Vasus had stolen the cow and became angry. He expressed :-- ``When the Vasus have stolen this my cow in utter defiance to my self, they must be born

amongst men.'' When the religious Varu\d{n}a's son Va\'sistha thus cursed the Vasus, they became very sorry and absent-minded; all of them went to Va\'sistha's hermitage and saw him there; they began to supplicate him as much as they could; and took refuge under him. Seeing the Vasus standing before him in an extremely distressed condition, the virtuous Muni Va\'sistha said :-- ``You all will be free from the curse within one year; but the Vasu Dyau will dwell amongst men for a long, long period as he had stolen direct my Nandini with her calf.''

40-60. While the Vasus, thus cursed, were returning, they saw on the way the chief river Gang\=a Dev\={\i} also cursed and therefore distressed; all of them bowed down to her simultaneously and said: ``O Dev\={\i}! A serious thought is troubling our minds, how can we, who live on nectar, take our birth in human wombs; so, O best river! You better be a woman and give birth to us. O Sinless one! You better be the wife of the sage King \'Santanu and no sooner we be born of your womb, kindly throw us in the river Gang\=a (your water). If you do thus, O Gang\=a we will certainly be freed of our curse.'' Gang\=a Dev\={\i} replied ``Well; that will be.'' Thus spoken, the Vasus went to their respective places; and Gang\=a Dev\={\i}, too, thinking on the subject again and again, went out of that place. At this time Mah\=abhi\d{s}a became born as a son of the king Prat\={\i}pa and became known as \'Santanu. He was exceedingly religious and true to his promise. One day while the King Prat\={\i}pa was praising the S\=urya Dev\={\i} (the sun) of unequalled energy, Gang\=a Dev\={\i} assumed an extraordinarily beautiful feminine form and came out of the waters and sat on the right thigh, resembling like a s\=al tree, of the king Prat\={\i}pa. The sage king Prat\={\i}p spoke out to the lady sitting on his right thigh, thus :-- ``O beautiful faced one! Why, unasked, have you sat on my auspicious right thigh?'' The lovely Gang\=a then replied :-- ``Hear why I have sat here. O best of Kurus! O king! Becoming attached to you, I have sat on your thigh; so please accept me.'' At this the king Prat\={\i}pa spoke to the beautiful lady, full of youth and beauty, ``I never go, simply out of passion to another's wife. There is another point; you have sat on my right thigh; that is the seat of sons and son's wives; so, when my desired son will be born, you will then, be my son's wife. And certainly, by your good will, my son will be born.'' The lady, of divine form, said, Well; that will be done! and went away. The king returned to his palace, thinking of the lady. After some time, he had a son born to him and when the son attained his teens, the king desired to lead a forest life and communicated this matter to his son. He said also, if the aforesaid beautifully smiling girl comes to you to marry, then marry her. And I am also ordering you not to question her anything ``who are you'' and so forth. If you take her as your legal wife, you will certainly be happy. Thus

saying to his son, the king Prat\={\i}pa handed over all his kingdom to his son and gladly retired into the forest. The king practised tapasy\=a in the forest and worshipped Ambik\=a; on quitting his mortal coil, he went by his sheer merit to the Heavens. The highly energetic king \'Santanu, on getting his kingdom, began to administer justice according to the laws of Dharma and governed his subjects.

Thus ends the third Chapter of the Second Skandha on the description of the curse on Gang\=a, Mah\=abhi\d{s}a and Vasus in the M\=ahapur\=a\d{n}am \'Sr\={\i} Mad Dev\={\i} Bh\=agavatam of 18,000 verses.



