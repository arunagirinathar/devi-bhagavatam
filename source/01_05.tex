\chapter{On the narrative of Hayagr\={\i}va}

1-4. The \d{R}i\d{s}is said :-- ``O S\=uta! Our minds are merged in the sea of doubt, hearing your this most wonderful saying, surprising to the whole world. The head of Jan\=ardan M\=adhava, the Lord of all, was severed out of His body! And He was afterwards known as Hayagr\={\i}va, the horse-faced! Oh! what more wonder can there be than this? Whom the Vedas even praise, all the Devas rest on Whom, Who is the Cause of all causes,

the \=Adi Deva Jagannath (the Lord of the universe), Oh! how is it that His head came to be severed! O highly intelligent one! Describe all this to me in detail''.

5-9. S\=uta said :-- O Munis! Hear all attentively the glorious deeds of the supremely energetic Vi\d{s}\d{n}u, the Deva of the Devas. Once on a time the eternal Deva Jan\=ardana became tired after the terrible continuous battle for ten thousand years. After this the Lord N\=ar\=ayana seated Himself on Padm\=asan (a kind of posture) in some lovely place on a level plot of ground and placing his head on the front of his bow with the bow strung and placed erect on the ground fell fast asleep. Vi\d{s}\d{n}u, the Lord of Ram\=a, was exceedingly tired and thus he fell soon into deep sleep. At this time Indra and the other Devas, with Brahm\=a and Mahe\d{s}\=a began a sacrifice.

10-13. Then they, for the sake of success in Deva's well, went to the region of Vaikuntha to meet with the Deva Jan\=ardana, the Lord of sacrifices. There the Devas, not finding Vi\d{s}\d{n}u, came to know by their Dhy\=an (meditation) where Bhagav\=an Vi\d{s}\d{n}u was staying and thither they went. They saw that the Lord Vi\d{s}\d{n}u, the Deva of the Devas was lying unconscious, being under the arms of Yoganidr\=a (the yogic sleep). Therefore they took their seats there. Seeing the Lord of the universe asleep, Brahm\=a, Rudra and the other Devas became anxious.

14-18. Indra then addressed the Devas :--``O best of the Suras! Now what is to be done! How shall we rouse Bhagav\=an from His sleep? Now think of the means by which this can be effected''. Hearing Indra's words \'Sambhu said :-- ``O good Devas! Now we must finish our sacrificial work. But if the sleep of Bhagav\=an be disturbed, He would get angry.'' Hearing \'Sankara's words, Paramesth\={\i} Brahm\=a created Vamr\={\i} insects (a sort of white ants) so that they might eat up the forepart of the bow that was lying on the ground causing the other end rise up and thus break His sleep. Thus the Deva's purpose will, no doubt, be fulfilled. Thus settling his mind, the eternal Deva Brahm\=a ordered the white ants Vamr\={\i}s to cut the bow string.

19-22. Hearing this order of Brahm\=a, Vamr\={\i} spoke to Brahm\=a, thus :-- ``O Brahm\=an! How can I disturb the sleep of the Devadeva, Lord of Lak\d{s}m\={\i}, the World Guru? To rouse one from one's deep sleep, to interrupt one in one's speech, to sever the love between a couple husband and wife, to separate a child from one's mother, all these are equivalent to Brahm\=ahaty\=a (murdering a Brahm\=an). Therefore, O Deva! how can I interrupt the happiuess of sleep of the Devadeva? And what benefit shall I derive by eating the bowstring, so that I may incur this vicious act? But a man can commit a sin if there be any interest of his; I am ready to eat this, if I get a personal interest''.

23-24. Brahm\=a said :-- We will give you, too, share in this our Yaj\~na (sacrifice); so hear me; do our work and rouse Vi\d{s}\d{n}u from His sleep. During the time of performing Homa whatever ghee will fall outside the Homa-Kund (the sacrificial pit) will fall to your share; so be quick and do this.

25-30. S\=uta said :-- Thus ordered by Brahm\=a, the Vamr\={\i} insect soon ate away the fore end of the bow that rested on the ground. Immediately the string gave way and the bow went up; the other end became free and a terrible sound took place. The Devas bcame afraid; the whole universe got agitated; the earth trembled. The sea became swollen; the aquatic animals became startled; violent wind blew; the mountains shook; ominous meteors fell. The quarters assumed a terrific aspect; the Sun went down the horizon. In that time of distress the Devas became anxious what evil might come down. O ascetics! while the Devas were thus cogitating, the head with crown on it of the Devadeva Vi\d{s}\d{n}u vanished away ; no body knew where it fell.

31-36. When the awful darkness disappeared, Brahm\=a and Mah\=adeva saw the disfigured body of Vi\d{s}\d{n}u with its head off. Seeing that headless figure of Vi\d{s}\d{n}u they were greatly surprised; they were drowned in the ocean of cares and, overwhelmed with grief, began to weep aloud. O Lord! O Master! O Devadeva! O Eternal one! what unforeseen extraordinary mishap occurred to us to-day! O Deva! Thou canst not be pierced nor cut asunder, nor capable of being burnt; how is it then that Thy head has been taken away! Is this the M\=ay\=a (majic) of some. Deva? O all pervading one! The Devas cannot live when Thy condition is thus; we do not know what affection dost Thou have towards us. We are crying because of our selfish ends; perhaps this therefore has occurred. The Daityas, Yak\d{s}as, or R\=akh\d{s}asas have not done this; O Lord of Laksm\={\i}! Whose fault will we ascribe this to? The Devas themselves have committed this loss to themselves?

37-41. O Lord of the Devas! The Devas are. now dependent! They are under Thee. Now where are we to go? What are we to do? There is none to save the dull stupid Devas!

At this juncture, seeing \'Siva and the other Devas crying, Brihaspati, supremely versed in the Vedas, consoled them thus :-- ``O highly fortunate one! what use there will be in thus crying and repenting? it ought you now to consider the means that you should adopt to redress your calamities. O Lord of the Devas! Fate and one's own exertion and intelligence are equal; if the success comes not through Fate (Luck or chance) one is certainly to show one's prowess and merit''.

42-46. Indra said :-- Fie to your exertion when, before our eyes, the head of Bhagav\=an Vi\d{s}\d{n}u Himself has been carried off! Fie, Fie to your prowess and intelligence! Fate is in my opinion, the supreme.

Brahm\=a said :-- Whatever, auspicious or inauspicious, is ordained Daiva (Fate), every one must bear that; no one can go beyond the Daiva. When one has taken up a body, one must experience pleasure and pain; there is no manner of doubt in this. See, in long-past days, by the irony of Fate, \'Sambhu severed my head; His generative organ, too, dropped down through curse. Similarly Hari's head has, to-day, fallen into the salt ocean. By the influence of time, Indra, the Lord of Sachi, had thousand genital marks over his body, was expelled from Heaven and had to live in the M\=anas sarovar in the lotuses and had to suffer many other miseries.

47-50. O Glorious ones! When such personages have suffered pains, then who else is there in the world, that dues not suffer! so you all cease sorrows and meditate on the Eternal Mah\=am\=ay\=a; who is the Mother of all, who is supporter of all, who is of the nature of Brahm\=avidy\=a (the Supreme Knowledge) and who is beyond the Gunas, who is the Prime Pr\=akriti, and who pervades the three Lokas, the whole universe, moving and unmoving; She will dispense our welfare. S\=uta said :-- Thus saying to the Devas, Brahm\=a ordered all the Vedas, that were incarnate there in their forms, for the successful issue of the Deva's work.

51-54. Brahm\=a said :-- ``OVedas! Now go on and chant hymns to the Sacred Highest Dev\={\i} Mah\=am\=ay\=a, who is Brahm\=avidy\=a, who brings all issues to their successful issues, who is hidden in all forms.'' Hearing His words, the all-beautiful Vedas began to chant hymns to Mah\=am\=ay\=a who can be comprehended by J\~n\=an, and who pervades the world.

The Vedas said :-- Obeisance to the Dev\={\i}! to the Mah\=am\=ay\=a! to the Auspicious One! to the Creatrix of the Universe! We bow down to Thee, who is beyond the Gunas, the Ruler of all the Beings! O Mother! Thou givest to \'Sankara even His desires. Thou art the receptacle of all the things; Thou art the Pr\=a\d{n}a of all the living beings; Thou art Buddhi, Lak\d{s}m\={\i} (wealth), \'Sobh\=a, K\'sham\=a (forgiveness), \'S\=anti (peace), Sraddh\=a (faith), Medh\=a (intellect), Dhriti (fortitude), and Smriti (recollection).

55. Thou art the vindu (m) over the Pr\=anava (om) and thou art of the nature of semi-moon; Thou art G\=ayattri, Thou art Vy\=arhiti; Thou art Jay\=a, Vijay\=a, Dh\=atri (the supportress), Lajj\=a (modesty), K\={\i}rti (fame), Ichch\=a (will) and Day\=a (mercy) in all beings.

56-57. O Mother! Thou art the merciful Mother of the three worlds; Thou art

the adorable auspicious Vidy\=a (knowledge) benefitting all the Lokas; Thou destroyest the Universe and Thou skilfully residest (hidden) in the V\={\i}ja mantras. Therefore we are praising Thee. O Mother! Brahm\=a, Vi\d{s}\d{n}u, Mahe\'svara, Indra, S\=urya, Fire, Sarasvat\={\i} and other Regents of the Universe are all Thy creation; so none of them is superior to Thee. Thou art the Mother of all the things, moving and non-moving.

58-61. O Mother ! When Thou dost will to create this visible Universe, Thou createst first Brahm\=a, Vi\d{s}\d{n}u and Mahe\'svara and makest them create, preserve and destroy this universe; but Thou remainest quite unattached to the world. Ever Thou remainest constant in Thy one form. No one in this Universe is able to know Thy nature; nor there is any body who can enumerate Thy names. How can he promise to jump across the illimitable ocean, who cannot jump across an ordinary well.

O Bhagavat\={\i}! No one amongst the Devas even knows particularly Thy endless power and glory. Thou art alone the Lady of the Universe and the Mother of the world.

62-68. The Vedas all bear testimony how thou alone hast created all this unreal and fleeting universe. O Dev\={\i}! Thou without any effort and having no desires hast become the cause of this visible world, thyself remaining unchanged. This is a great wonder. We cannot conceive this combination of contrary varieties in one. O Mother! How can we understand thy power, unknown to all the Vedas even, when thou thyself dost not know thy nature! We are bewildered at this. O Mother! It is that thou dost know nothing about the falling off of the Vi\d{s}\d{n}u's head! Or knowingly thou wanted to examine Vi\d{s}\d{n}u's prowess. Is it that Hari incurred any heinous sin. How can that be! Where is sin to thy followers who serve Thee! O Mother! Why art Thou so much indifferent to the Devas! It is a great wonder that the head of Vi\d{s}\d{n}u is severed! Really, we are merged in great misfortunes. Thou art clever in removing the sorrows of Thy devotees. Why art Thou delaying in fixing again the head on Vi\d{s}\d{n}u's body.

O Dev\={\i}! Is it that Thou taking offence on the gods hast cast that on Vi\d{s}\d{n}u! or was it that Vi\d{s}\d{n}u became proud and to curb that, Thou hast played thus! or is it that the Daityas, having suffered defeat from Vi\d{s}\d{n}u went and practised severe tapasya in some beautiful holy place, and have got some boons; and so Vi\d{s}\d{n}u's head has thus fallen off!

Or is it, O Bhagavat\={\i}! that Thou wert very eagerly interested to see Vi\d{s}\d{n}u's headless body and therefore Thou hast seen thus! O Prime Force! Is it that Thou art angry on the daughter of the Sindhu (ocean); Laksm\={\i} Dev\={\i}! Else, why hast Thou deprived Her of Her husband? Laksm\={\i} is born as a part of Thine; So Thou oughtest to forgive Her offence.

Therefore dost Thou gladden Her by giving back Her husband's life.

The principal Devas, engaged in Thy service, always make their Pr\=anams (bow down) to Thee; O Dev\={\i}! Beest Thou kind enough and make alive the Deva Vi\d{s}\d{n}u, the Lord of all and crossest us across this ocean of sorrows. O Mother! We cannot make out anything whatsoever where Hari's head has gone. We have no other protectress than Thee who canst give back His life? O Dev\={\i}! Dost Thou give life to the whole world as the nectar gives life to all the Devas.

69-73. S\=uta said :-- Thus praised by the Vedas with their Angas, with S\=amag\=anas (the songs from the S\=ama Veda), the Nirgu\d{n}\=a Mahe\'svari Dev\={\i} Mah\=am\=ay\=a became pleased. Then the auspicious voice came to them from the Heavens, gladdening all, and pleasing to the ears though no form was seen: ``O Suras! Do not care anything about it; you are immortal (what fear can you have?) Come to your senses. I am very much pleased by the praise sung by the Vedas. There is no doubt in this. Amongst men, whoever will read this My stotra with devotion, will get all what he desires. Whoever will hear this devotedly, during the three Sandhyas, will lie freed from troubles and become happy. When this stotra has been sung by the Vedas, it is equivalent to the Vedas.

74-75. Does anything take place in this world without any cause? Now hear why Hari's head was cut off. Once on a time, seeing the beautiful face of His dear wife Laksm\={\i} Dev\={\i}, Hari laughed in presence of Her.

76-82. At this Laksm\={\i} Dev\={\i} came to understand that ``He has seen surely something ugly in my face and therefore He laughed; otherwise why my Husband would laugh at seeing me. But what reason can there be as to see ugliness in my face after so long a time. And why shall He laugh without seeing something ugly, without any cause. Or it may be, He has made some other beautiful woman as my co-wife''. Thus arguing variously in her mind, Mah\=a Laksm\={\i} gradually got angry and Tamo guna slowly possessed Her. Then, by turn of Fate, in order that god's work might be completed, very fierce Tamas Sakti entered into Her body. She got very angry and slowly said :-- ``Let Thy head fall off''. Thus, owing to feminine nature and the destiny of Bhagvan, Laksm\={\i} cursed without any thought of good or bad, causing Her own suffering. By the T\=amas\={\i} \'Sakti possessing Her, she thought that a co-wife would be more painful than Her widowhood and thus She cursed Him.

83-86. Falsehood, vain boldness, craftiness, stupidity, impatience, over-greediness, impurity, and harshness are the natural qualities of women. Owing to that curse, the head of Vasudeva has fallen into the salt ocean. Now I will

fix the head on His body as before. O Sura Sattamas! There is another cause, also, regarding this affair. That will bring you great success. In ancient days a famous Daitya, named Hayagr\={\i}va practised severe tapasya on the bank of the Sarasvat\={\i} river.

87-92. Abandoning all sorts of enjoyments, with control over his senses and without any food, the Daitya did Japam of the (repeated) one syllabled M\=ay\=a-Vija-mantra and, meditating the form of the Utmost Sakti of Mine, adorned with all ornaments, practised very terrible austerities for one thousand years. I, too, went to the place of austerities in My T\=amas\={\i} form, meditated by the Daitya and appeared before him. There, seated on the lion's back, feeling compassion for his tapasya I spoke to him :-- ``O glorious One! O one of good vows! I have come to grant boon to Thee!'' Hearing the words of the Dev\={\i}, the Daitya instantly got up and falling down with devotion at Her feet, circumambulated Her. Looking at My form, his large eyes became cheerful with feelings of love and filled with tears; shedding tears, then, he began to chant hymns to Me.

93-95. Hayagr\={\i}va said :-- ``Obeisance to the Dev\={\i} Mah\=amaye! I bow down to Thee, the Creatrix, the Preserver, and the Destructrix of the universe! Skilled in shewing favour to Thy devotees! Giver of the devotee's desires! Obeisance to Thee! O Thou, the giver of liberation! O Thou! The auspicious one! I bow down to Thee. Thou art the cause of the five elements -- earth, water, fire, air, and Akasa! Thou art the cause of form, taste, smell, sound and touch. O Mahe\'svari! the five j\~n\=anendriyas (organs of perception) eyes, ears, nose, tongue, and skin and the five organs of action Karmendriyas :-- hands, feet, speech, arms, and the organ of generation are all created by Thee.

96-100. The Dev\={\i} said :-- ``O child! I am very much satisfied with your wonderful tapasya and devotion. Now say what boon do you want. I will give you the boon that you desire''. Hayagr\={\i}va said ;-- ``O Mother! grant me that boon by which death will not come to me, and I be invincible by the Suras and Asuras, I may be a Yogi and immortal''.

The Dev\={\i} said :-- `` Death brings in birth and birth brings in death; this is inevitable.'' This order of things is extant in this world; never its violation takes place. O best of the R\=ak\d{s}asas! Thus knowing death sure, think in your mind and ask another boon.

Hayagr\={\i}va said: -- ``O Mother of the universe! If it be that Thou art not willing at all to grant me immortality, then grant me this boon that my death may not occur from any other than from one who is horse-faced. Be merciful and grant me this boon that I desire.''

101-105. O highly fortunate one! ``Go home and govern your kingdom at your ease; death won't occur to you from any other beings then from one who is horse-faced.'' Thus granting the boon, the Dev\={\i} vanished. Becoming very glad on getting this boon, Hayagr\={\i}va went to his residence. Since then the wicked Daitya is troubling very much all the Devas and Munis. There is none in the three worlds to kill him. So let Visvakarm\=a take a horse's head and fix it on the headless body of Visnu. Then Bhagav\=an Hayagr\={\i}va will slay the vicious wicked Asura, for the good of the Deva\'s'.

106-112. S\=uta said :-- Thus speaking to the Devas, Bhagavat\={\i} \'Sarv\=an\={\i} remained silent. The Devas became very glad and spoke this to Visvakarm\=a :-- ``Kindly do this Deva work and fix Visnu's head. He will become Hayagr\={\i}va and kill the indomitable D\=anava.'' S\=uta said :-- Hearing these words, Visvakarm\=a quickly cut off with his axe, the head of a horse, brought it before the Devas and fixed it on the headless body of Visnu. By the grace of Mah\=am\=ay\=a, Bhagav\=an became horse-faced or Hayagr\={\i}va. Then, a few days after, Bhagav\=an. Hayagr\={\i}va killed that proud D\=anava, the Deva's enemy, by sheer force. Any man, hearing this excellent anecdote, becomes freed, certainly of all sorts of difficulties. Hearing or reading Mah\=am\=ay\=a's glorious deeds, pure and sin destroying, gives all sorts of wealth.

Thus ends the fifth chapter of the first Skandha on the description of the narrative of Hayagr\={\i}va in the Mah\=a Pur\=a\d{n}a \'Srimad Dev\={\i} Bh\=agavatam of 18,000 verses.



