\chapter{On the Glory of \'Sakti}

1-4. N\=arada said :-- ``O Bhagav\=an! I have heard all the anecdotes of Prakriti, as according to the \'S\=astras, that lead to the freedom from birth and death in this world. Now I want to hear the very secret history of \'Sr\={\i} R\=adh\=a and Durg\=a as described in the Vedas. Though you have told me about their glories, yet I am not satisfied. Verily, where is he whose heart does not melt away on hearing the glories of both of them! This world is originated from their parts and is being controlled by them. The devotion towards them frees one easily from the bonds of Sams\=ara (rounds of birth and death). O Muni! Kindly describe now about them.''

5-44. N\=ar\=aya\d{n}a said :-- O N\=arada! I am now describing the characters of R\=adh\=a and Durg\=a, as described in the Vedas; listen. I did not describe to anybody this Secret which is the Essence of all essences and Higher than the highest. This is to be kept very secret. Hearing this, one ought not to divulge it to any other body. R\=adh\=a presides over the Pr\=a\d{n}a and Durg\=a presides over the Buddhi. From these two, the M\=ulaprakriti has originated this world. These two \'Saktis guide the whole world. From the Mah\=avir\=at to the small insect, all, moving or non-moving, are under the M\=ulaprakriti. One must satisfy them. Unless these two be satisfied, Mukti cannot be obtained.

Therefore one ought to serve M\=ula Prakriti for Her satisfaction. Now of the two in M\=ula Prakriti, I will describe fully the R\=adh\=a Mantra. Listen. Brahm\=a, Vi\d{s}\d{n}u, and others always worship this mantra. The principal mantra is ``\'Sr\={\i} R\=adh\=ayai Sv\=ah\=a.'' By this six lettered mantra Dharma and other fruits all are obtained with ease. If to this six lettered M\=ula mantra Hr\={\i}m be added, it yields gems and jewels as desired. So much so, if thousand koti mouths and one hundred koti tongues are obtained, the glory of this mantra cannot be described. When the incorporeal voice of M\=ula Prakriti was heard in the Heavens, this mantra was obtained, first by Kri\d{s}\d{n}a in the R\=asa Mandalam in the region of Goloka where all love sentiments are played. (The Vedas declare him as Raso vai Sah). From Kri\d{s}\d{n}a, Vi\d{s}\d{n}u got the Mantra; from Vi\d{s}\d{n}u, Brahm\=a got; from Brahm\=a

Vir\=at got, from Vir\=at, Dharma, and from Dharma I have got this Mantra. Repeating that Mantra, I am known by the name of \d{R}i\d{s}i. Brahm\=a and the other Devas meditate always on the M\=ula Prakriti with greatest joy and ecstasy. Without the worship of R\=adh\=a, never can the worship of \'Sr\={\i} Kri\d{s}\d{n}a be done. So men, devoted to Vi\d{s}\d{n}u, should first of all worship R\=adh\=a by all means. R\=adh\=a is the Presiding Deity of the Pr\=a\d{n}a of \'Sr\={\i} Kri\d{s}\d{n}a. Hence \'Sr\={\i} Kri\d{s}\d{n}a is so much subject to R\=adh\=a. The Lady of the R\=asa Mandalam remains always close to Him. Without Her \'Sr\={\i} Kri\d{s}\d{n}a could not live even for a moment. The name R\=adh\=a is derived from ``R\=adhnoti'' or fulfills all desires. Hence M\=ula Prakriti is termed R\=adh\=a. I am the \d{R}i\d{s}i of all the mantras but the Durg\=a Mantra mentioned in this Ninth Skandha. G\=ayatr\={\i} is the chhanda (mantra) of those mantras and R\=adhik\=a is the Devat\=a of them. Really, N\=ar\=aya\d{n}a is the \d{R}i\d{s}i of all the mantras; G\=ayatr\={\i} is the chhanda; Pr\=a\d{n}ava (om) is the V\={\i}ja (seed) and Bhuvane\'svar\={\i} (the Directrix of the world) is the \'Sakti. First of all the principal mantra is to be repeated six times; then meditation of the great Dev\={\i} R\=adhik\=a, the \'Sakti of the \d{R}i\d{s}is is to be done, as mentioned in the S\=ama Veda. The meditation of R\=adh\=a is as follows :-- O Dev\={\i} Radhike! Thy colour is like white Champaka flower; Thy face is like the autumnal Full Moon; Thy body shines with the splendour of ten million moons, Thy eyes look beautiful like autumnal lotus; Thy lips are red like Bimba fruits, Thy loins are very heavy and decked with the girdle (K\=anch\={\i}) ornament; Thy face is always gracious with sweet smiles; Thy breasts defy the frontal globe of an elephant. Thou art ever youthful as if twelve years old; Thy body is adorned all over with ornaments! Thou art the waves of the ocean of \'Sring\=ara (love sentiments.) Thou art ever ready to show Thy grace to the devotees; on Thy braid of hair garlands of Mallik\=a and M\=alat\={\i} are shining; Thy body is like a creeping plant, very gentle and tender; Thou art seated in the middle of R\=asa Mandalam as the Chief Directrix; Thy one hand is ready to grant boons and another hand expresses ``Have no fear.'' Thou art of a peaceful appearance; Thou art ever youthful; Thou art seated on a jewel throne; Thou art the foremost guide of the Gop\={\i}k\=as; Thou art dearer to Kri\d{s}\d{n}a than even His life; O Parame'svar\={\i}! The Vedas reveal Thy nature. Meditating thus, one is to bathe the Dev\={\i} on a \'S\=alagr\=ama stone, jar, yantra or the eight petalled lotus and then worship Her duly. First the Dev\={\i} is to be invoked; then P\=adya and \=Asana, etc., are to be offered, the principal Mantra being pronounced at every time an offering

is given. After giving water for washing both the feet, Arghya is to be placed on the head and \=Achaman\={\i}yam water to be offered three times on the face. Madhuparka (an oblation of honey, milk etc.) and a cow giving a good quantity of milk are next to be offered. Then the yantra is to be thought of as the bathing place where the Dev\={\i} is to be bathed. Then Her body is to be wiped and a fresh cloth given for putting on. Sandalpaste and various other ornaments are next to be given. Various garlands of flowers with Tulas\={\i} Manjari (flower stalks) P\=arij\=ata flower and Satapatra etc., then, are to be offered. Then within the eight petals, the family members of the Dev\={\i} are to be thought of; worship is next to be offered in the right hand direction (with the hands of the watch). First of all, M\=al\=avat\={\i} on the petal in front of (on the east) the Dev\={\i}, then M\=adhav\={\i} on the southeast corner, then Ratnam\=al\=a on the south, Su\'s\={\i}l\=a on the south-west, Sa\'sikal\=a on the west, P\=arij\=ata on the north-west, Par\=avat\={\i} on the north and the benefactious Sundar\={\i} on the north-east corner are to be worshipped in order. Outside this, Br\=ahm\={\i} and the other M\=atrik\=as are to be worshipped and on the Bh\=up\=uras (the entrances of the yantra,) the Regents of the quarters, the Dikp\=alas and the weapons of the Dev\={\i}, thunderbolt, etc., are to be worshipped. Then all the attendant Deities of the Dev\={\i} are to be worshipped with scents and various other articles. Thus finishing the worship, one should chant the Stotra (hymns) named Sahasra-n\=ama (thousand names) Stotra with care and devotion. O N\=arada! The intelligent man who worships thus the R\=ase\'svar\={\i} Dev\={\i} R\=adh\=a, becomes like Vi\d{s}\d{n}u and goes to the Goloka.

He who performs the birthday anniversary of \'Sr\={\i} R\=adh\=a on the Full-Moon day of the month of K\=artik, gets the blessings of \'Sr\={\i} R\=adh\=a who remains near to him. For some reason R\=adh\=a, the dweller in Goloka was born in Brind\=aban as the daughter of Vri\d{s}av\=anu. However, according to the number of letters of the mantras that are mentioned in this chapter, Pura\d{s}chara\d{n}a is to be made and Homa, one-tenth of Pura\d{s}chara\d{n}a, is to be then performed. The Homa is to be done with ghee, honey, and milk; the three sweet things mixed with Til and with devotion.

45. N\=arada said :-- ``O Bhagav\=an; Now describe the Stotra (hymn) Mantra by which the Dev\={\i} is pleased.''

46-100. N\=ar\=aya\d{n}a said :-- O N\=arada! Now I am saying the R\=adh\=a Stotra. Listen. O Thou, the Highest Deity, the Dweller in R\=asa Mandalam! I bow down to Thee; O Thou, the Chief Directrix of the R\=asa Mandalam; O Thou dearer to Kri\d{s}\d{n}a than His life even, I bow down to Thee. O Thou, the Mother of the three Lokas! O Thou the Ocean of

mercy! Be pleased. Brahm\=a, Vi\d{s}\d{n}u and the other Devas bow down before Thy lotus feet. Thou art Sarasvat\={\i}; Thou art Savitr\={\i}; Thou art \'Sankar\={\i}; I bow down to Thee; Thou art Gang\=a; Thou art Padm\=avat\={\i}; Thou art Sasth\={\i}; Thou art Mangal\=a Chandik\=a; Thou art Manas\=a; Thou art Tulas\={\i}; Thou art Durg\=a; Thou art Bhagavat\={\i}; Thou art Lak\d{s}m\={\i}; Thou art all, I bow down to Thee. Thou art the M\=ula Prakriti; Thou art the Ocean of mercy. Obeisance to Thee! Be merciful to us and save us from this ocean of Sams\=ara (round of birth and death). O N\=arada! Anybody who remembers R\=adh\=a and reads this Stotra three times a day does not feel the want of anything in this world. He will ultimately go to Goloka and remain in the R\=asa Mandalam. O Child This great secret ought never to be given out to any. Now I am telling you the method of worship of the Durg\=a Dev\={\i}. Hear. When any one remembers Durg\=a in this world, all his difficulties and troubles are removed. It is not seen that anybody does not remember Durg\=a. She is the object of worship of all. She is the Mother of all and the Wonderful \'Sakti of Mah\=adeva. She is the Presiding Deity of the intellect (Buddhi) of all and She controls the hearts of all and She removes the great difficulties and dangers of all. Therefore She is named Durg\=a in the world. She is worshipped by all, whether a \'Saiva or a Vai\d{s}\d{n}ava. She is the M\=ula Prakriti and from Her the creation, preservation and destruction of the universe proceed. O N\=arada! Now I am saying the principal nine lettered Durg\=a Mantra, the best of all the Mantras. ``Aim Hr\={\i}m Kl\={\i}m Ch\=amund\=ayai Vichche'' is the nine lettered V\={\i}ja mantra of \'Sr\={\i} Durg\=a; it is like a Kalpa Vrik\d{s}a yielding all desires. One should worship this mantra by all means. Brahm\=a, Vi\d{s}\d{n}u, and Mahe\'sa are the \d{R}i\d{s}is of this mantra; G\=ayatr\={\i}, U\d{s}\d{n}ik and Anusthubha are the chhandas; Mah\=ak\=al\={\i}, Mah\=a Lak\d{s}m\={\i} and Sarasvat\={\i} are the Devat\=as; Rakta Dantik\=a, Durg\=a, and Bhr\=amar\={\i} are the V\={\i}jas. Nand\=a, S\=akambhar\={\i}, and Bh\={\i}m\=a are the \'Saktis and Dharma (Virtue), Artha (wealth) and K\=ama (desires), are the places of application (Viniyoga). Assign the head to the \d{R}i\d{s}i of the mantra (Ny\=asa); assign the chhandas to the mouth and assign the Devat\=a to the heart. Then assign the \'Sakti to the right breast for the success and assign the V\={\i}ja to the left breast.

Then perform the Sadamga Ny\=asa as follows :-- Aim Hriday\=aya namah, Hr\={\i}m \'Sirase Sv\=ah\=a, Kl\={\i}m \'Sikh\=ay\=am Va\d{s}at, Ch\=amund\=ayai Kavach\=aya Hum, Vichche Netr\=abhy\=am Vau\d{s}at, ``Aim Hr\={\i}m Kl\={\i}m Ch\=amund\=ayai Vichche'' Karatalapri\d{s}ih\=abhy\=am Phat. Next say touching the corresponding parts of the body :-- ``Aim namah \'Sikh\=ay\=am, Hr\={\i}m Namah'' on the right eye; ``Kl\={\i}m Namah'' on the left eye, ``Ch\=am Namah'' on

the right ear, ``Mum namah'' on the left ear, ndam Namah'' on the nostrils; ``Vim Namah'' on the face; ``Chchem Namah'' on the anus and finally ``Aim Hr\={\i}m Kl\={\i}m Ch\=amund\=ayai Vichche'' on the whole body. Then do the meditation (dhy\=an) thus :-- ``O Ch\=amunde! Thou art holding in Thy ten hands ten weapons, viz., Khadga (axe), Chakra (disc), Gad\=a (club), V\=a\d{n}a (arrows), Ch\=apa (bow), Parigha, S\=ula (spear), Bh\=u\'sund\={\i} Kap\=ala, and Khadga. Thou art Mah\=a K\=al\={\i}; Thou art three-eyed; Thou art decked with various ornaments. Thou shinest like Lil\=anjan (a kind of black pigment). Thou hast ten faces and ten feet. The Lotus born Brahm\=a chanted hymns to Thee for the destruction of Madhu Kaitabha; I bow down to Thee.'' Thus one should meditate on Mah\=a K\=al\={\i}, of the nature of K\=amav\={\i}ja (the source whence will comes). Then the Dhy\=anam of Mah\=a Laksm\={\i} runs as follows :-- O Mah\=a Lak\d{s}m\={\i}, the destroyer of Mahi\d{s}\=asura! Thou holdest the garland of Ak\d{s}a (a kind of seed), Para\'su (a kind of axe), Gad\=a (club), I\d{s}u (arrows), Kuli\'sa (the thunderbolt) Padm\=a (Lotus), Dhanu (bow), Ku\d{n}dik\=a (a student's waterpot), Kama\d{n}dalu, Danda (rod for punishment), \'Sakti (a kind of weapon), Asi (sword), Charma (shield) Padm\=a (a kind of waterlily), Ghant\=a (bell,) \'Sur\=ap\=atra (a pot to hold liquor), \'S\=ula (pickaxe), P\=a\'sa (noose) and Sudar\d{s}ana (a kind of weapon). Thy colour is of the Rising Sun. Thou art seated on the red Lotus. Thou art of the nature M\=ay\=av\={\i}ja (the source whence female energy comes). So Obeisance to Thee! (The V\={\i}ja and the Dev\={\i} are one and identical). Next comes the Dhy\=anam of Mah\=a Sarasvat\={\i} as follows :-- O Mah\=a Sarasvat\={\i}! Thou holdest bell, pickaxe, plough (Hala), Conch shell, Mu\d{s}ala (a kind of club), Sudar\'sana, bow and arrows. Thy colour is like Kunda flower; Thou art the destroyer of \'Sumbha and the other Daityas; Thou art of the nature of V\=a\d{n}\={\i}v\={\i}ja (the source whence knowledge, speech comes). Thy body is filled with everlasting existence, intelligence and bliss. Obeisance to Thee! O N\=arada! Now I am going to say on the Yantra of Mah\=a Sarasvat\={\i}. Listen. First draw a triangle. Draw inside the triangle eight petalled lotus having twenty-four leaves. Within this draw the house. Then on the Yantra thus drawn, or in the \'S\=alagr\=ama stone, or in the jar, or in image, or in the V\=a\d{n}ali\d{n}gam, or on the Sun, one should worship the Dev\={\i} with oneness of heart. Then worship the P\={\i}tha, the deities seated also on the dais, i.e., Jay\=a, V\={\i}jay\=a, Ajit\=a, Aghor\=a, Mangal\=a and other P\={\i}tha \'Saktis. Then worship the attendant deities called \=Avara\d{n}a P\=uj\=a :-- Brahm\=a with Sarasvat\={\i} on the east, N\=ar\=aya\d{n}a with Lak\d{s}m\={\i} on the Nairirit corner, \'Sankara with P\=arvat\={\i} on the V\=ayu corner, the Lion on the north of the Dev\={\i}, and Mah\=asura on the left side of the Dev\={\i}; finally worship Mahi\d{s}a (buffalo). Next worship

Nandaj\=a, Raktadant\=a, \'Sakambhar\={\i}, \'Siv\=a, Durg\=a, Bh\={\i}m\=a, and Bhr\=amar\={\i}. Then on the eight petals worship Brahm\=a, Mahe\'svar\={\i}, Kaum\=ar\={\i}, Vai\d{s}\d{n}av\={\i}, V\=ar\=ah\={\i}, N\=ara Simh\={\i}, Aindr\={\i}, and Ch\=amund\=a. Next commencing from the leaf in front of the Dev\={\i}, worship on the twenty four leaves Vi\d{s}\d{n}u M\=ay\=a, Chetan\=a, Buddhi, Nidr\=a (sleep), hunger, shadow, \'Sakti, thirst, peace, species (J\=ati), modesty, faith, fame, Lak\d{s}m\={\i} (wealth), fortitude, Vriti, \'Sruti, memory, mercy, Tusti, Pusti (nourishment), Bhr\=anti (error) and other M\=atrik\=as. Next on the corners of the Bh\=upura (gates of the Yantra), Gane\'sa K\d{s}ettrap\=alas, Vatuka and Yogin\={\i}s are to be worshipped. Then on the outside of that Indra and the other Devas furnished with weapons are to be worshipped as per the aforesaid rules. For the satisfaction of the World-Mother various nice offerings and articles like those given by the royal personages are to be presented to the Mother; then the mantra is to be repeated, understanding its exoteric and esoteric meanings. Then Sapta\'sati stitra (Chand\={\i} p\=atha) is to be repeated before the Dev\={\i}. There is no other stotra like this in the three worlds. Thus Durg\=a, the Deity of the Devas, is to be appeased every day. He who does this gets within his easy reach Dharma, Artha, K\=ama, and Mok\d{s}a, the four main objects of human pursuits (virtue, wealth, enjoyment and final beatitude). O N\=arada! Thus I have described to you the method of worship of the Dev\={\i} Durg\=a. People get by this what they want. Hari, Brahm\=a, and all the Devas, Manus, Munis, the Yog\={\i}s full of knowledge, the \=A\'sram\={\i}s, and Lak\d{s}m\={\i} and the other Devas all meditate on \'Siv\=an\={\i}. One's birth is attained with success at the remembrance of Durg\=a. The fourteen Manus have got their Manuship and the Devas their own rights by meditating on the lotus feet of Durg\=a. O N\=arada! Thus I have described to you the very hidden histories of the Five Prakritis and their parts. Then, verily, the four objects of human pursuits Dharma, Artha, K\=ama and Mok\d{s}a are obtained by hearing this. He who has no sons gets sons, who has no learning gets learning and whoever wants anything gets that if he hears this. The Dev\={\i} Jagaddh\=atr\={\i} becomes certainly pleased with him who reads with his mind concentrated on this for nine nights before the Dev\={\i}. The Dev\={\i} becomes obedient to him who daily reads one chapter of this Ninth Skandha and the reader also does what is acceptable to the Dev\={\i}. To ascertain before-hand what effects, merits or demerits, would accrue from reading this Bh\=agavata, it is necessary by examining through the hands of a virgin girl or a Br\=ahmi\d{n} child, the auspicious or inauspicious signs. First make a Sankalpa (resolve) and worship the book. Then bow down again and again to the Dev\={\i} Durg\=a. Then bring there a virgin girl, bathed well and worship her duly and have a golden pencil fixed duly in her

hand and placed in the middle on the body. Then calculate the auspicious or inauspicious effects, as the case may be, from the curves made by that pencil. So the effects of reading this Bh\=agavata would be. If the virgin girl be indifferent in fixing the pencil within the area drawn, know the result of reading the Bh\=agavata would be similar. There is no doubt in this.

Here ends the Fiftieth Chapter of the Ninth Book on the Glory of \'Sakti in the Mah\=apur\=a\d{n}am \'Sr\={\i}mad Dev\={\i} Bh\=agavatam of 18,000 verses by Mahar\d{s}i Veda Vy\=asa.

Here ends the Ninth Book.

The Ninth Book Completed.



