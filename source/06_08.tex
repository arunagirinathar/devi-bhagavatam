\chapter{On \'Sach\={\i}'s praising the Dev\={\i}}

1-11. Vy\=asa said :-- O King! Hearing that the wife of Indra had taken refuge under Brihaspati, the King Nahu\d{s}a became very angry towards Brihaspati and spoke to the Devas :-- `` O Devas! I hear that the stupid son of Angirasa has given protection to Indra and has kept her in his house; I will therefore kill him quickly.'' Seeing the terrible Nahu\d{s}a thus angry, the Devas and \d{R}i\d{s}is consoled him and said :-- O King of kings! Do not be angry; quit this vicious motive yours. See, the \d{R}i\d{s}is in all the Dharma \'S\=astras, have declared the holding of illicit connection with other's wives as a very heinous crime and have blamed it very much. You can consider that the daughter of Pulom\=a is always chaste, devoted to her husband and very good-natured. How can she, when her husband is alive, take another husband? O Lord! You are now the Lord of the three worlds and hence the Defender of Faith and Religion; and if a person like you act irreligiously, all the subjects will then go to annihilation. One who is a Lord should always observe the rules of good conduct. Besides there are many

other celestial women in this Heaven as beautiful as \'Sach\={\i}; you can satisfy your thirst with them. Mutual love is recognised by the wise as the true originator of amorous dealings; ravishing a woman by force destroys all amorous sentiments. O King! And if the mutual love be similar and equal in all respects, then comes the true happiness; you have now got the post of Indra; therefore quit this idea of holding illicit connection with other's wives and indulge in other good thoughts. Demerits destroy prosperity and merits increase it. Therefore, O King! Leave all these bad thoughts and make your heart take a good turn and be happy.

12-15. Nahu\d{s}a said :-- ``O Devas! Where were you all when Indra stole away the wife of Gautama and when the Moon stole away the wife of Brihaspati? It is easy to give advice to others but to act according to that is very rare in this world. O Devas! Let the qualified Dev\={\i} come to me you will derive much benefit from it and the Dev\={\i}, too, will get Her highest happiness; there is no doubt in this. I tell you truly that in no other way I will be satisfied; bring Indr\=an\={\i} here quickly, whether by good words or by force.''

16-17. The Devas and Munis heard the words of the king Nahu\d{s}a, smitten by the Cupid's arrows, got terrified and said :-- ``We will bring Indr\=an\={\i} to you by gentle words.'' Saying thus, they went to the house of Brihaspati.

18-21. Vy\=asa said :-- O King! The Devas, going to the house of Brihaspati, spoke thus with folded hands :-- O Guru! We know that Indr\=an\={\i} has taken shelter in your house; we will have to hand her over today to the king Nahu\d{s}a for we all united have made over the post of Indra to Nahu\d{s}a. Let this beautiful Lady now choose and worship him. Hearing these awful words of the Devas, Brihaspati said to them :-- ``O Devas! This chaste woman, devoted to her husband, has now taken my shelter; therefore I can never part with her.'' The Devas said :-- ``O Guru! Kindly advise then - if you do not part with \'Sach\={\i} Dev\={\i} - how the king Nahu\d{s}a be pleased; if he becomes angry, it will then be very difficult to please him.''

22-31. Brihaspati said :-- ``O Devas! Let \'Sach\={\i} now go to Nahu\d{s}a, and tempt him with enticing words and make this condition that when her husband's death will be known to her, she will then accept Nahu\d{s}a as her husband. How could she accept another husband when her husband was alive. Therefore let her now go in quest of her high-souled husband. Let \'Sach\={\i} thus make condition with him and, thus deceiving him, let her try

her best to bring back her husband. O King! Then, after coming to this conclusion, Brihaspati and other Devas went with Indr\=an\={\i} to the king Nahu\d{s}a. Seeing them come, especially looking at Indr\=an\={\i} the artificial king Nahu\d{s}a became very glad and said to Indr\=an\={\i} :-- ``O Beloved! Today I am become the real Indra. O beautiful-eyed One! Worship me as your husband; see the Devas now have made me to be worshipped by all the gods.'' When Nahu\d{s}a spoke thus, the Dev\={\i} \'Sach\={\i} became filled with great shame; she began to tremble and said to the king :-- ``O Lord of the Devas! I desire to ask a boon from you. Better wait till I ascertain whether Indra is dead or alive, there is this doubt in my heart whether he lives or whether he is dead. O King of kings! Let me, first of all, clear my doubts. Kindly excuse me and wait till then. I tell this truly that after I ascertain the fact, I will worship you. I do not know anything whether Indra is dead or whether he has gone any where else.'' When \'Sach\={\i} Dev\={\i} spoke thus, Nahu\d{s}a became very glad and saying ``let it be so'' dismissed her.

32-47. Thus having received permission from the King to depart, \'Sach\={\i} hurriedly went to the Devas and spoke to them to try their best to bring Indra back as soon as possible. O King! Hearing these sweet and holy words of Indr\=an\={\i} the Devas intently consulted with each other how they could get back Indra. They then went to Vaikuntha and began to praise with hymns the original Deva, the God Vi\d{s}\d{n}u, the Lord of the Universe, kind to those that seek His refuge. The Devas, skilled in speaking, spoke to Vi\d{s}\d{n}u with a very troubled heart :-- ``O Lord! Indra, the Lord of the Devas, is very much troubled with his sin Brahmahatty\=a. Where is he staying now, invisible to all the beings? O Lord! He is now overcome with the sin Brahmahatty\=a by killing Vritra, the best of the Br\=ahmi\d{n}s. We ask your skilful and intelligent advice. O Lord! You are the sole refuge of him as well as of us. We are now involved in a great difficulty. Kindly show us the way how we, as well as Indra, can get out of this difficult crisis.'' Hearing the pitiful words of the Devas, Vi\d{s}\d{n}u said :-- Let Indra perform the A\'svamedha sacrifice (Horse sacrifice) for the purification of his sins. By this Yaj\~na, that can destroy all sins, Indra will be purified and he will regain his Indraship; there is no doubt in this. The more so because the Dev\={\i}, the Universal Mother, will be pleased with his Horse sacrifice and will destroy all his sins, Brahmahatty\=a and others. Lo! Merely remembering Her destroys heaps of sins; and, if by this Horse sacrifice, She be pleased, what wonder is there that sins of a more grave nature would be destroyed! And let Indr\=an\={\i} worship Bhagavat\={\i} daily; happiness will undoubtedly be gained by worshipping that most Auspicious One! By this the King Nahu\d{s}a will be particularly deluded by the World

Mother and will then be quickly destroyed by the sin committed by himself. And Indra, purified by A\'svamedha, will soon regain his position and all his wealth. O king! Thus hearing the sweet beneficial words of Vi\d{s}\d{n}u of indomitable prowess, the Devas went to the spot, where resided Indra. Brihaspati and the other Devas consoled the distressed Indra and made him celebrate duly in right order, the Horse sacrifice the greatest of all sacrifices. Indra then distributed his sin Brahmahatty\=a amongst the trees, rivers, mountains, women, and the earth.

48-51. Thus casting aside his sin on all the above things, Indra became again free from his sin, and, getting rid of his fever and uneasiness, abided by the time and remained there invisible in the tubular stem of the lotus. Doing that wonderful act, the Devas started from there and reached their own abodes. The daughter of Pulom\=a, suffering from her bereavements from Indra, spoke then to Brihaspati with great sorrow :-- ``O Lord! Why is my husband still invisible to me, when he has performed the A\'svamedha sacrifice? Kindly show me the way how I can get a sight of him.''

52-62. Brihaspati said :-- ``O Dev\={\i}! Worship the most Auspicious Bhagavat\={\i}; surely She will make your husband sinless and you will see him. The Dev\={\i} Ambik\=a, the Upholdress of the Universe, will desist the King Nahu\d{s}a from doing the wrongful act and it is She that will delude him by Her M\=ay\=a and get his downfall from the Heavens. O King! When Brihaspati spoke thus, \'Sach\={\i} Dev\={\i} got initiated by him in the Dev\={\i} Mantram, capable to secure success in any undertaking. Thus getting the Mantram from her Guru, She began to worship the Dev\={\i} Bhuvane'svar\={\i} duly with flowers, sacrificial victims and other necessary articles for worship. Thus Indr\=an\={\i}, with a view to see her husband, performed the worship of the Dev\={\i}; she quitted all the articles of enjoyment and luxury and assumed the garb of an ascetic; thus some time passed away, when the Dev\={\i} was pleased and appeared before her on the back of a Swan, in Her peaceful form, ready to grant boons to Indr\=an\={\i}. She looked, then, fiery like thousands of Moons; Her lovely beauty appeared in rays like thousands and thousands of fixed lightnings. The four Vedas personified began to praise Her in hymns from the four sides. Her two hands were adorned with a noose and a goad, and Her two other hands made signs to grant boons and to discard all fear. The Vaijayant\={\i} garland of clear crystal-like gems suspended from Her neck up to Her feet. Her face was adorned with smiles and signs as if she would grant favours. She had three eyes and was the ocean of mercy and the Mother of all the J\={\i}vas from a worm up to Brahm\=a. Her two heavy breasts were filled with

unbounded ocean of nectar-like juice of Peace and Mukti. She was the Goddess of innumerable worlds, the Goddess of all and the Highest, endowed with all the knowledges and the Incarnate of the Undecaying and Immoveable Brahm\=a. The Dev\={\i}, then, began to address \'Sach\={\i}, the wife of Indra, in pleasant words and in voice deep like a rolling thunder.

63-69. The Dev\={\i} said :-- O Darling to Indra! Better now ask your desired boon. I am much pleased with your worship. O Beautiful One! I have come here to grant you boon. To see Me is not an easy task; by the collected merits, acquired in thousands and thousands of births one is able to See Me. Hearing the words of the Dev\={\i}, \'Sach\={\i} Dev\={\i}, the wife of Indra, fell prostrate before Her feet and began to speak to the Highest Goddess, the Bhagavat\={\i}, Who seemed graciously pleased :-- ``O Mother! I now desire from Thee, that I may see my husband whom I attained after great difficulty, that I be freed from the fear arising out of King Nahu\d{s}a and I want that Indra be reinstated as Indra as he was before.'' The Dev\={\i} said :-- ``O Lady of the Devas! Better go with this My messenger (D\=ut\={\i}) to M\=anasarovara; there is installed My fixed form, named Vi\'svak\=am\=a. You will see your Indra staying there very sorrowful and overwhelmed with terror. I will delude the King Nahu\d{s}a within a very short period. O large-eyed One! Be calm and quiet; I will fulfil your desires; soon I will delude that king and deprive him of the seat of Indra.''

70-71. Vy\=asa said :-- The wife of Indra accompanied the messenger of the Dev\={\i} and quickly reached the presence of her husband Indra. She was very pleased to see her long-wished for husband, in the state disguise.

Here ends the Eighth Chapter of the Sixth Book on the praising of the Bhagavat\={\i} by the wife of Indra and on getting the sight of Indra in the Mah\=apur\=a\d{n}am, \'Sr\={\i} Mad Dev\={\i} Bh\=agavatam of 18,000 verses by Mahar\d{s}i Veda Vy\=asa.



