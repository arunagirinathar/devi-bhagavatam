\chapter{On the Daitya armies getting ready}

1-14. Vy\=asa said :-- The very powerful Asura Mahi\d{s}a, puffed up with vanity on his getting the boon, obtained sovereignty and brought the whole world under his control! He, being the paramount power, began to protect the sea-girt earth acquired by the power of his own arms, over which he had the sole sovereignty, there being no other rival king nor any cause of the slightest fear. His Commander-in-Chief was then the very powerful Chik\d{s}ura, maddened with pride; and T\=amra was in charge of the Royal Treasury, guarded by many soldiers. There were, then, many generals Asilom\=a, Vid\=ala, Udarka, V\=askala, Trinetra, K\=ala, Bandhaka and others, very proud, and each in charge of his own corps respectively and occupying this sea-girt earth. O king! The powerful kings that reigned before were made subservient and tributary; and those, that fought valiantly befitting the K\d{s}attriya line, were slain by Mahi\d{s}a. The Br\=ahma\d{n}as over the earth became subservient to Mahi\d{s}a and gave their Yaj\~na offerings to him. When that Mahi\d{s}\=asura got the sole sovereign sway of this world, he, proud of his boons, desired to conquer the Heavens. Then Mahi\d{s}a, the Lord of the Daityas, desirous to send an envoy to Indra, the Lord of \'Sach\={\i}, instantly called for the messenger and spoke to him thus :-- Go, O hero! O valiant one! to Heaven. Act as my messenger and tell Indra fearlessly thus :-- ``O thousand-eyed one! Quit the Heavens; go anywhere you like, or offer your service to the high-souled Mahi\d{s}a! He is the lord; and if you take refuge unto him, he will certainly protect you. Therefore, O Lord of \'Sach\={\i}, better seek the protection of Mahi\d{s}a. If, O Balas\=udana! Not willing, wield your Vajra at once; we know your powers; you were, in days of yore, conquered by our ancestors. O chief of the S\=uras! You are the paramour of Ahaly\=a; your strength is well known, give battle or go anywhere you like.''

15-21. Vy\=asa said :-- Hearing the messenger's words, Indra became very indignant and laughed and said :-- I did not know, O you stupid, that you were maddened with vanity; I will shortly give medicines for your master's disease. Now I will extirpate him by the roots; wise persons do not slay messengers; I therefore let you go. Better go and tell him what I say :-- ``Son of a buffalo! If you are willing to fight, better come and do not delay. O Enemy of horse! (Buffaloes and horses are always at war with each other) Your strength is well known to me; you are a grass eater and your appearance is stupid, idiotic; out of your horns I will make a good bow. You depend on your horns for your strength;

that I well know. You are clever in striking with your horns; you don't know anything about warfare; therefore I will out off your both the weapons and render you powerless. You are very much puffed up with vanity due to that.''

22. Vy\=asa said :-- Indra having spoken thus, the messenger quickly returned to his haughty master Mahi\d{s}a and saluting, spoke :--

23-28. The messenger said :-- Indra counts you not even a fig, as he is surrounded by his Deva forces and considers himself quite sufficient. It ought one's servant to speak true and pleasant before one's master; how can I utter the words before my master, that are spoken by that brute Indra. Whereas the well known maxim reigns in my mind withal that I am your well-wishing servant and I ought to speak truth before you, my master, and that truth is to be pleasant to hear also. If pleasant words I speak only, then I fail in my duty; at the same time, harsh words ought not to be spoken by me, your sincere well-wisher. My Lord! The cruel poison-like words that come from the mouth of an enemy, how can I, a servant of yours, utter those harsh sayings! O Lord of the Earth! I will never be able to utter those rude sayings that Indra has spoken.

29-53. Vy\=asa said :-- Hearing the messenger's words full of meaning the grass-eater Mahi\d{s}a D\=anava got very angry and, waggling his tail behind his back, passed urine; then his eyes reddened with anger, he called the D\=anavas before him and said :--O D\=anavas! The Lord of the Devas is firmly resolved on battle; therefore collect your forces; we will have to conquer that devil, the chief of the S\=uras. Who can stand for him a my rival here! If hundreds and thousands of warriors like Indra come I do not fear any of them at all; O D\=anavas, we will thoroughly put a end to him. His heroism is before those only that are peaceful and quiet before the ascetics that have become lean and thin by the penances; he is licentious and can only seduce other's wives by craftiness and arts. He is a thorough rogue and hypocrite, vicious and faultfinding; otherwise why does he put obstacles before others, depending for his strength only on the beauties of the Apsar\=as or heavenly prostitutes. He is treacherous to his very core; therefore he, being afraid at the very outset, took oaths, and entered into agreement with the high-souled Namuchi; afterwards, when his time turned favourable, that villain broke his treaty and treacherously killed him. Again the powerful Vi\d{s}\d{n}u is a thorough master of treachery and hypocrisy, the mine in taking oaths and can only show his vanity and is expert in that. He can assume many forms at will by his Magic power. For these very reasons Vi\d{s}\d{n}u had to take the form of a boar and

kill Hira\d{n}y\=ak\d{s}a; and again he had to take up a man-lion form to kill Hira\d{n}ya Ka\'s\={\i}pu. O D\=anavas! Never shall I surrender myself to Vi\d{s}\d{n}u, for I never place my trust in the words or deeds of Vi\d{s}\d{n}u and his Devas. What can Indra or Vi\d{s}\d{n}u do against me, when the most powerful Rudra is not able to fight against me in the battle-field! I will instantly defeat Indra, Varu\d{n}a, Yama, Kuvera, Fire, Sun and Moon and get possession of their Heavens. On our conquering the Devas, we all shall get our share of Yaj\~nas and we along with other D\=anavas drink the Soma juice and enjoy ourselves in Heaven. O D\=anavas! I have got the boon; what do I now care for the Devas. My death is not from men too. What can a woman do to me? O my emissaries! Call without any delay the chief D\=anavas from the nether regions and the mountains and make them my generals? O D\=anavas! I can alone conquer all the Devas; only to make the war arrangements look nice, that I am taking you to defeat them. There is no fear of mine from the Devas, consequent on the boon conferred on me. I will kill them by my hoofs and horns. I am not to be killed by Suras, Asuras, as men; therefore get yourselves ready to conquer the Devas. O D\=anavas! After conquering the Heavens we will be garlanded with P\=arij\=ata wreaths and we will enjoy the Deva women in the Nandana Garden. We will drink the milk of the heavenly milching cow (the cow that yields all desires) and, intoxicated with the heavenly drinks, we will hear and see the music and singing the dancing of the Gandarbhas there. You will all be served there with various bottles of wine by Urvas\={\i}, Menak\=a, Rambh\=a, Ghrit\=ach\={\i}, Tillottam\=a, Pramadvar\=a Mah\=asen\=a, Mira Kes\={\i}, Madotkat\=a, Viprachitti and others. Then be all ready at once for this auspicious occasion to march to Heavens and fight there with the Suras. And be pleased to call that pure-souled Muni \'Sukr\=ach\=arya, the son of Bhrigu and the Guru of the Daityas and worship him and tell him to perform sacrificial ceremonies for the safety and victory of the D\=anavas. O king! Thus, ordering the chief D\=anavas, the wicked Mahi\d{s}a went to his abode, with gladness.

Here ends the Third Chapter of the Fifth Book on the Daitya armies getting ready in \'Sr\={\i} Mad Dev\={\i} Bh\=agavatam, the Mah\=apur\=a\d{n}am by Mahar\d{s}i Vedavy\=asa of 18,000 verses.



