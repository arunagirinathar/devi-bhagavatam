\chapter{On the subject of Gau\d{n}a Bhasma}

1-33. N\=ar\=aya\d{n}a said :-- O Knower of Brahm\=a! O N\=arada! The ashes prepared from ordinary fire are secondary (Gau\d{n}a). The greatness of this secondary ashes is to be considered by no means trifling; this also destroys the darkest ignorance and reveals the highest knowledge. It is of various kinds. Amongst the secondary ashes, that prepared from Viraj\=agni is the best; it is equivalent to that obtained from Agnihotra Yaj\~na and it is as glorious. The ashes obtained from the marriage sacrificial fire, that obtained from the burning of the Samidh fuel, what is obtained from the conflagration of fire are known as the secondary ashes. The Brahm\=a\d{n}as, K\d{s}attriyas and Vai\'syas should use the ashes from the Agnihotra and the Viraj\=a Fire. For the householders, the ashes from the marriage sacrificial fire are good. For the Brahm\=acharis, the ashes from the Samid fuel are good and for the \'S\=udras the fire of the cooking place of the Veda knowing Brahm\=a\d{n}as is good. For the other persons, the ashes obtained from the conflagration of fire are good. Now I will talk of the origin of the ashes obtained from the Viraj\=a fire. The chief season of the Viraj\=a fire sacrifice is the Full-moon night with Chitr\=a asterism with the Moon. If this does not take place, the sacrifice may be performed at other seasons; and it should be remembered that the fit place is where one adopts as one's dwelling place. The auspicious field, garden or forest is also commendable for the above sacrifice. On the Trayoda\'s\={\i} Tithi, the thirteenth night preceding the full-moon night, one is to complete one's bathing and Sandhy\=a; then one is to worship one's Guru and bow down before Him. Then, receiving his permission, the sacrificer is to put on pure clothing and perform the special P\=uj\=a. Then with his white sacrificial thread, white garlands, and white sandal paste one is to sit on the Ku\'sa seat with sacrificial (Ku\'sa) grass in one's hands. With his face towards the east or north he is to perform Pr\=a\d{n}\=ay\=ama thrice.

Then he is to meditate on \'Siva and Bhagavat\={\i} and get mentally their permissions. ``O Deva Bhagav\=an! O Mother Bhagavat\={\i}! I will perform this vow for my life-time.'' Thus making the resolve, he should start with this sacrifice. But this is to be known that this Vrata can be performed for twelve years, for six years, for three years, for one year, for six months, for twelve days, for six days, for three days, even at least for one day. But in every case, he must take mentally the permission of the Deva and the Dev\={\i}. Now, to perform the Viraj\=a Homa, one is to light the fire according to one's Grihya S\=utras and then perform Homa with ghee, Samidh (fuel) or with charu (an oblation of rice, milk, and sugar boiled together). Then on the fourteenth lunar day (Chaturda\'s\={\i}) one is to pray ``Let the tattvas (principles) in me be purified'' and then perform the Homa ceremony with Samidh, etc., as above-mentioned. Now recollecting that ``My principles in my body are purified,'' he is to offer oblations to the fire. In other words, uttering ``Priththitattvas me sudhyat\=am jyotiraham viraj\=a vip\=apm\=a bh\=uy\=asam Sv\=ah\=a'' one is to offer oblations to the Fire. Thus uttering the five element (Mah\=abhutas), five tanm\=atr\=as, five Karmendriyas (organs of action), five J\~n\=anendriyas (organs of perception), five Pr\=a\d{n}as, seven dh\=atus Tvak, etc., mind, buddhi (intellect), Ahamk\=ara (egoism), Sattva, Raja, Tamah gu\d{n}as, Prakriti, Puru\d{s}a, R\=aga, Vidy\=a, Kal\=a (arts etc,) Daiva (Fate), K\=ala (time), M\=ay\=a \'Suddhavidy\=a, Mahe\'svara, Sad\=a \'Siva, \'Sakti \'Sivatattva, etc., respectively by its own name, one is to offer oblations to the fire by the five-lettered Viraj\=a Mantra; then the sacrificer will become pure. Then form a round ball of fresh cowdung and purifying it by Mantram place it on fire and carefully watch it. On that day, the devotee is to take Havi\d{s}y\=anna (a sacred food of boiled rice with ghee). On the morning of the Chaturda\'s\={\i}, he is to perform his daily duties as above and then to perform Homa on that fire; uttering the five lettered Mantra. He is not to take any food the rest of the time. On the next day, that is, on the full-moon day, after performing the morning duties, he is to do the Homa ceremony, uttering the Five lettered Mantra and then take leave of the Fire (invoked for worship). He is, then, to raise up the ashes. Then the devotee is to keep Jat\=a (matted hair) or to shave clean his head or to keep only one lock of hair on the crown of the head. He is to take his bath, then; and if he can, then he should be naked or put on a red coloured cloth, hide, or one piece of rag or bark; he is to take a staff and a belt. Washing his hands and feet and sipping twice he by his two hands, is to pulverise the ashes and, uttering the six Atharvan Mantras, ``Fire is ashe\'s' and so forth, apply ashes from his head to foot. Then, as before, he is to apply ashes, gradually to his arms, etc., and all

over the body uttering the Pra\d{n}ava of \'Siva, ``Vam, Vam.'' He is to put on the Triy\=ayusa Tripundra on his forehead. After he has done this, the J\={\i}va (the embodied self) becomes \'Siva (the Free Self) and he should behave himself like \'Siva. O N\=arada! Thus, at the three Sandhy\=a-periods; he is to do like this. This P\=a\'supata vrata is the source of enjoyment as well as liberation and as well as of the cessation of all brutal desires. By the performance of this vrata the devotee is to free himself gradually of his animal feelings and then to worship Bhagav\=an Sada \'Siva in the form of a phallic symbol. The above bath ashes is highly meritorious and it is the source of all happiness. By holding the ashes, one's longevity is prolonged, one gets even great bodily strength, becomes healthy and his beauty increases and he gets nourishment. This using of ashes is for the preservation of one's own self; it is the source of one's good and of all sorts of happiness and prosperity. Those who use ashes (Bha\'sma) are free from the danger of plague and other epidemic diseases; this bhasma is of three sorts as it leads to the attainment of peace, nourishment, or to the fufilment of all desires.

Here ends the Tenth Chapter of the Eleventh Book on the subject Gau\d{n}a Bhasma (secondary ashes) in the Mah\=apur\=a\d{n}am \'Sr\={\i}mad Dev\={\i} Bh\=agavatam of 18,000 verses by Mah\=ar\d{s}i Veda Vy\=asa.



