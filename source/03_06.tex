\chapter{On the description of the Dev\={\i}'s Vibhutis (powers)}

1-10. Brahm\=a said :-- When I thus asked with great humility, the Dev\={\i} Bh\=agavat\={\i}, the Prime \'Sakti, She addressed me thus in the following sweet words :-- There is oneness always between me and the Puru\d{s}a; there is difference whatsoever at any time between me and the Puru\d{s}a (Male, the Supreme Self). Who is I, that is Puru\d{s}a; who is Puru\d{s}a, that is I. The difference between force and the receptacle of force is due to error. He who knows the subtle difference between us two, is certainly intelligent; he is freed from this bondage of Sams\=ara; there is no manner of doubt in this. The One Secondless Eternal ever-lasting Brahm\=a substance becomes dual at the time of creation. As a lamp, though one, becomes two by virtue of adjuncts; as a face, though one, becomes two, as reflected in a mirror; as one man becomes double by his shadow, we become reflected into many, by virtue of different Antah Kara\d{n}as (mind, buddhi, and ahank\=ara) created by M\=ay\=a. The necessity of creation, again and again, after the Pr\=akriti Pralayas is due to the fructification of those Karmas of the J\={\i}vas, whose fruits were not enjoyed before the Pralayas ; so when creation again commences, the above said dif-

ferences are found to appear; Brahm\=a is the material cause of these changes; without Brahm\=a as the basis, the existence of M\=ay\=a is simply impossible. It is therefore that in M\=ay\=a and M\=ay\=a's action, Brahm\=a is interwoven. For this reason as many differences are found in M\=ay\=a, so many differences exist in Brahm\=a.

The M\=ay\=a and Brahm\=a appear as two and hence all the differences, visible and invisible, have come forth. Only during creation are these differences conceived. When everything melts away, i.e., there comes the Pralaya or general dissolution, then, I am not female, I am not male, nor I am hermaphrodite. I then remain as Brahm\=a with M\=ay\=a latent in it. During the time of creation I am \'Sr\={\i} (wealth), Buddhi (intellect), Dhriti, (fortitude). Smriti (recollection), Sraddh\=a (faith), Medh\=a (intelligence), Day\=a (mercy), Lajj\=a (modesty), Kshudh\=a (hunger), Trishn\=a (thirst), Ksham\=a (forgiveness), Aksham\=a (non-forgiving), K\=anti (lustre), S\=anti (peace), Pip\=as\=a (thirst), Nidr\=a (sleep) Tandr\=a (drowsiness), Jar\=a (old age), Ajar\=a (non old-age), Vidy\=a (knowledge), Avidy\=a (non-knowledge), Sprih\=a (desires), V\=anchh\=a (desires), \'Sakti (force), A\'sakti (non-force), Vas\=a (fat), Majj\=a (marrow), Tvak (skin), Dristi (sight), Saty\=asatya V\=akya (true and untrue words) and it is I that become Par\=a, Madhyam\=a, Pa\'syanti, etc., the innumerable N\=adis (tubular organs of the body, e. g., arteries, veins, intestines, blood vessels, pulses, etc.); there are three koti and a half N\=adis (35 millions of Nadis).

11-13. O Brahm\=a! See what substance is there in this Sams\=ara, that is separate from Me? And what can you imagine with which I am not connected? So know this as certain that I am these all forms. O Creator! Say, is there any such thing, where you will not see my above mentioned positive form? So, in this creation, I am one, and I am many as well, in various forms. Know this as certain that it is I, that assuming the names of all the various Devas, exist in so many forms of \'Saktis. It is I that manifest power and wield strength.

14-27. O Brahm\=a! I am Gaur\={\i}, Br\=ahm\={\i}, Raudr\={\i}, V\=ar\=ah\={\i}, Vaisnav\={\i}, \'Siva, V\=aruni, Kauver\={\i}, N\=ara Sinh\={\i}, and V\=asav\={\i} \'Saktis. I enter in every substance, in everything of the nature of effect. Making that Puru\d{s}a the instrument, I do all the actions (rather Puru\d{s}a is the efficient cause, the immediate agent). I am the coolness in water, the heat in fire, the lustre in the Sun, the cooling rays in the Moon; and thus I manifest my my strength. O Brahm\=a! Verily, I tell you this as certain that this universe becomes motionless, if it be abandoned by Me. If I leave \'Sankara, he will not be able to kill the Daityas. A very weak man is declared to be as without any strength; he is not said to be without

Rudra, or without Vi\d{s}\d{n}u, nobody says like this; everyone says, he is without strength, without \'Sakti. Those who get fallen, tumbled, afraid, quiet, or under one's enemies are called powerless; no one says that this man is Rudraless and so forth. So the creation that you perform, know \'Sakti, power to be the cause thereof. When you will be endowed with that \'Sakti, you will be able to create this whole Universe. Hari, Rudra, Indra, Agni, Chandra, S\=urya, Yama, Vi\'svakarm\=a, Varu\d{n}a Pavana, and other Devas all are able to do their karmas, when they are united respectively with their \'Saktis. This Earth, when united with \'Sakti, remains fixed and becomes capable to hold all the J\={\i}vas and beings. And if this Earth be devoid of force, She cannot hold an atom even.

Thus Ananta, Kurma and all the other elephants of the eight points of the compass, become able to do their respective works, only by My help (when united with Me, the Force). O Lotus born! If I wish I can drink all the fire and waters today and I can hold wind in check. I do whatever I wish. If I say that I am creating this world then the inconsistency arises thus :-- ``When I am everything, then I am being eternal, all this universe, made up of Prapancha, becomes eternal.'' (Whereas this universe is not eternal in the sense that it is changing.) If it were said that this universe is different from Me, then My saying that I am everything becomes inconsistent. Thinking thus, do not plunge yourself in the doubt as to the reality and origin and separateness of the non-eternal universe. For what is unreal, how can that come into existence? The unreal substances can never come into existence; as the child of a barren woman, the flowers in the sky are simply absurd. What is real can only be born. In discussing about origin, birth, etc., the appearance and disappearance of real things is called their birth and dissolution. In the cold of earth there exists the previous existence of the jar and this is the cause of the appearance of the jar; the disappearance of the jar exists in the jar; hence this disappearance is the cause of the destruction of the jar. Thus the appearance and disappearance of the causal eternal things are called the Origin and Pralaya. Similarly in discussing on the causal nature, there does not arise an inconsistency in My being everything.

28-48. So there is nothing to fear. In discussing about the reality of effects, this is to be conceived, that today there does not exist here the earth in the form of jar, if it is destroyed, where it has gone? The conclusion is that the earth in the form of jar exists in atoms. O Brahm\=a! All substances eternal, existing for a moment only, the void, and the substances of the nature, real and unreal both, all are due to a cause.

Ahank\=ara is born first among them. Thus substances are of seven kinds: Mahat, etc. O Unborn One! Mahattattva first arises from Prakriti; from Mahattattva springs Ahamk\=ara; and from Ahamk\=ara arises other substances. Thus, in this order, you go on creating this Universe. O Brahm\=a! Now you better go to your respective places, and after creating the Universe, remain there and perform your respective functions ordained by Pr\=arabdha. Take this beautiful great \'Sakti Mah\=a Sarasvat\={\i}, full of Rajogu\d{n}a, and of a smiling nature. This \'Sakti, wearing white clothes, adorned with divine ornaments and sitting on Var\=asana, will always be your playmate. This beautiful woman will always be your boon companion; consider Her as My bibhuti (manifestation of power), and so most worshipful. Never show any sort of disrespect towards Her. Take Her and go immediately to Satyaloka; and from the seed of Mahattattva, create the fourfold beings from these. The subtle bodies (Linga sar\={\i}ra) and Karmas are remaining mixed up with each other. Separate them, as before, duly, in due time.

Now go on as before and according to K\=ala (time), Karma, and Svabh\=ava (nature), join them with their respective attributes (sounds and other qualities); in other words bestow fruits according to their gu\d{n}as and Karmas (Pr\=arabdhas), and to the time when these fruits are due.

Vi\d{s}\d{n}u is prominent in Sattvaguna and hence superior to You. So You should always respect and worship Him. Whenever any difficulty will come to you, Vi\d{s}\d{n}u will come down on earth to fulfil your ends. Jan\=ardan Vi\d{s}\d{n}u will sometimes be born in the wombs of birds and animals, be sometimes in the wombs of men and destroy the D\=anavas. The highly powerful Mah\=a Deva, too, will help you. Now create the Devas and enjoy as you like. The Br\=ahma\d{n}as, Kshattriyas, and Vaisy\=as will worship you, with devotion, in various sacrifices, endowed with due sacrificial fees. All the Devas will be always satisfied when my name ``Sv\=ah\=a'' will be uttered in the sacrificial oblations and ceremonies.

\'Siva, the incarnate of Tamo gu\d{n}a will be revered and worshipped by all persons in every sacrifice. When the Devas will be frightened by the Daityas, then V\=ar\=ah\={\i}, Vaisnav\={\i}, Gaur\={\i}, Nara Simh\={\i}, \'Sach\={\i}, \'Siva and My other \'Saktis will take excellent bodies and destroy your fear. So, O Lotus-born! Be at your ease and do work. You utter and repeat my nine-lettered mantra with V\={\i}ja and Dhy\=an and do your work.

O highly intelligent one! This nine-lettered mantra is the best of all the mantras. You are to keep this mantra, within your heart, for the accomplishment of all your ends.

Thus saying to me, Bhagavat\={\i} smiled and began to say to Vi\d{s}\d{n}u :-- O Vi\d{s}\d{n}u! Take this beautiful Mah\=a Lak\d{s}'m\={\i} and go. She will always reside within your breast; there is no doubt in this. This all auspicious giving \'Sakti I give to you for your enjoyment.

You should always shew respect to Her; never show hatred or contempt. For the good of the world, I unite thus Lak\d{s}'m\={\i} and N\=ar\=ayan. For your sustenance I create Yaj\~na. You three will act together in harmony unanimously.

You, Brahm\=a and \'Siva are my three Devas, born of my Gunas. You three will undoubtedly be respected and worshipped by the world.

The stupid man who will find any difference between you three, will go to hell; there is no doubt in this. He who is Hari, is \'Siva; He who \'Siva is Hari; to make difference between these will lead one to hell. So Brahm\=a is one and the same with \'Siva and Vi\d{s}\d{n}u; there no manner of doubt in this. O Vi\d{s}\d{n}u! But there are other differences in their Gu\d{n}as; I will tell this; listen, as far as meditation of the Supreme Self is concerned you will have Sattva Gu\d{n}a predominant within you; and Rajo Gu\d{n}a and Tamo Gu\d{n}a will be secondary. In various other pursuits and Vik\=aras (changes) better have Rajo Gu\d{n}a with Lak\d{s}'m\={\i} and always enjoy Her.

49-85. O Lord of Ram\=a! I give you V\=akv\={\i}ja, K\=amav\={\i}ja, and M\=ay\=av\={\i}ja that will lead you to the highest end. Take this Mantra and repeat it and enjoy as you like. O Vi\d{s}\d{n}u! By this, the danger of death, caused K\=ala, will never come to you. When the creation of this Universe will be completely done I will then destroy this whole thing, moving and non-moving. You all will then be dissolved in Me. You should add pra\d{n}ava this mantra with K\=amav\={\i}ja leading to Mok\d{s}a and repeat it always with auspicious motives. O Puru\d{s}ottama! Build your Vaikunthapur\={\i}; live there and think of this My Eternal Form and enjoy as you like.

Brahm\=a said :-- Saying thus to V\=asudeva, that Higher Prakriti Dev\={\i} who is all of the three Gu\d{n}as and yet transcending them, began to address Mah\=a Deva, the Deva of the Devas, in sweet words, thus :-- O Sankara! Accept this beautiful Mah\=a K\=al\={\i} Gaur\={\i}, build a new Kail\=a\'sa city and live there happily. Your primary Gu\d{n}as will be Tamas; Sattva and Rajas will be your secondary Gu\d{n}as. Have recourse to Rajo and Tamo Gu\d{n}as while you slay the Asuras and thus wander.

O sinless \'Sankara! Have recourse to peaceful Satto Gu\d{n}a, when you reflect on the Supreme Self and practise austerities. You all are for creating, preserving and destroying the Universe and you are all of the three

Gu\d{n}as. There is no such thing in this world as are devoid of these three Gu\d{n}as. Everything, that is visible, is endowed with the three Gu\d{n}as, and whatever will be or was before cannot exist without them. Only the Supreme Self is without these Gu\d{n}as; but He is not visible. O Sankara! I am the Par\=a Prakriti; at times I appear with Gu\d{n}as; and at others I remain without any Gu\d{n}as. O \'Sambhu! I am always of the causal nature; never I am of the nature of effect. When I am causal, I am with Gu\d{n}as; and when I am before the Highest Puru\d{s}a, I am, then, without any Gu\d{n}as on account of my remaining in the state of equilibrium (S\=amy\=a vasth\=a). Mahattattva, Ahamk\=ara, and sound, touch, etc., all the Gu\d{n}as perform the work of Sams\=ara, day and night, each preceding one being the cause and each subsequent one being the effect; never do they cease in their activities.

From the Reality (Sat vastu) springs Ahamk\=ara (Avyakta); therefore I am of the nature of causality; again Ahamk\=ara is embodied with the three Gu\d{n}as, and so the Pundits call it as an effect of mine. From Ahamk\=ara arises Mahattattva; this is denominated as Buddhi. So Mahattattva is the effect and Ahamk\=ara is its cause. From Mahattattva arises again another Ahamk\=ara; from this second Ahamk\=ara arise the five Tanm\=atr\=as or the subtle elements. From these five Tanm\=atr\=as, the five gross elements arise after a process called Panch\={\i}kara\d{n}a. From the S\=attvika part of the five Tanm\=atr\=as, arise the five organs of perception; from their R\=ajasik part, the five organs of action come; from their Panch\={\i}kara\d{n}a, came the five gross elements; from the S\=attvika portion of all the five elements comes mind. Thus sixteen things come into existence. These organs of perception, etc., and other effects together with the Mah\=a bh\=utas form one Ga\d{n}a, composed of the sixteen categories. The original Puru\d{s}a is the Supreme Self; He is neither cause nor is He any effect. O \'Sambhu! At the beginning of the creation, all the above things are born in the way already indicated. Thus I have described to you, in brief; about the creation. O Devas! Now get up in your aeroplane and go to your respective places and fulfil your respective duties. Whenever you get into any dire distress, then remember Me; I will appear before you. O Devas! You should remember always the Eternal Supreme Self and Me. When you will remember us both, all your actions, will, no doubt, be crowned with success.

Brahm\=a said :-- Bhagavat\={\i} Durg\=a gave us \'Saktis, full of Divine beauty and lustre; She gave Mah\=a Lak\d{s}m\={\i} to Vi\d{s}\d{n}u, Mah\=a K\=ali to \'Siva, and Mah\=a Sarasvat\={\i} to me and bade good bye to us. Thus given farewell to by the Dev\={\i}, we three went to another place and were born as males. We thought of the very wonderful nature and influence of the Dev\={\i} and

we got upon our divine aeroplane. When we ascended, we saw there was no Manidv\={\i}pa, there was no Dev\={\i}, there was no ocean of nectar, nothing whatsoever. Save our aeroplane, we did not see anything. We then got into our wide aeroplane and reached there where Vi\d{s}\d{n}u killed the two indomitable Daityas, in the great ocean, where I was born from the lotus.

Thus ends the Sixth Chapter of the Third Skandha on the description of the Dev\={\i}'s Vibhutis (powers) in the Mah\=apur\=a\d{n}am \'Sr\={\i}mad Dev\={\i} Bh\=agavatam of 18,000 verses by Mahar\d{s}\={\i} Veda Vy\=as.



