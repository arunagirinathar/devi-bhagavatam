\chapter{On Indra's living under disguise in the M\=anas Lake}

1-16. Vy\=asa said :-- O King! Now seeing Vritra slain, Vi\d{s}\d{n}u, the Deva of the Devas, went to Vaikuntha; but, with this fear reigning supreme in his mind that it was He that virtually slew him. Indra, too, then became afraid of the sin committed by him and returned to his Heavens. The Munis, too, became very anxious and thought what great sin they have committed in cheating Vritr\=asura. It is the company of Indra that now made their name ``Muni\'s' as meaningless. The Munis thought thus :-- ``Oh! Vritra on our words trusted Indra and we have thus turned out today traitors in company with that traitor Indra. Attachment and affection is the cause of all mischief. Fie on that attachment! It is, as it were, tied by the cord of affection that we had sworn falsely on oath and thus deceived Vritra. Those that deliberately guide others to vicious acts or those that advise or incite others to do sinful acts or those that side with the sinners certainly partake of the fruits of the sin committed. Vi\d{s}\d{n}u, too, committed the sin, though he had Sattva Gu\d{n}a preponderance, when he entered into the thunderbolt and thus helped Indra in killing Vritra. It seems that henceforth the people, when selfish, won't

hesitate to commit afterwards any sinful act when they will see that Bhagav\=an Vi\d{s}\d{n}u could have done, in concert with Indra, such a vicious thing. Of the four virtues Dharma, Artha, K\=ama, and Mok\d{s}a, Dharma and Mok\d{s}a are very rare in the three worlds. Artha (wealth) and K\=ama (desires) are everywhere recommended as excellent and therefore held very dear; Dharma is now merely in name and is the cause of the vanity of the Pundits (no one now really practises Dharma with devotion). Thus arguing, the Munis became very much afflicted in their minds and went back to their own hermitages respectively, broken-hearted and absent-minded. O Bharata! Hearing of the death of his son by Indra, Vi\'svakarm\=a wept very much and he become disgusted very much with the affairs of the world. He went to the place where lay his son Vritra and became pained very much to see him in that state; and he performed his cremation and other funeral obsequies according to the prescribed rules. He then bathed, performed his Tarpa\d{n}am (peace-offering) and funeral ceremonies due to a person in the first year of his death. Then his heart became afflicted with sorrow and he cursed the vicious Indra saying that as Indra had killed his son, enticing him by falsely swearing on oath, so Indra, in his turn would suffer a heavier suffering, to be inflicted by Vidhi (the Great Creator of Universe). O King! Thus cursing Indra, Vi\'svakarm\=a, very much afflicted due to the loss of his son, went to the top of the Mountain Meru and began to practise a hard tapasy\=a.

17. Janamejaya said :-- ``O Grandsire! First tell me what happiness or pain did Indra derive by killing Vritra, the son of Tvast\=a.''

18-40. Vy\=asa said :-- O fortunate One! What are you asking? and what is the nature of your doubt? The fruit of one's Karma is certainly to be enjoyed, whether it be auspicious or inauspicious. Be he weak or strong. Be he a Deva, an Asura or a human being, everyone in fact, will have to suffer for one's Karma, good or bad, to its full extent, whether it be done a little or too much. See! It was Vi\d{s}\d{n}u that gave advice to Indra and entered into his thunderbolt and helped him when Indra was ready to kill Vritra; but when there had been Indra's difficult time, Vi\d{s}\d{n}u did not help Indra in any way. Therefore, O King! It is clear that when one's time is favourable, everybody turns out friends; but when Fate turns adverse, nobody is seen to come forward to help. When Fate is against anybody, one's father, mother, wife, or brother, servant, friend or one's own son becomes quite incapable to help anybody. The man, who does good or bad acts, suffers for his deeds. When Vritra was killed, everyone went back to their respective homes; but Indra, the Lord of \'Sach\={\i}, became very much deprived of his energy and brilliancy due to the sin of his kill-

ing a Br\=ahmi\d{n}; all the Devas, then, blamed him as a Br\=ahmaghataka (the killer of a Br\=ahmi\d{n}). They talked further that no other body would have been able to even indulge the idea of killing a Muni who was an intimate friend and who placed full confidence on him when Indra had sworn on oath that he would be a friend to Vritra. O King! Everywhere then there was this gossip in the assemblages of the gods, in their gardens, at the meetings of the Gandharbas that Indra had deceived Vritra who had relied on him, on the words of the Munis and then killed him by pretext, and so had done, indeed, a horrible crime. Indra had now forsaken the eternal proofs of the Vedas; and he had become a Bauddha; therefore he could have easily killed Vritra. No other body, save Vi\d{s}\d{n}u and Indra, could have acted contrary to their words, as clearly evidenced by the manner in which Vritra had been killed. These remarks, similar to those mentioned above, became everywhere current and Indra heard all of them, tending to his own disgrace. O King! Fie on that man's life that is blamed everywhere! Fie on that man whose fame has been marred amongst the people. Such a person becomes laughed at by his enemies, when seen by them on the way. The royal saint Indradyumna (R\=ajar\d{s}\={\i}) was made to get down, though sinless, from Heavens when his good deeds expired. Why, then, would not vicious persons be made to descend? The king Yay\=ati had to get down from Heavens for his very little fault and had to pass eighteen Yugas in the form of a crab. What more can be said than the fact that even the Bhagav\=an Achyut Hari had to take several incarnations in the wombs of boar, crocodile, etc., out of the curse from a Br\=ahmi\d{n}, due to his cutting off the bead of the wife of Bhrigu. Though omnipresent, yet he had to take the appearance of a dwarf and had to beg from the King Vali's palace. What more troubles and miseries than this can be inflicted on those that had sinned viciously. O Ornament of Bharata! R\=amchandra, too, had to experience, due to the curse of Bhrigu, terrible miseries on the bereavement of S\={\i}t\=a Dev\={\i}. Similarly Indra, too, for his sin of killing a Br\=ahmi\d{n}, was so much terrified that he could not get his healthy condition though he remained in his own house, endowed with all sorts of prosperity and wealth. Seeing, then, Indra lustreless, knowledge-less, almost void of consciousness, and overwhelmed with fear, his wife \'Sach\={\i}, the daughter of Pulom\=a, spoke to him thus :-- ``O Lord! Your dreadful enemy has been killed; why are you, then, sighing so much, being afflicted with so much terror? O Lord! You have destroyed your enemy; then why are you so much anxious? why are you then so much remorseful and drawing such deep heavy sighs like an ordinary man? I am not seeing any other powerful enemy of yours; then, why do you look so anxious and bowed down with cares, as if you look quite unconscious.''

41-44. Indra said :-- ``O Dev\={\i}! True that I have no other powerful enemy, yet I do not find peace nor any happiness. I fear for the sin Br\=ahmahatty\=a in my house. O Dev\={\i}! This Nandana Garden, the city of Kuvera, the lord of riches, this nectar forest, the sweet music of the Gandbarbas, the beautiful dance of the Apsar\=as, all these now do not give the least pleasure to me. What more can I say than this that the beautiful Lady like you, most beautiful amidst the three worlds, and other beautiful ladies, the Heavenly cow, the Mand\=ara tree (one of the five trees of the celestial region), the P\=arij\=ata tree (the flower tree), the Sant\=ana tree, the Kalpa tree (yielding all desires) and the Harichandan (saffron tree) and others cannot give pleasure to me. What to do, where to go, so that I get happiness, O Beloved! This thought makes me uneasy. And so I am not able to get happiness in my own thought.''

45-60. Vy\=asa said :-- Thus speaking to his most distressed wife, Indra got out of his house and went to the exceedingly beautiful lake, named Manasarovara. Indra there entered into the tubular stalk of the lotus, his body becoming very lean and thin out of the fear and sorrow. Nobody could recognise him as he was overpowered by his terrible sin. He then began to behave himself, as regards feeding and enjoying, like a snake; and he became ovewhelmed with thought, helpless, and his organs were out of order, He remained hidden in the water. When Indra, the king of the Devas, thus fled away out of the fear of his Br\=ahmahatty\=a sin, the other Devas became very anxious; everywhere various evil signs manifested themselves. The \d{R}i\d{s}is, Siddhas and Gandharbas were very much panic-stricken, as various disturbances and violent symptoms covered all over the world without any king. Grains began to grow very scanty, due to want of rains; the streams were almost dry and very little water was there in the tanks. In such a state of anarchism, all the inhabitants of the celestial regions, the Devas and \d{R}i\d{s}is consulted and installed the king Nahu\d{s}a in the place of Indra. O King! Nahu\d{s}a, though virtuous, became, under the sway of Rajogu\d{n}a, influenced by lust and thus he got very much addicted to worldly enjoyments. He began to amuse himself in the Garden of Paradise, surrounded by the Apsar\=as or celestial nymphs. One day he heard of the excellent qualifications of \'Sach\={\i} Dev\={\i}, the wife of Indra, and desired to acquire her. Then he spoke to the \d{R}i\d{s}is :-- The Devas and you, united, have installed me in the office of Indra; but why does not the Indran\={\i} (the wife of Indra), come to me so long? If you want to do what I like, then quickly bring \'Sach\={\i} here before me for my gratification. I am now Indra and therefore the god of the Devas and all the worlds; therefore bring today quickly Indran\={\i} to my house. Hearing thus the words

of the king Nahu\d{s}a, the Devas and Devar\d{s}\={\i}s became anxious and went to \'Sach\={\i}, and, with their heads bowed down, spoke thus :-- ``O Wife of Indra! The wicked Nahu\d{s}a is now desiring you; he became angry and told as to send you to him quickly; O Dev\={\i}! We have made him Indra and are therefore under him; what shall we do now under these circumstances?'' \'Sach\={\i}, the wife of Indra, hearing their words, became absent-minded and spoke to Brihaspati, thus :-- ``O Br\=ahma\d{n}a! I now take refuge unto you.''

61-62. Brihaspati said :-- ``O Dev\={\i}! Do not be afraid of Nahu\d{s}a; he has been deluded by Moha. O Child! I won't forsake the eternal religion and thus I won't give you over to the hands of Nahu\d{s}a. No doubt that wretch suffers the severest torments in Hell to the end of Pralaya (the Great Dissolution) who quits and hands over the distressed person under one's refuge to another. O Good One! Be comfortable; I will never forsake you.''

Here ends the Seventh Chapter of the Sixth Book on Indra's living under disguise in the M\=anas Lake in the Mah\=apur\=a\d{n}am \'Sr\={\i} Mad Dev\={\i} Bh\=agavatam of 18,000 verses by Mahar\d{s}i Veda Vy\=asa.



