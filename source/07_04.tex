\chapter{On the conversation between the two A\'svins and the Princess Sukany\=a}

1-38. Vy\=asa said :-- O King! When the King \'Saryati departed, that virtuous lady devoted her time in serving her husband, and the Fire. She gave to the Muni for his food various delicious roots and fruits. She made him bathe with warm water; then making him put on the deer skin, she made him sit on the Ku\'s\=asan. (Seat made of the Ku\'sa grass.) Next she used to place in his front Ku\'sa, Til and Kamandalu and speak to him ``O best of Munis! You are now to perform your daily rites (Nitya Karma).'' When the Nitya Karma was finished, the lady used to catch him by his hand and make him sit on another seat. Next the

princess brought fresh ripe fruits and cooked rice, grown without cultivation and gave to the Muni for his food. When the husband finished his meals, she gave him devotedly water for cleansing his mouth; then washing his hands and feet gave him the betelnuts and p\=an leaves. Next she made him sit on an excellent seat, and with his permission, performed her own bodily purifications. She then, used to eat the remnants, fruits and roots of the dishes of her husband; and coming next to her husband addressed him affectionately ``O Lord! Order me what can I do now for you? If you like, I may shampoo your legs and feet.'' Thus every day that princess devotedly spent her time in serving her husband. At evening when the Homa ceremony was finished, she collecting delicious and soft fruits presented to him to eat. With his permission she, then, ate that were left of the fruits; next she prepared bedding soft to the touch and gladly made him sleep on it. When his dear husband thus laid himself on the bed, she shampooed his feet and legs and in the interval, asked him about the religious duties of the chaste wives of the family. When the Muni fell asleep in the night, she devotedly laid herself close to his feet and slept. During the summer season when her husband was perspiring, the lady used to fan him with cool breeze. She took off his troubles and thus served her husband. In the cold season, she collected firewood and lit them before him and used to ask him frequently ``O Muni! Are you feeling pleasure in this?'' That lady, dear to her husband, used to get up from her bed in the Br\=ahma Muh\=urta before Sunrise, next made her husband get up and took him some short way off and there made him sit for calling on nature. She kept ready water and earth and stood in readiness at a suitable distance apart. Knowing that he had finished his calls for nature, she took him back to the \=A\'sram, and washed his hands and feet with water duly making him sit on a convenient \=Asana. She then gave to him the pot for \=Achamana and then began to collect fuel. She used to bring pure clear water and made it hot; then she asked her husband fondly ``O Husband! Have you finished cleaning your teeth? Warm water is ready; may I bring it to you? You better bathe with that, uttering your Mantrams. The time is now for performing the morning Sandhy\=a and for performing Homa. Do the Homa regularly and worship the Devas.'' The princess, whose nature was pure and clean as anything, kept herself engaged daily in serving her husband, Chyavana Muni, with perfect gladness, austerity, and observing all the rules duly. Thus that beautiful-faced princess worshipped gladly Chyavana Muni, serving Fire and the guests daily. Then, once, on an occasion, the A\'svin twins, the sons of S\=urya came sporting and at their pleasure, close to the hermitage of the Chyavana Muni. At that

time the princess, beautiful in all respects, was returning home after her bath in a pure clear stream and came to the sight of the two A\'svins. Being enchanted by her extraordinary lovely beauty, they thought she might be a Deva Kavy\=a, quickly went to her and fondly questioned her :-- ``O slow moving One like an elephant! Look! We are the sons of Devas; we have come to you to ask some questions. O Excellent One! Wait for a moment; we request thus to you. O Sweet-smiling One! Please answer our questions truly and properly. O Lovely-eyed! Whose daughter are you? Who is your husband Why have you come here alone to bathe in this tank? O Lotus-eyed! You seem to be a second Lak\d{s}m\={\i}; O Beautiful One! We want to know something; please reply exactly. O Beloved! Your feet are exceedingly gentle; why have you not put on any shoes; why are you walking barefooted? Our hearts are being troubled to see you walking thus barefooted? O Thin bodied One! Your body is very soft that you ought to have gone in a car; why are you thus walking on foot and in such an ordinary dress in this forest? Why have not hundreds of maid-servants accompanied you? O lovely faced One! Speak truly whether you are a princess or Apsar\=a. O Sinless One! Blessed is your mother from whom you are born, blessed is your father. Specially the person with whom you are married, we are unable to describe his fortune. O Lovely eyed! This earth is being sanctified by the movements of your feet; consequently this garden is now purer today than the Devaloka. Boundless is the fortune of these deer and birds who can see you whenever they like; what more can we say than this that this forest is rendered very pure. O Fair One with fair eyes! It is needless to praise your beauty; speak truly who is your father and who is your husband; we like very much to see them.''

39-56. Vy\=asa said :-- O King! On hearing their words, the exquisitely beautiful princess bespoke to the twin Devas with much bashfulness :-- I am the daughter of \'Saryati; father has given me over, under the directions of the Daiva, to the Mahar\d{s}i Chyavana. I am his chaste dear wife; the Mahar\d{s}i is staying in this very place. O Twin Devas! My husband is a blind ascetic and he is very aged. I gladly serve him day and night according to the rules of chastity amongst women. Who are you? And why have you come here? My husband is staying in the \=A\'srama; kindly come and sanctify the A\'srama. O King! The two A\'svins heard her and said :-- ``O Auspicious One! Why has your father betrothed you, such a gem, to an old ascetic? It is very strange. Indeed! In this solitary forest you are shining like a steady lightning; what more can we say than this that we hardly find a beautiful lady like you, even

in the Devaloka! Alas! The Deva dress and a full set of ornaments and blue dyes look well on you; this deer-skin and barks of trees in no way fit you. O Beautiful One! Your eyes are very large; yet the Creator has given you a blind husband; specially a very aged one; and you are wearing away by constantly dwelling with your blind husband in this forest. What more can be wrong for the Creator than this? O deer-eyed One! In vain you have selected him for your husband. At this period of your youth and beauty it does not look at all well to see you with your blind husband. You are versed in dancing and music; but your husband is blind and aged; when in dancing you will shoot your darts of love, on whom then, will those arrows fall? O large-eyed One! Oh! The Creator is certainly of a very little understanding! Else why would he have made you, so full of youthful vitality, the wife of a blind man? O lovely-eyed One! You are never fit for him; select another husband. O Lotus-eyed One! Your husband is not only blind but an ascetic; so your life is quite useless; we do not consider it fit that you reside in this forest and put on this bark and deer-skin. O dark-eyed One! Your body and every limb thereof is very beautiful; judge well and make one amongst us your husband. O Proud One! Why are you being so very beautiful, spending your youth in vain in serving this Muni? No good signs are visible in this Muni; he cannot maintain nor protect you even ; why are you, then, serving him in vain? O spotless One! Leave at once this Muni, quite incapable in giving any sort of pleasure, and marry one of us. O Beloved! Then you will enjoy in the Nandana K\=anana or in the forest of Chaitratarha. O Proud One! How will you spend your time with the aged husband, being brought to so much humiliation and without any dignity and self-respect. You are endowed with all auspicious signs; moreover you are a princess; you are not ignorant of all enjoyments in this world; why then you like to live such an unfortunate life in vain in this forest? O Princess! Your face is exceedingly beautiful; your eyes are wide and your waist is thin. Your voice is sweet like a cuckoo. Who is more beautiful than you? Quit now your aged ascetic husband and marry one of us for your happiness; then you will be able to enjoy excellent celestial things in the heavens. O good-haired one! What pleasure can you derive by your staying in this forest with your blind husband! O deer-eyed One. It is very painful for you to serve at this young age of yours, to remain in this forest and serve this aged man. O Princess! Is it that you like troubles and nothing else. O One with a face lovely like the Moon! We see that you are of a very soft body; so to collect water and fruits is never a duty fit for you.

Here ends the Fourth Chapter in the Seventh Book on the conversa-

tion between the two A\'svins and the Princess Sukany\=a in \'Sr\={\i} Mad Dev\={\i} Bh\=agavatam, the Mah\=a Pur\=a\d{n}am, of 18,000 verses, by Mahar\d{s}i Veda Vy\=asa.



