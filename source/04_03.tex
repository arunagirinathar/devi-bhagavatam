\chapter{On the former curse of Vasudeva and Devak\={\i}}

1. Vy\=asa said :-- O king; The incarnation of Hari and the incarnation of the Amsa Avat\=aras of all the other Devas are accountable to many causes. The chief cause being Karma; the minor causes being many.

2. Hear, now, the cause of the incarnations of Vasudeva (Kri\d{s}\d{n}a's father), Devak\={\i} and Rohin\={\i} in detail.

3. Once, on an occasion, \'Sr\={\i}m\=an Ka\'syapa stole away the K\=amadhenu (the heavenly Cow, yielding all desires) of the Deva Varu\d{n}a for his sacrificial purpose; and though he was entreated by Varu\d{n}a often and often to return the cow, Ka\'syapa did not return to him that, the best of all the cows.

4. Varu\d{n}a became very sorry; he went to Brahm\=a, the Lord of the creation and told him humbly all that had happened and about his sorrows.

5-6. ``O Glorious One! Mahar\d{s}i Ka\'syapa is now almost infatuated with his sacrifice; and though I have tried all my means, he is not returning me my cow. I could not hear the pitiful cries and wailings of the calves or bereavement from their mother; and I cursed Ka\'syapa saying `You would go down and take birth in the human world as a cow-herd; and your two wives also are to go there as human mortals, suffering under the greatest difficulties and dangers.'''

7. O Br\=ahma\d{n}a! On seeing the distressed condition of the calves I cursed Aditi a second time that she would be put to prison, her children would be still born, and she would suffer lots of troubles.

8. O Janamejaya! Hearing this, the Lotus-born Brahm\=a called Ka\'syapa before him and asked.

9. O Fortunate One! Why have you stolen away all the cows of the Varu\d{n}a Deva, the Guardian of a quarter of the world? And why have you committed an offence in not returning the cows to Him?

10. Bhagav\=an! You are intelligent, you know everything fully; knowing that it is a sin to steal other's property, why have you committed the unlawful act of stealing away the cows.

11. Oh! What is the wonderful influence of covetousness! Even those that are great are not free from the clutches of greed. Covetousness is the source of all sins, is unapproved by the Sages and leads to hell.

12. Lo! Mahar\d{s}i Ka\'syapa is not able to leave this vicious habit even now; what shall I do? I will hence count greed as more powerful than even Fate, the Ruler of all destinies.

13. Blessed are those saints that have devoted themselves wholly to the attainment of peace, who are tranquil-hearted, lead a hermit life and don't ask themselves of any thing from any body. Verily those are blessed.

14. This covetousness is a powerful enemy; it is always unholy and odious. See! Its influence has overpowered the Mahar\d{s}i Ka\'syapa and has tied him down to an ordinary affection and has urged him to commit a sinful act.

15-16. Then the Praj\=apati Brahm\=a, to preserve and keep the prestige in the name of Justice and Religion, cursed his own very dear grandson Ka\'syapa, the best of the Munis, and said :-- Go to the earth in your Am\'sa, and take your birth in the Yadu clan, be united with your wives and work as a Cowherd.

17. Vy\=asa said :-- O king! Thus was cursed the Mahar\d{s}i Ka\'syapa by Brahm\=a and Varu\d{n}a to come down to the earth as Ams\=a Vat\=ara to relieve the earth of her burden.

18. Diti, too, becoming grieved much with sorrows, cursed Aditi that seven of her sons would be killed consecutively after their births.

19. Janamejaya said :-- O best of Munis! Why was it that Diti so cruelly cursed his sister Aditi, the mother of Indra? Kindly explain to me the cause of this and oblige. I am sorry to hear of this curse.

20. S\=uta said :-- Thus asked by the son of Parik\d{s}it, Vy\=asa, the son of Satyabat\={\i}, himself replied to the king about their causes in the following words :--

21. Vy\=asa said :-- Dak\d{s}a Praj\=apati had two daughters, Diti and Aditi; these two, of high rank, were married to Ka\'syapa; and they were his favourites.

22. Aditi gave birth to the very powerful Indra, the king of the Devas. Diti, too, asked for a son of the same strength, prowess, and splendour as those of Indra.

23. Diti, of beautiful dark blue eyes, entreated to her husband and said, ``Give me a son, O giver of due respects to every body! who shall be a hero as strong as Indra, religious and of indomitable energy.''

24. The Muni said to her :-- O Dear! Be peaceful; I advise you to take a vow, practise a rite, and when the period of your practice will be over, you will get a son like Indra.

25. Diti promised to act according to his word and took an oath; and when she practised the vow, Mahar\d{s}i Ka\'syapa impregnated the seed in her womb. Diti also bore the seed in her womb according to the usual rite.

26. The excellent fair complexioned Diti remained sacred, observed all the rules and, deeply intent on her vow, subsisted only on milk and slept or the ground.

27-28. Thus when the foetus was fully developed, Diti began to look white and full of splendour. On seeing her thus, Aditi became anxious and thought if there be born of Diti a son like the powerful Indra, then my son will no doubt be deprived of his brilliancy and splendour.

29. The proud Aditi, thinking thus, said to Indra :-- O Son! There, in the womb of Diti, is your powerful enemy.

30. O Beautiful One! Even now think out how you can kill your enemy. Before the child is born of the womb, try to destroy
it.

31. Since the time I have looked, on my co-wife Diti, of beautiful eyes and proud, this is the one and only thought that troubles the peace in the innermost of my hearts.

32. The enemy, if he firmly gets hold like a fully developed consumption, cannot be killed; therefore the intelligent persons should destroy the enemies, when they are in their buds.

33. O \'Satakratu! My heart is being pierced wholly by an iron spoke when I see the womb of Diti; kill it by any means you can!

34. O High minded One! If you like my welfare, then destroy the foetus, in the womb of Diti, by any of the existent means, S\=ama, D\=ana or strength and thus remove the cause of grief in my heart.

35. Vy\=asa said :-- On hearing his mother, Indra, the King of the Immortals, thought over all the means and went then to his step-mother Diti.

36. That evil minded Indra bowed down at the foot of Diti with humility and addressed her with words, sweet but full of poison.

37. O mother! You have become very weak, lean and thin in the practise your vow. I have come to serve you; order me now what I can do for you.

38. O chaste one to your husband! I want to shampoo your feet. To serve one's Guru means to earn righteousness and immortality.

39. O mother! I swear, on oath, I don't make any difference between you and my mother Aditi. Saying thus, he touched her feet and began to shampoo her legs.

40. The beautiful eyed Diti, tired of the vow, lean and thin, thus being shampooed and having full faith in Indra's words, fell to deep sleep.

41-42. Seeing her asleep, Indra, with thunderbolt in his hand, took subtle form and by the influence of his yogic power, entered carefully into her womb quickly and cut asunder the foetus in the womb into seven parts.

43-44. The child in the womb, struck by the thunder bolt, cried out. Indra spoke to the child gently :-- ``Do not cry,'' and in the mean while cut each of the seven parts into seven parts again. Thus, O king! The forty-nine Maruts were born.

45. When the good natured Diti awoke, she came to know that Indra has treacherously cut the foetus in her womb and became very sorry and angry.

46-47. Knowing that all these treacherous acts are really done under the advice of her sister, the truthful Diti; who was under the vow, cursed Aditi, and Indra, saying that as her son Indra has treacherously cut the foetus in her womb, Indra's kingdom over the three worlds would be destroyed.

48-49. And as the sinful Aditi has secretly caused the destruction of my son, her sons, too, would also die after their birth consecutively and she would dwell in the prison house in much trouble and anxiety and would also bear still born sons in her next birth.

50. Vy\=asa said :-- O king! Mahar\d{s}i Ka\'syapa, the son of Mar\={\i}chi, hearing the curse, allayed her anger with loving words.

51. O Blessed One! Do not be angry. Your sons would all become very powerful and would be called Maruts. They would be companions and friends to Indra.

52. O Dear! Your curse won't be fruitless; in the 28th Manvantara, at the end of the Dv\=apara Yuga, your curse will bear fruit. Then Aditi, sinful for her jealousy and anger, will go down on earth to take the human birth through her Amsa (part) and suffer according to your curse.

53. Varu\d{n}a, too, had become very grieved and cursed her. And, due to both these curses, this Aditi will be born as a woman.

54. O King! The fair complexioned Diti, thus consoled by her husband, became glad and did not utter any more unpleasant words.

55. O king! Thus I have narrated to you the cause of the previous curse. O best of kings! Thus Aditi was born as Devak\={\i} out of her Am\'sa.

Here ends the third Chapter of the Fourth Book of the Mah\=a Pur\=a\d{n}am \'Sr\={\i} Mad Dev\={\i} Bh\=agavatam, of 18,000 verses on the former curse of Vasudeva and Devak\={\i} by Mahar\d{s}i Veda Vy\=asa.



