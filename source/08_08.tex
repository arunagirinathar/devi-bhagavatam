\chapter{On the description of Il\=avrita}

1-11. N\=ar\=aya\d{n}a said :-- In those Var\d{s}as, Vi\d{s}\d{n}u and the other Devas used to worship always the Great Dev\={\i} with Japam and meditation and to chant hymns to Her. The forests there are ornamented with all sorts of fruits, flowers and leaves, in all the seasons. In those excellent forests, and on the mountains, in those Var\d{s}as and on the clear waters interspersed with full blewn lotuses and \'S\=arasas (cranes) and in those countries where varieties of mountain trees are standing and where varieties of birds frequent and echo all around, the people play in waters and engage themselves with a great many pleasant occupations; and the beautiful women, as well, roam there with the knitting of their eyebrows. The people there enjoy as they like, surrounded by young women; N\=ar\=aya\d{n}a, the \=Adipuru\d{s}a Bhagav\=an Himself, worships the Dev\={\i} there, to shew his extreme grace to all the inhabitants of the Navavar\d{s}a. The people also worship the Deity. By worshipping the Dev\={\i} only, the Bhagav\=an remains there in Sam\=adhi, surrounded with Aniruddha and his other Vy\=uhachatustaya (the four forms). But in Il\=avrita Var\d{s}a, the Bhagav\=an Rudra, originated from the eye-brows of Brahm\=a, resides only with women. No other person can enter there; for the Bh\=avan\={\i}, the \'Sakti of Rudra has cursed that any male entering there would be transformed into a female. The Lord of Bhavan\={\i}, surrounded by innumerable women, remains here engaged in the worship of the unmanifested unborn Bhagav\=an Samkar\d{s}a\d{n}a. For the good of humanity, with intense meditation, He worships His Own Tur\={\i}ya Form of the nature of Tamas, thus :--

12-19. \'Sr\={\i} Bhagav\=ana said :-- ``Obeisance to Thee! the Bhagav\=an, the Great Puru\d{s}a, endowed with all the qualities (the principal six Ai\'svaryas or prosperities), the Ananta (the Infinite) and to the Unmanifested! We worship Thee, Whose lotus feet are the refuge of all. Thou art the great storehouse of all the superhuman powers and the divine

faculties of omnipotence, etc. Thou art always present to the Bhaktas. Thou art creating all these beings. Thou givest Mok\d{s}a to the Bhaktas and destroyest their attachment to the world and Thou bindest Thy non-devotees in bondage to this world. Thou art the Lord. We worship Thee. We are entirely under the control of the passions, anger, etc., and our minds are always attached to the senses; but though Thou art always looking at this world for its creation, etc., Thy mind is not a bit attached to it. So who will not turn to Thee, desirous to conquer his self. Thou art appearing by Thy M\=ay\=a as one whose sight is ignorant; Thou lookest dreadful with Thy eyes reddened with the drink of Madhu (wine). By the touch of Thy feet, the mental faculties are very much enchanted; hence the women folk of the N\=agas cannot in any way worship Thee, out of bashfulness. The \d{R}i\d{s}is say that though Thou art the Only One to create, preserve and destroy, yet Thou art quite unconcerned with them. Thou art Infinite and Thou hast innumerable heads. This vast universe is like the mustard seed resting somewhere on one of these heads, which Thou canst not feel even. The Mahat Tattva is Thy body manifested first. It is built of Sattva, Raja and Tamo Gu\d{n}as. Brahm\=a has come out of this and I again have sprung from this Brahm\=a and am nurtured by the Sattva and the other Gu\d{n}as and with the help of the Teja, created these elements and the senses. These Mahat Tattvas and we all are controlled by Thy Extraordinary Form. Thou hast kept us in our respective places by Thy Kriy\=a \'Sakti as birds are kept duly by the strings through them. Mahat Tattva, Ahamk\=ara, and the Devas, elements and the senses, before mentioned all united create this Universe by Thy Grace. Thy creation is very big and grand; for this reason the gross thinkers, deluded by Thy power, never understand it. This M\=ay\=a is the only means to get the Sams\=ara Nivritti, Mok\d{s}a, the real Goal of man: and this M\=ay\=a, again involves them in the Karma entanglements, very hard to get through. Coming in and going out, both of these are Thy forms; so we bow down to Thee.

20-23. N\=ar\=aya\d{n}a said :-- Thus the Bhagav\=an Rudra, with His Own persons in Il\=avrita Var\d{s}a used to worship the Dev\={\i} and the Sankar\d{s}a\d{n}a, the Controller of all the Lokas. The son of Dharma, well known by the name of Bhadra\'srava and all the persons born of his family and his attendants, worship thus the Dev\={\i}. This form is well known to all by the name of Hayagr\={\i}va and worshipped thus. All the persons there worship Him with the intense meditation and Sam\=adhi and realise Him thoroughly. Then they praise Him, according to due customs and get the thorough Siddhis (success in getting extraordinary powers).

24-29. The Bhadra\'srav\=as said :-- Obeisance to Thee, the Bhagav\=an, the Incarnate of Virtue, and to Him who destroys completely the desires, attachments, etc., to worldly objects! Ho! How wonderful are the feats of the Bhagav\=an! Death always destroys all, but people, seeing this, seem not to see this. Seeing that the son meets with death, the father desires to live long not for a virtuous purpose but for sense enjoyments, what is called Vikarma. Those who are skilled in J\~n\=ana and Vij\~n\=ana say that this Universe that is seen is very transient. Moreover those Pundits who are endowed with much J\~n\=ana, see vividly the transitoriness of this Universe. Still, O Unborn One! When practically they come to deal with this, they all become overpowered with the influence of M\=ay\=a. So Thy Pastime (L\={\i}la) is wonderfully variegated. (Instead of spending our time uselessly in discussing on \'S\=astras) we bow down to Thee, and Thee alone. Thou art the Self-manifest Chaitanya. Thou are not the object to be covered by M\=ay\=a. Thou dost not do anything in the sort of creation, etc. Thou remainest simply as the Witness thereof. Sill the Vedas declare that Thou createst, preservest and destroyest the Universe. It is quite reasonable and nothing to be wondered at. Thou art the \=Atman of all. When the Pralaya (the time of great dissolution) comes, the Vedas were stolen by the Daityas and taken to the nether regions, the Ras\=atala. Thou, in the form of Hayagr\={\i}va (Horse-faced), rescued the Vedas and gave them to the Grandsire Brahm\=a who was very eager to get them back and understand their meanings. Thou art the true Sankalap (resolve); we bow down to Thee. Thus the Bhadra\'srav\=as praise the Haiyagr\={\i}va form of Hari and sing the glorious deeds of Him. He who reads these narratives of the Mah\=a Puru\d{s}a (the high-souled personage) or he who makes others hear these things, both of them, quitting their sinful bodies, go to the Dev\={\i} Loka.

Here ends the Eighth Chapter of the Eighth Book on the description of Il\=avr\={\i}ta in the Mah\=apur\=a\d{n}am \'Sr\={\i} Mad Dev\={\i} Bh\=agavatam, of 18,000 verses, by Mahar\d{s}i Veda Vy\=asa.



