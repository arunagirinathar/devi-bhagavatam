\chapter{On the defeat of the Devas by Vritra}

1-17. Vy\=asa said :--- O King! The Suras that wanted to create hindrance in Vritra's tapasy\=a, seeing him firmly resolved, became disappointed in the fulfilment of their objects and returned to their own abodes. Thus full one hundred years passed away. The four-faced Brahm\=a, the Grandsire of the Lokas, came there mounted on his carrier the Swan, and said :-- ``O Vritra! Be happy; now quit your meditation and ask boon; I will grant you the boon that you choose. O Child! Your body has become very lean and thin through your penance. I am now very pleased to see your this very hard tapasy\=a. Welfare be to you. Now ask the boon that you desire.'' Vy\=asa said :-- O King! Hearing thus the clearly distinct nectar-like sweet words of the Creator Brahm\=a, Vritra shed tears of joy and suddenly stood up. And going to him, bowed down gladly before His feet, and, with folded hands, spoke to Him, Who is desirous to grant him boons, in a tremulous voice. O Lord! Today I have been fortunate to see Thee who art generally seen with great difficulty; and I have acquired thus the posts of all the Devas; O Lotus-seated One! I have got an insatiable desire burning within me. Thou art omniscient, Thou knowest everything; still I am speaking out my mind. O Lord! Grant that my death does not occur with iron, wood, dry or wet substances or with bamboos or any other weapons and let my strength and valour be increased very much in the battle; for, then, I will be unconquerable by all the Devas with all

their armies. Vy\=asa said :-- O King! Thus prayed for, Brahm\=a said to him smiling :-- ``O Child! get up; I grant that your desired boon will always be fulfilled; now go to your own place. Your death won't occur with dry or wet substances or with stones or wood. I say this truly unto you.'' Thus granting the boon, Brahm\=a went to His Brahm\=aloka. Vritra, too, became very glad on receiving his desired object, and returned to his own abode. The highly intelligent Vritra informed the father about the boon granted to him; Vi\'svakarm\=a became very glad to hear it. O highly fortunate One! Let all bliss and good fortune come unto you; kill Indra, my greatest enemy. Go and kill the murderer of my son Tri\'sir\=a, the vicious Indra and return to me. Be victorious in the battle and become the Lord of all the Devas and remove my mental agony due to the killing of my son. A son becomes then really a son when he obeys the commands of his father and when he feeds plentifully good many people on the Sr\=addha day (after his father's death) and when he offers Pinda at Gay\=a. Therefore, O Son! Keep my words and try to remove my sorrows. Know this as certain that Tri\'sir\=a never vanishes from my mind. Tri\'sir\=a was very truthful, amiable and good-natured; he was an ascetic and foremost amongst the Vedic scholars. The wicked Indra killed my dear son without any offence.

18-33. Vy\=asa said :-- O King! Hearing the father's words, that extremely indomitable Vritr\=asura mounted on his chariot and quickly got out of his father's house. The proud Asura, then, marched to the battle, accompanied with his vast army, to the sounding of the conch-shells and war drums. Vritra, versed in politics and morals, exhorted his soldiers before marching and said :-- ``To-day we will kill Indra and possess the kingdom of the Immortals, freed of all enemies.'' O King! Thus, accompanied by his soldiers, and raising a tremendous war-cry terrifying to the Devas, the Asura set out for battle. O Bh\=arata! The King of the Devas, knowing that the Asura is quite at hand, became overwhelmed with terror and ordered at once the soldiers to be ready for the battle and called quickly all the Lokap\=alas and sent them all for the battle. The highly lustrous Indra, the tormentor of the foes, arrayed his troops in order according to Gridhra Vy\=uha (the method in which the vultures arrange themselves while flying) and stayed there. On the other hand Vritra, the slayer of enemies, dashed unto that place with all swiftness. A dreadful fight then ensued between the Devas and D\=anavas; the two parties, desirous to get victory over the other, fought awfully hard. When the blaze of the battle fire shone to a very high pitch, the Devas dropped with sorrow while the Asuras became

excited with joy. The Devas and D\=anavas struck each other with Tomaras, Bhindip\=alas, axes, Para\'sus, Patti\'sas, and various other weapons. When the dreadful battle rose to a high pitch causing horripilation, Vritra became very angry and suddenly caught hold of Indra and denuding him of all clothes and armours swallowed him; he, then, remembering his former enmity, became very glad and stayed there. When Indra was thus devoured by Vritra, the Devas were overwhelmed with terror and cried out frequently, with great distress :-- ``O Indra! O Indra!'' All the Devas became very dejected and grieved in their hearts to see Indra denuded of his armour and clothes in the belly of Vritra and bowed down to Brihaspati and said :-- ``O Indra of the Br\=ahmans! You are our best Guru what are we to do now? Though the gods tried their best to save Indra still Vritra has devoured him. We are all powerless, what can we do without Indra? O Lord! Perform quickly magic spells (Abhich\=ara process) which will lead to our Indra's liberation.''

34. Brihaspati said :-- ``O Suras! The king of the gods is swallowed by Vritra, he has been quite disabled; but Indra is living in his bowels; attempt therefore must be made that he comes out while living.''

35-54. Vy\=asa said :-- O King! The Devas became very anxious to see Indra in that plight and took all the ways and means carefully how he might be freed. Then they created a state tending to cause yawning, very powerful and irresistible and calculated to destroy one's enemy. Vritr\=asura then yawned and his mouth got widely opened and extended. In the meanwhile Indra, the destroyer of one's enemie\'s strength, contracted all his limbs and came out of the expanded mouth of the Asura and fell down. Since that time, this state of yawning has become prevalent amongst the beings. The Devas were all glad to see Indra thus come out. When Indra thus got out, he fought again with Vritra for 10,000 years (Ajuta years). The fight was very dreadful, causing horripilation. On one side all the Devas joined in the battle; on the other side, the pre-eminently powerful Vritra, the son of Vi\'svakarm\=a fought. When Vritr\=asura got more and more energy in the battle, Indra became gradually dwindled and was at last defeated. Indra became very much grieved when he found himself defeated; the Devas also were very dejected to see this. Indra and the other Devas quitted the battle-field and fled away. Vritr\=asura too, quickly arrived and occupied the Heavens. Vritra began to enjoy by force the Heavenly gardens and took the Air\=avata elephant. O King! The Asura, the son of Tvast\=a, took away all Vim\=anas (the self-moving chariots of gods), Uchchai\'srava, the best of horses, the heavenly cow, the giver of desires, the P\=arij\=ata tree, the Apsar\=as, and all other jewels of the Heavens. The Devas, on the other hand, deprived of their shares in sacrifices

and driven away from their Heavens, suffered very much. Vritr\=asura became puffed up with vanity, when he got possession of the Heavens. Vi\'svakarm\=a, too, became very happy at that time and began to enjoy pleasures along with his son. O Bharata! The Devas, then, united with the Munis and they began to consult about their own welfare. When the Devas took Indra with them and went to Mah\=a Deva in the Mount Kail\=a\'sa and bowed down to His feet very humbly and, with folded hands, spoke thus :-- O Deva of the Devas! O Mah\=a Deva! Thou art the Mahe\'svara and the unbounded Ocean of Mercy! We are defeated by Vritr\=asura and we are very much terrified. Save us, O \'Sambhu! Thou dost good to all the beings; dost thou tell us, therefore, truly what are we to do now, when that powerful D\=anava has dispossessed us of our Heavens. O Mahe\'sa! Now dislodged, where are we to go? We are not finding any remedy by which our miseries can be destroyed. O Bh\=uta Bh\=avana! We are very much pained; help us; O merciful One! That Vritr\=asura has become intoxicated with vanity due to his being granted the boon. Therefore destroy him.

55-57. \'Sankara said :-- ``O Devas! We will keep Brahm\=a in the front and let all of us go to the residence of Hari and there consult with Him how to destroy this unruly Vritra. The Jan\=ardana V\=asudeva is fully capable to do all actions. He is powerful, knower of pretexts, highly intelligent, ocean of mercy, and fit to be asked by all for protection. Without Him, the Deva of the Devas, no success is possible in any action. Therefore all of us ought to go there for the success in our undertaking.''

58-62. Vyasa said :-- O King! Thus settling their plan of action, Indra and other Devas took \'Sankara and Brahm\=a with them and went to the abode of Hari, who protects all and is gracious to His devotees. They, then, began to chant Puru\d{s}as\=ukta hymns to Him and thus they praised the God Hari, the Guru of this Universe. The Jan\=ardan Hari, the Lord of Kamal\=a, then, appeared before them and, after showing his respect, addressed them thus :-- O Lord of the several Lokas! What have brought you all together with Brahm\=a and \'Sankara hither? O best of Suras! Please tell me the reason of your coming here. Vyasa said :-- O King! Thus hearing Hari's words, the Devas could not reply anything; rather almost all of them remained with an anxious look with their hands folded, overwhelmed with cares.

Here ends the Fourth Chapter of the Sixth Book on the defeat of the Devas by Vritra in the Mah\=a Pur\=a\d{n}am \'Sr\={\i} Mad Dev\={\i} Bh\=agavatam of 18,000 verses by Mahar\d{s}i Veda Vy\=asa.



