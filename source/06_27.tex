\chapter{On the marriage of N\=arada and his face getting transformed into that of a monkey}

1-13. N\=arada said :-- On hearing these words of her daughter from her nurse, the King addressed the queen Kaikey\={\i}, of lovely eyes, standing close by, thus :-- ``Have you heard what the nurse has said? Damayant\={\i} has mentally chosen the monkey-faced N\=arada as her husband. What has she thought? Whatever it be, it is no doubt, an act of great foolishness. His face is monkey-like; how can I betroth my daughter to him? Where is an ugly beggar N\=arada? And where is my daughter Damayant\={\i}? The marriage between them is quite unjust; never it should take place. O Beautiful One of good hairs! Better call her before you in private and show her reasons approved of the \'S\=astras and of the aged persons and make her desist from such a rash course.'' On hearing her husband's words, the mother of Damayant\={\i} called her in private and said :-- ``O Child! Where is your this beautiful face? And where is the monkey-like face of N\=arada? You are smart and quick; how have you been, then, deluded by such a Moha? O Child! You are the daughter of a king! Your body is gentle like a creeper. And N\=arada always besmears his body with ashes; so his body is very rough. O Spotless One! How will you change your words with him? Why do you shew your attachment to an ugly person? What pleasure do you feel thereby? You would be married to a beautiful prince; never follow this rash course; your father is very sorry to hear these from your nurse. O One of soft body! Judge this yourself, what intelligent man is there that is not sorry at the soft M\=alat\={\i} creeper entwining a thorny tree? Even a stupid silly man would never

feed a camel, that likes thorns, with soft betel-leaves. When your marriage time arrives, say yourself, who will not be sorry to see you going to N\=arada and embracing him by his arms! Nobody likes to speak with an ugly faced one; how will you be able to spend your time with him till your death!''

14-29. N\=arada said :-- On hearing the mother's words, the gentle Damayant\={\i}, with her mind intently fixed on me, spoke to her mother, very much depressed in her spirits. ``O Mother! What good face and beautiful form will avail, who is not in the path of love and who is quite ignorant of amorous feelings and sentiments! And what will the wealth and kingdoms of that unskilled illiterate person avail! The deer, that roam in the forest, getting enchanted by the N\=ada (sound) Rasa, give up their lives even to the singers. So they are fortunate. But fie to the persons who are illiterate and void of feelings of love! O Mother! N\=arada \d{R}i\d{s}i is well conversant with the science of music with seven Svaras. No other man save Mah\=a Deva knows this. Living with an illiterate person is courting death at every moment. One devoid of qualifications should be always avoided, by all means, though he be wealthy and of a beautiful form. Fie on the friendship with kings that are illiterate and puffed up with vain arrogance! A well-qualified man, be he even a beggar, is far better to be cultivated friendship with. Leaving other circumstances out of account, even to change words with such a well qualified man, makes one highly delighted. The man is very rare in this world, though he be weak, if he be well versed in the science of music and if he knows Svara, Gr\=ama, Murchchan\=a and be skilled in eight sentiments of love. [Note :-- Svara - Sadaja, Ri\d{s}abha, G\=a\d{n}dh\=ara, Madhyama, Panchama, Dhaivata and Ni\d{s}\=ada. Gr\=ama - the gradual increase and decrease in Svaras. Murchchana - the rising of sounds, an intonation; a duly regulated rise and fall of sound conducting the air and the harmony through the keys in a pleasing manner; changing the key or passing from one key to another; modulation; melody]. The man versed in the knowledge of Svara leads one to the Heaven of Kail\=a\'sa as the rivers Ganges and Sarasvat\={\i} by their own merits lead one to Kail\=a\'sa. There is not the least doubt in this. He is a Deva in his human body who knows the Svara measure; and he who does not know the Svara and its seven grades is a beast though he has a human form - he who finds no delight when he hears the tune regulated by Murchchan\=a and the seven Svaras. Do not consider the deer as beasts for they get enchanted when they hear the musical notes. The venomous snakes, though they have no ears, get delighted to hear the enchanting Svara N\=ada by their eyes. They even are to be praised;

but fie on those human beings who have ears but who do not find any delight when they hear the N\=ada! The little children feel intense pleasure to hear the music, but fie, fie on those elders who are void of this musical sentiments! Does not my father know that N\=arada is ornamented with many qualifications? Who is there in the three worlds like him in singing the S\=ama songs! For this very reason, indeed! I have already selected him as my husband; afterwards, due to a curse, the Muni, the ocean of qualifications, got his face changed into that of a monkey. The Kinnaras, skilled in the science of music, have their faces horse-like; but are they not dear to all? What business have they to get good faces? They enchant the Devas even by their sweet ravishing songs. O Mother! Kindly tell my father that I have already chosen N\=arada as my husband. Therefore let him deliver me to his hands, without making any further requests in this matter.''

30-40. N\=arada said :-- On hearing the words of her daughter Damayant\={\i}, that unblameable pure queen knowing her attachment deep towards me, spoke to the King thus :-- ``O King! Now celebrate in an auspicious day and on an auspicious moment the auspicious marriage of Damayant\={\i}; the daughter has said that she has already selected N\=arada as her bridegroom and it cannot be other-wise.'' Thus prompted by the queen, the King \'Sanjaya performed the marriage ceremony of her daughter in accordance with due rites and customs and in an exceedingly becoming manner. O \d{R}i\d{s}i! Thus I entered into the married life and remained there though my heart constantly burned with the thought of my monkey-face. Whenever the princess used to come to me for my service, I used to get tormented with the remembrance of my monkey-face; but her face beamed with gladness whenever she saw me; never she became sorry nor dejected, even for a moment, to see my face monkey-like. Thus time passed on. One day the Muni Parvata suddenly came there, after making his sojourn to many places of pilgrimages. I showed him a great respect and gladly loved him and greeted him duly; he got himself seated in an excellent \=Asana and became very sorry to see me. I am his uncle and have entered into a married life; my face has become monkey-like. Therefore I am very much depressed in spirits and worried with the sad thought and has become lean and thin. Seeing this he was overwhelmed with pity. He then said :-- ``O Muni! The curse that I cast on you before out of my anger, I now withdraw. Hear. O Mahar\d{s}i! Let your face be by my merits, again as good as it was before; I now feel pity for the daughter of the King.''

41-52. Hearing thus, my heart also became gentle and instantly with a view to free him of my curse, I said :-- ``Let your journey to

the Heavens be re-established. I now make this special favour on you as regards my curse on you before.'' O Dvaip\=ayana! At his word, before our sight, my face became exceedingly handsome as it was before. The princess Damayant\={\i} became very glad and instantly she went to the mother and said :-- ``O Mother! At the word of Parvata, the great Muni, the curse of your son-in-law has been removed and his face has become handsome as before and the lustre of his body has also increased.'' The queen was very much filled with ecstasy and joy at Damayant\={\i}'s words and went hurriedly and informed the King. The King \'Sanjaya gladly went at once to see the Muni. The great King became very glad and gave lots of wealth, gems and jewels to me and my nephew Parvata as a dowry. O Dvaip\=ayana! Thus I have described to you my old story how I felt the strong influence of M\=ay\=a. O Fortunate One! Owing to the illusory nature of the Gu\d{n}as, like a magic, no embodied being in this world could have been happy before, or he is happy now or he will be happy hereafter. Lust, anger, greed, jealousy, attachment, egoism, and vanity, each one of these is very powerful; nobody is able to conquer these. O Muni! The three Gu\d{n}as S\=attva, R\=ajas and T\=amas are the entire causes of the coming into this bodily existence of every being. O Dvaip\=ayana! Once I was passing with Bhagav\=an Vi\d{s}\d{n}u, laughing and joking, making merriments through a forest, when suddenly I was transformed into a woman. Next I became the wife of a king enchanted by M\=ay\=a, I remained in his house and gave birth to many children.

53-56. Vy\=asa said :-- O Devar\d{s}i! A great doubt has now arisen in my mind at your word. O Muni! You are very wise; how then did you get womanhood; how again did you regain your manhood? Who was the king at whose house you stayed and how did you give birth to children; describe fully and satisfy my curiosity. Describe to me, now, the nature of M\=ay\=a, extremely wonderful, by which this entire universe, moving and non-moving, all are enchanted. O Muni! Though I have heard your nectar-like words, capable to remove all the doubts, embodying the essence of all the \'S\=astras, yet I am not fully satiated.

Here ends the Twenty-seventh Chapter of the Sixth Book on the marriage of N\=arada and his face getting transformed into that of a monkey in \'Sr\={\i} Mad Dev\={\i} Bh\=agavatam of 18,000 verses by Mahar\d{s}i Veda Vy\=asa.



