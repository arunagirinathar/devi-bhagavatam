\chapter{On Vy\=asa doing his duties}

1-8. The \d{R}i\d{s}is said :-- ``O S\=uta! What did Veda Vy\=asa do, when the highest Yogi \'S\=uka, Deva-like, acquired all the excellent supernatural powers? Kindly describe all these in detail.''

Hearing this question, S\=uta spoke :-- O Rishis! Vy\=asa already had with him many disciples Asita, Devala, Vai\'samp\=ayana, Jaimini, Sumantu and others, all engaged in the study of the Vedas. After their studies were over, they all went out to propagate Dharma on the earth. Then Vy\=asa, seeing that the disciples went to the earth and his son \'S\=uka Deva had got to the next world, became very much distressed with sorrow and wanted to go to some other place. He then decided to go to his birth place and went to the banks of the Ganges and there remembered his auspicious mother Satyavat\={\i}, forsaken by him before, very sorrowful, and the daughter of a fisherman. He then quitted that heaven-like mountain, the source of all happiness and came to his own birth place. Reaching the island where he was born, he enquired the whereabouts of the beautiful faced, the fisherman's daughter as well the wife of a king. The fishermen replied that their king had given her in marriage to the king \'Santanu. Then the king of fishermen, seeing Vy\=asa there, gladly worshipped him and gave him a cordial welcome and spoke with folded palms, thus :--

9-16. O Muni! When I have become so fortunate as to see you, rare even to the Devas, then my birth has been sanctified today and you have purified my family. O Br\=ahmin! Kindly say what for have you come?

My wife, son and all my riches and every other thing that I have are at your disposal. Thus hearing the history of his mother Satyavat\={\i}, Vy\=asa erected an \=A\'srama on the beautiful banks of the river Sarasvat\={\i} and remained there in tapasy\=a with an enlightened mind.

Some time elapsed when the highly energetic \'Santanu got through his wife Satyavat\={\i} two sons. Vy\=asa Deva considered them as his two brothers and became very glad, though he himself used to live in the forest. The first son of the king \'Santanu was Chitr\=angada, endowed with all auspicious qualities, exceedingly beautiful, and tormenting his foes; the second son was Vichitra-v\={\i}rya; he was endowed with all qualities. The king \'Santanu became very happy to get these children. \'Santanu had one son before through his wife Gang\=a; he was a great hero and very powerful; and the two sons of Satyavat\={\i} were equally powerful. The high souled \'Santanu now seeing the three sons, all endowed with all auspicious qualities, began to think that the Devas were incapable to defeat him.

17-34. After some time, the religious \'Santanu quitted his worn-out body as a man quits his clothes worn out in due time. After the king \'Santanu had ascended the Heavens, the energetic Bh\={\i}\'sma performed duly his funeral obsequies and gave various things in charity to the Br\=ahma\d{n}as. He did not accept the kingdom himself; but placed Chitr\=angada on the throne and became known by the name of Devavrata (truthful in vow like the Devas). The pure souled Chitr\=angada, born of Satyavat\={\i} became so much powerful by sheer force of his arms, and became so great a hero that the enemies felt endless troubles. Now once on an occasion, the greatly powerful Chitr\=angada, surrounded by a great army, went on an hunting excursion to the forest in quest of rur\=u deer, etc., when the Gandarbha Chitr\=angada, seeing the king on the way, alighted from his chariot.

O ascetics! A fierce battle then ensued for three years on that sacred and wide expanse Kuruk\d{s}ettra between the two heroes, both equally powerful. In the battle, the king Chitr\=angada, the son of \'Santanu was slain by the Gandarbha Chitr\=angada and went up to Heavens. Bh\={\i}\'sma, born of the womb of Gang\=a, hearing the above news, expressed his sorrows and, being surrounded by the ministers completed all the funeral obsequies and installed Vichitrav\={\i}rya on the throne. The beautiful Satyavat\={\i} became very much agitated by the death of her son; but when the ministers and the highsouled spiritual teachers consoled her, she became glad when she saw that her youngest son became king. Vy\=asa Deva, too, felt himself glad to hear that his youngest brother had been made king. After some time when the all auspicious, Satyavat\={\i}'s son Vich\={\i}trav\={\i}rya attained his youth,

Bh\={\i}\'sma began to think of his marriage. At this time the king of K\=as\={\i} (K\=a\'s\={\i}r\=aj) called an assembly Svayamvara (where the kings are invited and the bride selects the bridegroom) for the marriage of her three daughters, endowed with all auspicious qualities, at one and the same time. Thousands and thousands of kings and princes from various countries were invited there in the assembly; and, worshipped duly, they went and decorated the hall. At that time the highly energetic fiery Bh\={\i}\'sma alone, mounting on his chariot, attacked the infantry and cavalry, and defeated all the kings assembled there, and perforce carried away the three daughters of K\=a\'s\={\i}r\=aj and took them to Hastin\=apur. Bh\={\i}\'sma behaved towards those three daughters as if they were mothers, sisters or daughters and informed Satyavat\={\i} without any delay of everything that had happened.

35-39. Then he called for the astrologers and Br\=ahmi\d{n}s, versed in the Vedas and enquired about the auspicious day for their marriage. When the day was fixed and when every preparation was made, the religious Bh\={\i}\'sma wanted Vichitrav\={\i}rya to marry them. At this time, the eldest daughter, beautiful-eyed spoke out modestly to the Gang\=a's son Bh\={\i}\'sma :-- ``O Gang\=a's son, the illustrious son of your family and the best of the Kurus! You are the best knower of Dharma; therefore what more shall I say to you. In the Svayamvara assembly I mentally selected \'S\=alva and it struck me that he, too, looked on me with a very loving heart towards me. So, O tormentor of foes! Now do what is fit for that sacred family; O Gang\=a's son! Not only you are extraordinarily powerful but you are also the foremost of the religious. S\=alva mentally wanted to marry me; now do as you like.''

40-44. When the eldest daughter spoke thus, Bh\={\i}\'sma asked the aged Br\=ahma\d{n}as, ministers and his mother ``What ought to be done now'' and, taking the opinions of all, spoke to that daughter :-- ``O beautiful one! You can go wherever you like.'' Thus saying, Bh\={\i}\'sma released her. Then the beautiful daughter of K\=as\={\i}r\=aj went to the house of S\=alvar\=aj and expressed to him her heart's desire :-- ``O great king! Knowing me attached to yourself, Bh\={\i}\'sma has quitted me according to the laws of Dharma; I have therefore come to you now; marry me. O best of the kings! I will be your legal wife, for already I used to think you as my husband and you, too, must have thought me your wife.''

45-47. \'S\=alva replied as follows :-- ``O beautiful one! When Bh\={\i}\'sma caught hold of your arm before me and took you to his chariot, then I won't marry you. You can say yourself what intelligent man can marry a woman touched by another? Therefore I won't marry you, though

Bh\={\i}\'sma has quitted you, in the light of another.'' Hearing these words of \'S\=alva, the daughter of K\=as\={\i}r\=aj wept bitterly; yet \'S\=alva quitted her. Therefore, finding no other way, she went back to Bh\={\i}\'sma weeping, and said as follows :--

48-50. O great warrior! \'S\=alva did not consent to marry me, as you first took me to the chariot and afterwards left me. So, O Mah\=abh\=aga! You better look to Dharma and marry me, as you know best what is Dharma. If you do not marry me, I will certainly quit my life. Hearing her words Bh\={\i}\'sma said :-- O beautiful one! How can I accept you, when your mind has become attached towards another. So, O fair one! You better go back soon to your own father with a calm, clear mind. When Bh\={\i}\'sma said thus, that daughter of K\=as\={\i}r\=aj did not go back, out of sheer shame, to her father's house, but went to a forest and in a greatly solitary place of pilgrimage began to practise asceticism.

51-56. Now the other two daughters of K\=as\={\i}r\=aj, beautiful and all auspicious Amb\=alik\=a and Ambik\=a became the wives of the king Vich\={\i}trav\={\i}rya. Thus the powerful king Vich\={\i}trav\={\i}rya began to enjoy various pleasures in the palace and in the gardens and thus passed his time. For full nine years the king Vich\={\i}trav\={\i}rya enjoyed the sexual pleasures and became attacked with consumption and fell into the jaws of death. Hearing the death news of her son Vich\={\i}trav\={\i}rya, Satyavat\={\i} became very sorry and surrounded by her ministers, performed his funeral obsequies. Then she spoke privately to Bh\={\i}\'sma with a grievous heart :-- ``O highly fortunately son! now you better govern your father's kingdom and see that the family of Yay\=ati does not become extinct. So better take your brother's wife and try your best to continue your family line.

57-74. Bh\={\i}\'sma then said :-- "O Mother! Did you not hear of the promise that I already made before my father? So I cannot ever marry and govern the kingdom.'' Hearing these words of Bh\={\i}\'sma, Satyavat\={\i} became anxious. She began to think as follows :-- ``How now the continuity of the family be kept! And it is not advisable to remain idle when the kingdom has become kingless; no happiness can be derived in this state.'' Thus thinking, she became exceedingly distressed; then the Gang\=a's son, Bh\={\i}\'sma spoke to her :-- ``O respected one! Do not worry your mind with cares; now take steps so as to secure a son from Vich\={\i}trav\={\i}rya's wife. Call some best Br\=ahmin, born of a good family and unite him with Vich\={\i}trav\={\i}rya's wife. There is no fault, as far as I know, in doing thus to keep up the family line. O sweet smiling one! Thus having begotten the grandson, give him this kingdom; I will also obey his commands.'' Hearing these reasonable words of Bh\={\i}\'sma, Satyavat\={\i} remem

bered her own son, the sinless Vy\=asa Deva, who was born to her during her virginity. As soon as Vy\=asa was remembered, he, the great ascetic and effulgent like the sun, came there and bowed down to his mother. The highly energetic Vy\=asa was then worshipped duly by Bh\={\i}\'sma and welcome by Satyavat\={\i} and began to rest there like a smokeless fire. The mother Satyavat\={\i} then spoke to the chief Muni :-- ``O son! Now procreate a beautiful son from your sperm and the ovum of Vich\={\i}trav\={\i}rya's wife.'' Hearing the mother's words, Vy\=asa considered them as Veda's injunction and thought they must be obeyed and promised before her that he must obey and and fulfil her orders. He remained there, waiting for the menstruation period. When the due period of menstruation arrived, Ambik\=a bathed and had a sexual intercourse with Vy\=asa and begot a very powerful son, but a blind one (since she closed her eyes at the sight of Vy\=asa during her intercourse). Seeing the son born blind Satyavat\={\i} became exceedingly sorry; she, then, asked her other son's wife :-- ``Go soon and get a son born of you in the aforesaid manner.'' When the menstruation period arrived, Amb\=alik\=a during the night time went to Vy\=asa and mixed and became pregnant. In due time a son was born; that child became of a very pale colour; so Satyavat\={\i} thought the new child, too, unfit for the kingdom; therefore at the end of the year again asked her son's wife Amb\=alik\=a to go to Vy\=asa. She asked Vy\=asa also for the same purpose and sent Amb\=alik\=a to his bed room. But Amb\=alik\=a became afraid, and could not go herself but sent her maid servant for the purpose. Thus from the womb of the maid servant the high souled Vidura was born, having Dharma's parts and the most auspicious towards all. Thus Vy\=asa begot three very powerful sons Dh\d{r}tar\=a\d{s}\d{t}ra, Pandu and Vidura for the continuity of the family line. O sinless Mahar\d{s}is! Thus I have described to you how my Guru Vy\=asa Deva, who knows well all the Dharmas, kept up the continuity of his family and how he begot sons in the womb of his brother Vich\={\i}trav\={\i}rya's wives, according to the laws of Dharma, to keep up a family.

Thus ends the twentieth chapter of the 1st Skandha as well as the first Skandha on Vy\=asa doing his duties in the Mah\=apur\=a\d{n}am \'Sr\={\i} Dev\={\i} Bh\=agavatam of 18,000 verses by Mahar\d{s}i Veda Vy\=asa.



