\chapter{On the birth of Lak\d{s}m\={\i} in the discourse of N\=arada and N\=ar\=aya\d{n}a}

1-2. N\=arada said :-- ``O Lord! How did the eternal Dev\={\i} Mah\=a Lak\d{s}m\={\i} the dweller in Vaikuntha, the beloved of N\=ar\=aya\d{n}a, the Presiding Deity of Vaikuntha, come down to the earth and how She, became the daughter of the ocean? By whom was She first praised? Kindly describe all these in detail to me and oblige.''

3-10. N\=ar\=aya\d{n}a said :-- O N\=arada! In ancient days when on Durv\=as\=a's curse, Indra was dispossesed of his kingdom, all the Devas came down to earth. Lak\d{s}m\={\i}, too, getting angry, quitted the Heavens, out of pain and sorrow and went to Vaikuntha and took the shelter of N\=ar\=aya\d{n}a. The Devas, then, went to Brahm\=a with their hearts full of sorrow and, taking Him from there, they all went to N\=ar\=aya\d{n}a in Vaikuntha. Going there they all took refuge of the Lord of Vaikuntha. They were very much distressed and their throats, palate and lips were quite dry. At that time Lak\d{s}m\={\i}, the wealth and prosperity of all, came down on earth by the command of N\=ar\=aya\d{n}a and became born in part as the daughter of the ocean. The Devas, then with the Daityas churned the K\d{s}\={\i}roda Ocean and, out of that, Mah\=a

Lak\d{s}m\={\i} appeared. Vi\d{s}\d{n}u looked on Her. Her joy knew no bounds. She smiling, granted boons to the Devas and then offered a garland of flowers on the neck of N\=ar\=aya\d{n}a (as a symbol of marriage celebrated). O N\=arada! the Devas, on the other hand, got back their kingdoms from the Asuras. They then worshipped and chanted hymns to Mah\=a Lak\d{s}m\={\i} and since then they became free from further dangers and troubles.

11-12. N\=arada said :-- ``O Bhagav\=an! Durv\=as\=a was the best of the Munis; he was attached to Brahm\=a and had spiritual knowledge. Why did he curse Indra? What offence had he committed? How did the Devas and Daityas churn the ocean? How, and by what hymns Lak\d{s}m\={\i} became pleased and appeared before Indra? What passed on between them. Say all this, O Lord.''

13-25. N\=ar\=aya\d{n}a said :-- In ancient days, Indra the Lord of the three worlds, intoxicated with wine and becoming lustful and shameless, began to enjoy Rambh\=a in a lonely grove. After having enjoyed her, he became attracted to her; his mind being wholly drawn to her, he remained there in that forest, his mind becoming very passionate. Indra then saw the Muni Durv\=as\=a on his way from Vaikuntha to Kail\=a\'sa burning with the fire of Brahm\=a. From the body of the \d{R}i\d{s}i, emitted, as it were, the rays of the thousand midday Suns. On his head was the golden matted hair. On his breast there was the hoary holy thread; he wore torn clothes; on his hands there was the Danda and Kamandalu; on his forehead there was the bright Tilaka in the form of the Crescent Moon. (Tilaka - a sectarian mark on the forehead made with coloured earth or sandalpaste.)

One hundred thousand disciples, thoroughly-versed in the Vedas and the Ved\=angas, were attending him. The intoxicated Purandara, seeing him, bowed down to him and he began to chant with devotion hymns to his disciples also. They were very glad. The \d{R}i\d{s}i with his disciples then blessed Indra and gave him one P\=arij\=ata flower.

When the Muni was returning from the region of Vaikuntha, Vi\d{s}\d{n}u, gave him that beautiful P\=arij\=ata flower. Old age, death, disease, sorrows, etc., all are removed by the influence of the flower; and the final liberation is also attained. The Devendra was intoxicated with his wealth; so taking the flower given by the \d{R}i\d{s}i, he threw it on the head of the elephant Air\=arata. No sooner the elephant touched the flower, than he became suddenly like Vi\d{s}\d{n}u, as it were, in beauty, form, qualities, fire and age. The elephant, then, forsook Indra and

entered into a dense forest. The Lord of the Devas could, in no way, get him under his control. On the other hand, the Muni Durv\=as\=a seeing that Mahendra had thus dishonoured the flower, became inflamed with rage and cursed him saying, ``O Indra! You are so mad with wealth that you have dishonoured me. The flower that I gave you so lovingly, you have thrown that, out of vanity, on the elephant's head!

26-46. No sooner one gets the food, water, fruits that had been offered to Vi\d{s}\d{n}u, one should eat that at once. Otherwise one incurs the sin of Brahmahatty\=a. If anybody forsakes the things offered to Vi\d{s}\d{n}u, that he has got perchance, he becomes destitute of wealth, prosperity, intelligence, and his kingdom. And if he eats the food already offered to Vi\d{s}\d{n}u with devotion, he then elevates his hundred families passed before him and he himself becomes liberated while living. If anybody daily eats Vi\d{s}\d{n}u's Naivedyam (food offered to Vi\d{s}\d{n}u) and bow down before Him or worships Hari with devotion and chants hymns to Him, he becomes like Vi\d{s}\d{n}u in energy and wealth. By mere touch with the air round about his body, the places of pilgrimage become all purified. O You Stupid! The earth becomes purified by the contact of the dust of the feet of such a one devoted to Vi\d{s}\d{n}u. If anybody eats the food unoffered to Hari and flesh that is not offered to any Deity; if he eats the food of any unchaste woman, any woman without husband and sons, the food offered at any \'S\=udra's Sr\=adh (funeral) ceremony, the food offered by a Br\=ahma\d{n}a, who is a priest to the \'S\=udras in honour of a \'Siva Lingam, the food of a Brahm\=a\d{n} priest who subsists on the presents of a temple, the food of one who sells his daughter, the food of one who subsists on dealing with womb concerns, the leavings of others, the stale food left after all others had eaten, the food of the husband of an unmarried girl (twelve years old in whom menstruation has commenced), the driver of oxen, the food of one uninitiated in one's Istamantram, of one who burns a corpse, of a Br\=ahmi\d{n} who goes to one not fit for going, the food of a rebel against friends, of one who is faithless, treacherous, who gives false evidence, the food of a Br\=ahmi\d{n} who accepts offerings in a sacred place of pilgrimage, all his sins (incurred in the ways above-mentioned) will be removed if he eats the pras\=adam of Vi\d{s}\d{n}u, i.e., the food offered to Vi\d{s}\d{n}u. Even if a Ch\=and\=ala be attached to the service of Vi\d{s}\d{n}u, he sanctifies his millions of persons born in his family before him. And the man who is devoid of the devotion to Hari is not able even to save himself. If anybody takes unknowingly the remains of an offering (such as flowers) made to Vi\d{s}\d{n}u,

he will certainly be freed from all the sins incurred in his seven births. And if he does this knowingly and with intense devotion, he will certainly be freed of all sorts of sins incurred in his Koti births. So, O Indra! I am a devotee of \'Sr\={\i} Hari. And when you have cast away the P\=arij\=ata to flower offered by me on the elephant's head, then I say unto you that the Mah\=a Lak\d{s}m\={\i} will leave you and She will go back to N\=ar\=aya\d{n}a. I am highly devoted to N\=ar\=aya\d{n}a; so I do not fear anybody, I fear neither the Creator, nor K\=ala, the Destroyer, nor old age, nor death; what to speak of other petty persons! I do not fear your father Praj\=apati Ka\'syapa nor do I fear your family priest Brihaspati. Now he, on whose head there lies the flower P\=arij\=ata offered by me, verily he should be worshipped by all means.'' Hearing these words of Durv\=as\=a, Indra became bewildered with fear, and being greatly distressed and holding the feet of Muni, cried out loudly. He said :-- ``The curse is now well inflicted on me; and it has caused my delusion to vanish. Now I do not want back my R\=aja Lak\d{s}m\={\i} from you; instruct me on knowledge. This wealth is the source of all coils; it is the cause of the veil to all knowledge, it hides the final liberation and it is a great obstacle on the way to get the highest devotion.''

47-67. The Muni said :--``This birth, death, old age, disease, and afflictions, all come from wealth and the manipulation of great power. Being blind by the darkness of wealth, he does not see the road to Mukti. The stupid man that is intoxicated with wealth is like the one that is intoxicated with wine. Surrounded by many friends, he is surrounded by the unbreakable bondage. The man that is intoxicated with wealth, blind with property and overwhelmed with these things has no thought for the real knowledge. He who is R\=ajasik, is very much addicted to passions and desires; he never sees the path to Sattvagu\d{n}a. The man that is blind with sense-objects is of two kinds, firstly, R\=ajasik and secondly T\=amasik. He who has no knowledge of the \'S\=astras is T\=amasik and he who has the knowledge of the \'S\=astras is R\=ajasik. O Child of the Devas! Two paths are mentioned in the \'S\=astras; one is Pravritti, going towards the sense objects and the other is Nivritti, going away from them. The J\={\i}vas first follow the path of Pravritti, the path that is painful, gladly and of their own accord like a mad man. As bees, blind with the desire of getting honey, go to the lotus bud and get themselves entangled there, so the J\={\i}vas, the embodied souls, desirous first of getting enjoyments come to this very painful circle of births and deaths, this worldly life, which in the end is realised as vapid and the only cause of old age, death, and sorrow and get themselves enchained there.

For many births he travels gladly in various wombs, ordained by his own Karmas, till at last by the favour of gods, he comes in contact with the saints. Thus one out of a thousand or out of a hundred finds means to cross this terrible ocean of world. When the saintly persons kindle the lamp of knowledge and show the way to Mukti, then the J\={\i}va makes an attempt to sever this bondage to the world. After many births, many austerities and many fastings, he then finds safely the way to Mukti, leading to the highest happiness. O Indra! What you asked me, I thus heard from my Guru.'' O N\=arada! hearing the words of the Muni Durv\=as\=a Indra became dispassionate towards the Sams\=ara. Day by day his feeling of dispassion increased. One day, when he returned to his own home from the hermitage of the Muni, he saw the Heavens overspread by the Daityas and it had become terrible. At some places outrage and oppression knew no bounds; some places were devoid of friends; at some places, some persons had lost their fathers, mothers, wives, relations; so no rest and repose could be found. Thus, seeing the Heavens in the hands of the enemies, Indra went out in quest of Brihaspati, the family preceptor of the Devas. Seeking to and fro Indra ultimately went to the banks of the Mand\=akin\={\i} and saw that the Guru Deva had bathed in the waters of the Mand\=akin\={\i} and sitting with his face turned towards the East towards the Sun, was meditating on Para Brahm\=a, Who has His faces turned everywhere. Tears were flowing from his eyes and the hairs of the body stood erect with delight. He was elderly in knowledge; the spiritual Teacher of all, religious, served by all great men; he was held as most dear to all the friends. Those who are J\~n\=anins regard him as their Guru. He was the eldest of all his brothers; he was considered as very unpopular to the enemies of the Devas. Seeing the family priest Brihaspati merged in that state of meditation, Indra waited there. When after one Prahara (three hours), the Guru Deva got up, Indra bowed down to his feet and began to weep and cry out repeatedly. Then he informed his Guru about his curse from a Br\=ahmi\d{n}, his acquiring the true knowledge as so very rare, and the wretched state of Amar\=avat\={\i}, wrought by the enemies.

68-92. O Best of Br\=ahma\d{n}as! Hearing thus the words of the disciple, the intelligent speaker Brihaspati spoke with his eyes reddened out of anger. ``O Lord of the Devas! I have heard everything that you said; do not cry; have patience; hear attentively what I say. The wise politicians of good behaviour, with moral precepts, never lose their heads and get themselves distressed in times of danger. Nothing is everlasting; whether property or adversity; all are transient;

they only give troubles. All are under one's own Karma; one is master of one's own Karma. What had been done in previous births, so one will have to reap the fruits afterwards. (Therefore property or adversity, all are due to one's own Karma.) This happens to all persons eternally, births after births. Pain and happiness are like the ring of a rolling wheel. So what pain is there? It is already stated that one's own Karma must be enjoyed in this Holy Bh\=arata. The man enjoys the effects of his own Karmas, auspicious or inauspicious. Never the Karma gets exhausted in one hundred Koti Kalpas, without their effects being enjoyed. The Karma, whether auspicious or inauspicious must be enjoyed. Thus it is stated in the Vedas and as well by \'Sr\={\i} Kri\d{s}\d{n}a, the Supreme Spirit. Bhagav\=an \'Sr\={\i} Kri\d{s}\d{n}a addressed Brahm\=a, the lotus-born, in the S\=ama Veda S\=akh\=a that all persons acquire their births, whether, in Bh\=arata or in any other country, according to the Karma that he had done. The course of a Br\=ahma\d{n}a comes though this Karma; and the blessings of a Br\=ahma\d{n}a come again by this Karma. By Karma one gets great wealth and prosperity and by Karma again one gets poverty. You may take one hundred Koti births; the fruit of Karma must follow you. O Indra! The fruit of Karma follows one like one's shadow. Without enjoyment, that can never die. The effects of Karma become increased or decreased according to time, place, and the person concerned. As you will give away anything to persons, of different natures, in different times and in different places, your merit acquired will also vary accordingly. Gifts made on certain special days bring in Koti times the fruits (merits, pu\d{n}yam) or infinite times or even more than that. Again gifts, similar in nature, made in similar paces yield pu\d{n}yam the same, in character also. Gifts made in different countries yield pu\d{n}yams, Koti times, infinite times, or even more than that. But similar things given to similar persons yield similar pu\d{n}yams. As the grains vary in their natures as the fields differ, so gifts made to different persons yield different grades of pu\d{n}yas infinitely superior or infinitely inferior as the case may be.

Giving things to a Br\=ahma\d{n}a on any ordinary day yields simple pu\d{n}ya only. But if the gift be made to a Brahm\=a\d{n} on an Amavasy\=a day (new moon day) or on a Sankr\=anti day (the day when the Sun enters another's sign) then hundred times more pu\d{n}yam is acquired. Again charities made on the Ch\=aturm\=asya period (the vow that lasts for four months in the rainy season) or on the full moon day, yield infinite pu\d{n}yams. So charities made on the occasion of the lunar eclipses yield Koti times the result and if made on the occasion of the solar eclipse yield ten times more pu\d{n}yams. Charities made on Ak\d{s}ayaya Trit\={\i}y\=a or the Navam\={\i} day yield infinite and endless results. So charities on other holy days yield religious merits

higher than those made on ordinary days. As charities made on holy days yield religious merits, so bathing, reciting mantrams, and other holy acts yield meritorious results. As superior results are obtained by pious acts; inferior results are obtained by impious acts. As an earthen potter makes pots, jars, etc., out of the earth with the help of rod, wheel, earthen cups or plates and motion, so the Creator awards respective fruits to different persons, by the help of this thread (continuity) of Karma. Therefore if you want to have cessation of this fruition of Karma, then worship N\=ar\=aya\d{n}a, by whose command all these things of Nature are created. He is the Creator of even Brahm\=a, the Creator, the Preserver of Vi\d{s}\d{n}u, the Preserver, the Destroyer of \'Siva, the Destroyer and the K\=ala (the great Time) of K\=ala (the Time). \'Sankara has said :-- He who remembers Madhus\=udana (a name of Vi\d{s}\d{n}u) in great troubles, his dangers cease and happiness begins.'' O N\=arada! The wise Brihaspati thus advised Indra and then embraced him and gave him his hearty blessings and good wishes.

Here ends the Fortieth Chapter of the Ninth Book on the birth of Lak\d{s}m\={\i} in the discourse of N\=arada and N\=ar\=aya\d{n}a in the Mah\=a Pur\=a\d{n}am \'Sr\={\i} Mad Dev\={\i} Bh\=agavatam of 18,000 verses by Mahar\d{s}i Veda Vy\=asa.



