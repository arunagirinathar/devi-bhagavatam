\chapter{On the glories of the Rudr\=ak\d{s}a beads}

1-21. \'Sr\={\i} N\=ar\=aya\d{n}a said :-- O N\=arada! There are the six kinds of \=Achamana :-- (1) \'Suddha, (2) Sm\=arta, (3) Paur\=a\d{n}ik, (4) Vaidik, (5) T\=antrik and(6) \'Srauta. The act of cleaning after evacuating oneself of urine and faeces is known as \'Suddha \'Saucha. After cleaning, the \=Achaman, that is

performed according to rules, is named as Sm\=arta and Paur\=a\d{n}ik. In places where the Brahm\=a Yaj\~na is performed, the Vaidik and \'Srauta \=Achamanas are done. And where acts, e.g., the knowledge of warfare are being executed, the T\=antrik \=Achaman is done. Then he is to remember the G\=ayatr\={\i} Mantra with Pra\d{n}ava (Om) and fasten the lock of hair on the crown of his head, thus controlling all the hindrances (Bighna Bandhanam). Sipping again, he is to touch his heart, two arms, and his two shoulders. After sneezing, spitting, touching the lower lip with teeth, accidentally telling a lie, and talking with a very sinful man, he is to touch his right ear (where the several Devas reside). On the right ear of the Br\=ahma\d{n}as reside Fire, Water, the Vedas, the Moon, the Sun, and the V\=ayu (wind). Then one is to go to a river or any other reservoir of water, and there to perform one's morning ablutions and to cleanse his body thoroughly. For the body is always unclean and dirty and various diets are being excreted out of the nine holes (doors) in the body. The morning bath removes all these impurities. Therefore the morning bath is essentially necessary. The sins that arise from going to those who are not fit for such purposes, from accepting gifts from impure persons or from the practice of any other secret vices all are removed by the morning ablutions. Without this bath, no acts bear any fruit. Therefore everyday, this morning bath is very necessary. Taking the Ku\'sa grass in hand, one is to perform one's bath and Sandhy\=a. If for seven days, the morning ablutions are not taken, and if for three days, the Sandhy\=as are not performed, if for twelve days, the daily Homas be not performed, the Br\=ahma\d{n}as become \'S\=udras. The time for making the Homa in the morning is very little; therefore lest ablutions be done fully which would take a long time and hence the time for the Homa might elapse, the morning bath should be performed quickly. After the bath the Pr\=a\d{n}\=ay\=ama is to be done. Then the full effects of bath are attained. There is nothing holier in this world or in the next than reciting the G\=ayatr\={\i}. It saves the singer who sings the G\=ayatr\={\i}; hence it is called G\=ayatr\={\i}. During the time of Pr\=a\d{n}\=ay\=ama, one must control one's Pr\=a\d{n}a and Ap\=a\d{n}a V\=ayus, i.e., make them equal. The Br\=ahmi\d{n}, knowing the Vedas and devoted to his Dharma, must practise Pr\=an\=ay\=ama three times with the repetition of G\=ayatr\={\i} and Pra\d{n}ava and the three Vy\=arhitis (Om Bhu, Om Bhuvar, Om Svah).

While practising, the muttering of G\=ayatr\={\i} is to be done three times. In Pr\=an\=ay\=ama, the Vaidik mantra is to be repeated, never a Laukika Mantra is to be uttered. At the time of Pr\=an\=ay\=ama, if anybody's mind be not fixed, even for a short while, like a mustard seed on the apex of a cow-horn, he cannot save even one hundred-and one persons in his father's

or in his mother's line. Pr\=an\=ay\=ama is called Sagarbha when performed with the repetition of some mantra; it is called Agarbha when it is done simply with mere meditation, without repeating any mantra. After the bathing, the Tarpa\d{n}am with its accompaniments, is to he done; i.e., the peace offerings are made with reference to the Devas, the \d{R}i\d{s}is, and the Pitris (whereby we invoke the blessings from the subtle planes where the highsouled persons dwell). After this, a clean pair of clothes is to be worn and then he should get up and come out of the water. The next things preparatory to practise Japam are to wear the Tilaka marks of ashes and to put on the Rudr\=ak\d{s}a beads. He who holds thirty-two Rudr\=ak\d{s}a beads on his neck, forty on his head, six on each ear (12 on two ears), twenty four beads on two hands (twelve on each hand) thirty-two beads on two arms (sixteen on each), one bead on each eye and one bead on the hair on the crown, and one hundred and eight beads on the breast, (251 in all) becomes himself Mah\=a Deva. One is expected to use them as such. O Muni! You can use the Rudr\=ak\d{s}as after tieing, stringing together with gold or silver always on your \'Sikh\=a, the tuft of hair on the head or on your ears. On the holy thread, on the hands, on the neck, or on the belly (abdomen) one can keep the Rudr\=ak\d{s}a after one has repeated sincerely and with devotion the five lettered mantra of \'Siva, or one has repeated the Pr\=a\d{n}\=ava (Om). Holding the Rudr\=ak\d{s}a implies that the man has realised the knowledge of \'Siva Tattva. O Brahm\=a\d{n}! The Rudr\=ak\d{s}a bead that is placed on the tuft or on the crown hair represents the T\=ara tattva, i.e., Om K\=ara; the Rudr\=ak\d{s}a beads that are held on the two ears are to be thought of as Deva and Dev\={\i}, (\'Siva and \'Siv\=a).

22-37. The one hundred and eight Rudr\=ak\d{s}a beads on the sacrificial thread are considered as the one hundred and eight Vedas (signifying the Full Knowledge, as sixteen digits of the Moon completed; on the arms, are considered as the Dik (quarters); on the neck, are considered as the Dev\={\i} Sarasvat\={\i} and Agni (fire). The Rudr\=ak\d{s}a beads ought to be taken by men of all colours and castes. The Br\=ahma\d{n}as, K\d{s}attriyas and Vai\'syas should hold them after purifying them with Mantras, i.e., knowingly; whereas the \'S\=udras can take them without any such purification by the Mantras, i.e., unknowingly. By holding or putting on the Rudr\=ak\d{s}a beads, persons become the Rudras incarnate in flesh and body. There is no doubt in this. By this all the sins arising from seeing, hearing, remembering, smelling, eating prohibited things, talking incoherently, doing prohibited things, etc., are entirely removed with the Rudr\=ak\d{s}a beads on the body; whatever acts, eating, drinking, smelling, etc., are done, are, as it were, done by Rudra Deva Himself. O Great Muni! He who feels

shame in holding and putting on the Rudr\=ak\d{s}a beads, can never be freed from this Sams\=ara even after the Koti births. He who blames another person holding Rudr\=ak\d{s}a beads has defects in his birth (is a bastard). There is no doubt in this. It is by holding on Rudr\=ak\d{s}a that Brahm\=a has remained steady in His Brahm\=ahood untainted and the Munis have been true to their resolves. So there is no act better and higher than holding the Rudr\=ak\d{s}a beads. He who gives clothing and food to a person holding Rudr\=ak\d{s}a beads with devotion is freed of all sins and goes to the \'Siva Loka. He who feasts gladly any holder of such beads at the time of \'Sr\=adh, goes undoubtedly to the Pitri Loka. He who washes the feet of a holder of Rudr\=ak\d{s}a and drinks that water, is freed of all sins and resides with honour in the \'Siva Loka. If a Br\=ahma\d{n}a holds with devotion the Rudr\=ak\d{s}a beads with a necklace and gold, he attains the Rudrahood. O Intelligent One! Wherever whoever holds with or without faith and devotion the Rudr\=ak\d{s}a beads with or without any mantra, is freed of all sins and is entitled to the Tattvaj\~n\=ana. I am unable to describe fully the greatness of the Rudr\=ak\d{s}a beads. In fact, all should by all means hold the Rudr\=ak\d{s}a beads on their bodies.

Note :-- The number one hundred and eight (108) signifies the One Hundred and Eight Vedas, the Brahm\=a\d{n}, the Source of all Wisdom and Joy.

Here ends the Third Chapter of the Eleventh Book on the glories of the Rudr\=ak\d{s}a beads in the Mah\=a Pura\d{n}am \'Sr\={\i} Mad Dev\={\i} Bh\=agavatam of 18,000 verses by Mah\=ar\d{s}i Veda Vy\=asa.



