\chapter{On the narration to Haihaya the stealing away of Ek\=aval\={\i}}

1-10. Ya\'sovat\={\i} spoke :-- O King! One day Ek\=aval\={\i} got up early in the morning and went to the banks of the Ganges, accompanied by her companions; they began to fan her with a chowrie. The armed guards accompanied her. Slowly she went where there were the lotuses in order to sport with them. I, too, went with her playing with the lotuses to the banks of the Ganges and both of us began to play with lotuses with the Apsar\=as. When both of us were deeply engaged in the play, then one powerful D\=anava, named K\=alaketu, came up there suddenly with many R\=ak\d{s}asas armed with parighas, swords, clubs, bows, arrows and tomaras and many other weapons. Ek\=aval\={\i} was playing with the best lotuses when K\=alaketu saw her in that state, blooming with beauty and youth as if like Rat\={\i}, the Goddess of Love. O King! I then spoke to Ek\=aval\={\i} :-- ``Look! Who is this Daitya that has come here unexpectedly; O Lotus-eyed One! Let us go into the central part of our armed guards.'' O King! My companion and myself consulting thus, went out of fear immediately into the centre of the armed guards. K\=alaketu was seized with the arrows of Cupid, and no sooner he looked at that beautiful young lady than he, with a very big club in his hand, hurriedly came to us, drove away the guards, and took away my lotus eyed companion, of thin waist. Then the young lady, helpless, began to tremble and cried aloud.

11-22. Seeing this, I spoke to the D\=anava :-- Leave her and take me. The passionate D\=anava did not take me but he went away, taking my

companion. The guards exclaimed :-- ``Wait, wait; don't fly away with the girl; we are giving you a good lesson.'' Thus saying, they made the powerful D\=anavas stop and both the parties engaged in a very terrible conflict, astounding to all. The followers of the D\=anavas, more cruel and all fully armed began at once to fight with great enthusiasm for their Master's cause. K\=alaketu himself began to fight afterwards terribly and killed the guards. He, then, with his followers, carried away my companion towards his own city. I, too, followed my companion, when I saw her thus carried away by the D\=anava and crying out of fear. I also walked crying aloud by those tracks as would enable my Sakh\={\i} to see me. She, too, seeing me, became somewhat consoled. Crying out repeatedly I approached her. She was already very distressed and when she saw me, she clasped me closely around my neck, perspiring and stunned and, becoming more distressed, cried aloud. K\=alaketu then showed his liking for me and told that my quick-eyed companion was very afraid and that I might comfort her. He told me thus :-- ``O Dear! My city is like the Deva's abodes; you will soon be able to go there. From today I become your slave, bound by love. Do not cry thus distressedly; be comforted.'' In these words he told me to comfort my dear companion. Thus speaking, that villain made both of us mount on the beautiful chariot and making us sit by his sides went gladly and quickly to his own beautiful palace, followed by his army.

23-30. That Demon placed both of us in a beautiful house white washed and mirror-like and kept hundreds and thousands of R\=ak\d{s}asas to watch and protect us. On the second day he called me in private :-- ``Your companion is very much distressed on the bereavement from her father and mother and is lamenting; make her understand and console her.'' He told me to speak the following words to my companion :-- ``O One of beautiful hips! Be my wife and enjoy as you like. O One with a face beautiful like the Moon! This kingdom is yours; ever I am your obedient slave.'' Hearing his unbearable harsh words I said :-- ``O Lord! I will not be able to speak her these words, disagreeable to her. You better speak this yourself.'' When I spoke thus, that wicked D\=anava struck by the arrows of Cupid began to speak gently to my dear companion of thin belly, thus :-- ``O Dear One! Today you have successfully cast on me the Vas\={\i}kara\d{n}a Mantra (one of the T\=antrik processes by which a lover is made to come under subjection); O Beloved! Therefore it is that my heart is stolen and so much brought under your subjection; this has converted me into a veritable slave of yours; then know this as certain that I am your slave; O Sweet One!

I am very much troubled by the Cupid's arrows and I am semi-unconscious; therefore O Lean-bellied One! Worship me. O One of beautiful thighs! This youth is a very rare and transient thing; O Auspicious One! Now embrace me as your husband and make your youth a veritable success.''

31-36. Ek\=aval\={\i} said :-- ``O Fortunate One! My father wanted to hand me over to a prince named Haihaya; I also mentally adopted him as my husband. You are certainly aware of the \'S\=astras; how can I now abandon the eternal religion and the virtue of a woman and take up another husband. The girl must accept him to whom the father betrothes. The girl is under every circumstances dependent. Never do they get any independence.'' Though Ek\=aval\={\i} said thus, the vicious Daitya struck by Cupid's arrows, did not desist and did not leave me and that larged-eyed companion. His city lies in P\=at\=ala and is a very dangerous place; always it is guarded by R\=ak\d{s}asas and surrounded by a moat; inside is built a beautiful and strong fort. Now my dear companion, the queen of my heart, is staying there with a grievous heart and I am here wandering hither and thither very much troubled on account of her bereavements.

37-46. Ekav\={\i}ra said :-- ``O Beautiful-faced One! How have you been able to get away from the city of that wicked demon and how have you been able to come here? I am perfectly at my wit's end. Say quickly all these. O Proud One! I doubt your words; the father of your dear companion resolved to give his daughter to Haihaya in marriage; now I am that Haihaya. I am the King by that name, on this earth; there is no other King by the name of Haihaya. Is it that your dear companion is meant for me? O Bh\=amin\={\i} (passionate woman)! Remove my doubts; I will kill that villain R\=ak\d{s}asa and bring just now your dear companion; there is no doubt in this. O One of good vows! Shew me that place, if it be known to you. Has anybody informed her father that She is suffering from so many troubles? Has her father come to know that her daughter has been stolen and carried away? And has he made any effort to rescue her from the hands of that villain R\=ak\d{s}asa? Is it that the King is calm and quiet, knowing that his daughter has been kept in prison? Or is it that he is unable to free her from bondage? Say quickly all these things before me. O Lotus-eyed One! You have captivated my mind by describing the extraordinary qualities of your dear companion and have made me passionate too. Alas! When will it be that I will free my beautiful beloved from the greatest perilous situation and shall see her face and her eyes, beaming with joy. O Sweet-speaking One! Say, by what means I can go to that impassable city. How have you been able to come from there?''

47-63. Ya\'sovat\={\i} said :-- O King! In my early age I got the Mantram of the Dev\={\i} Bhagavat\={\i} with its seed Mantram (mystic syllable involving in it the power connoted by the Dev\={\i}) and how to meditate it. While I was in the D\=anava's place I thought out that at that juncture I would worship the powerful Chandik\=a who gives instantaneously one's own desires. If I worship that \'Sakti, That fructifies all desires, That is all mercy to Her Bhaktas, certainly She will free my dear companion from this her bondage. Though She is really without form, yet She, without anybody's help, by Her own force, She is creating, preserving, and at the end of the Kalpa, destroying this Universe. Oh! She is very wonderful indeed! Thus thinking I began to meditate that auspicious red-robed and red-eyed Dev\={\i}, the Lady of the Universe, and recollected mentally Her form and repeated silently Her V\={\i}ja Mantram. When I meditated thus for merely one month, Chandik\=a Dev\={\i} became, through my devotion, manifest to me in my dreams and began to speak in sweet nectar-like words :-- ``You are now asleep; go quickly to the beautiful banks of the Ganges. The enemy destroyer, the powerful Ekav\={\i}ra, the greatest of all the kings, will come there. Datt\=atreya, the Great Lord of the Munis, has given him my Mantra named Mah\=avidy\=a; the King also constantly worships me devotedly with that. His mind is constantly attached to Me and he constantly worships Me. What more to say than this fact that the king, extremely devoted to Me, meditates on Me as the internal controller of all beings. That intelligent son of Lak\d{s}m\={\i} will come for sport to the banks of the Ganges and will remove all your sorrows. That king Ekav\={\i}ra, versed in all the \'S\=astras will kill the R\=ak\d{s}asas in a dreadful battle and will rescue Ek\=aval\={\i}. So now you pay heed to my word.'' Lastly, She told me that my companion should marry that beautiful King, endowed with all auspicious qualifications. Thus saying, She disappeared and I instantly woke up. Then I informed my lotus-eyed dear Sakh\={\i} all the details of my dream as well my worshipping the Dev\={\i}; hearing this, her lotus-face beamed with joy and gladness. That sweet-smiling Ek\=aval\={\i} very gladly told me, ``O dear Companion! Go at once for our success. That truth-speaking Bhagavat\={\i} Ambik\=a Dev\={\i} will release us from our bondage.'' O King! When my dear companion ordered me thus, I thought it proper, as also dictated to me in my dream, to go out and soon I did it. O King! Due to the grace of the Great Dev\={\i}, I came to know the way and I also got the quick motion. Thus I have described to you the cause of my sorrow. O Hero! Who are you, whose son are you? Speak truly to me.

Here ends the Twenty-second Chapter in the Sixth Book on the narration to Haihaya the stealing away of Ek\=aval\={\i} in \'Sr\={\i} Mad Dev\={\i} Bh\=agavatam of 18,000 verses by Mahar\d{s}i Veda Vy\=asa.



