\chapter{On Manas\=a's story}

1-25 N\=ar\=aya\d{n}a said :-- O N\=arada! I have now narrated the anecdote of Sasth\={\i} as stated in the Vedas. Now hear the anecdote of Mangala Chand\={\i}, approved of by the Vedas and respected by the literary persons. The Chand\={\i}, that is very skilled in all auspicious works and who is the most auspicious of all good things, is Mangal Chand\={\i}k\=a. Or the Chand\={\i} who is an object of worship of Mangala (Mars), the son of earth and the bestower of desires is Mangala Chand\={\i}k\=a. Or the Chand\={\i} who is an object of worship of Mangala of the family of Manu who was the ruler of the whole world composed of seven islands and the bestower of all desires is Mangala Chand\={\i}. Or it may be that the M\=ula Prakriti, the Governess, the Ever Gracious Durg\=a assumed the form of Mangala Chand\={\i} and has become the Ista Devat\=a of women. When there was the fight with Tripur\=asura, this Mangala Chand\={\i}, higher than the highest was first worshipped by Mah\=adeva, stimulated by Vi\d{s}\d{n}u, on a critical moment. O Br\=ahmi\d{n}! While the fighting was going on, a Daitya threw out of anger one car on Mah\=adeva and as that car was about to fall on Him, Brahm\=a and Vi\d{s}\d{n}u gave a good advice when Mah\=adeva began to praise Durg\=a Dev\={\i} at once. Durg\=a Dev\={\i} that time assuming the form of Mangala Chand\={\i} appeared and said ``no fear no fear'' Bhagav\=an Vi\d{s}\d{n}u will be Thy Carrier buffalo. I will be also Thy \'Sakti in the action and Hari, full of M\=ay\=a, will also help Thee. Thou better slayest the enemy that dispossessed the Devas. O Child! Thus saying, the Dev\={\i} Mangala

Chand\={\i} disappeared and She became the \'Sakti of Mah\=a Deva. Then with the help of the weapon given by Vi\d{s}\d{n}u, the Lord of Um\=a killed the Asura. When the Daitya fell, the Devas and \d{R}i\d{s}is began to chant hymns to Mah\=adeva with devotion and with their heads bent low. From the sky, a shower of flowers fell instantaneously on Mah\=a Deva's head. Brahm\=a and Vi\d{s}\d{n}u became glad and gave their best wishes to Him. Then ordered by Brahm\=a and Vi\d{s}\d{n}u, \'Sankara bathed joyously. Then He began to worship with devotion the Dev\={\i} Mangala Chand\={\i} with p\=adya, Arghya, \=Achaman\={\i}ya and various clothings. Flowers, sandal paste, various goats, sheep, buffaloes, bisons, birds, garments, ornaments, garlands, P\=ayasa (a preparation of rice, ghee, milk and sugar), Pistaka, honey, wine, and various fruits were offered in the worship. Dancing, music, with instruments and the chanting of Her name and other festivals commenced. Reciting the Dhy\=an as in M\=adhyandina, Mah\=adeva offered everything, pronouncing the principal Radical Mantra, ``Om Hr\={\i}m \'Sr\={\i}m Kl\={\i}m Sarvapujye Dev\={\i} Mangala Chand\={\i}ke Hum Phat Sv\=ah\=a'' is the twenty-one lettered Mantra of Mangala Chand\={\i}. During worship, the Kalpa Vrik\d{s}a, the tree yielding all desires, must be worshipped. O N\=arada! By repeating the Mantra ten lakhs of times, the Mantra Siddhi (success in realising the Deity inherent in the Mantra) comes. Now I am saying about the Dhy\=anam of Mangal Chand\={\i} as stated in the Vedas and as approved by all. Listen. ``O Dev\={\i} Mangala Chand\={\i}ke! Thou art sixteen years old; Thou art ever youthful; Thy lips are like Bimba fruits, Thou art of good teeth and pure. Thy face looks like autumnal lotus; Thy colour is like white champakas; Thy eyes resemble blue lilies; Thou art the Preserver of the world and thou bestowest all sorts of prosperity. Thou art the Light in this dark ocean of the world. So I meditate on Thee.'' This is the Dhy\=anam. Now hear the stotra, which Mah\=adeva recited before Her.

26-37. Mah\=adeva said :-- Protect me, Protect me, O Mother! O Dev\={\i} Mangal Chand\={\i}ke! Thou, the Destroyer of difficulties! Thou givest joy and good. Thou art clever in giving delight and fortune. Thou the bestower of all bliss and prosperity! Thou, the auspicious, Thou art Mangala Chand\={\i}k\=a. Thou art Mangal\=a, worthy of all good, Thou art the auspicious of all auspicious; Thou bestowest good to the good persons. Thou art worthy to be worshipped on Tuesday (the Mangala day); Thou art the Deity, desired by all. The King Mangala, born of Manu family always worships Thee. Thou, the presiding Dev\={\i} of Mangala; Thou art the repository of all the good that are in this world. Thou, the Bestower of the auspicious Mok\d{s}a. Thou, the best of all; Thou, the respository of all good; Thou makest one cross all the Karmas; the people worship Thee on every Tuesday; Thou bestowest abundance of Bliss to all. Thus praising Mangala Chand\={\i}k\=a with this stotra, and worshipping on every

Tuesday, \'Sambhu departed. The Dev\={\i} Sarva Mangal\=a was first worshipped by Mah\=adeva. Next she was worsh ipped by the planet Mars; then by the King Mangala; then on every Tuesday by the ladies of every household. Fifthly she was worshipped by all men, desirous of their welfare. So in every universe Mangal Chand\={\i}k\=a, first worshipped by Mah\=adeva, came to be worshipped by all. Next she came to be worshipped everywhere, by the Devas, Munis, M\=anavas, Manus. O Muni! He who hears with undivided attention this stotra of the Dev\={\i} Mangala Chand\={\i}k\=a, finds no evils anywhere. Rather all good comes to him. Day after day he gets sons and grandsons and so his prosperity gets increased, yea, verily increased!

38-58. N\=ar\=aya\d{n}a said :-- O N\=arada! Thus I narrated to you the stories of Sasth\={\i} and Mangala Chand\={\i}k\=a, according to the Vedas. Now hear the story of Manas\=a that I heard from the mouth of Dharama Deva.

Manas\=a is the mind-born daughter of Mahar\d{s}i Ka\'syapa; hence she is named Manas\=a; or it may be She who plays with the mind is Manas\=a. Or it may be She who meditates on God with her mind and gets rapture in Her meditation of God is named Manas\=a. She finds pleasure in Her Own Self, the great devotee of Vi\d{s}\d{n}u, a Siddha Yogin\={\i}. For three Yugas She worshipped \'Sr\={\i} Kri\d{s}\d{n}a and then She became a Siddha Yogin\={\i}. \'Sr\={\i} Kri\d{s}\d{n}a, the Lord of the Gop\={\i}s, seeing the body of Manas\=a lean and thin due to austerities, or seeing her worn out like the Muni Jarat K\=aru called her by the name of Jarat K\=aru. Hence Her name has come also to be Jarat K\=aru. Kri\d{s}\d{n}a, the Ocean of Mercy, gave her out of kindness, Her desired boon; She worshipped Him and \'Sr\={\i} Kri\d{s}\d{n}a also worshipped Her. Dev\={\i} Manas\=a is known in the Heavens, in the abode of the N\=agas (serpents), in earth, in Brahm\=aloka, in all the worlds as of very fair colour, beautiful and charming. She is named Jagad Gaur\={\i} as she is of a very fair colour in the world. Her other name is \'Saiv\={\i} and she is the disciple of \'Siva. She is named Vai\d{s}\d{n}av\={\i} as she is greatly devoted to Vi\d{s}\d{n}u. She saved the N\=agas in the Snake Sacrifice performed by Pariksit, she is named Nage\'svar\={\i} and N\=aga Bhagin\={\i} and She is capable to destroy the effects of poison. She is called Vi\d{s}ahari. She got the Siddha yoga from Mah\=adeva; hence She is named Siddha Yogin\={\i}; She got from Him the great knowledge, so she is called Mah\=a J\~nanayut\=a, and as she got Mritasamj\={\i}van\={\i} (making alive the dead) she is known by the name of Mritasanj\={\i}van\={\i}. As the great ascetic is the mother of the great Muni \=Ast\={\i}k, she is known in the world as \=Ast\={\i}ka m\=at\=a. As She is the dear wife of the great high-souled Yogi Muni Jarat K\=aru, worshipped by all, she is called as Jarat
K\=arupriya, Jaratk\=aru, Jagadgaur\={\i}, Manas\=a, Siddha Yogin\={\i}, Vai\d{s}\d{n}av\={\i},

N\=aga Bhagin\={\i}, \'Saivi, Nage\'svar\={\i}, Jaratk\=arupriy\=a, \=Astikam\=at\=a, Vi\d{s}ahari, and Mah\=a J\~nanayut\=a; these are the twelve names of Manas\=a, worshipped everywhere in the Universe. He who recites these twelve names while worshipping Manas\=a Dev\={\i}, he or any of his family has no fear of snakes. If there be any fear of snakes in one's bed, if the house be infested with snakes or if one goes to a place difficult for fear of snakes or if one's body be encircled with snakes, all the fears are dispelled, if one reads this stotra of Manas\=a. There is no doubt in this. The snakes run away out of fear from the sight of him who daily recites the Manas\=a stotra. Ten lakhs of times repeating the Manas\=a mantra give one man success in the stotra. He can easily drink poison who attains success in this stotra. The snakes become his ornaments; they carry him even on their backs. He who is a great Siddha can sit on a seat of snakes and can sleep on a bed of snakes. In the end he sports day and night with Vi\d{s}\d{n}u.

Here ends the Forty-seventh Chapter of the Ninth Book on Manas\=a's story in the Mah\=a Pur\=a\d{n}am \'Sr\={\i} Mad Dev\={\i} Bh\=agavatam of 18,000 verses by Mahar\d{s}i Veda Vy\=asa.



