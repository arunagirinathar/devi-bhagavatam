\chapter{On the killing of Chanda and Munda}

1-17. Vy\=asa said :-- O King! Thus ordered, the two strong warriors Chanda and Munda hurriedly went to the battle, accompanied by a vast army. There they saw the Dev\={\i}, intent on doing good to the Gods. Then they began to address Her in conciliatory words. O Lady! Dost Thou not know that the extraordinarily strong \'Sumbha and Ni\'sumhha, the Lords of the Daityas have crushed down the Devas and vanquished Indra and have become intoxicated with their victory? O Fair One! Thou art alone! Only K\=alik\=a and Thy Lion are with Thee! It is Thy foolishness that Thou art desirous to conquer \'Sumbha, who is endowed with all power. I think there is no adviser to Thee, man nor woman; the Devas have sent Thee here simply for Thy destruction. Think, O Delicate One! over the powers of Thine as well as those of Thy enemy. Vainly dost Thou boast of Thy having eighteen hands. Before the great warrior \'Sumbha, the conqueror of the Devas, many hands and many weapons will be useless; they will prove mere burdens. So dost Thou fulfil what reigns in the heart of \'Sumbha, the destroyer of the legs and the uprooter of the teeth of Air\=avata elephant. Vain is Thy boast, O Beloved! Follow my sweet words; they will do good to Thee, O Large-eyed One! They will destroy Thy pains and give Thee bliss. Those actions that lead to pain are to be avoided by the wise; and those that bring in happiness are to be served by the Pundits, versed in the \'S\=astras. O Sweet speaking one! Thou art clever. Look at the great strength of \'Sumbha with Thy eyes. He has enhanced his glory by crushing down the Devas. And if Thou thinkest the gods superior, that is false, for the wise men do not rely on the mere guess, full of doubts; they believe what they actually see. \'Sumbha, hard to be conquered in battles, is the great enemy of the Gods; they have been crushed down by him, and have therefore sent Thee here. O Sweet smiling One! Thou hast been deceived by their sweet words; they, prompted by their selfish ends, have sent Thee here simply to give Thee trouble. The friends that come with certain business and selfish ends are to be rejected. Friends out of religious motives are only to be sought for refuge. Verily I tell Thee that the gods are terribly selfish. Therefore dost Thou serve \'Sumbha, the conqueror of Indra and the lord of the three worlds; he is a hero, beautiful, lovely, cunning and thoroughly expert in the

science of amorous love. Thou wilt get the prosperity of all the worlds by the mere command of \'Sumbha; therefore make a firm resolve and serve that splendid husband \'Sumbha.

18-30. Vy\=asa said :-- O King! The Universal Mother, hearing the words of Chanda spoke with a voice deep as thunder. O Boor! Why do you use false deceitful words? Fly away just now. Why shall I make \'Sumbha my husband, disregarding Hari, Hara and the other Devas? O You, a veritable Fool! I have no necessity for My lord; I have got nothing to do with my lord. I Myself am the Lord of all the beings; and I preserve this whole Universe with all the lords and beings therein. Note this. In ancient times I saw thousands and thousands of \'Sumbha and Ni\'sumbha and I slew them all. I sent hundreds and hundreds of Daityas and Demons to the realm of Death. Before Me the hosts of Devas were destroyed in yugas after yugas. Today the Daityas again will go unto destruction. The Time has come to destroy the Daityas; why, then, are you struggling in vain with your followers for your lives? Fight now and keep the Dharma of the warriors; death is inevitable; thinking thus, the high-souled ones should keep their name, fame, and respect. What business have you to do with \'Sumbha and Ni\'sumbha? Follow the warrior's Dharma and go to Heavens, the abode of the gods. \'Sumbha, Ni\'sumbha and your other friends and followers, all will follow you and will come here no doubt. O Stupid One! I will put an end to all the D\=anavas today. Therefore cast aside your weakness and go on, fight. I will slay you and your brother just now; next I will kill the proud Rakta V\={\i}ja, Ni\'sumbha and \'Sumbha and the other D\=anavas in the battle field and will then go to My desired place. Now remain here if you like or fly away quickly. You have been fed in vain because you fear to fight. What use is there now in using sweet words like a weak and distressed man. Well! Take up your arms now and fight.

31-61. Vy\=asa said :-- O King! Chanda and Munda, elated with pride got excited at the Dev\={\i}'s words, became angry and made a violent noise with their bow strings. The Dev\={\i}, too, blew Her conchshell so loudly that the ten quarters of the sky reverberated; in the meanwhile, the powerful lion became very angry and roared loudly. Hearing that sound Indra and other Devas, the Munis, Yak\d{s}as, Siddhas, and Kinnaras became all very glad. A dreadful fight than ensued between Chandik\=a and Chanda with arrows, axes and other weapons, causing terror to the weak. Then Chandik\=a Dev\={\i} became very wrathful and cut off to pieces all the arrows shot by Chanda and then hurled arrows serpent-like on him. Then the sky over the battle ground seemed to be overcast with arrows just as the

clouds get covered over with locusts, dreadful to the cultivators. In the meanwhile Munda, exceedingly terrible, came up to the field, taking with him his army and becoming impatient with anger began to shoot arrows. Seeing that multitude of arrows, Ambik\=a got very angry; out of Her frowny look, Her eyebrows became crooked, Her face became black, and Her eyes turned red like Kadal\={\i} flowers; at this time suddenly came out of Her forehead K\=al\={\i}. Wearing the tiger's skin, cruel, covering Her body with elephant's skin, wearing a garland of skulls, terrible, with a belly like a well dried up, mouth wide open, with a wide waist, lip hanging loosely, with axe, noose, \'Siva's weapon, in Her hands, She looked very terrible like the Night of Dissolution. She began to lick frequently and forcibly dashed into the D\=anava army and began to destroy it. She angrily began to take the powerful D\=anavas by Her arms and pouring them into Her mouth crushed them with Her teeth. Taking the elephants with bells by Her own power in Her hands She put them all into Her mouth and swallowed them all with their riders and began to laugh hoarsely. Thus camels, horses and charioteers with chariots all She put into Her mouth and began to chew them all grimly. O King! Seeing that the forces were being thus destroyed, the two great warriors Chanda and Munda began to shoot arrows after arrows without intermission and covered the Dev\={\i} with them. Chanda hurled the Sudar\d{s}an-like disc, lustrous like the Sun, with great force against the Dev\={\i}, and frequently shouted thundering cries. Seeing him roaring and the lustrous disc coming towards Her like another sun, She shot at him arrows sharpened on stones so that the warrior Chanda became overpowered by them and lay down senseless on the ground. The powerful Munda seeing his brother unconscious became very much distressed with grief; but he got angry and began to shoot arrows immediately on the Dev\={\i}. Chandik\=a Dev\={\i} hurled the weapon named \=I\d{s}ik\=a and thus cut off to pieces all the dreadful arrows of Munda in a moment and shot Ardha Chandra (semi-circular) arrow at him. With this arrow the powerful Asura was deprived of his pride and made to lie down unconscious on the earth. Munda thus lying on the ground, a great uproar arose amidst the army of the D\=anavas; and the Devas became delighted in the sky. In the meanwhile Chanda became conscious and taking a very heavy club hurled it violently on the right hand of K\=alik\=a. K\=alik\=a rendered that blow useless and instantly tied down that Asura by Her P\=asa weapon, purified by Mantras. Munda again rose up, and, seeing his brother in that fastened condition, came to the front well armoured and with an exceedingly strong weapon called \'Sakti. Seeing the Asura coming, She instantly fastened him down like his brother. Taking

the powerful Chanda and Munda like hares and laughing wildly, K\=al\={\i} went to Ambik\=a, and said :-- ``I have brought the two beasts very auspicious as offerings in this sacrificial war. Kindly accept these.'' Seeing the two D\=anavas brought, as if they were the two wolves, Ambik\=a told her sweetly :-- O Thou, fond of war! Thou art very wise; so dost not commit the act of envy nor dost leave them; think over the purport of my words and know that it is Thy duty to bring the Dev\={\i}'s work to a successful issue.

62-65. Vy\=asa said :-- O King! Hearing thus the words of Ambik\=a, K\=alik\=a spoke to Her again :-- ``In this war-sacrifice there is this axe which is like a sacrificial post; I will offer these two as victims to Thy sacrifice. Thus no act of envy will be committed (i.e., killing in a sacrifice is not considered as envy).'' Thus saying, the K\=alik\=a Dev\={\i} cut off their heads with great force and gladly drank their blood. Thus seeing the two Asuras killed, Ambik\=a said gladly :-- Thou hast done the service to the gods; so I will give Thee an excellent boon. O K\=alik\=a! As Thou hast killed Chanda and Munda, henceforth Thou wilt be renowned in this world as Ch\=amund\=a.

Here ends the Twenty-sixth Chapter of the Fifth Book on the killing of Chanda and Munda in \'Sr\={\i} Mad Dev\={\i} Bh\=agavatam, the Mah\=a Pur\=a\d{n}am, of 18,000 verses by Mahar\d{s}i Veda Vy\=asa.



