\chapter{On the dialogues of Nara N\=ar\=aya\d{n}a}

1. Vy\=asa said :-- O best of kings! There is no need of dwelling at length on this point; suffice to say, that in this world, are found persons very rare that are religious, and free from egoism, jealousy, anger, etc.

2. O king of kings! Even in the Satya Yuga, the Golden age, this world, moving and unmoving, was covered with feelings of jealousy and anger. What to say in this Kali Yuga (Dark Age)! (There is no wonder that this world would be full of these vicious things.)

3. O best of kings! When the Devas are deceitful, jealous, and filled with feelings of anger, what is to be said with human beings and other lower creations!

4. O Lord of the Earth! It is natural, that injury be inflicted on those persons that commit injury; but when peaceful persons, void of any enmity, are injured, that is certainly an act wicked and mischievous.

5. Whenever, any devout ascetic, calm and quiet, is engaged in prayer and meditation, and silent muttering of one's mantrams, the king of the Immortals throws hindrance in his asceticism. (This is certainly a mischievous act.)

6. (Holy, unholy and mixed persons exist in all the yugas). To those that are holy, all the yugas are the Satya yuga; to the unholy ones always it is the Kali yuga (Dark age); and to the mixed ones, always it is Tret\=a and Dv\=apara.

7. You will very seldom find a few persons, following really the True Religion; otherwise, you would have found all the persons in the different yugas religious, appropriate to those yugas.

8. O king! In all cases where the conservation of religions and religious affairs are concerned, know that the original wish and desire is the cause. If this desire be impure and sullied, religion becomes also sullied for, verily, this impurity in one's desire is one's cause of ruin in every respect. (Therefore the impure desires are never to be cherished and indulged.)

9. A son, named Dharma, was born of the heart of Brahm\=a; he was devoted to Brahm\=ajn\=ana (the knowledge of Brahm\=a), truthful, and always engaged in rites and ceremonies and in accordance with the Vedic religion.

10. This high souled Muni Dharma was a householder and married duly, according to the proper procedure, to the ten daughters of Dak\d{s}a Praj\=apati.

11. This Dharma, the foremost amongst the followers of truth, impregnated them and had four sons, named respectively Hari, Kri\d{s}\d{n}a, Nara, and N\=ar\=aya\d{n}a.

12. Hari and Kri\d{s}\d{n}a, amongst the four, used to remain always in the practising of the yoga.

13. Nara and N\=ar\=aya\d{n}a came over to the Him\=alay\=an mountains and, in the hermitage of Badarik\=a, commenced the difficult religious asceticism and penance.

14. The foremost of the ascetics, those two ancient Munis, began to recite that highest mantra of Para Brahm\=a, the G\=ayatr\={\i}, on the wide spacious bank of the Ganges.

15. The two Risis named Nara and N\=ar\=aya\d{n}a, born of Hari's Am\'sa, practised excellent tapasy\=a for full one thousand years.

16. The whole Universe, moving and unmoving, became hot through the Fire of their Tapas. Indra became also perplexed.

17-18. The thousand-eyed Indra became anxious, thought and within himself thus :-- What is to be done now? These two sons of Dharma are practising Tapas and are in meditation. If they succeed, they can occupy my excellent seat in Heaven; how can I break their Tapasy\=a and what steps shall I take to hinder them.

19-20. Lust, anger, and insurmountable avarice Indra brought into existence and, intent on hindering their tapasy\=a, mounted on the elephant Air\=avata, went quickly to the hill Gandham\=adan, and approaching the holy hermitage, saw the two ancient \d{R}i\d{s}is.

21. Their bodies were incandescent by Tapasy\=a, as if they were the two rising Suns. Were they Brahm\=a, Vi\d{s}\d{n}u manifested there or were they the two shining sources of light? These two \d{R}i\d{s}is were the sons of Dharma. What would they do with their Tapasy\=a?

22-23. Thinking thus, the lord of \'Sach\={\i} seeing them addressed thus :-- O highly fortunate ones! O two \d{R}i\d{s}is the sons of Dharma! Please tell me what are your objects? I have come here to give thee excellent boons; I am very pleased with your Tapasy\=a; therefore ask boons from me; and even if they be not worth giving, I will give them to you.

24-25. Vy\=asa said :-- The \d{R}i\d{s}is were deeply immersed in meditation and seemed very firm and resolute; they, therefore, did not reply anything, though Indra, standing before them, repeatedly urged them to ask boons from him. Seeing this, the king of the Immortals began to terrify them with his supernatural enchanting fearful m\=ayic powers.

26. He created lions, tigers, wolves and other murderous animals and began to terrify the two \d{R}i\d{s}is with them; Indra also produced rains, hurricanes and fires very frequently so that they might yield.

27. In spite of Indra's attempt to terrify them by his wonderful M\=ay\=a, the two Munis, Nara N\=ar\=aya\d{n}a, the two sons of Dharma, could not be brought under his control. And Indra returned to his own place.

28-31. And he became very sorry and thought thus :-- These two Munis could not be tempted away with boons, nor did they fly away from their place of worship, though terrified with fire, wind, wolves, tigers and lions. No one, I think, would be able to break their meditation. When fear and temptations have not distracted their meditation, they are certainly meditating on the Eternal Mah\=a Vidy\=a \'Sr\={\i} Bhuvane\'swar\={\i}, the Prime Force of Nature, the Source of all M\=ay\=as, and the Goddess the Creatrix of all the worlds, the wonderful highest Prakriti; what other expert in emitting M\=ay\=a there can be? Who can break their meditation!

32. Indeed! how can this whole host of M\=ay\=as that are created by Gods and Asuras overpower those purged of all their sins, who are meditating their Creator, the Supreme M\=ay\=a, that Illusion by which one considers the unreal Universe as really existent and as distinct from the Supreme Spirit, whence the Gods and Asuras have derived all their supernatural powers.

33. He in whose heart reign the seed mantras of V\=ak, K\=ama and M\=ay\=a, called V\=agv\={\i}jam, K\=amavijam, M\=ay\=av\={\i}jam, no one is able to stand against and overpower him.

34-35. O king! Indra, enchanted by M\=ay\=a, did not desist from tempting the two \d{R}i\d{s}is, but he went on thinking other means by which their asceticism could be baffled and asked K\=ama and Vasanta (the god of Lust and the season spring) to come before him and addressed them, thus :-- O K\=ama! You now be united with your wife Rati and Vasanta (the God of spring) and go to the hill Gandham\=adan, accompanied by all the Apsar\=as (celestial damsels) and with all the Rasas (love sentiments).

NOTE :-- The Gandham\=adan is the mountain like unsurpassable intoxicating happiness of the senses.

36-37. There you will find the two ancient excellent \d{R}i\d{s}is Nara and N\=ar\=aya\d{n}a practising asceticism in solitude, in the hermitage of Badarik\=a. O Manamatha! You better go before them, and with the influence of your arrows, do now my work and make their hearts extremely lustful.

38. O Fortunate One! Charm over them by means of your arrows, make them leave their asceticism by magical spells.

39. Who is there in this world of Devas, Daityas, human beings, that, being whipped by your arrows, do not come under your control?

40. When Brahm\=a, I, Mah\=adeva, Moon and Fire are all fascinated by your arrows, then is there any doubt that these two \d{R}i\d{s}is would not be fascinated by them!

41. I am sending these public women as your assistants. Rambh\=a and other beautiful celestial nymphs would all follow you.

42. You alone, or Rambh\=a or Tilottam\=a alone can do this work. Will there be any doubt if you all unite in this?

43. O Good One! Do this work for me; I will confer on you your desired objects.

44. O Manmatha! I tempted them with boons but these two ascetics, of controlled minds, could not be displaced from their seats. My efforts were rendered useless.

45. I frightened them very much with all the M\=ayic powers; yet they could not be dislocated from their deep thoughts. It seems that they are quite heedless in the preservation of their bodies.

46. Vy\=asa said :-- K\=amadeva, on hearing the king of the Devas, addressed him thus :-- O Indra! Today I will fulfill all your desires.

47. But there is one word. If these two ascetics be meditating Vi\d{s}\d{n}u, \'Siva or Brahm\=a or the Sun, then I will be able to bring them under my control.

48. And if they be meditating on the Great Seed Mantra, the root of all M\=ay\=a, and the great K\=amav\={\i}jam, the king of the K\=ama, I will never be able to subdue such a devotee of the Highest Dev\={\i}.

49. If these two ascetics have devotedly taken refuge of the Great Power Mah\=a Dev\={\i}, then they will not come under the sight of my arrows.

50. Indra said :-- O Blessed One! Go now with your assistants, ready to do your work. No body but you, I find, that can fulfill my this beneficial, though very difficult work.

Vy\=asa said :-- Thus ordered by Indra, they all departed to where the Dharma's sons Nara, and N\=ar\=aya\d{n}a were performing their hard Tapasy\=as.

Here ends the Fifth Chapter in the Fourth Book of \'Sr\={\i} Mad Dev\={\i} Bh\=agavatam, the Mah\=a Pur\=a\d{n}am of 18,000 verses by Mahar\d{s}i Veda Vy\=asa.



