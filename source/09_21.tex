\chapter{On the meeting of Mah\=adeva and \'Sankhach\=uda for an encounter in conflict}

1-33. \'Sr\={\i} N\=ar\=aya\d{n}a spoke :-- Then the D\=anava, the devotee of \'Sr\={\i} Kri\d{s}\d{n}a, got up from his flower strewn bed, meditating on \'Sr\={\i} Kri\d{s}\d{n}a, early in the morning time, at the Brahm\=a Muh\=urta. Quitting his night dress, he took his bath in pure water and put on a fresh washed clothing. He then put the bright Tilak mark on his forehead and, performing the daily necessary worship, he worshipped his Ista devat\=a (The Deity doing good to him). He then saw the auspicious things such as curd, ghee, honey, fried rice, etc., and distributed as usual, to the Br\=ahma\d{n}as the best jewels, pearls, clothing and gold. Then for his marching to turn out auspicious, he gave at the feet of his Guru Deva priceless gems, jewels, pearls, diamonds, etc., and finally he gave to the poor Br\=ahmi\d{n}s with great gladness, elephants, horses, wealth, thousands of stores, two lakhs of cities and one hundred kotis of villages. He then gave over to his son, the charge of his kingdom and of his wife, and all the dominions, wealth, property, all the servants and maid servants, all the stores and conveyances. He dressed himself for the war and took up bows and arrows and arrow cases. By the command of the King, the armies began to gather. Three lakhs of horses, one lakh elephants, one ayuta chariots, three Kotis of bowmen, three Kotis armoured soldiers and three Kotis of trident holders got themselves ready. Then the King counted his forces and appointed one Commander-in-Chief, (Mah\=aratha), skilled in arts of

warfare, over the whole army. Thus the generals were apppointed over the three lakh Ak\d{s}auhi\d{n}\={\i} forces and their provisions were collected by three hundred Ak\d{s}auhi\d{n}\={\i} men. He, then, thinking of \'Sr\={\i} Hari, started for war, accompanied by his vast army. (Note :-- One Ak\d{s}auhi\d{n}\={\i} consists of a large army consisting of 21,870 chariots, as many elephants, 65,610 horses, and 109,350 foot). He then mounted on a chariot built of excellent jewels and, headed by his Guru and all his other elders, went to \'Sankara. O N\=arada! Bhagav\=an Mah\=adeva was at that time, staying on the banks of Pu\d{s}pabhadra. That place was Sidh\=a\'srama (the hermitage where the yogic successes had been obtained and can easily be acquired in future for the Siddhas as well a Siddha K\d{s}ettra.) It was the place where the Muni Kapila practised Tapasy\=a, in the holy land of Bh\=arata. It was bounded on the east by the western ocean, on the west by the Malaya mountain, on the south, by the \'Sr\={\i} \'Saila mountain and on the north by the Gandha M\=adana Mountain. It was five yojanas wide and one hundred times as long. This auspicious river in Bh\=arata yields great religious merits and is always full of clear, sparkling running water. She is the favourite wife of the Salt Ocean and She is very blessed. Issuing from \'Sar\=avat\={\i} Him\=alay\=as, She drops into the ocean. Keeping the river Gomat\={\i} (Goomti) by her left; She falls into the west ocean. \'Sankhach\=uda, arriving there, saw Mah\=adeva under a Peepul tree near its root with a smiling countenance, like one Koti Suns seated in a yogic posture. His colour was white like a pure crystal; as if the Fire of Brahm\=a was emitting from every pore of His body (burning with Brahm\=a-Teja); He was wearing the tiger skin and, holding the trident and axe. He dispels the fear of death of His Bhaktas; His face is quite calm. He, the Lord of Gaur\={\i}, is the Giver of the fruits of Tapasy\=a and of all sons of wealth and prosperity. The smiling face of \=A\'suto\d{s}a (one who is pleased quickly) is always thinking of the welfare of the Bhaktas; He is the Lord of the Universe, the Seed of the universe, the All-form (all-pervading), and the Progenitor of the universe. He is omnipresent, All pervading, the Best in this universe, the Destroyer of this universe, the Cause of all causes, and the Saviour from the hells. He is the Awakener and Bestower of Knowledge, the Seed of all knowledges, and He Himself is of the nature of Knowledge and Bliss. Seeing that Eternal Puru\d{s}a, the King of the D\=anavas at once descended from his chariot and bowed down with devotion to Him and to Bhadra K\=al\={\i} on His left and and to K\=artikeya on his front. The other attendants did the same. \'Sankara, Bhadra K\=al\={\i} and Skanda all blessed him. Nandi\'svara and others got up from their

seats on seeing the D\=anava King and began to talk with each other on that subject. The King addressed \'Siva and sat by Him. Bhagav\=an Mah\=adeva, the Tranquil Self, then, spoke to him, thus :-- O King! Brahm\=a, the knower of Dharma and the Creator of the world, is the Father of Dharrna, The religious Mar\={\i}chi, a devotee of Vi\d{s}\d{n}u, is the son of Brahm\=a. The religious Praj\=apati Ka\'syapa is also the Brahm\=a's son. Dak\d{s}a gladly gave over to Ka\'syapa in marriage, his thirteen daughters. Danu, fortunate and chaste, is also one of them.

34-64. Danu had forty sons, all spirited and known as D\=anavas. The powerful Viprachitti was the prominent amongst them. Viprachitti's son was Dambha, self controlled and very much devoted to Vi\d{s}\d{n}u. So much so that for one lakh years he recited the Vi\d{s}\d{n}u mantra at Pu\d{s}kara. His Guru (spiritual teacher) was \'Sukr\=ach\=arya; and, by his advice, he recited the mantra of \'Sr\={\i} Kri\d{s}\d{n}a, the Highest Self. He got you as his son, devoted to Kri\d{s}\d{n}a. In your former birth, you were the chief attendant Gopa (cow-herd) of Kri\d{s}\d{n}a. You were very religious. Now, by R\=adhik\=a's curse, you are born in Bh\=arata, as the Lord of the D\=anavas, powerful, heroic, valorous, and chivalrous. All the things from Brahm\=a down to a blade of grass, the Vai\d{s}\d{n}avas regard as very trifling; even if they get S\=alokya, S\=arsti, S\=ayujya and S\=am\={\i}pya of Hari, they do not care a straw for that. Without serving Hari, they do not accept those things, even if those are thrust on them. Even Brahm\=ahood and immortality, the Vai\d{s}\d{n}avas count for nothing. They want to serve Hari (Sev\=a-bh\=ava). Indrahood, Manuhood, they do not care. You, too, are a real Kri\d{s}\d{n}a Bhakta. So what do you care for those things that belong to the Devas, that are something like false to you. Give back to the Devas their kingdoms thus and please Me. Let the Devas remain in their own places and let you enjoy your kingdom happily. No need now for further quarrels. Think that you all belong to the same Ka\'syapa's family. The sins that are incurred, for example, the murder of a Br\=ahmi\d{n}, etc., are not even one-sixteenth of the sins incurred by hostilities amongst the relatives. If, O King! You think that by giving away to the Devas their possessions, your property will be diminished, then think that no one's days pass ever in one and the same condition. Whenever Prakriti is dissolved, Brahm\=a also vanishes. Again He appears by the Will of God. This occurs always. True, that knowledge is increased by true Tapasy\=a; but memory fails then. This is certain. He who is the creator of this world, does his work of creation gradually by the help of his Knowledge-power (J\~n\=ana-\'Sakti). In the Satya Yuga, Dharma reigns in full; in the Tret\=a Yuga, one quarter is diminished; again in the Dv\=apara only one-half remains. And in the K\=al\={\i} Yuga, only one quarter remains. Thus Dharma gets increase and decrease. At the end of the

K\=al\={\i}, the Dharma will be seen very feeble as the phase of the Moon is seen very thin on the Dark Moon night. See, again, the Sun is very powerful in summer; not so in winter. At midday the Sun is very hot; it does not remain so in the morning and evening. The Sun rises at one time; then he is considered as young at another time he becomes very powerful and at another time he goes down. Again in times of distress (i.e., during the cloudy days) the Sun gets entirely obscured. When the Moon is devoured by R\=ahu (in the Lunar Eclipse), the Moon quivers. Again when the Moon becomes liberated (i.e., when the eclipse passes away) She becomes bright again. In the Full-Moon night She becomes full but She does not remain so always. In the Dark fortnight She wanes every day. In the bright fortnight She waxes every day. In the bright fortnight, the Moon becomes healthy and prosperous and in the dark fortnight, the Moon becomes thinner and thinner as if attacked with consumption. In the time of eclipse She becomes pale and in the cloudy weather, She is obscured. Thus the Moon also becomes powerful at one time and weak and pale at another time. Vali now resides in P\=at\=ala, having lost all his fortunes; but, at some other time, he will become Devendra (the Lord of the Devas). This earth becomes at one time covered with grains and the resting-place of all beings; and, at another time, She becomes immersed under water. This universe appears at one time and disappears at another. Everything, moving or non-moving, sometimes appears and again, at another time, disappears. Only Brahm\=a, the Highest Self, remains the same. By His grace, I have got the name Mrityunjaya (the Conqueror of Death). I, too, am witnessing many Prakritik dissolutions, I witnessed repeatedly many dissolutions and will in future, witness many dissolutions. The Param\=atman becomes of the nature of Prakriti. Again it is He that is the Puru\d{s}a (male principle). He is the Self; He is the individual soul (J\={\i}va). He thus assumes various forms. And, again, Lo! He is beyond all forms! He who always repeats His Name and sings His Glory, can conquer, at some occasion, death. He is not to come under the sway of this birth, death, disease, old age and fear. He has made Brahm\=a the Creator, Vi\d{s}\d{n}u the Preserver and Me the Destroyer. By His Will, we are possessed of those influences and powers. O King! Having deputed K\=ala, Ag\d{n}i and Rudra, to do the destruction work, I Myself repeat only His name and sing His glory, day and night, incessantly. My name is, on that account, Mrityunjaya. By His Knowledge Power, I am fearless. Death flies away fast from Me as serpents fly away at the sight of Gar\=uda, the Vinat\=a;s son. O N\=arada! Thus saying, \'Sambhu, the Lord of all, the Progenitor of all, remained silent. Hearing the above words of \'Sambhu, the King thanked Mah\=adeva again and again and spoke in sweet humble words.

65-74. \'Sankhach\=uda said :-- The words spoken by Thee are quite true. Still I am speaking a few words. Kindly hear. Thou hast spoken just now that very great sins are incurred by kindred hostilities. How is it, then, that He robbed Vali of his whole possessions and sent him down into P\=at\=ala? Gad\=adhara Vi\d{s}\d{n}u could not recover Vali's glory. But I have done that. Why did the Devas kill Hira\d{n}y\=aksa and Hira\d{n}y\=aka\'sipu, \'Sumbha and the other D\=anavas? In by gone days, we laboured hard when the nectar was obtained out of the churning of the ocean; but the best fruit was reaped by the Devas only. However, all these point that this universe is but the mere sporting ground of Param\=atman, Who has become of the nature of Prakriti (the polarities of the one and the same current to produce electric effects). Whomsoever He grants glory and fortune, he only gets that. The quarrel of the Devas and the D\=anavas is eternal. Victory and defeat come to both the parties alternately. So it is not proper for Thee to come here in this hostility. For Thou art the God, of the nature of the Highest Self. Before Thee, we both are equal. So it is a matter of shame, no doubt, for Thee to stand up against us in favour of the gods. The glory and fame that will result to Thee, if Thou art victorious, will not be so much as it will be if we get the victory. On the contrary the inglory and infamy that will result to Thee if Thou dost get dire defeat will be inconceivably much more than what would come to us if we are defeated. (For we are low and Thou art Great.)

75-79. Mah\=adeva laughed very much when he heard the D\=anava's words and replied :-- O King! You are descended from the Br\=ahmi\d{n} family. So what shame shall I incur if I get defeat in this fighting against you. In former days, the fight took place between Madhu and Kaitabha; again between Hira\d{n}ya Ka\'sipu and Hira\d{n}y\=aksa and \'Sr\={\i} Hari. I also fought with the Asura Tripur\=a. Again the serious fight took place also between \'Sumbha and the other Daityas and the Highest Prakriti Dev\={\i}, the Ruler of all, and the Progenitrix of all and the Destructrix of all. And, then, you were the P\=ari\d{s}ada attendant of \'Sr\={\i} Kri\d{s}\d{n}a, the Highest Self.

Note :-- \'Sr\={\i} Kri\d{s}\d{n}a is the Eternal Puru\d{s}a beyond the Gu\d{n}as. He creates Prakriti. All the creation is effected by Him. He is the Master of all the \'Saktis. These \'Saktis come from Him and go into Him. \'Sr\={\i} Kri\d{s}\d{n}a plays with these \'Saktis, these lines of Forces, very powerful and terrible, indeed, that go to create, preserve and destroy the whole universe. These Lines of Forces have their three properties :-- (1) Origin; (2) direction and (3) magnitude. And finally they come back to their origin. This makes one Kalpa, one Life, one Moment, one in the Full One. The Gu\d{n}as come out of these \'Saktis, these Lines of Forces. \'Sr\={\i} Kri\d{s}\d{n}a

is the Great Reservoir, the Great Centre of Forces, Powerful, Lovely and Terrible. All the events as described here, appear in the intermediate stages when the Fourth Dimension passes into the Third Dimension, etc. The Fourth Dimension does not at once turn out into the Third Dimension but it takes place by degrees. This explains our dreams, visions, etc., which, if seen when the mind is pure, turn out to be true.

80-82. So the Daityas, that were killed before, cannot be compared with you. Then why shall I feel shame in fighting against you? I am sent here by \'Sr\={\i} Hari for saving the Devas. So either give back to the Devas their possessions, or fight with Me. No need in speaking thus quite useless talks. O N\=arada! Thus speaking, Bhagav\=an \'Sankara remained silent. \'Sankhach\=uda got up at once with his ministers.

Here ends the Twenty-first Chapter in the Ninth Book on the meeting of Mah\=adeva and \'Sankhach\=uda for an encounter in conflict in the Mah\=apur\=a\d{n}am \'Sr\={\i} Mad Dev\={\i} Bh\=agavatam of 18,000 verses by Mah\=arsi Veda Vy\=asa.



