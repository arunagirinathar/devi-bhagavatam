\chapter{On G\=ayatr\={\i} Hridaya}

1-2. N\=arada said :-- O Bhagav\=an! I have heard from you all about the Kavacha and the Mantra of \'Sr\={\i} G\=ayatr\={\i}. O Deva Deva! O Thou, the Knower of the present, the past, and the future! Now tell about the Hridaya, the highest, the interior or esoteric Essence of the G\=ayatr\={\i}, holding which, if one repeats the G\=ayatr\={\i}, he acquires all the pu\d{n}yam (merits), I am desirous to hear this.

3-8. N\=ar\=aya\d{n}a said :-- O N\=arada! This subject on the Hridaya of G\=ayatr\={\i} is explicitly written in the Atharva Veda. Now I will speak on that, the great secret, in detail. Listen. First, consider the G\=ayatr\={\i}, the Dev\={\i}, the Mother of the Vedas as of a Cosmic Form (Vir\=a trup\=a) and meditate all the Devas as residing on Her Body. Now in as much as the Pinda and Brahm\=anda are similar, consider yourself as of the form of the Dev\={\i} and meditate within yourself on the Devat\=as, thus :-- The Pundits, the Knowers of the Vedas, say this :-- He is not yet fit to worship the Deva and he is not an Adhik\=ar\={\i} as yet who has not been able to make himself a Deva; therefore to establish the knowledge of the oneness of the Deva and himself, he is to meditate the Devas within his body, thus :--

O N\=arada! Now I will speak on the Hridaya of G\=ayatr\={\i}, knowing which every man becomes able to become all the Devas. Listen. The \d{R}i\d{s}i of this G\=ayatr\={\i} Hridaya is N\=ar\=aya\d{n}a; the Chhandas is G\=ayatr\={\i}; and \'Sr\={\i} Parame\'svar\={\i} G\=ayatr\={\i} is the Devat\=a. Perform the Ny\=asa of this as mentioned before and taking your seat in a lonely place, meditate intently on the Dev\={\i} with your heart and head well collected. Now I am speaking of the Arthany\=asa. Hear. Meditate on the Devat\=a Dyau on your head; the twin A\'svins on the rows of the teeth; the two Sandhy\=as on your upper and lower lips; the Agni, Fire, within your mouth; Sarasvat\={\i}, on the tongue; Brihaspati on the neck; the eight Vasus on the two breasts; the V\=ayus, on the two arms; the Paryanya Deva on the heart; \=Ak\=a\'sa, on the belly; Antar\={\i}k\d{s}am (the middle space) on the navel; Indra and Agni, on the loins; Praj\=apati, the condensed form, as it were, of Vij\~n\=ana, on the hip joints; the Kail\=a\'sa and the Malaya mountains on the two thighs; the Visvedev\=as on the two knees; Vi\'sv\=amitra on the shanks; the Sun's northern and southern paths, the Uttar\=ayana and Dak\d{s}i\d{n}\=ayana

on the anus; the Pitris on the thighs; the Earth on the legs; the Vanaspati on the fingers and toes; the \d{R}i\d{s}is on the hairs of the body; the Muh\=urtas on the nails; the planets on the bones; the Ritus (seasons) on the blood and flesh; the Samvatsaras on the Nimi\d{s}a (twinkling of eye) the Sun and the Moon on the day and night respectively. Thinking thus, repeat ``I take refuge of the Divine Holy G\=ayatr\={\i}, the Chief and most Excellent One, the Thousand eyed and I take refuge wholly unto Her.''

Then repeat ``I bow down to Tat savitur vare\d{n}yam,'' ``I bow down to the Rising Sun on the East,'' ``I bow down to the Morning Aditya,'' ``I bow down to the G\=ayatr\={\i}, residing in the Morning Sun'' and ``I bow down to all.'' O N\=arada! Whoever recites this G\=ayatr\={\i} Hridaya in the morning finds all the sins committed in the night all destroyed! Whoever recites this in the evening gets his sins of the day all destroyed! Whoever recites this in the evening and in the morning can rest assured to have become free of sins; he gets the fruits of all the T\={\i}rthas; he is acquainted with all the Devas; he is saved if he has spoken anything that ought not to have been spoken; if he has eaten anything that is not fit to be eaten; if he has chewn and sucked anything that ought not to have been chewn and sucked; if he has done anything that ought not to have been done and if he has accepted hundreds and thousands of gifts that ought never to have been accepted.

The sins incurred by eating with the others in a line cannot touch him. If he speaks lies, he will not be touched by the sins thereof; even if a non-Brahmach\=ari recites this, he will become a Brahmach\=ar\={\i}. O N\=arada! What more shall I say to you of the results of G\=ayatr\={\i} Hridaya than this :-- that whoever will study this will acquire the fruits of performing thousand sacrifices and repeating the G\=ayatr\={\i} sixty thousand times. In fact, he will get Siddhi by this. The Br\=ahm\=a\d{n}a, who daily reads this in the morning will be freed of all the sins and go upwards to the Brahm\=a (Loka) and is glorified there. This has been uttered by Bhagav\=an N\=ar\=aya\d{n}a Himself.

Here ends the Fourth Chapter of the Twelfth Book on G\=ayatr\={\i} Hridaya in the Mah\=apur\=a\d{n}am \'Sr\={\i} Mad Dev\={\i} Bh\=agavatam of 18,000 verses by Mahar\d{s}i Veda Vy\=asa.



