\chapter{On the anecdote of Tulas\={\i}}

1-19. \'Sr\={\i} N\=ar\=aya\d{n}a said :-- O N\=arada! The wife of Dharmadhvaja was M\=adhav\={\i}. Going to the Gandham\=adan mountain, She began to enjoy, with great gladness, the pleasures with the king Dharmadhvaja. The bed was prepared, strewn with flowers and scented with sandal-paste. She smeared all over her body with sandal-paste. The flowers and cool breeze in contact with the sweet scent of sandal-paste began to cool the bodies. M\=adhav\={\i} was the jewel amongst women. Her whole body was very elegant. Besides it was adorned all over with jewel ornaments. As she was humorous, so the king was very expert in that respect. It seemed as if the Creator created especially for Dharmadhvaja, the humorous lady M\=adhav\={\i} expert in amorous affairs. Both of them were skilled in amorous sports. So no one did like to desist from amorous enjoyments. One hundred divine years passed in this way, day and night passed unnoticed. The king then got back his consciousness and desisted from his amorous embraces. But the lustful woman did not find herself satisfied. However, by the Deva's influence, she became pregnant and conceived for one hundred years. In the womb there was the incarnation of Lak\d{s}m\={\i}; and the body's lustre increased day by day. Then, on an auspicious day, on an auspicious moment, auspicious Yoga, auspicious Lagna, auspicious Amsa, and on an auspicious combination of planetary

rulers and their houses, she delivered on the full moon night of the month of K\=artik one beautiful daughter, the incarnation in part of Lak\d{s}m\={\i}. The face of the baby looked like the autumnal moon; Her two eyes resembled autumnal lotuses and her upper and lower lips looked beautiful like ripe Bimba fruits. The daughter began, no sooner it was born, to look on all sides of the lying-in-room. The palm and lower part of feet were red. The navel was deep and below that there were three wrinkles. Her loins were circular. Her body was hot in the winter and cold in the summer and pleasant to touch. Her hairs on the head were hanging like the roots of the fig tree. Her colour was bright like Champaka; She was a jewel amongst women. Men and women cannot compare her beauty. The holy wise men named Her Tulas\={\i}. As soon as she was born, she looked of the female sex, full in every way. Though prevented repeatedly by all, She went to the forest of Badar\={\i} for practising Tapasy\=a. There she practised hard Tapasy\=a for one lakh divine years. Her main object was to get N\=ar\=aya\d{n}a for her husband. In summer she practise Panchatap\=a (surrounded by fire on four sides and on the top); in the winter she remained in water and in the rainy season she remained in the open air and endured the showers of the rain, twenty thousand years. She passed away thus in eating fruits and water. For thirty thousand years she subsisted only on the leaves of trees. When the forty thousandth year came, she subsisted only on air and her body became thinner and thinner day by day. Then for ten thousand years afterwards she left eating anything whatsoever and without any aim, stood on only one leg. At this time the lotus-born Brahm\=a, seeing this, appeared there to grant her boons. On seeing Him, Tulas\={\i} immediately bowed down to Brahm\=a, the Four-faced One riding on His vehicle, the Swan. [Note: -- The vehicle theory of the Devas came from Egypt. The Devas were without vehicles at first and were faced half-beasts. Then they were rendered men and their vehicles were fancied as beasts. The face of the D\=urg\=a Dev\={\i} was thought of as that of a tiger.]

20. He then addressed her and said :-- ``O Tulas\={\i}! Ask a boon that you like. Whether it be devotion to Hari, servantship to Hari, freedom from old age or freedom from death, I will grant that to you.''

21-27. Tulas\={\i} said :-- ``Father! I now say you my mind. Hear. What is the use of hiding away my views out of fear or shame to One who knows everything reigning in One's Heart.

I am Tulas\={\i} Gop\={\i} (cowherdess); I used to dwell before in the Goloka. I was a dear she-servant of Radhik\=a, the beloved of Kri\d{s}\d{n}a. I was also born of Her in part, Her Sakhis (female attendants) used to love me. Once in R\=asa Mandalam I was enjoyed by Govinda; but I was not satiated and while

I was lying down in an unconscious state, R\=adh\=a, the Governess-in-chief of the R\=asa circle, came there and saw me in that state. She rebuked Gobinda and, out of anger, cursed me :-- `Go at once and be born as a human being.' At this Govinda spoke to me :-- `If you go and practise Tapas in Bh\=arata, Brahm\=a will get pleased and He will grant you boon. When you will get N\=ar\=aya\d{n}a, the Four-armed, born of Me in part as your husband.' O Father! Thus speaking, \'Sr\={\i} Kri\d{s}\d{n}a disappeared out of sight. Out of R\=adh\=a's fear, I quitted my body and am now born in this world. Now grant me this boon that I get the peaceful, lovely, beautiful Nara for my husband.''

28-37. Brahm\=a said :-- ``O Child Tulas\={\i}! The Gopa (cowherd) Sud\=am\=a was born of \'Sr\={\i} Kri\d{s}\d{n}a's body. At the present time he is very energetic, He too, under the curse of R\=adh\=a, has come and taken his birth amongst the D\=anavas. He is named \'Sankha Ch\=uda. No one is equal to him in strength. In Goloka, when he saw you before, he was overpowered with passion for you. Only out of R\=adh\=a's influence, he could not embrace you. That Sud\=am\=a is J\=atismara (knows all about his previous births); and you, too, are J\=ati Smar\=a. There is nothing unknown to you. O Beautiful One! You will now be his wife. Afterwards you will get N\=ar\=aya\d{n}a, the Beautiful and Lovely for your husband. Thus under the curse of N\=ar\=aya\d{n}a, you will be transformed into the world purifying Tulas\={\i} tree. You will be the foremost amongst the flowers and will be dearer to N\=ar\=aya\d{n}a than His life. No one's worship will be complete without Thee as leaf. You will remain as a tree in Bindr\=aban and you will be widely known as Vrind\=aban\={\i}. The Gopas and Gopis will worship M\=adhava with Your leaves. Being the Presiding Deity of the Tulas\={\i} tree, you will always enjoy the company of Kri\d{s}\d{n}a, the best of the Gopas.'' O N\=arada! Thus bearing Brahm\=a's words, the Dev\={\i} Tulas\={\i} became very glad. Smile appeared in her face. She then bowed down to the Creator and said :--

38-40. ``O Father! I speak now truly to Thee that I am not as devoted to the four-armed N\=ar\=aya\d{n}a as I am devoted to \'Sy\=ama Sundara, the two-armed. For my intercourse with Govinda \'Sr\={\i} Kri\d{s}\d{n}a was suddenly interrupted and my desire was not gratified. It is because of \'Sr\={\i} Govinda's words that I prayed for the four-armed. Now it appears certain that by Thy grace I will get again my Govinda, very hard to be attained. But, O Father! Do this that I be not afraid of R\=adh\=a.''

41-48. Brahm\=a said :-- ``O Child! I now give you the sixteen lettered R\=adh\=a mantra to you. By Her Grace you will be dear to R\=adh\=a as Her life. R\=adhik\=a will not be able to know anything of your secret

dealings. O Fortunate! You will be dear to Govinda like R\=adh\=a.'' Thus saying, Brahm\=a, the Creator of the world, gave her the sixteen lettered R\=adh\=a mantra, stotra, Kavacha and mode of worship and pura\'schara\d{n}a and He blessed her. Tulas\={\i}, then, engaged herself in worshipping R\=adh\=a, as directed. By the boon of Brahm\=a, Tulas\={\i} attained Siddhi (success) like Lak\d{s}m\={\i}. Out of the power of the Siddha mantra, She got her desired boon. She became fortunate in getting various pleasures, hard to be attained in this world. Her mind became quiet. All the toils of Tapasy\=a disappeared. When one gets the fruit of one's labour, all the troubles then transform to happiness. She then finished her food and drink and slept on a beautiful bed strewn with flowers and scented with sandal paste.

Here ends the Seventeenth Chapter of the Ninth Book on the anecdote of Tulas\={\i} in \'Sr\={\i} Mad Dev\={\i} Bh\=agavatam of 18,000 verses by Mahar\d{s}i Veda Vy\=asa.



