\chapter{On the description of G\=ayatr\={\i}}

1-7. N\=arada said :-- O Deva! The rules of Sad\=ach\=ara (right ways of living) and the all-sin-destroying unequalled Glories of the Dev\={\i} Bhagavat\={\i} have been described by Thee. And I, too, have heard the nectar of the Glories of the Dev\={\i} from Thy lotus mouth. The Ch\=andr\=aya\d{n}a and other Vratas; described by Thee, are very difficult to practise. So they are impracticable with the ordinary persons. Therefore, O Lord! Kindly describe those actions which can easily be carried out by common persons, at the same time, the Dev\={\i}'s Grace and Siddhis can be obtained by those practices. Again what thou hast described about the G\=ayatr\={\i} in connection with Sad\=ach\=ara kindly say which are the chief and foremost as well as those that are more meritorious. O Best of the Munis! Thou hast told that there are the twenty-four syllables in the G\=ayatr\={\i}. Kindly describe now their \d{R}i\d{s}is, Chhandas, Devat\=as and other things that should be known regarding them and thus satisfy my longings.

8-27. \'Sr\={\i} N\=ar\=aya\d{n}a said :-- O N\=arada! The twice-born would have done what they ought to do if they be engaged in repeating their G\=ayatr\={\i} only, whether they be able or not able to practise the Ch\=andr\=aya\d{n}a and the other vratas. Whichever Br\=ahmi\d{n} repeats the G\=ayatr\={\i} three thousand times and offers Arghya to the Sun in the three Sandhy\=a times, the Devas worship him; what to speak of other ordinary persons! Whether he practises Ny\=asa or not, if anybody sincerely repeats the G\=ayatr\={\i} Dev\={\i}, Whose Nature is Existence, Intelligence, and Bliss and meditates on Her, even if he attains siddhi in one syllable even, then as a result of that, he can vie with the best of the Br\=ahma\d{n}as, the Moon, and the Sun; nay, with Brahm\=a, Vi\d{s}\d{n}u, and Mahe\'svara even! O N\=arada! Now I will tell in due order the \d{R}i\d{s}is, Chhandas, and the Devat\=as of the twenty-four syllables of the G\=ayatr\={\i}. The \d{R}i\d{s}is, in due order, are (1) V\=ama Deva, (2) Attri, (3) Va\'sistha, (4) \'Sukra, (5) Ka\d{n}va, (6) Par\=a\'sara, (7) the very fiery Vi\'svamitra, (8) Kapila, (9) \'Sau\d{n}aka, (10) Y\=aj\~navalkya, (11) Bharadv\=aja, (12) the ascetic Jamadagni, (13) Gautama, (14) Mudgala, (15) Vedavy\=asa, (16) Loma\'sa, (17) Agastya, (18) Kau\'sika, (19) Vatsya, (20) Pulastya, (21) M\=anduka, (22) the ascetic in chief Durv\=as\=a, (23) N\=arada and (24) Ka\'syapa.

Now about the chhandas :-- (1) G\=ayatr\={\i}, (2) U\d{s}\d{n}ik, (3) Anustup, (4) Brihat\={\i}, (5) Pankti, (6) Tri\d{s}\d{n}up, (7) Jagat\={\i}, (8) Atijagat\={\i}, (9) \'Sakkar\={\i}, (10) Ati \'Sakkar\={\i}, (11) Dhriti, (12) Ati Dhriti, (13) Vir\=at, (14) Prast\=arapankti, (15) Kriti, (16) Pr\=akriti, (17) \=Akriti, (18) Vikriti, (19) Samkriti, (20) Ak\d{s}arapankti, (21) Bhuh, (22) Bhuvah, (23) Svah and (24) Jyoti\d{s}mat\={\i}. The Devat\=as of the several letters in due order, are :-- (1) Agni, (2) Praj\=apati, (3) Soma, (4) \=I\'s\=ana, (5) Savit\=a, (6) \=Aditya, (7) Brihaspati, (8) Maitr\=avaru\d{n}a, (9) Bhagadeva, (10) Aryam\=a, (11) Ga\d{n}e\'sa, (12) Tvastr\=a, (13) P\=u\d{s}\=a, (14) Indr\=agn\={\i}, (l5) V\=ayu, (16) V\=amadeva, (17) Maitr\=a varun\={\i} (18) Vi\'svadeva, (19) M\=atrik\=a, (20) Vi\d{s}\d{n}u, (21) Vasu, (22) Rudra Deva, (23) Kuvera, and (24) the twin A\'svin\={\i} Kum\=aras. O N\=arada! Thus I have described to you about the the Devat\=as of the twenty-four syllables. The hearing of this destroys all sins and yields the full results of repeating the mantra G\=ayatr\={\i}. (Note :-- The Devat\=as, mentioned in the G\=ayatr\={\i} Brahm\=a Kalpa are different from those mentioned here.)

Here ends the first Chapter of the Twelfth Book on the description of G\=ayatr\={\i} in the Mah\=apur\=a\d{n}am \'Sr\={\i} Mad Dev\={\i} Bh\=agavatam of 18,000 verses by Mahar\d{s}i Veda Vy\=asa.



