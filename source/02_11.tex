\chapter{On the Sarpa Yaj\~na}

1-4. S\=uta said :-- ``O Munis! seeing now the king lifeless, and his son a mere boy, the ministers themselves performed all his funeral ceremonies. First they burned the king on the banks of the Ganges without uttering any Mantra, as his death was an accidental one due to snake bite, afterwards they had an effigy of the king made of ku\'sa grass and placed it on a funeral pyre and burned it, with sandal and scented wood. The priest then performed and completed his funeral obsequies, repeating duly the Vedic mantras, and distributed various things in charities to the Br\=ahmi\d{n}s, together with sufficient quantity of gold, and varieties of food and clothings so that the king may attain heaven.

5-7. Next, on an auspicious moment, the ministers installed the boy prince on the throne that gladdened the hearts of the subjects and all the popu-

lace of the city, towns, and villages acknowledged the boy prince Janamejaya, endowed with all royal qualities as their king. The Dh\=atreyi gave all instructions to the king about his duties. The boy prince gradually grew in years and became endowed with great intellect.

8-15. When Janamejaya became eleven years old, the family priest initiated him duly with the G\=ayatr\={\i} mantra and he also studied it duly. Then Krip\=a ch\=arya taught him perfectly the science of archery (Dhanurveda) as Dron\=ach\=arya taught Arju\d{n}a and Para\'sur\=ama taught Kar\d{n}a. Janamejaya learnt by and by all the sciences and became very powerful and indomitable to his enemies as he was skilled in the science of archery, he was similarly in the other branches of the Vedas. Truthful, self-controlled, religious, the king Janamejaya acquired full knowledge in the Dh\=arma\'s\=astras (philosophies and law books) and Artha\'s\=astras (economics) and governed his kingdom like the Dharma's son Yudhisthira.

The king of K\=a\'s\={\i} gave his all-auspicious daughter Vapustam\=a in marriage to the king Janamejaya wearing golden coat of armour. The king Janamejaya, with the beautiful Vapu\'sam\=a casting side-long looks, looked very happy as was the king Vich\={\i}trav\={\i}rya, when he got for his wife the daughter of K\=ashir\=aj and also when Arjuna got his Subhadr\=a. Then the king began to enjoy his lotus-eyed Vapustam\=a in forest, and gardens like \'Satakratu and \'Sach\={\i}. The able ministers conducted satisfactorily the reins of government; and the subjects, well governed passed away their time with cheerful hearts.

16-32. In the meanwhile, a Muni, named Uttanka, being much troubled by Tak\d{s}aka, thought who could help him in his taking revenge on Tak\d{s}ak and, seeing the king Par\={\i}ksit's son the king Janamejaya a proper person came to Hastin\=a to the king and spoke out thus :-- ``O good king! Thou dost not know when to do a thing that ought to be done; Thou art doing at present what ought not to be done; and thou art not doing what should be done now. There is nothing of anger or energy within Thee; Thou dost things as a child does; so Thou dost not know the meaning of the \'S\=astras nor dost Thou know Thy former enemy; so what shall I pray before Thee?'' Hearing this Janamejaya said :-- ``O highly fortunate one! I do not know who is my enemy; what wrong is there to be redressed? please speak out what I am to do.'' Uttanka said :-- ``O king! the wicked Tak\d{s}ak killed Thy father; ask about the death of Thy father from Thy councillors.'' Hearing these words, the king Janamejaya asked his ministers; they replied ``Thy father died out of the snake Tak\d{s}aka's bite.'' Then the king spoke :-- ``The cause of my father's death is the Br\=ahmi\d{n}'s curse; what is the fault of Tak\d{s}aka in this matter; please say.'' Uttanka said :-- It was Tak\d{s}aka that

gave abundance of wealth to K\=a\'syapa who was coming to cure Thy father of Tak\d{s}aka's poison and made him desist from his purpose; so O king! Is not that Tak\d{s}aka, then, Thy father's great enemy and his slayer?

O King! In former days, when Pramadvar\=a, the dearest wife of the Muni Ruru, died of snake bite in her unmarried state, Ruru made her alive again. But Ruru made then the promise ``whichever serpent will see, I will take away its life by striking it with a club.'' O King! Thus making the resolve, he began to kill snakes wherever he found with his club, and thus, in his course of travel all round the earth, he saw within a forest an aged terrible water-snake (Dhonda serpent) and immediately lifted his club to kill it and angrily struck a blow on it, when the snake replied :-- ``O Br\=ahma\d{n}a! Why are you striking me thus? I have not caused any offence to you.'' Ruru said :-- ``O serpent! My dearest wife died of snake bite; since then I have made this resolve, under great provocation and sorrow, to kill snakes.'' Hearing thus, the water-snake Dundubha replied :-- ``I do not bite; those who bite are a different class of snakes; simply on account of my bearing a body similar to them that you will strike me is not quite proper.'' Hearing these beautiful humane words from the mouth of a serpent, Ruru asked :-- ``Who are you? Why have you become this Dundubha snake?''

33-45. The snake replied :-- ``O Br\=ahma\d{n}a! I was formerly a Br\=ahma\d{n}; there was a friend of mine named Khy\=as, very religious, truthful and self-controlled. Once he was staying in his Agnihotra room and I foolishly terrified him much by placing before him an artificial snake created by me of the leaves of trees. He became so much bewildered with fear and shuddered so terribly that he at length cursed me saying :-- ``O one of blunt intellect! As you have terrified me by this snake, having no poison, so you better be a snake of that type.'' Immediately I turned into a snake and when I much entreated that Br\=ahma\d{n}a, his anger abated a little and he said again :-- ``O snake! Pramati's son Ruru will no doubt free you of this curse.'' I am that snake; and you are also that Ruru; now hear my words in conformity with Dharma. The highest Dharma of the Br\=ahma\d{n}a, is non-killing. There is no doubt in this. The wise Br\=ahma\d{n}as ought to show mercy to all. No harm or killing is to he committed anywhere except in Yaj\~na (sacrifice); killing is only allowed in a Yaj\~na; for at the sacrifice, the animal killed attains the highest goal; hence killing in sacrifice is not reckoned as an act of killing. Uttanka said :-- That Br\=ahma\d{n}a was then freed of the serpent body; and Ruru, too, desisted from killing since then. O King! Ruru gave life back to that girl and married her but even then, remembering the former enmity he killed the snakes. But, O chief of Bharata's family! Thou art staying without

any care, without any anger to the snakes and without any revenge to the previous wrong. O king of kings! Thy father died high up in the air without any bath or charity due to be done at the time of death. So rescue thy father by killing his enemies, the snakes. That son is dead, though living, who does not consider the act of his father's enemy as inimical. Until Thou dost kill the snakes, Thy father's enemies, Thy father's hell life will not be freed. O king! Now remember the wrong done to Thy father and perform the sacrifice to the Great Mother, denominated as the Sarpa Yaj\~na (the sacrifice of snakes).

46-55. S\=uta said :-- Hearing the words of Uttanka, the king Janamejaya sadly wept and shed tears, and thought within himself :-- ``Alas! Fie to me! I am a great stupid; hence I feel myself proud but in vain. Where can his honour be whose father, bitten by a snake, has gone down to hell. Now I will, no doubt, commence the Sarpa Yaj\~na and ensure the destruction of all the snakes in the blazing sacrificial fire and thus deliver my father from hell.'' Thus coming to a conclusion, he called all his ministers and said :-- ``O ministers! Better make arrangements duly for a great sacrifice. Have a suitable holy site on the banks of the Ganges, selected and measured by the Br\=ahma\d{n}as and have a beautiful sacrificial hall built up on one hundred pillars and prepare a sacrificial altar within this. O Ministers! When all these preliminaries will be completed, I will commence with great eclat the great Sarpa Yaj\~na (sacrifice of snakes). In that Yaj\~na, the snake Tak\d{s}ak will be the animal victim; and Uttanka, the great Muni, will be the sacrificial priest; so invite early the all-knowing Br\=ahma\d{n}as, versed in the Vedas. Thus at the command of the king, the able ministers collected all the materials of the sacrifice and prepared a big sacrificial altar. When the oblations were offered on the sacrificial fire, calling on the snakes, Tak\d{s}ak became greatly distressed with fear and took refuge of Indra saying ``Save my life''. Indra, then, gave hopes to Tak\d{s}aka, trembling with fear, and made him sit on his \=Asana, encouraged him with words ``No fear''. O! snake do not fear any more.

56-65. The Muni Uttanka, seeing that Tak\d{s}ak had taken Indra's protection and that Indra had given him hopes of ``no fear'', called on Tak\d{s}aka with Indra to come to fire with an anxious heart; Tak\d{s}ak, then, seeing no other way, took refuge of the greatly religious \=Astik, the son of the Muni Jarat K\=aru, born of the family of Y\=ay\=avara. The Muni's son \=Astik came to the sacrificial hall and chanted hymns in praise of Janamejaya; the king, too, seeing the Muni boy greatly learned worshipped him and said :-- ``What for have you come? I will give you what you desire.'' Hearing this, \=Astika prayed :-- ``O highly enlightened one! Let you desist from this sacrifice.''

The truthful king, prayed thus again and again, stopped the Sarpa Yaj\~na to keep the Muni's word. Vai\'samp\=ayana then recited the whole Mah\=abh\=arata to the king to cheer up his heart. But the king, hearing the whole Mah\=abh\=arata could not find peace and asked Veda Vy\=asa ``how can I get peace; my mind is constantly being burned with sorrow; say what am I to do? I am very miserable; hence my father Par\={\i}ksit the son of Abhimanyu has died an unnatural death. O lucky one! See that a K\d{s}attriya's death in a deadly battlefield or in an ordinary battle is praiseworthy; even his death in his own house, if followed up according to natural laws and Vidhis (rules) is commendable; but my father did not die such a death; under the Br\=ahma\d{n}a's curse why did he, quite senseless, quit his life high up in the air? O son of Satyavat\={\i}! Now advise me so that my father who is now in hell can again go up to the heavens, and that my heart may find its way to peace.''

Thus ends the Eleventh Chapter of the Second Skandha on the ``Sarpa Yaj\~na'' in the Mah\=apur\=a\d{n}a \'Sr\={\i}mad Dev\={\i} Bh\=agavatam of 18,000 verses.



