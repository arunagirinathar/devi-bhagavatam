\chapter{On the one thousand and eight names of the G\=ayatr\={\i}}

1-3. N\=arada said :-- O Bhagav\=an! O All-knowing One! O Thou versed in all the \'S\=astras! I have heard from Thy mouth all the secrets of \'Srutis and Smritis. Now I ask Thee, O Deva! How can the knowledge of that Veda Vidy\=a (Learning) be obtained by which all sins are rooted out and destroyed, how is Brahmaj\~n\=anam obtained and how can Mok\d{s}a be obtained? How can death be conquered and how can

the best results be obtained in this world and in the next. O Lotus-eyed One! Thou ought'st to describe fully all these to me.

4-9. N\=ar\=aya\d{n}a said :-- O N\=arada! O Highly Learned One! S\=adhu! S\=adhu! You have now put a nice question indeed! Now I will describe one thousand and eight names of the G\=ayatr\={\i} Dev\={\i}. Listen attentively. These all sin-destroying auspicious names were composed by Brahm\=a and first recited by Him. Its \d{R}i\d{s}i is Brahm\=a; the Chhandas is Anustup; the Devat\=a is G\=ayatr\={\i}; its V\={\i}ja is Halavar\d{n}a (consonants) and its \'Sakti is Svaravar\d{n}a (vowels). Perform the Anga Ny\=asa and the Kara Ny\=asa by the M\=atrik\=a var\d{n}as (that is, by the fifty syllables). Now hear its Dhy\=anam, that will do good to the S\=adhakas (the practisers). [N.B. :-- Amga Ny\=asa - Touching the limbs of the body with the hand accompanied by appropriate Mantras. Kara Ny\=asa - assignment of the various parts of fingers and hand to different deities which is usually accompanied with prayers and corresponding gesticulations.] I worship the Kum\=ar\={\i} (virgin) G\=ayatr\={\i} Dev\={\i}, the Lotus-eyed One, riding on the Swan (the Pr\=a\d{n}as), and seated on a lotus (creation); Who is three-eyed and of a red colour; and Who is bright and decorated with gems and jewels of red, white, green, blue, yellow and other variegated colours; Who is holding in Her hands Kundik\=a, the rosary, lotus and making signs as if ready to grant the desired boons and on whose neck is suspended the garland of red flowers. I worship the Dev\={\i} G\=ayatr\={\i}. [Note :-- The colours are the various emotions and feelings.]

10-16. Now I will recite the one thousand and eight names of the G\=ayatr\={\i}, beginning with the syllable ``a'' and going on a, \=a, i, \={\i}, etc., in due order of the alphabets. Listen! Her ways and actions cannot be comprehended by intellect (Buddhi); She is therefore Achintya Lak\d{s}a\d{n}\=a; She is Avyakt\=a (unmanifested; unspeakable); She is Artham\=atrimahe\'svar\={\i}, (because She is the Controller of Brahm\=a, etc.); She is Amrit\=arnava madhyasth\=a, Ajit\=a and Apar\=ajit\=a. Thou art A\d{n}im\=adigu\d{n}\=adh\=ar\=a, Arka mandalasamsthit\=a, Ajar\=a, Aj\=a, Apar\=a, Adharm\=a (she has no dharma, caste, etc.), Ak\d{s}as\=utradhar\=a, Adhar\=a; Ak\=ar\=adak\d{s}ak\=ar\=ant\=a (beginning with the syllable ``a'' and ending with the syllable ``k\d{s}a'', thus comprising the fifty syllables), Ari\d{s}advargabhedin\={\i} (destroying the five passions), Anjan\=adriprat\={\i}k\=a\'s\=a, Anjan\=adriniv\=asin\={\i}, Aditi, Ajap\=a, Avidy\=a, Aravindanibhek\d{s}a\d{n}\=a, Antarvahihsthit\=a, Avidy\=adhvamsin\={\i}, and Antar\=atmik\=a. Thou art Aj\=a. Ajamukh\=av\=as\=a (residing in the mouth of Brahm\=a), Aravindanibh\=anan\=a, (Vyanjanavarn\=atmik\=a, therefore called) Ardham\=atr\=a, Arthad\=anaj\~n\=a (because She grants all the Puru\d{s}\=arthas.)

Arimandalamarddin\={\i}, Asuraghn\={\i}, Am\=av\=asy\=a, Alak\d{s}\={\i}ghn\={\i}, Antyaj\=archit\=a. Thus end Her names beginning with ``A''. Now the names with

``\=A'' Thou art \=Adi Lak\d{s}m\={\i}, \=Adi \'Sakti, \=Akriti, \=Ayat\=anan\=a, \=Adityapadavich\=ar\=a, \=Adityaparisevit\=a, \=Ach\=ary\=a, \=Avartan\=a; \=Ach\=ar\=a, and \=Adi M\=urti niv\=asin\={\i}.

17-18. Thou art \=Agney\={\i}, \=Amar\={\i}, \=Ady\=a, \=Ar\=adhy\=a, \=Asanasthit\=a, \=Adh\=ara nilay\=a (seated in the Mul\=adh\=ara), \=Adh\=ar\=a (the Refuge of all), and \=Ak\=a\'s\=anta niv\=asini (of the nature of Aham tattva; Thou art \=Ady\=ak\d{s}ara sam\=ayukt\=a, \=Antar\=ak\=a\'sar\=upin\={\i}, \=Adityamandalagat\=a, \=Antaradhv\=antan\=a\'sin\={\i}, (i.e., destroyer of the Moha of J\={\i}vas). Then come the names beginning with ``I.''

19-25. Thou art Indir\=a, Istad\=a, Ist\=a Ind\={\i}varanivek\d{s}a\d{n}\=a, Ir\=avat\={\i}, Indrapad\=a, Indr\=a\d{n}\={\i}, Indur\=upi\d{n}\={\i}, Ik\d{s}ukodandasamyukt\=a, I\d{s}usandh\=anak\=arin\={\i}, Indran\={\i}lasamak\=ar\=a, Id\=api\d{n}galar\=upin\={\i}, Indr\=ak\d{s}\={\i}, \=I\'svar\={\i}, Dev\={\i} \=Ih\=atrayavivarjit\=a. Thou art Um\=a, U\d{s}\=a, Udunibh\=a, Urv\=arukaphal\=anan\=a, Uduprabh\=a, Udumat\={\i}, Udup\=a, Udumadhyag\=a, \=Urdha, \=Urdhake\'s\={\i}, \=Urdhadhogatibhedin\={\i}, \=Urdhav\=ahupriy\=a, \=Urmim\=al\=av\=aggranthad\=ayin\={\i}. Thou art Rita, \d{R}i\d{s}i, Ritumat\={\i} (the Creatrix of the world), \d{R}i\d{s}idevanamaskrit\=a, Rigved\=a, Ri\d{n}ahartr\={\i}, \d{R}i\d{s}imandala ch\=arin\={\i}, Riddhid\=a, Rijum\=argasth\=a, Rijudharm\=a, Rijuprad\=a, Rigvedanilay\=a, Rijv\={\i}, Lupta dharma pravartin\={\i}, L\=ut\=arivarasam bh\=ut\=a, L\=ut\=adivi\d{s}ah\=ari\d{n}\={\i}.

26-30. Thou art Ek\=ak\d{s}ar\=a, Ekam\=atr\=a, Ek\=a, Ekaikanisthit\=a, Aindr\={\i}, Air\=avat\=ar\=udh\=a, Aihik\=amu\d{s}mikaprad\=a, Omk\=ar\=a, O\d{s}adh\={\i}, Ot\=a, Otaprotaniv\=asin\={\i}, Aurbb\=a, Au\d{s}adhasampann\=a, Aup\=asanaphalaprad\=a, Andamadhyasthit\=a, Ahk\=aramanur\=upin\={\i}. (Visargar\=upi\d{n}\={\i}). Thus end the names beginning with vowels.

Now begin the names beginning with consonants. Thou art K\=aty\=ayan\={\i}, K\=alar\=atri, K\=am\=ak\d{s}\={\i}, K\=amasundar\={\i}, Kamal\=a,. K\=amin\={\i}, K\=ant\=a, K\=amad\=a, K\=alakanthin\={\i}, Karikumbha stana bhar\=a, Karav\={\i}ra Suv\=asin\={\i}, Kalya\d{n}\={\i}, Kundalavat\={\i}, Kuruk\d{s}etraniv\=asin\={\i}, Kuruvinda, dal\=ak\=ar\=a, Kundal\={\i}, and Kumud\=alay\=a.

31-32. Thou art K\=alajhibh\=a, Kar\=al\=asy\=a, K\=alik\=a, K\=alar\=upin\={\i}, K\=aman\={\i}yagu\d{n}\=a, K\=anti, Kal\=adh\=ar\=a, Kumudvat\={\i}, Kau\'sik\={\i}, Kamal\=ak\=ar\=a, K\=amach\=araprabhanjin\={\i}. Thou art Kaumar\={\i}, Karu\d{n}\=ap\=ang\={\i}, Kakubanta (as presiding over all the quarters), and Karipriy\=a.

33-37. Thou art Ke\'sar\={\i}, Ke\'savanut\=a, Kadamba Ku\'sumapriy\=a, K\=alind\={\i}, K\=alik\=a, K\=a\~nch\={\i}, Kala\'sodbhavasamstut\=a. Thou art K\=amam\=at\=a, Kratumat\={\i}, K\=amar\=up\=a, Krip\=avat\={\i}, Kum\=ar\={\i}, Kunda nilay\=a, Kir\=at\={\i}, K\={\i}rav\=ahana, Kaikey\={\i}, Kokil\=al\=ap\=a, Ketak\={\i}, Kusumapriy\=a, Kamandaludhar\=a, K\=al\={\i}, Karmanirm\=ulak\=ari\d{n}\={\i}, Kalahansagati, Kak\d{s}\=a, Krit\=a, Krita, Kautukamangal\=a, Kast\=ur\={\i}tilak\=a, Kamr\=a, Kar\={\i}ndra Gaman\=a, Kuh\=u, Karp\=uralepan\=a, Kri\d{s}\d{n}\=a, Kapil\=a, Kuhar\=a\'sray\=a, K\=utasth\=a, Kudhar\=a, Kamr\=a, Kuk\d{s}isth\=akhilavistap\=a.

Thus end the names with Ka. Now come those with Kha.

38-62. Thou art Khadga Khetadhar\=a, Kharbh\=a, Khechar\={\i}, Khagav\=ahan\=a, Khatt\=anga dh\=ari\d{n}\={\i}, Khy\=ata, Khagar\=ajoparisthit\=a, Khalaghn\={\i}, Khanditajar\=a, Khad\=aksy\=anaprad\=ayin\={\i}, Khandendu tilak\=a.

Thou art Gang\=a, Ga\d{n}e\'sa guhap\=ujita, G\=ayatr\={\i}, Gomat\={\i}, G\={\i}t\=a, G\=andh\=ar\={\i}, G\=analolup\=a, Gautam\={\i}, G\=amin\={\i}, G\=adh\=a, Gandharv\=apsarasevit\=a, Govinda chara\d{n}\=a kr\=a\d{n}t\=a, Gu\d{n}atraya vibh\=abit\=a, Gandharv\={\i}, Gahvar\={\i}, Gotr\=a, Gir\={\i}\'s\=a, Gahan\=a, Gam\={\i}, Guh\=av\=as\=a, Gu\d{n}avat\={\i} (of good qualities), Gurup\=apapra\d{n}\=a\'sin\={\i}, Gurbh\={\i}, Gu\d{n}avat\={\i} (of the three gu\d{n}as), Guhy\=a, Goptavy\=a, Gu\d{n}ad\=ayin\={\i}, Girij\=a, Guhyam\=atang\={\i}, Garudadhvajavallabh\=a, Garv\=apah\=ari\d{n}\={\i}, God\=a (grating Heaven), Gokulasth\=a, Gad\=adhar\=a, Gokar\d{n}anilay\=a sakt\=a, and Guhyamandala vartin\={\i}.

Now the names with ``Gha''. Thou art Gharmad\=a, Ghanad\=a, Ghant\=a, Ghora D\=anava marddin\={\i}, Ghri\d{n}\={\i} mantra may\={\i} (of the S\=urya mantra, Ghri\d{n}\={\i} is to shine). Gho\d{s}\=a, Ghanasamp\=atad\=ayin\={\i}, Ghant\=ara-vapriy\=a, Ghr\=a\d{n}\=a, Ghri\d{n}isantustik\=ari\d{n}\={\i} (giving pleasure to the Sun), Ghan\=arimandal\=a, Gh\=ur\d{n}\=a, Ghrit\=ach\={\i}, Gha\d{n}avegin\={\i}, G\~n\=anadh\=atumay\={\i}. Thou art Charch\=a, Charchit\=a, Ch\=aruh\=asin\={\i}, Chatul\=a, Chandik\=a, Chitr\=a, Chitram\=alyayi bh\=u\d{s}it\=a, Chaturbhuj\=a. Ch\=aru dant\=a, Ch\=atur\={\i}, Charitaprad\=a, Ch\=ulik\=a, Chitravastr\=ant\=a, Cha\d{n}dramah Kar\d{n}a Kundal\=a, Chandrah\=as\=a, Ch\=arud\=atr\={\i}, Chakor\={\i}, Ghandrah\=asin\={\i}, Chandrik\=a, Chandradh\=atr\={\i}, Chaur\={\i}, Chor\=a, Chandik\=a, Chanchadv\=agv\=adin\={\i}, Chandrach\=ud\=a, Choravin\=a\'sin\={\i}, Ch\=aruchandana lipt\=ang\={\i}, Chanchachch\=amarav\={\i}jit\=a, Ch\=arumadhy\=a, Ch\=arugati, Chandil\=a, Chandrar\=upin\={\i}, Ch\=aruhoma priy\=a, Ch\=arv\=a, Charit\=a, Chakrab\=ahuk\=a, Chandramandalamadhyasth\=a, Chandramandala Darpa\d{n}\=a, Chakrav\=akastan\={\i}, Chest\=a, Chitr\=a, Ch\=aruvil\=asin\={\i}, Chitsvar\=up\=a; Chandavat\={\i}, Chandram\=a, Chandanapriy\=a, Chodayitr\={\i} (as impelling the J\={\i}vas always to actions), Chirapraj\~n\=a, Ch\=atak\=a, Ch\=aruhetuk\={\i}.

Thou art Chhatray\=at\=a, Chhatradhar\=a, Chh\=ay\=a, Chhandhahparichchhad\=a, Chh\=ay\=a Dev\={\i}, Chhidranakh\=a, Chhannendriyavisarpi\d{n}\={\i}, Chhandonustuppratisth\=ant\=a, Chhidropadrava bhedin\={\i}, Chhed\=a, Chhatre\'svar\={\i}, Chhinn\=a, Chhurik\=a, and Chhelanpriy\=a. Thou art Janan\={\i}, Janmrarahit\=a, J\=ataveda, Jaganmay\={\i}, J\=ah\d{n}av\={\i}, Jatil\=a, Jatr\={\i} (Jetr\={\i}), Jar\=amara\d{n}a varjit\=a, Jambu dv\={\i}pa vat\={\i}, Jv\=aI\=a, Jayant\={\i}, Jalas\=alin\={\i}, Jitendr\={\i}y\=a, Jitakrodh\=a, Jit\=amitr\=a, Jagatpriy\=a, J\=atar\=upamay\={\i}, Jihv\=a, J\=anak\={\i}, Jagat\={\i}, Jar\=a (Jay\=a) Janitr\={\i}, Jah\d{n}utanay\=a, Jagattrayahitai\d{s}in\={\i}, Jv\=alamul\={\i}, Japavat\={\i}, Jvaraghn\={\i}, Jitavistap\=a, Jit\=akr\=antamay\={\i}, Jv\=al\=a, J\=agrat\={\i}, Jvaradevat\=a, Jvalant\={\i}, Jalad\=a, Jyesth\=a, Jy\=agho\d{s}\=a sphota dinmukh\={\i}, Jambhin\={\i}, Jrimbha\d{n}\=a, Jrimbh\=a, Jvalanm\=a\d{n}ikya Kundal\=a. Jhinjhik\=a, Jha\d{n}anirgho\d{s}\=a, Jhanjh\=a

M\=aruta vegin\={\i}, Jhallak\={\i}v\=adya ku\'sal\=a, Nr\=up\=a, Nbhuj\=a, Ta\d{n}ka bhedin\={\i}, Tanka b\=a\d{n}asam\=ayukt\=a, Tankin\={\i}, Ta\d{n}ka bhedin\={\i}, Tank\={\i}ga\d{n}akrit\=agho\d{s}\=a, Ta\d{n}kan\={\i}ya mahoras\=a, Ta\d{n}k\=ara K\=ari\d{n}\={\i}, Tha tha \'savdanin\=adin\={\i}.

63-80. Now come the names beginning with ``Da.'' They are :-- D\=amar\={\i}, D\=akin\={\i}, Dimbh\=a, Dundam\=araikanirjit\=a, D\=amar\={\i}tantramargasth\=a, Dandadamarun\=adin\={\i}, Dind\={\i}ravasah\=a, Dimbhalasat kr\={\i}d\=apar\=ayan\=a (dancing with joy in battles). Then Dhundhi vigh\d{n}e\'sa janan\={\i}, Dhakk\=a hast\=a, Dhilivraj\=a (followed by \'Siva ga\d{n}as), Nityaj\~n\=an\=a, Nirupam\=a, Nirgu\d{n}\=a and Narmad\=a river. Now :-- Trigu\d{n}\=a, Tripad\=a, Tantr\={\i}, Tulas\={\i}, Tarun\=a, Tara, Trivikramapad\=a kr\=a\d{n}t\=a, T\=ur\={\i}yapadag\=amin\={\i}, Tarun\=a ditya samka\'s\=a, T\=amas\={\i}, Tuhin\=a, Tur\=a, Trik\=alaj\~n\=ana Sampann\=a, Trival\={\i}, Trilochan\=a, Tri \'Sakti, Tripur\=a, Tung\=a, Turangavadan\=a, Timingilagil\=a, Tibr\=a, Trisrot\=a, T\=amas\=adin\={\i}, Tantra mantravi\'se\d{s}aj\~n\=a, Tanumadhy\=a, Trivipstap\=a, Trisandhy\=a, Tristan\={\i}, To\d{s}\=asamsth\=a, T\=alaprat\=apin\={\i}, T\=atankin\={\i}, Tu\d{s}\=ar\=abh\=a, Tuhin\=achala v\=asin\={\i}, Tantuj\=alasam\=ayukt\=a, T\=arah\=ar\=a valipriy\=a, Tilahomapriy\=a, T\={\i}rth\=a, Tam\=ala kusum\=a kriti, T\=arak\=a, Triyut\=a, Tanv\={\i}, Tri\'sam kupariv\=arit\=a, Talodar\={\i}, Tirobh\=a\d{s}\=a, T\=atamka priyav\=adin\={\i}, Trijat\=a, Tittir\={\i}, Tri\d{s}\d{n}\=a, Tribidh\=a, Taru\d{n}\=a krit\={\i}, Tapta k\=anchanasamk\=a\'s\=a, Tapta k\=a\~ncha\d{n}a bh\=u\d{s}an\=a, Traiyambak\=a, Trivarg\=a, Trik\=alaj\~n\=anad\=ayin\={\i}, Tarpa\d{n}\=a, Triptid\=a, Tript\=a, T\=amas\={\i}, Tumvarustut\=a, T\=ark\d{s}yasth\=a, Trigu\d{n}\=ak\=ar\=a, Tribhang\={\i}, Tanuvallar\={\i}, Th\=atk\=ar\={\i}, Th\=arav\=a, Th\=ant\=a, Dohin\={\i}, D\={\i}navatsal\=a, D\=anav\=anta kar\={\i}, Durg\=a, Durg\=asuranivahri\d{n}\={\i}, Devar\={\i}ti, Div\=ar\=atri, Draupad\={\i}, Dunda bhisvan\=a, Devay\=an\={\i}, Dur\=av\=as\=a, D\=aridrya bhedin\={\i}, Div\=a, D\=amodarapriy\=a, D\={\i}pt\=a, Digv\=as\=a, Digvimohin\={\i}, Danda k\=ara\d{n}ya nilay\=a, Dandin\={\i}, Deva p\=ujit\=a, Deva vandy\=a, Divi\d{s}\=ad\=a, Dve\d{s}i\d{n}\={\i}, D\=anav\=a kriti, D\={\i}nan\=a thastut\=a, D\={\i}k\d{s}\=a, Daiva\'s\=a disvarupi\d{n}\={\i}, Dh\=atri, Dhanurdhar\=a, Dhenur Dh\=ari\d{n}\={\i}, Dharmach\=ari\d{n}\={\i}, Dhurandhar\=a, Dhar\=adhar\=a, Dhanad\=a, Dh\=anya dohin\={\i}, Dharma\'s\={\i}l\=a, Dhan\=adhyak\d{s}\=a, Dhanurvedavi\'s\=arad\=a, Dhriti, Dhany\=a, Dhritapad\=a, Dharmar\=ajapriy\=a, Dhruv\=a, Dh\=umavat\={\i}, Dh\=umake\'s\={\i} Dharma\'s\=astrapraka\'sin\={\i}.

81-98. Nand\=a, Nandapriy\=a, Nidr\=a, Nrinut\=a, Nandan\=atmik\=a, Narmmad\=a Nalin\={\i}, N\={\i}l\=a, N\={\i}lakanthasam\=a\'sray\=a, Rudr\=a\d{n}\={\i}, N\=ar\=aya\d{n}apriy\=a, Nity\=a, Nirmmal\=a, Nirgu\d{n}\=a, Nidhi, Nir\=adh\=ar\=a, Nirupam\=a, Nitya\'suddh\=a, Niraj\~n\=an\=a, N\=adabindu Kal\=at\={\i}t\=a, N\=adavindu Kal\=atmik\=a, Nrisimhin\={\i}; Nagadhar\=a, Nripan\=aga vibh\=u\d{s}it\=a, Naraka Kle\'san\=a\'sin\={\i}, N\=ar\=aya\d{n}apadodbhav\=a, Niravady\=a, Nir\=ak\=ar\=a, N\=aradapriyak\=ari\d{n}\={\i}, N\=an\=ajyotih, Nidhid\=a, Nirmal\=atmik\=a, Navas\=utradhar\=a, N\={\i}ti, Nirupa drava k\=ari\d{n}\={\i}, Nandaj\=a, Navaratn\=adhy\=a, Naim\={\i}\d{s}\=ara\d{n}ya v\=asin\={\i}, Navan\={\i}tapriya, N\=ar\={\i}, N\={\i}la j\={\i}m\=uta nisvan\=a, Nime\d{s}i\d{n}\={\i}, Nad\={\i}r\=up\=a, N\={\i}lagr\={\i}v\=a, Ni\'si\'svar\={\i}, N\=am\=aval\={\i}, Ni\'sumbhaghn\={\i}, N\=agaloka niv\=asin\={\i}, Navaj\=amb\=u nadaprakhy\=a, N\=agalok\=a

dhidevat\=a, N\=up\=ur\=a Kr\=antacharan\=a, Narachitta pramodin\={\i}, Nimagn\=a rakta nayan\=a, Nirgh\=ata-sama-nisvan\=a, Nandanody\=anilay\=a, Nirvya hoparich\=ari\d{n}\={\i}.

99-107. P\=arvat\={\i}, Paramod\=ar\=a, Parabrahm\=atmik\=a, Par\=a, Pa\~nchko\'savinirmukt\=a, Pa\~nchap\=ataka-n\=a\'sin\={\i}, Para chitta vidh\=anaj\~n\=a, Pa\~nchik\=a, Pa\~nchar\=upi\d{n}\={\i}. P\=urnim\=a, Param\=a Pr\={\i}ti, Paratejah praka\'sin\={\i}, Pur\=a\d{n}\={\i}, Pauru\d{s}\={\i}, Pu\d{n}y\=a, Pundar\={\i} kanibhek\d{s}an\=a, P\=at\=ala tala nirmmagn\=a, Pr\={\i}t\=a, Pr\={\i}tivivardhin\={\i}, P\=avan\={\i}, P\=ada sahit\=a, Pe\'sal\=a, Pavan\=a\'sin\={\i} Praj\=apati, Pari\'sr\=ant\=a, Parvatastana mandal\=a, Padmapriy\=a, Padmasamsth\=a, Padm\=ak\d{s}\={\i}, Padmasambhav\=a, Padmapatr\=a, Padmapad\=a, Padmin\={\i}, Priyabh\=a\d{s}i\d{n}\={\i}, Pa\'sup\=a\'sa vinirmukt\=a, Purandhr\={\i}, Purav\=asin\={\i}, Pu\d{s}kal\=a, Puru\d{s}\=a, Parbh\=a, P\=arij\=ata Kusumapriy\=a, Pativrat\=a, Pativrat\=a, Pavitr\=ang\={\i}, Pu\d{s}pah\=asa par\=aya\d{n}\=a, Praj\~n\=avat\={\i}sut\=a, Pautr\={\i}, Putrap\=ujy\=a, Payasvin\={\i}, Pattip\=a\'sadhar\=a, Pankti, Pitrilokaprad\=ayin\={\i}, Pur\=an\={\i}, Pu\d{n}ya\'sila, Pr\=a\d{n}at\=arti vin\=a\'sin\={\i}, Pradyumnajanan\={\i}; Pust\=a, Pit\=amahaparigrah\=a, Pundar\={\i}kapur\=av\=as\=a, Pundar\={\i}kasam\=anan\=a, Prithujangh\=a, Prithubhuj\=a, Prithup\=ad\=a, Prith\=udar\={\i}, Prav\=ala\'sobh\=a, Ping\=ak\d{s}\={\i}, P\={\i}tav\=as\=ah, Prach\=apal\=a, Prasav\=a, Pustid\=a, Pu\d{n}y\=a, Pratisth\=a, Pr\=a\d{n}av\=a, Pati, Pa\~nchavarn\=a, Panchav\=a\d{n}\={\i}, Pa\~nchik\=a, Panjarasthit\=a, Param\=ay\=a, Parajyotih, Parapr\={\i}ti, Par\=agati, Par\=ak\=asth\=a, Pare\'san\={\i}, P\=avan\={\i}, P\=avaka Dyut\={\i}, Pu\d{n}yabhadr\=a, Parichchhedy\=a. Pu\d{s}pah\=as\=a, Prith\=udar\=a, P\={\i}t\=ang\={\i}, P\={\i}tavasan\=a P\={\i}ta\'say\=a, Pi\'s\=achin\={\i}, P\={\i}takriy\=a, Pi\'s\=achaghn\={\i}, P\=atal\=ak\d{s}\={\i}, Patukriy\=a, Pa\~nchabhak\d{s}apriy\=ach\=ar\=a, Putan\=a pr\=a\d{n}agh\=atin\={\i}, Pu\d{n}y\=agavanamadhyasth\=a, Pu\d{n}yat\={\i}rthanisevit\=a, Panch\=ang\={\i}, Par\=a\'sakti, Param\=adh\=ada k\=ari\d{n}\={\i}, Pu\d{s}pak\=andasthit\=a, P\=u\d{s}\=a, Po\d{s}it\=akhilavistap\=a, P\=anapriy\=a, Pa\~ncha\'sikh\=a, Pannagopari\'s\=ayin\={\i}, Pa\~ncham\=atr\=atmik\=a, Prithv\={\i}, Pathik\=a, Prithudohin\={\i}, Pur\=a\d{n}any\=ayam\={\i}mans\=a, P\=atal\={\i}, Pu\d{s}pagandhin\={\i}, Pu\d{n}yapraj\=a, P\=arad\=atr\={\i}, Param\=argaikagochar\=a, Prav\=ala\'sobh\=a, P\=ur\d{n}\=a\'s\=a, Pr\=a\d{n}av\=a, Palhabodar\={\i}.

108-149. Phalin\={\i}, Phalad\=a, Phalgu, Phutk\=ar\={\i}, Phalak\=akrit\={\i}, Phani\d{n}dra bhoga\'sayan\=a, Pha\d{n}imandalama\d{n}dit\=a, B\=alab\=al\=a, Bahumat\=a, B\=aI\=atapanibh\=am\'suk\=a, Balabbadrapriy\=a, Vandy\=a, Badav\=a, Buddhisamstut\=a, Band\={\i}dev\={\i}, Bilavat\={\i}, Badi\'saghin\={\i}, Balipr\={\i}y\=a, B\=andhav\={\i}, Bodhit\=a, Buddhirbandh\=ukakusumapriy\=a, B\=ala bh\=anuprabh\=ak\=ar\=a, Br\=ahm\={\i}, Br\=ahma\d{n}a devat\=a, Brihaspatistut\=a. Brind\=a, Brindavana vih\=arin\={\i}, B\=al\=akin\={\i}, Bil\=ah\=ara, Bilavas\=a Bah\=udak\=a, Bahunetr\=a, Bahupad\=a, Bahukar\d{n}\=avatamsik\=a, Bahub\=ahuyut\=a, Bijar\=upin\={\i}, Bahur\=upi\d{n}\={\i}, Bindun\=adakal\=atit\=a, Bindun\=adasvar\=upi\d{n}\={\i}, Baddhagodh\=angulitr\=a\d{n}\=a, Badary\=a\'sramav\=asin\={\i}, Brind\=arak\=a, Brihatskandh\=a, Brihat\={\i}, B\=a\d{n}ap\=atin\={\i}, Brind\=adhyak\d{s}\=a, Bahunut\=a, Vanit\=a, Bahuvikram\=a, Baddhapadm\=asan\=as\={\i}na, Bilvapatratalasthit\=a, Bodhidrumanij\=av\=as\=a, Badisth\=a, Bindu darpa\d{n}\=a, B\=al\=a, V\=a\d{n}\=asanavat\={\i}, Badav\=analavegin\={\i}, Brahm\=anda

bahirantasth\=a, Brahmakanka\d{n}as\=utri\d{n}\={\i}, Bhav\=an\={\i}, Bh\={\i}\d{s}a\d{n}avat\={\i}, Bh\=avin\={\i}, Bhayah\=arin\={\i}, Bhadrak\=al\={\i}, Bhujang\=ak\d{s}\={\i}, Bh\=arat\={\i}, Bh\=arat\=a\'say\=a, Bhairav\={\i}, Bh\={\i}\d{s}a\d{n}\=ak\=ar\=a, Bh\=utid\=a, Bhutim\=alin\={\i}, Bh\=amin\={\i}, Bhoganirat\=a, Bhadrad\=a, Bh\=urivikram\=a, Bh\=utav\=as\=a, Bhrigulat\=a, Bh\=argav\={\i}, Bh\=usur\=archit\=a, Bh\=ag\={\i}rath\={\i}, Bhogavat\={\i}, Bhavanasth\=a, Bhi\d{s}agvar\=a, Bh\=amin\=a, Bhogin\={\i}, Bh\=a\d{s}\=a, Bhav\=an\={\i}, Bh\=uridak\d{s}i\d{n}\=a, Bharg\=atmik\=a, Bh\=amavat\={\i}, Bhavabandhavimochin\={\i}, Bhajan\={\i}y\=a, Bh\=utadh\=atri-ranjit\=a, Bhuvane\'svar\={\i}, Bhujangavalay\=a, Bh\={\i}m\=a, Bherund\=a, Bh\=agadheyin\={\i}; Thou art M\=at\=a, M\=ay\=a, Madhumat\={\i}, Madhujihav\=a, Manupriy\=a, Mah\=adev\={\i}, Mah\=abh\=ag\={\i}\=a, M\=aliri, M\={\i}nalochan\=a, M\=ay\=at\={\i}t\=a, Madhumat\={\i}, Madhum\=ans\=a, Madhudrav\=a, M\=anav\={\i}, Madhusambh\=ut\=a, Mithil\=apurav\=asin\={\i}, Madhukaitabhasamhartr\={\i}, Medin\={\i}, Megham\=alin\={\i}, Mandodar\=a, Mah\=a M\=ay\=a, Maithil\={\i}, Masri\d{n}apriy\=a, Mah\=a Lak\d{s}m\={\i}, Mah\=a K\=al\={\i}, Mah\=a Ka\d{n}y\=a, Mahe\'svar\={\i}, M\=ahendr\={\i}, Merutanay\=a Mand\=arakusum\=archit\=a, Manjumanj\={\i}rachara\d{n}\=a, Mok\d{s}ad\=a, Manjubha\d{s}i\d{n}\={\i}, Madhuradr\=avin\={\i}, Mudr\=a, Malay\=a, Malay\=anvit\=a, Medh\=a, Marakata\'sy\=am\=a, M\=agadh\={\i}, Menak\=atmaj\=a, Mah\=am\=ar\={\i}, Mah\=av\={\i}r\=a, Mah\=a\'sy\=am\=a, Manustut\=a, M\=atrik\=a, Mihir\=abh\=as\=a, Mukundapada Vikram\=a, M\=ul\=adh\=arasthit\=a, Mugdh\=a, Ma\d{n}ip\=uraniv\=asin\=a, Mrig\=aks\={\i}, Mahi\d{s}\=ar\=udh\=a, Mahis\=asuramardin\={\i}. Thou art Yog\=asan\=a, Yogagamy\=a, Yog\=a, Yauvanak\=a\'sray\=a, Yauvan\={\i}, Yuddhamadhyasth\=a, Yamun\=a, Yug\=adhari\d{n}\={\i}, Yak\d{s}i\d{n}\={\i}, Yogayukt\=a, Yaksar\=ajapras\=utin\={\i}, Y\=atr\=a, Y\=ana bidhanaj\~n\=a, Yaduva\d{n}\'sasamudbhav\=a, Yak\=ar\=adi-Ha K\=ar\=ant\=a, (all \=antahstha var\d{n}as), Y\=aju\d{s}\={\i}, Yaj\~n\=a r\=upi\d{n}\={\i}, Y\=amin\={\i}, Yoganirat\=a. Y\=atudh\=ana, bhayamkar\={\i}, Rukmi\d{n}\={\i}, Rama\d{n}\={\i}, R\=am\=a, Revat\={\i}, Re\d{n}uk\=a, Rat\={\i}, Raudr\={\i}, Raudrapriy\=ak\=ar\=a R\=ama m\=at\=a, Ratipriy\=a, Rohi\d{n}\={\i}, R\=ajyad\=a, Rev\=a, Ras\=a, R\=aj\={\i}valochan\=a, R\=ake\'s\={\i}, R\=upasampann\=a, Ratnasimh\=a\'sanasthit\=a, Raktam\=aly\=ambaradhar\=a, Raktagandh\=anu lepan\=a, R\=aja hamsa sam\=ar\=udh\=a, Rambh\=a, Raktavalipriy\=a, Rama\d{n}\={\i}yayug\=adh\=ar\=a, R\=ajit\=akhilabh\=utal\=a, Rurucharmapari-dh\=an\=a, Rathin\={\i}, Ratnam\=alik\=a, Roge\'s\={\i}, Roga\'saman\={\i}, R\=avin\={\i}, Romahar\d{s}i\d{n}\={\i}, R\=amachandra pad\=a Kr\=ant\=a, Rava\d{n}achchhedak\=ari\d{n}\={\i}, Ratnavastra parichchhinv\=a, Rathasth\=a, Rukma bh\=u\d{s}a\d{n}\=a, Lajj\=adhidevat\=a, Lol\=a, Lalit\=a, Lingadh\=ari\d{n}\={\i}, Lak\d{s}m\={\i}, Lol\=a, Luptavi\d{s}\=a, Lokin\={\i}, Lokavi\'srut\=a, Lajj\=a, Lambodar\={\i}, Lalan\=a, Lokadh\=arin\={\i} Varad\=a, Vandit\=a, Vidy\=a, Vai\d{s}\d{n}av\={\i}, Vimal\=akriti, V\=ar\=ah\={\i}, Viraj\=a, Var\d{s}\=a, Varalak\d{s}m\={\i}, Vil\=asin\={\i}, Vinat\=a, Vyomamadhyasth\=a, V\=arij\=asanasamsthit\=a, V\=aru\d{n}\={\i}, Ve\d{n}usambhut\=a, V\={\i}tihotr\=a, Vir\=upi\d{n}\={\i}, V\=ayumandalamadhyasth\=a, Vi\d{s}\d{n}ur\=up\=a, Vidhikriy\=a, Vi\d{s}\d{n}upatn\={\i}, Vi\d{s}\d{n}umat\={\i}, Vi\'s\=al\=ak\d{s}i, Vasundhar\=a, V\=amadevapriy\=a, Vel\=a, Vajri\d{n}\={\i}, Vasudohin\={\i}, Ved\=ak\d{s}arapar\={\i}t\=amg\={\i}, V\=ajapeya-phalaprad\=a, V\=asav\={\i}, V\=amajanan\={\i}, Vaikunthanilay\=a, Var\=a, Vy\=asapriy\=a Varmadhar\=a, V\=alm\={\i}kiparisevit\=a.

Thou art \'Sakambhar\={\i}, \'Siv\=a, \'Sant\=a, \'Sarad\=a, \'Sara\d{n}\=agati, \'S\=atodar\={\i}, \'Subh\=ach\=ar\=a, \'Sumbh\=asuramardin\={\i}, \'Sobh\=abati, \'Siv\=ak\=ar\=a, \'Samkar\=ardha\'sar\={\i}ri\d{n}i, \'So\d{n}\=a (red), \'Subh\=a\'say\=a, \'Subhr\=a, \'Sirahsandh\=anak\=ari\d{n}\={\i}, \'Sar\=avat\={\i}, \'Sar\=anand\=a, \'Sarajjyotan\=a, \'Subb\=anan\=a, \'Sarabh\=a, \'S\=ulin\={\i}, \'Suddh\=a, \'Sabar\={\i}, \'Sukav\=ahan\=a, \'Sr\={\i}mat\={\i}, \'Sr\={\i}dhar\=anand\=a, \'Srava\d{n}\=anandad\=ayin\={\i}, \'Sarv\=a\d{n}\={\i}, \'Sarbhar\={\i}vandy\=a, Sadbh\=a\d{s}\=a, Sadritupriy\=a, Sad\=adh\=arasthit\=adev\={\i}, Sa\d{n}mukhapriyak\=ari\d{n}\={\i}, Sadamgar\=upasumati, Sur\=asuranama\d{s}krit\=a.

150-155. Thou art Sarasvat\={\i}, Sad\=adh\=ar\=a, Sarvamangalak\=ari\d{n}\={\i}, S\=amag\=anapriy\=a, S\=uk\d{s}m\=a, S\=avitr\={\i}, S\=amasambhav\=a, Sarvav\=as\=a, Sad\=anand\=a, Sustan\={\i}, S\=agar\=ambar\=a, Sarvai\'syaryapriy\=a, Siddhi, S\=adhubandhupar\=akram\=a, Saptar\d{s}imandalagat\=a, Somamandalav\=asin\={\i}, Sarvaj\~n\=a, S\=andrakaru\d{n}\=a, Sam\=an\=adhikavarjit\=a, Sarvottung\=a, Sangah\={\i}n\=a, Sadgu\d{n}\=a, Sakalestad\=a, Saragh\=a (bee), S\=uryatanay\=a, Suke\'s\={\i}, Somasamhati, Hira\d{n}yavar\d{n}\=a, Hari\d{n}\={\i}, Hr\={\i}mk\=ar\={\i}, Hamsav\=ahin\={\i}, K\d{s}aumavastrapar\={\i}t\=a\d{n}g\={\i}, K\d{s}\={\i}r\=abdhitanay\=a, K\d{s}am\=a, G\=ayatr\={\i}, S\=avitr\={\i}, P\=arvat\={\i}, Sarasvat\={\i}, Vedagarbh\=a, Var\=aroh\=a, \'Sr\={\i} G\=ayatr\={\i}, and Par\=amvik\=a.

156-159. O N\=arada! Thus I have described to you one thousand (and eight) names of G\=ayatr\={\i}; the hearing of which yields merits and destroys all sins and gives all prosperity and wealth. Specially in the Astam\={\i}tithi (eighth lunar day) if after one's meditation (dhy\=anam) worship, Homa, and japam, one recites this in company with the Brahm\=a\d{n}as, one gets all sorts of satisfactions. These one thousand and eight names of the G\=ayatr\={\i} ought not to be given to anybody indiscriminately. Speak this out to him only who is very devoted, who is a Brahm\=a\d{n}a, and who is an obedient disciple. Even if any devotee, fallen from the observances of \=Achar\=a (right way of living), be a great friend, still do not disclose this to him.

160-165. In whatever house, these names are kept written, no cause of fear can creep in there and Lak\d{s}m\={\i}, the Goddess of wealth, though unsteady, remains steady in that house.

This great secret yields merits to persons, gives wealth to the poor, yields mok\d{s}a to those who are desirous of it, and grants all desires. If anybody reads this, he gets cured of his diseases, and becomes freed from bondages and imprisonment. All the Great Sins, for example, murdering Br\=ahma\d{n}as, drinking wine, stealing gold, going to the wife of one's Guru, taking gifts from bad persons, and eating the uneatables, all are destroyed, yea, verily destroyed! O N\=arada! Thus I have recited to you this Great Secret. All persons get, indeed, united with Brahm\=a (Brahama s\=ayujya) by this. True. True. True. There is not the least trace of doubt here.

Here ends the Sixth Chapter of the Twelfth Book on the one thousand and eight names of the G\=ayatr\={\i} in the Mah\=apur\=a\d{n}am \'Sr\={\i} Mad Dev\={\i} Bh\=agavatam of 18,000 verses by Mahar\d{s}i Veda Vy\=asa.



