\chapter{On the peace of the world}

1-11. Janamejaya said :-- O best of \d{R}i\d{s}is! I have now seen the wonderful excellent deeds of the Dev\={\i} for the enhancement of peace in this world. Though I have heard from thy lotus face these nectar-like words, still I am not satisfied. O best of Munis! What did the chief Devas do when the Goddess disappeared, kindly say to me. O Bhagav\=an! I think those J\={\i}vas cannot fully comprehend these excellent sacred deeds of the Dev\={\i}, that are less fortunate and have done not many meritorious deeds in this world. O Muni! What to speak of the less fortunate souls, even the Mah\=atmas who are well versed in hearing such things, can hardly be satiated on hearing the Dev\={\i}'s deeds. O! Fie to those, that do not hear of these things, the essence of essences, on hearing which men become Immortals. The Mother's L\={\i}l\=a is to preserve the Devas as well as the great Munis and to serve as a boat for the human beings to cross this ocean of world. How can, then, the grateful souls forsake Her? The Pundits versed in the Vedas declare, that the Dev\={\i}'s life is able to fulfil all the desires. Therefore the liberated souls that want liberation, the worldly souls, the diseased all ought to drink incessantly the nectar-like nectar of Dev\={\i}'s doings. Especially the kings that are engaged in Dharma, in earning wealth and in enjoyments, ought to hear Her life. O Muni! When the liberated souls drink the nectar-like doings of the Dev\={\i}, what doubt can there be with the ordinary human beings, to listen with rapt devotion those wondrous things! O Best of Munis! It is those that worshipped the Goddess Bhav\=an\={\i} in their previous births with

beautiful Kunda flowers, Champaka flowers and Bel leaves, they have, it is inferred, in their present births become possessed of rich enjoyments. And those devoid of any devotion, that obtained this human body in the land of Bh\=arata and did not worship the Mother Goddess, they are, in the present births, without grains and riches, diseased, and void of any issues. Wander they always as servants, carrying out orders, and bearing on the burden loads; day and night, they seek for their own selfish ends, yet they cannot get their belly full meals. The blind, deaf and dumb; lame and lepers suffer pain and misery in this earth; seeing them, it should be inferred that they never worshipped the Goddess Bhav\=an\={\i}. And those that are wealthy, prosperous, attended by numerous attendants and are always enjoying, like kings, it is to be inferred that they certainly worshipped the lotus feet of the Mother Goddess in their past lives.

12-15. Therefore O Son of Satyavat\={\i}! As you are kind-hearted, kindly narrate before me the excellent deeds of the Dev\={\i}. O best of Munis! Where did the Goddess, Mah\=a Lak\d{s}m\={\i}, created out of the energies of all the gods, depart after She had slain the Mahi\d{s}\=asura and had been worshipped and praised by the Devas? O highly Fortunate one! You told me that She vanished from the sight of the Devas; now I like to know where is She staying now, whether in the Heavens or in the Land of Mortals? Did She melt away then and there or did She descend to Vaikuntha or did She go to the mountain Sumeru? O Muni! Narrate all these duly before me.

16-50. Vy\=asa said :-- O King! I told you before about the beautiful Ma\d{n}i Dv\={\i}pa; that island is the place of sport to the Dev\={\i} and very dear to Her. In that place Brahm\=a, Vi\d{s}\d{n}u, Mah\=adeva were transformed into females; they afterwards became males and were engaged in their respective duties. That place is grand and splendid and is in the centre of the ocean of Nectar; the Dev\={\i} Ambik\=a assumes various forms there as She likes; and She sports there. To that Ma\d{n}i Dv\={\i}pa the auspicious Dev\={\i} departed after She had been praised by the Gods, to that place where sports always the eternal Bhagavat\={\i} Bhuvane\'svar\={\i}, the incarnate of Para Brahm\=a. When the Highest Goddess vanished, the Devas installed, on the throne of Mahi\d{s}\=asura, the powerful King \'Satrughna, endowed with all auspicious qualities, the Lord of Ajodhy\=a and descended from the Solar line. After making him thus the King, Indra and the other Devas went to their respective abodes on their own conveyances. O King! The Devas having gone to their places, the subjects were governed on this earth according to Dharma; and they passed their times in ease and comfort. It used to rain, then, timely and the earth was covered with plenty of grains and wealth; the

trees were all filled with fruits and leaves and gave enjoyment to people. The cows with their udders full like earthen pots gave such a profuse quantity of milk that men began to milk them whenever they liked. The river\'s waters were all clear and cooling; and they flowed full in regular channels; the birds grouped round them. The Br\=ahma\d{n}as, versed in the Vedas, were engaged in performing sacrifices; the K\d{s}attriyas observed their virtues and were engaged in doing charities and in their education; the kings held their rods of justice and were engaged in governing their subjects; though the several kings were busy with various arms and weapons, they all became fond of peace. Thus no wars nor quarrels were seen amongst the subjects; and the mines yielded plenty of wealth to the people. O best of Kings! There were the Br\=ahma\d{n}s, K\d{s}attriyas, Vai\'syas and \'S\=udras who became the devotees of the Goddess. The Br\=ahma\d{n}as and K\d{s}attriyas used, then, to perform so many sacrifices that, at every nook and corner in this globe, the sacrificial altars and the sacrificial posts became visible. The female sex became gentle and of good behaviour, truthful and chaste towards their husbands respectively. Atheism and unrighteous acts vanished entirely from the face of the earth; the people left all dry discussions; they argued only about the \'S\=astras that did not go in contra-distinction to the Vedas. Nobody liked to quarrel with each other; poverty, and evil inclinations were checked; the people everywhere lived in happiness. Untimely death was not there; so the people had no bereavements with their friends; no distress was seen. Famine, want of rains, and deadly plagues were out of sight. The people had no illness even; and jealousies and quarrels vanished. O King! all men and women began to sport merrily everywhere like the Gods in Heaven. Theft, atheism, deceit, vanity, hypocrisy, lustfulness, stupidity, and the anti-Vedic feelings were not to be seen. O Lord of the Earth! All the men were then extremely devoted to their Dharma and engaged in serving the Br\=ahma\d{n}as. The Br\=ahmi\d{n}s were also, according to the three-fold plan of the creation, S\=attvik, R\=ajasik and T\=amasik. The S\=attvik Br\=ahmi\d{n}s were all versed in the Vedas, clever and truthful; they were kind, they controlled their passions and they did not accept any presents from others. Filled with their ideas of Dharma, they used to perform their Purod\=a\d{s}a and other such sacrifices with S\=attvik rice, etc., but never, never did they immolate any animals.* O King! The S\=attvik Br\=ahma\d{n}as gave charities, studied the Vedas and offered sacrifices for themselves. These were their three ordained actions. They were busy in these. O King! The R\=ajasik Br\=ahma\d{n}as were versed in the Vedas and acted as priests to the K\d{s}attriyas

*N.B. - Where the victim is fastened during the time of immolation.

and ate flesh as sanctioned by recognised rules. They were busy with their six duties. They offered sacrifices on their own behalf, assisted others in sacrifices, took gifts, made charities, studied and taught others the Vedas. The T\=amasik Br\=ahma\d{n}as were angry, attached to worldly objects, and jealous. They studied very little of the Vedas and spent most of their time in serving the kings. O King! Mahi\d{s}\=asura was killed, all the Br\=ahma\d{n}as were glad and began to practise Dharma according to the Vedas, observed vows and made charities. The K\d{s}attriyas began to govern the subjects, the Vaisya carried on their trading business and the other tribes went on with their agriculture, preservation of the cows, and lending money on interest. Thus all men became vary glad on the death of Mahi\d{s}a. Devoid of cares and anxieties, the subjects got much wealth! The cows were endowed with suspicious signs and gave plenty of milk and the rivers flowed full of waters. The trees looked splendid with abundance of fruits; men were without diseases: in short, people had no mental agony and too much or too little of rains were not there; \'Salavas, mice, birds, and seditions we not extant. O King! The beings died not prematurely; rather enjoyed incessantly, their full health and possessed lots of riches; especially beings, engaged in the Vedic Dharma, served the lotus feet of Chandik\=a and thus spent their lives.

Here ends the Twentieth Chapter of the Fifth Book on the peace of the world in the Mah\=apur\=a\d{n}am \'Sr\={\i} Mad Dev\={\i} Bh\=agavatam of 18,000 verses by Mahar\d{s}i Veda Vy\=asa.



