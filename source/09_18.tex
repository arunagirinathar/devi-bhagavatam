\chapter{On the union of \'Sankhach\=uda with Tulas\={\i}}

1-26. N\=ar\=aya\d{n}a said :-- Thus highly pleased, Tulas\={\i} went to sleep with a gladdened-heart. She, the daughter of Vri\d{s}adhvaja, was then in her blooming youth and while asleep, the Cupid, the God of five arrows, shot at her five arrows (by which one gets enchanted and swooned). Though the Dev\={\i} was smeared with sandal paste and She slept on a bed strewn with flowers, her body was felt as if being burnt. Out of joy, the hairs stood on their ends all over her body; her eyes were reddened and her body began to quiver. Sometimes She felt uneasiness, sometimes dryness; sometimes She got faint; sometimes drowsiness and sometimes again pleasantness; sometimes she became conscious, sometimes sorrowful. Sometimes she got up from her bed; sometimes she sat; and sometimes she fell again to sleep. The flower-bed, strewn with sandal-paste, appeared to her full of thorns; nice delicious fruits and cold water appeared to her like poison. Her house appeared to her like a hole in a ground and her fine garments seemed to her like fire. The mark of Sind\=ura on her forehead appeared, as it were, a boil, a sore. She began to see in her dreams that one beautiful, well clothed, humorous, young man with smile in his lips, appeared to her. His body was besmeared with sandal-paste and decked with excellent jewels; garlands of forest flowers were suspending from his neck. Coming there, he was drinking the honey of her lotus face. He was speaking on love themes and on various other sweet topics. As if he was embracing amorously and enjoying the pleasures of intercourse. After the intercourse he was going away; again he was coming near.

The lady was addressing him, ``O Darling! O Lord of my heart! Where do you go. Come close.'' Again when she became conscious, she began to lament bitterly. Thus on entering in her youth, the Dev\={\i} Tulas\={\i} began to live in the hermitage of Badari (Plum fruit, it may signify womb. Those who visit Badari are not to enter again in any womb). On the other hand the great Yogi \'Sankhach\=uda obtained the Kri\d{s}\d{n}a Mantra from Mahar\d{s}i Jaig\={\i}\d{s}avya and got siddhi (success) in Pu\d{s}kara T\={\i}rtha (sacred place of pilgrimage where one crosses the world). Holding on his neck the Kavacha named Sarvamangalamaya and obtaining the boon from Brahm\=a as he desired, he arrived at Badari, by Brahm\=a's command. The signs of the blooming youth had just begun to be visible in the body of \'Sankhach\=uda as if the God of Love incarnated in his body; his colour resembled that of white Champakas and all his body was decked with jewelled ornaments. His face resembled the autumnal full moon; his eyes were extended like the lotus leaves. The beautiful form was seen to sit in an excellent aerial car, made of pearls and jewels. Two jewel earrings, nice and elegant, suspended upto his cheek; his neck was adorned with P\=arij\=ata flower garlands; and his body was smeared with Kumkum and scented sandal-paste. O N\=arada! Seeing \'Sankhach\=uda coming near to her, Tulas\={\i} covered her face by her clothing and she, with a smiling countenance, cast repeatedly sidelong glances at him and bent her head low abashed in the expectation of a fresh intercourse. How beautiful was that clear face of her! It put down the autumnal moon in the background. The invaluable jewelled ornament were on her toes. Her braid of hair was surrounded by sweet scented M\=alat\={\i} garlands. The invaluable jewelled wonderful earrings like the shape of a shark were hanging up to her cheek. Extraordinarily beautiful necklaces were seen being suspended to the middle of her breasts and added to the beauty thereof; on her arms and hand were jewelled bangles and conch ornaments; jewelled armlets and on fingers; excellent jewelled rings were seen. O Muni! Seeing that lovely beautiful chaste woman of good nature, \'Sankhach\=uda came to her and taking his seat addressed her as follows :--

27. ``O Proud One! O Auspicious One! Who are you? Whose daughter are you? You look fortunate and blessed among women. I am your silent slave. Talk with me.''

28-30. That beautiful eyed Tulas\={\i}, full of love, replied to \'Sankhach\=uda with smiling countenance and face bent low :-- ``I am the daughter of the great king Vri\d{s}adhvaja. I have come to this forest for tapasy\=a and am engaged in this. Who are you? What business have you to talk with me? You can go away wherever you like. I have heard

in the \'S\=astras that persons born of a noble family never speak with ladies of a respectable family in privacy.

31-71. Only those that are lewd, void of any knowledge in the Dharma \'S\=astras, void of the Vedic knowledge and who are not Kul\={\i}nas, like to speak with women in privacy. And those woman, too, that look externally beautiful but very passionate and the Death of males, who are sweet tongued but filled with venom in their hearts, those who are sweet externally but like a sword internally, those that are always bent in achieving their own selfish ends and those that become obedient to their husbands for their own selfish ends otherwise behaving as they like, those that are filled inside with dirty things and outside looking pleasant in their faces and eyes, whose characters are pronounced as defiled, what intelligent, learned and noble-minded man can trust them? Those women do not discriminate who are their friends or who are their enemies; they want always new persons. Whenever they see a man well dressed, they want to satisfy their own passions. And they pretend with great care that they are very chaste. They are the vessels of passion; they always attract the minds of others and they are very enthusiastic in satisfying their own lust. They verbally shew that they want other men to go away but at hearts, feelings for intercourse remain preponderant; whenever they see their paramours in private, they laugh and become very glad but externally their shame, knows no bounds. When they do not have their intercourses with their paramours, they become self-conceited; their bodies burn with anger and they begin to quarrel. When their passions are satisfied fully, they become glad and when there is a deficiency in that, they become sorrowful. For the sake of good and sweet food and cold drinks, they want beautiful young persons, qualified and humorous. They consider witty young persons clever in holding intercourses, more dearly than their sons. And if that beloved one becomes incapable or aged, then he is considered as an enemy. Quarrels and anger then ensue. They devour these men as serpents eat rats. They are boldness personified and they are the source of all evils and vices. Even Brahm\=a, Vi\d{s}\d{n}u and Mahe\'sa remain deluded before them. They cannot find out any clue of their minds. They are the greatest obstacle in the path of tapasy\=a and the closed doors for liberation. Devotion to Hari cannot reach those women. They are the repositories of M\=ay\=a and they hold men fast by iron chains in this word. They are like magicians and false like dreams. They enchant others by external beauty; their lower parts are very ugly and filled with excrements, faeces, of foul scent and very unholy and smeared with blood. The Creator Bhagav\=an has created them as such, the M\=ay\=a to

the M\=ay\=avis and the venom to those who want liberation, and as invisible to those that want to have them.'' Thus saying Tulas\={\i} stopped. O N\=arada! \'Sankhach\=uda, then smilingly addressed her as follows :-- ``O Dev\={\i}! What you have spoken is not wholly false; partly it is true and partly it is false. Now hear. The Creator has created this all-enchanting female form into two parts. One is praiseworthy and the other is not. He has created Lak\d{s}m\={\i}, Sarasvat\={\i}, Durg\=a, S\=avitr\={\i} and R\=adh\=a and others as the primary causes of creation; so there are the prime creations: Those women that are born of their parts, are auspicious, glorious and much praiseworthy. \'Satar\=up\=a, Devah\=ut\={\i}, Svadh\=a, Sv\=ah\=a, Dak\d{s}i\d{n}\=a Chh\=ay\=avat\={\i}, Rohi\d{n}\={\i}, Varun\=an\={\i}, \'Sach\={\i}, the wife of Kuvera, Diti, Aditi, Lop\=amudr\=a, Anas\=uy\=a, Kautabh\={\i} (Kotar\={\i}), Tulas\={\i}, Ahaly\=a, Arundhat\={\i}, Men\=a, Tar\=a, Mandodar\={\i}, Damayant\={\i}, Vedavat\={\i}, Gang\=a, Manas\=a, Pusti, Tusti, Smriti, Medh\=a, K\=alik\=a, Vasundhar\=a, Sasth\={\i}, Mangalacha\d{n}d\={\i}, M\=urti, wife of Dharma. Svasti, \'Sraddh\=a; \'S\=anti, K\=anti, K\d{s}\=anti, Nidr\=a, Tandr\=a, K\d{s}udh\=a, Pip\=as\=a, Sandhy\=a, R\=atri, Div\=a, Sampatti, Dhriti, K\={\i}rt\={\i}, Kr\={\i}y\=a, \'Sobh\=a, Prabh\=a, \'Siv\=a, and other women born of the Prime Prakritis, all are excellent in every Yuga. The prostitutes of the heavens are also born of the above women in their parts and parts of parts. They are not praiseworthy in the universe; they are all regarded as unchaste women. Those women that are of S\=attva Gu\d{n}as are all excellent and endowed with influence. In the universe they are good, chaste and praiseworthy. This is not false. The Pandits declare them excellent. Those that are of R\=ajo Gu\d{n}as, and T\=amo Gu\d{n}as are not so praiseworthy. Those women that are of R\=ajo Gu\d{n}as are known as middling. They are always fond of enjoyments, yield to them, and always ready to achieve their own ends. These women are generally insincere, delusive, and outside the pale of religious duties. Therefore they are generally unchaste. The Pandits consider them as middling. Those women that are of T\=amo Gu\d{n}as are considered as worst. Those born of noble families, can never speak with other wives in a private place or when they are alone. By Brahm\=a's command I have come to you. O Fair One! I will marry you now according to the Gandharba method. My name is \'Sankhach\=uda. The Devas fly away from me out of terror. Before I was the intimate \'Sakh\=a (friend) of \'Sr\={\i} Hari, by the name of Sud\=am\=a. Now, by R\=adhik\=a's curse I am born in the family of the D\=anavas. I was a P\=ari\d{s}ad (attendant) of \'Sr\={\i} Kri\d{s}\d{n}a and the chief of the eight Gopas. Now, by R\=adhik\=a's curse I am born as \'Sankhach\=uda, the Indra of the D\=anavas. By \'Sr\={\i} Kri\d{s}\d{n}a's grace and by His mantra, I am J\=atismar\=a (know of my past births). You, too, are J\=atismar\=a Tulas\={\i}. \'Sr\={\i} Kri\d{s}\d{n}a enjoyed you before. By

R\=adhik\=a's anger, you are now born in Bh\=arata. I was very eager to enjoy you then; out of R\=adhik\=a's fear I could not.''

72-87. Thus saying, \'Sankhach\=uda stopped. Then Tulas\={\i} gladly and smilingly replied :-- ``Such persons (like you) are famous in this world; good women desire such husbands. Really, I am now defeated by you in argument. The man who is conquered by woman is very impure and blamed by the community. The Pitri Lokas, the Deva Lokas, and the Gandharba Loka, too, look upon men, overpowered by women, as mean, despicable. Even father, mother, brother, etc., hate them mentally. It is said in the Vedas that the impurities during birth and death are expiated by a ten days observances for the Br\=ahma\d{n}as, by twelve days observances for the K\d{s}attriyas, by fifteen days observances for the Vai\'syas and by one month's observances for the \'S\=udras and other low castes. But the impurity of the man who is conquered by women cannot be expiated by any other means except (his dead body) being burned in the funeral pyre. The Pitris never accept willingly the pindas and offerings of water (Tarpa\d{n}as) offered by the women-conquered men. So much so that the Devas even hesitate to accept flowers, water, etc., offered by them on their names. Those whose hearts are entirely subdued by men, do not acquire any fruits from their knowledge, Tapasy\=a, Japam, fire sacrifices, worship, learning and fame. I tested you to ascertain your strength in learning. It is highly advisable to choose one's husband by examining his merits and defects. Sin equivalent to the murder of a Br\=ahmi\d{n} is committed if one gives in marriage one's daughter to one void of all qualifications, to an old man, to one who is ignorant, to a poor, illiterate, diseased, ugly, very angry, very harsh, lame, devoid of limbs, deaf, dumb, inanimate like, and who is impotent. If one gives in marriage a daughter to a young man of good character, learned, well qualified and of a peaceful temper, one acquires the fruits of performing ten horse sacrifices. If one nourishes a daughter and sells her out of greed for money, one falls to the Kumbh\={\i}p\=aka hell. That sinner drinks the urine and eats the excrements of that daughter, remaining in that hell. For a period equal to the fourteen Indra's life-periods they are bitten by worms and crows. At the expiry of this period, they will have to be born in this world of men as diseased persons. In their human births they will have to earn their livelihood by selling flesh and carrying flesh.''

88-100. Thus saying, when Tulas\={\i} stopped, Brahm\=a appeared on the scene and addressed \'Sankach\=uda :-- O \'Sankhach\=uda! Why are you spending uselessly your time in vain talks with Tulas\={\i}? Marry her soon by the Gandharba method. As you are a gem amongst

males, so She is a gem amongst females. It is a very happy union between a humorous lover and a humorous beloved. O King! Who despises the great happiness when it is at one's hand! He who forsakes the pleasure is worse than a beast in this world. O Tulas\={\i}! And what for are you testing the nobly qualified person who is the tormentor of the Devas, Asuras and D\=anavas. O Child! As Lak\d{s}m\={\i} Dev\={\i} is of N\=ar\=aya\d{n}a, as R\=adhik\=a is of Kri\d{s}\d{n}a; as is My S\=av\={\i}tr\={\i}, as Bhava's is Bhav\=an\={\i}, as Boar's is Earth, as Yaj\~na's is Dak\d{s}\={\i}\d{n}\=a, Atri's Anas\=uy\=a, Gautama's Ahaly\=a, Moon's Rohi\d{n}\={\i}, Brihaspati's T\=ar\=a, Manu's \'Satar\=up\=a, Kandarpa's Rati, Ka\'syapa's Aditi, Va\'sistha's Arundhat\={\i}, Karddama's Devah\=uti, Fire's Sv\=ah\=a, Indra's \'Sach\={\i}, Ga\d{n}e\'sa's Pusti, Skanda's Devasen\=a, and Dharma's M\=urti, so let you be the dear wife of \'Sankhach\=uda. Let you remain with \'Sankhach\=uda, beautiful as he is, for a long time, and enjoy with him in various places as you like. When \'Sankhach\=uda will quit his mortal frame, you would go to Goloka and enjoy easily with the two-armed \'Sr\={\i} Kri\d{s}\d{n}a, and in Vaikuntha with the four-armed Kri\d{s}\d{n}a and with great gladness.

Here ends the Eighteenth Chapter of the Ninth Book on the union of \'Sankhach\=uda with Tulas\={\i} in the Mah\=apur\=a\d{n}am \'Sr\={\i} Mad Dev\={\i} Bh\=agavatam of 18,000 verses by Mahar\d{s}i Veda Vy\=asa.



