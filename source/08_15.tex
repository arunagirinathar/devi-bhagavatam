\chapter{On the motion of the Sun}

1-45. N\=ar\=aya\d{n}a said :-- O N\=arada! I will now describe the motion of the Sun. Hear. It is of three kinds; \'S\={\i}ghra (perihelionic), Manda (Aphelionic), and even. O Surasattama! Every planet has three positions. The name of the Madhyagati position is J\=aradgava, the name of the northern position is Air\=avata; and the name of the southern position is Vai\'sv\=anara. The asterisms A\'svin\={\i} Krittik\=a and Bhara\d{n}\={\i} are known by the term N\=agav\={\i}th\={\i}. Rohi\d{n}\={\i}, \=Ardr\=a, and Mriga\'sir\=a are named Gaja V\={\i}th\={\i}; Pu\d{s}y\=a, A\'sle\d{s}\=a, and Punarvasu are named Airavat\={\i}v\={\i}th\={\i}. The three V\={\i}th\={\i}s, above-mentioned are called Uttara M\=arga. Purvaphalgun\={\i}, Uttara Phalgun\={\i} and Magh\=a are named A'r\d{s}abh\={\i} V\={\i}th\={\i}. Hast\=a, Chitr\=a and Sv\=at\={\i} are called Gov\={\i}th\={\i}; Jyesth\=a, Vi\'s\=akh\=a and Anur\=adh\=a are named J\=aradgav\={\i} V\={\i}th\={\i}. These three V\={\i}th\={\i}s are named Madhyam\=a M\=arga. M\=ul\=a, Purb\=a\d{s}\=adh\=a, Uttar\=a\d{s}\=adh\=a are termed Ajav\={\i}th\={\i} \'Srava\d{n}\=a, Dhani\d{s}th\=a and \'Satabhi\d{s}\=a are termed Mriga V\={\i}th\={\i}. Uttara bh\=adrapada, Purvabh\=adrapada, and Revat\={\i} are called Vai\'sv\=anar\={\i}v\={\i}th\={\i}. These three V\={\i}th\={\i}s (paths) are called Dak\d{s}i\d{n}am\=arga. During the Uttar\=aya\d{n}a time, as the Dhruva attracts the rope of air from both the sides of the Yuga, orbit (or axis), the chariot of the Sun ascends, (i.e., is drawn up by the rope). Thus when the Sun enters within the sphere, the motion of the chariot becomes slower and the day is lengthened and the night is shortened. O Sura Sattama! Know this to be the course of the path of the Sun.

When the cord draws towards the south, the Chariot descends and as the Sun then comes out of the sphere, the motion becomes quick. The day shortens and the night is lengthened. Again when the cord is neither tightened nor is it slackened, rather its motion is exactly mid-way, the Sun also remains in a medium position and his Chariot enters within a sphere of equilibrium and the day and night become equal. When the cord of air, in a state of equilibrium is attracted by the Polar Star, then it is that the Sun and the Solar system revolves; and when the Polar Star slackens its attraction over the cord of air, the Sun coming out of the middle sphere, revolves; and the Solar system also revolves. On the east of Meru is established the city of Indra and the Devas dwell there. It is called therefore Devadh\=anik\=a. On the south

of the Meru, is the famous city of Yama, the God of Death, named Samyaman\={\i}. On the west of Meru, is the great city of Varu\d{n}a, named Nimnochan\={\i}. On the north of Meru is the city of the Moon, named Vibh\=avar\={\i}. O N\=arada! The Brahmav\=ad\={\i}s say that the Sun first rises in the city of Indra. At noon the Sun goes to Samyaman\={\i}; at evening the Sun goes to Nimnochan\={\i} and He is said to set. In the night the Sun remains in Vibh\=avar\={\i}. O Muni! The going of the Sun round Meru is the cause of all the beings getting themselves engaged in their respective duties. The inhabitants of the Meru see the Sun always in the central position. The Sun moves on, eastwards towards the stars, keeping the Meru to his left; but if the Zodiac be taken into account, it would appear that the Meru is left towards the south of the Sun. The rising and the setting of the Sun are always considered in front of Him. O Devar\d{s}i! Every point, every quarter, every person, seeing the Sun says that the Sun has risen there; again where he becomes invisible, He is considered to set there. The Sun always exists; so there is no rising nor setting for Him. It is His appearance and disappearance that make men say that the Sun rises or sets. When the Sun is in the Indra's city, He illumines the three cities, those of Indra, Yama, and the Moon and illumines the north-east and east-west corners. So when He rests in the city of Fire, he illumines north-east, east-west, and south-west, the three corners, and at the same time the cities of Indra and Yama; and so on for the other cities and corners. O N\=arada! The Mont Meru is situated towards the north of all the Dv\={\i}pas and Var\d{s}as. So whenever any person sees the Sun rise he calls that side ``east.'' But Meru exists towards the left of the Sun; so it is said. If the Sun travels in 15 (fifteen) Ghatik\=as, the distance from Indrapur\={\i} to Yamapur\={\i}, He is said to travel within that time a distance equal to 2\sfrac{1}{4}; Kotis, 12\sfrac{1}{2}; lakhs and 25000 Yojanas (22695000 Yojanas). The thousand-eyed and thousand rayed Sun God is the Manifester of Time. He travels in the aforesaid way the cities of Varu\d{n}a, Chandra and Indra respectively. He is the diadem of the Svarloka; and the Zodiac is his \=Atman. He travels thus, to mark off time to all persons. O N\=arada! The Moon and the other planets and stars rise and set in the aforesaid manner. Thus the powerful chariot of the Sun travels in a Muh\=urta 142,00000 Yojanas. By the force of Pravaha V\=ayu (air), the Sun God, the Incarnate of the Vedas travels round the cities, the Zodiac, in one Samvatsara (year). The wheel of the Sun's Chariot is one year; twelve months are the spokes; three Ch\=aturm\=asyas are the nave and the six seasons are the outer ring or circumference of the wheel. The learned men call this chariot as the

Samvatsara (one year). The axis or axle points to the Meru on one side and to M\=anasottara mountain on the other. The end or circumference of the wheel marks off other divisions of the time as Kal\=a, K\=asth\=a, Muh\=urta, Y\=ama, Parahara, day and night, and fortnights. The wheel is fixed on the nave. The Sun goes on this wheel, like an oilman's on his oil-machine, round and round the M\=anasottara mountain. The eastern side of the wheel is on that axis and the other part is fixed on the Pole Star. The dimension of the first axis is (15750000 Yoyanas). The second axis measures one-fourth of the above (3937500 Yoyanas). It resembles the axis of an oil-machine. The upper side of that is considered to belong to the Sun. The seat of the Sun on his chariot measures 36 Lakh Yoyanas wide. The Yuga measures in length one fourth of the above dimensions, that of his seat. The Chariot is is moved by seven horses, consisting of the seven Chhandas, G\=ayatr\={\i}, etc., driven by Aru\d{n}a. The horses carry the Sun for the happiness of all. Though the charioteer sits in front of the Sun, his face is turned towards the west. He does his work as a charioteer in that state. Sixty thousand V\=alakhilya \d{R}i\d{s}is, of the size of a thumb, chant the sweet Vedic hymns before Him. Other \d{R}i\d{s}is, Apsar\=as, Uragas, Gr\=ama\d{n}\={\i}s, R\=ak\d{s}asas, and all the Devas, each divided in groups of seven, worship every month that highly lustrous Sun-god. The earth measures 90152000 Kro\'sa Yuga Yoyanas (1 Kro\d{s}a - \sfrac{1}{4} Yoyana). The Sun passes over this distance in a moment. He does not take rest in his this work even for a day; no, not even for a moment.

Here ends the Fifteenth Chapter of the Eighth Book on the motion of the Sun in the Mah\=apur\=a\d{n}am, \'Sr\={\i} Mad Dev\={\i} Bh\=agavatam, of 18000 verses, by Mahar\d{s}i Veda Vy\=asa.



