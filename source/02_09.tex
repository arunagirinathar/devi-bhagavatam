\chapter{On the account of Ruru}

1-17. Par\={\i}ksit said :-- When the Muni Ruru went to his room to sleep, his mind having become perturbed with passion, his father Pramati seeing him sorrowful, asked him :-- ``O Ruru! Why do you look so

absent minded?'' Ruru was passionate then; so he said to his father :-- ``I saw a girl named Pramadvar\=a in the hermitage of Sth\=ulake\'sa; I wish that she might become my wife.'' Hearing this, Pramati went immediately to the hermitage of Sth\=ulake\'sa, and pleased him by various conversations and asked for her beautiful daughter when Sth\=ulake\'sa promised that he would give her daughter in marriage on an auspicious day. Then both the high-souled persons Pramati and Sth\=ulake\'sa began to work in co-operation and make arrangements for marriage ceremony and collected various articles in that hermitage when the fair eyed girl Pramadvar\=a, while playing in the courtyard in the house, trod on a serpent and was bitten by it and consequently died. Seeing then Pramadvar\=a dead, all the Munis of the place assembled and cried and wept with sorrowful hearts, when a great tumultuous uproar ensued. Though the life departed from Pramadvar\=a's body, yet seeing the brilliant lustre of her lifeless body lying on the ground, her nourisher and father Sth\=ulake\'sa became very sorry and wept aloud. Hearing this cry of his, Ruru came there to see what had happened and perceived the girl, though lifeless, yet seeming alive and lying on the ground.

Seeing Sth\=ulake\'sa and other Risis weeping, Ruru went out from that place and with a grievous heart, began to cry aloud. ``Alas! Fate has certainly sent this serpent as the cause of all my miseries and to mar all my happiness. Alas! What am I to do now? Where to go? When my beloved has fallen unto the jaws of death, I do not want to live any longer, bereft of my wife. Oh! What an unfortunate creature I am? I have not been able to embrace this beautiful darling of mine. I am deprived of kissing her face and marrying her. Alas! Fie to my human birth! Let my life get out just now in as much as I could not, out of mere shame, throw myself on the burning pyre along with my beloved! Oh! When death comes not to the sorrowful person, even when prayed for, how then can I expect divine happiness in this world? So let me now drop myself down in a lake or enter in to a burning fire or drink venom or strangle myself by tieing rope round my neck!''

18-31. Thus Ruru wailed much on the bank of the river and long reflecting in his mind found out a way and thought what would be the advantage in death? ``Rather an irretrievable sin would be incurred in committing suicide; and my father and mother would be sorry. Seeing me commit suicide, my bad luck and enemies will be gladdened; there is no manner of doubt; in this. What benefit will my beloved gain if I commit suicide or if I be distressed for her bereavement. Suppose

I die, even then my beloved will not become mine in the next world; so there are many faults in my committing suicide but there is no fault if I preserve my life.'' Thus coming to a conclusion Ruru bathed, performed \=Achaman and became pure. He then took water in his hand and said :-- ``Whatever good works, worshipping the gods, etc., that I have done and if I have performed, with devotion, the service to my preceptors and teachers and superiors, homa ceremonies, Japam, tapasy\=a, if I have studied all the Vedas and if I have recited G\=ayatri and worshipped the Sun then let my beloved have life and get up as an outcome of my Pu\d{n}yam. If my beloved does not get back her life, I will certainly quit my life.'' Thus saying, he worshipped the Devas mentally and threw that water of his hands on the ground. Thus Ruru, with a sorrowful heart, was weeping. The Deva's messenger came down and said :-- ``O Br\=ahma\d{n}! Don't make this bold attempt; how can your beloved get back her life? The life-period of this beautiful girl, born of Gandharva's sperm and Apsar\=a's ovum is now exhausted; now look for another beautiful woman. O one of very dull understanding! Why are you crying in vain? Where is the affection between you and this girl; she died in an unmarried state (without marrying you).'' At this Ruru said :-- ``O Deva messenger! I won't marry any other lady, whether my beloved gets back her life or does not get back her life; in case she does not regain this life, I will also forego my life at this instant.'' At this greatest importunity of Ruru, the Deva messenger became glad and spoke the following truthful beneficent yet beautiful words :--

32-51. ``O Br\=ahma\d{n}a! I will suggest one way to you; kindly hear. The Devas ordained this long, long ago. You can give up your half life period, and with that you can make this girl alive soon.''

Ruru said :-- ``O Deva messenger! I give half my life-period to this girl; there is no doubt in this. Let my beloved get back her life soon and get up.''

The king said :-- O Ministers! At this time Visv\=avasu, knowing that his daughter Pramadvar\=a is dead, descended from the Heavens in a celestial car and came to the place; then the Gandharva king and the Deva messenger both went to Yama, the Dharmar\=aj, and spoke thus :-- ``O Dharmar\=aj! This Visv\=avasu's daughter Pramadvar\=a, the wife of Ruru, the \d{R}i\d{s}i\'s son was bitten by a snake and has now come to your place. The Dvija Ruru is now desirous to quit his life; so, O Sun's son! Now let the girl again get her life through the influence of Ruru's brahmacharya (purity) as a consequence of his giving away half his life period for the girl.''

Dharma said :-- ``O Deva messenger! if you want to make the girl alive again, let her get life as a consequence of half the life-period of Ruru being subtracted. Go immediately and give the girl to Ruru.''

The king said :-- ``O Ministers! Yama having said thus to the Deva messenger, he went away immediately and made Pramadvar\=a alive and handed her over to Ruru.

Thus, on an auspicious day, Ruru married her. Thus the \d{R}i\d{s}i\'s daughter Pramadvar\=a though fallen dead, got again her life by proper means. So, O Councillors! to save life, one should resort one's best duty according to the \'S\=astras, by the use of gems, mantras, and herbs and plants.''

Thus speaking to the ministers, the king Par\={\i}ksit had a fine building of seven floors in height erected, placed the principal guards around it and stationed also the most powerful men well versed in the knowledge of mani (gems), mantrams, and plants for protection and immediately ascended to this building. To appease the wrath of the Muni \'Sring\={\i}, the king sent the Muni named Gaurmukha to him and requested him repeatedly ``Let the crime of the humble devotee be forgiven.'' Then, for self preservation, the king brought from all sides the Br\=ahma\d{n}as, who are perfect in their knowledge and application of the mantras. The minister's son placed the elephants in proper places so that no body can ascend to the top of the building; what more can be said than the fact that even air could not find entrance there when once ordered ``no admission'' what to speak of others! The king Par\={\i}ksit remained there and counted the number of days of the serpent Tak\d{s}aka's coming there; he performed his bath, Sandhy\=a Bandanams and fooding; even he consulted with his ministers and governed his kingdom from there. O \d{R}i\d{s}is! At this time a Br\=ahma\d{n} named Ka\'syapa, versed in the mantras, heard of the curse of the king and thought that he would get abundant wealth if he could free the king from Tak\d{s}aka's poison and proposed to himself that he would go to the place where the cursed king Par\={\i}ksit was staying with the Br\=ahma\d{n}as. Pondering thus, the Br\=ahma\d{n}a went out of his house, on the expectation of wealth from the king.

Thus ends the ninth chapter of the second Skandha on the account of Ruru in the Mah\=apur\=a\d{n}am \'Sr\={\i} Mad Dev\={\i} Bh\=agavatam of 18,000 verses.



