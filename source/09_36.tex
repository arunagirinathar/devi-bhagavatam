\chapter{On the destruction of the fear of the Yama of those who are the worshippers of the Five Devat\=as}

1-7. S\=avitr\={\i} said :-- ``O Dharmar\=ajan! O Highly Fortunate One! O Thou! Expert in the Vedas and the Amgas thereof! Now kindly describe that which is the essence of the various Pur\=a\d{n}as and Itih\=asas, which is the quintessence, which is dear to all, approved of by all, which is the seed by which the Karmic ties are cut asunder, which is high, noble and happy is this life. Kindly describe the above by which man can acquire all his desires, and what is the only source of all the good and auspicious things. All by knowing which man has

not to face any dangers or troubles, nor has he to go to the dreadful hells that thou hast severally just now described and that by which men can be freed of those various wombs. Kindly now describe all these. O Bhagavan! What is the size of the several kundas or hells that Thou hast just now enumerated? How do the sinners dwell there? When a man departs, his body is reduced to ashes. Then of what sort is that other body by which the sinners enjoy the effects of their Karmas and why do not those bodies get destroyed when they suffer so much pains for so long a time? What sort of body is that? Kindly describe all these to me.''

8-33. N\=ar\=aya\d{n}a spoke :-- Hearing the questions put forward by S\=avitr\={\i}, Dharmar\=aja remembered \'Sr\={\i} Hari and began to speak on subject that sever the bonds of Karma :-- O Child! O One of good vows! In the four Vedas, in all the books on Dharma, (Smritis) in all the Samhit\=as, all the Itih\=asas, all the Pur\=a\d{n}as, in the N\=arada Pa\~nchar\=atram, in the other Dharma \'S\=astras and in the Ved\=angas, it is definitely stated that the worship of the Pa\~ncha Devat\=as (the five Devat\=as) \'Siva, \'Sakti Vi\d{s}\d{n}u, Gane\'sa, and S\=urya is the best, the highest, the destroyer of the old age, disease, death, evils and sorrows, the most auspicious and leading to the highest bliss. In fact, the worship of these Pa\~ncha Devat\=as is the source of acquiring all the Siddhis (the success) and saves one from going to the hells. From their worship springs the Bhaktic Tree and then and then only the Root of the Tree of all Karmic bonds is severed for ever and ever. This is the step to Mukti (final liberation) and is the indestructible state. By this one can get S\=alokya, S\=arsti, S\=ar\=upya, and S\=am\={\i}pya, the different state of beatitudes in which the soul (1) resides in the same world with the Deity, (2) possesses the same station, condition, or rank, or equality with the Supreme Being in power and all the Divine attributes (the last of the four grades of Mukti), (3) possesses the sameness of form or gets assimilated to the Deity or (4) gets intimately united, identified or absorbed into the Deity. O Auspicious One! The worshipper of these five Devat\=as has never to see any of the hells, watched by My messengers. Those who are devoid of the devotion to the Dev\={\i} see My abode; but those who go to the T\={\i}rthas of Hari, who hold Hariv\=asaras (festivities on the days of Hari) who bow down at the feet of Hari and worship Hari, never come to My abode named Samyamana. Those Br\=ahma\d{n}as that are purified by their performing the three Sandhy\=as and by the following the pure \=Ach\=aras (customs and observances), those that find no pleasure until they worship the Dev\={\i}, those that are attached to their own Dharmas and their own \=Ach\=aras, never come to My abode.

My terrible messengers, seeing the devotees of \'Siva, run away out of terror as snakes run away terrified by Garuda. I also order My messengers with nooses in their hands never to go to them. My messengers go mostly to other persons than the servants of Hari. No sooner do My Messengers see the worshippers of the Kri\d{s}\d{n}a Mantra, than they run away as snakes get terrified at the sight of Garuda. Chitragupta, too, one of the beings in Yama's world, recording the vices and virtues of mankind, strike off the names of the Dev\={\i} worshippers, out of fear and prepare Madhuparka, etc., for them (a mixture of honey; respectful offering made to a guest or to the bridegroom on his arrival at the door of the father of the bride). They rise higher than the Brahm\=a Lokas and go to the Dev\={\i}'s abode, i.e., to Ma\d{n}idv\={\i}pa. Those that are the worshippers of the \'Sakti Mantra and are highly fortunate, whose contact removes the sins of others, they deliver the thousand generations (from the downward course). As bundles and bundles of dry grasses become burnt to ashes, no sooner they are thrown into fire, so the delusion at once becomes itself deluded at the sight of the forms of those devotees. At their sight, lust, anger, greed, disease, sorrow, old age, death, fear, K\=ala (time that takes away the life of persons), the good and bad karmas, pleasures and enjoyments drop off to a great distance. O Fair One! Now I have described to you the states of those persons that are not under the control of K\=ala, good and bad karmas, pleasures and enjoyments, etc., and those that do not suffer those pains. Now I am speaking of this visible body. Listen. Earth, water, fire, air, and ether are the five Mah\=a Bh\=utas (the great elements); these are the seeds of this visible body of the person and are the chief factors in the work of creation. The body that is made up of earth and other elements is transient and artificial, i.e., that body becomes burnt to ashes. Within this visible body, bound, is there a Puru\d{s}a of the size of a thumb; that is called the J\={\i}va Puru\d{s}a; the subtle J\={\i}va assumes those subtle bodies for enjoying the effects of karmas. In My world, that subtle body is not burnt by the burning fire. If that subtle body be immersed in water, if that be beaten incessantly or if it be struck by a weapon or pierced by a sharp thorn, that body is not destroyed. That body is not burnt nor broken by the burning hot and molten material, by the red hot iron, by hot stones by embracing a hot image or by falling into a burning cauldron. That body has to suffer incessant pains. O Fair One! Thus I have dwelt on the subject of the several bodies and the causes thereof according to the \'S\=astras. Now I will describe to you the characters of all the other Kundas. Listen.

Here ends the Thirty-sixth Chapter of the Ninth Book on the destruction of the fear of the Yama of those who are the worshippers of the Five Devat\=as, in the Mah\=a Pur\=a\d{n}am \'Sr\={\i} Mad Dev\={\i} Bh\=agavatam of 18,000 verses by Mahar\d{s}i Veda Vy\=asa.



