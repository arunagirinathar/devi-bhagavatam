\chapter{On the coming in this world of Lak\d{s}m\={\i}, Gang\=a and Sarasvat\={\i}}

1-10. N\=ar\=aya\d{n}a said :-- ``O N\=arada! Sarasvat\={\i} lives always in Vaikuntha close to N\=arada. One day a quarrel arose with Gang\=a, and by Her curse, Sarasvat\={\i} came in parts as a river here in this Bh\=arata. She is reckoned in Bh\=arata as a great sanctifiying holy and merit-giving river. The good persons serve Her always, residing on Her banks. She is the Tapasy\=a and the fruit thereof of the ascetics. She is like the burning fire to the sins of the sinners. Those that die in Bh\=arata on the Sarasvat\={\i} waters with their full consciousness, live for ever in Vaikuntha in the council of Hari. Those that bathe in the Sarasvat\={\i} waters, after committing sins, become easily freed of them and live for a long, long time in Vi\d{s}\d{n}u-Loka. If one bathes even once in the Sarasvat\={\i} waters, during Ch\=aturm\=asya (a vow that lasts four months), in full moon time, in Ak\d{s}yay\=a or when the day ends, in Vyat\={\i}p\=ata Yoga, in the time of eclipse or on any other holy day or through any other concomitant cause or even without any faith and out of sheer disregard, one is able to go to Vaikuntha and get the nature of \'Sr\={\i} Hari. If one repeats the Sarasvat\={\i} Mantra, residing on the banks of the Sarasvat\={\i}, for one month, a great illiterate can become a great poet. There is no doubt in this. Once shaving one's head, if one resides on the banks of the Sarasvat\={\i}, daily bathes in it, one will have not to meet with the pain of being again born in the womb. O N\=arada! Thus I have described a little of the unbounded glories of Bh\=arata that give happiness and the fruits of all desires.''

11. S\=uta said :-- ``O Saunaka! The Muni N\=arada hearing thus, asked again at that very moment to solve his doubts. I am now speaking of that. Hear.''

12-15. N\=arada said :-- ``O Lord! How did the Dev\={\i} Sarasvat\={\i} quarrel with the Dev\={\i} Gang\=a and how did she by Her curse turn out in India, into a holy river in giving virtues. I am becoming more and more eager and impatient to hear about this critical incident. I do not find satiety in drinking your nectar-like words. Who finds satiety in getting his good weal? Why did Gang\=a curse Sarasvat\={\i}, worshipped everywhere. Gang\=a is also full of Sattva Gu\d{n}as. She always bestows good and virtue to all. Both of them are fiery and it is pleasant to hear the cause of quarrels between those two. These are very rarely found in the Pur\=a\d{n}as. So you ought to describe that to me.''

16-21. N\=ar\=aya\d{n}a said :-- Hear, O N\=arada! I will now describe that incident, the hearing of which removes all the sins. Lak\d{s}m\={\i}, Sarasvat\={\i} and Gang\=a, the three wives of Hari and all equally loved, remain always close to Hari. One day Gang\=a cast side-long glances frequently towards N\=ar\=aya\d{n}a and was eagerly looking at Him, with smile on Her lips. Seeing this, the Lord N\=ar\=ayana, startled and looked at Gang\=a and smiled also. Lak\d{s}m\={\i} saw that, but she did not take any offence. But Sarasvat\={\i} became very angry. Padm\=a (Lak\d{s}m\={\i}) who was of Sattva Gu\d{n}a, began to console in various ways the wrathful Sarasvat\={\i}; but she could not be appeased by any means. Rather Her face became red out of anger; she began to tremble out of her feelings (passion); Her lips quivered; and She began to speak to Her husband.

22-38. The husband that is good, religious, and well qualified looks on his all the wives equally; but it is just the opposite with him who is a cheat. O Gad\=adhara! You are partial to Gang\=a; and so is the case with Lak\d{s}m\={\i}. I am the only one that is deprived of your love. It is, therefore, that Gang\=a and Padm\=a are in love with each other; for you love Padm\=a. So why shall not Padm\=a bear this contrary thing! I am only unfortunate. What use is there in holding my life? Her life is useless, who is deprived of her husband's love. Those that declare you, of Sattva Gu\d{n}as, ought not to be ever called Pundits. They are quite illiterate; they have not the least knowledge of the Vedas. They are quite impotent to understand the nature of your mind. O N\=arada! Hearing Sarasvat\={\i}'s words and knowing that she had become very angry, N\=ar\=aya\d{n}a thought for a moment and then went away from the Zenana outside. When N\=ar\=aya\d{n}a had thus gone away, Sarasvat\={\i} became fearless and began to abuse Gang\=a downright out of anger in an abusive language, hard to hear :-- ``O Shameless One! O Passionate One! What

pride do you feel for your husband? Do you like to show that your husband loves you much? I will destroy your pride today. I will see today, it will be seen by others also, what your Hari can do for you?'' Saying thus Sarasvat\={\i} rose up to catch hold of Gang\=a by Her hairs violently. Padm\=a intervened to stop this. Sarasvat\={\i} became very violent and cursed Lak\d{s}m\={\i} :-- ``No doubt you will be turned into a tree and into a river. In as much as seeing this undue behaviour of Gang\=a, you do not step forward to speak anything in this assembly, as if you are a tree or a river.'' Padm\=a did not become at all angry, even when she heard of the above curse. She became sorry and, holding the hands of Sarasvat\={\i}, remained silent. Then Gang\=a became very angry; Her lips began to quiver frequently. Seeing the mad fiery nature of the red-eyed Sarasvat\={\i}, she told Laksm\={\i} :-- ``O Padme! Leave that wicked foul-mouthed woman. What will she do to me? She presides over speech and therefore likes always to remain with quarrels. Let Her shew Her force how far can she quarrel with me. She wants to test the strength of us. So leave Her. Let all know today our strength and prowess.''

39-44. Thus saying, Gang\=a became ready to curse Sarasvat\={\i} and addressing Lak\d{s}m\={\i}, said :-- ``O Dear Padme! As that woman has cursed you to become a river, so I too curse her, that she, too, be turned into a river and she would go to the abode of men, the sinners, to the world and take their heaps of sins.'' Hearing this curse of Gang\=a, Sarasvat\={\i} gave her curse, ``You, too, will have to descend into the Bhurloka (the world) as a river, taking all the sins of the sinners.'' O N\=arada! While there was going on this quarrel, the four-armed omniscient Bhagav\=an Hari came up there accompanied by four attendants of His, all four-armed, and took Sarasvat\={\i} in His breast and began to speak all the previous mysteries. Then they came to know the cause of their quarrels and why they cursed one another and all became very sorry. At that time Bhagav\=an Hari told them one by one :--

45-67. O Lak\d{s}m\={\i}! Let you be born in parts, without being born in any womb, in the world as the daughter in the house of the King Dharma-dhvaja. You will have to take the form of a tree there, out of this evil turn of fate. There \'Sankhach\=uda, the Indra of the Asuras, born of my parts will marry you. After that you will come back here and be my wife as now. There is no doubt in this. You will be named Tulas\={\i}, the purifier of the three worlds, in Bh\=arata. O Beautiful One! Now go there quickly and be a river in your parts under the name Padm\=avat\={\i}. O Gange! You will also have to take incarnation in Bh\=arata as a river, purifying all the worlds, to destroy the sins of

the inhabitants of Bh\=arata. Bhagiratha will take you there after much entreating and worshipping you; and you will be famous by the name Bhagirath\={\i}, the most sanctifying river in the world. There, the Ocean born of my parts, and the King \'S\=antanu also born of my parts will be your husbands. O Bharat\={\i}! Let you go also and incarnate in part in Bh\=arata under the curse of Gang\=a. O Good-natured One! Now go in full Amsas to Brahm\=a and become His wife. Let Gang\=a go also in Her fullness to \'Siva. Let Padm\=a remain with Me. Padm\=a is of a peaceful nature, void of anger, devoted to Me and of a S\=attvika nature. Chaste, good-natured, fortunate, and religious woman like Padm\=a are very rare. Those women that are born of the parts of Padm\=a are all very religious and devoted to their husbands. They are peaceful and good-natured and worshipped in every universe. It is forbidden, nay, opposed to the Vedas, to keep three wives, three servants, three friends of different natures, at one place. They never conduce to any welfare. They are the fruitful sources of all jealousies and quarrels. Where, in any family females are powerful like men and males are submissive to females, the birth of the male is useless. At his every step, he meets with difficulties and bitter experiences. He ought to retire to the forest whose wife is foul-mouthed, of bad birth and fond of quarrels. The great forest is better for him than his house. That man does not get in his house any water for washing his feet, or any seat to sit on, or any fruit to eat, nothing whatsoever; but in the forest, all these are not unavailable. Rather to dwell amidst rapacious animals or to enter into fire than remain with a bad wife. O Fair One! Rather the pains of the disease or venom are bearable, but the words of a bad wife are hard to bear. Death is far better than that. Those that are under the control of their wives, know that they never get their peace of mind until they are laid on their funeral pyres. They never see the fruits of what they daily do. They have no fame anywhere, neither in this world nor in the next. Ultimately the fruit is this:-- that they have to go to hell and remain there. His life is verily a heavy burden who is without any name or fame. Never it is for the least good that many co-wives remain at one place. When, by taking one wife only a man does not become happy, then imagine, how painful it becomes to have many wives. O Gange! Go to \'Siva. O Sarasvat\={\i}! Go to Brahm\=a. Let the good-natured Kamal\=a, residing on the lotus remain with Me. He gets in this world happiness and Dharma and in the next Mukti whose wife is chaste and obedient. In fact he is Mukta, pure and happy whose wife is chaste; and he whose wife is foul-natured, is rendered impure unhappy and dead whilst he is living.

Here ends the Sixth Chapter of the Ninth Book on the coming in this world of Lak\d{s}m\={\i}, Gang\=a and Sarasvat\={\i} in the Mah\=apur\=a\d{n}am \'Sr\={\i} Mad Dev\={\i} Bh\=agavatam of 18,000 verses by Mahar\d{s}i Veda Vy\=asa.



