\chapter{On the Tapta Krichchhra vrata and others}

1-20. N\=ar\=aya\d{n}a said :-- The best S\=adhaka, then uttering after his meals, the mantra ``Amrit\=apidh\=anamasi.'' (O Water-nectar! Let Thou be the covering to the food that I have taken), should make \=Achaman (sip one Gandu\d{s}a water) and distribute the remnant food (the leavings) to those who take the leavings. ``Let the servants and maid-servants of our family that expect the leavings of food be satisfied with what leavings I give to them.'' ``Let those inhabitants of the Raurava hell or other unholy places who have remained there for a Padma or Arbuda years and want to drink water, be satisfied with this water that I offer to them and let this water bring unending happiness to them.'' Repeating the above two mantras let the house-holder distribute the leavings of food to the servants and the water to those who want water respectively. Then opening the knot of the Pavitra (a ring of Ku\'sa grass worn on the fourth finger on certain religious occasions), let him throw this on the square mandalam or the ground. The Br\=ahma\d{n}a that throws this Ku\'sa grass on the vessel (P\=atra) is said to defile the vow of Br\=ahma\d{n}as, taking their food. The Br\=ahma\d{n}a that has not yet washed his face after taking the food, or touching another such Br\=ahma\d{n}a or a dog, or a \'S\=udra, should fast one day and then drink Pa\~nchagabya and thus purify himself. And in case the Uchchista Br\=ahmi\d{n} (who has not washed his mouth and hands after meals) be touched by another Br\=ahmi\d{n}, who is not Uchchista, then simply bathing will purify him. By offering this Ek\=ahuti (oblation once) according to rules mentioned above, one obtains the fruit of performing ten million sacrifices; and by offering this oblation five times one gets the

endless fruit, of performing fifty million sacrifices, and if one feeds such a man who knows well how to do this Pr\=an\=agnihoma, then he as well as he whom he feeds both derive full benefits and they ultimately go to heaven. The Br\=ahma\d{n}a acquires while taking each of his mouthful of food the fruit of eating Pa\~nchagavya, who takes his food duly with the holy Pavitra Ku\'sa grass tied on his finger. During the three times of worship, the devotee is to do his daily Japam, Tarpa\d{n}am and Homa and he should feed the Br\=ahmi\d{n}s. Thus the five limbed Pura\'schara\d{n}a is completely done. The religious man should sleep on a low bedding (lie on the ground); he is to control his senses and anger; he is to eat moderately, the things that are light, sweet and good; he is to be humble, peaceful and calm. He is to bathe thrice daily and not to hold any unholy conversation with any woman, a \'S\=udra, one who is fallen, without any initiation, and who is an atheist; as well he should not speak in a language spoken by the ch\=and\=alas. One is to bow down before him who is in the act of performing the Japam, Homa and worship, etc; one is not to talk with him. Never by deed, mind or word, on all occasions never speak about sexual intercourses; nor hold any contact with such people. For the relinquishment of this subject is called Brahmacharyam (continence) of the kings as well as of the house-holders. But one should go to one's legal wife during the night time after her menstruation duly according to the rules of the \'S\=astras; the Brahmacharyam is not thereby destroyed. Man cannot repay the three-fold debts and he cannot aspire for mok\d{s}a without procreating sons or without doing the duties of the house-holders, as prescribed by the \'S\=astras. An attempt to do so becomes entirely fruitless like the breast on the neck of a goat. Rather it drags one downward. So the \'Srutis say. So let yourself be free first from the debts due to the Devas, the debts due to the \d{R}i\d{s}is and the debts due to the Pitris. Make sacrifices first and then be free from the Deva\'s debt. Hold Brahmacharyam and be free from the \d{R}i\d{s}i\'s debt. Offer til and water; that is, do \'Sr\=addhas and tarpa\d{n}ams and be free from the debt due to the Pitris. Then do readily practise your own Varna\'srama Dharma.

21-33. One is to practise Krichchra ch\=andr\=a yana Vrata and to take for his food, milk, fruits, roots and vegetables, Habi\d{s}yannam and food obtained by begging so that one may become sinless. One is to make japam for Pura\'schara\d{n}am. One is to avoid salt, salty or alkaline substances, acid, garlic, turnips, eating in K\=amsa vessels, chewing betels, eating twice, putting on impure clothings, the intoxicating things and the uns\=astric nocturnal japam; also one is not to waste one's time over blaming and

trying to find faults with the relatives, playing at dice, or talking at random with one's wife (so that evil effects may arise). One is to spend one's time in worshipping the Devas, reciting the hymns of praise, and studying the \'S\=astras. One is to sleep on the ground, practise Brahmacharyam, and the vow of silence, bathe thrice, not practise anything which befits the \'S\=udras only. One is to worship everyday, make charities duly and be always happy, recite stotras daily, do occasional Deva worships, have faith in one's Guru and Deva. These twelve rules are to ensure success to the devotee who does Pura\'schara\d{n}am. One is to daily praise the Sun, with one's face turned towards Him, do japam before Him; or one is to worship one's own Deity in front of fire or the image of any god, and do japam simultaneously. The devotee who practises Pura\'schara\d{n}am is to bathe, worship, do japam, meditate, practise Homa, Tarpa\d{n}am, is to have no desires and to surrender all fruits to one's own desired Deity, etc. These are necessarily to be observed by him. Therefore while doing japam, Homa, etc., the devotee's mind is to remain always pleasant and satisfied. One should be ready to practise tapasy\=a, to see the \'S\=astras and be merciful to all the beings. As asceticism leads one to to heaven and to the attainment of one's desires, therefore know this that all the powers come to an ascetic. An ascetic can cause another's death (m\=aran); he can injure others, cure diseases and kill all. Whatever the several \d{R}i\d{s}is wanted from the Dev\={\i} G\=ayatr\={\i} and to that end made Pura\'schara\d{n}am and worshipped Her, they obtained from Her all those things. O N\=arada! I will speak of \'S\=anti Karmas etc., in a future chapter. Here I will speak of those rules, etc., that are to be observed in Pura\'schara\d{n}am in as much as they play the principal part to success.

First of all shave yourself and have your hairs and nails, etc., cut off and bathe and be pure. Then perform the Pr\=aj\=apatya pr\=aya\'schitta for one's peace and purification and next do the pura\'schara\d{n}am of the G\=ayatr\={\i}. Do not speak the whole day and night. Keep your thoughts pure. If words are to be spoken, speak only what you take as true. First recite Mah\=avy\=arhiti and then the S\=avitr\={\i} mantra with Pra\d{n}ava prefixed. Then recite the sin-destroying mantra ``\=Apohisth\=a, etc.,'' and Svasti mat\={\i} S\=ukta and ``P\=avam\=an\={\i} S\=ukta.'' In every action, in its beginning and at its end one is to understand the necessity of doing the Japam, why and what for one is doing that.

One is to repeat the Pra\d{n}ava, the three Vy\=arhitis and S\=avitr\={\i} ayuta times or one thousand times or one hundred times or ten times. Then offer with water, the peace offerings (tarpa\d{n}am) to the \=Ach\=arya, \d{R}i\d{s}i, Chhandas, and the Devas. Being engaged in action, do not speak any impure language

of the Mlechchhas or talk with any \'S\=udra or any bad person. Do not talk with wife in the period of menstruation, with one who has fallen, with the low-class person, with any hater of the Devas and the Br\=ahma\d{n}as, \=Ach\=aryas and Gurus, with those who blame the fathers and mothers; nor show any disrespect to anybody. Thus I have spoken in due order about all the rules of Krichchhra vrata. Now I will speak of the rules of the Pr\=aj\=apatya Krichchhra, \'S\=antapana, Par\=aka Krichchhra and Ch\=andr\=aya\d{n}a.

34-54. One becomes freed of all the sins, if one performs the above five Ch\=andr\=aya\d{n}as. By the performance of the Tapta Krichchhra, all sins are burnt off in an instant. By the performance of the three Ch\=andr\=aya\d{n}as the people get purified and go to the Brahma Loka. By doing eight Ch\=andr\=aya\d{n}as, one sees face to face one's Devat\=a, ready to grant boons. With ten Ch\=andr\=aya\d{n}as, one gets the knowledge of the Vedas and one acquires all what one wants.

In the observance of the Krichchhra Pr\=aj\=apatya Vrata, one has to take food once in midday for three days, once in the evening for three days, and for the next three days whatever one gets without asking anything from anybody. For the next three days one is not to take any thing at all and go on with one's work. These twelve day\'s work constitutes the Pr\=aj\=apatya Vrata.

Now about the rules of the \'S\=antapana Vrata. On the preceding day one has got to eat food consisting of the mixture of cow urine, cow-dung milk, curd, ghee and the water of the Ku\'sa grass; the day following he is to fast. These two day\'s work constitutes the \'S\=antapana Vrata.

Now about the Ati Krichchhra vrata. For the first three days, one is to eat one mouthful of food a day and for the next three days one is to fast. This is the Ati Krichchhra vrata. This vrata repeated three times is called Mah\=a \'S\=antapana vrata. Note :-- According to the opinion of Yama, the fifteen day\'s work constitutes Mah\=a \'S\=antapana. For the three days one has to eat cow-urine; for the next three days, cow-dung, for the next three days, curd; for the next three days milk; and for the next three days one has to take ghee. Then one becomes pure. This is called the all sin-destroying Mah\=a \'S\=antapan Vrata. Now I am speaking of the nature of the Tapta Krichchhra Vrata.

The Tapta Krichchhra vrata is carried out for the twelve days. For the first three days, one has to drink hot water; for the next three days, hot milk; for the next three days, the hot ghee and for the next three days, air only. Everyday one has to bathe once only under the above rules,

and remain self-controlled. If one drinks water simply everyday under the above conditions, that is called the Pr\=aj\=apatya vrata.

To remain without any food for twelve days according to rules is called the Par\=aka Krichchhra vrata. By this vrata, all sins are destroyed.

Now about the rules of taking food in the Ch\=andr\=aya\d{n}a vrata. In the dark fortnight one will have to decrease one mouthful of food every day and in the bright fortnight one will have to increase one mouthful every day and one has to fast completely on the Am\=avasy\=a (new moon) day. One has to bathe thrice daily during every Sandhy\=a time. This is known as the Ch\=andr\=aya\d{n}a Vrata.

In the \'Si\'su Ch\=andr\=aya\d{n}a Vrata one will have to take four mouthfuls of food in the midday and four mouthfuls in the evening. In the Yati Ch\=andr\=aya\d{n}a one has to take eight mouthfuls in the midday and to control his passions.

55. These abovementioned vratas are observed by the Rudras, \=Adityas, V\=asus, and Maruts; and they are enjoying thereby their full safety.

Each of the above vratas purifies the seven Dh\=atus of the body in seven nights simply! First skin, then blood, then flesh, bones, sinews, marrows and semen are purified. There is no doubt in this. Thus purifying the \=Atman by the above vratas, one is to do religious actions. The work done by such a purified man is sure to be met with success. First control the senses, be pure and do good actions. Then all your desires will be undoubtedly fructified. Fast for three nights, without doing any actions and see the result. (You will not do anything and you want self control! Is this a child's play?) Perform for three days the nocturnal vratas. Then proceed with your desired duties. If one works according to these methods, one gets the fruits of Pura\'schara\d{n}am. O N\=arada! By the Pura\'schara\d{n}am of \'Sr\={\i} G\=ayatr\={\i} Dev\={\i} all desires are fulfilled and all sins are destroyed. Before doing Pura\'schara\d{n}am purify your body by performing the above vratas. Then you will get all your desires completely fulfilled. O N\=arada! Thus I have spoken to you of the secret rules of Pura\'schara\d{n}am. Never disclose this to any other body. For it is recognised equivalent to the Vedas.

Here ends the Twenty-third Chapter of the Eleventh Book on the Tapta Krichchhra vrata and others in the Mah\=a Pur\=a\d{n}am \'Sr\={\i} Mad Dev\={\i} Bh\=agavatam of 18,000 verses by Mahar\d{s}i Veda Vy\=asa.



