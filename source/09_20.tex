\chapter{On the war preparations of \'Sankhach\=uda with the Devas}

1-21. N\=ar\=aya\d{n}a said :-- Brahm\=a, then putting \'Siva to the task of killing \'Sankhach\=uda went to His own abode. The other Devas returned to their homes. Here under the beautiful Bata tree, on the banks of the river Chandrabh\=ag\=a, Mah\=adeva pitched His big tent and encamped Himself to get the victory of the Devas. He then sent Chitraratha, the Lord of the Gandharbhas, as a messenger to \'Sankhach\=uda, the Lord of the D\=anavas. By the command of Mah\=adeva, Chitraratha went to the city of the king of Daityas, more beautiful than Indra's place and more wealthy than the mansion of Kuvera. The city was five yojanas wide and twice as much in length. It was built of crystals of pearls and jewels. There were roadways on all sides. There were seven trenches, hard to be crossed, one after another, encircling the city. The city was built of countless rubies and gems, brilliant like flames. There were hundreds of roadways and markets and stalls, in the wonderful Vedis (raised platforms) built of jewels. All around were splendid palatial buildings of traders and merchantmen, filled with various articles. There were hundreds and kotis of beautiful buildings, adorned with various ornaments and built of variegated red stones looking like Sind\=uras. Thus he went on and saw, in the middle, the building of \'Sankhach\=uda, circular like the lunar sphere. Four ditches all filled with fiery flames, encircled one after another, his house. So the enemies could not in any way cross them; but the friend could easily go there. On the top were seen turrets built of jewels, rising high to the heavens. The gate-keepers were watching the twelve gates. In the centre were situated lakhs and lakhs of excellent jewel built houses. In every room there were jewelled steps and staircases and the pillars were all built of gems and jewels, and pearls. Pu\d{s}padanta (Chitraratha) saw all this and then went to the first gate and saw one terrible person, copper coloured, with tawny eyes, sitting with a trident in his hand and with a smiling countenance. He told he had come as a messenger and got his entrance. Thus Chitraratha went one afer another to all the entries, not being prohibited at all though he told that he had come as a messenger on war service. The Gandharbha reached one after another, the last door and said :-- ``O Door keeper! Go quickly and inform the Lord of the D\=anavas all about the impending war.'' When the messenger had spoken thus, the gate-keeper allowed him to go inside. Going inside, the Gandarbha saw \'Sankhach\=uda, of excellent form, seated in the middle of the royal assembly, on a golden

throne. One servant was holding on the king's head an umbrella, decked with divine excellent gems, the inner rod of the umbrella being made up of jewels, and decorated with expanded artificial flowers made of gems. The attendants were fanning him with beautiful white ch\=amaras; he was nicely dressed, beautiful and lovely and adorned with jewel ornaments. He was nicely garlanded, and wore fine celestial garments. Three Koti D\=anavas were surrounding him; and seven Koti D\=anavas, all armed, were walking to and fro.

22-53. Pu\d{s}padanta was thunderstruck when he saw thus the D\=anava and he addressed him thus :-- O King! I am a servant of \'Siva; My name is Pu\d{s}padanta; hear what \'Siva has commanded me to tell you. You better now give back, to the Devas, the rights that they had before. The Devas went to \'Sr\={\i} Hari and had taken His refuge. \'Sr\={\i} Hari gave over to \'Siva one \'S\=ula weapon and asked the Devas to depart. At present, the three eyed Deva is residing under the shade of a Bata tree on the banks of the Pu\d{s}pabhadr\=a river. He told me to speak this to you, ``Either give over to the Devas their rights, or fight with me. Please reply and I will speak to Him accordingly.'' \'Sankhach\=uda, hearing the messenger's words laughed and said, ``Tomorrow morning I will start, ready for war. Better go away today.'' The messenger went back to \'Siva and replied to Him accordingly. In the meantime the following personages joined \'Siva and remained seated on excellent aerial cars, built of jewels and gems. The following were the persons :-- Skanda, V\={\i}rabhadra, Nand\={\i}, Mah\=ak\=ala, Subhadraka, Vi\'s\=al\=ak\d{s}a, B\=a\d{n}a, Pingal\=ak\d{s}a, Vikampana, Vir\=upa, Vikriti, Ma\d{n}ibhadra, V\=ask\=ala, Kapil\=ak\d{s}a, D\={\i}rgha Da\d{n}gstra, Vikata, T\=amralochana, K\=al\=akantha, Bal\={\i}bhadra, K\=alaj\={\i}hba, Kut\={\i}chara, Balonmatta, Ra\d{n}a\'sl\=agh\={\i}, Durjaya, Durgama, (these eight Bhairavas), eleven Rudras, eight Vasus, Indra, the twelve \=Adityas, fire, moon, Vi\'svakarm\=a, the two A\'svins, Kuvera, Yama, Jayanta, Nala K\=ubara, V\=ayu, Varu\d{n}a, Budha, Mangala, Dharma, \'San\={\i}, I\'s\=ana, the powerful K\=amadeva. Ugradamstr\=a, Ugracha\d{n}d\=a, Kotar\=a, Kaitabh\={\i}, and the eight armed terrible Dev\={\i} Bhadrak\=al\={\i}. K\=al\={\i} wore the bloody red clothings and She smeared red sandal paste all over Her body.

Dancing, laughing; singing songs in tune, very jolly, She bids Her devotees discard all fear, and terrifies the enemies. Her lip is terrible, lolling, and extends to one Yojana. On Her eight arms She holds conch, disc, mace, lotus, axe, skin, bow and arrows. She was holding in Her hands, the bowl shaped human skull; that was very deep and one Yojana wide. Her trident reached up to the Heavens; Her weapon called \'Sakti (dart) extended to one Yojana. Besides there were Mudgara

(mace), Musala (club), Vajra (thunderbolt), Kheta (club), brilliant Phalaka (shield), the Vai\d{s}\d{n}ava weapon, the Varu\d{n}a weapon, the \=Agney\=astra (the fire weapon), N\=agap\=a\'sa (the noose of serpents), the N\=ar\=aya\d{n}\=astra, the Gandharva's weapons, the Brahm\=a's weapons, the Gadud\=astram, the P\=arjanay\=astram, the P\=a\'supat\=astram, the Jrimbha\~n\=astram, the P\=arvat\=astram, the Mahe\'svar\=astram, the V\=ayavy\=astram, and the Sanmohanam rod and various other infallible divine weapons. Besides hundreds of other divine weapons were with Her. Three Kotis of Yogin\={\i}s and three Kotis and a half of terrible D\=akin\={\i}s were attending Bhadrak\=al\={\i}. Bh\=utas, (demons) Pretas, Pi\'s\=achas, Kusm\=a\d{n}das, Brahma R\=ak\d{s}asas, R\=akh\d{s}asas, Vet\=alas, Yak\d{s}as and Kinnaras also were there in countless numbers. At this time K\=artikeya came there and bowed down to his father Mah\=adeva. He asked him to take his seat on His left side an asked him to help. Then the army remained there in military array. On the other hand, when \'Siva's messenger departed, \'Sankhach\=uda went to the zenana and informed Tulas\={\i} of the news of an impending war. No sooner She heard than her throat and lips and palate became dried. She then with a sorrowful heart spoke in sweet words :-- ``O my Lord! O my Friend! O the Ruler of My life! Wait for a moment and take your seat on My heart. Instil life in Me for a moment. Satisfy My desire of human birth. Let me behold you fully so that my eyes be satisfled. My breath is now very agitated. I saw by the end of the night one bad dream. Therefore I feel an internal burning.'' Thus at the words of Tulas\={\i}, the king \'Sankhach\=uda finished his meals and began to address her, in good and true words, beneficent to her :-- ``O My Lady! It is K\=ala (the time that brings out these various combinations by which the Karmic fruit is enjoyed; it is K\=ala that awards auspicious and inauspicious things; the K\=ala is the Sole Master to impart pain, fear, and good and bad things.

54-84. Trees grow up in time; their branches, etc., come out in time; flowers appear in time and fruits come out in time. Fruits are ripen in time and after giving the fruits, they die out also in time. O Fair One! The universe comes into existence in time and dies away in time. The Creator, Preserver, and Destroyer of the universe, are creating, preserving and destroying the worlds with the help of time. Time guides them in every way. But the Highest Prakriti is the God of Brahm\=a, Vi\d{s}\d{n}u, and Mahe\'sa (i.e., the Creatrix of Time). This Highest Prakriti, the Highest God is creating, preserving and destroying this universe. She makes the Time dance. By Her mere Will, She has converted Her inseparable Prakriti into M\=ay\=a and is thus creating all things, moving and unmoving. She is the Ruler of all; the Form of all, and She is the Highest God. By Her is being done this creation of

persons by persons, this preservation of persons by persons, and this destruction of persons by persons. So you better now take refuge of the Highest Lord. Know it is by Her command the wind is blowing, by Her command the Sun is giving heat in due time, by Her command Indra is showering rains; by Her command, Death is striding over the beings; by Her command fire is burning all things and by Her command the cooling Moon is revolving. She is the Death of death, the Time of time, Yama of yama (the God of death), the Fire of fire and the Destroyer of the destroyer. So take Her refuge. You cannot find and fix who is whose friend in the world; so pray to Her, the Highest God, Who is the Friend of All. Oh! Who am I? And who are you either? The Creator is the combiner of us two and so He will dissociate us two by our Karma. When difficulty arises, the ignorant fools become overwhelmed with sorrow; but the intelligent Pundits do not get at all deluded or become distressed. By the Wheel of Time, the beings are led sometimes into happiness; sometimes into pain. You will certainly get N\=ar\=aya\d{n}a for your husband; for which you practised Tapas before, in the hermitage of Vadari (the source of the Ganges, the feet of Vi\d{s}\d{n}u). I pleased Brahm\=a by my Tapasy\=a and have, by his boon, got you as my wife. But the object for which you did your Tapasy\=a, that you may get Hari as your husband; will certainly be fulfilled. You will get Gobinda in Vrind\=abana and in the region of Goloka. I will also go there when I forsake this, my Demon body. Now I am talking with you here; afterwards we will meet again in the region of Goloka. By the curse of R\=adhik\=a, I have come to this Bh\=arata, hard to be attained. You, too, will quit this body and, assuming the divine form will go to \'Sr\={\i} Hari. So, O Beloved! You need not be sorry.'' O Muni! Thus these conversations took them the whole day and led them to the evening time. The king of the demons, \'Sankhach\=uda then slept with Tulas\={\i} on a nicely decorated bed, strewn with flowers, and smeared with sandal paste, in the Ratna Mandir (temple built of jewels.) This jewel temple was adorned with various wealth and riches. The jewel lamps were lighted. \'Sankhach\=uda passed the night with his wife in various sports. The thin bellied Tulas\={\i} was weeping with a very sorrowful heart, without having taken any food. The king, who knew the reality of existence, took her to his breast and appeased her in various ways. What religious instructions he had received in Bh\=and\={\i}ra forest from \'Sr\={\i} Kri\d{s}\d{n}a, those Tattvas, capable to destroy all sorrows and delusions, he now spoke carefully to Tulas\={\i}. Then Tulas\={\i}'s joy knew no bounds. She then began to consider, everything as transient and began to play with a gladdened heart. Both became drowned in the ocean of bliss; and the bodies of both of them were filled with joy

and the hairs stood on their ends. Both of them, then, desirous to have amorous sports, joined themselves and became like Ardhan\=ari\'svara and so one body. As Tulas\={\i} considered \'Sankhach\=uda, to be her lord, so the D\=anava King considered Tulas\={\i} the darling of his life. They became senseless with pleasureable feelings arising out of their amorous intercourses. Next moment they regained their tenaciousness and both began to converse on amorous matters. Thus both spent their time sometimes in sweet conversations, sometimes laughing and joking, sometimes maddened with amorous sentiments. As \'Sankhach\=uda was clever in amorous affairs, so Tulas\={\i} was very expert. So none felt satiated with love affairs and no one was defeated by the other.

Here ends the Twentieth Chapter of the Ninth Book on the war preparations of \'Sankhach\=uda with the Devas in the Mah\=apur\=a\d{n}am \'Sr\={\i} Mad Dev\={\i} Bh\=agavatam of 18,000 verses by Mahar\d{s}i Veda Vy\=asa.



