\chapter{On the anecdote of S\=avitr\={\i}, on gifts and on the effects of Karmas}

1. N\=ar\=aya\d{n}a said :-- Yama got thunderstruck at these queries of S\=avitr\={\i}. He then began to describe, with a smiling countenance, the fruition of the several works of the J\={\i}vas.

2-8. He said :-- ``O Child! You are now a daughter only twelve years old. But you speak of wisdom like the Highest J\~n\=anins and Yogis, Sanaka and others. O Child! By virtue of the boon granted by S\=avitr\={\i}, you have become incarnate of Her in part. The King A\'svapati got you before by performing severe penances. As Lak\d{s}m\={\i} is dear and fortunate with regard to Vi\d{s}\d{n}u, as Mah\=adevi is to Mah\=adeva, Aditi to Ka\'syapa, Ahaly\=a to Gautama, so you are to Satyav\=ana in respect of affection and good-luck and other best qualities. As \'Sach\={\i} is to Mahendra, as Rohi\d{n}\={\i} is to Moon, as Rati is to K\=ama, as Sv\=ah\=a is to Fire, as Svadh\=a is to the Pitris, as Sanj\~n\=a is to the Sun, as Varu\d{n}\=an\={\i} is to Varu\d{n}a, as Dak\d{s}i\d{n}\=a is to Yaj\~n\=a, as Earth is to Var\=aha, as Devasen\=a is to K\=artika, so you are fortunate and blessed with respect to Satyav\=ana. O S\=avitr\={\i}! I myself grant you this boon of my own accord. Now ask other boons. O highly fortunate One! I will fulfil all your desires.''

9-12. S\=avitr\={\i} said :-- ``O Noble One! Let there be one hundred sons of mine by Satyav\=ana. This is the boon that I want. Let there be one hundred sons of my Father as well; let my Father-in-law get back his (lost) eyesight and may he get back his lost kingdom. This is another boon that I want. Thou art the Lord of the world. So grant me this boon, too, that I may have this my very body for a lakh years when I may go to Vaikuntha with Satyav\=ana. Now I am eager to hear the various fruitions of Karmas of several J\={\i}vas. Kindly narrate them and oblige.''

13-70. Dharma said :-- You are very chaste. So what you have thought will verily come to pass. Now I describe the fruition of Karmas of the J\={\i}vas. Listen. Excepting this holy land of Bh\=arata, nowhere do the people enjoy wholly the fruition of their two-fold Karmas, good and bad. It is only the Suras, Daityas, D\=anavas, Gandharvas, R\=ak\d{s}asas, and men that do Karmas. The beasts and the other J\={\i}vas do not do Karmas. The special J\={\i}vas, e.g., men, etc., experience the fruition of their Karmas in Heavens, hells and in all the other Yonis (wombs). Specially, as the J\={\i}vas

roam in all the different Yonis, they enjoy their Karmas, good or bad, as the case may be, carved in their previous births. The good works get fructified in Heavens; and the bad works lead the J\={\i}vas to hells. This Karma can be got rid of by Bhakti. This Bhakti is of two kinds :-- (1) Nirgu\d{n}\=a of the nature of Nirv\=a\d{n}a; and (2) towards Prakriti, of the astute of Brahm\=a, and with M\=ay\=a inherent. Diseases come as the result of bad and ignorant actions and healthiness comes from good and certain scientific Karmas. Similar are the remarks for short and long lives for happiness and pain. By bad works, one becomes blind or deformed in body. So by doing excellent Karmas, one acquires Siddhis, etc.. These are spoken generally. I will now speak in detail; listen. This is very secret even in Pur\=a\d{n}as and Smritis. In this Bh\=aratavar\d{s}a men are the best of all the various classes of beings. The Br\=ahma\d{n}s are the best of men and are best in all Kinds of Karmas. They are responsible, too, for their actions. O Chaste One! Of the Br\=ahmi\d{n}s, again, those that are attached to the Br\=ahma\d{n}as are the best. The Br\=ahma\d{n}as are of two kinds as they are Sak\=ama (with desires) or Ni\d{s}k\=ama (without desires). The Ni\d{s}k\=am\={\i} Br\=ahma\d{n}as are superior to the Sak\=am\={\i} Br\=ahma\d{n}as. For the Sak\=am\={\i}s are to enjoy the fruits of their Karmas, while the Ni\d{s}k\=am\={\i} Br\=ahma\d{n}as are perfectly free from any such disturbances (they have not to come back to this field of Karma). The Ni\d{s}k\=ama Bhakta after they quit their bodies, go to a place free from sickness or disease, pure and perfect. From there they do not come back. The Ni\d{s}k\=ama Bhaktas assuming the divine forms go to the Goloka and worship the Highest God, the Highest Self, the two-armed Kri\d{s}\d{n}a. The Sak\=am\={\i} Vai\d{s}\d{n}avas go to Vaikuntha; but they come back in Bh\=arata and get into the wombs of the twice-born. By degrees they also become Ni\d{s}k\=ama when they certainly acquire pure undefiled Bhakti. The Br\=ahma\d{n}as and Vai\d{s}\d{n}avas that are Sak\=am\={\i}s in all the births, never get that pure undefiled intellect and never get the devotion to Vi\d{s}\d{n}u. The Br\=ahma\d{n}as, living in the T\={\i}rthas (sacred places of pilgrimages) and attached to Tapas go to Brahmaloka (the region of Brahm\=a); they again come down to Bh\=arata. Those that are devotedly attached to their own Dharma (religion) and reside in places other than T\={\i}rthas, go to Satyaloka and again come to Bh\=arata. The Br\=ahma\d{n}as, following their own Dharma and devoted to the Sun go to the world of the Sun and again come to Bh\=arata. And those who are devoted to M\=ula Prakriti and devoted to Ni\d{s}k\=ama Dharma go to Ma\d{n}i Dv\={\i}pa and have not to come back from thither. The Bhaktas of \'Siva, \'Sakti, and Gane'sa, and attached to their own Dharmas respectively go to the \'Siva Loka and return from thence. Those Br\=ahma\d{n}as that worship the other Devas and attached to their

own Dharmas go to those regions of theirs respectively and again come to Bh\=arata. Attached to their own Dharmas, the Ni\d{s}k\=am\={\i} Bhaktas of Hari go by their Bhakti step by step to the region of \'Sr\={\i} Hari. Those that are not attached to their own Dharmas and do not worship the Devas and always bent on doing things as they like without any regard to their \=Ach\=aras go certainly to hells. No doubt in this. The Br\=ahma\d{n}as and the other three Var\d{n}as, attached to their own Dharmas all enjoy the fruits of their good works. But those who do not do their Svadharma, go verily down into hells. They do not came to Bh\=arata for their rebirth, they enjoy their fruits of Karmas in hells! Therefore the four Var\d{n}as ought to follow their own Dharmas of the Br\=ahma\d{n}as, they are to remain attached to their own Dharmas and give their daughters in marriage to the similarly qualified Br\=ahma\d{n}as. They then go to the Chandraloka (the region of the Moon). There they remain for the life periods of the fourteen Indras. And if the girl be given, with ornaments, the results obtained would be twice. If the girl be given with a desire in view, then that world is obtained; but if the girl be given without any desire but to fulfil the God's will and God's satisfaction only, then one would not have to go to that world. They go to Vi\d{s}\d{n}u Loka, bereft of the fruits of all Karmas. Those that give to the Br\=ahma\d{n}as pasture ground and cattle, silver, gold, garment, fruits and water, go to the Chandraloka and live there for one Manvantara. They live long in those regions by virtue of that merit. Again those that give gold, cows, copper, etc., to the holy Br\=ahma\d{n}as, go to the S\=urya Loka (the region of the Sun) and live there for one Ayuta years (10,000 years), free from diseases, etc., for a long time. Those that give lands and lots of wealth to the Br\=ahmi\d{n}s, go to the Vi\d{s}\d{n}u Loka and to the beautiful \'Sveta Dv\={\i}pa (one of the eighteen minor divisions of the known continents). And there they live as long as the Sun and Moon exist. O Muni! The meritorious persons live long in that wide region. Note :-- \'Sveta Dv\={\i}pa may mean Vaikuntha, where Vi\d{s}\d{n}u resides. Those who give with devotion dwelling places to the Br\=ahma\d{n}as, go to the happy Vi\d{s}\d{n}u Loka. And there, in that great Vi\d{s}\d{n}u Loka, they live for years equal to the number of molecules, in that house. He who offers a dwelling house in honour of any Deva, goes to the region of that Deva and remains there for a number of years equivalent to the number of particles in that house. The lotus-born Brahm\=a said that if one offers a royal palace, one obtains a result four times and if one offers a country, one gets the result one hundred times that; again if one offers an excellent country, twice as much merit one acquires. One who dedicates a tank for the expiation of

all one's sins, one lives in Janar Loka (one of the pious regions) for a period equivalent to the number of particles therein). If any man offers a V\=ap\={\i} (a well) in preference to other gifts, one gets ten fold fruits thereby. If one offers seven V\=ap\={\i}s, one acquires the fruits of offering one tank. A V\=ap\={\i} is one which is four thousand Dhanus long and which is as much wide or less (Note :-- Dhanu equals a measure of four hastas). If offered to a good bridegroom, then the giving of a daughter in marriage is equivalent to a dedication of ten V\=ap\={\i}s. And if the girl be offered with ornaments, twice the merits accrue. The same merit accrues in clearing the bed of the mud of a pond as in digging it. So for the V\=ap\={\i} (well). O Chaste One! He who plants an A\'svattha tree and dedicates it to a godly purpose, lives for one Ayuta years in Tapar Loka. O S\=avitr\={\i}! He who dedicates a flower garden for the acquirement of all sorts of good, lives for one Ayuta years in Dhruva Loka.

O Chaste One! He who gives a V\={\i}m\=ana (any sort of excellent carriage) in honour of Vi\d{s}\d{n}u, in this Hindoosth\=an, lives for one Manvantara in Visnuloka. And if one gives a V\={\i}m\=ana of variegated colours and workmanship, four times the result accrues. And one who gives a palanquin, acquires half the fruits. Again if anybody gives, out of devotion, a swinging temple (the Dol Mandir) to Bhagav\=an \'Sr\={\i} Hari, lives for one hundred Manvantras, in the region of Vi\d{s}\d{n}u. O Chaste One! He who makes a gift of a royal road, decorated with palatial buildings on either side, lives with great honour and love in that Indraloka for one Ayuta years. Equal results follow whether the above things are offered to the Gods or to the Br\=ahma\d{n}as. He enjoys that which he gives. No giving, no enjoying. After enjoying the heavenly pleasures, etc., the virtuous person takes birth in Bh\=arata as a Br\=ahmi\d{n} or in other good families, in due order, and ultimately in the Br\=ahma\d{n}a families. The virtuous Br\=ahma\d{n}a, after he has enjoyed the heavenly pleasures, takes his birth again in Bh\=arata in Br\=ahma\d{n}a, K\d{s}attriya or in Vai\'sya families. A K\d{s}attriya or a Vai\'sya can never obtain Br\=ahma\d{n}ahood, even if he performs asceticism for one Koti Kalpas. This is stated in the \'Srutis. Without enjoying the fruits, no Karma can be exhausted even in one hundred Koti Kalpas. So the fruits of the Karmas must be enjoyed, whether they be auspicious or inauspicious. By the help of seeing the Devas and seeing the T\={\i}rthas again and again, purity is acquired. O S\=avitr\={\i}! So now I have told you something. What more d o you want to hear? Say.

Here ends the Twenty-Ninth Chapter of the Ninth Book on the anecdote of S\=avitr\={\i} on the fruits of making gifts and on the effects of Karmas in \'Sr\={\i} Mad Dev\={\i} Bh\=agavatam of 18,000 verses by Mahar\d{s}i Veda Vy\=asa.



