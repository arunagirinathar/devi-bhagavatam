On the preparation for war by Madhu Kaitabha

p. 19

 

1-44. The Risis said :-- “O Saumya! Just now you have spoken of the fight for five thousand years, in that great ocean, between Bhagavân S’auri and Madhu Kaitabha. How was it that the two greatly powerful Dânavas, invincible of the Devas came to be born there? And why did Bhagavân Hari kill them? O highly intelligent one! Kindly describe that greatly wonderful event. We all are extremely eager to hear it, and you are the great Pundit and speaker, present before us. It is our good luck that we have come across you here. As the contact with the illiterate is very painful, so the contact with the literate is very happy like nectar. The animals in this world live like illiterates; they eat, call for their nature, void urines and faeces, and know wonderfully well the sexual intercourse. Only they want discriminative knowledge of right and wrong, of the real

 

 

and unreal, and a knowledge of discrimination, leading to Moksa or final liberation; this is the only point of difference. Therefore the persons that have no liking to hear of Bhâgavata and books like it, are like beasts; there is no doubt in this. Behold! Deer and some other animals can enjoy well the sense of hearing like men; and the serpents, though wanting in the organ of hearing, become charmed quite like men, as if tasting the pleasure of hearing sweet sounds. Verily out of the five organs of perception the organ of hearing and the organ of sight are benefitting, for the knowledge of things arises from hearing and the heart is pleased by seeing. Therefore the Pundits divide in three classes, the objects of hearing, as :-- (1) Sâttvik, (2) Râjasik and (3) Tâmasik. The Vedas and other S’âstras are Sâttvik; the literature Sahitya is the Râjasik and war news and finding fault with others is Tâmasik. The wise persons again sub-divide the Sâttvik in three sub-classes :-- good, middling and worst. That which gives Moksa is good or excellent; that which gives Heavens is middling and that which gives this worldly pleasure is worst. In the same way, the literature Sahitya is of three kinds :-- That which describes the people to live with their legitimate wives is the best; which describes about prostitutes is the middling; and that which makes people live with other's wives is the worst.

 

The seers of Sâstras, the learned men divide the subjects of the Tâmasik hearing into three classes :-- That in which fight with the enemies is described is best; where the fight, as of the Pandavas, with the enemies out of hatred, ill-feeling, is described is middling; and that where fighting is described without any cause is worst. Therefore, O highly intelligent one! Hearing the Purânas is far superior to hearing other S’âstras, for thereby sins are destroyed, intellect is increased and Punyam (good merits) is stored. So, O intelligent one! Kindly describe to us, the Purânas, fulfilling all the requirements of life, that you heard before from the mouth of Krisna Dvaipâyana”. Hearing these words of the Risis, Sûta said :-- “O highly fortunate ones! When you all are desirous to hear the Purânas and I am ready to tell them, then both of us are blessed on the surface of the earth”.

 

In days of yore, in the time of Pralaya (universal dissolution) when the three lokas and the entire universe dissolved in water, when the Devadeva Janârdana was lying asleep on the bed of Ananta, the thousand headed serpent, arose from the was of the ear of Bhagavân Visnu, the two very powerful Daityas Madhu and Kaitabha; they grew in the waters of the ocean and played around in the waters and thus passed some of their time. Once, on a time, when the two huge bodied Dânavas were playing with each other like two brothers, they thought that the general

 

p. 21

 

rule of the universe is that no effect takes place without a cause and nothing can rest without the receptacle thereof. But we cannot understand what is our receptacle or who is resting on us. Whereon rests this pleasant expanse of wide ocean? Who was it that created this? How was this created? Why are we living here merged within the water? Who created us? and who are our father and mother. Nothing of a these we know. Thus thinking, when they could not come to any conclusion, Kaitabha spoke to Madhu, beside him, within the waters :-- “O Brother! It seems to me the great immoveable force that makes us rest in this water is the cause of all. This whole mass of water, too, pervaded by that force, rests on that; that Highest Devî must be the Cause of us”.

 

When the two Asuras, merged in this thought, understood this, they heard in the air the beautiful Vâgvîja (the seed mantra of Vâk, the speech, the Devî Sarasvatî). They then began to pronounce repeated the Vâgvîja mantra and practised it with the great steadfastness. Next they saw, risen high up in the air, the auspicious lightning and thought that certainly our mantra that we are repeating has made Herself visible in this form of light and thus we have seen certainly in the air, the saguna form (form with attributes) of Sarasvatî, the goddess of Speech. Thus thinking in their minds they, without any food, with their minds controlled, constantly thought of that, with their whole mind collected on that, and repeating and meditating the mantra became one with that. Thus they passed one thousand years in practising that great tapas; when the Highest Âdyâ S’akti became pleased with them and seeing the two Dânavas, steadfast in the practice of Tapas, tired, address them, invisibly in the way of celestial voice thus :-- “O two Dânavas! I am exceedingly pleased with your tapasyâ; so ask boon whatever you desire; I will grant it.” Hearing, then, the celestial voice, thus the two Dânavas said :-- “O Devî! O Suvrate! Grant us that we will die when we will.” Hearing this, Vagdevî said :-- “O two Dânavas! Certain by My grace, you two will die when you will and you two brothers will be invincible of all the Suras and Asuras. There is no doubt in this”.

 

Sûta said :-- When the Devî granted them this boon, the two Dânavas, puffed up with pride, began to play with the aquatic animals in the ocean. O Brâhmins! Some days thus passed away when the two powerful Dânavas saw the Brahmâ, the Prajâpati, seated on the lotus of navel of Hari. Doubt came on their minds and they told him with a view to fight :-- “O Suvrata! Either fight with us, or leave off this lotus seat and go any where you like. If you be so weak, this auspicious lotus seat not fit for you. For this should be enjoyed by the heroes. So if you

 

p. 22

 

be a coward, leave it quickly”. Hearing these words of the Dânavas, Prajâpati, engaged in the practice of Tapasyâ, saw the two great powerful heroes and began to think anxiously “What should be done now” and waited there.

 

Thus ends the sixth chapter of the first Skandha on the preparation for war by Madhu Kaitabha in the Mahâpurâna S’rîmad Devî Bhâgavatam by Maharsi Veda Vyâsa.