\chapter{On Adharma}

1. The King spoke :-- O highly honoured and intelligent one! I have heard the anecdote just related to me by you. I am very much bewildered. This Sams\^ara (world) is vice incarnate. I wonder how the J\^ivas, entangled in its meshes, can again be freed!

2. When the son of Ka\'syapa, whose kingdom is the three worlds, can commit such an heinous act, what wonder, then, that any other ordinary person would do more blameable things!

3. On the pretence of serving and on a solemn oath, when a man, can enter into his step-mother's womb and take away the life of the son, what more heinous and dreadful can take place than this!

4. When the preserver and controller of religion, the ruler of the three worlds can do such acts, you cannot expect that any other person would desert from committing heinous, contemptible acts.

5. O World Teacher! Indeed my grandfather did unjustifiable horrible mean acts in the battle field of Kuruksettra.    It is really wonderful!

6-7. Bh\^isma, Drona, Kripa, Karna, even Yudhisthira, who is the part incarnate of Dharma all these were urged by V\^asudeva into this contrary religious act. These personages are all born of Dev\^amsas, devoted to religion, and intelligent. These know the transitory nature of this world; how can these commit such mean blameable things!

8. O Glory of the Br\^ahmins! What faith or regard can we have for a religion, when such high souled persons commit such irreligious acts! Indeed, there is doubt whether religion exists at all or not! O Best of the Munis! My heart is agitated very much on hearing these incidents.

9. If it be said that the word of the \^Aptas (seers) is a sufficient guarantee for the entity of religion, it may then be questioned where there is such an \^Apta, holding such a pure religious body? All those persons who are attached to worldliness are bent on all worldly objects with their whole head and heart; these, therefore, cannot be \^Aptas.

10. When self interest is obstructed, jealousy and anger arise; and to secure one's self interest, out of jealousy, arise untruthful words.

11. Even the pure, excellent, virtuous \'Sr\^i Krisna, with full consciousness, had to assume under pretence a Br\^ahmin form for killing Jar\^asandha.

12. Just as the holy \'Sr\^i Hari assumed a false appearance to kill Jar\^asandha, similarly Arjuna, too, did a false sacrifice to accomplish his ends. Where is, then, one who can claim to be an \^Apta? And what proof is there of the existence of such an \^Apta?

13. What sort of sacrifice was this? Did it lead to heaven in the next world or did it lead to glory or did it serve the cause of any good will? Why was it deprived of that peace and rest? (It was performed with a view to kill \'Sisup\^ala and others).

14-15. The Pundits, of yore, declare that truth is the first P\^ada, cleanliness, the second; compassion, the third; and charity is the fourth P\^ada (foot) of Dharma (Religion). Thus, devoid of these, how can Religion stand with due regards from all?

16. How can an act bear good fruits, which has no trace of virtue in it? It seems no one had any trace of faith and steadiness in one's religion. (The P\^andavas did sacrifice out of arrogance; how can they be \^Aptas?)

17-20. Visnu, the Lord of the Universe, assumed His Dwarf (V\^amana) Incarnation with the express object to cheat Vali, the king of the Daityas. Now, O Muni! The king Bali, performed one hundred sacrifices; he was the protector of the Vedas, virtuous, charitable, truthful and self controlled; why was such a man dislodged from his position by Visnu, the Powerful. Who was victorious in this affair? Was he the Vali, who was cheated? Or was it V\^amana Deva, the expert in making nice pretence? Who was the better of the two? I have got grave doubts on this point. O the best of the twiceborn! You are the composer of the Pur\^anas, virtuous, and liberal hearted. Speak what is true (and thus tranquil my heart).

21-23. Vy\^asa said :-- O king! The victory was certainly Bali's, in as much as he fulfilled his promise and gave over his kingdom of earth to Visnu. And in as much as Visnu in his 5th or dwarf Incarnation deceived Bali, he had to become a dwarf (i. e. a small mean person indicated even by the shortness of his body). O king! There is nothing superior in religion to truth. See! \'Sr\^i Hari even had to become, for his falsehood, a gate keeper of Vali. O king! It is hardly possible for a human being to observe in every way the injunctions of truth.

24. Powerful, indeed, is M\^ay\^a, composed of the three qualities and of various forms. By Her is created this Universe, made manifold by the admixture of the three qualities (Sattwa, Rajas and Tamas).

25. How can you expect therefore, truth to be observed wholly, without the least violation, by a deceiver. This world is made up of the mixture of Rajas; O king! Know this as the every day routine of things in nature.

26-27. It is only the Munis and Hermits that can observe pure truth; and that is why they are without any attachment; they do not accept any thing from any other body; they are desireless; and they all have no rough wear and tear of the world. They exist as perfect examples; their case is quite separate. All the others are caught under the meshes of the three M\^ayic Gunas.

28. O Best of kings! The Dharma \'S\^astras, Pur\^anas and the Angas the Vedas are full of diverse opinions on any one point under consideration for their composers were under the influence of the different Gunas.

29. The Saguna persons (i. e. persons under M\^ay\^a) do Saguna works (works composed of qualities) and the Nirguna persons (i. e. persons above M\^ay\^a) do not do any Saguna work. And when the Gunas are mixed with one another, they cannot remain pure (i. e. they exhibit qualities of those Gunas with which they are mixed).

30. O king! One is influenced by M\^ay\^a no sooner one takes one's birth in this world; so that no body can remain steady in this pure, steady maxim of truth, untainted by any falsehood or deceit.

31. The sense organs, Indriyas, confound the Buddhi (reason) and make one follow the path of enjoying sensual things. Mind is attached to senses and follows diverse ways, urged on furiously by the three Gunas.

32. O king! All the beings, Br\^ahm\^a down to the moving and non-moving things, fall under the delusion of M\^ay\^a; She plays with them.

33. This M\^ay\^a is always imposing on all; and She is incessantly making formations and transformations in this Universe; O king of kings! The man under the influence of action takes recourse to this untruth (i.e., actions arise first from this untruth) from the very moment of his birth.

34. Persons when they do not get their desired objects after they have pondered how to secure the sensual objects, take recourse to pretext, and, from that pretext do many sinful acts.

35. Lust, anger, and avarice; these three are very powerful enemies. The J\^ivas under their influence cannot distinguish the right from the wrong.

36. When wealth, might and rank come to a person, he gets deep-rooted Ahamk\^ara, and becomes very egoistic; from Ahamk\^ara, delusion comes and, from delusion, insensibility and death ensue.

37-38. Here men argue mentally many plans; and thence jealousy, intolerance and enmity spring in the heart; next arise, out of delusion, hope, thirst, misery, low-spiritedness, arrogance and irreligiousness.

39. It is through Ahamk\^ara that people are led to perform sacrifices, charities, visit places of pilgrimages, practise vows and rules for religious rites and ceremonies.

40. Hence these sacrificial acts, etc., proceeding from Ahamk\^ara, are unable to remove the clouds of impurity from the mind, as observance of purity and cleanliness does. Especially when any action is done through greed or undue affection, as its motive, it cannot be pure in every respect.

41. Therefore, at the commencement of any sacrifice, the wise persons look at the purity of sacrificial things; (Dravya Suddhi); those articles that are collected without injuring others, are the best in religious acts.

42. O best of Kings! If the things, acquired by injuring others, be utilised in any auspicious act, they yield contrary results at the time of fruition.

43. It is he only, whose mind is very pure and undefiled, who gets the results wholly auspicious from any sacrificial act. Minds defiled do not acquire their proper desired objects.

44-45. When the preceptor and the priests ordained are sincere and pure; moreover, when the place, moment, act, sacrificial things, the mantras, and the sacrificer are all holy, then and there only, the full results accrue in their entirety to the sacrificer.

46. If the sacrifice be intended for the destruction of one's enemy or for a personal motive and one's gain, it converts auspicious results into those that are inauspicious and lead to ruin in the end.

47. Selfish persons are unable to ascertain, which actions are auspicious and which are not; they depend on the circumstances what they call Daiva, and the people do acts sinful instead of virtuous.

48-49. The Devas and demons all are created by Br\^ahm\^a, the Praj\^apati, the Creator; they all are selfish; hence they are at war and war with each other. The Devas are born from the Sattva Guna; the human beings are sprung from the Rajas and the birds are sprung from the Tamas.

50. O King! When the Devas, born of the Sattva Guna are always engaged in inimical actions, what wonder, then, is there, that the lower ones would be at war with one another!

51-52. O King! When the Devas are always discontented, filled with jealousy and envy, at war amongst each other, and obstructors of the ascetics and the austere persons, then know that this Universe has sprung from Ahamk\^ara (egoism). How can you expect them to be free from feelings of anger, jealousy; etc.!

Here ends the Fourth Chapter of the Fourth Book, the Mah\^a Pur\^anam \'Sr\^i Mad Dev\^i Bh\^agavatam of 18,000 verses on Adharma by Maharsi Veda Vy\^asa.