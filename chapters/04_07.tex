\chapter{On Ahamk\^ara}

1. Vy\^asa said :-- O king! The Dharma's son, of excellent prowess, hearing thus, the words of these damsels, thought within himself, thus :-- what shall I do under the above circumstances.

2. If I indulge now in sexual pleasures, I will be an object of laughter amongst the Munis. This present trouble has, no doubt, arisen from my Ahamk\^ara (egoism). This Ahamk\^ara is the first and foremost in ruining one's Dharma.

3-5. The wise sages have declared this Ahamk\^ara as the root of this tree of world. I did not observe the vow of silence on seeing those damsels come here; I have held much conversations with them; therefore I have got into this troublesome anxiety and sorrow. I have created these damsels at the expense of my Dharma and Tapasy\^a. The beautiful and lovely damsels sent by Indra are now full of lust; and are bent on ruining my tapasay\^a. Now if through Ahamk\^ara I had not created the damsels, I would not have fallen into this difficulty. Now I am caught firmly in the meshes of my own creation like a spider; what am I to do next!

6-7. If I abandon these damsels, thinking that there is no necessity to reconsider the matter, then these would be broken hearted; and they would go away cursing me?

Yet I would be free from this present danger at least and then be able to practise excellent tapasy\^a in a lonely place. Therefore, now, I will get angry and tell these damsels go away from me.

8. Vy\^asa said :-- O King! The Muni N\^ar\^ayana thought that he would become thus happy; but, at the next moment, he discussed in his mind thus :--

9. The second great enemy is anger; it is greater than causing hurt to others; and it is greater than lust and avarice.

10. Out of anger people commit murder; this murder is the source of hell and is giving pains to all.

11. As trees, rubbing against each other, generate fire; and are themselves burnt up in this fire, so fire arising from this body ultimately burns this body to death.

12. Vy\^asa said :-- The younger brother Nara on seeing his elder brother anxious and low-spirited, spoke out what is right, as follows :--

13. O N\^ar\^ayana! You are very intelligent and very good; therefore reliquish this feeling of anger and betake to quietude and peace, and kill the dreadful anger.

14. Do you not remember that it is through this Ahamk\^ara and anger that our tapasy\^a was destroyed on a previous occasion; and we had to fight severely with Prahl\^ada, the Lord of the Asuras for one full divine thousand years.

15-16. O Lord of the Devas! We were put to much difficulties then; therefore O Lord of the Munis! Get rid of this anger; be quiet! The sages declare the peace is the root cause and the only object of Tapasy\^a.

17.Vy\^asa said :-- On hearing these words of his younger brother Nara, the Dharma's son N\^ar\^ayana took to peace.

18. Janamejaya said :-- O Lord of the Munis! The high souled Prahl\^ada was a devotee of Visnu and of a peace loving heart: how it was that, in the ancient days, the battle took place between him and these Risis; how could the Risis fight? There is this great doubt in my mind.

19-20. These two Dharma's sons were ascetics and peace loving; how the fight could come to pass between these and the Daity\^asuras? How did these two Risis fight with the high souled Prahl\^ada?

21-22. Prahl\^ada was very religious, full of knowledge and very much devoted to Visnu. Nara N\^ar\^ayana were Sattvik and ascetics; therefore if there had occurred enmity between those, it appears that the religion and asceticism, Tapasy\^a and Dharma were matters in name only; and the labour was spent in vain in the golden age even. What was the value of asceticism and meditation and muttering silently the mantras! No one can make out.

23. Oh! Persons like them could not conquer their hearts full of anger and egoism! Anger and jealousy cannot spring unless there be at the bottom a feeling of egoism (Ahamk\^ara).

24. All the passions, lust, greed, anger, etc., come out of Ahamk\^ara (egoism), there is no doubt of it; one hundred lakh years of severe asceticism are rendered quite useless by the cropping up afterwards of a bit of Ahamk\^ara.

25. As darkness is dispelled entirely on sunrise, so no trace of religious merit can exist on the rising of a bit of Ahamk\^ara.

26. When Prahl\^ada could fight with \'Sr\^i Bhagav\^an Hari, then, Oh! all his merits in this world are rendered of no use whatsoever.

27. Where is the religious merit and where is peace when the quiet souled persons Nara N\^ar\^ayana, the two Risis began to fight, without paying any heed to their highest end, the Tapasy\^a?

28. When Ahamk\^ara became invincible by the two Risis, then what can be expected from the weak trivial persons like us in the matter of subjugating this Ahamk\^ara?

29. Who can be free from Ahamk\^ara in these three worlds, when the high souled persons like the above were not free from it? I am now quite confident that, in this Universe, no body was ever before free from Ahamk\^ara nor will there be any such in the distant future.

30. One can be free if bound by an iron or a wooden chain; but when one is pierced by Ahamk\^ara, one can never become free from it.

31. This whole Universe, moving and unmoving, is rolling in this Sams\^ara (migration and transmigration) polluted by urine and faeces, being covered by Ahamk\^ara.

32. Where is, then the Brahm\^a Jñana? O Good One in vows! The Karma theory, according to the Mim\^amsakas, seems reasonable and true.

33. O Muni! What can you expect from the weak-minded persons like me in this Kali yuga, when the great persons are always overpowered with lust, anger, etc.

34-35. Vy\^asa said :-- O Descendant of Bharata! How can the effect be different from its cause? Gold and golden ear-rings though different in form owing to up\^adhis, are both similar to their original cause, the metal gold.

Thread is the cause of cloth; therefore as cloth cannot be different from its thread, so this whole universe, moving and unmoving, is sprung from Ahamk\^ara; then how can it be free from Ahamk\^ara?

36. All this, moving and unmoving, including a blade of grass, are fashioned out of the three qualities of M\^ay\^a; so if it be formed of those qualities, what repentance can come to those who are wise and know every phenomenon as unreal?

37. O Best of kings! Brahm\^a, Visnu or Mahe\'sa, even these are all rolling in this vast ocean of Sams\^ara, being bewildered and fascinated by Ahamk\^ara.

38. The great sages like Va\'sistha, N\^arada and the other Munis are frequently taking their births in this Sams\^ara.

39. In this Trilok\^i, there is not even one embodied soul, who is entirely free from this M\^ay\^a and has become quiet and immersed in the high bliss of the Supreme Self.

40. O Best of kings! Lust, anger, avarice, and fascination, all, arise from Ahamk\^ara. These do not leave any embodied person.

41-42. Studying all the Vedas and Pur\^anas, going to all the sacred places pilgrimages, making charities, thinking on Param\^atman and worshipping the gods, doing all these, the people still get attached to sensual objects and act like a thief.

43. O Son of Kuru! In the three yugas, the Satya, Tret\^a, Dw\^apara, Dharma had been pierced and wounded very much; what to say of Dharma in this Kali Yuga!

44. You will find quarrels, avarice, anger raging always in this Kali yuga. Therefore there is no wonder that you will not find any one thinking and doing what is worth thinking, and doing what is not worth doing?

45. Free from envy, anger, and jealousy, such persons are rare now-a-days in this Kali yuga. Some peaceful persons exist here and there to keep up the ideal.

46. The king said :-- O Muni! They are blessed and holy who are free from this fascination of M\^ay\^a, self controlled, who have conquered their passions, and who follow good conduct. They have risen above the Trilok\^i.

47. O Best of Munis! My high minded father put a dead serpent round the neck of an ascetic without any fault; I am very sorry to think of his act.

48. Therefore, O Muni ! Kindly suggest any means by which I can now redress that act. O Bhagavan! I do not know what will be the result of this act, committed out of the bewildering of intellect.

49. Fools in search of honey see only honey before them but not the falls, whence they might tumble down and die. So the stupid men do disgraceful acts and do not get afraid of the tortures of hell.

50. Kindly describe, in detail, how the fight incurred between Prahl\^ada and N\^ar\^ayana in ancient times.

51. How was it that Prahl\^ada went out of P\^at\^ala (the nether regions) and went to the great holy place, the hermitage of Badarika\'srama in the S\^arasvata country, the great place for pilgrimage.

52. O Muni! What was it that led the best of the Munis, the two ascetics to fight with Prahl\^ada?

53. The enmity springs where there is wealth, wife, or land. The two Maharsis were desireless, had nothing of these; how, then, without any cause, they fought such a battle!

54. Prahl\^ada was also very religious and knew that those two Risis were the Devas; knowing this, why did he fight with them?

55. So describe in detail the cause of all these.

Here ends the Seventh Chapter in the 4th Book of \'Sr\^imad Dev\^i Bh\^agavatam of 18,000 verses on Ahamk\^ara by Maharsi Veda Vy\^asa.