\chapter{On the fight between the Risis and Prahl\^ada}

1. Vy\^asa said :-- After performing duly his religious rites there, the son of Hiranya Ka\'sipu saw before him an unbrageous peepul tree.

2-3. There he saw along with the feathers of vultures, the terrible, sharpened under a stone, various glittering arrows, arrayed in due order; and he was surprised to think who could have kept such arrows, well guarded in this very holy hermitage of the Risis.

4-5. While Prahl\^ada was thus meditating in his mind, he saw before him, wearing the skin of a black antelope, the two sons of Dharma, the two Munis Nara N\^ar\^ayana, loaded on their heads with high clots of hairs. Before them were placed the two white bows named \'S\^arngam and \^Ajagavam, (Pin\^aka) the bows of Visnu and \'Siva respectively, bearing their qualified marks, as well as their two inexhaustible big quivers.

6-10. The Lord of the Daityas, then, saw those two blessed ones, the two Risis Nara N\^ar\^ayana, the two sons of Dharma, deeply absorbed in meditation. Seeing this, he became very much enraged, his eyes became red, and he began to address them thus :-- O two Ascetics! Has vain arrogance possessed your mind to destroy religion? It is never seen nor even heard, that the practice of severe asceticism and the holding of the bows and arrows, were carried into effect simultaneously by one man in any of the Four Ages. These two are contradictory things. That may be worthy in the Kali Yuga? This asceticism is fit for the Br\^ahmanas; why, then are the bows and arrows held by you? There is an irreconcilable difference between the holding of clotted hairs on the head and the holding in the hand of the bows and arrows. Therefore, do you practise religious rites, with feelings befitting your divine positions!

11. Vy\^asa said :-- O Descendant of Bharata! On hearing thus the Prahl\^ada's words, the Nara Risi said :-- O Lord of the Daityas! What matters it to you? Why do you, for nothing, trouble yourself with our tapasy\^a.

12. An able man can accomplish any thing. It is widely known in the three worlds, that we are able to accomplish these two things contemporaneously. O thou of weak understanding!

13. In the battle field as well as in asceticism, we can shew our prowess. What have you got to do with us in these matters? The road before you is unobstructed, you can go wherever you like; why do you brag of your own merits?

14. You are very dull and stupid; what can you understand of a Br\^ahmanic glory that is very rare and attained with great difficulty? Those that want happiness need not meddle with the Br\^ahmanas.

15-16. Prahl\^ada said :-- Blunt headed and vain braggarts are you! When I am present in this T\^irtha, I who am the upholder of Dharma, I won`t allow you to practise any irreligious things here! O Ascetics! Better show me your skill in fight to-day.

17-18. Vy\^asa said :-- O king! The Risi Nara on hearing his words replied :-- Give us battle if you are so desirous. O wretched amongst the Asuras! In to-day's battle, I will knock your head down, and then you shall never in future desire to fight with any body.

19-20. Vy\^asa said :-- On hearing these words, the chief of the Daityas was very much angry and promised to conquer these two self controlled Risi ascetics, Nara N\^ar\^ayana by any means whatsoever.

21-22. Vy\^asa said :-- Thus saying, the Daitya took up his bow and quickly stretched it with arrow and the bow string made a terrible noise. Then Nara, too, too up with anger his bows and began to shoot arrows at Prahl\^ada lots of sharpened arrows and weapons.

23. The lord of the Daityas quickly selected the arrows, glittering like gold, and with them tore asunder the arrows thrown by Nara. Nara, seeing his arrows, cut asunder, became infuriated with anger and began to hurl as quickly lots of other arrows.

24. Prahl\^ada then cut asunder with arrows, of quick velocity, the Nara's weapons and struck violently on the breast of Nara. Nara, too, with anger pierced the arms of Prahl\^ada with five quick arrows.

25. Indra and the other Devas came on their respective aeroplanes to see their fight and began to give cheers sometimes to Nara and sometime to Prahl\^ada from above the skies.

26. The Lord of the Daityas taking up his bow, began out of furious anger to hurl various weapons on Nara as incessantly as clouds give rain over the mountain peaks. The Nara Muni now became very much exhausted and weary, being struck with Prahl\^ada's arrows.

27. N\^ar\^ayana then seeing Nara exhausted, became very much amazed and bolding his unequalled \'S\^arnga bow, began to quit arrows, shining with with golden lustre.

28. O Lord of the earth! Then Nar\^ayana and Prahl\^ada both were desirous to win the victory, and a terrible fight ensued. The Devas gladly poured forth flowers on their heads from the skies.

29. The king of the Daityas got very much enraged and began to hurl arrows with tremendous quickness. N\^ar\^ayana, the son of Dharma, immediately cut asunder those weapons with his very sharp arrow.

30-32. N\^ar\^ayana too, threw arrows sharpened under stones with high velocity and very much troubled the lord of the Daityas, who now became very much restless.

33-34. The sky was covered over with arrows and arrows from both the parties and the day looked like the night. Then the Devas and the Daityas were very much astonished and told each other, ``We never saw before a terrible fight like this.''

35. Then the Devarsis, Gandarbhas, Yaksas, Kinnaras, Pannagas, Vidy\^adharas and Ch\^aranas were all very much confounded.

36-37. The two Risis N\^arada and Parvata came also to witness their fight; the Devarsi N\^arada told the Parvata Risi he never saw before such a dreadful fight. There were awful battles with T\^arak\^asura and Vritr\^asûra and also the battle between Hari and Madhukaitava; but they were all inferior and cannot stand in comparison.

38. It seemed that Prahl\^ada was very powerful; otherwise how could an equal fight last so long a time with such an accomplished person, perfect with all the supernatural powers and of such heroic deeds as N\^ar\^ayana.

39-42. Vy\^asa said :-- O king! Day and night the Daityas and the ascetic N\^ar\^ayana went on fighting terribly with each other. Then N\^ar\^ayana cut off, with the arrow, the bow of Prahl\^ada; Prahl\^ada soon took another bow; the expert handed N\^ar\^ayana quickly broke into two that bow. Thus though Prahl\^ada's arrows were repeatedly cut asunder still he began to take up fresh bows and N\^ar\^ayana began to cut them repeatedly.

43-44. Thus, then, when all the bows of Prahl\^ada were destroyed, the Daitya R\^aj\^a took up Parigha (iron club), became enraged and threw it on N\^ar\^ayana's arm. The powerful Bhagav\^an N\^ar\^ayana, seeing the dreadful iron club, cut it asunder with nine arrows and pierced Prahl\^ada with ten arrows.

45-47. Then Prahl\^ada, enraged, threw the iron gad\^a on N\^ar\^ayana's thighs. The exceedingly powerful Dharma's son was not at all agitated and stood firm like a rock and taking up arrows quickly cut asunder the iron gad\^a of the Daitya. Then the visitors were much startled.

48-49. Then Prahl\^ada, intent on killing his enemy, became very much angry and threw the \'Sakti darts, spears and missiles instantly on N\^ar\^ayana's thighs with great velocity. N\^ar\^ayana with one arrow cut that easily into seven parts and with seven arrows pierced Pr\^ahlad\^a.

50. Thus for one thousand Deva years the terrible fight lasted between Prahl\^ada and N\^ar\^ayana in that hermitage; and the whole universe was struck with surprise.

51-52. Then Gad\^adhara with yellow robes and four hands quickly came there and called Prahl\^ada. The son of Hiranya Ka\'sipoo, Prahl\^ada, seeing the Lord of Laksmi, four armed, N\^ar\^ayana with lotus and disc in His hands come there, bowed low, and, with folded hands, began to speak to him with great devotion.

53-54. O Deva of the Devas! You are the Lord of the universe and devoted to your devotees. O M\^adhava! I have fought for full one hundred Deva years; still I have not not been able to defeat these ascetics. I do not know why. I am surprised at it.

55-56. Visnu said :-- O Forgiving One! These two Risis Nara N\^ar\^ayana are, the perfect ascetics, self controlled and born of my Amsas. Therefore you have not been able to defeat them. What wonder is there! O king! Better go now to your P\^at\^ala and keep your steadfast devotion on me. O Intelligent one! Do not quarrel any more with these two ascetics.

57. Vy\^asa said :-- O king! The Daitya king Prahl\^ada then advised by Visnu went out of that place with his Asura followers; and the two Nara N\^ar\^ayanas began again to practise their Tapasy\^as.

Here ends the Ninth Chapter of the Fourth Book of \'Sr\^i Mad Dev\^i Bhag\^avatam, the Mah\^a Pur\^anam; of 18,000 verses, on the fight between the Risis and Prahl\^ada by Mahars\^i Veda Vy\^asa.

