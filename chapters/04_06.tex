\chapter{On the origin of Urva\'s\^i}

1. Vy\^asa said :-- O king! First there appeared on the mountain, the king of the seasons, Vasanta, the Spring. All the trees flowered and became very beautiful; and the bees began to hum round all sides.

2. Mangoes, Bokul trees, the beautiful Tilaka trees, the good Kimsukas, S\^al, T\^al, Tam\^al and Madhuka trees assumed unequalled beauties, ornamented with their flowers.

3. Cuckoos began to coo coo (warble) beautifully on the tops of trees; the creepers flowered and began to embrace the trees.

4. The creatures became enamoured with love and began to look on their paramours with amorous eyes and began to hold pleasant sexual intercourses.

5. The southern breeze blew gently, full of pleasant odours and agreeable to touch. The sensual organs became very powerful and could no longer be brought under their control by the Munis.

6. Then K\^ama, united with Rati, hurriedly entered into the Hermitage of Badarik\^a with the five arrows in his hands.

7. Rambh\^a, Tilottam\^a, and other prominent Apsar\^as all went to that beautiful hermitage and began to sing in perfect tune with gamuts, key notes and respective pauses.

8. The two Maharsis awoke on hearing the sweet music, the warbling of the cuckoos and the nice hummings of the bees.

9. Nara N\^ar\^ayana became anxious to see the untimely bursting of the Vasanta (vernal season) and the flowering of the trees.

10. How can the spring season come now at such an untimely season. I see, all the creatures are become extremely amorous with each other and infatuated with passionate lust.

11. It is very unusual that untimely things should happen. How has this come to pass? Struck with wonder, N\^ar\^ayana began to speak to Nara with eyes wide apart.

12. N\^ar\^ayana said :-- O Brother! See these trees look very elegant with flowers on them; the cuckoos are sounding sweet notes on all sides; the bees are humming on all sides.

13. The spring, the lion of the seasons, has burst asunder the fierce elephant, the winter season, by its sharp nails, as testified by the budding of Pal\^asa flowers.

14-18. O Brahman! See how beautiful and excellent has become this hermitage with the presence of the Goddess Spring Laksmi? O Devars\^i! The Rakt\^a\'soka flower is the palm of her hand; Kim\'suka flower, her excellent feet; N\^il\^asoka flowers, her black hairs on her head, the full-blown lotuses, her eyes; the bel fruits, her breast; the jolly Kunda flowers, her teeth; Manjari, her beautiful ears; red Bandhu flowers, her lips; Sindhub\^ara, her wonderful nails; the peacocks, her ornaments; the sounds of S\^arasa birds, the jingling of her feet ornaments; the wreaths of flowers, her waist ornaments; the mad gooses, her gait; Kadamba flower's filaments, her hairs on her body; O best of ascetics! With all these, the Vasanta Laksm\^i has assumed a wonderful nice appearance.

19. Why has this occurred untimely? Think over it; O Devarsi! I am struck with wonder; surely this is obstructive of our penances.

20. Hear! There the Apsar\^as are singing sweetly the song, tending to destroy our Tapasy\^as; it seems, these are the means, no doubt, adopted by Indra to pollute our Tapasy\^a.

21. Why is this spring season now generating our pleasures? It is clear that that Indra, the enemy of the Asuras, is become afraid of our Tapasy\^a and is creating these obstructions to disturb our asceticisms.

22. Lo! The cool, odorous, and pleasant breezes are blowing; no other cause can be traced than the wicked deed of Indra.

23. When the best of the Br\^ahmins, the Deva N\^ar\^ayana was addressing thus, the whole host of Cupid became visible before their sight.

24. And the two Risis were very much surprised on seeing them.

25-27. They saw near to them the Cupid with his attendants Menak\^a, Rambh\^a, Tilottam\^a, Puspagandh\^a, Suke\'s\^i, Mah\^a\'svet\^a, Manoram\^a, Pramodvar\^a, Ghrit\^ach\^i, Ch\^aruh\^asin\^i, the expert in music, Chandra Prabh\^a, the cuckoo voiced Som\^a, the lotus eyed Vidyunm\^al\^a, K\^anchana malin\^i, and others.

28. Eight thousand and five hundred Apsar\^as and long multitudes of the hosts of Cupid, the Munis saw and were surprised.

29. Then those prostitutes of the Devas, dressed with their heavenly ornaments and the heavenly flowers, appeared before the Munis and bowed down their heads on the ground.

30. The Apsar\^as began their enchanting songs, exciting much passion and rarely heard or seen in this world.

31-32. The two Munis Bhagav\^an Visnu-like Nara N\^ar\^ayana were pleased with their music and addressed them thus :-- O thin waisted good looking Apsar\^as! You have come here as guests, I see, from your Heavenly world. Stay here in peace and all comfort; we will gladly serve you as your hosts.

33-34. Vy\^asa said :-- O king! The two Munis, thinking that Indra has sent these Apsar\^as to obstruct their Tapasy\^a, were filled with egoism and determined to create, out of their strength of Tapasy\^a a new Apsar\^a, who would be very much more beautiful and possessing far more heavenly graces than the present ones, who are ordinary looking and clumsy in their behaviours.

35. And the Munis, by clapping or striking their thighs, instantly created a woman, exquisitely beautiful in all respects.

36. This good looking woman was named Urva\'s\^i, since she was produced from the thighs. And all the other Apsar\^as present there were very much thunderstruck on seeing that Urva\'s\^i.

37. Then the Muni N\^ar\^ayana easily created as many women as there were Apsar\^as to serve them.

38. The just produced Apsar\^as brought with them all sorts of offering in their hands, and, singing and smiling, came before the Munis and with clasped hands bowed down before them.

39. The heavenly damsels sent by Indra, though enchanting to others were themselves now bewildered on beholding Urva\'si, beautiful in all respects and produced out of the Tapasy\^a of the Munis; and their hairs over the bodies stood on their ends. Then they tried to make their faces as beautiful as they could and began to address the Munis thus :--

40. O Munis! We are ignorant girls; how can we praise you and the greatness of your Tapasy\^a and at your steadiness. Oh! There is no one in this Universe, that is not burnt with the passion by the arrows of our sharp eyesight? But there is no trace of mental disturbance and defilement in you; Oh! Wonderful is your greatness, indeed!

41. We are convinced that both of you are the Amsas of Visnu and that your treasures are your incessant peace and control of mind. We have come here not to serve you but to hinder you in your penances, that we may fulfill the desires of Indra.

42. By what good luck of ours we have got a sight of thee, we do not know; we do not know also what merits we did? We have committed great offence to you; still you have not cursed us. You have considered us as those of your own family and have pardoned us. Therefore our minds are free from sorrow and anxiety. Much praise be to your forgiveness! Wise saints do not squander away their occult powers, derived from austerities, in trivial ways like cursing others.

43. Vy\^asa said :-- Very pleased were those two Dharma's sons, the two Maharsis, self controlled and desireless, to hear these words of those godly behaved heavenly damsels; they then spoke to the damsels, blazing with the fire of their Tapas.

44-45. Nara and N\^ar\^ayana said :-- O Damsels! We are pleased with you; better ask from us your desired boons; we will instantly grant them to you. You better take with you to your Heaven this beautiful eyed Urva\'s\^i, born of our thighs as a present to your Deva R\^aja, the Indra.

46. Now peace be to all the Devas; you better go to your own places; do not, in future, disturb the Tapasy\^a of others.

47. The damsels said :-- Where will we go now? We have reached your lotus feet through our devotion, and our joy knows no bounds; O N\^ar\^ayana the Supreme amongst the Gods!

48. O Lord! O Madhusûdana! O Lotus-eyed! If Thou art pleased with us and dost want to give us our desired boons, we disclose to you our wished for object.

49. O Lord of the Devas! Thou art the Lord of the world; so beest Thou the Lord of us. O Destroyer of the foes! We will gladly put ourselves at the service of your feet.

50. Let those sixteen hundred and fifty beautiful-eyed damsels including Urva\'s\^i, that are your creation and that are now existing here, let them go unto Heaven by your command.

51. And we, the sixteen hundred and fifty damsels that have come before, may be allowed to remain here at your service.

52. O M\^adhava! You are the Lord of the Devas; be true to your word and give us our desires. Those seers, the Munis, who know what is Dharma, declare that it is sin, equivalent to murder, to destroy the hopes of those women that are struck with passion.

53. We are very fortunate to come here from Heaven and we are filled with extreme love for you, O Deve\'sa! You are the Lord of the world; you can do all things; therefore do not leave us.

54. N\^ar\^ayana said :-- O thin bodied damsels! I am practising at this place the tapasy\^a for full one thousand years, controlling my passions; how can I now break it by engaging myself to enjoy sensual things.

55. I have no inclination to indulge in sexual pleasures, tending to destroy the Highest Bliss as well as the Highest Dharma. What intelligent person will like to indulge like a beast in sensual pleasures.

56-57. The Apsar\^as said :-- Of the five senses; sound, etc., the pleasures attained through the sensation of touch are excellent, and are reckoned as the source of Bliss; no other pleasures stand equal to it. Therefore do then fulfill our words, and enjoy incessantly this highest bliss and roam freely in this Gandham\^adan mountain.

58. If you like to go to Heaven, be pleased to know that there is no Superior Heaven to Gandham\^adan (the mountain like intoxicating happiness of the senses). Dost thou enjoy the highest bliss, the pleasant sexual intercourse with us, the heavenly damsels in this very beautiful and lovely place.

Thus ends the Sixth Chapter in the Fourth Book of \'Sr\^imad Dev\^i Bh\^agavatam, the Mah\^a Pur\^anam of 18,000 verses by Maharsi Veda Vy\^asa on the origin of Urva\'s\^i.