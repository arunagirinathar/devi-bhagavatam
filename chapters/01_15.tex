\chapter{On the dispassion of \'S\^uka and the instructions of Bhagavat\^i to Hari}

1-67. Hearing these words of Vy\^asa Deva \'S\^uka Deva said :-- O Father! I do not like at all to take to a householder's life; as I see clearly that it fastens men, as a cord fastens animals, and is a source of incessant pain. O Father! Where can you expect happiness from a householder who is always loaded with anxiety how and whence to get wealth? Those, who have greed for wealth, oppress their poor relatives, even; and extort money. Even one who is the lord of the three worlds, who

is their Indra, he also is not so happy as a beggar, that has no desires. See, then, who else can be happy in this world? Whenever an ascetic is seen to practise severe asceticism, Indra, the lord of the Devas becomes anxious and sorry, and raises various obstacles in his way. See also that Brahm\^a is not happy with his big sams\^ara (his creation which is his house). Bhagav\^an Visnu, though He has got His beautiful Kamal\^a, the presiding Deity of all wealth and prosperity, is always suffering, since He is incessantly engaged in fighting with the Asuras; and though He is the husband of Laksm\^i and full of prosperity, He practises, almost, every now and then, terrible asceticism with great care and earnestness. So who else is there, who is possessed of constant happiness? I know also Bhagav\^an \'sankara, too, suffers incessant troubles and has to fight against the Daity\^as. So, then, O Father: how can a poor householder be happy when the rich householder cannot sleep happily, with his constant care for wealth. O highly fortunate one! Knowing full well this truth of the world, why are you plunging me, your son, in this terrible Sams\^ara, full of pains and agonies.

O Father! What shall I say to you about the miseries of the world! There is pain in birth, pain in old age, pain in death, and pain in the life in the womb full of urines and faeces; but the pain, arising from desire and greed, is more terrible than all the pains mentioned above; and then, the pains experienced while asking for them are greater than the pangs of death. Alas! There is no other way for the Brahm\^ans to earn their livelihood than to accept gifts from others. Therefore the Brahm\^ans have to suffer daily death-like pangs in having to wait in expectation from others; can there be anything more regrettable than this? The Brahm\^anas, studying all the Vedas and Dharma \'s\^astras and acquiring wisdom, have got at last to go to the rich and praise them (in expectation of some money) carefully. O Father! if one does not become a householder, then what care is there to feed one's own belly? If there be contentment in the mind, any how the belly can be filled with leaves, roots and fruits; but if there be wife, sons and grandsons and many dependent relatives, then to feed them all, much trouble and anxiety are experienced. So how can you expect, O Father! perfect happiness in the world? So teach me, O Father! the \'S\^astras on Yoga and eternal truth that will give perfect happiness; no advice in karma k\^anda (the series of actions) will bring me pleasure. Now advise me how the karmas can be exhausted; how the root of the three sorts of karmas, Sanchita, Pr\^arabdha, and Vartam\^ana, giving torments of birth, death, etc., the Avidy\^a, the great ignorance, can be destroyed? The fools do not understand how the women suck the blood out of persons like leeches, for they get themselves deluded by their gestures and postures! The lady of the house, whom the people

call k\^ant\^a, the beautiful one, steal away the semen virile, the strength and energy in the way of giving them happiness as sexual intercourse, and their minds and wealth and everything by their crooked love conversations; so see what greater thief can there be than a woman? In my opinion, those that are ignorant are certainly deluded by the Creator; they accept wife to destroy their own pleasure of happiness. They can never understand that the women can never be the source of pleasure; they are the source of all miseries. Hearing these words of \'S\^uka, Vy\^asa became merged in the deep sea of cares and anxieties, thinking what to do then. The incessant tears of pain flowed from his eyes; his whole body began to shiver and his mind became too much worried.

Seeing this distressed and sorrowful state of his father, \'s\^ukdeva, with eyes full of wonder, said :-- Oh! What a power has M\^ay\^a got?

Oh!  He, whose words are accepted by all, with great love and care as equivalent to the Vedas, who is the author of the Ved\^anta Dar\'sana, and before whom nothing is veiled in ignorance, Oh! that greatest Pundit, the knower of all the Tattvas, is now deluded by M\^ay\^a? Oh! what is that M\^ay\^a who has been able to delude Vy\^asa Deva, the son of Satyavati, so skilled in the knowledge of Brahm\^a Vidy\^a; I also do not know how, with what great care, one is to practise S\^adhan\^a towards Her.

Alas! He who has composed eighteen Mah\^a Pur\^anas and the great Mah\^a Bh\^arata, who has divided the Vedas in four parts, the same Veda Vy\^as has today been deluded by the power of M\^ay\^a! What to speak of other persons! Oh! M\^ay\^a has deluded Brahm\^a, Visnu, Mahe\'svara and others and the whole universe; then who is there in the three worlds that is not fascinated by Her influence! I therefore, take refuge unto the Internal Governess, the Dev\^i Mah\^a M\^ay\^a. Oh! what wonderful power She wields? By her own M\^ayic power, She has kept God even under Her control, who is omniscient and the Controller of all. The Pundits, who know the Pur\^anas say, that Vy\^asa Deva is born of the part of Visnu; but, see the wonder, that he is today plunged in the sea of delusion like a merchant whose ship has been wrecked. Alas! How great is the wonderful power of M\^ay\^a! The all-knowing Vy\^asa is today under the control of M\^ay\^a and is weeping like an ordinary man! So I have come to the firm conclusion that the wise Pundits are incapable to surpass the strength of M\^ay\^a. What a great error arises through the power of M\^ay\^a! See! indeed!! who is he and who am I? What for we have come here? There is no certainty, nothing whatsoever, about that. And, see, also, how he has got the nice idea of ``father'' on his body and the idea of ``his son'' in my body, that are composed of five elements.

This is now quite evident to me that, when the Br\^ahmin Maharsi Krisna Dvaip\^ayan is weeping under the influence of M\^ay\^a, She is the strongest of all; even those who are skilled in the great M\^ay\^a fall under Her prey.

Then \'S\^uka Deva bowed down mentally to the Dev\^i, Mah\^a M\^ay\^a, who is the Creatrix of Brahm\^a and the other Devas and who is the Controller of them all; and then began to speak the following auspicious words pregnant with reason, to his father Vy\^asa Deva, greatly distressed and plunged in the sea of sorrows :-- ``O Father! you are exceedingly fortunate, for you are the son of the high souled Par\^a\'sara and you yourself are the instructor of real truth, the tattva jn\^ana, to all persons; so, O Lord! why are you giving vent to sorrows, like an ordinary bewildered man? O Mah\^abh\^aga! why are you plunging yourself in this great error, though you are a high souled personage! See, it is quite true that now I am born as your son; but this I don't know what relation existed between you and me in my previous birth? So, O highly intelligent one! Open your eyes of wisdom, and be patient; do not throw yourself, in vain, in the sea of sorrows. All this universe is like a net of delusion; knowing this, abandon all your grief; why are you feeling yourself so much weak and distressed, for your attachment towards your son? Hunger is satisfied by eating something, and thirst is satisfied by the drinking of water; hunger is not satisfied by seeing the son. So the organ of scent is satisfied by smelling sweet scents; and the organ of hearing is satisfied by hearing sweet music; and when thirst arises to enjoy women, that is satisfied only by sexual intercourse; but what satisfaction can a son give? So what shall I do to you by remaining as your son? The son, in fact, is not the cause of any satisfaction to one's self. For this reason, in ancient days, the poor Brahmin Ajigarta gave his son to the king Harischandra, for necessary equivalent price in money, who wanted a man for his sacrifice where human beings are to be sacrificed as victims. In fact, those things that are urgently required as necessities give happiness; and all these articles can be obtained by wealth; so if you want to enjoy happiness, then earn money; of what use shall I be to you as your son? O Muni! you can see subtle things and you are greatly intelligent; so I pray to you, to look upon me as your son and open my eyes of wisdom, that I can be free for ever from this womb of birth. O Sinless one! To get a human birth in this land of Karma (in India) is very difficult; again to get a Br\^ahmin birth is extremely rare; so when I have got this so very rare birth, why shall I spend my time in vain? O Father! Though I have served many spiritual teachers, fraught with wisdom, for many years, yet the firm idea ``I am, as it were, bound up in this net of Sams\^ara'' the notion covered with dark darkness of ignorance, caused by desires, this net of Sams\^ara does not leave me.

When the son \'S\^uka Deva of extraordinary power and intelligence spoke thus, Vy\^asa saw that his son was strongly inclined to take to the four \^a\'sram, that of Sanny\^asa and spoke thus :-- O Son! If your mind has become so, then read Bh\^agavat Pur\^ana, composed by me, highly auspicious, voluminous, and the second Vedas.

In this you have the chapters on Creation (Sarga) and secondary creation (upa sarga), etc., the five characteristics as in other Pur\^anas and it is sub-divided into twelve Skandhas. Hearing of this Bh\^agavata brings up to the mind that Brahm\^a alone is real and all the universe is unreal and knowledge both intuitive and indirect springs up. For this very reason, the Bh\^agavata treatise is considered as the ornament of the Pur\^anas. Therefore, O highly intelligent one! you better study the Pur\^ana. O Child! In days of yore, at the end of a Kalpa, Bhagav\^an Hari was lying, as a small child on a floating leaf of a banyan tree, and was thinking thus :-- ``Who is the Intelligent One that has created me a small child? What is His object? Of what stuff am I made of? and how am I created? whence can I know all this?'' At this moment the Dev\^i Bhagavat\^i Who is all chaitanya, seeing the high-souled Bhagav\^an Hari musing thus, spoke out in the form of a celestial voice in the following half-stanza:-- ``All this that is seen is I Myself; there is existent nothing other that is eternal.'' Bhagav\^an Visnu, then, began to think deeply the above celestial voice :-- ``Who has uttered this word, pregnant of truth, to me? How shall I come to know the speaker, whether that is female, male or a hermaphrodite?'' Pondering over this for a long time, when he could not come to a definite conclusion, he began to repeat (make japam) frequently that word of Bhagavat\^i with a whole heart. When Hari, lying on a banyan tree leaf, became very anxious to know what the above words implied, then the all-auspicious Dev\^i Bhagavat\^i with a beautiful face, calm and quiet appearance, appeared before Bhagav\^an Visnu, of unrivalled splendour, in the form of Mah\^a Laksm\^i, who is all of Sattva Guna, surrounded by Her Vibh\^utis, Her manifestations of attendents, Her smiling companions of the same age, decked with ornaments, and wearing divine clothings, and holding each in their four divine hands, conch shell, disc, club, and lotus.

The lotus eyed Visnu was very much surprised to see that beautiful Dev\^i, standing without anything to rest on that water; He saw that on four sides of the Dev\^i, were staying Rati, Bh\^uti, Buddhi, Mati, K\^irti, Smriti, Dhriti, \'sraddh\^a, Medh\^a, Svadh\^a, Sv\^ah\^a, Kshudh\^a, Nidr\^a, Day\^a, Gati, Tusti, Pusti, Ksam\^a, Lajj\^a, Jrimbh\^a Tandr\^a and other personified forces, each possessing a clear distinct form, and endowed with a clear distinct feeling. In the hands of them all were divine weapons; on their necks, necklaces and garlands of

Mand\^ara flowers; and all the limbs of their bodies were decorated with divine ornaments. Seeing in that one mass of ocean the Dev\^i Laksm\^i and Her \'saktis, Bhagav\^an Jan\^ardan, the soul of all, became greatly astonished and thought within Himself thus :-- ``What is this? Is this M\^ay\^a that I am witnessing? Whence have appeared these women? and whence have I come here, lying on this banyan leaf? How has the banyan tree come to existence in this one mass of ocean? And who is it, that has placed me here in the form of a child? Is this my Mother? Or is this some M\^ay\^a that can create impossible things?

Why has She made Herself manifest before me now? Or is there some hidden motive that She has appeared thus? What ought I to do now? Or shall I go to some other place? or shall I continue remaining here in this form of the child, silent and with vigilance.

Thus ends the fifteenth chapter of the 1st Skandha on the dispassion of \'s\^uka and the instructions of Bhagavat\^i to Hari in the Mah\^apur\^ana \'Sr\^i Mad Dev\^i Bh\^agavatam of 18,000 verses by Maharsi Veda Vy\^asa.