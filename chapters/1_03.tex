\chapter{On praising the Purânas and on each Vyâsa of every Dvâpara Yuga}

 

p. 6

 

1-11. Sûta said :-- “O best of the Munis! I am now telling you the names of the Purânas, etc., exactly as 1 have heard from Veda Vyâsa, the son of Satyavati; listen.

 

The Purâna beginning with "ma" are two in number; those beginning with “bha” are two; those beginning with “bra" are three; those beginning with "va” are four; those beginning respectively with “A”, “na”, “pa”, “Ling”, “ga”, “kû” and “Ska” are one each and “ma” means Matsya Purâna, Mârkandeya Purâna; “Bha” signifies Bhavisya, Bhâgavat Purânas; “Bra” signifies Brahmâ, Brahmânda and Brahmâvaivarta Purânas; “va” signifies Vâman, Vayu, Visnu and Varaha Purânas; “A” signifies Agni Purâna; “Na” signifies Narada Purâna; “Pa” signifies Padma Purâna; “Ling” signifies Linga Purânam; “Ga” signifies Govinda Purânam; Kû signifies Kurma Purâna and “Ska” signifies Skanda Purânam. These are the eighteen Purânas. O Saunaka! In the Matsya Purâna there are fourteen thousand slokas; in the wonderfully varied Markandeya Purânam there are nine thousand slokas. In the Bhavisya Purâna fourteen thousand and five hundred slokas are counted by the Munis, the seers of truth. In the holy Bhâgavata there are eighteen thousand S’lokas; in the Brahmâ Purâna there are Ajuta (ten thousand) S’lokas. In the Brahmânda Purâna there are twelve thousand one hundred S’lokas; in the Brahmâ Vaivarta Purânam there are eighteen thousand S’lokas. In the Vaman Purâna there are Ajuta (ten thousand) S’lokas; in the Vayu Purânam there are twenty-four thousand and six hundred S’lokas; in the greatly wonderful Visnu Purâna there are twenty-three thousand S’lokas; in the Agni Purânam there are sixteen thousand S’lokas; in the Brihat Narada Purânam, there are twenty-five thousand S’lokas, in the big Padma Purâna there are fifty-five thousand s'lokas; in the voluminous Linga Purâna eleven thousand s’lokas exist; in the Garuda Purânam spoken by Hari nineteen thousand s'lokas exist; iu the Kurma Purâna, seventeen thousand s'lokas exist and in the greatly wonderful Skanda Purâna there are eighty-one thousand s'lokas, O sinless Risis! Thus I have described

 

p. 7

 

to you the names of all the Purânas and the number of verses contained in them. Now hear about the Upa Purânas.

 

12-17. The first is the Upapurâna narrated by Sanat Kumâra; next comes Narasimha Purâna; then Naradiya Purâna, S’iva Purâna, Purâna narrated by Durvasa, Kapila Purâna, Manava Purâna, Aus’anasa Purâna, Varuna Purâna. Kalika Purâna, Samva Purâna, Nandi Kes’wara Purâna, Saura Purâna, Purâna spoken by Parâs’ara, Âditya Purâna, Mahesvara Purâna, Bhâgavata and Vasistha Purâna. These Upa Purânas are described by the Mahatmas.

 

After compiling the eighteen Purânas, Veda Vyâsa, the son of Satyavati composed Mahabharata, that has no rival, out of these Purânas.

 

18-24. At every Manvantara, in each Dvâpara Yuga, Veda Vyâsa expounds the Purânas duly to preserve the religion. Veda Vyâsa is no other person than Visnu Himself; He, in the form of Veda Vyâsa, divides the (one) Veda into four parts, in every Dvâpara Yuga, for the good of the world. The Brahmânas of the Kali age are shortlived and their intellect (Buddhi) is not sharp; they cannot realise the meaning after studying the Vedas; knowing this in every Dvâpara Yuga Bhagavân expounds the holy Purâna Samhitas. The more so because women, S’udras, and the lower Dvijas are not entitled to hear the Vedas; for their good, the Purânas have been composed. Tne present auspicious Manvantara is Vaivasvata; it is the seventh in due order; and the son of Satyavati, the best of the knowers of Dharma, is the Veda Vyâsa of the 28th Dvâpara Yuga of this seventh Manvantara. He is my Guru; in the next Dvâpara, Yuga Asvatthama, the son of Drona will be the Veda Vyâsa. Twenty-seven Veda Vyâsas had expired and they duly compiled each their own Purâna Samhitas in their own Dvâpara Yugas.

 

25-35. The Risis said :-- “O highly fortunate Sûta! kindly describe to us the names of the previous Veda Vyâsas, the reciters of the Purânas in the Dvâpara Yugas.

 

Sûta said :-- In the first Dvâpara, Brahmâ Himself divided the Vedas; in the second Dvâpara, the first Prajapati Vyâsa did the same; so S’akra, in the third, Brihaspati, in the fourth, Surya in the fifth; Yama, in the sixth, Indra, in the seventh, Vasistha, in the eighth; Sarasvata Risi in the ninth, Tridhama, in the tenth; Trivrisa, in the eleventh, Bharadvâja, in the twelfth; Antariksa, in the thirteenth; Dharma, in the fourteenth; Evaruni in the fifteenth; Dhananjaya, in the sixteenth; Medhatithi in tba seventeenth; Vrati, in the eighteenth; Atri, in the nineteenth; Gautama in the twentieth, Uttama, whose soul was fixed on Hari, in the twenty-first, Vâjasravâ Vena, in the twenty second; his family descendant Soma

 

p. 8

 

iu the twenty-third; Trinavindu, in the twenty-fourth; Bhârgava, in the twenty-fifth; Sakti, in the twenty-sixth, Jâtûkarnya in the twenty-seventh and Krisna Dvaipâyana became the twenty-eighth Veda Vyâs in the Dvâpara Yugas. Thus I have spoken of the 28 Veda Vyâsas, as I heard. 1 have heard the holy S’rimad Bhâgavat from the month of Krisna Dvaipayana. This removes all troubles, yields all desires, and gives Moksa and is full of the meanings of the Vedas. This treatise contains the essence of all the S’astras and is dear always to the Mamuksas (those who want Moksa or liberation).

 

36-43. O best Munis! Thus, compiling the Purânas Veda Vyâsa thought this Purâna to be the best; so (without teaching it to other persons) he settled that his own son the high-sould S’uka Deva born of the dry woods used for kindling fire (excited by attrition), having no passion for the worldly things, would be the fit student to be taught this Purâna and therefore taught him; at that time I was a fellow student along with S’aka Deva and I heard every thing from the mouth of Vyâsa Deva and realised th« secret meanings thereof. This has happened through the grace of the merciful Guru Veda Vyâsa.

 

Here ends the Third Chapter of S’rimad Devi Bhâgavatam on praising the Purânas and on each Vyâsa of every Dvâpara Yuga.