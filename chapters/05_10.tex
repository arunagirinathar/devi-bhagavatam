\chapter{On the messenger's news to Mahisa}

1-16. Vy\^asa said :-- The Mah\^a M\^ay\^a, that Excellent Lady, hearing thus the words of the prime minister of Mahisa, laughed and spoke with a voice, deep like that of a cloud, thus :-- O Minister-in-chief! Know Me as the Mother of the gods; my name is Mah\^a Laksm\^i. It is I that destroy all the Daityas. I am requested by all the Devas to kill D\^anava Mahisa; they have been oppressed and deprived of their share of Yajña offerings. Therefore I have come here today alone, without any army to take away his life. O Good One! I am pleased with your sweet words of welcome, in showing me marks of respect. Had you not behaved thus, I would have certainly burnt you to ashes by my fiery sight, which is the universal conflagration at the break up of the world. O Minister! Who is there that gets not pleased with sweet words! Go you to Mahisa and speak to him the following words of mine ``O Villain! Go down to P\^at\^ala (the nether regions) at once if you have any desire to live. Otherwise, I will slay you, the wicked one, in the battle-field; you will have to go to the house of Death, pierced by my mass of arrows. O Stupid One! Know that this is merely kindness shown unto you, that I have told you to go soon to P\^at\^ala and that the Devas get possession of their Heaven, with no delay. O One of weak intellect! Therefore dost Thou leave possession of this sea-girt earth and go alone without any delay to P\^at\^ala, before my arrows are shot at you. O Asura! Or if you desire to fight, then come at once with your powerful warriors; I will destroy all of them. O One of dull intellect! I will kill you in battle, just as I killed before in yugas after yugas countless Asuras like you. O Passionate creature! Better shew that your efforts in holding weapons have been crowned with success by your being engaged in battle against Me; otherwise they will all be useless. O Stupid! You thought that you would be vulnerable alone to women hence you oppressed the Devas entitled to worship; O wicked one! No longer show your pride on the strength of your getting the boon from Brahm\^a, that you would be vulnerable only to the females. Thinking it advisable to observe the words of the Creator, I have assumed this incomparable Eternal Female appearance and I have come here to slay you, O wicked one! O stupid one! If you have any desire for your life, then quit this Heaven and go to P\^at\^ala, infested with snakes, or anywhere else you like.''

17-28. Vy\^asa said :-- Hearing these words of the Dev\^i, that minister, surrounded by forces, replied in reasonable words thus :-- ``O Dev\^i! You are speaking in words befitting a woman and puffed up with pride. You are a woman; the lord of the Daityas is a hero; how can a battle be engaged between you two. It seems to me impossible. Your body is delicate, a girl in full youth; especially you are alone and Mahisa is of huge body and powerful; so the fight comes next to impossibility. He has elephants, horses, chariots, infantry, etc., and countless soldiers all armed with weapons. Therefore, O Beautiful One! He will find no difficulty in killing you in battle as an elephant finds no difficulty in treading over the M\^alati flowers. Rather, if I utter anything harsh to you, that would go against the sentiment of love, with you; therefore I cannot speak rudely to you out of my fear not to interrupt the above feeling. True, that our king is an enemy of the gods; but be has become extremely devoted to you. Therefore it is wise to speak words full of conciliation or generosity. Were it otherwise, I would have shot arrows at you and would have killed you in as much as you have thus boasted in vain and spoken so dire a falsehood, resting merely on the strength of your youthful pride and cleverness. My master has become fascinated on hearing your extraordinary beauty hardly to be seen in this world; it therefore behoves me to speak sweet words to you for the sake of pleasing my master. O Large-eyed! This kingdom and the wealth thereof are all yours; in fact, Mahisa will be your obedient servant; therefore, better forsake your anger, leading to your death; and cultivate friendship with him. O Sweet Smiling One! I am falling at your feet; you better go to him and become at once queen-consort. O Handsome Woman! No sooner you become the queen of Mahisa than you will get at once all the pure wealth of the three worlds and the unbounding happiness of this world.''

29-45. The Dev\^i said :-- ``Minister! I now speak what is pregnant with goodness and wisdom to you, according to the rules of the \'S\^astras, keeping in view also the cleverness that you have shown in using your words. Now I come to understand from your talk, that you are the chief secretary of Mahisa; and therefore your nature and intelligence are like those of a beast. And how can he be intelligent, whose ministership is occupied by a man of your nature! Nature has ordained connection between two persons of like nature. O Stupid One! Did you think a little beforehand the meaning of your words when you told me of my feminine nature? Though I am not apparently a man, yet my nature is that of the Highest Purusa (Man); I shew myself simply in a feminine form. Your master asked before from Brahm\^a that he would prefer death, if possible, at the hands of a woman ; therefore, I consider him quite illiterate and ignorant of the sentiment, worthy of a hero. Because to die at the hands of a woman is very painful to one who is a hero; and this is gladly welcome to one who is a hermaphrodite. Now see that your master Mahisa has shown his intelligence, when he courted his death from the hands of a woman. For that very reason, I have come here in the shape of a woman to effect my purpose; why shall I fear, then, to hear your words, contradictory to those of the \'S\^astras. When Fate goes against any one, a grass comes like a thunderbolt; and when fate goes in favour of anyone, a thunderbolt becomes as soft as a bundle of cotton. What does it avail even when one possesses an extensive army or various weapons in abundance, taking shelter in a wide extending fort? What will his soldiers do to him, whose death has come close at hand? Whenever, in due time, the connection of the J\^iva (the human soul) with this body is brought about, then his pleasures, pains and death are written. Know this as certain, very certain, that death will come to him in the manner as written by the hands of Fate; it will never be otherwise. As the birth and death of Brahm\^a and other gods are ordained, your death has been similarly ordained; no, there is no need of taking the example further than this. Those who are tied up by the hands of death are surely fools and of extremely blunt intellect, if they think simply on the strength of their getting some boons ``that they would never die.'' Therefore go quickly to your king and speak to him what I have said; you will then surely obey what he commands you to do. If he wants his life, he, with his retinue, would at once go down to P\^at\^ala; let Indra and the other Devas get possession of the Heavens and their share of Yajñas. If he holds a contrary opinion, let him be eager to go to the house of Death and come and fight with Me. If he thinks that Visnu and the other Devas have fled from the battle-fields, he has nothing to boast of; for he has not shewn his manliness at all even then; for his victory is solely due to his having got the boon from Brahm\^a.''

46-52. Vy\^asa said :-- Hearing these words of the Dev\^i, the D\^anava began to think whether I ought to fight or to go to Mahisa? The King has become very enamoured and has sent me hither to negotiate for marriage; how then will I be able to go to him if I make this affair unpleasant and interrupted in the middle in its course of harmony. Now it is wise for me to go to the King without fighting; let me then go as early as possible in this way and inform him about this whole affair. The King is exceptionally intelligent and experienced; he will consult with his other experienced ministers and do what is best. Therefore I ought not to fight here rashly; for victory or defeat would alike be distasteful to my monarch. Whether this Lady kills me, or I kill this Lady, the king will be angry in either case. I will therefore go now to the king and tell him what the Dev\^i has said; he will do whatever he likes.

53-66. Vy\^asa said :-- Thus that intelligent son of the minister argued and went to the king. Then, bowing down before him, he began to say thus :-- O King! That excellent woman, fascinating to the world, the beautiful Dev\^i is sitting on a lion with weapons in all her eighteen hands. O King! I told her ``O Beautiful Lady! Be attached to Mahis\^asura; you will become, then, the queen-consort of the king, the lord of the three worlds. You will certainly then be his queen-consort; he will pass his life, ever obedient to you like an obedient servant. O Beautiful One! If you choose to make Mahisa your husband, you will become fortunate amongst women and will enjoy ever all the wealth of the three worlds.'' Hearing my these words, that large-eyed woman, puffed up with egoism, laughed a little and said thus :-- ``Your king is born of a buffalo and is the worst of brutes; I will sacrifice him before the Dev\^i for the benefit of the gods. Is there any woman in this world so stupid as to select Mahisa as her husband? O You stupid! Can a woman like me ever indulge in bestial sentiments! A female buffalo has got horns; she, being excited with passion, may select your Mahisa with horns as her husband and come to him bellowing. I am not stupid nor like her so as to make him my husband. O Villain! I will fight and destroy the enemies of the gods in the battle-field. Or if he desires to live, let him flee to P\^at\^ala. O King! Hearing those rough words uttered by Her in a moment of madness, I have come to you, thinking also how to redress this wrong. O King! Only I feared not to interrupt in your love sentiment; and therefore I did not fight with Her; especially, without Your command, how can I engage myself in useless excitement? O Lord of the Earth! That handsome woman rests maddened on Her own strength; I do not know what is in the womb of future or whatever is destined to happen, will surely come to pass. You are the sole master in this matter; I will do whatever you order me. The matter is very difficult to be reflected upon; whether it is better to fight or it is better to fly away, I cannot say definitely.''

Here ends the Tenth Chapter of the Fifth Skandha on the messenger's news to Mahisa, in \'Sr\^i Mad Dev\^i Bh\^agavatam the Mah\^a Pur\^anam, of 18,000 verses by Maharsi Veda Vy\^asa.