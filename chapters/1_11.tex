On the birth of Budha

 

p. 36

 

1-86. The Risis said :-- “O Sûta! Who is that King Pururavâ? and who is the Deva girl Urvas'î? And how did that high-souled King Pururavâ come into trouble? O son of Lomaharsana! Kindly describe now all these to me. We are now desirous to hear sweet words from the lotus of your mouth. O Sûta! Your words are sweeter and more full of juice than nectar even; so we are not satiated by hearing them as gods are not satiated with the drink of nectar.”

 

Hearing this Sûta said :-- O Munis! I am now telling you, as far as my intelligence goes, what I heard from the mouth of S'rî Vyâsa. Now hear that beautiful divine incident.

 

Once on a time the exceedingly beautiful dear wife of Brihaspati, named Târâ, full of youth and beauty, of the most beautiful limbs and intoxicated with pride went to the house of Chandra Deva (the Moon), the yajamân (Employer of priest Brihaspati at any sacrifice) of Brihaspati. Seeing Târâ of beautiful face like Moon, the Moon became very passionate; Târâ also fell amorous at the sight of Moon. Thus both of them became very much passionately attached to each other. Then they, the Moon and Târâ, being smitten by the arrows of Cupid and intoxicated with amorous passions began their sexual intercourse with reciprocal feelings of passionate love. Some days passed in this state. Brihaspati, then, being distressed with the pang of separation from his wife, sent his pupil to bring back Târâ; but Târâ was then submissive of Chandra and therefore refused to come. Brihaspati sent over and over again his pupil and when Chandra Deva returned the messenger, Brihaspati became very angry and went personally to Chandra's house and spoke angrily to the Moon who was puffed up with arrogance and somewhat smiling :-- “O Moon! Why are you committing this vicious act, contrary to Dharma? Why are you keeping my beautiful wife in your house? I am your Guru; you are my client; O stupid! why are you enjoying your Guru's wife and keeping her in custody in your house? Do you not know that he who kills a Brâhman, who steals gold, who drinks, who goes to wife of one's Guru are Mahâpâtakis (great sinners) and those who keep company of these are the fifth Mahâpâtakis? Therefore if you had enjoyed my wife, you are exceedingly vicious, blameable and a Mahâpâtaki (great sinner); you are not fit to live amongst the Devas. O wicked

 

p. 37

 

one! Now I say that even now you better forsake Târâ, who is of a blue colour and whose look is askance; I won't go from here without having Târâ. And if you do not give back Târâ, then you are certainly with her and undoubtedly I will curse you. When Brihaspati said so, Chandra, the husband of Rohinî, spoke to his Guru Deva, who was very angry, sorry and afflicted at the separation from his beloved wife, thus :-- In this world, the Brâhmins that know the Dharma S'âstras, that are devoid of anger, are fit to be worshipped; and those that are not so, they are objects of disrespect and not to be worshipped by all for their anger. O sinless one! The beautiful one will surely go to your house; what harm is there to you, if she waits here for a few days?

 

She is staying here of her own accord to enjoy pleasures and will go back of her own will. One word more :-- You gave out before this opinion of the Dharma S'âstras that as a Brâhman though guilty of committing vicious deeds, becomes pure again by the practice of Karmas as enjoined in the Vedas, so a woman, too, though guilty of adultery, becomes pure again when she is again in the period of her menstruation. At these words of Chandra Deva, Brihaspati, the Guru of the Devas, became exceedingly sorry and anxious and went back immediately to his own house, with a grievous heart, full of amorous feelings. After staying in his own house for some days Brihaspati, worn out with anxiety, came again quickly to the house of Chandra; but, no sooner he was to enter the gate, he was stopped by the door-keepers; he became very angry and stopped at the gate way. And when he saw that Chandra did not make his appearance, he became exceedingly angry and thought :--  “Oh! What a wonder is this? this irreligious Chandra, being my disciple, has done this vicious act and took by violence the wife of his Guru, who is reckoned as the mother; and I will now teach him a good lesson.”

 

Standing on the entrance gate way Brihaspati began to speak aloud :-- “O stupid, vicious, vilest of the Devas! Why do you now sleep in your inner room? Do return quickly my wife; else I will curse you at once. In case you do not bring me back my wife at once, I will now reduce you to ashes.” Hearing these foul words of Brihaspati, Chandra Deva the king of the Dvijas, quickly came out of the house and said smiling :-- “O Brâhmin! Why are you spending your stock of words for nothing? That all-auspicious lady, of a blue colour and looking askance, is not fit for you; therefore take another comparatively uglier woman for your use. Exceedingly youthful and lovely woman like her is not fit for a beggar's house. O stupid one! I see, you don't know anything about the Kâma S'âstra (the book that dwells

 

p. 38

 

on amorous passion); those wise men who are skilled in this S'âstra assign for the women, their lovers equivalent to their beauty in matters of amorous dealings. So, O stupid man! go away wherever you like. I won't give you back your wife. Do whatever lies in your power. I won't return your wife. When you have become passionate, your curse won't affect me in any way. This I say finally unto you :-- “O Guru! I will not give you back your wife; do what you wish.” Thus spoken by Chandra, Brihaspati became vary anxious and angry; he then went away quickly to the Indra's house.

 

Seeing Guru Deva morose and sorry, the very liberal minded Indra Deva worshipped him duly with pâdya, arghya, and Âchamanîya and asked :-- “O highly fortunate one! Why do you look so anxious? O great Muni! Why are you grievous and sorry? You are my Guru; is it that you are insulted by any one in my kingdom; say freely. All the Regents of the several quarters (the Lokapâlas) and all the Deva armies are under your command. Brahmâ, Visnu, and Mahes'var and other Devas are ready to lend you every assistance, no doubt. So say what is the cause of your anxiety?” Hearing these words of Indra, Brihaspati said :-- “Chandra has stolen my beautiful-eyed wife. I asked for her, again and again, but that wicked soul is not returning me my wife at all. O Lord of the Devas! What am I to do now? You are my help and guide. O S'atakratu! You are the lord of the Devas; therefore I pray to you with a very grievous heart, help me in this matter.” Hearing this, Indra said :-- “O knower of Dharma! Do not be sorry. O Suvrata! I am your servant. O highly intelligent one! Surely I will bring you back your wife. I will send a messenger and even then if Chandra, mad with pride, do not return your wife, I will wage war with him and fight against him, with all our Deva armies.” Thus consoling Brihaspati, Indra sent a very clever man, who was a good speaker and wonderful in his capabilities, to Chandra. The clever and wise messenger went to the Chandra Loka (the region of the Moon) and spoke to Chandra, the husband of Rohinî, thus :-- “O Mahâbhâga! Indra has sent me to you to communicate his message to you. So O intelligent one! I will tell you what he has ordered me; hear.” He said :-- “O highly fortunate one! You know well Dharma and Nîti S'âstra (the science of morals); the more so, because the virtuous Maharsi Atri is your father. Therefore, O Suvrata! You ought not to commit such blameable act. See, all beings should protect their own wives always without remaining idle to the best of their powers; therefore, no doubt, quarrels would ensue necessarily on that point. O Sudhânidhi! as far as this point of protecting one's wife is concerned, your Guru Deva ought also to do his best. You ought to consider all persons like your own self.

 

p. 39

 

O Sudhâkara! You have got twenty-eight exceedingly beautiful wives, who are the daughters of Daksa; why then do you desire to enjoy the wife of your Guru? The beautiful Apsarâs (celestial nymphs) Menakâ and others are always residing in the Heavens; you can enjoy them to your heart's content; leave off the wife of your Guru. In case any powerful man commits an unworthy act out of egoism, the illiterate ones would follow them; so the Dharma will decline. Therefore, O highly lucky one! Do such as does not lead, for nothing, quarrels amongst the gods and leave your Guru's wife, even beautiful.” Hearing these words from the messenger, the Moon (Chandra Deva) became somewhat angry and, making gestures, replied to the messenger, as if to Indra, thus :--

 

O mighty armed one! As you yourself are the lord of the Devas and the knower of Dharma, so your priest, too, has become like you; the head of both of you are the same. You will find many that can show their learning and give advice to others, but you will find always very rare such persons as will act themselves to their own advices when occasion arises and wants them to fulfil their own words. O Lord of the Devas! All the persons take the opinion of the S'âstras framed by Brihaspati then why the quarrel would ensue with me and the Devas when I an enjoying, according to his dictates, a woman who is herself willing? See also, that the rule in this world is might is right; all things go to the powerful man who can take by force; nothing falls to the lot of the weak; moreover this woman is mine and that woman is of another, this false notion comes to those whose brains are weak. When Târâ, is so much attached to me and is not at all attached to Brihaspati, the above rule applicable to me all the more; how then can I quit the lady so much attached to me, according to the laws of Dharma and the morals? You can see also, that happiness reigns in that family where the wife is according to the will of the husband; how, then, can the household happiness exist when the lady of the house is always dissatisfied? Therefore the household happiness of the Guru is impossible as Târâ is dissatified with Brihaspati since he enjoyed the wife of his younger brother Samvarta. Then the result comes to this, O thousand eyed one! How have you come to be thousand eyed! However that may be, you are the lord of the Devas; you can do whatever you like. O messenger! go and tell your lord of the Devas all that I have spoken; I will not return by any means that beautiful Târâ.

 

When Chandra spoke thus, the messenger went back to Indra and communicated to him all that Chandra Deva had spoken. Hearing this, Indra became angry and ordered all the Deva forces to be ready at once. Hearing this news of war, S'ukrâchârya, out of enmity to Brihaspati, went to

 

p. 40

 

Chandra and spoke thus :-- “O highly intelligent one! never return Târâ; in case if war ensues between you and Indra, I will help you by my Mantra-S'akti.” On the other hand, Bhagavân S'ankara, hearing of the vicious act of Chandra's, taking his Guru's wife, and knowing that S'ukrachârya was the enemy of Brihaspati, came to the assistance of the Devas. The great war, then, ensued between the Devas like the terrific war of Târakâsura; it continued for many years. Then the grandfather Brahma, seeing the great havoc in the lives of the Devas and Asuras, came there on his vâhan Hamsa, to secure peace and talked to Chandra :-- “Quit the Guru's wife; if not, I will call Visnu and destroy all of you party.” He also desisted the son of Bhrigu, S'ukrâcharya,  saying :-- “O highly intelligent one! why has this wicked idea possessed your mind? Is it due to the bad association?” Then S'ukrâchârya also told Chandra, the lord of the medicinal plants, not to wage war and said :-- “Better quit you now the Guru's wife. Your father Maharsi Atri has sent me to you for this purpose.” Chandra, then, hearing the strange words of S'ukrâchârya, returned to Brihaspati his wife Târâ, though she was not satisfied with him and became herself pregnant.

 

Brihaspati returned with joy to his house, accompanied by his wife; the Devas and Dânavas went away to their respective places. Brahmâ went to Brahmaloka and S'ankara went to Kailâs'a.

 

Brihaspati began to pass his time happily with his beautiful wife; Some days went away when the wife of Brihaspati, Târâ, gave birth to an all-auspicious son, having all the qualities of Chandra, on an auspicious day and under the influence of an auspicious star; seeing this new-born child, Brihaspati gladly performed the natal ceremonies of the child. Hearing that a son is born to him, Chandra sent a messenger to Brihaspati saying that “That the child is not his; but it is born out of the semen of mine; why, then, have you performed the natal ceremonies out of your own will?” Hearing these words of Chandra's messenger, Brihaspati said :-- “No, this child is mine, no doubt, as he resembles quite like me.” When Brihaspati said this, war again ensued. The Devas and Dânavas met each other again in battle field; and councils of war were held. Then, for the preservation of peace, Prajâpati Brahmâ went there; and before all desisted the Devas; and Dânavas, mad for war, and ready to fight against each other. Brahmâ, then, asked Târâ :-- “O auspicious one! say truly whose child is this? O beautiful one! if you say truly, then this war resulting in the loss of so many lives, will cease.” The handsome Târâ, looking askance, lowered her head with shame and gently spoke to Brahmâ :-- “This is the Chandra's child” and went inside. Chandra Deva, then, became very glad and took the child, put down its name as Budha and carried it, to his own house. Bhagavân

 

p. 41

 

Brahmâ, Indra and the other Devas went back to their respective places. All the spectators went also to their own places whence they came. O Munis! I have now described the birth of Budha, as the son of Chandra and in the womb of Brihaspati's wife, as I heard it from the mouth of Vyâsa Deva, the son of Satyavatî.

 

Thus ends the eleventh chapter of the 1st Skandha on the birth of Budha in the Mahâpurâna S'rî Mad Devî Bhâgavatam of 18,000 verses by Maharsi Veda Vyâs.