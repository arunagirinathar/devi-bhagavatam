\chapter[]{On the questions by \'Saunaka and others}
1. I meditate on the beginningless Brahm\^avidy\^a who is Sarvachaitanyar\^up\^a, of the nature of all-consciousness;  May She stimulate our buddhi to the realisation of That (or who stimulates our buddhi in different directions).

2. \'Saunaka said :-- “O highly fortunate S\^uta! O noble S\^uta! You are the best of persons; you are blessed inasmuch as you have thoroughly studied all the auspicious Pur\^anas.

3. O sinless one! you have gone through all the eighteen Pur\^anas composed by Krisna Dvaip\^ayana; these are endowed with five excellent characteristics and full of esoteric meanings\footnote{Note: The characteristics are to make the mantras reveal to one's own self, to realise, to transfer to others the \'Sakti, force thereof, to prove the various manifestations of the several effects thereof, etc.}.

4-5. O Sinless one! It is not that you have read them like a parrot, but you have thoroughly grasped the meaning of them all as you have learnt them from Vy\^asa himself, the son of Satyavati. Now it is our good merits that you have come at this divine holy excellent Vi\'svasan Ksettra (place), free from any defects of the Kali age.

6-10. O S\^uta! These Munis assembled here are desirous to hear the holy Pur\^ana Samhit\^a, that yields religious merits. So describe this to ns with your mind concentrated. O all-knowing S\^uta! Live long and be free from the threefold sorrows of existence. O highly fortunate one! Narrate to us the Pur\^ana equivalent to the Vedas. O S\^uta! Those persons that do  not hear the Pur\^anas, are certainly deprived by the Creator, though they have apparently the organ of hearing, of the power of tasting the sweet essence of words; because, the organ of hearing is gratified then and then only when it hears the words of the wise men, just as the organ of taste is satisfied then and then only when it tastes the six kinds of rasas (flavour, taste) (sweet, sour, pungent, bitter, salty, and astringent). This is known to all. The serpents that are void of the organ of hearing are enchanted by sweet music; then why should not those persons that have the organs of hearing and are averse to hear the Pur\^anas, be thrown under the category of the deaf? 

11-18. O Saumya! Hence all these Br\^ahmins, being distressed with the fear of this Kali, have come here to this Naimis\^aranya, eager to hear attentively the Pur\^anas, and are staying here with this one object. Time must be spent away anyhow or other; those that are fools while away their times in sports and other evil practices and those that are learned pass away their times in meditating on the \'S\^astras; but these \'S\^astras are too vast and very varied; they contain Jalpas (debates or wrangling discussions to win over the opposite party), Vadas (sound doctrines to arrive at just conclusions), and various Arthav\^adas (explanations and assertions, recommending Vidhis or precepts by stating the good arising from its proper observance and evils arising from its omission and also by adducing historical instances for its support; praises and eulogies) and filled with many argumentations. And, amongst these \'S\^astras again, the Ved\^anta is the S\^attvik, the Mim\^amsas are the R\^ajasik and the Ny\^aya \'S\^astras with Hetuv\^adas, are the T\^amasik; so the \'S\^astras are varied. Similarly, the Pur\^anas are of three kinds :-- (1) \'Sattvik, (2) R\^ajasik and (3) T\^amasik. O Saumya! (one of gentle appearance) you have recited those Pur\^anas endowed with five characteristics and full of many narratives; of these, the fifth Pur\^ana, equivalent to the Vedas and with all the good qualities, the Bh\^agavata yields Dharma and K\^ama (religion and desires), gives liberation to those who desire for emancipation and is very wonderful; you mentioned this before but ordinarily; you did not dwell on this specially. Now these Br\^ahmanas are eager to hear gladly this divine auspicious Bh\^agavata, the best of the Pur\^anas; so kindly describe this in detail.

19-25. O knower of Dharma! By your faith and devotion to your Guru, you have become S\^attvik and thus have thoroughly known the Pur\^ana Samhit\^as spoken by Veda Vy\^as. O Omniscient one! Therefore it is that we have heard many Pur\^anas from your mouth; but we are not satisfied as the Devas are not satisfied with the drinking of the nectar. O S\^uta! Fie to the nectar even  as the drinking of nectar is quite useless in giving Mukti. But hearing the Bh\^agavata gives instantaneous Mukti from this Sams\^ara or round of birth and death. O S\^uta! we performed thousands and thousands of Yajñas for the drinking of the nectar (Amrita), but never we got the full peace. The reason being that Yajñas lead to heaven only; on the expiry of the period of punya (good merits, the heavenly life ceases and one is expelled, as it were from the Heavens. Thus incessant sojourns in this wheel of Sams\^ara, the constant rounds of births and deaths never end. O Knower of every thing! Thus, without Jñ\^ana (knowledge, wisdom) Mukti never comes to men, wandering in this wheel of Time (K\^alachakra) composed

of the three Gunas. So describe this holy Bh\^agavata, always beloved of the Mumuksas (those that desire Mukti), this secret work yielding liberation, holy and full of all sentiments (rasas).

Thus ends the first chapter of the first Skandha on the questioning about the Pur\^ana by Saunaka and other Rishis in the Mah\^apur\^ana \'Sr\^imad Dev\^i Bh\^agavatam of 18,000 verses by Maharsi Veda Vy\^asa.

Here ends the First Chapter of the First Skandha of \'Sr\^imad Devi Bh\^agavatam on the questions by \'Saunaka and other Risis.