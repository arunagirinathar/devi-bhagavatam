On the description of the curse on Gangâ, Mahâbhisa and Vasus

 

p. 84

 

1-8. The Risis said :-- “O Sinless Sûta! You have described to us in detail the birth of Vyâsa, of unrivalled fire, and of Satyavatî; but we have one great doubt in our minds though, O Knower of Dharma! which is not being removed by your words. O Sinless one! First, as regards the mother of Vyâsa, the all auspicious Satyavatî, we have this doubt how she came to be united to the virtuous S’antanu? The king S’antanu, of the family of Puru is a greatly religious man; how could he have married Satyavatî knowing her to be a fisherman's daughter and born of a low family? Now say who was the first wife of S’antanu and how Bhîsma, the intelligent son of S’antanu came to be born of the parts of Vasu? O Sûta! You have told before that Bhîsma, of indomitable valour, made the Satyavatî's son, the brave Chitrângada, king; and subsequent to his death made his younger brother Vichîtravîrya king. But when the elder brother Bhîsma, the greatly religious and beautiful was present, how was it that Chitrângada and Vichîtravîrya having been installed by Bhîsma himself could have reigned.”

 

9-12. Again on the demise of Vichîtravîrya, Satyavatî became very much grieved and got two sons born of her two son's wives by Vedavyâsa? How can we explain this fact? Why did she do this? Why did she not give to Bhîsma the kingdom? Why did Bhîsma not marry? And how was it that the elder brother Vyâsa Deva, of indomitable valour, did such an irreligious act as to beget two (Goloka) sons from the wives of the brothers? Vyâsa composed the Purânas and knew everything of religion; how then did he go to other's wives, especially, of his brother's wives?

 

13-14. O Sûta! Why did Vyâsa Deva do such a hateful act, in spite of his being a Muni? The actions of Vedas are inferred from their subsequent good conducts; how can this act of Vyâsa be calculated as one amongst them? O Intelligent one! You are the disciple of Vyâsa; therefore you are the best man to solve our doubts. We all of this Dharmakshettra Naimisâranya are very eager to hear this.

 

15-39. At this Sûta said :-- In ancient days there reigned a king named Mahâbhisa, in the family of Iksâku endowed with all the qualities of a great king; he was the foremost of all the kings, truthful and religious. That highly intelligent king performed thousand horse-sacrifices (Asva

 

p. 85

 

medhas) one hundred Vâjapeya sacrifices and thereby satisfied Indra, the king of the Devas and went to Heavens. Once, on an occasion, that king went to the abode of Brahmâ; the other gods also went there to serve Prajâpati. The great river, Gangâ Devî, too, assuming the feminine form, went to Brahmâ to serve him. Now, in the interval, violent winds arose and the clothing of Gangâ Devî went off; at this the Devas did not look at her; rather kept their faces downwards; but the king Mahâbhisa continued gazing at her. Gangâ also came to know the king and that he had become attached to her. Brahmâ, seeing that both of them are love-stricken and are shameless, became angry and cursed them immediately :-- O king! you better take your birth again in the human world and practise great meritorious deeds and come again to this Heaven. Thus saying, Brahmâ looked at Gangâ, who was attached to the king, and addressed her :-- “You too better go to the human world and become his wife.” Both of them, the king as well as Gangâ, came out of Brahmâ's abode, very much grieved in their hearts. The king Mahâbhisa thought of coming to this world and reflected on the kings thereof and settled to make the king Pratîpa of Puru's family his father. At this time the eight Vasus with their wives wandering in various places and enjoying as they liked came to the hermitage of Vas'istha. Amongst the aforesaid eight Vasus Prithu and others, one Vasu Dyau's wife seeing Nandini, the sacrificial cow (Kâmadhenu) of Vas'istha asked her husband :-- “Whose is this excellent cow that I see?  Vasu then replied as follows :-- “ O Beautiful one! This is Vas'istha's cow. Whoever, be he a man or woman drinks her milk gets his longevity extended to ten thousand years and his youth never ends.” Hearing this, the Vasu's wife said :-- “There is a very beautiful comrade (Sakhî) of mine, the daughter of the Rajarsî-Us'îna in the world, of auspicious qualities. O Mahâbhâga! Kindly bring to me from Vas'istha's hermitage that auspicious sacrificial milch cow Nandini together with her calf that yields all desires; my Sakhî will then drink her milk and be thereby free from disease, old age and become the chief amongst all mankind. Hearing thus, his wife's word, the Vasu Dyau, though sinless, stole away together with Prithu and the other Vasus the cow Nandini in utter defiance to the self-controlled Muni Vas'istha. When the cow Nandini had been stolen, the great ascetic Vas'istha came quickly to the hermitage with abundance of fruits.

 

The ascetic Muni Vas'istha, not finding, in his hermitage, his cow with her calf, searched for her in many forests and caves; but he, the son of Varuna, could not find out his cow even after prolonged searches; he, then, took recourse to meditation and came to know that the Vasus had stolen the cow and became angry. He expressed :-- “When the Vasus have stolen this my cow in utter defiance to my self, they must be born

 

p. 86

 

amongst men.” When the religious Varuna's son Vas'istha thus cursed the Vasus, they became very sorry and absent-minded; all of them went to Vas'istha's hermitage and saw him there; they began to supplicate him as much as they could; and took refuge under him. Seeing the Vasus standing before him in an extremely distressed condition, the virtuous Muni Vas'istha said :-- “You all will be free from the curse within one year; but the Vasu Dyau will dwell amongst men for a long, long period as he had stolen direct my Nandini with her calf.”

 

40-60. While the Vasus, thus cursed, were returning, they saw on the way the chief river Gangâ Devî also cursed and therefore distressed; all of them bowed down to her simultaneously and said: “O Devî! A serious thought is troubling our minds, how can we, who live on nectar, take our birth in human wombs; so, O best river! You better be a woman and give birth to us. O Sinless one! You better be the wife of the sage King S’antanu and no sooner we be born of your womb, kindly throw us in the river Gangâ (your water). If you do thus, O Gangâ we will certainly be freed of our curse.” Gangâ Devî replied “Well; that will be.” Thus spoken, the Vasus went to their respective places; and Gangâ Devî, too, thinking on the subject again and again, went out of that place. At this time Mahâbhisa became born as a son of the king Pratîpa and became known as S’antanu. He was exceedingly religious and true to his promise. One day while the King Pratîpa was praising the Sûrya Devî (the sun) of unequalled energy, Gangâ Devî assumed an extraordinarily beautiful feminine form and came out of the waters and sat on the right thigh, resembling like a sâl tree, of the king Pratîpa. The sage king Pratîp spoke out to the lady sitting on his right thigh, thus :-- “O beautiful faced one! Why, unasked, have you sat on my auspicious right thigh?” The lovely Gangâ then replied :-- “Hear why I have sat here. O best of Kurus! O king! Becoming attached to you, I have sat on your thigh; so please accept me.” At this the king Pratîpa spoke to the beautiful lady, full of youth and beauty, “I never go, simply out of passion to another's wife. There is another point; you have sat on my right thigh; that is the seat of sons and son's wives; so, when my desired son will be born, you will then, be my son's wife. And certainly, by your good will, my son will be born.” The lady, of divine form, said, Well; that will be done! and went away. The king returned to his palace, thinking of the lady. After some time, he had a son born to him and when the son attained his teens, the king desired to lead a forest life and communicated this matter to his son. He said also, if the aforesaid beautifully smiling girl comes to you to marry, then marry her. And I am also ordering you not to question her anything “who are you” and so forth. If you take her as your legal wife, you will certainly be happy. Thus

 

p. 87

 

saying to his son, the king Pratîpa handed over all his kingdom to his son and gladly retired into the forest. The king practised tapasyâ in the forest and worshipped Ambikâ; on quitting his mortal coil, he went by his sheer merit to the Heavens. The highly energetic king S’antanu, on getting his kingdom, began to administer justice according to the laws of Dharma and governed his subjects.

 

Thus ends the third Chapter of the Second Skandha on the description of the curse on Gangâ, Mahâbhisa and Vasus in the Mâhapurânam S’rî Mad Devî Bhâgavatam of 18,000 verses.