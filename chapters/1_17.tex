On S’ûka’s displaying his self-control amidst the women of the palace of Mithilâ

 

p. 60

 

Sûta said :-- Thus speaking to his father about his intention to go to Mithilâ, the high-souled S’ûka Deva fell prostrate at his feet and with folded palms said :-- O highly fortunate one! Your word must be obeyed by me; now I desire to see, as you say, the kingdom of Janaka; kindly give me permission. O father! Again the doubt is coming within me how the king Janaka is governing his kingdom without sentencing any body? And if there be no punishment within his kingdom, no one will remain in the path of virtue. It is for the sake of preserving religion that Manu and the other sages have always prescribed for punishment; how, then, religion can be preserved without inflicting punishment. O Mahâbhâga! What you have spoken to me appears to me true like the sentence “My mother is barren.” So, O destroyer of foes! Permit and I will start for Mithilâ. Seeing the greatly wise son S’ûka, void of any desires, earnestly anxious to go to Mithilâ, gave him a cordial embrace and said :-- “O highly intelligent son S’ûka! Peace be on you! Have a long life. O child! Speak truly before me and go. O son! Say that after going to Mithilâ you will come back again to this Âs'rama; never that you will go anywhere else. O son! Seeing the lotus face of yours, I am passing my days happily; if I do not see you, I will suffer extreme pain. What more than this, that you are my life and soul. I am saying, therefore, after seeing Janaka and clearing your doubts come here again and remain at peace, and study on the Vedas.”

 

p. 61

 

Vyâsa having spoken thus, S’ûka bowed down and circumambulated his most worshipful father, and went out; he began to walk very fast like an arrow, leaving the bow, and when it has left the bow. On his journey he saw various countries, various classes of persons, earning money, various gardens and forests, various trees; in some places he saw fields with green grains and grains standing on them; at others he saw ascetics practising asceticism, and initiated Yâjniks (performing yajñas, or sacrifices); in some places he saw yogis practising yoga, the high-souled Vânaprasthîs (in the third stage of life) residing in the forest, and at others he saw devotees of S'iva, S’akti, Ganes'a, Sûryâ and Visnu and many others. Thus he went on in his journey, in great wonder, towards his destination. In his passage he crossed Meru in two years and the Mount Himâlayâs in one year and then reached the city Mithilâ. Going there he found the place, full of wealth, corn, grains, etc., and all prosperities and the people were all very happy and they observed the rule of conduct as in the S'âstras. When he was about to enter into the city the guard in front of the gate stopped him, asking “Who are you? Sir. What for are you come here?” When the guard asked him thus, he replied nothing and went away to a distance and with great wonder could not help laughing in his mind and remained motionless like a statue. At this the guard said :-- “O Brâhman! Why have you remained silent? Kindly say what for you have come here? I know this well that no body goes anywhere without having something to do? The king has forbidden strangers whose parentage and character are unknown. Therefore O Brâhmin! Every one has to take the king's permission before he goes into the city.

 

It seems that you are a very energetic Brâhman and that you know the Vedas; so O giver of honour! speak out to me your object-in-view and go into the city as you like.” Hearing these words of the guard, S’ûka Deva began to say :-- “I have come to see the city of Videha Janaka; but now I see that persons like me find great difficulty to enter here; so O Gatekeeper! I have got the answer from you. I was a great fool that I was so much deluded that to see the king I crossed many countries and over topped the two mountains and have come here. O Mahâbhâga! What blame can I put on others? It is my father that has deluded me; or my karma done in my previous birth is now making me wander about. Alas! In this world greed for money is the sole cause to make a man knock about; but I have not got that even; my erroneous idea has brought me so far. I now realise that a man, having no desires, gets constant happiness if he be not plunged in the net of delusion; else he cannot have any such. O Mahâbhâga! Though I have no desire of anything, yet I am

 

p. 62

 

plunged in the sea of Moha. Alas! Where is Meru? and where is Mithilâ (a great distance intervenes) I have walked so great a distance on foot; alas! this is this the result of my so long a journey! Therefore I am thoroughly convinced that the Creator has deceived me. One must have to suffer for his Prârabdha karma, be it auspicious or inauspicious. One must make one’s effort, being always under the control of this Law of Karma. Though there be no apparent desire or cause, yet this Prârabdha Karma always puts a man into different actions.

 

This place is not a Tîrath (holy place) nor there are the Vedas personified here, that I have taken so much pains and trouble to come here -- only there is one thing here and that is the king Janaka; but there is no chance to see him even; for I have not been able to enter even within his kingdom.” Thus saying, S’ûka remained silent and began to stay as one who has taken the vow to remain silent. The guard then took him to be a very wise Brâhman and spoke in sweet words :-- “O Brâhman! Go to the place, as you like, where you have got your work. O Brâhman! I stopped you; so please excuse me for any offence incurred by me. Free persons like you have mercy alone as their greatest strength.” Hearing this S’ûka Deva said :--What is your fault? you are dependent on another; the servant ought to obey the words of his master and serve him by all means; and there is no fault of the king, too, in your not allowing me to enter; for the wise persons ought to ascertain by all means, whether the new comers are enemies or thieves? Hence when I am quite a stranger suddenly come to this place, that the fault is wholly mine. Every person knows that it is lowering oneself to go to another's house. The guard then said :-- “O great Brâhman! what is happiness? and what is pain? what ought to be done to by your well wisher? who is your enemy? and who is your benefactor? Now advise me on all these points.” Hearing this S’ûka Deva said :-- Everywhere men are divided, as far as their internal natures are concerned, into two classes; they are called attached or unattached. And the minds of these two classes are again of two kinds. The “attached” man is stupid and cunning and the “unattached” is sub-divided into three classes knowing, unknowing and middling. The cunning man is divided again into two classes :-- Whether his cunningness is according to the dictates of S’âstras or arising from his intellect. Again intellect is sub-divided into two whether it is Yukta (one-pointed) or Ayukta (Diverted) The guard spoke :-- “O Learned one! I cannot understand what you say; so explain them to me what they mean.” S’ûka Deva said :-- Those who are attached to this world are said to be “attached” persons. These attached persons feel frequently various pleasures and pains. When they get wives, sons, wealth, honour, rise, etc., they get pleasures; and if they

 

p. 63

 

do not get any of these they feel at every moment intense pain. Now the attached person sought to take such means as will secure them the pleasures of this world; so whoever acts against those means are denominated as breakers of their happiness and so they are enemies; and whoever aids in their acquiring pleasures are denominated as their friends. Of these the attached but at the same time cunning man does not get confounded and bewildered by them; whereas stupid attached man gets always bewildered everywhere. The man that is dispassionate and engaged in determining the “self” dwells in a solitary place, meditates on “self”, finds pleasure in studying the Vedânta  S’âstras and feels pain in all the topics on worldly affairs. The wise man that wants his real welfare and is averse to the worldly enjoyments finds that he has many enemies; lust, anger, palaces, etc., are his so many enemies. Contentment is his only friend in the three lokas and no one is his real self.

 

Hearing these words of S’ûka Deva, the watchman considered S’ûka Deva a very wise man and soon led him to a very beautiful compartment. S’ûka Deva then began to see that the town was full of three sorts of men, good, middling, and bad; and the shops were filled with various articles of merchandise. The many things were being incessantly purchased and sold there. Within that town, filled with many men, money and all sorts wealth and prosperities, almost everywhere were seen instances of attachment, hatred, lust, anger, greed, vanity and delusion; at some parts there were seen persons quarrelling with each other. Seeing thus the three sorts of persons, the highly energetic S’ûka, blazing like a second Sun went to the royal palace when the gateman stopped him. He stood there like a log of wood and began to meditate on “Moksa” (Liberation). He began to think the light and darkness as same; the greatly ascetic S’ûka became merged in Dhyâna (meditation) and remained at one place motionless. In an instant, a royal minister came out and saluting him with folded hands, took him to a second compartment. Here the minister showed him beautiful divine gardens adorned nicely with rows of divine trees bearing fruits and gave him a good reception and took him to a very beautiful palace. The minister next ordered the public women in royal service, expert in music and playing with instruments, and skilled in Kâma-S’âstra (the science of amorous dealings) to attend on S’ûka Deva and went out of the palace. S’ûka, the son of Vyâsa, remained there. Those prostitutes then prepared various dishes, suited to the time and place, and sought the satisfaction of S’ûka and then worshipped him duly with greatest devotion. Those ladies, then, residing within the four walls became enamoured to see the beauty of S’ûka Deva and showed him the gardens that existed in the inner compound. S’ûka was young and beautiful; over this he was extremely lovely, of nice limbs; his speech was soft

 

p. 64

 

and gentle; so he looked like a second Cupid (the god of love); all the ladies, struck with Cupid’s arrows, lost their consciousness. Then recovering, they considered S’ûka Deva to be the great controller of passions and began to serve him with great care. The pure minded S’ûka, born of Arani, looked on them like his mother. S’ûka, finding pleasure in self and the controller of anger was not pleased or displeased with anything; so though be saw that the ladies were disturbed with amorous feelings, he remained quite undisturbed, calm and quiet. The ladies, then prepared a very nice bed whereon S’ûka Deva would sleep; it was spread over with nice clean bed sheet; many nice pillows were placed. He, then, washed his feet and with vigilance, put on his finger the ring prepared of Kus'a grass, and completing his evening Sandhyâ, became merged in Dhyâna. Meditating on Supreme Brahmâ for three hours (one Prahara), slept for 6 hours and getting up, again became merged in Brahmâ Dhyâna for the last three hours of the night. Then at the Brahmâ mûhurta (one hour preceding the sunrise) he took his bath and completing his morning duties, became immersed in Samâdhi (inner enlightenment) and sat at ease.

 

Thus ends the 17th chapter of the 1st Skandha on S’ûka's displaying his self-control amidst the women of the palace of Mithilâ in the Mahâpurâna S’rî Mad Devî Bhâgavatam.