\chapter{On the installation of Durg\^a Dev\^i in the city of Benares}

1. Vy\^asa said :-- Hearing the Dev\^i's words, the king Sub\^ahu began to say with great devotion thus :--

2-3. O Dev\^i! If there be made a comparison between the kingdom of the Devas and the world on the one hand and the vision of Thine on the other hand, then it must be acknowledged that the kingdom of the Devas and the earth cannot stand in comparison before Thee. O Dev\^i! There cannot be anything, in this Trilok\^i, that is more exalted than Thy vision; therefore, O Mother! What other boon may I ask from Thee. I am very thankful and blessed; all my desires are fulfilled, when I have seen Thee.

4-5. O Auspicious Mother! I ask from You this boon, my desire that my devotion may remain constant, fixed, and unflinching towards You. O Mother! You would remain always in this city of mine being celebrated under the name of \'Sr\^i Durg\^a Dev\^i, Your \'Sakti. This is my desire.

6-9. O Dev\^i! As you have cleared off all the obstacles of Sudar\'sana and saved him from this danger, so remain here in this city of Benares and protect it, so long as this city stands on the face of the earth and make it firm and well established and renowned. O Durg\^a, I pray that you may grant me these boons. O Dev\^i! Grant me also various other desires of mine and destroy my enemies and extirpate all the irreligious and wicked people in this city. O Goddess of mercy! What more can I ask from you?

10-11. Vy\^asa said :-- Thus praising and praying, the king Sub\^ahu stood, with folded hands, before the Dev\^i Durg\^a, the remover of all calamities, when She addressed thus :-- O king! I will remain no doubt, in this city of Benares, the place of salvation, as long as it stands on the face of the earth and protect all the people here.

12. Then came there Sudar\'sana, heartily gladdened; and he bowed to Her and began to praise Her with intense joy and devotion.

13. O Mother of this Universe! Everyone in this world shows mercy to those that are devoted to him; but, O Mother! I see, in Your case, You take it as if Your bounden duty, to save those, that are void of any devotion towards You; for You have saved my life, though I am devoid of any devotion towards you. Therefore how can I describe the boundless ocean of mercy that reigns in You!

14. O Goddess! I have heard that You have created all this Universe, with its elements, and You are preserving this Your own creations and again You will destroy it in due time. Therefore O Mother! What wonder is there that you have saved me!

15. O Goddess! Now order me early what work of Yours shall I do now? Where shall I go? O Mother! Now I am unable to make out my duty; therefore kindly order me whether I will remain here or go anywhere else or remain anywhere, I like, at my leisure?

16. Vy\^asa said :-- On Sudar\'sana thus petitioning before the Dev\^i, She said with much kindness :-- ``O good soul! Go to Ayodhy\^a and govern the country befitting your family.

17. O king! Constantly remember Me and worship Me with great care. I will always look after the welfare of your kingdom.

18. Especially in the eighth, fourteenth, and in the ninth day of the lunar half month, worship Me according to the prescribed rites and rules and offer me victims (sacrifices).

19. O sinless one! Establish my image in this city and worship it three times, morning, mid-day and evening carefully and with devotion.

20. It is noteworthy that My Great Puja in autumn for the nine nights (Navar\^atra) ought to be done with the greatest devotion.

21-22. O king! In the month of Chaitra, M\^agh, Â\'sv\^in, and Âs\^adha, My grand festival should be done on the four Navar\^atris respectively; and especially on the fourteenth and on the eighth day of the black half, all persons ought to worship Me with their minds full of devotion towards Me.''

23. Vy\^asa said :-- After the Dev\^i, the Goddess Durg\^a, the Destroyer of all dangers, had finished Her sayings, Sudar\'sana bowed down to Her and praised Her much. The Dev\^i, giving him the above mentioned advices, disappeared.

24. Seeing Her disappear, all the kings went to Sudar\'sana and bowed to him, as the Devas go to their lord, the Indra.

25. The king of Benares, Sub\^ahu, too, gladly bowed down and stood before him. Then all the kings began to address Sudar\'san, the king of Ayodhy\^a.

26. ``O king! You are our lord and governor; we are always your servants; protect us as the king of Ayodhy\^a.

27. O king! It is through your grace only that we have seen the Supreme Force, the Goddess of this Universe, the most Auspicious, the Eternal Bhav\^an\^i, the Giver of the fourfold desires.

28. O king! It is for your sake that the Eternal, Highest Prakriti Dev\^i appeared; therefore you are very fortunate, auspicious, and most blessed in this world. Your have finished, as it were, all that you had to do.

29. O king! We all are deluded by the M\^ay\^a of that Mah\^am\^ay\^a Chandik\^a Dev\^i; therefore none of us is able to know Her prowess.

30. We are always engaged in thinking of wealth, sons and wives; there we are merged in this awful ocean of delusion, infested with crocodiles, etc., in the shape of lust, anger, greed, etc.

31. O Blessed one! You are highly enlightened and you know everything; hence we ask you What is this Force; whence has She sprung? How is Her prowess? Kindly describe all these to us.

32. O Descendant of Kakud! The saints are always merciful; kindly therefore relate to us the glory of the Excellent Goddess, that serves the purpose of a boat in crossing this ocean of world (transmigration).

33. O king! I am intensely desirous to hear the prowess and nature of the Dev\^i.''

Note :-- Kakud is an epithet of Puranjaya, son of \'Sas\^ada, a king of the solar dynasty, and a descendant of Ikshv\^aku. The Mythology relates that when in their war with the demons, the gods were often worsted; they, headed by Indra went to the powerful king Puranjaya and requested him to be their friend in battle. The latter consented to do so, provided Indra carried him on his shoulders. Indra accordingly assumed the form of a bull and Puranjaya seated on its hump, completely vanquished the demons. Puranjaya is therefore Kakutstha ‘standing on a hump.'

34. Vy\^asa said :-- When the kings had thus asked, the son of Dhruvasandhi, the king Sudara\'sana became very glad and, meditating on the Goddess, began to say thus :--

35. ``O kings! Indra and the other Devas, even Brahm\^a, Visnu, and Mahe\'sa are unable to fathom the most exalted deeds of that Goddess; how, then, can I describe to you the great glory of the Mah\^am\^ay\^a.

36-38. O kings! The Bhagavat\^i Bhav\^an\^i is present, as it were, being divided into four parts. She who is the first and foremost, the excellent S\^attvic Energy, worshipped by all, is always engaged in the preservation of this world. That part which is engaged in creating this world, is called the R\^ajasik Energy; and that part which is engaged in destroying the world is called the T\^amasik Energy, and that part which is the cause of all, Brahm\^a, etc., that Highest \'Sakti, the Bestower of all desires, is called the fourth \'Sakti, the Nirgun\^a \'Sakti.

39. O kings! Those who are not Yogis, will never be able to grasp the Nirgun\^a \'Sakti. The Sagun\^a Force can be easily served. All those middle Adhik\^aris (fit persons) and learned men always meditate and worship the Sagun\^a Aspect of Her.''

40-41. The kings said :-- ``O king! You got afraid and went in your very early age to the forest; how is it, then, that you have been able to know the excellent Goddess Mah\^am\^ay\^a. How did you worship and pray to Her? That She, becoming so glad, has favoured you and so helped you?''

42-43. Sudar\'san said :-- ``O kings! Early in my childhood, I got the excellent root-mantra of desires, K\^amav\^ija; daily I meditated and silently uttered that mantram. After that I came to realise through the Risis That Eternal Auspicious Mother; and since that time, day and night, I always used to remember that Highest Deity; with the greatest devotion.''

44. Vy\^asa said :-- Hearing the words of Sudar\'sana, the kings came to know that the Goddess which they saw was the Highest Force and filled with the greatest devotion towards Her, returned to their own homes.

45. The king of Benares, Sub\^ahu, returned to his own city after bidding good-bye to Sudar\'sana. The virtuous Sudar\'sana, too, went towards his Kosala kingdom.

46. The ministers were very glad to hear the death of \'Satrujit and to see the victory of Sudar\'sana.

47-48. The inhabitants and armies of S\^akata (Ayodhy\^a) hearing that Sudar\'sana is coming and knowing him to be the son of the king Dhruvasandhi, became highly delighted and approached to him with various offerings.

49-50. Sudar\'sana, with his new consort, arrived at Ayodhy\^a with his heart highly gladdened, and shewed his due regard and respect towards all his subjects. Then the ministers came and sainted him; the women threw at him offerings of L\^aja (fried rice) and flowers; the bards began to praise loudly. Thus, honoured by various auspicious ceremonies, the king entered into his palace.

Here ends the 24th chapter on the installation of Durg\^a Dev\^i in the city of Benares and the return to Ayodhy\^a of Sudar\'sana in the Mah\^a Pur\^anam \'Sr\^imad Dev\^i Bh\^agavatam of 18,000 verses by Maharsi Veda Vy\^asa.

