\chapter{On \'S\^uka’s desiring to go to Mithil\^a to see Janaka}

Vy\^asa spoke :-- Then the Dev\^i Mah\^a Laksm\^i, seeing the Deva Jan\^ardana lying on a banyan leaf and surprised, spoke to him, smiling :-- O Visnu! Why are you becoming so much astonished? Before this, since times immemorial (without any beginning) there had been many dissolutions (Pralayas); and many Sristis (creations); and at the beginning of every creation You came first into existence and every time I was united with You; but now You have forgotten me under the spell of that Mah\^a \'sakti.

That Highest Mah\^a \'sakti is transcendent of all the Gunas; but you and I are with Gunas. Know me as the \'Sakti, all of Sattva Guna who is widely known as Mah\^a Laksm\^i. After this the Praj\^apati Brahm\^a, full of Rajo Gunas, the creator of all the Lokas, will come into existence from your navel lotus and will create the three worlds. Then he will perform severe tapasy\^a and acquire the excellent power to create, and will create the three worlds by his Rajo Guna. That highly intelligent Praj\^apati will create first, the five Mah\^a bh\^utas (great elements), all endowed with Gunas and then create mind with sensory organs and the presiding deities of the senses, and thus with all the ingredients, fit for creation, will create all the worlds. Therefore He is denominated by all as the Creator of Brahm\^anda. O highly fortunate one! You will be the Preserver of this Universe. When the Praj\^apati Brahm\^a will be angry at the beginning of the creation on his four mind-born sons, Rudra Deva will appear.

He will appear then from the centre of his eye brows. On being born this Rudra Deva will practise very severe tapasy\^a and will get the Samh\^ara \'Sakti, who is all of Tamo Guna and at the end of the Kalpa will destroy all this universe of five elements. O highly intelligent one! So I have come to you for this work of creation, etc. So take me to you as your Vaisnav\^i \'Sakti full of Sattva Guna. O Madhus\^udana! I will take refuge at your breast and will remain always with you. Hearing all this, Bhagav\^an Visnu spoke :-- ``O beautiful Dev\^i! The half stanza was ere long heard by me, in clear words; by whom was that spoken? Kindly speak to me on this great auspicious secret first. For a great doubt has come and possessed my mind. What more shall I say than this that as a poor man always thinks of wealth, so I am thinking of that again and again.'' Hearing these words of Visnu, the Dev\^i Mah\^a Laksm\^i smilingly said, with great affection :-- ``O Strong and Energetic one! I am now speaking in detail on this; listen. O Four-armed one! It is because I have come to you with form and endowed with Gunas that you have come to know me; but you have not known that \^adya \'Sakti, the Prime force, transcending all the Gunas, though She is the substratum of all the Gunas. O highly fortunate one! That Dev\^i Bh\^agavat\^i, transcendant of all the Gunas, uttered that all auspicious, highly sanctifying semistanza, the essence of all the Vedas. O destroyer of the enemies! I think that there is the highest grace of that Highest \'Sakti on you, that She spoke the greatest secret to you for your benefit. O one performing good vows! know those words uttered by Mah\^avidy\^a, as the essence of all the \'S\^astra. So firmly retain them within your heart; never forget them. There is no other thing, save that, worth being known in earnest. Because you are most beloved by the Dev\^i! that She has spoken this to you.'' Hearing the words of the Dev\^i Mah\^a Laksm\^i, the four-armed Bhagav\^an took that semi stanza as a Mantra to be repeated in right earnest within his mind and cherished that for ever within his heart. After some time, Brahm\^a born of the lotus of the navel of Visnu, became afraid of the two Daity\^as Madhu and Kaitabha, took refuge of Bhagav\^an Visnu; Visnu killed the two demons and began to do distinctly the japam of the semi-stanza. Brahm\^a, born of the lotus, then asked Visnu with a gladdened heart :-- ``O Lord of the Devas! what japam are you doing? Lotus eyed! Is there any other body more powerful than you? O Lord of the Universe! whom do you think and thus feel yourself so happy?'' Hearing Brahm\^a, Bhagav\^an Hari said :-- ``O highly fortunate one! Think out yourself once of the Primordial Force, the auspicious Bh\^agavat\^i \^adya \'Sakti who is reigning everywhere as the cause and effect and you will be able to understand everything. My presiding Deity is that immeasurable eternal Mah\^a \'Sakti

Brahmamy\^i; on whose \'Sakti, as a receptacle with form on this ocean rests the whole Universe; I am thinking of that, by which is created (often and often) this entire Universe, moving and non-moving. When the Dev\^i Bh\^agavat\^i, the giver of boons, become graciously pleased, the human beings become freed of this bondage of Sams\^ara; and again that highest Eternal Wisdom, the cause of Mukti, becomes the source of bondage to this world, of those who are deluded by Her.

She is the \^i\'svar\^i of the \^i\'svaras of this universe. O Brahm\^a! You, I and all other things of the entire Universe are born of the Chit \'Sakti (the power of consciousness) of Her and Her alone; there is no manner of doubt in this. The semi-stanza by which She has sown within me the seed of Bh\^agavata will get expanded by the beginning of the Dv\^apara Yuga. While Bhagav\^an Brahm\^a was resting on the navel lotus of Visnu, He got the seed of Bh\^agavata. Then He gave it to His own son N\^arada, the best of the Munis. N\^arada gave it to me and I have expanded that into twelve Skandhas. Therefore, O Mah\^abh\^aga! You now study this Bh\^agavata Pur\^ana, equal to the Vedas and endowed with five characteristics. In this the wonderful glorious deeds and life of the Dev\^i Bh\^agavat\^i, the hidden meanings of the Vedas and the wisdom, the truth are all described; hence this is the best of all the Pur\^anas and sanctifying like the Dharma \'S\^astra. It is the substratum of Brahm\^a Vidy\^a; therefore if men study this, they will easily cross this sea of world; and those that are stupid and deluded get pleasure in hearing the death of Vritr\^asura and many other narrations that are interspersed

in this book. Therefore, O Mah\^abh\^aga! hear this sanctifying Bh\^agavata Pur\^anam and retain it firmly within your heart. O best of persons! You are the foremost of those that are intelligent; so you are worthy to read this Pur\^ana. Eighteen thousand \'slokas are in that Pur\^ana and you better get them by heart; for if anybody reads or bears this Pur\^ana, fit to be praised in every way, all-auspicious, capable to increase posterity by the addition of sons and grandsons, giving long life, happiness and peace, he sees the Sun of Wisdom, resting in his breast and dispelling all darkness of ignorance. Thus speaking to his son \'S\^uka Deva, Krisna Dvaip\^ayan, my Guru, studied us the Pur\^ana and thought it was voluminous. I got the whole of it by heart. \'s\^uka studied the Pur\^ana and stayed in Vy\^asa's \^a\'srama. But he was naturally dispassionate like Sanat Kum\^ara, etc., the mind-born sons of Brahm\^a; therefore he could not get peace in studying the contents of the Pur\^ana which deal with Karma-K\^anda (actions) fit for the house-holders. He remained in a solitary place, his heart being troubled very

much. He appeared, as if, with his heart void. He did not mind much for his food and he did not fast also. Once Vy\^asa Deva seeing his son \'s\^ukdeva so thoughtful, said :-- ``O Son! What do you think constantly? And why are you troubling yourself so much? Like an impoverished man, entangled in debt, you are always disturbed by your thoughts. O child! When I your father is living, what for do you care? Leave aside your inmost sorrows and be happy. Cast off all other thoughts and think of the wisdom contained in the \'S\^astras and try your best to acquire Vijñ\^ana, the essence of wisdom. O Suvrata! If you do not get peace by my words then go, at my word to Mithil\^a, the city of the King Janaka. O Mah\^abh\^aga! That king Janaka, who is liberated while living, whose soul is religious and who is the ocean of truth will cut asunder the net of your delusion. O Son! Go to the king and question him on Varn\^a\'sram Dharma (Dharma relating to caste and stages of life) and remove your doubts.

That royal sage Janaka, the greatest Yogi, the knower of Brahm\^a and liberated while living, is of pure soul, truthspeaking, of a calm and quiet heart and always fond of Yoga.'' Hearing these words of Vy\^asa Deva, the highly spirited \'S\^uka deva of unrivalled energy replied :-- ``O virtuous one! Your word can never turn out false; but when I hear that the king Janaka is gladly governing his kingdom still he is liberated while living, and disembodied while he has body -- this your word appears to me quite contradictory like light and darkness at one and the same place and time, and seems that these two epithets simply indicate vanity and nothing else. O Father! This is my greatest doubt how can the royal sage Janaka govern his kingdom, being disembodied. It appears that your word about Janaka is quite false as the son of a barren woman. O Father! I have now got a desire to see the disembodied king Janaka; for my mind is plunged in great doubt how can he remain in sams\^ara unattached like a lotus leaf in water? O Greatest Orator! Is the liberation of Janaka according to Buddhistic doctrines or like the opinions of the materialistic Ch\^arv\^akas! O highly intelligent one! How can the royal sage Janaka, in spite of his being a householder, quit the usages of his senses? I cannot comprehend this. How can the things enjoyed by him appear to him, as if, unenjoyed and and how can his doings be his non-doings? How can the ideas of mother, wife, son, sister, prostitutes and various persons having different relations, arising within him vanish again altogether? And if that be not the case, how can his Jivanmuktahood be possible? If his taste be present of pungent, sour, astringent, bitter, and sweet things, then it is clear that he is enjoying all the most excellent things, O Father! This is my greatest

wonder and doubt, that if he has got the sense of heat and cold, pleasure and pain, how can he be a Jivanmukta? That king is thoroughly expert in reigning his kingdom; how then the ideas of enemy, friend, taste and distaste, remaining absent in him, he can govern his state? How can he look with the same eyes a thief and an ascetic? And if he makes any distinction, how then is his liberation effected? I have never seen such a man, that is liberated while living and at the same time an expert king in governing his subjects. For these reasons, great doubt has arisen in me. I cannot understand how can the king Janaka be liberated, while he is remaining in his house? Whatever it be, I desire now greatly to see him after his Jivanmuktahood; so I desire to go to Mithil\^a to solve my doubts.''

Thus ends the Sixteenth Chapter of the first Skandha on \'S\^uka's desiring to go to Mithil\^a to see Janaka, in the Mah\^apur\^ana \'Sr\^imad Dev\^i Bh\^agavatam of 18,000 verses.