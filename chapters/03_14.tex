\chapter{On the narration of the glories of the Dev\^i}

1-2. Janamejaya spoke :-- O Dv\^ija! I have heard in detail the Dev\^i Yaj\~na, performed by \'Sr\^i Visnu. Now describe Her Glory and glorious deeds. After hearing these, the Dev\^i's glorious deeds, I will also perform that, the best of all sacrifices. Thereby I will no doubt be pure through your favour.

3. Vy\^asa said :-- O king! Hear, I am describing to you the history of the most auspicious mighty deeds of the Dev\^i, according to the Pur\^anas.

4-5. In days of yore, there reigned in the country of Kosala, the king Dhruvasandhi of the Solar Dynasty. He was the son of Puspa and cele-

brated on account of his great prowess. He was truthful, religious, engaged in doing good to his subjects, obeying the laws of the four castes and Â\'sramas. He being pure, performed his regal duties in the flourishing city of Ayodhy\^a.

6. The Br\^ahmanas, Ksattriyas, Vai\'syas, and \'Sûdras and other good persons all lived religiously under his rule, each abiding by his own profession.

7. No thieves, cheats, cunning persons, vain and arrogant persons, treacherous and illiterate men were allowed to remain in his kingdom.

8. O host of Kurus! Thus ruling virtuously, the king had two wives, both of them young, fair and beautiful and well able to give delights and enjoyments to the king.

9. The first and lovely wife was Manoram\^a, and the second was L\^il\^avat\^i. Both of them were exceedingly handsome, intelligent and qualified.

10. The king enjoyed much with them in palaces, gardens, romantic hills, lakes, and various beautiful mansions.

11. In the auspicious moment, Manoram\^a gave birth to a beautiful child, endowed with all royal qualifications. The child was named, in due course, Sudar\'sana.

12. Next within one month, his second wife the fair L\^il\^avat\^i in the auspicious fortnight and in the auspicious day, gave birth to an excellent child.

13. The King then performed the J\^ata-Karma (ceremonies on the birth of a child) and being very glad, made lots of presents, wealth, etc., to the Br\^ahmanas.

14. The King shewed affection equally to the two children; never he made any distinction between them.

15. The king, the tormenter of the foes, was very glad and performed duly the chud\^a karana ceremony according to his position and wealth.

16. The sight of these two sons delighted very much the people. Now seeing these Kritachûdas, and playing, the king was merged in the ocean of pleasure.

17-18. Sudar\'sana was the eldest; but Satrujit, the second beautiful son by L\^il\^avat\^i was of sweet and persuasive speech. His beautiful figure and sweet words gave very much delight to the king, and for these qualities, the child Satrujit turned out also a favourite of the people and ministers.

19. The king could not show so much affection to the unfortunate Sudar\'sana as he showed to Satrujit.

20-21. Thus some days having passed, one day the king Dhruvasandhi went out on an hunting expedition to the forest. He killed in the forest many deer, Ruru (a kind of deer), elephants, boar, hare, buffaloes, rhinoceros, camels and amused himself very much with this hunting affair.

22-23. While he was hunting thus, a lion got very much enraged, and, from a bush, suddenly jumped and came upon the king. That king of the beasts was already struck with arrows; now seeing the king in front, he loudly roared.

24-25. He angrily lifted his long tail high up in the air and, puffing up his manes, jumped up high in the air to attack and to take the life of the king. Seeing this, instantly the king took sword in his right hand and shield in his left and stationed himself like another lion before him.

26. The king's followers, one and all, angrily shot arrows on the lion.

27. Then a loud uproar ensued; and all began to hurl arrows as best as they could. But, after all, that dangerous lion fell upon the king.

28-29. Seeing thus, the king struck him with his sword, but the lion also tore asunder the king, with his sharp nails. The king thus struck by the lion, fell on the spot and died. The soldiers cried aloud and killed the lion with arrows.

30. Thus both the king and lion lay dead on the spot; and the soldiers turned back to the palace and gave all the informations to the royal ministers.

31. When the munis heard the demise of the king, they went to the forest, performed the burning of the dead body of the king.

32. The Maharsi Va\'sistha performed duly on the same spot, all the funeral ceremonies, thus ensuring the king the safe journey to the next world.

33. All the subjects and the citizens and the Muni Va\'sistha counselled each other to install Sudar\'sana on the throne as the king.

34-35. The minister-in-chief as well as the other members proposed that as Sudar\'sana is the son of the legal wife, calm and quiet, beautiful and endowed with all the royal qualifications, he is fit for the throne. Maharsi Va\'sistha said, the royal son, though not attained to proper age is still religious; therefore he is really fit be installed as king on the royal throne.

36. When the wise aged ministers thus decided, Yudh\^ajit, the king of Ujjain, on hearing the decision hastened to the spot.

37. He was the father of L\^il\^avat\^i; on hearing the demise of his son-in-law he came there, so that his daughter's son might get the kingdom.

38. Next, V\^irasena, the king of the country of Kalinga and the father of Manoram\^a, came there also with the object that his daughter's son Sudar\'sana be the Emperor.

39. The two kings, accompanied respectively by their own army and soldiers, began to counsel with the aged ministers, each trying so that his daughter's son may get the throne.

40. Yudh\^ajit made the question :-- ``Who is the eldest of the two sons? Is it always the case that the eldest will inherit the kingdom? Will not the youngest ever be able to acquire it?''

41. V\^irasena said :-- O king! He who is the son of the legal wife inherits the kingdom; this I have heard from the learned who are proficient in the knowledge of the \'S\^astras.

42. Hearing V\^irasena, Yudh\^ajit repeated ``Sudar\'sana is not so qualified with royal qualifications and other matters as this son of the late king, Satrujit. How can then Sudar\'sana inherit the throne?''

43. O King! Then quarrels ensued amongst the two kings. Now, at this critical juncture, who is able to solve their doubts?

44. Yudh\^ajit then addressed the ministers :-- ``You all are prompted by selfish ends; you want to acquire a good deal of money by making Sudar\'sana the king.

45-46. I have come to know by your gestures and postures that your decision is to the above affect. After all, as Satrujit possesses many more qualifications than Sudar\'sana, he has more claims to the throne; and therefore he is fit to occupy the throne and no other. Morever, let me see as long as I live who can set aside the claims of a qualified prince, in possession of an army, and put forward the claims of a prince who has no qualifications it all.

47. I am ready to fight and I will tear the earth into two pieces by my sword. What more have you to say on this?''

48. Hearing this, V\^irasena addressed Yudh\^ajit ``I see the two boy's intelligence the same. You are intelligent; kindly mention where is the difference?''

49. O king! The two kings quarrelling with each other, remained there; the subjects and the Risis, seeing this, were very anxious.

50. Hundreds of tributary princes wanting that the two kings might be

involved into quarrels with each other, came to the spot, with their soldiers, though they had to undergo great hardships in doing so.

51. Many aborigines, from the inhabitants of Sringaverpur, hearing the demise of the late king, also appeared on the scene with the sole object to plunder.

52. The two princes are minors; and hearing their parties at war with each other, many robbers from various adjoining countries came also there.

53. Thus when the war broke out between the two kings, the great confusion and tumult across within the kingdoms; on the other hand, Yudh\^ajit and V\^irasena both became ready to fight.

Thus ends the Fourteenth Chapter on the narration of the glories of the Dev\^i and the death of the Kosala king Dhruva Sandhi in the 3rd Adhay\^aya of \'Sr\^i Mad Dev\^i Bh\^agavatam.