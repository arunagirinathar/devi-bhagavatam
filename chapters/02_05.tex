\chapter{On the marriage of Satyavat\^i}

1-20. The Risis said :-- ``O son of Lomaharsana, O S\^uta; you have described to us how the eight Vasus, cursed by Va\'Sistha, took their birth and how Bh\^isma was born. O knower of Dharma! Now describe to us in detail how the greatly religious king \'Santanu married the auspicious Yojanagandh\^a, the chaste Satyavat\^i, the mother of Vy\^asa,

knowing full well that she was a fisherman's daughter? O Suvrata! Remove our this doubt. S\^uta then said :-- The sage king \'Santanu always used to go to forests on hunting expeditions, with his heart addicted to hunting buffaloes, deer and various other wild animals.

Thus, for four years that king went out a hunting, with his son Bh\^isma, deer and other wild animals and got the highest happiness as Mah\^adeva finds in company with K\^artikeya. Once, on an occasion, while he was shooting arrows at rhinoceros and boar, etc., he went so far as to reach a forest on the banks of the Yamun\^a, the chief of the rivers. There he began to smell an excellent nice smell that could not be described in words; he tried to find out the source and wandered here and there, and on all sides of the forest; and thought that this enchanting smell was not that of Mand\^ara flowers, musk, champaka nor that of M\^alat\^i nor that of Ketaki flower; the air was blowing saturated with peculiar fragrant smell that he never experienced before. Thus thinking of that smell, the king \'Santanu, being enchanted with that, followed to its source in that forest. At last he came to a spot on the banks of the Yamun\^a the chief of the rivers, where a very beautiful girl, calm and quiet and with feminine gestures and posture amorous, lovely but ill-clad, was sitting; and he found out that the above beautiful smell was coming out of her body. The form of the lady was extremely beautiful; the smell was very wonderful, and captivating the hearts of all; her age then entered to youth and she was very auspicious. The king was greatly surprised and was eager to know who the lady was; whence she had come; whether she was a Deva girl, or a human or a Gandarbha daughter or a N\^aga daughter? But, being unable to come to a definite conclusion and becoming passionate, he remembered Gang\^a and asked that lady sitting on the bank of the Yumn\^a, thus :-- ``O dear! Who are you? Whose daughter are you? Why are you alone in this lovely forest? O beautiful eyed! Are you married? Or are you as yet unmarried? So answer to all these. O lovely eyed one!  Seeing your lovely enchanting form I am become extremely passionate. So, O dear! Describe in detail to me, who are you? and what do you intend to do?'' When the king spoke thus the lotus eyed nice-teethed lady replied as follows :-- ``O king! Know me as a fisherman's daughter and I am completely under my father's command. O king of kings! For Dharma's sake I carry ferry across this Yumn\^a river. My father has gone to-day to our house. O Master of wealth! Thus I have spoken truth to you.'' Thus saying the lady desisted; the passionate king then spoke to her :-- ``I am the foremost hero of the Kuru family; so choose me as your husband; then your this youth will not go in vain.

21-32. O fawn-eyed one! I have no other wife existing; so you will be my legal wife. O Dear! Passion is giving much pains to me; therefore I am now become your obedient servant for ever. O Beloved! My former dear wife has abandoned me and gone away; but I have not married since then. Now seeing you beautiful, in all respects, I cannot bring my mind under control.''

Hearing these nectar-like beautiful words of the king, the sweet scented fisherman's daughter, though also turned extremely passionate, held patience and exclaimed :-- ``O king! I also desire that which you have expressed; I am of opinion to act according to your wishes. But, what am I to do? I am not dependent. You are to know this. My father alone can give me in marriage to you. So better ask my father for me. Though I am a fisherman's daughter, still I am not wanton and wilful. I am always obedient to my father; if my father wills, you can marry me. And I will be obedient to you. O king! The god of love is tormenting me, who is endowed with youth; he does not torment you so much. Still I must regard my family manners and customs coming down from ancient times. I must hold my patience.'' S\^uta said :-- Thus passionately pleased with these fascinating words of the lady, the king went to the fisherman's house for the lady. Seeing the king coming, the fisherman was greatly bewildered and astonished and bowed down with great devotion and said :-- ``O King! I am Thy servant. I am blessed by Thy presence. O great King! Now be graciously pleased to order me the cause of Thy arrival.'' Hearing the fisherman's words, the King said :-- ``O sinless one! This I tell you truly that if you give me your daughter in marriage I will certainly make her my legal wife.'' The fisherman replied :-- ``O king! What must be given ever, how can I say that is not to be given. Therefore if Thou askest for my daughter, I will certainly give her to Thee. But, O great King! Thou wilt have to make her son, the king of Thy kingdom; no other son of Thine could be king, after Thy absence.''

33-40. Hearing these words of the fisherman, the king \'Santanu became very anxious. He remembered G\^angeya and could not speak anything. He, being sick with love anxiously returned home; but he abandoned bathing, fooding, sleeping, etc. At this, the son G\^angeya Bh\^isma whose vow was equal to that of the gods, marking that the king was being troubled with some thought, went to him and asked why he was anxious :-- ``O king! Say truly what is your anxiety; who is your enemy that is not conquered; whom do you went to bring under your control? O king! What use is there of having a son who does

not understand the difficulties of his father, or does not try to remove these difficulties. A son can be called really the son, who is born to repay the debts incurred by him in previous births; there is no manner of discussion in this. See, Raghu's son D\^asarath\^i R\^am abandoned his kingdom under the orders of his father and repaired to Chitrak\^uta forest with his brother Laksman and wife S\^it\^a. The son of king Haris Chandra, Roh\^ita, ready to repay the debt of his father and sold by his father, worked as a servant at a Br\^ahmana's house. So the famous \'Sunah\'Sephah, sold by his high souled father Ajigarta was tied for sacrifice in a sacrificial post; but he was afterwards freed by the G\^adhi's son Vi\'Sv\^amitra.

41-59. It is well known that, in ancient days, the Jamadagni's son Para\'Sur\^am cut off his mother's head under the orders of his father. He considered the father's words more important, and hence could do such an unjust act. O king! This my body is at your disposal; I can certainly do what you order me to do. So say what am I to do? as long as I am living, you need not express any sorrow; if you permit, I will do what is even hardly practicable to do. O king! Say why you are anxious; I will remove that at once with this bow in my hand. If my body goes in carrying your mandate know that your desire will be fructified. Fie to that son, who, being capable, is averse to do what his father desires! What use is there in having a son who does not remove the cause of his father's anxiety? Hearing the words of the son, the king \'Santanu felt much ashamed in his heart and said :-- ``O son! This is now my gravest care that you are my only son; besides you are a hero very powerful, honoured and never showing your back in battles; therefore, if, out of ill-luck you become dead in some battlefield, I will become issueless; under such circumstances what am I to do? So, O son! My life is fruitless when I have got only one son; this is my gravest care; therefore I am sorry. O son! I have no other cares that I can mention to you.'' Hearing these words of the father, G\^angeya called the old ministers and said that the king was too ashamed to speak out to me the real matter; so I ask you all to know exactly the king's cares and communicate them to me as they are; I can carry them out, without any hitch, then. At these Bh\^isma's words, the ministers went to the king, and learned the true cause, and spoke to Bh\^isma; learning this, he began to think what ought to be done.

The Gang\^a's son Bh\^isma, then, accompanied by the ministers, quickly went to the house of the fisherman, and with words of humility and affection, spoke :-- ``O tormentor of foes! I pray to you to give your beautiful daughter in marriage to my father. Your daughter will

be my mother and I will be her servant.'' The fisherman, then, said :-- ``O highly lucky prince! Then the king's son will not be able to become king, in your presence; so kindly marry yourself my daughter.'' At this Bhisma again said :-- Let your daughter be my mother; I will never accept the kingdom. The son of your daughter will, no doubt, become king. The fisherman said :-- ``I know your words are true; but if your son be powerful, he can take forcibly the kingdom for himself.'' At this Bh\^isma again said :-- ``O Sire! Know my words as true; I will never marry; from to-day I have accepted this difficult vow.'' S\^uta said :-- Hearing this firm resolve of Bh\^isma; the fisherman gave over his beautiful daughter to the king \'Santanu. Thus \'Santanu married the dear Satyavat\^i; but he was quite unaware of the wonderful birth of Vy\^asa Deva.

Thus ends the fifth Chapter of the second Skandha on the marriage of Satyavat\^i in the Mah\^apur\^ana \'Sr\^imad Dev\^i Bh\^agavatam of 18,000 verses.

