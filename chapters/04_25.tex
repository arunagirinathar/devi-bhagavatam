\chapter{On the Dev\^i's Highest Supremacy}

1. The King said :-- O Best of Munis! Hearing these sorrows of \'Sr\^i Krisna, the part incarnate of Visnu Bhagav\^an, I am in doubt on your utterances.

2. Behold! Bhagav\^an V\^asudeva is the part incarnate of N\^ar\^ayana; how could the Asura \'Sambara steal away His son from the lying-in-chamber!

3. The beautiful Dv\^arak\^a city is specially well-guarded; the lying-in-chamber is again within the centre of that; under these circumstances, how was it that the Daitya could enter there and steal away the child!

4. O Son of Satyavat\^i! How was it that V\^asudeva could not know that! This appears very strange to me!

5. O Br\^ahmana! Please explain to me why was that child stolen away from the lying-in-chamber, though \'Sr\^i Krisna was staying there at the city; and how was it that he was not able to know this beforehand!

6. Vy\^asa said :-- O King! The M\^ay\^a called \'S\^ambhav\^i (P\^arvat\^i's) is the cause; it fascinates and deludes the minds of human beings. Thus it is known to us. Who is there in this world that is not deluded by this M\^ay\^a?

7. The J\^ivas, no sooner they are born as human beings, are immediately overcome with human qualities; the Deva or Asura qualities or their natures do not then visibly exist.

8-9. O King! Hunger, thirst, sleep, fear, lassitude, delusion, sorrow, doubt, pleasure, egoism, old age, disease, death, non-knowledge, knowledge, displeasure, envy, jealousy, pride and weariness; all these human qualities are seen to exist in human embodiments.

10-11. Behold! The night wanderer R\^aksasa M\^aricha assumed, by his M\^ay\^a, the form of a golden deer and came before \'Sr\^i R\^amchandra; and R\^amchandra was not the least aware of it. Then the stealing away of S\^it\^a, the death of Jat\^ayu, R\^ama's going to the forest on the very day of his installation to the throne of Ayodhy\^a; the death of his father due to his bereavement, all these \'Sr\^i R\^amchandra did not know a bit beforehand.

12. When R\^avana stole away J\^anak\^i and carried her by force R\^ama did not know this before or after that event had happened. He wandered from forest to forest in search of Her, like a quite ignorant man.

13. Afterwards He killed B\^ali, the son of Indra and with the help of the monkeys, erected a bridge across the ocean, and, crossing it, went to Lank\^a.

14. He sent the chief monkeys to all the quarters in search of S\^it\^a and had to undergo all the troubles of deadly battles in the great battlefield.

15. The most powerful Raghunandana was tied down by N\^agap\^a\'sa (snakes) and was afterwards freed from it by Garuda.

16. Then, being furiously enraged, the great R\^aghava slew Kumbhakarna, Nikumbha, Megha N\^ada and R\^avana.

17. The Jan\^ardan R\^amchandra was not aware of the innocence of S\^it\^a; and therefore He made her take an oath about the purity of Her character and even made Her undergo an ordeal of fire.

18. Afterwards R\^amachandra, the son of Da\'saratha, had to banish his dear blameless S\^it\^a on the mere ground of bad name, imputed to her by some ignorant person and that he would be thus blamed by the public.

19. He did not know that Ku\'s\^i and Lava were His two sons, born in the forest. Afterwards when the Muni V\^almik\^i told him, He came to know of them.

20. Behold also R\^amachandra could not know about the departure of S\^it\^a to P\^at\^ala; getting angry once He was about to kill his brother Laksmana even.

21. R\^ama, the slayer of the R\^ak\'sasa Khara did not know that K\^ala Purusa was coming to him. He, incarnating in the human body, did acts all becoming to a man. Similarly \'Sr\^i Krisna, the descendant of Yadu, taking human birth did acts all like a man. What more discussion can there be in this?

22. Lo! From the very outset He fled, out of fear of Kamsa, to Gokula; afterwards he fled out of fear of Jar\^asandha to the Dw\^ark\^a city.

23. Knowing all the rites and ceremonies of the San\^atan Dharma (the Eternal Religion) He stole away Rukmin\^i who was chosen as bride elect by \'Si\'sup\^ala. This act was very unreligious of Him.

24. \'Sambara Daitya stole away his newly born child and Krisna lamented for this. Afterwards on coming to know of the real state of things from the Goddess Bhagavat\^i, He was very glad. Therefore it can be easily seen from all these circumstances that He had to yield to pleasures and to undergo remorse like ordinary human beings.

25. Again, under the orders of his wife Satyabh\^am\^a, He had to go to Heaven to bring the P\^arij\^ata tree and He had to fight with Indra. This shows clearly that He was under the subjection of His wife.

26. In that battle Hari with disc in hand defeated Indra; the Lord of the Devas, took away the Kalpa tree and retained the prestige of His respected wife (whom He had offended).

27. Again Satyabh\^am\^a tied down Hari against a tree and presented Him as a gift to N\^arada; afterwards she, the passionate woman, freed Krisna on paying an equivalent of gold coins.

28-29. On seeing Rukmin\^i's many sons, Pradyûmna and others, all qualified with diverse qualifications, His wife J\^ambavat\^i prayed to \'Sr\^i Krisna with humility, so that she may have also many beautiful sons. For her sake, Krisna firmly resolved to practise tapasy\^a and went to the place where the great devotee of \'Siva, Upamanyu, was staying.

30. Hari desiring to have sons engaged Upamanyu as His spiritual guide and obtained from him the Mantram called P\^a\'supata Mantra and became a Dundee (holder of a staff) and shaved His head.

31-32. In the first month He subsisted on fruits only and meditated on \'Siva and repeated silently the \'Siva mantra. Thus He practised very severe austerities. In the second month He subsisted on water only and stood on only one leg. In the third month he lived on air only and stood on the end of His great toe.

33-36. Thus time passed away. In the sixth month the God Rudra, holding Moon on His forehead, was pleased with His asceticism and devotion and appeared before Him on that spot. The God Mah\^a Deva came on a bull; He was attended by Brahm\^a  and Visnu, Indra and the other Devas, Yakshas and Gandarbhas and addressed thus :-- ``O high minded Krisna of Yadu's descent; I am pleased with Your severe asceticism; now ask Your desired boon; I will grant it just now. I fulfil all the desires of all my devotees; what desire, then, there can be that is not fulfilled, when I am seen by the devotees!''

37-38. Vy\^asa said :-- The son of Devak\^i was very glad to see the God \'Samkara and fell prostrate at His feet. Then that eternal supreme God of the Devas began to recite hymns in praise of Him in a tone as deep as the rumbling of a cloud.

39. Krisna said :-- O Deva of the Devas! O Lord of the world! You alone destroy the misfortunes and sorrows of all the beings. O Destroyer of Asuras! You are the Cause and Creator of this universe. I salute Thee.

40. O One having a blue throat! I bow down to Thee! O Holder of trident! I again and again salute Thee! O Lord of P\^arvat\^i! You destroyed Daksa's sacrifice. I salute Thee.

41. I am blessed by Thy sight and think myself as having discharged all my duties and satisfied. O Virtuous One! My human birth is crowned with success by saluting Thy feet.

42. O Lord of everything! O three-eyed! I am tied down to this world by my attachment towards my wives; now I take refuge unto Thee to free me from these bonds.

43. O Destroyer of sorrows! I am very much troubled on attaining this human birth; O Bhava! I am afraid of this world; and hence I take refuge unto Thee; now save me.

44-45. O Destroyer of cupidity! I experienced a good deal of troubles in the womb; next out of fear to Kamsa I had to go to Gokula where I suffered much pains; there I had to obey the orders of cow-herds; there I had to attend as Nanda's cow-herd, the pasturing of his cows and was constantly suffocated with the awful dust thrown up by the cows; I had to wander constantly in the wild forests of Brind\^aban.

46. O Omnipresent One! I had to leave my dear ancestral place, the city of Mathur\^a, a rare place to be found anywhere else, out of the great fear of K\^ala Yavana, the king of the Mlechchas and had to go to Dw\^arak\^a city.

47-48. O Lord! In order to preserve the cause of religion, I had to hand over the best prosperous kingdom to Ugrasena, due to the curse of Yay\^ati. My elders made him the king of the Y\^adavas; following their examples, I gave him the kingdom and am now serving him always like his servant.

49. O \'Sambhu! The householder's life is exceedingly troublesome;  it makes one subject to one's wife and go against his religion. There we are always dependent on others; and no word is heard or dreamt even, how to free oneself from those bondages of the world. Oh! What an irony of Fate.

50. O Destroyer of cupid! My wife J\^ambavat\^i, on seeing the sons of my wife Rukmin\^i has urged me to practise this Tapasy\^a so that she might get excellent sons born to her also.

51. O Lord of the Devas! O Lord of the world! I am engaged in this asceticism with the desire to get sons; O Deva! I feel shame in asking you for the sons!

52. You are the lover of your devotees; You give eternal freedom; You are the Lord of all the Devas. By worshipping and satisfying You, who is so fool as to ask for this trivial and transient thing!

53. O Omnipresent One! O \'Sambhu! O Lord of the world! Knowing You as the giver of salvation, I, still deluded by M\^ay\^a, ask from You, being requested by my wife, this happiness that sons be born to me of my wife.

54-55. O \'Samkara! This world and its concerns are the abode of all sorrows; it is the cause that brings in all sorts of pains and troubles, and it is transient and will go to destruction. I know all these; still my mind does not desist from it.

56. Vy\^asa said :-- O great and powerful king! The God of Gods, Mah\^a Deva, thus praised and adored by Govinda, the Destroyer of enemies, replied :-- You will get many sons.

57. You will get sixteen thousand one hundred wives and no doubt you will get ten sons of each of them. These sons will be very powerful and valorous.

58-60. The good-looking \'Samkara saying these words remained silent; then \'Sr\^i Krisna bowed down at the feet of Girij\^a, the wife of \'Samkara. Then the Goddess P\^arvat\^i addressed repeatedly to V\^asudeva and said :-- O mighty armed! O Krisna! O best of human beings! You will be the typical exemplary householder; (all people will try to follow you). When one hundred years will pass away, your race will be extinct, due to the curse of the Br\^ahmana and G\^andh\^ari.

61. Your sons and the other Y\^adavas will lose their senses on drinking liquor; they will kill each other in the battle field and thus will be extirpated\footnotemark.

\footnotetext{Here Visnis and Andhkas are meant.}

62. Then you and your elder brother Balabhadra (Balar\^ama) will give up your bodies and will ascend to the Heavens; O Mighty Person! Do not grieve in matters that cannot be avoided.

63. You should know that there can be no remedy to what will inevitably come to pass; therefore no one is to grieve for them; this is all along my view.

64. O Madhusûdana! After Your death, due to the curse of Ast\^avakra Muni, your wives will be forcibly stolen away by indomitable robbers. There is no doubt in this.

65. Vy\^asa said :-- When Dev\^i P\^arvat\^i thus spoke, \'Sambhu, with the other gods disappeared; Krisna too, bowed down to Upamanyu and went back to the city Dv\^ark\^a.

66-67. Therefore, O King! Though Brahm\^a  and the other Devas are heard to be the lords of the world, still they are all being tossed hither an thither by the waves of the ocean of M\^ay\^a. They are all like wooden dolls subject to M\^ay\^a.

68. As their previous karmas, so their several manifestations in the field of action, by the Great M\^ay\^a, the incarnate of Par\^a Brahm\^a.

69. She has no differences nor any want of mercy; That Goddess of the universe is always leading the J\^ivas towards the Eternal Freedom (freedom from M\^ay\^a).

70. Had She not created this world, moving and unmoving and if She had not remained there as the Controller of the J\^ivas in the shape of unshakeable consciousness the Kûtasthya Chaitanya, this whole world would have become devoid of any consciousness, like an insentient substance and would have dissolved in the T\^amas\^i M\^ay\^a (sheer darkness). There is no doubt in this.

71. Therefore that Goddess of the Universe has, through Her mercy, created all these worlds and J\^ivas, and resting incarnate in each J\^iva, is directing each and every of them according to his karmic merits and demerits.

72. Therefore it is a matter not to be doubted that Brahm\^a  and the other gods are all under this M\^ay\^a; the Suras and Asuras are subject to Her.

73. Therefore, O king! Know this as certain that the Great Goddess moves and enjoys freely according to Her will; She is not dependent on anybody. Therefore it is the duty of everyone to serve and worship, with whole head and heart, that Dev\^i.

74. In these three worlds there is nothing higher or more excellent than Her. Therefore this birth cannot be crowned with success in any other way than remembering that Highest Force, the Par\^a \'Sakti and Her place.

75-77. One should always think, without any difference, that Eternal World Mother, thus ``Let me not be born in that family which has not that Supreme Goddess for its presiding Deity; I am that Goddess Bhagavat\^i and no other; I am Brahm\^a, untouched by sorrows.'' One should hear first from the mouth of one's Spiritual Guide; next by hearing Vedanta and other religious scriptures, one should first form an idea of that Bhagavat\^i; and then if one daily meditates on That Goddess, the Highest Self incarnate with one minded devotion, one will get, within a short period, the Eternal Freedom; else there is not the least chance, even if one performs lots of innumerable good works of becoming free.

78. \'Svet\^a\'svatara and other pure hearted Risis obtained this freedom from the bondages of M\^ay\^a by meditating, in their hearts, this Highest Self and nothing else.

79. Brahm\^a, Visnu and the other Devas, Gaur\^i, Laksm\^i and other goddesses, all worship This Supreme Goddess, of Sachchid\^ananda Par\^a Brahm\^an\^i.

80. O pure-hearted king! I answered all that you asked me, terrified with the fears of this world; what more do you want to hear?

81-82. O king! I have described this wonderful Pur\^ana narrative, destructive of sins, productive of virtue. He who daily listens to this Bh\^agavatam equal alike to Veda, becomes freed from all sorts of sins and goes to the region of the Highest Goddess and passes his time in the midst of the Highest Glory. There is no doubt in this.

83. Sûta said :-- ``O Risis! This Sr\^i Mad Bh\^agavatam, called otherwise the Fifth Pur\^anam was recited, in detail, in days of yore by Vy\^asa. Whatsoever I heard from him, I have now told exactly the same to you.''

Here ends the 25th Chapter in the Fourth Skandha of Sr\^i Mad Dev\^i Bh\^agavatam, the Mah\^a Pur\^anam, of 18,000 verses, by Maharsi Veda Vy\^asa on the Dev\^i's Highest Supremacy.

Note :-- The best mantra is the whole hearted devotion to one's Guru, and devotion and surrender of one's Self to the Supreme Mother, doing works without attachment to the fruits thereof. This will lead to dispassion and Renunciation. To one who is faithful in this, all the other mantras will be duly revealed and all his desires will be found to be true and fulfilled.

Here ends as well as the Fourth Skandha.