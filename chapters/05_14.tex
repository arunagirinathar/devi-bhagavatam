\chapter{On the killing of T\^amra and Chiksura}

1-4. Vy\^as\^a said :-- Hearing the death news of Durmukha, Mahis\^asura became blind with anger and began to utter repeatedly to the D\^anavas, ``O! What is this? What is this? Alas! That delicate woman has slain in battle the two heroes Durmukha and V\^askala! Lo! Now look at the wonderful workings of the Daiva (Fate). It is the acts virtuous, or otherwise that make men dependent; and the powerful Time awards pleasure or pain accordingly. The two powerful Demons are killed; what are we to do hereafter? You all judge and say what is reasonable at this critical juncture.''

5-23. Vy\^asa said :-- When the powerful Mahisa said thus, his general Chiksura, the great warrior spoke as follows :-- ``O King! Why are you so anxious as to take away the life of a delicate woman? I will kill Her;'' thus saying he departed for battle, mounted on his chariot and accompanied by his own army. The powerful T\^amra accompanied him as his attendant; the sky and all the quarters became filled with the clamour of their vast army. The auspicious Dev\^i Bhagavat\^i saw them before Her and She made an extraordinary wonderful sound with Her conchshell, with Her bow string and with Her great bell. The Asuras heard that and trembled and fled, speaking amongst each other, ``What is this?'' The Chiksur\^aksa seeing them turning their backs, told them very angrily, ``O D\^anavas! What fear has now overcome you? I will slay today this vain woman in the battle with arrows; so you should quit your fear and remain steady in battle.'' Thus saying, the D\^anava Chieftain Chiksura came fearlessly before the Dev\^i with bows and arrows in his hands and, accompanied by his army, angrily spoke thus :-- ``O Thou of large and broad eyes! Why are you roaring to terrify the weak persons! O the Soft limbed One! I have heard all about your deeds but I am not a bit afraid of You. O One of beautiful eyes! It is a matter of disgrace, rather sin, to kill a woman; knowing this my heart wants to pass over this act (does not like to do it, if my purpose be served without it).

O Beautiful One! The women fight with their side glances and amorous gestures; but I have never heard a woman like you coming to fight with arms and weapons. Even the delicate flowers, M\^alati, etc., cause pain on the bodies of beautiful women like you; so it is not advisable to fight against you with flowers even; what to speak of sharpened arrows! Fie on those who spend their lives according to the Ksatriya Dharma! Oh! Who can praise that Dharma which allows this dear body of ours to be pierced by sharpened arrows? This dear body is nourished by sweet food, by being smeared with oil, and by smelling the scents of beautiful flowers. Ought, then, one to destroy it by arrows from an enemy? Men get their bodies pierced by arrows and then become rich. Now is it possible for the riches to give pleasure afterwards when they caused such pains in the beginning? Even if this be so, fie on those riches! O Beautiful One! It seems you are not intelligent; why have you desired to fight instead of to enjoy sexual pleasures. O beautiful! What merits have you found in the battle that you have chosen this. Where you see the action of the axes and spears, striking each other with clubs, and hurling of sharpened arrows and weapons and where, when death comes, jackals come and feed upon the dead bodies, what merits have you been able to trace out in these things! It is only those cunning poets that praise these; they say that those who die in battle go to heaven! O Beautiful! Those sayings are, no doubt, mere flatteries. Therefore, O Excellent One! Go away anywhere else you like; or accept this king Mahisa, the tormentor of the Devas, as your husband.''

24-30. Vy\^asa said :-- O King! The D\^anava Chiksura speaking thus, the Divine Mother addressed him thus :-- O Stupid! Why are you speaking false words, having no significance, like a literary man giving out mere words only? You do not know anything of politics, ethics, metaphysics; you serve the illiterate and stupid; therefore, you are also a first class illiterate; you do not know what are the royal duties; then what are you speaking before me? I will kill that Mahis\^asura in battle make the soil muddy with his blood, thus establish firmly My pillar of Fame and then go happily to My abode. Surely will I slay that vain vicious demon, the tormentor of the Devas. Better fight steadily. O Stupid! Better go to P\^at\^ala with all the D\^anavas, if you and Mahisa desire to live any longer. And if you like to go unto death, then be ready and fight without any delay; I will slay you all; this is My firm resolve.

31-39. Vy\^asa said :-- O King! Hearing the Dev\^i's words, the D\^anava, proud of his own strength, began to hurl instantly on Her showers of arrows, as if another shower of rain burst upon Her. The Dev\^i cut off those arrows into pieces by Her sharp arrows and shot at him dreadful arrows like poisonous snakes. Then their fight became astounding to the public; the Divine Mother, then, struck him with Her club so much that he fell down from his chariot. That vicious demon, thus struck by the club, remained senseless near to his chariot for two muh\^urtas, fixed like a mountain. T\^amra, the tormentor of the foes, seeing him thus, could not remain steady and came forward to fight with Chandik\^a. The Dev\^i seeing him laughed and said, ``O D\^anava! Come, Come, I will instantly send you unto death. Or, what is the use of your coming? You are so weak that you can be called lifeless. What is that stupid Mahisa doing now? Is he thinking out the way to save his life? You all are too weak; no use in killing you, all my labours will go in vain, if that wicked Mahisa, the enemy of the gods, be not slain. Therefore, do you go to your home and send here your king Mahisa. I am staying here in that form in which that wicked one likes very much to see Me.''

40-56. Hearing Her words, T\^amra became very angry and drawing his bow up to his ear, began to hurl arrows after arrows on Chandik\^a Dev\^i. The Bhagavat\^i, too, had her eyes reddened with anger and drawing Her bow began to shoot arrows quickly at the demon, wishing to kill, as early as possible, the enemy of the gods. In the meanwhile, Chiksura regained his senses, and taking up again his bow in an instant, came before the Dev\^i. Then Chiksura and T\^amra, the two valiant warriors, began to fight dreadfully with the Dev\^i. Mah\^a M\^ay\^a then, became very angry and began to hurl arrows after arrows so incessantly that all the armours of all the D\^anavas became pierced and were cut down to pieces. The Asuras, thus pierced by arrows, became infuriated with anger and hurled angrily a network of arrows upon the Dev\^i. The D\^anavas, thus struck with sharp arrows and filled with cuts and wounds looked like the red Kim\'suka flowers in the spring. The fight then grew so severe between T\^amra and Bhagavat\^i that the seers, the Devas, were all struck with wonder. T\^amra struck on the head of the lion with his dreadful hard Musala (club), made of iron, and laughed and shouted aloud. Seeing him thus vociferating, the Dev\^i became angry and cut off his head by her sharp axes in no time. The head being thus severed from the body, T\^amra, though headless, for a moment turned round his Musala and then fell down on the ground. The powerful Chiksura, seeing T\^amra thus falling down, instantly took up his axe and ran after Chandik\^a. Seeing Chiksura with axe in his hand, the Bhagavat\^i quickly shot at him five arrows. With one arrow, his axe was cut down, with the second arrow his hands were cut and with the remaining ones his head was severed from his body. Thus when the two cruel warriors were slain, their soldiers soon fled away in terror in all directions. The Devas were exceedingly glad at their downfall and showered gladly flowers from the sky and uttered shouts of Victory to the Dev\^i. The Risis, Gandarbhas, the Vet\^alas, the Siddhas and Ch\^aranas were all very glad and began to utter repeated, ``O Goddess! Victory, victory be Yours.''

Here ends the Fourteenth Chapter of the Fifth Skandha on the killing of T\^amra and Chiksura in \'Sr\^i Mad Dev\^i Bh\^agavatam, the Mah\^a Pur\^anam, of 18,000 verses by Maharsi Veda Vy\^asa.