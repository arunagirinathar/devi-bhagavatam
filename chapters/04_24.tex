\chapter{On the stealing away of Pradyûmna}

1-2. Vy\^asa said :-- On the other hand, there, at the house of Nanda, early in the next morning, commenced the grand birth day festivity. Kamsa came to know, afterwards, by his spies as will as by general rumour, that at Gokula, in the house of Nanda, a very joyous festival on a grand scale was being performed; he also knew before that the other wives of V\^asudeva, his animals and his servants were all staying at Nanda's residence in Gokula.

3-4. O Bh\^arata! Thus all these gave reasons to Kamsa to suspect the place Gokula. Especially N\^arada told him before that the residents, the cowherds at Gokula, Nanda and others, their wives, Devak\^i and V\^asudeva were all Devas incarnates; and consequently they were his enemies.

5-7. Thus being brought to more confidence by N\^arada's words, that vicious Kamsa, the disgrace to his family, was very angry and sent there his demons Pûtan\^a, Baka, Vatsa, the great Asura, the powerful Dhenuka, Pralamba. They were all killed by Krisna, of unsurpassable prowess. \'Sr\^i Krisna held aloft also the hillock Gobardhana (to protect the cow herds and cows, etc.) Hearing all these, Kamsa became certain also of his own death.

8. Lastly when the evil minded Kamsa heard that the Daitya Ke\'s\^i was also slain, then he made arrangements for a sacrifice, known as Dhanuryajña; and under this pretext wanted to bring over there at Mathur\^a the two brothers Krisna and Balar\^ama.

9. The evil minded Kamsa to effect the death of those two, R\^ama and Krisna, of unsurpassable prowess, sent Akrûra to Gokula to bring them over to Mathur\^a.

10. Akrûra, the son of Gandin\^i, under the orders of Kamsa, went to Gokula and brought the two boys on a chariot to Mathur\^a.

11-12. On arriving at Mathur\^a, R\^ama and Krisna first broke the bow; killed R\^ajaka, the elephant Kubalaya, Ch\^anûr, Mustika, \'Sala, To\'sala and other athletes and warriors. Last of all, Hari, the Lord of the Devas, holding Kamsa by his hair, killed him with utmost ease.

13. The enemy-destroyer Krisna removed the sorrows of his father and mother and released them from their prisons and gave over the kingdom of Mathur\^a to Ugrasena (the father of Kamsa).

14-15. The high minded V\^asudeva, then, with the triple girdle made of Munja grass, performed the Upanayana ceremonies (wearing the sacred thread round the body) of R\^ama and Krisna and made them accept the vow of Brahmacharya. They then departed to the hermitage of the holy Muni \'Sand\^ipana to acquire knowledge. Learning all the vidy\^as (knowledges) there, they returned quickly to Mathur\^a.

16. The two sons of \^Anakadundubhi stayed there and on attaining their twelfth year, became proficient in all the branches of learning and became very powerful.

17. That time Jar\^asandha, being grieved at the killing of his son-in-law Kamsa, collected a strong and numerous army and marched to Mathur\^a.

18. Seventeen times Jar\^asandha, the king of Magadha, attacked Mathur\^a and seventeen times he was defeated by the ingenuity of that highly intelligent \'Sr\^i Krisna, who was of firm resolve and was then residing in Mathur\^a.

19. Lastly, Jar\^asandha sent the K\^alayavana (Black Yavana) to invade Mathur\^a. These Yavanas were brave and the lords of all Mlechch\^as (untouchables) and extremely terrific to the Y\^adavas.

Note: K\^alayavana - A king of Yavanas\footnotemark and enemy of Krisna and an invincible foe of the Y\^adavas. Krisna finding it impossible to vanquish him in the field of battle, cunningly decoyed him to the cave where Muchukunda was sleeping who burnt him down.

\footnotetext{Yavana means a Greek, an Ionian; then any foreigner, or barbarian (the word is applied at present to a Mahomedan or a European also).}

20-21. Hearing that K\^ala Yavana was coming to attack the Y\^adavas, Krisna the destroyer of M\^adhu, called all the Y\^adavas and Baladeva and addressed them thus :-- ``O blessed ones! Now a cause of great terror has appeared amongst us; K\^ala Yavana is being sent by our powerful enemy Jar\^asandha to attack Mathur\^a. Now what to do? It is better to save one's life by leaving aside all our homes, wealth and army.

22. You should all know that is the place of our fathers and forefathers where we can safely and happily dwell; where there is a constant source of anxiety and uneasiness that, though the place of our fathers and forefathers, ought to be avoided; never ought anyone to dwell there.

23. If you want to dwell at ease and comfort, you ought to dwell in that country or place which is adjacent to a sea or a mountain; where there is no fear from an enemy, the sages will always remain there.

24. See! The Bhagav\^an Hari, being afraid, as it were, of his enemy has taken refuge on the body of the thousand headed \'Sesa serpent as his sleeping place and is sleeping at ease and comfort on the ocean. It seems likely that the enemy of Tr\^ipur\^a, the great \'Siva is also dwelling on the Kail\^a\'sa mountain.

25. We, too, are being constantly worried by our enemies here; therefore we ought not to live here any longer. We should all go to Dw\^ark\^a city with our friends, relatives and wealth.

26. Garuda, the king of the birds, has given us the detailed information of the city Dw\^ark\^a. That beautiful city is situated on the sea shore in the vicinity of the Raivataka mountain.''

27. Vy\^asa said :-- The Y\^adava chiefs, hearing \'Sr\^i Krisna's utterances fraught with their welfare, were ready to depart to that place Dw\^ark\^a, attended by their friends, relatives, and appurtenances.

28. They then collected their camels, mares, and buffaloes and filled their conveyances with wealth, gems and precious stones and marched out of their place.

29. R\^ama and Krisna went in front; the Y\^adavas and other subjects then marched in groups (several parties).

30. Marching some days, they all reached Dv\^ar\^avat\^i. Then the portions of the city that were dilapidated or destroyed, \'Sr\^i Krisna had them repaired by engineers, artisans and craftsmen.

31. Placing the Y\^adavas there, Ke\'sava and Baladeva quickly returned to Mathur\^a and began to stay in that desolated city.

32. The extremely powerful king of the Yavanas arrived then at  Mathur\^a. Krisna knowing that the Yavana chief had come there, went out of the city.

33. The Bhagav\^an Madhusûdana, the destroyer of the boastings of Asuras and other people, dressed in yellow robes, appeared on foot before the K\^alayavan with smile on his lips.

34. Seeing the lotus-eyed Krisna before him, the treacherous Lord of the Yavanas, pursued him on foot to catch hold of him.

35. Where the powerful R\^ajarsi Muchukunda was sleeping soundly, the Bhagav\^an Hari led K\^alayavana there.

36. There \'Sr\^i Krisna, saw Muchukunda and vanished away at once; the king of the Yavanas on arriving there found the R\^ajarsi (the royal sage) there in deep sleep.

37. The wicked Yavana mistaking Muchukunda for \'Sr\^i Krisna, gave him a good kick. The powerful king Muchukunda got up and was very angry; his eyes became red and reduced that vicious Yavana instantly into ashes.

38. When Muchukunda burnt the Yavana, he saw the lotus-eyed Krisna; he bowed down to that Supreme Deva, V\^asudeva, and went to forest.

39. \'Sr\^i Krisna then went back to the city Dw\^ark\^a with R\^ama and made Ugrasena there the king and began to enjoy at his will.

40. At the marriage ceremony of \'Si\'sup\^ala, at the palace of the king of Vidarbha, Jan\^ardan Visnu carried away by force Rukmin\^i, the bride elect from the Svayambara assembly (where the husband is self elected by the bride herself) and afterwards married her according to the rule called R\^akhsasa Vidhi (one of the eight forms of marriage in Hindu Law in which a girl is forcibly seized and carried away after the defeat or destruction of her relatives in battle).

41-42. Afterwards He brought also J\^ambavat\^i, Satyabh\^am\^a, Mitravind\^a, K\^alind\^i, Laksman\^a, Bhadr\^a, and auspicious N\^agnajit\^i (the daughter of the king Nagnajit) on various occasions and married them. O Lord of the earth! These eight women were the best and most beautiful of \'Sr\^i Krisna's wives.

43. Rukmin\^i first gave birth to the beautiful child Pradyûmna and \'Sr\^i Krisna performed the religious ceremony at the birth of his child.

44. Then the powerful D\^anava named \'Samvara stole away the little baby from the lying-in-chamber and carried him to his own city and made him over under the charge of M\^ay\^avat\^i.

45. Coming to know that His son had been stolen away, \'Sr\^i Krisna became very much overpowered with sorrow and took the shelter of the Supreme Goddess, the Dev\^i, with a heart full of devotion.

46-47. \'Sr\^i Krisna then began, to chant, in sweet auspicious tone, hymns in alphabets, conveying the highest meanings, in adoration of the Yoga M\^ay\^a, Who slew Vritr\^asura and other Daityas with ease and alacrity.

48. O Mother! I, in my former birth as the son of Dharma, appeased You by my ascetic practices in the hermitage of Badari and worshipped You with various offerings; O Mother! Have you now forgotten all my devotion to You?

49. O Mother! Has any evil minded enemy stolen away my son from the lying-in chamber? Or have You Yourself done this to make a fun and see the amusement? It seems that some one of my enemies has done so to insult me; however, You, O Mother! ought not to put your devotee under this shameful condition.

50. O Mother! This Dw\^arak\^a city is well guarded; a very strong fort is built in its middle and my place is in the midst of that again; and the lying in-chamber is again in the middle; I therefore must say that it is due to my bad luck that the child is stolen away!

51. O Mother! I did not go to the house of my enemy; the Y\^adavas also did not go there; this city is guarded by valiant soldiers; then how is it, under what charm, the baby has been stolen? O Mother! Now I come to know that it is due to Your M\^ay\^a; such things are common due to Your M\^ay\^a in the three worlds.

52. O Mother! When I am ignorant of your deepest mysteries, how can there exist anyone among the little minded J\^ivas that can know your doings? My watchmen could not see anything, where my child was taken away and who has stolen it. O Mother! I come to the conclusion that it is hidden behind the screen of Your M\^ay\^a.

53. O Mother! It is not strange with You; to the chaste woman, Rohin\^i Dev\^i, though situated at a great distance and not connected with any male persons, You, in the fifth month, moved away the son to my knowledge from the womb of my mother; and thus Baladeva was born to Rohin\^i. This is now known to all.

54. Mother! You are incessantly creating, preserving, and destroying this whole universe by the mixture of the three qualities. Who can know Your sin-destroying doings? Mother! There is no need of dwelling at length. Suffice it to say that You, no doubt, are doing all that are being done in this whole universe.

55. You first create the joy at the birth of a child; again You load us with heavy burdens of sorrows due to the separation from that child; thus you are always sporting; otherwise how my joy at the birth of my child would thus be rendered quite useless?

56. The mother of that child is always weeping like an ewe, straying from a flock; she is giving vent to her sorrows always to me; O Kind-hearted! Being thus endowed with illimitable prowess and understanding, do You not know my troubles! O Mother! You are the only source of consolation to one, suffering from the sorrows of this world. There is no doubt in this.

57. O Goddess! The wise seers say that the birth of a child in any house is the highest bliss there, and the death of a child is the greatest sorrow that can befall to any house. Therefore, O Mother! What shall I do in this? What shall I say more than that my heart is going to burst, due to the disappearance of my child.

58. O Mother! I will perform all the necessary sacrifices, take up vows, perform all sorts of worship to the entire satisfaction of the Great Fate (Ordainer of things); You be pleased to remove my sorrow. O Mother! If my son be alive, kindly shew him once to me. Mother! There is no other than You Who is fully capable to destroy this my pain and sorrow, raging in my heart.

59. Vy\^asa said :-- He who brings into practice, things that are considered impracticable for the Devas and removes the load of the Goddess Earth with ease and alacrity, the same Saviour of world, \'Sr\^i Krisna thus chanted hymns in adoration of the Great Goddess. The Dev\^i then became visible to him and said.

60. O Lord of the Devas! Do not any longer be sorrowful and miserable; there had been a curse on you before; and, for that reason, the Daitya \'Sambara has stolen away your son by his demonic magic.

61. Therefore, when your son will grow sixteen years old, then he will, by My Grace, kill the Daitya perforce and will return to you. There is no doubt in this.

62. O king! Thus saying these words full of hope and confidence, the Great Goddess Chandik\^a, of formidable prowess, disappeared. Krisna too, quitted his sorrows, due to the bereavement of his child, and began to spend his time in happiness and peace.

Here ends the Twenty-fourth Chapter of the Fourth Book of \'Sr\^i Mad Dev\^i Bh\^agavatam, the Mah\^a Pur\^anam of 18,000 verses by Maharsi Veda Vy\^asa on the stealing away of Pradyûmna.