\chapter{On the part incarnations of the several Devas}

1. Janamejaya asked :-- O grandfather! What bad act did that child commit, that no sooner he was born than he was killed by Kamsa?

2. Especially, Maharsi N\^arada is the the best amongst the Munis and foremost amongst the Brahm\^a-vids (Knowers of Brahm\^a), always doing virtuous acts, and learned; why did he become the agent in this very sinful act?

3. Pundits declare that the doers and stimulators of any evil deed both are equally responsible; then how is it that N\^arada, being the best of the Munis, instigated the wicked Kamsa to do this evil act!

4. I am very much in doubt on this point. Kindly describe, in detail, the act that the child did as the result of which be had to meet with this fate of being killed.

5. Vy\^asa said :-- The Devarsi N\^arada is always fond of seeing quarrels brought about amongst parties; he always likes thus to see the fun. Here specially to serve the god\'s purpose he went to Kamsa and incited him to such an act.

6. Really he never intends to speak a lie; he is always truth speaking; pure hearted, and always ready to serve the gods.

7. Thus the six sons were born to Devak\^i; and Kamsa, too, killed those six sons consecutively as they were born. These six sons named Sadgarbha, were killed just after their births, owing to their having been previously cursed.

8. O King! Hear why they were cursed before. In the reign of Sv\^ayambhuva Manu, were born to Urn\^a; the wife of Maharsi Mar\^ichi, the six powerful sons, all of a virtuous disposition.

9-11. Once, on an occasion, the Praj\^apati Brahm\^a, on seeing his daughter, became passionate, and was ready to hold sexual intercourse with her. At this, those six sons laughed at him. Brahm\^a cursed them saying ``You all go quickly and take your birth in the wombs of the asuras.'' Therefore those six sons became the sons of K\^alanemi in their first birth. At their second birth, they became the sons of Hiranyaka\'sipu. This second time they had the fear of curse in their minds and therefore were born endowed with knowledge.

12. In this birth they became peaceful and, collecting all their energies, they began to practise austerities. Brahm\^a was pleased at this and asked the Sadgarbha to take boons.

13. Brahm\^a said :-- O my sons! I was very angry to you before and cursed you; now I am very much pleased with you; ask boons from me that you all desire.

14-15. Vy\^asa said :-- Hearing Brahm\^a's words, they were very glad and becoming very anxious to secure their objects of desire, said :-- O our grand sire! Today thou art pleased unto us; now favour us with our desired boons. That we may be invulnerable to all the Devas, human beings, the big serpents the Gandarbhas, and the Lord of Siddhas, (semi-divine beings supposed to be of great purity and holiness and said to be particularly characterised by eight supernatural faculties called Siddhis).

16. Vy\^asa said :-- Brahm\^a told them ``What you have asked, you would certainly get; O blessed ones! better go now; my words will be found to be literally true. No doubt in this.''

17-19. Granting them boons, Brahm\^a went away; they then became very glad. O best of Kurus! Hiranyaka\'sipu began to think ``My sons now have pleased the Grandsire Brahm\^a and are now regardless of me'' and got very angry and said :-- You all are become very proud on account of receiving boons; and since you have ceased your good feelings towards me I also henceforth cut off my connection with you. Now better go to P\^at\^ala; you will be known in this world as Sadgarbha.

20-21. At present you would be always involved in deep sleep and remain in P\^at\^ala for many years; and when you will be born one after another in the womb of Devak\^i, then your father K\^alanemi of previous birth will be born as Kamsa; and he would be cruel hearted and surely kill you all, no sooner you be born.

22. Vy\^asa said :-- Thus because they were cursed, they took their births repeatedly and Kamsa, too, being urged on by the same curse, killed those sons of Devak\^i, the Sadgarbha, no sooner they were born.

23-24. In the seventh womb of Devak\^i, Ananta made his appearance. The foetus in the womb was attracted by Yoga m\^ay\^a and placed in the womb of Rohin\^i. But there was the rumour that there was miscarriage in the womb of Devak\^i in the fifth month; and this became known to the public.

25. Kamsa came to know that there had been miscarriage. That wicked soul became exceedingly glad to hear this gladdening news.

26. And at about this time the Bhagav\^an, the Protector of the devotee appeared in the eighth womb of Devak\^i to serve the purpose of the gods and to relieve the load of the Earth.

27-28. The King said :-- O best of Munis! ``You have described the part incarnations of (1) Ka\'syapa as V\^asudeva and (2) of Bhagav\^an Hari to relieve the burden of the Goddess Earth as prayed by Her; and (3) of Ananta Deva; but you have not described the part incarnations of the other Devas. How the other Devas incarnated as their parts on this earth, kindly describe them now.''

29. Vy\^asa said :-- The part incarnations of Suras and Asuras on this earth, and their names I am now saying to you in brief; hear.

30-32. V\^asudeva was the part incarnation of Ka\'syapa, Devak\^i was of Aditi, Baladeva, of Ananta; V\^asudeva \'Sr\^i Krisna, of \'Sr\^im\^an N\^ar\^ayana; the son of Dharma existing even at that time in his physical body; Arjuna, of Nara, the younger brother of N\^ar\^ayana.

33. Yuidhisthira was part incarnate of Dharma, Bhimasena, of V\^ayu, the powerful twins of M\^adri, Nakul and Sahadeva, of A\'svin\^i-kum\^aras?

34. The valiant hero Karna, born of Kunt\^i, was part incarnate of the Sun, and the high minded Vidura, the knower of the Supreme Essence, was incarnate of Yama, the king Dharmar\^aj. Drona, the \^Ach\^arya of the Kurus and the P\^andavas was the part incarnate of Brihaspat\^i; and his son A\'svatth\^am\^a was part incarnate of Rudra Deva.

35. \'Santanu was the part incarnate of the Ocean; his wife, of the river Ganges in human farm. It is stated in the Pur\^anas that the king Devaka was part incarnate of the Lord of Gandarvas.

36-41. The Grand-father of the Kauravas, the foremost of the heroes, Bh\^isma Deva was the incarnate of Vasu; Vir\^ata, the Lord of Matsya was the part incarnate of Maruts; Dhritar\^astra, of the Daitya Hamsa, the son of Arista Nemi; Kripa and Krita Varm\^a, of Maruts; Duryodhana, of Kali and \'Sakuni, of Dv\^apara; Suvarch\^akhya Somapraru, of the son of the Moon; Dhristadyumna was part incarnate of Fire and \'Sikhand\^i of R\^aksasa; Pradyumna was part incarnate of Sanatkum\^ara; the king Drupada was part incarnate of Varuna; Draupad\^i, of Laksm\^i; Draupad\^i's five sons, of Visve-devas; Kunt\^i was incarnate of Siddhi; M\^adri, of Dhriti; G\^andh\^ar\^i, of Mati; the wives of \'Sr\^i Krisna were the heavenly public women; thus all the Devas came as their part incarnations, urged on by Indra.

42-43. Amongst the Asuras, \'Si\'sup\^ala was the incarnate of Hiranyaka\'sipu; Jar\^asandha, of Biprachitti, \'Salya, of Prahl\^ada; Kamsa, of K\^alanemi and Ke\'s\^i, of Haya \'Sir\^a. The Asura named Arista of the form of a cow that was killed by Krisna was the son of Bali.

44. Dhristaketu was part incarnate of Anuhr\^adha, Bhagadatta, of V\^askala; Pralamba, of Lamba; Dhenuka, of Khara.

45. Ch\^anûra and Mu\'stika, the two athletes, were part incarnates of V\^ar\^aha, and Ki\'sora, the two dreadful Daityas.

46-47. Kubalaya, the elephant of Kamsa, was part incarnate of Arista, the sun of Diti. Vak\^i was the daughter of Bali, Vaka was her younger. The powerful son of Drona, A\'svatth\^am\^a, though known as the part incarnate of Rudra, was really born of the four parts of Yama, Rudra, Cupidity and Anger.

48-49. The Daityas and R\^aksasas that were born to relieve the heavy burden of the Earth were all incarnates of Asuras. O king! I have thus narrated to you in order the incarnations of the Suras and Asuras, as they are stated duly in the Pur\^anas.

50-51. When Brahm\^a and the other Devas went to Visnu and prayed to Him then Hari gave to Brahm\^a one hair of a black colour and one hair of a white colour. The Bhagav\^an \'Sr\^i Krisna was born of that black hair and \'Sankarsana Baladeva was born of the white hair. They were both the incarnations of Visnu.

Note here the black is the younger and the stronger; and they also represent the polarities. The J\^ivas are points of those hairs.

52. He who hears with devotion the story of these part incarnations becomes freed of all sins and passes away his time merrily, surrounded by his circle of friends; there is no doubt in this.

Thus ends the 22nd chapter in 4th book of \'Sr\^imad Dev\^i Bh\^agavatam the Mah\^a Pur\^anam, of 18,000 verses by Maharsi Veda Vy\^asa on the part incarnations of the several Devas.