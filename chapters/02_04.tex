On the birth of the Vasus

 

p. 87

 

1-8. Sûta said :-- On the king Pratîpa, ascending to the Heavens, the truly hero-king S’antanu went out a hunting tigers and other forest animals. Once, while he was roaming in a wild wilderness, on the banks of the Ganges, he saw a fawn-eyed well decorated beautiful woman. No sooner the king S’antanu saw her than he became addicted to her and thought within himself thus :-- “Certainly my father spoke of this beautiful faced woman who is looking like a second Laksmî, endowed with beauty and youth.” The king could not rest satisfied simply with seeing the lotus-like face. The hairs on his body stood on their ends and his heart was very much attracted to her. Gangâ Devî, too, knew him to be the king Mahâbhisa and became, in her turn, very much attached to him. She then went smiling towards the king. Seeing the blue-coloured lady looking askance at him, the king became very happy and consoled her in sweet words and said :-- “O, one of beautiful thighs! Are you Devî; Mânusî (human kind) Gandharvî; Yakshî, the daughter of Nâgas (serpents), or a celestial nymph? Whoever you may be, O beautiful one! be my wife; your sweet smiles, it seems, are brimful of love; so be my legal wife to-day.”

 

9-26. Sûta said :-- The king S’antanu could not recognise the lady to be Gangâ; but Gangâ knew that he is the king Mahâbhisa and is born as S’antanu. Hearing the above words of the king, Gangâ, out of her previous affections, spoke out to the king, smiling :-- “O king! I know that you are the son of the king Pratîpa. Behold! Though it is inevitable that woman will get their husbands, yet who is that beautiful lady that does not husband according to her liking and qualifications? But I can take you as my husband, if you make a certain promise to me. Hear my resolve afterwards I will marry you. O king! Whatever I will do, be it good or bad, auspicious or inauspicious, you must not hinder or interrupt me

 

p. 88

 

nor ever say that is not to your liking and satisfaction. Whenever you will break this my resolve, I will instantly quit you and go to another place wherever I like. The king S’antanu, then, said :-- “Well! That will be” and promised to the above effect; then Gangâ Devî recollected within herself Vasus' words and thought of the attachment of the king Mahâbhisa and accepted S’antanu as her husband. Thus married to the king S’antanu, the beautiful Gangâ in human form went to his abode. The king, on having got possession of her, began to enjoy in pleasant gardens. The lady, too, appreciated his mental feelings and began to serve him to his satisfaction. Thus many years elapsed in lovely enjoyments and intercourses between the couple who looked like Indra and his pair Sachî; and they did not feel at all how the time passed. The lady endowed with all qualities and the clever king, well-versed in the art of loving, began to enjoy incessantly like Laksmî and Nârâyana, in their divine palace.

 

Thus many years passed when the lovely eyed lady became pregnant of the king S’antanu's sperm and, in due time, gave birth to a son who was a Vasu. No sooner the son was born than Gangâ Devî threw it in the waters of the Ganges. Thus the second, third, fourth, fifth, sixth and seventh sons were threwn successively on the waters. Then the king became very anxious and thought within himself :-- “What am I to do now? How my family be preserved? This my wife, incarnate of sin, has killed my seven sons; if I now desist her, she will instantly leave me and go away. Now this is the eighth pregnancy as desired by me. Now if I do not interrupt her, she will certainly throw my son in the Ganges. Whether a son will be born again or not is doubtful; and even if that be born, it is doubtful whether she will preserve that child; now what am I to do in this doubtful point? However I will try my best to continue the thread of my family line.”

 

27-46. Now, in due time, the Vasu who, having been influenced by his wife had stolen Vas’istha's cow Nandini, became born as the eighth son of Gangâ Devî; the king S’antanu, seeing this son fell unto Gangâ's feet and said :-- “O thin-bodied woman! I pray to you to give my life to-day; better nourish this my one son. O beautiful one! You killed in succession my seven exceedingly beautiful sons. O one of beautiful hips! I now fall at your feet. O beautiful one! save the life of this child of mine. If you ask from me any other thing to day, even that be very rare, I will give it to you; but you better now keep the thread of my family line. The Pundits, versed in the Vedas, say that he who has no issues cannot go to Heaven; so, O Beautiful one! To-day I pray to you to keep the life of this my eighth son.” Though thus spoken by S’antanu,

 

p. 89

 

Gangâ Devî was eager to take away the son to throw in the waters; the king became very sorrowful and angrily spoke out “O vile and vicious woman! What are you going to do? Do you not fear hell! of what villain are you the daughter, that you are always doing this vicious deed? O Sinner! go away wherever you will or remain here as yon like, it matters little; but my son will remain here. When you attempt to bring my family to extinction, what use is there in living with you?” When the king thus spoke to the woman who was ready to take away the son she angrily spoke as follows :-- O King! When you have acted against my promise, my word is broken and my connection with you has stopped from to-day. Therefore I will take this son to the forest, where I will nourish him. I am Gangâ; to fulfil God's work I have come here. The high-souled Vas'istha cursed before the eight Vasus :-- “Better be born as men”; they became very anxious; and seeing me they prayed :-- “O Sinless one! let you be mother of us all.” O best of kings! I granted them what they desired; and then for the purpose of serving god's ends, I became your wife. Know this my history. The seven Vasus already were born and were freed; now this is the last Vasu and he will remain here for sometime as your son. O S’antanu! now take this son offered by Gangâ. Know this to be the Deva Vasu and enjoy the pleasure of having a son. O highly fortunate one! This son will be famous by the name of Gângeya (Gangâ's son) and will be the most powerful of all. O King! To-day I will take this son to the place where I chose you as my husband; I will nourish him and when he attains his youth, I will return him to you. For, this son, if deprived of mother, will not be happy nor will he live.” Thus saying, Gangâ vanished with the son; the King S’antanu became very sorrowful and passed away his time in his palace. The king thought always of the separation from his wife and son and thus painfully governed his kingdom.

 

47-69. Thus some time passed on, when, once on an occasion, the king S’antanu went out a hunting and killed, with arrows, buffaloes, boar, and other wild animals and came to the banks of the Ganges. Here he saw with great wonder that a boy was playing with a great bow and was shooting arrows after arrows. The king's attention was more attracted towards the boy, but whether that boy was his or not, did not at all come to his mind. Looking at his extraordinary feats, his agility in shooting arrows with ease and quickness, his learning that can have no equal and his beautiful form, as if of Cupid, he became greatly surprised and asked him :-- “O Sinless one! whose son are you?” The hero boy did not reply anything but went away shooting his arrows. The king thought within himself “Who is this boy? Whose son is he? What to do now? To whom

 

p. 90

 

shall I go now?” Thus pondering, he recollected within himself and began to recite verses in praise of Gangâ; Gangâ, assuming her beautiful form as before, became visible to the king. Seeing her, the king said :-- “O Gangâ! Who is this boy that has just gone? Will you show him once more to me now?” Hearing these words of S’antanu, Gangâ said :-- “O king of kings! He is your son, he is that eighth Vasu. So long I have nourished him and now I hand him over to you. O Suvrata! This is the great ascetic Gângeya. He is the illustrious scion of your family. The glory of your line will be enhanced. I have taught him the whole science of archery. This pure son of yours dwelt in the hermitage of Vas'istha and has become versed in all the Vidyâs and skilled in all the actions. Your this son knows everything that Jamadgni Paras'urâm knew. So, O king of kings! Take now your son and be happy. Thus saying, Gangâ gave him his son and vanished; the king also became very glad and embraced his son; he smelled his head and took him to his chariot and drove towards his own city. On returning to Hastinâpur, the king held a great festival (utsab) in honor of the arrival of his son; he called all his astrologers and enquired what day was auspicious. He then called all his subjects and ministers and installed Gângeya as the Crown Prince. The religious S’antanu became very happy on making Gângeya, the Crown Prince; he forgot the pains due to Gangâ's bereavement. Sûta said :-- “Thus I have described to you the cause of the curse on Vasus, the birth of Bhîsma from the womb of Gangâ, the union of Gangâ and S’antanu, etc., He who hears in this world this holy story of Gangâ's birth and the birth of the Vasu, is freed of all sins and gets mukti. O Munis! I have described these meritorious holy accounts, as I heard from the mouth of Vyâsa. Any body who hears this holy S’rîmad Bhâgavatam, endowed with five characteristics and filled with various anecdotes, that came out of the mouth of Vyâsa, finds all his sins destroyed and attains peace and blessedness. O Munis! Thus has been described completely to you this holy history.

 

Thus ends the fourth Chapter on the birth of the Vasus in the Mahâpurâna S’rîmad Devî Bhâgavatam of 18,000 verses.

