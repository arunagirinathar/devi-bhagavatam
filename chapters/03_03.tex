\chapter{On seeing the Dev\^i}

1-5. Brahm\^a said :-- ``We were very much astonished not to find water where our beautiful aeroplane landed us. We saw earth resonated with the sweet cooings of the cuckoos, filled with beautiful fruit-laden trees, forests and gardens. Big rivers, wells, tanks, ponds, water-springs, small pools, women, men all are there. Next we saw, in front of us, a nice city enclosed by a divine wall, containing many sacrificial halls and various palatial buildings and magnificent edifices. Oh! We thought :-- It is Heaven! What a great wonder! Who built this?

6-11. Next we saw a king looking like a Deva is going out on a hunting excursion in the forest. The Dev\^i Ambik\^a, Whom we saw before, is staying on the chariot. In an instant, our aeroplane, propelled by air got high up above the sky and reached in the twinkling of an eye at a lovely place. We saw there a divine Nandana garden. There Surabhi, the cow of plenty, was staying under the shade of the Parij\^ata tree. Close by her, there was an elephant having four tusks; and Menaká and other hosts of Apsar\^as were there with their various gestures and postures, playing, dancing and singing. There were hundred of Yaksas, Gandharbhas, Vidy\^adharas within that Mand\^ara garden playing and singing. Within this there was the Lord Satakratu with \'Sach\^i, the daughter of Pulom\^a.

12-34. Next we saw with great wonder, Varuna, the lord of the aquatic animals, Kuvera, Yama, Sûrya (sun), fire and the other Devas; then we saw that in our front, Indra the Lord of the Devas, was coming out from a well decorated city. He was there situated in his palanquin, calm and quiet and carried by men. Then the car, where we were situated, began to

get up high in the sky, and in the twinkling of an eye, we reached Brahm\^a loka, that is saluted by all the Devas. There \'Sambhu and Ke\'sava were greatly bewildered to see Brahm\^a of that place. In the council hall of Brahm\^a, the Vedas with their Angas, the serpents, hills, oceans and rivers were seen. Seeing all these, Visnu and Mahe\'svara asked me :-- ``O Four-faced one! Who is this eternal Brahm\^a? I replied :-- I do not know who is this Brahm\^a? Who am I? and who is He? why has this error come over me? You, too, also are gods so you can better ponder over it.'' Next our car, going with the swiftness of mind went, in the twinkling of an eye, to the beautiful all auspicious Kail\^a\'sa mountain surrounded by bliss-giving Yaksas. It was beautified by the Mand\^ara garden, resonated by the sweet cooings of \'Sukas and cuckoos and the sweet sounds of lutes and small drums and tabors. When we reached there we saw the five faced, three-eyed Bhagav\^an \'Sashi \'Sekhara, with ten hands, wearing tiger skin, and the upper garment of the elephant skin. He was then, getting out of his abode, riding on a bull. His two sons, the great heroes, Gane\'sa and K\^artikeya, beautifully adorned, were attending Him as His body guards. Nandi and all other hosts were following Him, chanting victories to Him. O Muni Nar\^ada! we were greatly wondered to see another \'Sankara, surrounded by the Matrik\^as. So much so, that perplexed with doubts, I sat down there. Next our aeroplane went on with the force of wind; and in an instant reached the abode of Vaikuntha, the amusement court of Laksm\^i. O Sûta! There at Vaikuntha, we saw a wonderful manifestation of power. Our companion Visnu was greatly surprised to see that excellent city. We saw there four-armed Visnu, of the colour of Âtasi flower, wearing yellow garments, adorned with divine ornaments sitting on Garuda. Laksm\^i Dev\^i is fanning wonderful chowry to Him. Struck with wonder at the sight of the eternal Visnu, we took our seat on the car and looked at one another's face.

Next the balloon ascended with the swiftness of wind; and, in the twinkling of an eye, reached to the ocean of nectar, the Sudh\^a-S\^agar, with waves playing sweetly on it. This ocean Sudh\^a S\^agara is filled with aquatic animals and agitated with ripples. We saw and went along and came to a very wonderful place called the Mani Dv\^ipa (the island of gems) in the midst of the Ocean. It was adorned with Mand\^ara and Pàrij\^ata an other heavenly flower trees (plants?), with various beautiful carpets, with variegated trees A\'soka, Vakula, Ketak\^i, Champaka, Kuravaka, etc., adorned with lustrous gems and pearls. It was resonated with the sweet cooings of the cuckoos and the humming sounds of bees; and it presented the sight of a sweet harmonious music playing there.

35-67. Sitting on our aeroplane, we saw, from a distance, within that

Dv\^ipa, a beautiful cot known as \'Siv\^ak\^ara (i.e. whose four legs represent Brahm\^a, Visnu, Rudra, etc., and whose top portion represents Sad\^a \'Siva looking like a rainbow, with exquisitely beautiful carpet spread over it and decked with various gems and jewels and inlaid with pearls. We saw a Divine Lady, sitting on the cot, wearing a red garment and a garland of red cloth and bedewed with red sandal paste. Her eyes were dark-red; that beautiful faced red-lipped lady looked more beautiful than ten millions of lightnings and ten millions of Laksm\^is and lustrous like the Sun. The Bhagavat\^i Bhuvane\'svar\^i was sitting with a sweet smile on Her lips and holding in Her four hands noose, goad, and signs indicating as if She was ready to grant boons and asking Her devotees discard all fear. We never saw before such a form. Even the birds of that place repeat the mystic incantation Hrim and serve that Lady, Who is of the colour of the rising Sun, all merciful, and in the full bloom of youth. That lotus-faced smiling lady was adorned with all the beauties of Nature. Her high breasts defied the lotus bud. She was holding various jewelled ornaments, e.g., armplates, bracelets, diadems, etc.

Her lotus-face looked exceedingly beautiful with jewelled ear-rings of the shape of the \'Sr\^i Yantra (yantra of Tripur\^a Sundar\^i). Hrillekh\^a and other Deva girls were surrounding Her. There were Sakhis on the four sides -- always chanting hymns to Mahe\'svar\^i, the Lady of the world. She was surrounded on Her all sides by Ananga kusuma and other Dev\^is. She was sitting in the middle of the Satkona (six angled) Yantra. We were all wondered at the sight of this Wonderful Form never seen before and we thought :-- ``Who is this Lady? What is Her name? we know nothing of Her, from such a distance.'' Thus while we were gazing at Her, that four armed Lady became gradually thousand eyed, with thousand hands and thousand feet; so it seemed to us. O N\^arada! We became very much embarrassed with doubts and thought within ourselves ``Is She Apsar\^a (nymph) or a Gandharva daughter or any other Deva Girl? who is She ?'' At this juncture Bhagav\^an Visnu saw closely the sweet smiling Dev\^i and by his intelligence came to a definite conclusion and spoke to us :-- ``This is the Dev\^i Bhagavat\^i Mah\^avidy\^a Mah\^a M\^ay\^a, undecaying and eternal; She is the Full, the Prakriti; She is the Cause of us all. This Dev\^i is inconceivable to those who are of dull intellects; only the Yogis can see Her by their Yoga-powers. She is eternal (Brahm\^a) and also non-eternal (M\^ay\^a). She is the Will-force of the Supreme Self. She is the First Creatrix of this world.

This Dev\^i with wide eyes, the Lady of the Universe, has produced the Vedas. The less-fortunate persons cannot worship Her. During the time of Pralaya, She destroys all the Universe, draws within Her body all

the subtle bodies (Linga-Sar\^iras), and plays. O two Devas! At present She is residing in the form of the Seed of the Universe. Behold! On Her sides are seen duly all the Vibhûtis (manifestations of powers). They are all adorned with divine ornaments and anointed with divine scents and are serving Her. O Brahm\^an! O \'Sankara! To-day we are blessed and highly fortunate that we have got the sight of this Dev\^i. The tapasy\^a (asceticisms) that we practised of yore have yielded to us this fruit. Else why Bhagavat\^i has shown so carefully Her own form? Those who are highly meritorious by tapasy\^as and gifts of abundant wealth, those high souled persons are able to see this all-auspicious Bhagavat\^i. The person attached to sensual objects can never see Her. It is She that is the Mûl\^a Prakriti, united with the Chid\^ananda Person. It is She that creates this Brahm\^anda and exhibits it to the Param\^atm\^a (the Supreme Self). O two Devas! This whole Universe and all the Seers and Seen and other things contained therein owe to Her as their sole cause. She is the M\^ay\^a assuming all forms; She is the Goddess of all. Where is I myself! Where are the Devas! Where are Laksm\^i and the other Dev\^is! We cannot compare to one-hundred thousandth part of Her. It is this all-excellent Lady, Whom I saw in the great Ocean when She reckoned Me who was baby then with greatest gladness. In former days, when I was sleeping on the cot made of immoveable fixed leaves of a banyan tree and licking my toe, making it enter within my mouth and playing like an ordinary baby, this Lady rocked my gentle body to and fro on the banyan leaves singing songs like a Mother. Now I recollect all what I felt before at Her sight and recognise that She is the Bhagavat\^i. These very things I now communicate to you. Hear attentively that She is this Lady and She is our Mother.''

Thus ends the third chapter of the Third Skandha on seeing the Dev\^i in the Mah\^a Pur\^anam \'Sr\^imad Dev\^i Bh\^agavatam of 18,000 verses by Maharsi Veda Vy\^as.