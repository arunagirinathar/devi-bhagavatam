\chapter{On questions put by S’aunaka and other Rsis}

1-5. S'rî Sûta said :-- “I am highly fortunate; I consider myself blessed and I am purified by the Mahâtmâs (high souled persons); inasmuch as I am questioned by them about the highly meritorious Purâna, famous in the Vedas. I will now speak in detail about this Purâna, the best of the Âgamas, approved of by all the Vedas and the secret of all the S’âstras.

 

O Brahmins! I bow down to the gentle lotus feet, known in the three Lokas, of the Devî Bhagavatî, praised by Brahmâ and the other devas Visnu, Mahes’a and others, meditated always by the Munindras and which the Yogis contemplate as their source of liberation. Today I will devotedly describe, in detail and in plain language, that Purâna which is the best of all the Purânas, which gives prosperity and contains all the sentiments (Rasas) that a human being can conceive, the S’rimad Devî Bhâgavatam.

 

May that Highest Primal S’akti who is known as Vidyâ in the Vedas; who is omniscient, who controls the innermost of all and who is skilled in cutting off the knot of the world, who cannot be realised by the wicked and the vicious, but who is visible to the Munis in their meditation, may that Bhagavatî Devî give me always the buddhi fit to describe the Purâna!

 

I call to my mind the Mother of all the worlds who creates this universe, whose nature is both real (taking  gross, practical point of view) and and unreal (taking a real point of view), preserves and destroys by Her Râjasik, Sâttvik and Tâmasik qualities and in the end resolves all these into Herself and plays alone in the period of Dissolution - at this lime, I remember my that Mother of all the worlds.

 

6-10. It is commonly known that Brahmâ is the creator of this universe; and the knowers of the Vedas and the Purânas say so; but they also say that Brahmâ is born of the navel-lotus of Visnu. Thus it appears that Brahmâ cannot create independently. Again Visnu, from whose navel lotus Brahmâ is born, lies in Yoga sleep on the bed of Ananta (the thousand headed serpent) in the time of Pralaya; so how can we call Bhagavân Visnu who rests on the thousand headed serpent Ananta as the creator of the universe? Again the refuge of Ananta is the water of the

 

ocean Ekârnava; a liquid cannot rest without a vessel; so I take refuge of the Mother of all beings, who resides as the S’akti of all and thus is the supporter of all; I fly for refuge unto that Devî who was praised by Brahmâ while resting on the navel lotus of Visnu who was lying fast asleep in Yoga nidrâ. O Munis! meditating on that Maya Devî who creates, preserves and destroys the universe who is kuown as composed of the three gunas and who grants mukti, I now describe the whole of the Purânas; now you all better hear.

 

11-16. The Purâna S’rimad Bhâgavat (Devî Bhâgavat) is excellent and holy; eighteen thousand pure S’lokas are contained in it. Bhagavân Krisna Dvaipâyan has divided this Purâna into twelve auspicious Skandhas (Books) and three hundred and eighteen chapters. Twenty chapters compose the first Skandha; twelve chapters in the second Skandha; thirty chapters in the the third Skandha; twenty-five chapters in the fourth Skandha, thirty-five, in the fifth; thirty-one, in the sixth; forty, in the seventh; twenty-four, in the eighth; fifty chapters in the ninth; thirteen, iu the tenth; twenty-four in the eleventh and fourteen chapters are contained in the twelfth Skandha, O Munis! Thus the Dvaipâyan Muni has arranged his chapters in each Skandha.

 

17-20. Thus the Mahâtmâ Veda Vyâs has divided this Bhâgavata Purâna. into so many Skandhas and into so many chapters; and that the number of verses is eighteen thousand is already stated. That is denominated as Purâna which contains the following five characteristics :-- (1) Creation of the universe, (2) Secondary creation, (3) Dynasties (4) Manvantaras and (5) The description of Manus and other kings.

 

S’iva is beyond Prâkritic attributes, eternal and ever omnipresent; She is without any change, immutable, unattainable but by yoga; She is the refuge of the universe and Her nature is Turîya Chaitanya. Mahâ Lakshmi is Her Sattvikî S’akti; Sarasvati is Her Râjasik S’akti and Mahâ Kâlî is Her Tâmasik S’akti; these are all of feminine forms.

 

21-25. The assuming of bodies by these three S’aktis for the creation of this universe is denominated as "Sarga" (creation) by the high souled persona (Mahârpurusa), skilled in S’astras. And the further resolution of these three S’aktis into Brahmâ, Visnu and Mahes'a for the creation, preservation, and destruction of this universe is denominated (in this Purâna) as Pratisarga (secondary ereation.) The description of the kings of the solar and lunar dynasties and the families of Hiranya Kasipu and others is known as the description of the lineages of kings and their dynasties. The description of Svâyambhûva and, other Manus

 

 

and their ruling periods is known as Manvantaras. And the description of their descendants is known as the description of their families. (Thus these are the five characteristics in the Purânas.) O best of Munis! all the Purânas are endowed with these five characteristics.

 

26-32. So is Mahâbhârata writen by Vedavyâsa, characterised by these five things. This is known as the fifth Veda and Itihâsa (history.) In this are something more than one lakh slokas. S’aunaka said :-- “O Sûta! What are those Purânas and how many verses are contained in each? Speak all those in detail in this holy Ksettra; we, the residents of Naimisâranya are all very eager to hear this. (Why we call ourselves as the residents of Naimisâranya, hear; you will realise then that no other place exists in this Kali age for hearing the holy discourses on religion) :-- When we were afraid of the Kali age, Brahmâ gave us a Manomaya Chakra (wheel) and I said to all of us :-- Follow this wheel, go after it and the spot where the felly of the wheel will become thin (so as to break) and will not roll further, that country is the holy place; Kali will never be able to enter there; you all better remain there until the Satya age comes back. Thus, acording to the saying of Brahmâ, we have got orders to stay here. On hearing the words of Brahmâ, wo went out quickly keeping the wheel go on, our object being to determine which place is best and holiest. When we came here, the felly of the wheel become thin and shorn before my eyes; hence this Ksettra is called Naimis; it is the most sanctifying place.

 

Kali cannot enter here; hence the Mahatmas, Munis and Siddhas, terrified by the Kali age, have followed me and resorted to this place. We have performed yajñas with Purodâsa (clarified butter as is offered in oblations to fire) where no animals are sacrificed; now we have no other important work to do except to pass our time here until the arrival of Satyayuga. O S’ûta! we are extremely fortunate in all respects that you have come here; purify us to-day by narrating to us the names of the Purânas equivalent to the Vedas. O S’ûta! you are also a learned orator; we, too, are ardent listeners, with no other works to bother our heads; narrate to us to-day the auspicious holy Bhâgavata Purâna. O S’ûta! Long live you; and no ailings, internal, external, or from the Devas torment you. (this is our blessing to you). We have heard that in the most sanctifying Purâna, narrated by Maharsi Dvaipâyan, all about Dharma (religion), Artha (Wealth) and Kama (desires) are duly described as well the acquiring of Tattvajñan and liberation are also spoken of. O S’ûta! our desires are not satisfied the more we hear of those beautiful holy words. Now describe to us the highly pure S’rîmad Devî Bhâgavatam where all the Lilas (the dramatic acts) of the Mother of the three worlds purifying the sins, adorned with all the qualifications are described as yielding all the desires like the Kalpa Vriksa (the celestial tree yielding all desires).

Thus ends the second chapter of the first Skandha on the description of the Purâna (the text) in Mahâ Purâna S’rîmad Devî Bhâgavatam of 18,000 verses by Maharsi Veda Vyâs.

Her ends the Second Chapter of S’rîmad Devî Bhâgavatam on questions put by Saunaka and other Risis.

