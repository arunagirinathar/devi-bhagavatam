\chapter{On the narration of what are to be done in the Navar\^atri}

1. Janamejaya said :-- ``O Best of the Br\^ahmins! What are men to do in the time of Navar\^atra? Especially in the Navar\^atra ceremony during the autumnal season how is the ceremony to be performed? Kindly relate all this with the prescribed rules and regulations.

2. O intelligent one! What are the fruits therein of the Navar\^atra ceremony? and what are the rules to be observed? Kindly describe all these to me.''

3-5. Vy\^asa said :-- O king! Hear about the vow of auspicious Navar\^atra. This has to be performed with loving devotion in the vernal season; but its special season is autumn. The two seasons, autumn and spring, are famous as the teeth of Yama, the God of Death; and these are the two seasons, very hard for the persons to cross over. Therefore every goodfaring man should everywhere perform this vow very carefully.

6-8. O king! The people are very much afflicted with various terrible diseases in these two seasons autumn and spring and many lose their lives during these portions of the year. Therefore the wise should unquestionably worship with great devotion the Chandik\^a Dev\^i in these auspicious months of Chaitra and Â\'svin.

9-11. On the day previous to the commencement of the vow, when the Am\^avasy\^a tithi commences, one should collect the materials that will be required in the worship and should eat only once in that tithi what is called Habisy\^anna (sacred food, boiled rice with ghee) and should on that day prepare an open shade in a temporary building, twenty four (24) feet in dimensions, on a level piece of ground, that is considered holy; it is to be equipped with a post and a flag. Next, this is to be heaped over with yellow earth and cow dung. Then a raised platform called the Ved\^i, six feet wide and one and a half foot high, level and hard, is to be erected, and provided with an excellent space thereon for the seat of the Dev\^i. Provisions are to be made also for ornamented gate ways and an awning over the top.

12-17. One should invite then, those Br\^ahmins, that observe fully the customs and usages, who are self restrained and versed in the Vedas and Ved\^angas, especially those who are skilled in the ceremony of worshipping the Dev\^i.

Next, in the Pratipad tithi (the first day of the bright half), one should take one's morning ablutions in a river, or in a lake, tank or a well or in one's own residence, according to rules, and one should perform one's every day practices of Sandhy\^a Bandanam. Afterwards he should appoint the Br\^ahmins and give them water for washing their feet and Arghya (offerings of grass, rice, etc.,) and Madhuparka (an oblation of honey and milk, etc.) and give then, as his means permit, clothings and ornaments to them. If he happens to be rich, he should never show his miserliness here in making these gifts; for if the Br\^ahmins be satisfied, they will try their best to make the ceremony a complete success. O king! The Chand\^i paths (the reading of the book called Chand\^i) and Bh\^agavata paths (the reading of some portions of the book named Bh\^agavat) are done on this occasion, for the satisfaction of the Goddess; and either nine Br\^ahmins or five or three or at least one Br\^ahmin should be appointed for the purpose. Moreover one other Br\^ahmin, of a restrained and calm nature, is to be appointed, who would observe the fasting on the day previous (p\^ar\^ayana). All these being done, the able man is to perform the ceremony preparatory to the solemn Dev\^i worship, (in which the priest utters the Ved\^ic mantra Svasti-v\^achana, Svasti na Indro vriddha\'srav\^ah, etc.). Om Hr\^im \'Sr\^im Dûm Dûrg\^ayai namah is the nine lettered Dûrg\^a mantra.

18-20. O king! When the ceremony has been thus commenced, one should place on the Ved\^i (a raised platform; an altar), the throne fitted with double silken clothes; and, on that throne, he should place the image of the Dev\^i. The Dev\^i, the Eternal World-Mother, is to be four-armed or eighteen armed, (4 or 18) fully provided with all the weapons, ornamented with garlands of pearls and jewels, decorated with various ornaments of gems and precious stones, wearing excellent heavenly clothings, all the parts of the image being artistically finished and endowed with all the auspicious signs, mounted on a lion, and holding conch shell, wheel, club, and lotus in Her hands.

Note :-- The Dev\^i, here, is represented with four (4) or eighteen (18) hands.

21-22. In the absence of the image, one should place an earthen water-pot, on that throne, thoroughly purified by the Ved\^ic Mantras, filled with gold and jewels, and filled fully with the water, brought from a sacred river or a sacred place of pilgrimage and with five young shoots of plants, the extremities of branches bearing new leaves immersed in water. Beside the water-pot on the throne, there should be a symbol (Diagram or Yantra) with the nine lettered Mantram (Om Hr\^im \'Sr\^im Chandik\^ayai namah) in it for the purpose of worship.

23. One should place on one's side all the materials of worship in their due places, and then have the music and other sounding drums played, for the good fortune and prosperity of the family.

Note :-- Look for the mantras in the book Mantramaho Dadhi.

24. O king! If the first day be the Nand\^a tithi (i.e., the first day of the bright half with the asterism Hast\^a in the ascendant), then that is the best time for worshipping duly the Holy Goddess. There is no doubt that special fortunate results would arise on this.

25. On the previous night, one should observe fasting, or on the previous day one should take only one meal of Habisy\^anna (boiled rice and ghee) and on the next day one should make a Sankalpa (an avowal of the purpose to perform a rite) and then begin worship.

26. One should pray before the Goddess thus, ``O Mother, Mother of the World! I will perform this excellent Navar\^atra vow; be pleased to help me in every respect.''

27. One is to observe, as far as possible, all the rules enjoined in this vow and then utter the mantras and do the worship according to the prescribe rules.

28-31. First of all, one should worship duly the Goddess Jagaddh\^atri, presenting Her Chandan (sandal paste), Aguru (a fragrant wood, the aloe wood), Camphor, the flowers Mand\^ara (one of the five trees of the celestial regions), Karaja a kind of fragrant flower)! A\'soka, Champaka, Karavir, M\^alat\^i, and Br\^ahm\^i and various lovely sweet scented flowers and good Bel leaves, Dhûpa (incense, a fragrant gum burnt before idols) and lamps. Next one should present the fruits cocoanut, M\^atulinga, the pomegranate, bananas, oranges, the jack fruits, Bel and various other delicious fruit and then, offering Her arghya, present boiled rice and other food with a heart, full of devotion.

32. Those who eat meat, they can sacrifice animals in this worship of the Dev\^i; and, for this purpose, goat and wild boars are the best.

33-34. O sinless one! The goats, etc., offered as a sacrifice before the Dev\^i attain to unending heavens. Therefore persons offering the sacrifices of goats do not incur any sin. O king! The goats, etc., and other beast offered as a sacrifice before the Devas undoubtedly go to the heavenly regions; therefore, in all the \'S\^astras, it has been decided that this killing of animals in a sacrifice is considered as non-killing.

35. Now, for doing the Homa ceremony one should prepare, according to one's requirements, a triangular pit from one to ten hands in dimension and a triangular level piece of ground covered with sand.

36. Daily, thrice, one should worship the Dev\^i with various lovely

articles and finally make a great festivity with dancing, singing and music.

37. Everyday he should sleep on the ground and worship the virgins (young girl from the age of two to the age of ten) with nectar like sweetmeats and beautiful clothings aud ornaments.

38. Everyday one virgin or increased by one, two, or three every day or nine virgins in all the days respectively are to be worshipped.

39. O king! One should perform worshipping this Kum\^ar\^i (virgin) Puj\^a for the satisfaction of the Dev\^i, as his means allow; never one is to shew miserliness in this.

40. O king! Hear the rules of the virgin worship that I am going to tell you. The virgin, aged one year, is not to be worshipped; for they are quite ignorant as to smell and tasting various delicious things.

41-43. The virgin aged two years is named the Kum\^ar\^i; aged three years is named the Trimurt\^i four years, is called the Kaly\^an\^i; five years, Rohin\^i; six years, K\^alik\^a; seventh year, Chandik\^a; eighth year, \'S\^ambhav\^i; ninth year, Dûrg\^a; and a virgin, aged ten years, is called Subhadr\^a. Virgins aged more than ten years are not allowed in all ceremonies.

44. One should worship these virgins, taking their names and observing all the rules. I am now mentioning the different results that arise from the worship of these nine classes of virgins.

45. The worship of Kum\^ar\^i leads to the extinction of miseries and poverty, to the extirpation of one's enemies and the increment of riches, longevity and power.

46. The Trimurt\^i Puj\^a yields longevity, and the acquisition of the three things, Dharma, wealth, and desires, the coming in of riches, sons and grandsons.

47. Those who want learning, victory, kingdom and happiness, they should worship the Kaly\^an\^i, the fructitier of all desires.

48-49. Men should worship Rohin\^i duly for the cure of diseases. For the destruction of enemies, the worship of the K\^alik\^a with devotion is the best. For prosperity and riches, Chandik\^a is to be worshipped with devotion. O king! For the enchanting and overpowering of one's enemies, for the removal of miseries and poverty, and for victory in battles, \'S\^ambhav\^i worship is the best.

50-51. For the destruction of awfully terrible enemies and for happiness in the next world, the worship of Dûrg\^a is the safest and best. People worship Subhadr\^a when they want their desires to be fulfilled.

52. People should, with great devotion, worship the Kum\^ar\^is (virgins) with the mantrams ``\'Sr\^irastu'' or other mantrams, beginning with ``\'Sr\^i'' or with the seed mantrams.

53. The Goddess who can create without any difficulty all the sacred tattvas of the Kum\^ar K\^artikeya and who effects, as if in sport, the creation of all the Devas Brahm\^a and others; I am worshiping the same Kum\^ar\^i Dev\^i.

54. She who is appearing under the three forms as differentiated by the three gunas S\^attva, R\^ajas, and T\^amas, and who is appearing in multiple forms, owing to the differentiations of the three gunas again into various minor differences, I am worshipping Her the Trimûrt\^i Dev\^i.

55. She who being worshipped always fares us with auspicious things, I am worshipping Her, with devotion, the Kum\^ar\^i Kaly\^an\^i, the awarder of all desires.

56. I am worshipping the Rohin\^i Dev\^i with a heart, full of devotion who is germinating all the karmas in seed forms, that have accumulated owing to past deeds.

57. She who, at the end of a Kalpa gathers unto Her in the form of K\^al\^i all this Universe, moving and unmoving, I worship that K\^alik\^a Dev\^i with devotion.

58. She, who is furious and wrathful and hence is called Chandik\^a and who killed the two Demons Chanda and Munda I bow down to Her humbly with devotion, to that Chandik\^a Dev\^i, who destroys the terrible sins.

59. I worship that \'S\^ambhav\^i Dev\^i, the giver of all pleasures and happiness, whose form is the Veda Brahm\^a, and whose origin is without any cause, and whe is so recited in the Vedas.

60. She who saves from danger her devotees and who always delivers from various difficulties and troubles, whom all the Devas are incapable to know, I worship with devotion that Dûrg\^a Dev\^i the destroyer of all calamities.

61. I, with my mind devoted, offer my salutations to that Subhadr\^a Dev\^i, Who procures all auspiciousness to Her devotees and removes all inauspicious incidents.

62. Thus, in the mantrams, above described, people should always worship the virgin girls, giving them clothings, ornaments, garlands, scents, and various other articles.

Here ends the 26th Chapter on the narration of what are to be done in the Navar\^atri in the Mah\^a Pur\^anam in \'Sr\^i Mad Dev\^i Bh\^agavatam of 18000 verses, by Maharsi Veda Vy\^asa.