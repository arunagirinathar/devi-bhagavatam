On deciding who is to be worshipped

 

p. 26

 

1-7. The Risis said :-- “O highly fortunate one! A great doubt has arisen on your statement. This is ascertained by all the wise men as written in the Vedas, Purânas and other Sâstras that Brahmâ, Visnu and Mahes'var, these three Devas are eternal. None is superior to them in this Brahmânda. Brahmâ creates all the beings, Visnu preserves and Mahes'var destroys all in due time. These are the causes of creation, preservation and destruction. The Trinity Brahmâ, Visnu and Mahes'a are really one form, indeed, Trinity in Unity and Unity in Trinity.

 

Being endowed respectively with Sattva, Raja and Tamo Gunas they do their respective works. Amongst these, again, Purushottam Âdideva Jagannâth Hari, the husband of Kamalâ is the best; for he is capable of doing all the actions; no other than the Visnu, of unrivalled prowess is so capable. How is it, then that Yogamâyâ has overpowered Hari with sleep and made him altogether senseless? O highly fortunate one! whither did, then, go that extraordinary self knowledge and power, etc., of Hari while alive? This is our greatest doubt; so kindly advise us that our this doubt be removed and our well-being be thus ensured.

 

8-30. What is that S’aktî? Which you mentioned to us before; as well by whom Visnu is conquered? Whence is She born? What is the power of that S’aktî and what is Her nature? O Suvrata! explain to us these fully.

 

How was it that Yogamâyâ overpowered with sleep the Highest Deity Bhagavân Visnu who is everlasting-intelligence bliss! who is the God of all, the Guru of the whole world, the Creator, Preserver and Destroyer, who is omni-present, an incarnate of purity and holiness and beyond Rajoguna; how was such a personage brought under the control of sleep? O Sûta! You are very intelligent and the pupil of Vyasa Deva; destroy our this doubt by the sword of wisdom.

 

Hearing this, Sûta said :--“O highly fortunate Munis! There is none in the three Lokas who can clear your this doubt; the mind-born sons of Brahmâ, Nârada, Kapila and other eternal sons get bewildered by these questions; what can I, then, say on this very difficult point! See, some persons call Bhagavân Visnu omnipresent, the preserver of all and the best of all the Devas; according to them all this universe moving and non-moving, is created by Visnu; they bow down before

 

p. 27

 

the Highest Nârâyan Hrisikes'a Janârdana Vâsudeva and worship Him, whereas others worship Mahâdeva S’ankara, having Gauri for the other half of his body, endowed with all powers, residing in Kailâs'a, surrounded by hosts of bhutas, that destroyed the Daksha's sacrifice, who is mentioned in the Vedas as S’as'is'ekhara (having moon on his forehead), with three eyes and five faces and holding trident in his hand and known as Vrisadhaja and Kaparddi. O highly intelligent ones! There are some other persons, that know the Vedas and worship the Sun everyday in the morning, mid-day and in the evening with various hymns.

 

In all the Vedas, it is stated that the worship of the sun is excellent and they have named the high-souled sun as Paramâtmâ (the Highest Deity). Whereas there are other Vedavits (the knowers of the Vedas) who worship the Devas, Fire, Indra, and Varuna. But the Maharsis say, that as Gangâ Devi (the river Ganges), though one, is expressing Herself by many channels, so the one Visnu is expressing in all the Deva forms. Those who are big Pundits, declare perception, inference, and verbal testimony as the three modes of proofs. The Naiyâyik Pundits add to the above three, a fourth proof which they call upamâ, resemblance, similitude and some other intelligent Pundits add another fifth proof called Arthâpatti, an. inference from circumstances, presumption, implication. It is deduction of a matter from that which could not else be; it is assumption of a thing, not itself perceived but necessarily implied by another which is seen, heard or proved; whereas the authors of the Purânas add two other, called Sâksî and Aitijhya, thus advocating. seven modes of proofs. Now the Vedânta S’âstra says that the supreme being (Param Brahmâ), the Prime cause of the Universe, cannot be comprehended by the above-mentioned seven proofs. Therefore, first of all, adopt the reason leading to sure belief, the Buddhi, according to the words of the Vedas and discriminate and discuss again and again and draw your inference about Brahmâ. And the intelligent person should adopt what is seen by perception as self-evident and what is inferred by the observance of good conduct. The wise persons say, and it is also stated in the Purânas, that the Prime Force is present in Brahmâ as the Creative Force; is present in Hari as the Preservative Force; is present in Hara as the Destructive Force; is present in Kurma (tortoise) and in Ananta (the thousand headed Snake) as the earth supporting Force; is present in fire as the Burning Force, is present in air as the moving Force, and so is present everywhere in various manifestations of forces.

 

31-51. In this whole Universe, whoever he may be, all are incapable of any action if he be deprived of his force; what more than this, if S’iva be deprived of Kula Kundalinî S’aktî, He becomes a lifeless corpse; O great

 

p. 28

 

ascetic Risis! She is present everywere thus in every thing in this universe from the highest Brahmâ to the lowermost blade of grass, all moving and non-moving things. Verily everything becomes quite inert, if deprived of force; whether in conquering one's enemies, or in going from one place to another or in eating -- one finds oneself quite incapable, if deprived of force. Thus the omnipresent S’aktî, the wise call by the name of Brahmâ. Those who are verily intelligent should always worship Her in various ways and determine thoroughly the reality of Her by every means. In Visnu there is the Sattviki S’aktî; then He can preserve; otherwise He is quite useless; so in Brahmâ there is Rajasi S’aktî and He creates; otherwise He is quite useless; in S’iva, there is Tamasi S’aktî and He destroys; else He is quite useless. Thus, arguing again and again in one's mind, everyone should come to know that the Highest Âdya S’aktî by Her mere will creates and preserves this Universe and She it is who destroys again in time the whole Brahmânda, moving and non-moving; no one is capable to do his respective work be he Brahmâ, Visnu, Mahes'var, Indra, Fire, Sun, Varuna or any other person whatsoever; verily all the Devas perform the respective actions by the use of this Âdya S’aktî. That She alone is present in cause and effect and is doing every action, an be witnessed vividly. The intelligent ones call that S’aktî twofold; one is Sagunâ and the other is Nirgunâ. The people, attached to the senses and the objects, worship the Sagunâ aspect, and those who are not so attached worship the Nirguna aspect. That conscious S’aktî is the Lady of the fourfold aims of life, religion, wealth, desires, and liberation. When She is worshipped according to due rules, She awards all sorts of desires. The worldly persons, charmed by the Mâyâ of this world, do not know Her at all; some persons know a little and charm others; whereas some stupid and dull-deaded Pundits, impelled by Kali, start sects of heretics, Pâsandas for the sustenance of their own bellies. O highly fortunate Munis! In no other Yugas were found acts as prevalent in this Kali Yuga, based on various different opinions and altogether beyond the pale of the Vedic injunctions. Behold again, if Brahmâ, Visnu and Mahes’a be the supreme Deities, then why do these three Devas meditate on another One beyond speech, beyond mind and practise, for years, hard austerities; and why do they perform Yajñas (sacrifices) for their success in creation, preservation, and destruction? They know, verily, the Highest Supreme Being, Brahmâni Devî S’aktî eternal, constant and therefore they meditate Her always in their minds. Therefore the wise man, knowing this firmly, should serve in every way the Highest S’aktî. O Munis! This is the settled conclusion of all the Sâstras. I have heard of this great hidden secret from Bhagavân Krisna Dvaipâyan. He heard it from Nârada, and Nârada heard it from

 

p. 29

 

his own father Brahmâ. Brahmâ heard this from Visnu. O Munis it is well that the wise even should not hear or think anything to the contrary from other sources; they should with their concentrated heart serve the Brahmâ Sanâtanî S’aktî. It is clearly witnessed in this world that if there be any substance wherein this conscious S’aktî does not exist, that becomes inert, quite useless for any purpose. So know this fully that it is the Highest Divine Mother of the Universe that is playing here, residing in every being.

 

Thus ends the eighth chapter of the first Skandha on deciding who is to be worshipped in the Mahapurâna Sri Mad Devî Bhâgavatam of 18,000 verses by Maharsi Vedavyâsa.