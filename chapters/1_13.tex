On Urvas’î and Pururavâ

 

p. 45

 

1-34. Sûta said :-- O Maharsis! When the king Sudyumna had gone up to heavens, the religious king Pururavâ, endowed with great beauty and many qualities, and able to please the minds of his subjects began to govern his kingdom well, according to Dharma, with his heart intent on governing his people. No body knew what his counsel was, but he was very clever in knowing others' counsels. He was always highly energetic and his lordly power was great. The four methods of warfare, (1) con-ciliation, (2) giving gifts, (3) sowing dissensions and (4) war, were fully under his control. He saw that his subjects practised religion according to Varnâs'ram (Colours and stages of life), and thus he began to govern his kingdom. Pururavâ, the lord of men, performed various sacrifices with abundance of Daksinâs (sacrificial fees) and also gave away much in various charities, causing great wonder and astonishment. His extra-ordinary beauty and qualities, liberality and good nature, his unbounded wealth and prowess made the Apsarâ Urvas’î (celestial nymph) think of him often and she wanted him to satisfy her. Some time passed when that procud Urvas'î had to descend on this earth, due to a curse pronounced on her by a Brâhmin; and she chose the king Pururavâ as her husband, thinking him to be endowed with all the qualities. She then addressed the king and made the following contract :-- “O king, giving honour due to every body! I keep these two young sheep in trust and deposit with you; kindly look after these and, then, my honour will be preserved by you. O king! I will take ghee daily and nothing else for my food; and there is one word more; O king! Let me not see you naked, except when you hold sexual intercourse. O king I say this truly, that, in case there be any breach in this contract I will instantly leave you and go away.” The king accepted this contract of Urvas’î. Urvas’î, too, remained there according to the above contract and also with a view to pass away the period of her curse. During this time the king was fascinated with the love of Urvas’î and became so very much attached to her, that he left all his duties and dharma and remained long for many, many years in enjoying Urvas’î. The king's mind was so deeply absorbed in her, that he could not remain alone without her,

 

p. 46

 

even for a moment. Thus many years passed away, when, once on a time, Indra, the lord of the Devas, not seeing Urvas'î, asked the Gandarbhas and said :-- “O Gandarbhas! Better go and steal away the two young sheep in a proper time from the palace of Pururavâ, and then bring Urvas'î here. My Nandana garden is now void of all beauty without Urvas’î; so bring the lady here any how or other.” Thus spoken by Indra, Vis'vâvasu and other Devas went to Pururavâ's palace; and when it was dark in the night, and when Pururavâ was holding sexual intercourse with Urvas’î, stole away the two young sheep. They, when being carried away in the sky, cried out so piteously that Urvas’î came to hear that as if of her two sons, and angrily spoke to the king :-- “O king! Now the contract that I made with you is verily fulfilled! It is that I placed my implicit confidence on you that this my misfortune has befallen on me; See! the thieves are stealing away the two sheep, my sons as they were! How then are you sleeping here like a woman? Alas! I am ruined in having an impotent husband who vainly boasts of his prowess!

 

Where are my two young sheep to-day that are dearer to me than my life?” Thus seeing Urvas’î wailing, the king Pururavâ, the lord of the Universe, ran after the Gandharvas instantly without any sense as it were, left in him, naked. The Gandharvas, then, cast rays of lightning in that room, and Urvas’î, willing to leave, saw the king naked when the Gandharvas left the two young sheep there and went away. The tired king brought the two sheep back to his house in that naked state. Then Urvas’î, too, seeing the husband naked, went away immediately to the Devî loka. Seeing Urvas’î going away, the king wailed very much with a very grievous heart. Then, being very much bewildered by the bereavement of Urvas’î, with his senses beyond control, and deluded by passion, wandered about in various countries, crying and giving vent to sorrow. Thus, wandering all over the globe, he came once to Kuruksettra and saw Urvas’î; then with a gladdened face said :-- “O beloved! Wait, wait for a moment; my mind is all absorbed in you; it is quite innocent and submissive to you. So you ought not to forsake me in such a dire difficult time. O Devî! For the sake of you, I have travelled very far. O Beautiful one! The body that you embraced before, will now, forsaken by you, fall here and will be devoured by crows and wolves, and other carnivorous animals.” Seeing the king, tired and passion stricken, greatly distressed and with a very sorrowful heart and wailing, Urvas’î spoke out :-- “O king! You are certainly a quite senseless man; whither has gone your extraordinary knowledge now? O king! Do you not know that the pure unalloyed love of women cannot take place with any other as the love of a wolf cannot fall on any man. Therefore the earthly men ought never to trust a bit to women and thieves. So go

 

p. 47

 

back to your palace and enjoy the pleasures of the kingdom; do not drown your mind further in sorrows.” The king Pururavâ, though thus brought to senses by Urvas’î, was so much fascinated by her love that his heart did not feel any consolation; rather he felt indescribable pain, being held up in bondage by the love of Urvas’î. O Munis! Thus I have described to you the character of Urvas’î; it is described, in detail, in the Vedas; I have stated this in brief.

 

Thus ends the thirteenth chapter of the 1st Skandha of the characters Urvas’î and Pururavâ; in the Mahâpurânam S’rî Mad Devî Bhâgavatam of 18,000 verses by Maharsi Veda Vyâsa.