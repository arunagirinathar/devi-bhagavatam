On the birth of Âstika

 

p. 115

 

1-4. Sûta said :-- Hearing these words of the king, Vyâsa Deva, the son Satyavatî addressed to him before the assembly, thus :-- “O king! I am now reciting to you a Bhâgavata Purâna, holy, wonderful, filled with many anecdotes, and leading to auspicious results; listen. Before I made my son S'uka study this Purâna; O king! I will now recite before you that highest Purâna, with all the secrets contained therein. I have extracted this from all the Âgamas; it brings in Dharma (religion), Artha (wealth), Kâma (fructification of desires) and Moksa (liberation); hearing this gives always happiness and good results.

 

5-6. At this Janamejaya said :-- “O Lord! Whose son is this Muni Âstika? Why did he come as an obstacle in my Sarpa Yajña (sacrifice of snakes)? And what object had he in preserving the snakes? O highly fortunate one! Kindly describe all this in detail; after this recite the Purâna, also, in detail to me.”

 

7-18. Vyâsa Deva said :-- “O king! In former days there lived a Muni named Jaratkâru. He always remained in the path of peace; and did not marry. Once he saw, in a cave in a forest, his fathers and forefathers pendant. They spoke to Jaratkâru thus :-- “O son! Marry; we will thereby be greatly pleased; if there be a son of good character born to you, we all will be freed from all troubles and we would then able to go to Heavens.” Hearing this Jaratkâru said :-- “O Forefathers! If I get a girl of my name, without begging and asking and if she be entirely obedient to me, I will marry and lead a householder's life; thus

 

p. 116

 

I have spoken truly to you.” Thus saying to his forefathers, Jaratkâru went on tour to the holy places. Now it happened so, that at that very time Kadru, the mother of snakes cursed her sons, saying “May you be burnt by fire.” The matters of this incident run as follows :-- “At that moment Kadru and Vinatâ, the two co-wives of Kas'yapa saw the horses yoked in the chariot of the sun and thus argued with each other :-- Kadru, seeing the sun's horse, first asked Vinatâ “O good one! Tell me soon, what is the colour of this horse?” Vinatâ said :-- “O auspicious one! What do you think?” I said, the colour of the horse is white; you also better say before hand what is its colour? We will then lay a wager (and challenge). Kadru said :-- “O Smiling one! I think the the horse is black. Now come; let us challenge; whoever will be defeated will become the slave of the other.” Thus saying, Kadru told her sons that were obedient :-- “Cover by your bodies all the pores of the body of the horse of the chariot of the Sun, so it may look black; go and do it.” At this some snakes replied “That cannot be.” Kadru then cursed them saying :-- “Let you fall on the sacrificial fire of Janamejaya.” Then the other snakes tried to please their mother and coiled round the back of that horse so that the horse began to look black. Kadru and Vinatâ, the two co-wives went together and saw the horse. Vinatâ saw it black and became very sorry.

 

19-21. Now, Garuda, Vinatâ's son, very powerful and devourer of snakes was passing that way and seeing his mother very distressed asked her :-- “O Mother! Why do you look so very sorry? It seems as if you are weeping. Aruna, the charioteer of the Sun and I myself are your two sons living. Fie to us that, while we are living, you will have to suffer pains. O beautiful one! If mother suffers while the son is living, then what use is there in having such a son? So, O Mother, give out the cause of your grief and I will remove it at once.”

 

22-31. Hearing this Vinatâ said :-- “O son! What shall I say to you of my misery; I am now become the slave of my rival wife. By some pretext she defeated me and is now telling me to carry her on my back. O son! For this reason I am sorry.” Hearing these words of the mother, Garuda said :-- “Very well I will carry her on my shoulders wherever she wishes to go. O auspicious one! You need not be sorry; I will remove all your cares.” Vyâsa Deva said :-- Thus spoken to by Garuda, Vinatâ went to Kadru. At that time the highly powerful Garuda went there also to free his mother of her slavery and carried Kadru with all her sons on his back to the other side of the ocean. When Garuda went across the ocean, Garuda spoke to Kadru :-- “O mother! I bow down to thee; kindly say how my mother can be freed

 

p. 117

 

of your slavery. Hearing this Kadru said :-- “O son! If you can bring to-day by your sheer force nectar from the Deva loka and give it to my sons then you will be able to free your helpless mother. When Kadru said so, the highly powerful Vinatâ's son, Garuda immediately went to the abode of Indra and, fighting hard, stole away the jar of nectar and brought and gave it to Kadru and freed his mother Vinatâ from the slavery of Kadru. In the meantime, the snakes went for their bath, after which they would drink the nectar. Indra stole away that jar which contained nectar. O king! Thus, by the sheer strength of arms of Garuda Vinatâ was freed of her slavery. On the other hand, when the snakes returned from their bath and found that there was no jar of nectar, they began to lick the Kusa grass over which the jar of nectar was kept, thinking that they would thereby get some drops of nectar which might have trickled over; and the result was that by the sharp edges of kusa grasses, the tongues of all the snakes were cut asunder into two; hence the snakes are called Dvijihva.

 

32-36. The snake Vâsuki and others, whom Kadru, the mother of snakes, cursed, went to Brahmâ and took his refuge and informed all of the cause of their terror, the curse from their mother; when Brahmâ spoke to them :-- “Go and give the sister of Vâsuki, named Jaratkâru, in marriage to the great Muni Jarat Kâru, (both of the same name). In her womb, a son named Âstika will be born; and he will certainly deliver you from your difficulties. Hearing those beneficial words of Brahmâ, Vâsuki went to the forest and requested humbly the great Muni Jarat Kâru to accept in marriage her own sister when the Muni, knowing the girl to be of his name, spoke out thus :-- “But when your sister will act against my wishes, I will forsake her at once.”

 

37-46. Under these conditions, the Muni married her. And Vâsuki, after giving her sister in marriage according to her own wishes to the Muni, returned to her own abode. O Tormentor of foes! Then the Muni Jaratkâru built a white hut of leaves in that great forest and began to pass his days happily in enjoyment with his wife. Once, on an occasion, after he had taken his dinner he slept and told his wife not to awaken him under any circumstances and fell fast asleep. The beautiful sister of Vâsuki sat by his side. When the evening time came and the sun began to set, the Vâsuki's sister Jaratkâru became afraid at the thought that the evening Sandhya might not be performed by the Muni and thought thus :-- “What am I to do now? My heart finds not rest if I do not awaken him; and if I awaken him, he will forsake me at once. Now if I do not awaken him, the evening will pass away to no purpose. Whatever it be, if he quits me or if my death ensues, that is better than the non-observance of Dharma; for when Dharma is destroyed, hell ensues.

 

p. 118

 

Thus thinking, the girl awakened him saying :-- “O One of good vows! It is evening time; so get up; etc.” The Muni got up in great anger and addressed his wife :-- “When you have disturbed my sleep, I now go away from you; you also better go to your brother's house.” When the Muni said so, Vâsuki's sister spoke out, trembling :-- “O One of indomitable lustre! How will the object be served for which my brother has given me in marriage with you.”

 

47-50. The Muni then spoke firmly to his wife Jaratkâru :-- “That is within your womb.” Jaratkâru then, forsaken by the Muni, went to the abode of Vâsuki. When her brother Vâsuki asked her about her son, she said :-- “The Muni has forsaken me, saying that the son is within your womb.” At this Vâsuki trusted; and said :-- “The Muni won't ever tell lies” and gave shelter to his sister. O Kurusattama! After some time, a famous boy named the Muni Âstika was born.

 

51-56. O king! That Muni boy, the knower of truth, had desisted you from your sacrifice of snakes for the preservation of his mother's family. It is well and good, befitting you, that you respected the words of the Muni Âstika, born of Yâyâvara family and the cousin of Vâsuki. O Mighty-armed! Let all auspiciousness come to you; you have heard the whole Mahâbhârata and gave away lots of things in charities. You have worshipped innumerable Munis. But, O king! Though you have done so many good things, yet your father has not attained heaven and you have not been able to sanctify your family. So, O king Janamejaya! Now install a capacious temple of the Devî with the highest devotion; then all your desires will be fulfilled. The all auspicious Devî, the Giver of all desires, makes the kingdoms more stable and increases the family, if She be always worshipped with the highest devotion.

 

57-64. O king! You better perform duly the Devîmakha Yajña Yotistoma and others, pleasing to the Devî, and hear the great Purâna S’rîmad Devî Bhâgavatam, filled with accounts of the glorious deeds of the Devî. I will make you hear now that Divine Purâna, filled with various sentiments, highly sanctifying and capable to carry one across this ocean of world. O king! There is no other subject in this world worthy to be heard than the above Purâna and there is no other thing to be worshipped then the lotus feet of the Devî. O king! Those are certainly fortunate, those are intelligent and blessed, in whose hearts of love and devotion reign always the Devî Bhagavatî. O illustrious scion of Bharata's family! Know them to the always afflicted with troubles who do not worship in this world the great Mother Mahâmâyâ. O king! Who is there that will not worship Her when Brahmâ and all the Devas

 

p. 119

 

are always engaged in Her devotional service. O king! He who hears always this Purâna gets all his desires fulfilled; in former days Bhagavatî Herself spoke this excellent Purâna to Visnu. O king! Your heart will be appeased and become peaceful when you hear this; and, as a result of your hearing this Purânam, all your ancestors will attain endless Heavenly life.

 

Thus ends the Twelfth Chapter of the Second Skandha on the birth of Âstika in the Mahâpurânam S'rîmad Devî Bhâgavatam of 18,000 verses by Maharsi Veda Vyâsa. Here ends as well the Second Book.

 

