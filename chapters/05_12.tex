\chapter{On the holding of counsel by Mahis\^asura}

1. Vy\^asa said :-- The World-Mother, hearing T\^amra's words, spoke laughing a little and with a deep voice like that of a rumbling thunder cloud.

2-13. The Dev\^i spoke :-- ``O T\^amra! Go and say to your Lord Mahisa who, it seems, is stupid, whose end is nigh, who has become very passionate, and who is void of knowledge what is proper and what is improper. I am not like your grown up mother, the she-buffalo, having horns, eating grass, with a long tail and a big-belly. I do not like to have Visnu, the god \'Sankara, Brahm\^a, Kuvera, Varuna, or Fire. How then can I select a beast? If I do so, I will be an object of much censure amongst the several worlds. I am not desirous of any more husband; my Husband is existing; though He is the Lord of all, Witness of All, yet He is not the Actor; He is without any desires and He is calm and tranquil. He, the \'Siva, is devoid of any Prakritic qualities, without any attachment, the Great Infinite, without anyone to rely on, without any refuge, omniscient, omnipresent, the Great Witness, the Full, and the seat of the Full, the Auspicious. He is the abode of all, capable to do all, the peaceful, capable to create everything and He is seeing everywhere. How can I then leave Him and try to serve the dull, stupid Mahisa? Let him come and fight with this understanding that he will be defeated and be made the conveyance of Yama, the God of Death or the carrier of water of the human beings. And if that impious heretic desire to live, let him fly at once to P\^at\^ala with all his demon comrades; else I will slay him in battle. See! The combination of similar substances leads to happiness; and if out of ignorance, the connection takes place between things entirely different in their natures, it becomes at once the source of all pains and troubles. You are a thorough illiterate when you ask me to worship your lord; do you not see me endowed with exquisite beauty? and what is your Mahisa? A buffalo with horns; how can then creation become possible between us? Better fly away or fight if you like; I will kill you and your friends, and if you leave the region of Heavens and the share of Yajñ\^a, then you will become happy.''

14-30. Vy\^asa said :-- O King! Thus saying, the Dev\^i howled and roared so loudly that it appeared strange and it caused a great terror to the D\^anavas who took it as the great dissolution of the universe at the end of a Kalpa. The earth and the mountains trembled; the wives of the D\^anavas, had miscarriages at that terrific noise. T\^amra hearing that sound was terrified; his mind became unsteady and he at once fled to Mahisa. O King! The D\^anavas present in the city became deaf; thy fled and became very anxious and were absorbed in the thought whence and how that sound came. The lion, too, enraged and, raising up its manes, roared so loud that the Daityas became very much terrified. Mahisa, too, became confounded to see T\^amra returning; he then held a council with his ministers what ought to be done next? Mahis\^asura said :-- ``O best of the D\^anavas! Shall we now take our shelter within the forts? Or shall we go out and fight? Or will it be favourable to us if we fly away? You all are intelligent and versed in all the \'S\^astras and unconquerable by your foes; therefore ought you all to consult over the the matter in utmost privacy for our success at the present moment. The root of Kingdom lies in the council in the secret place (cabinet) and Statesmanship; if this counsel be kept well preserved in secrecy, then that Kingdom is also kept entire; therefore it is highly incumbent that the plan be kept in strictest confidence amongst the good and virtuous ministers. If the plan be out, then destruction comes both to the King and his Kingdom; hence the plan must be kept secret by those wanting glory, lest it be taken advantage of and rendered ineffective by other persons. O Ministers! Now declare, taking due consideration of time and place, after duly discussing and ascertaining what is the best course to adopt, what would be beneficial and full of reason and intelligence. First find out the cause why this powerful woman, created by the Devas has come here alone and helpless? That woman is challenging us to fight. What more wonder can there be than this? Who can say in the three worlds what the result will be, whether it will be good or otherwise? Victory comes not to many persons nor defeat comes to a single individual; therefore victory or defeat lies at the hands of the Luck and Destiny. Those who plead for place, policy, statesmanship, they say what is Fate? Is there anyone who has seen Fate? (Adrista) No one has seen His appearance. It may be argued that there may exist such a thing as Fate; to which it might be replied, what proofs are there for such an existence? Thus the weak persons alone hold it out as their only hope; nowhere are seen energetic persons who can fulfil their ends by their own efforts, by those who enrol themselves under Fate. Therefore ``effort,'' ``energy'' are the words of the heroes and ``Fate'' is the word of the cowards. You should all consider today these subjects fully and intelligently and then decide what are we to do?''

31-39. Vy\^asa said :-- Thus hearing the King, the famous Vid\^al\^aksa with folded hands spoke thus :-- O King! First it should be definitely ascertained whose wife is she, this woman possessing large eyes? Whence and for what purpose has she come here; next what ought to be done should be decided. It seems to me that the Devas, knowing that your death will ensue from the hands of a woman, have created very carefully this lotus-eyed woman out of their own essences. And they are lying in wait, unknown to anybody in the celestial space with a desire to see the battle but really to fight with you. In due time, they will undoubtedly help this woman. When the war will ensue, Visnu and the other Devas will put this woman in front and slay us all. Whereas this Dev\^i will slay you. This is their earnest desire. O King! I have come to know this beforehand; but what will be the actual result I cannot say. I cannot say also whether it is advisable for You to fight now; therefore it would be better if you consider yourself well on this matter of the gods and do accordingly. Our duty, the duty of your servants lies in this :-- That we should sacrifice at any moment our lives for the preservation of your prestige and to enjoy with you whenever you are enjoying. But, O King! It is extremely advisable to ponder over this very carefully when we see that this woman, though alone, is challenging us to fight who are armed with powerful soldiers.

40-44. Durmukha said :-- O King! I know for certain, that we will not get victory in this battle; still we ought not to show our backs; for that would lead us to sheer disgrace. Even in our encounter with Indra and other Devas, we did nothing hateful and blameable; then how can any of us fly away when we come face to face with a helpless woman? Therefore fight we must; that is certain; let whatever happen. What is inevitable, must come to pass. Thus considered, what need we care for the result? If we die in the battle, we will get name and fame; if we be victorious, we will get happiness. Thus thinking both the cases, we must fight today. Death is inevitable when our longevity expires; our prestige will suffer if we fly away; therefore we ought not to spend uselessly our time in thus expressing our vain regret for life or for death.

45-51. Vy\^asa said :-- O King! Hearing thus the Durmukha's words, V\^askala, the eloquent speaker, thus spoke to the king, with clasped hands and his head bowed down. O King! You need not think thus in agony with this unpleasant affair; alone I will kill that Chandik\^a, of unsteady eyes. O Best of kings! To be always prompt and energetic indicates that one is steady in one's heroic valour; to consider one's enemy as dreadful is contrary to above; so we ought now to take recourse to heroic valour. O King! Therefore I will discard fear altogether and fight out valiantly; I will no doubt, send Chandik\^a in the battlefield to the abode of Death. I fear not Yama, nor Indra, nor Kuvera, nor Vayu nor Agni, nor Visnu, nor \'Sankara, nor Moon nor Sun; I do not fear any of them; what fear can I, then, entertain of that vain arrogant woman, who has got none to support her. I will kill Her with these arrows, sharpened on stones. You can see today the prowess of my arms and enjoy peace; you will not have to go to battle anymore to fight with Her.

52-65. Vy\^asa said :-- O King! V\^askala having said thus to the lord Mahisa in a haughty spirit, Durdhara bowed down and said thus :-- O Lord of the earth! Let the purpose be whatsoever, with which the beautiful Dev\^i with eighteen hands, the creation of the gods, may come hither, I will vanquish Her. O King! I think, it is simply to terrify you, as the Suras have thus created this M\^ay\^a woman; therefore, do you forsake your delusion by knowing this merely as a scare. O King! Such is the statesmanship; now hear about the workings of the ministers. Ministers in this world are of three kinds :-- (1) S\^attvik; (2) R\^ajasik and (3) T\^amasik. Those ministers in whom the Sattva quality is predominant, they perform their Master's duties according to their own strength. The S\^attvik Mantris (ministers) are well versed in their Mantra \'S\^astras (the policies and statesmanship), virtuous and one-pointed in their thoughts, they never do any injury to their king and they fulfil their own purposes. The R\^ajasik Mantris are of different sorts; they are always after their own interests; at times, whenever they like, they do the State duties. The T\^amasik Mantris always look of their own interests out of their greedy nature; they serve their ends even by ruining the regal interests. It is the T\^amasik Mantris that are influenced by the bribes from the enemies, become separated at their hearts from their own masters and give out the secrets to the enemies, while staying in their homes. They always advise alienation policy like the sword ensheathed in a scabbard; and when the time of war comes, they always frighten their masters. Therefore, O King! Never put your trust on ministers; if you do so, they will always hinder you in your actions and counsels; what harm cannot be done by those ministers that are treacherous, greedy, deceitful and void of any intelligence and always addicted to vicious acts, when they are trusted! Therefore, O King! I will go myself to the battle and serve your purpose; you need not be at all anxious in this matter. I will soon bring before you that vicious woman; I will do your actions by my own strength and powers. Let you be calm; and look at my strength, fortitude and valour.

Here ends the Twelfth Chapter of the Fifth Book on the holding of counsel by Mahis\^asura in \'Sr\^i Mad Dev\^i Bh\^agavatam the Mah\^a Pur\^anam, of 18,000 Slokas by Maharsi Veda Vy\^asa.