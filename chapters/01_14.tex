\chapter{On the birth of \'S\^uka Deva and on the duties of householders}

1-70. S\^uta said :-- O Maharsis! (Now hear the main topic). Seeing the dark-blue lady looking askance at him, Vy\^asa Deva thought :-- ``Indeed! What is to be done now? This Devakany\^a Apsar\^a Ghrit\^ach\^i is not fit for my household.'' Then, seeing Vy\^asa Deva thus thoughtful, the Apsar\^a thought that the Muni might curse her and got terrified. Confounded by terror, she assumed the form of a \'S\^uka bird and fled away; Vy\^asa, too, became greatly surprised to see her in the form of a bird. The moment Vy\^asa saw the extraordinary beautiful form of Ghrit\^ach\^i, the Cupid entered then, into his body, and his mind was filled with the thought of sweet feminine form and was gladdened and all his body was thrilled with pleasure so that the hairs of the body stood on their ends. The Muni Vy\^asa Deva tried his best and exerted his power of patience to its utmost, but failed to control his restless mind to enjoy the woman. Though he was very energetic, and he tried repeatedly to control his heart, enchanted with the beautiful form of Ghrit\^ach\^i, yet he could not, as due to a state of things pre-ordained by God, control his mind. At this state, when he was rubbing the fire sticks to get the sacred fire, the two pieces of wood used in kindling the fire, his seed (semen) fell upon the Aran\^i (the two pieces of wood used in kindling the sacred fire). But he did not take any notice of that, and he went on rubbing the firesticks when arose from that Aran\^i the wonderfully beautiful form of \'S\^uka deva, looking like a second Vy\^asa. This boy, born of Aran\^i fuel, looked there brilliant like the blazing fire of the sacrificial place, whereon oblations of ghee are poured. Seeing that son,Vy\^asa Deva was struck with great wonder and thought thus :-- ``What is this? How is it that my son is born without any woman.'' Thinking for a while, he came to the conclusion, that this had certainly come to pass as the result of boon granted to him by \'siva. No sooner the fiery \'S\^uka Deva, was born of Aran\^i, he looked brilliant, like fire, by his

own tejas (spirit). At that time Vy\^asa Deva began to look with one steady gaze the blissful form of his son as a second G\^arhapatya Fire, brilliant with the Divine fire. O hermits! The river Ganges came there from the Himalayas and washed all the inner nerves of the child \'S\^uka Deva, by her holy waters and showers of flowers were poured on his head.

Vy\^asa Deva next performed all the natal ceremonies of the high-souled child; the celestial drums were sounded and the celestial nymphs began to dance and the lords of the Gandharvas Visv\^avasu, N\^arada, Tumburu and others began to sing with great joy for the sight of the son. All the Devas and Vidy\^a Dharas began to chant hymns with gladdened hearts at the sight of the Divine form, the son of Vy\^asa, born of aran\^i. O twice born ones! Then were dropped down from the sky the divine rod (Danda), Kamandalu, and the antelope skin. No sooner the extraordinarily brilliant \'S\^uka Deva was born than he grew up, and Vy\^asadeva, who is master of endless learning and how to impart them to others, performed the son's Upanayana ceremony. No sooner the child was born than all the Vedas with all their secrets and epitomes began to flash in the mind of \'S\^uka Deva, as it reigned in Vy\^asa Deva. O Munis! Bhagav\^an Vy\^asa Deva gave the name of the child as \'S\^uka as during the moment of his birth he saw the form of Ghrit\^ach\^i in the form of the \'S\^uka bird. \'S\^uka then accepted Brihaspati as his guru and began devotedly, with his whole head and heart to perform duly the Brahmacharya vow (the life of studentship and celebacy). The Muni \'S\^uka remained in the house of his Guru and studied the four Vedas with their secrets and epitomes and all the other Dharma \'s\^astras and gave Daksin\^a to the Guru duly according to proper rules, and returned home to his father Krisna Dvaip\^ayan. Seeing his son \'S\^uka, Vy\^asadeva got up and received him with great love and honour and embraced him and took the smell of his head. The holy Vy\^asa asked about his welfare and about his studies and requested him to stay in that auspicious \^a\'srama. Vy\^asa then thought of \'S\^uka's marriage and he became anxious and began to enquire where a beautiful girl of a Muni can be found. And he spoke to his son :-- ``O highly intelligent one! You have now studied all the Vedas and Dharma \'s\^astras. Therefore, O sinless one! better marry now. O son! Take a beautiful wife, and leading a householder's life, worship the Devas and Pitris, and free me from debt. There is no other way of issueless persons; he can never go to heaven; so O highly fortunate son of mine! Now enter into the life of a householder and make me happy. O highly intelligent one! I have big expectations from you; now try to fulfill them. O greatly wise \'S\^uka! After a very severe asceticism, I have got you who are

verily a Deva born without any womb. I am, therefore, your father; save me.'' When Vy\^asa spoke thus to \'S\^uka, making him sit close by, the highly dispassionate S\^uka at once made out that his father was terribly attached to the world and replied :-- ``O knower of Dharma! you have, by the power of your great intelligence, divided Veda into four parts; why are you therefore advising me so now? I am your disciple; so give me true advice. Certainly I will obey your order.'' At this Vy\^asa deva said :-- ``O son! I have got you after I had performed very severe tapasy\^a, for one hundred years, and worshipped Bhagav\^an \'sankara in the sole object of having you. O highly wise one! I will ask some king and will give you sufficient wealth for your family expenses. So that you, having attained this much desired youth, enjoy the householder's life.'' Hearing these words of the father, \'S\^uka Deva said :-- ``O father! Kindly say this to me what pleasure is there in this earth that is not mixed with pain. The happiness, that is mixed with pain, is not called happiness by the wise. O highly fortunate one! when I will marry, I will become certainly submissive to that woman; see then how happiness can be possible to one who is dependent; especially to one, dependent on one's wife. Rather freedom can be obtained one day when one is tied to an iron or wooden pillar; but never freedom will come to that man who is tied by his wife and children. As the body of man is full of urine and faeces, so is the body of the woman. The more so, when I am born of no womb, how can I find happiness there; not only in this birth, but in my previous birth, too, I had no desire to be born of any womb. How can I desire now to enjoy the pleasure of urine and faeces in the face of the bliss of self that has got no other bliss equal to it? The high-souled persons, that find pleasure in their selves, never go after the sensual pleasures of the objects of enjoyments? When I studied first, the Veda in detail, it struck me that the Vedas dealt with the \'s\^astra of Karma m\^arga (the way of action); and it is all full of Hims\^a (injury to others). Then I took Brihaspati as my Guru to shew me the way to true wisdom; but soon I found that he, too, was attacked with the dreadful disease Avidy\^a (ignorance) and plunged in the terrible ocean of world, full of M\^ay\^a. So it became quite clear to my mind, how could he save me? If the physician be diseased himself, how can he effect cures to other diseases? When I am desirous of liberation, how can I get it from a Guru who is himself deeply attached to the world; how can such a one treat my case to free me, from the disease of attachment to this world? It would be merely a farce. I bowed down to the Guru and now I am come to you to save me, frightened by this terrible serpent of Sams\^ara. Day and night the J\^ivas travel in this awful wheel of Sams\^ara, this constellation of Zodiac; they are moving like the Sun and never get any rest. O father! If

we discuss about the truth of \^atman, we will at once find that there is no trace of happiness in this Sams\^ara. As the worms enjoy pleasures in the midst of faeces, so the ignorant persons find pleasures in this Sams\^ara. Those who have studied the Vedas and other \'S\^astras and yet are attached to the world, are certainly deluded and blind like horses, pigs and dogs; no one is more stupid and ignorant than those persons. Getting this extremely rare human birth and studying the Vedanta and other \'S\^astras, if they be attached to this world, then who are the men that will attain freedom? What more wonder can you find in this world than the fact that persons, attached to wives, sons and houses; are denominated as Pundits? That man who is not bound by this Sams\^ara, composed of the three Gunas of M\^ay\^a, is Pundit; that man is intelligent and he has understood the real import of the \'S\^astras. What use can there be in studying the \'S\^astras, in vain, that teach how to bind men more firmly in this Sams\^ara, full of M\^ay\^a.

That \'S\^astra ought to be studied, which tells how a man would be liberated. The house is called "Griha" because it catches hold of a man firmly. So what happiness can you expect from the house which is like a prison? O father! I am therefore afraid. Those Pundits are certainly stupid and they are certainly deceived by the Creator, who having the birth even of men, become again imprisoned.'' Hearing these words of \'S\^uka, Vy\^asa spoke as follows :-- ``O Son! The house is never a prison, nor is it the cause of any bondage; the householder whose mind is unattached, can get Moksa, in spite of his being such. Truthful, holy, earning wealth by just means and performing, according to rules the rites and ceremonies, as stated in the Vedas and doing \'sr\^addhas duly, a householder can certainly get Moksa. See a man who is a Brahmach\^ari, who is an ascetic, who is a V\^anaprasth\^i or follows any other method or vow, all have got to worship the householder after mid-day. The religious householder, too; welcomes them all, with sweet words, and gives them food, with great love and respect, and thus does them an amount of good. For this reason the householder's stage is the most excellent of all; and I have not seen or heard of any other \^a\'srama superior to it. For this reason Va\'sistha and other \^ach\^aryas resorted to householder's life, in spite of their being endowed with great wisdom O highly fortunate one! If one performs duly the rites and ceremonies of the Vedas, there is nothing that is impracticable to him. Be it the birth in a good family, or the enjoyment of heavens say, or be it Moksa, whatever desires, it is fructifled to success. Also there is no such rule that one will have to remain in one and in the same \^a\'srama throughout his life. The Pundits who know Dharma say that pupils can go from one \^asrama to another, Therefore, O child! accept Agni (the

householder's fire) and try your best to do unremittingly your duties. O Son! Enter into a householder's life and appease the Devas, Pitris and men; procreate sons and enjoy the pleasures of household life. When old age will come, quit the house and take up the V\^anaprasth\^ashram (the third stage) and go to a forest and perform the excellent vows and then take up the dharma of the Sanny\^asa (renunciation of everything).

O Fortunate one! He who does not take a wife, is certainly maddened by these indomitable five organs of action, five organs of senses and mind. Therefore, the makers of the \'S\^astras say, that to save one self from the pernicious influences of these vicious senses, one is to take wife during his youth time and then be engaged in performing tapasy\^a during his old age. O fortunate one! In days of yore, the fiery R\^ajarsi Vi\'sv\^amitra practised very severe tapasy\^a without any food for three thousand years, and thought he was very strong and shining like fire, he was fascinated by the charm of the celestial nymph Menak\^a. And an auspicious daughter was born from the womb of Menak\^a by Vi\'sv\^amitra. My father Par\^a\'sara, though a great ascetic, was struck with Cupid's arrows at the sight of the daughter of a fisherman, named K\^ali and accepted her in the boat. What more than this, that Brahm\^a seeing his  own daughter Sandhy\^a was struck by passion and ran after her, when Bhagav\^an Rudra Deva made him unconscious by his Humk\^ar sound and made Brahm\^a desist from the attempt.

So, O fortunate one! Take my word pregnant of good issues and marry a lady, born of a good family, and follow the path presented in the Vedas.''

Thus ends the fourteenth Chapter of the 1st Skandha, on the birth of \'S\^uka Deva and the duties of householders in the Mah\^a Pur\^ana \'sr\^imad Dev\^i Bh\^agavatam of 18,000 verses by Maharsi Vedavy\^as.

