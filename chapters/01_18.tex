\chapter{On Janaka’s giving instructions on truth to \'S\^uka Deva}

1-22. S\^uta said :-- Mah\^arsis! When the king Janaka heard of the arrival of \'S\^uka Deva, the son of his Guru, he took his priest before him and attended by his ministers came before him in pure spirit. Then he duly worshipped \'S\^uka, offering him P\^adya, Arghya and an excellent seat, and a cow, yielding milk and then enquired about his welfare. \'S\^uka Deva accepted duly all the things offered by the king; and informed him of his well-being and asked the king in return, of his welfare and took his seat at ease on the \^asana. The king Janaka asked the son of Vy\^asa, full of peace :-- ``O Mah\^abh\^aga Muni Sattama! You are devoid of any attachment and you have no desires. May I enquire why such a person as your honoured self has come to me.'' \'S\^uka Deva said :-- O great king! my father told me thus :-- O child; take a wife; for the house-holder's life is the best of all the \^a\'sramas but I thought that will be the source of my bondage to this world and therefore did not obey his word, though he was my highest Guru. He then again said to me :-- If one takes a household life, it does not at once follow that he will be held in bondage; yet I did not agree to that. Then the Muni, thinking me still to be in some doubt, spoke this word of advice to me :-- ``O Son! Do not be sorry; go to Mithil\^a and have your doubts solved. There my disciple

the king Janaka, is governing his kingdom without any source of danger. He is Jivanamukta (liberated while living) and is free from the ideas of body, etc., so everybody knows. When that royal sage, Janaka, though governing his kingdom, is not seen tied up by M\^ay\^a, then O Son! why are you afraid of this Sams\^ara, when you are living this forest life.

Therefore, O Mah\^abh\^aga! Trust me and marry; and in case you doubt very much, then go and see the king Janaka; ask him and remove your doubts. He will certainly solve your doubts. But, O Son! After hearing him, come again quickly to me.'' O king! When my father spoke thus, by his permission I have come now to your capital. O king! I don't want any thing, save Moksa (liberation); therefore O Sinless one! Kindly advise what am I to do, so that I attain Moksa. O Lord of kings! Practising asceticism, going to the holy places of pilgrimage, holding vratas (vows), performing sacrifices, studying the Vedas, or earning wisdom, whatever is the cause of Moksa, kindly say that. Hearing this, Janaka said :-- ``O son of my Guru! I am telling what ought to be done by the Br\^ahmanas, following the path of Moksa; listen. After having the holy thread, a Br\^ahmin should live in the house of his Guru to study the Vedas, the Ved\^antas and pay the Dakshin\^a (the fee) to the Guru according to rules; he will then return home and marry and enter into the householder's life; he should lead a life of contentment, be free from desires, sinless and truthful and earn his livelihood with a pure heart and according to the sanction of justice and conscience. He is to perform the Agnihotra and other sacrifices; and after getting sons and grandsons, he is to leave his wife under the care of his son and then to take the life of a V\^anaprastha (3rd stage of life). That Br\^ahman, the knower of Dharma, must practise tapasy\^a and become master of his six passions (enemies); and when he gets disgusted with the world and when the Vair\^agyam (dispassion) will arise within him, he would enter into the fourth \^a\'srama. For, the man is first to enter into the householder's life and when he will be quite dispassionate towards the world, he will then have a right to take the \^a\'Srama of Sanny\^asa (Renunciation). A course contrary to this can never entitle one to the \^a\'Srama of Sany\^asa.

This is the beneficial word of the Vedas and it must hold true; it cannot be false; this is my firm belief. O \'S\^uka! In the Vedas are mentioned forty-eight Samsk\^aras (consecrations; purificatory rites); out of which the learned Mah\^atmas have reserved forty Samsk\^aras for the householders and the last eight Samsk\^aras (\'sama, Dama, etc.,) for the Sanny\^asins. And this good usage is heard to come down from very ancient times. A Br\^ahmana ought to complete his previous \^a\'Sramas successively and then enter into the succeeding \^a\'Srama.

23-30. \'S\^uka said :-- If the pure Vair\^agyam (dispassion) arising out of knowledge and wisdom (jñ\^an and Vijñ\^an) already arises (before taking to the grihasth \^a\'sram), is it still necessary to pass through house holder’s life, V\^anaprastha life, etc., or is one entitled then to take up at once the Sanny\^asa \^a\'srama, quit everything and reside in the forest? Janaka said :-- O! One giving honour to the \'S\^astras and Gurus! Though the powerful passions seem to be under control in the period of unripened Yoga (the imperfect yogic state), yet one ought not to trust them; for, it is generally seen, many imperfect Yogins find themselves disturbed by one or other of the senses. If the mind of one who has already entered into the Sanny\^asa \^a\'Sram gets perturbed in his course, then, how can he, you can see this for yourself, satisfy desires of eating good things, sleeping nicely, seeing his son, or wishing any other desires, knowing them to lead to his degradation? He is then in a very serious state. The net of desires is very difficult to be conquered by men; that can never die out. Therefore, to put an end to them, the advise is to cut them slowly and slowly. He who sleeps on an elevated place has the danger of tumbling down; but one who sleeps in a low place has no such danger. So any man who has once taken the highest dharma Sanny\^asa, and if he be fallen, then he never gets hold of the real track. As an ant begins to get from the root of tree, and, by and by, gets to the topmost part of the branches, so human beings go by degrees from one \^a\'srama to another till they go to the highest; then and then only they are able to get easily their desired truth. The birds without anticipating any danger, get up to the skies very quickly and soon they get tired and cannot go to their desired place but the ant goes with rest to its desired place. This mind is very difficult to be controlled; for this reason the men of unripened minds, cannot conquer it all at once; and are advised to conquer it, by and by, observing the laws of one \^a\'Srama after another.

31-37. See also if anybody, remaining in his household life be of a quiet temper and of good intellect, and if he takes success and failure in the same light, and be not elated in times of pleasure and not depressed in times of pains and does his duty for duty’s sake without troubling his mind with cares, and anxieties, then that householder acquires pure happiness by the realisation of his self and acquires Moksha. There is no manner of doubt in this. O Sinless One! See, I am liberated while living, though I am engaged in preserving kingdom; if any source of pain or pleasure arises, I am not in any way affected by them. As I will attain in the end Videha Mukti (liberation from bodies) though I am always wandering at my free will, enjoying various things as I like

and do various things as it pleases me, so you can do your duties and then be liberated in the end.

O Son of my Guru! When this material world, the cause of all error according to the Vedanta \'S\^astras, is simply an object of sight then how can this material substance, an object of sight, be the source of bondage to the \^atman, the Self? O Br\^ahman! Though the five material elements can be seen, their qualities or Gunas can be known only by inference, so the self is to be inferred; it can never be an object of sight; and also this self, known by inference, changeless and without any impurity or stain can never be bound by the visible changeful material thing. O Br\^ahman! This impure heart is the source of all pleasure and pains; so when the heart becomes pure and quiet, all the things then become fully pure, O Br\^ahmana!

38-41. If going often and often to all Tiraths and bathing there, do not make one's heart pure and holy, then all one's troubles are taken in vain. O Destroyer of enemies! It is the mind that is the cause of bondage or freedom; and not the body, nor the Jiv\^atm\^a (the embodied soul), nor the senses. The Self or \^atman is always pure consciousness and is ever free so, truly speaking, it can never be bound. Bondage and freedom reside on in the mind; so when the Mind gets peace, the bondage of Sams\^ara is also at an end. He is an enemy, he is a friend, he is neither an enemy nor friend, all these different thoughts reside in the mind and arise out of duality; how can the ideas of differences exist, when everything has become all one pervading self?

42-47. J\^iva is Brahm\^a; I am that Brahm\^a and nothing else; there is nothing to be discussed here. It is owing to the dualities that monism appears not clear and differences between J\^iva and Brahm\^a arise. O Mah\^abh\^aga! This difference is due to Avidy\^a and by which this difference vanishes, that is termed Vidy\^a.

This difference between Vidy\^a and Avidy\^a ought to be always kept in view, by those that are clever.

How can the pleasure from the cooling effect of the shadow, be felt, if the heating effect of the rays of the Sun be not previously experienced? So how Vidy\^a is to be experienced if Avidy\^a be not felt before? Sattva, Rajas and Tamo Gunas reside naturally in things, made of Gunas; and the five principal elements reside naturally in substances made up of elements; so the senses reside naturally in their own forms, etc.; so how can there be any stain to the \^atman which is unattached? Yet to teach humanity, the high souled persons preserve always with greatest care the respect of the Vedas. If they do not do this, then, O Sinless One! the ignorant persons would act lawlessly according to their wishes, like

the Ch\^arv\^akas; and Dharma will become extinct. When Dharma will become extinct, the Varn\^a\'Srama will gradually die out; so the well-wishers should always follow the path of the Vedas.

48-56. \'S\^uka said :-- ``O King! I have now heard all that you have said; still my doubt remains; it is not solved. O King! In the Dharma of the Vedas, there is Hims\^a (act of killing and injuring); and we hear that there is much of Adharma (sin) in the above Hims\^a.

So how can the Dharma of the Vedas give Moksha? O King! One can see before one's eyes that the drinking of Soma rasa, the killing of animals, the eating of fish and flesh and so are advised in the Vedas; so much so that in the sacrificial ceremony named Sautr\^amana the rule of drinking wine and many other vratas are clearly mentioned; even gambling is advised in the Vedas. So how can Mukti be obtained by following the Veda Dharma? It is heard that, in ancient times, there was a great king, named \'sa\'savindu, very religious, truthful, and performing sacrifices, very liberal; he protected the virtuous, and chastised those that were wicked and going astray. He performed many Yajñas, where many cows and sheep were sacrificed according to the rules of the Vedas and abundant Dakshin\^as (sacrificial fees) were presented to every one that performed their parts in the sacrifices. In these sacrifices, the hides of the cows that were sacrificed as victims, were heaped to such an enormous extent that they looked liked a second Bindhy\^achal mountain. Then the rains fell and the dirty water coming out of that enormous heap of skins flowed down and gave rise to a river which was thence called the Charmanvat\^i river. And what a wonder? That cruel king left behind him an ineffaceable fame and went to Heavens. Whatever it may be, it can never come to my head that I should perform the Veda Dharma, filled with so many acts of killing and cruelties. Again, when the man find pleasure in sexual intercourses and when they do not have that intercourse, they experience pain, how can you expect such persons to attain liberation.''

57-61. Janaka said :-- ``The killing of animals in a sacrificial ceremony is not killing; it is known as Ahims\^a; for that hims\^a is not from any selfish attachment; therefore when there is no such sacrifice and the animals are killed out of selfish attachment, then that is real hims\^a; there is no other opinion in this. Smoke arises from a fire when fuels are placed in it; and smoke is not seen when no fuel is added. So, O Munisattama! The hims\^a, as prescribed in the Vedas, is free from all blemishes, selfish attachment, etc., and therefore it is unblameable. So it follows the hims\^a committed by persons attached to objects, is the real hims\^a; that can be blamed, but the hims\^a of those persons who

have no desires is not that sort of hims\^a. Therefore the learned men that know the Vedas declare that the hims\^a done by the dispassionate persons, with their hearts free from egoism, is no hims\^a done at all. O Dvija! Really speaking, the killing of animals done by the house-holder attached to senses and their objects, and done under their impulses can be taken into account as a real act of killing; but, O Mah\^abh\^aga of those whose hearts are not attached to anything of those self controlled persons, desirous of moksa, if they do an act of Hims\^a out of a sense of duty, with no desires of fruits and with their hearts free from egoism that can never be reckoned as a real act of killing.''

Thus ends the 18th Chapter of the 1st Skandha on Janaka's giving instructions on truth to \'S\^uka Deva in the Mah\^apur\^anam \'Sr\^imad Dev\^i Bh\^agavatam.