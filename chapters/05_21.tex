\chapter{On the conquest of the Heavens by \'Sumbha and Ni\'sumbha}

1-6. Vy\^asa said :-- O King! I am describing to you that excellent pure life and doings of the Dev\^i that destroy all the sins of all the beings and make them happy. In days of yore, there were two very powerful demons \'Sumbha and Ni\'sumbha; they were two brothers, strong heroes and invulnerable by the male persons. Those two wicked Asuras were surrounded with numerable D\^anavas; they tormented always the Devas. Then the Goddess Ambik\^a, for the good of the Devas, killed \'Sumbha and Ni\'sumbha with all their attendants in a very dreadful battle. In the battlefield the Dev\^i killed their main assistants Chanda Munda and the exceedingly terrible Rakta V\^ija and Dhumralochana. When the Dev\^i destroyed those D\^anavas, the Devas became fearless; the Devas then went to the beautiful Sumeru mountain and praised Her and chanted hymns to Her.

7-8. Hearing about the names of \'Sumbha and Ni\'sumbha, Janamejaya asked :-- O best of Munis! Who were those two Asuras? How came they to be most powerful? Who put them here? Why were they vulnerable to women only? Under whose tapasy\^a and under whose boon did they become so strong? And why did that great Dev\^i kill them? Describe all these to me in detail.

9-20. Vy\^asa said :-- O King, I am describing to you that beautiful anecdote where the Dev\^i's holy deeds are involved. Hear. This incident full of all that is good, destroys the hearer's all sins and grants them all their desired ends. In days of yore, \'Sumbha and Ni\'sumbha, the two fair and good looking brothers came out of P\^at\^ala to this earth. These two Asuras, when they grew to their manhood, performed severe asceticism in Puskara, the holy place of pilgrimage, the most purifying place in this world and they refused to eat rice and water. They became so very skilled in their Yoga practices that they passed away in their one posture and seat one Ajuta (10,000) years. Thus they performed very difficult Tapasy\^a. Then the God Brahm\^a, the Grandsire of all, became pleased with their asceticism and appeared before them, riding on His vehicle, the Swan. The Creator, seeing them thus deeply merged in meditation, asked them to get up from that state and told them thus :-- ``I have become pleased with your asceticism. I fulfil the desires of all the Lokas; I have now come to you, pleased to see you so very strong in your ascetic practices; better ask your desired boons from me; I will grant them to you.'' Vy\^asa said :-- O King! Hearing thus the Grandsire's wards, \'Sumbha and Ni\'sumbha got up from their meditation; concentrating their attention towards Him, circumambulated Him and bowed down to Him with their hearts full of reverence. The two Asuras were very weak, lean and thin by their hard tapasy\^a and they looked very humble. They fell down before Him like a piece of wood and began to speak in a sweet voice, choked by intense feelings. O Br\^ahman! O Deva of the Devas! O Thou, the Ocean of Mercy! O Destroyer of fear of the devotees! O Lord! If Thou art pleased then dost Thou grant us immortality. There is nothing in this world more fearful than death; we two have taken refuge unto Thee, being afraid of this death. O Thou, Ocean of mercy! O Creator of the world! O Lord of the Devas! O Universal Soul! Protect us from this fear due to the terrible Death.

21-23. Brahm\^a said :-- Is this the boon that you ask? This is in every way, against the Law of Nature; for no one, in these three Lokas, can grant this boon to anybody. When one becomes born, one must die; and when one dies, one must be born again. This Law is ordained in this world by the Supreme Creator of this Universe, from time immemorial. Therefore all the beings must die; there is no doubt in this. Better ask any other boon that you desire; I will grant that to you.

24-27. Vy\^asa said :-- O King! Hearing thus the words of Brahm\^a, the two D\^anavas pondered over the matter and bowed down to the Praj\^apati, the Lord of the Creation and said :-- O Merciful One! Grant us then so that we shall be invulnerable to any of the male of the Immortal Devas down to human beings and birds and deers; this is the boon that we ask. Where exists the woman so powerful as to kill us? We never fear any woman in all the three Lokas. O Lotus-born! We, the two brothers, want not to be killed by any male; the females are naturally weak therefore we need not fear them.

28-58. Vy\^asa said :-- O King! Hearing their words, the Grandsire Brahm\^a gladly granted them their desired boon and returned to His own abode. On Brahm\^a going away, the two D\^anavas, too, returned to their own places. They then appointed the Muni Bhrigu as their priest and began to worship him. Bhrigu, the best of the Munis, then, on an auspicious day and when the star was benign, got a beautiful golden throne built and gave it to the king. \'Sumbha, being the eldest, was then installed on the auspicious throne as the king; the other brave and excellent demons began to assemble there quickly for serving him. The two great warriors Chanda and Munda, proud on account of their great strength came there with their large armies, chariots, horses, and elephants. Similarly the valiant warriors Dhumralochana, hearing that \'Sumbha had become their King, came there with his own army. There came up also at that time the great warrior Rakta V\^ija, more powerful on account of his getting a boon, attended by his army of two Aksauhin\^i soldiers. O King! Hear why this Rakta V\^ija became so very unconquerable; whenever this Asura was wounded by any weapon, if one drop of blood fell on the ground, at once would be created so many innumerable Asuras, resembling his wicked nature and with similar weapons in their hands. The Asuras born of this blood would have similar appearances and would be similar in strength and ready to fight at once when they were born. That great warrior, the great Demon Rakta V\^ija was unconquerable in battle for this very reason and no being could now kill him. The other Asuras, when they heard that \'Sumbha had become their king, came up there with their armies consisting of four divisions of elephants, chariots, cavalry and infantry and began to serve him. The army of \'Sumbha and Ni\'sumbha thus became countless; and they forcibly conquered and got possession of all the kingdoms that existed then on the surface of the earth. Then Ni\'sumbha, the destroyer of enemies, collected his army and marched up to the Heavens without any delay to conquer Indra, the Lord of \'Sach\^i. He fought very hard with all the Lokap\^alas on all sides when Indra struck him on his breast with His thunderbolt. Ni\'sumbha fell unconscious on the ground with that blow when his soldiers, defeated in the battle, fled away on all sides. \'Sumbha, the destroyer of the enemie\'s forces, hearing the unconscious state of the younger brother, came up at once on the field and shot at the Devas with multitudes of arrows. The untiring \'Sumbha fought so violently that Indra and the other Devas and Lokapalas were defeated. \'Sumbha then took away, perforce, the position of Indra and he occupied the Celestial Tree and Heavenly milching cow that yielded all desires and other excellent things over which Indra used to reign. In fact, that high-souled Asura got the dominion of the three Lokas and took away all those that were offered at the sacrifices. He became highly glad on getting the Nandana Garden and was extremely delighted when he drank the celestial nectar. He then defeated in battle Kuvera, the god of wealth and occupied his kingdom. He defeated the Moon, Sun, and Yama, the God of Death and occupied their positions. Surrounded by his army, Ni\'sumbha dispossessed Varuna, Fire, and Air of their kingdoms and began to reign in their stead. Thus deprived of their kingdoms, prosperity and wealth, the Devas left the Nandana Garden and fled, out of terror, to the caves of hills and mountains. Thus deprived of all their rights, the Devas without any weapons, without any lustre, without any home, and without anywhere to go, began to wander in lonely forests. O King! All the Immortals began to knock about in lonely gardens, mountain caves and rivers; and nowhere they found happiness; for happiness depends entirely unto the hands of Fate. O Lord of men! Even those fortunate souls, who are powerful, and wealthy and wise, meet at times with distress and poverty. O King! How marvellous are the ways and manners of Time! It makes kings and donors beggars; it renders the powerful, weak; literates, illiterates; and it makes great warriors into terrible cowards. O King! V\^asava performed one hundred horse-sacrifices and got the excellent Indra's position; but again be fell into extreme difficulties; thus runs the wheel of Time.

59. It is Time that bestows the gem of knowledge to a person and it is Time again that deprives that very same man of his wisdom and makes him a great sinner.

60-61. The Bhagav\^an Visnu takes incarnations, under the control of Time, in several lower wombs as boar, etc., and Mah\^a Deva carries on His body the human skulls, that are not even fit to be touched. When Brahm\^a, Visnu, Mahe\'sa and others suffer such painful things, then one need not wonder at the workings of the Great inscrutable Time.

Here ends the Twenty-first Chapter of the Fifth Book on the conquest of the Heavens by \'Sumbha and Ni\'sumbha in \'Sr\^i Mad Dev\^i Bh\^agavatam, the Mah\^a Pur\^anam of 18,000 verses by Maharsi Veda Vy\^asa.

