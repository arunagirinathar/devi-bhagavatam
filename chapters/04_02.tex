\chapter{On the supremacy of the effects of Karma}

1. Sûta said :-- The learned Vy\^asa, the son of Satyavat\^i, and the knower of the Pur\^anas, when thus asked by Janamejaya, the son of Par\^iksit, whose heart had become calm, replied in the following words, capable to remove all his doubts. Vy\^asa said :--

2. O king! You would better know, that in this Universe the course of Karma is not easily comprehensible; even the Devas are not capable to comprehend the wonderful effects of actions; what to speak of men!

3. When this Universe composed of the three Gunas arose, it was through Karma, that everything had its origin.

4-5. It was the seed of Karma whence the J\^ivas (the individual embodied souls) arose with neither any beginning nor any end. Those J\^ivas go often and often incarnating in numberless varieties of wombs and then go to dissolution. When this Karma ceases, the J\^ivas then are never to have any more connection with any other body.

6. The Karmas done by J\^ivas are of three kinds :-- auspicious, inauspicious and mixed (partly auspicious and partly inauspicious); of which the auspicious is the Sattvik Karma, the inauspicious is the Tamasik Karma and the mixed, is the R\^ajasic Karma. Thus have been said by the Sages.

7. These three again are subdivided into three. They are Sanchita (accumulated), Bhavisya (impending in future) and Pr\^arabdha (commenced). All these Karmas are in dwelling always with the body.

8-9. O king! Everybody, even Brahm\^a, Visnu and Mahe\'sa all under the influence of this Karma! And they experience pleasure, pain, old age, disease and death, joy and sorrow, lust, anger, greed and other bodily qualities, out of the effects of this Karma, which we call ordinarily Fate.

10-11. Therefore love, hatred and other bodily qualities all predominate equally in all bodies. Anger, jealousy, hatred, and other similar qualities arise in the Devas, men, and birds owing to some sort of dislikes on previous occasions; and love, compassion, pity, etc., arise out of some sort of likings, existing already.

12-13. O king! No individual can arise without some sort of action or other. It is through Karma that the Sun traverses in the sky; it is through Karma that the Moon was attacked with consumption disease; and it is through Karma that the Rudra holds the disc of skull bone. This Karma, therefore, has no beginning nor end (till Moksa); now that this Karma is the sole cause in the production of this Universe.

14-16. For this reason, this whole Universe, moveable and immoveable, is real; but Munis are deeply absorbed in meditation to ascertain about its reality or unreality. They cannot definitely know it for certain whether this world is real or unreal; for where M\^ay\^a is prevalent, the universe is there. Where there is the cause fully existing in all respects, there is no effect, how can we say? The M\^ay\^a is eternal and always acts as the Prime Cause of all.

17. Therefore, O king! the sages declare that the seed of Karma is eternal. This whole universe changes incessantly, being controlled by this karma.

18. O king of kings! They say, it is through the will of Visnu, of unbounded energy and splendour, that all this universe enters, again and again, into all sorts of wombs, whether good or evil.

19. Now, if the birth of Visnu, of infinite prowess, takes place according to His will, then why is it that He travels through many impious births? Why is it that Bhagav\^an Visnu goes, in different Yugas, to take His births in low, vile origins?

Where is that self dependent man, who, leaving his abode Vaikuntha and all sorts of pleasures and happiness, desires to live in this mortal temple, filled with urine, faeces and other filthy matters.

20. No intelligent man will leave comfortable resting places and amorous sports and gathering flowers for the sake of dwelling in this uterus in the womb?

21. Who likes to live with his face downwards in the womb, when he can enjoy fine heavenly soft downs, puffed up with cotton or silk.

22. Who will abandon singing, dancing and music, where all sorts of love feelings are being manifested, and think of coming down to this veritable Hell?

23. Who will abandon the wonderful ambrosial nectar and prosperity given by Laksm\^i, that cannot be easily renounced, and then like to taste this urine and faeces.

24. There is no hell more aggravating in the three worlds than this existence in the wombs. The Munis, afraid of these, perform difficult asceticisms in this wondrous world.

25. Wise, intelligent persons renounce their kingdoms and enjoyments and resort to forests. Who is there so stupid as to enter willingly in the various wombs?

26. Worms and insects torment the J\^ivas in the womb; the digestive fire of the stomach heats it from below, whereas it is always fearfully tied down on all sides by the flesh, enclosing its fat or marrow. O King; Not a trace of happiness is visible there.

27. It is far better to live in a prison house, fettered by hard iron chains. Whereas it is not desirable to live for a moment in the womb.

28. It is very hard and painful to be in the womb for ten months. To come out of the hard and terrible womb is extremely troublesome.

29. J\^ivas get trouble in their childhood; they cannot speak, and they do not know what to say, when they are hungry or thirsty; they depend entirely on others and they are grieved.

30. When the child becomes hungry and cries, the mother becomes anxious. When the child is afflicted with diseases and cries, the mother then knows and administers medicines.

31. Thus many troubles arise in childhood. Sages do not therefore find any happiness and do not desire, of their own accord, to come here.

32. O king, no sane man, would forego incessant heavenly pleasures and prefer before the Devas to this toilsome and painful state of being born in the womb.

33. O king of kings! All the Devas, Brahm\^a and others have to enjoy full the effects of their Karmas done, whether they are pleasant or painful.

34. O best of kings! The fruits of karma must have to be experienced, whether auspicious or inauspicious, be he a Deva, or human being or an animal; any one who has embodied himself in fine or gross bodies!

35. Human beings, by dint of their practise of penance, religious austerities alms givings and sacrifices, rise to Indrahood. Indra, in his turn, when the effects of his good actions cease, comes down to inferior births! there is no doubt of it.

36. In the Rama Incarnation, the Devas had to incarnate themselves a Monkeys; and in the Krisna incarnation, the Devas had to incarnate themselves as human beings, Cow-herds (Gopas) and Y\^adavas.

37. Thus being urged on by Brahm\^a, Visnu Bhagav\^an incarnates Himself many times, yugas after yugas, to preserve the religion.

38. O king of mortals! Thus, like a carwheel, Bhagavàn Hari incarnated Himself in various wombs successively in a wonderful manner.

39. The destruction the of Daityas was done by Hari in His many secondary incarnations.

40. Now I will narrate to you the auspicious facts of the birth of Krisna, Who incarnated Himself in the family of Yadu (Yadu Kula).

41. O king! The illustrious V\^asudeva, born of the part of the Muni Ka\'syapa, had to take his birth again as a human being due to his previous curse and had to maintain his livelihood by tending cows.

42. O best of kings! And the two wives Ka\'syapa, Aditi and Suras\^a had to take their births as the two sisters, Devaki and Rohin\^i, on account of the curses cast on them. O Descendant of Bharata! We have thus heard that they were greatly cursed at one time by Varuna, the water deity, who got very much angry. The king said.

43-47. What fault was committed by Ka\'syapa that he had to take his birth along with his wife as cowherds. And why was it that the Everlasting uninterrupted Atman Visnu N\^ar\^ayana had to take his birth in Gokula. He whose abode is Vaikuntha, who is the Lord of Rama! who is Bhagav\^an and the Supreme amongst the gods, who is the upholder of the universe and the yugas! Under Whose order can such a being abandon his abode and take his birth in the world like an ordinary mortal? There is this grave doubt, then, of mine on this point.

48-51. Obtaining this depraved human coil, one is always perplexed with various thoughts, sometimes with lust, anger, jealousy, intoleration, sorrow, enmity sometimes with pleasurable feelings, happiness, fear, sufferings, penury, sometimes with straight-forwardness, good or bad deeds, faithfulness, treachery, unsteadiness, supporting others; sometimes with remorse, hesitation, bragging, greed, vain boasting, delusion, or hypocrisy and sometimes with remorse; these different feelings exist in men.

52. How then can Visnu Bhagav\^an abandon His eternal pleasures and have recourse to this human birth, full of many perplexing thoughts.

53. O best of Munis! What peculiar happiness is there in the pleasures of human births, that \'Sr\^i Bhagav\^an Hari has to undertake the burden of dwelling thus in the human wombs?

54-55. O Mun\^indra! The sufferings that are experienced, while in the womb, the pain during the time of delivery, the misfortunes in the early childhood, the troubles of passionate lust in youth, the greater sorrows and difficulties in the householder's life, all these are existent there; how then Bhagav\^an Visnu incarnate Himself often in these various human births.

56-57. What an amount of enormous difficulties had Brahm\^a-born Hari to undertake in His R\^ama incarnation! That high souled One had to suffer for his exile in forest, for the stealing away of his wife S\^it\^a, for the frequent wars, for the final separation from his wife S\^it\^a.

58-59. Likewise in the Krisna Avat\^ara, the birth in a prison, the departure to Gokul, tending cows, the killing of Kamsa, departure to Dw\^ark\^a with great difficulty and all sorts of household difficulties were there. Why had He to suffer all these?

60. Who amongst the wise and the emancipated, of his own accord condescends to take on his shoulders so many hard sufferings? This is the grave doubt in my mind; be graciously pleased to remove my this grave doubt and make my mind tranquil.

Here ends the Second Chapter in the Fourth Book of \'Sr\^i Mad Dev\^i Bhag\^avatam of the Mah\^a Pur\^anam of 18000 verses by Mahars\^i Veda Vy\^asa.