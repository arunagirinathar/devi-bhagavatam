\chapter{On the excellency of the Dev\^i}

1-3. The Risis said :-- O Saumya! How was \'Suka Deva born? Who studied these Pur\^ana Samhit\^as; by which wife of Vy\^asa Deva? And How? O highly intelligent one! You have just spoken that \'Suka Deva was not born from womb, in the natural way; he was born of the dry pieces of wood for Homa sacrifice. But we heard before that the great ascetic was Yogi even in his mother's womb, so a great doubt comes to our minds. You better remove that to-day; how he studied also these Pur\^anas, as vast in their nature; say this.

4-5. S\^uta said :-- In long-past days, Satyavati's son Vedas Vy\^as, while in his own hermitage on the banks of the river Sarasvati, was greatly wondered to see a pair of Ch\^atakas (Sparrows). He saw the pair putting  the beak of their young one, just born of the egg, of beautiful body, red mouth, and greasy body. They do not care at all for their own hunger and toil; all they are caring for is to nurture their young one. He said also that the pair are rubbing their bodies over the body and kissing lovingly the mouth of the young one and feeling the highest pleasure. Seeing this wonderful affection of the two sparrows towards their young, Veda Vy\^as became very anxious and thought over the following in his mind.

9-14. Oh! What wonder is there, when the birds have so much filial affection towards their child, that men, who want services from their sons, would show their affection towards their sons! This pair of sparrows will not perform the happy marriage of their young one and will not see the face of their son’s wife; nor when they will grow old, that their child would become very religious and serve them to attain great merits in Heaven. Nor do they expect that their child would earn money and satisfy them nor the child would perform when they die, their funeral obsequies duly and help them in their sojourn in the next world; nothing of all these. Nor will the child perform the \'Sr\^adh ceremony at Gay\^a; nor will the child offer the oblation of a blue bull on the day of offering the sacrifice to its ancestor (the bull is then let loose and held sacred); yet the pair of sparrows have so much affection towards their young one! Oh! in this world to touch the body of the son, especially to nurture the sons, is the highest happiness in life.

15-27. There is no prospect in the after birth of the sonless; never, never will Heaven be his. Without son, there is none other who can be of help in the next world. Thus in the Dharma \'S\^astras, Manu and other Munis declare that the man who has sons goes to Heaven and the sonless one can never go to Heaven. The man possessing a son is entitled to the Heavenly pleasures can be vividly seen, rather than imagined. The man with son is freed from sins; this is the word of the Vedas. The sonless man becomes very much distressed even at the time of death and while lying on bed that is ground at that time, mournfully thinks. ``This all my vast wealth, various things, this my beautiful house, who will enjoy all these?''

When the sonless man is thus perplexed in his mind at the time of his death and becomes restless, then it is sure that his future career is full misfortunes; unless one’s mind is calm and serene at the time of death, can never attain a good goal. Thus thinking variously, the Satyavati's son Veda Vy\^as sighed heavily and became unmindful. He thought of various plans and at last, coming to a definite conclusion, went to the Sumeru mountain to perform Tapasy\^a. On reaching there, he thought which Deva he will worship! Visnu, \'Siva, Indra, Brahm\^a, Surya, Gane\'Sa, K\^artikeya, Agni, or Varuna? Who will grant him boon quickly and thus satisfy his desires. While thus cogitating in his mind, came there the Muni N\^arada, of one mind with lute in hamd, accidentally in his course of travels. Seeing N\^arada, the Satyavati’s son Veda Vy\^asa gave him a hearty welcome, with great gladness, offering him Arghya and \^asan (seat) and asked about his welfare. Hearing this question of welfare, N\^arada Muni spoke :-- ``O Dvaip\^ayan! Why do you look so care worn! First speak this out to me''.

28-30. Veda Vy\^asa said :-- `` The sonless man has no goal; therefore there is no happiness in my mind; I am always anxious to get a son and therefore I am very sorry. To-day my mind is sorely troubled with the one idea, which Deva I may satisfy by my tapasy\^a, who will grant me my desires; now I take your refuge. O merciful Maharsi! You are omniscient; say this quickly; which Deva I will take for my refuge, who will grant me a son''.

31-37. S\^uta said :-- Thus questioned by Krisna Dvaip\^ayan Veda Vy\^asa, the high souled N\^arada Muni, well versed in the Vedas, became very glad and spoke thus :-- O highly fortunate Par\^a\'sar\^a’s son. The question that you have asked me to-day was formerly asked by my father to N\^ar\^ayana. At this, N\^ar\^ayana Vasudeva, the Deva of the Devas, the Creator, Preserver and Destroyer of the Universe, the husband of Laksm\^i, the four armed, wearing yellow garment, holding conchshell, discus, club and with the mark \'Sr\^ivatsa (a mark or curl of hair on the heart of Visnu) adorning His breast and decorated with  Kaustuvagem, the Divinity Himself, became merged in great Yoga; at this my Father became greatly surprised and said :-- ``O Jan\^ardana! Thou art the Deva of the Devas; the Lord of the Present, the Past and the Future, the Lord of this Universe; why art thou meditating in Yoga? And what is it that Thou art meditating? O best of the Devas! Thou art the Lord of the entire Universe and yet Thou art now merged in deep meditation. At this I am greatly surprised (my surprise is not without foundation; Thou canst Thyself see). What more wonderful than this can happen?

38-43. O Lord of Rama! I am sprung from the lotus from thy navel and have become the Lord of this whole universe; who is there in this universe that is superior to Thee; kindly say this to me. O Lord of the world? Thou art the Origin of all, the Cause of all causes, the Creator, Preserver and Destroyer and the capable Doer of all actions. O Maharaja! at Thy will, I create this whole universe and Rudra destroys iu due time this world. He is always under Thy command. O Lord! By Thy command the Sun roams in the sky; the wind blows in various auspicious or inauspicious ways and the fire is giving heat and the cloud showers rain. I don’t see in the three Lokas any one superior to Thee. Then whom art Thou meditating while being questioned by his very intelligent son \'Suka Deva! not born in the usual way from womb, Dvaip\^ayana expounded all the secret excellent meanings of the Pur\^ana and thereby I also came to know them also. O saintly persons! Thus \'Suka Deva, sincerely earnest to cross this endless bottomless ocean of \'Samsara, tasted of the wonderful traits of the Veda, the Kalpa tree, this \'Srimad Bh\^agavata with its numerous stories and anecdotes with great eagerness and intense pleasure.

38-43. Oh! Who is there in this world that is not freed from this terror of Kali, after he has heard this Bh\^agavata. Even if the greatest sinner, void of the right ways of living and Achara as ordained in the Vedas, hears on a pretence this excellent Dev\^i Bh\^agavata, the chief of the Pur\^anas, he enjoys all the great enjoyments of this world and in the end attains the eternal place occupied by the Yogis. She who is rare, in Her Nirguna aspect, to even Hari and Hara, who is very dear as Tattva Vidya to the Jñanins whose real nature can be realised only in Sam\^adhi, She resides always in the cavity of the heart of the hearers of the Bh\^agavata Pur\^ana. He who getting the all qualified human birth and getting the reciter of this Pur\^ana, the boat to cross, as it were, this world, does not hear this blissful Pur\^ana, he is certainly deprived by the Creator. How is it that the way-ward dull-headed persons, getting the vicious ears, can hear always the faults and calumnies of others, that are entirely useless, and cannot hear this pure Pur\^ana that contains the four Vargas :-- Dharma, Artha, Kama, and Mokhsa?

This is my main point of doubt. O One of good vows! I am Thy devotee; be merciful to me and speak this to me. There is almost nothing that is secret to Mah\^apurusas; this is a well-known fact''.

44-50. Thus hearing Brahm\^a's words, Bhagavan N\^ar\^ayana spoke :-- ``O Brahm\^an! I now speak out my mind to you; listen carefully. Though the Devas, D\^anavas and men and all the Lokas know that You are the Creator, I am the Preserver and Rudra is the Destroyer, yet it is to be known that the saints, versed in the Vedas, have come to this conclusion by inference from the Vedas that the creation, preservation, and destruction are performed by the creative force, preservative force and destructive force. The Rajasik creative force residing in you, the  Sattvik preservative force residing in me, and the Tamasik destructive force residing in Rudra are the all-in-all. When these Saktis become absent, you become inert and incapable to create, I to preserve and Rudra to destroy.

O intelligent Suvrata! We all are always under that Force directly or indirectly; hear instances that you can see and infer. At the time of Pralaya, I lie down on the bed of Ananta, subservient to that Force; again I wake up in the time of creation duly under the influence of Time.

51-54. I am always subservient to that Maha \'Sakti; (under Her command) I am engaged in Tapasy\^a for a long time; (By Her command) some time I enjoy with Lakshm\^i; some time I fight battles, terrible to all the Lokas, with the D\^anavas, involving great bodily troubles. O Know of Dharma! It was before Your presence that I fought hand to hand fight for five thousand  years before Your sight on that one great ocean in long-past days with the two demons Madhu and Kaitabha, sprung from the wax of my ear, maddened with pride; and by the grace of the Dev\^i, successfully killed the two D\^anavas.

55-61. O highly fortunate one! you realised then the great \'Sakti, higher than the highest and the cause of all causes; then why are you asking again and again that question. By the will of that \'Sakti, I have got this idea of man and roam on the great ocean; in yuga after yuga, I assume by Her will, the Tortoise, Boar, Man-Lion, and Dwarf incarnations. No one likes to take birth in the womb of inferior animals (especially birds). Do you think that I willingly take unpleasant births as in the womb of boars, tortoise, i.e., certainly not. What independent man is there who abandons the pleasurable enjoyment with Laksmi and takes birth in inferior animals as fish, etc. or leaves his seat on the seat of Gaduda and becomes engaged in great war-conflicts. O Svayambhu! In ancient days you saw before your eyes that my head was cut off when the bowstring suddenly gave way; and then you, brought a horse's head and by that help, the divine artist Visvakarma, stuck that on to my headless body. O Brahm\^a! Since then I am known amongst men by the name of  ``Hayagr\^iva''. This is well-known to you. Now say, were I independent, would such an ignominy have happened to me? Never. Therefore I am not independent; I am in every way under that \'Sakti. O Lotus-born! I always meditate on that \'Sakti; and I do not know any other than this \'Sakti''.

62-66. N\^arada said :-- Thus spoke Visnu to Brahm\^a. O Muni Vedavy\^as! Brahm\^a spoke these to me. So you, too, better meditate the lotus feet of Bh\^agavati calmly in the lotus of your heart for the success of your idea. That Dev\^i will give you all that you wish. S\^uta said :-- At these words of N\^arada, Satyavati's son Veda Vy\^asa went out to the hills for Tapasy\^a, trusting the lotus feet of the Dev\^i as the all-in-all in this world.

Thus ends the fourth chapter of the first Skandha on the excellency of the Dev\^i in the Mahapur\^ana \'Srimad Dev\^i Bh\^agavatam of 18,000 verses.

