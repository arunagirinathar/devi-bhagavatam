\chapter{On the installation of the Dev\^i in Ayodhy\^a and Benares}

1-4. Vy\^asa said :-- The king Sudar\'sana, surrounded by his friends, on coming to the palace at Ayodhy\^a, bowed down to L\^il\^avat\^i, the mother of \'Satrujit, and said :-- ``O mother! I swear by touching your feet, that I have not killed in battle your son \'Satrujit nor your father Yudh\^ajit; it is the Dev\^i Durg\^a that has killed them; I am not to be blamed a bit in this. O mother! You need not be sensitive in this; there is no remedy for what will inevitably come to pass; therefore you do not be sorry for the death of your son; you must know that the J\^ivas enjoy pleasure and pain as the results of their own Karmas.

5. O mother! I am your servant; you are entitled to the same respect and worship as Manoram\^a, my own mother; there is no difference whatsoever between her and you.

6. O mother! One must bear the effects of one's Karma, good or bad; therefore when pleasure or pain arises, you should not be glad or otherwise.

7. When pain arises, more pain is said to be conceived and when pleasure arises, more pleasure is seen. But the learned say that man ought not to subject himself to excessive pleasure or pain.

8. O mother! This whole world is under Fate, Destiny; nothing of it is yours. Therefore the intelligent persons ought not to grieve their hearts at any time with sorrow.

9. As the wooden dolls dance in a stage as danced by the actor, so the individual souls here work as the results of their past Karmas; there is no doubt in this.

10. O mother! I know that the effect of one's own Karma, must have to be borne; it is, on that account, that I never felt sorrow in my exile in the forest.

11. You are quite aware that my mother's father was killed here, and my mother, becoming very much afraid and sorrowful, took me and escaped to the forest.

12-13. The robbers robbed us of everything save our clothes on our bodies; I was then very young; my mother was without any shelter; she carried me with this minister Vidalla and my helpless nurse to the hermitage of Bh\^aradv\^aja.

14. There the kind hermit and his wife and the other wives of the hermits protected our lives in that forest, with the roots and fruits, that can be obtained there in that forest. Thus our time passed.

15. Mother! I had felt no pain then; nor do I feel any pleasure at present, when wealth is flowing unto me. What more to say, I have no feeling of jealousy or envy whatsoever in my mind.

16. O mother! Rather it is better, in my eyes, to subsist on roots and fruits than to enjoy kingdoms; for the kings go to hell; but the ascetics living on roots and fruits never meet with that result.

17. The wise should undoubtedly practice Dharma and control their passions and thus save themselves from being led into hells.

18-19. O mother! The human birth in this auspicious Bh\^aratvarsa is seldom obtained. The enjoyments in eating and drinking are possible in every womb, but it is highly incumbent on us when we have got the privilege of this human birth, to earn Dharma, leading to the Heavens and salvation which can be very rarely attained in being born in other wombs.''

20-21. Vy\^asa said :-- When Sudar\'sana had told thus, L\^il\^avat\^i became very abashed; she cast aside the sorrow for the death of her son, told him with tears in her eyes :-- ``O my son Sudar\'sana! I am very much guilty on account of my father Yudh\^ajit killing your mother's father and taking hold of the sovereignty of this kingdom.

22. I could not then hinder my father and son; whatever unlawful evil and cruel deeds were then committed, all were done by my father Yudh\^ajit. Therefore, my child, I am not to be made guilty in any way in these doings.

23. Both my father and son were killed out of the wickedness of their own actions; how can you account for those wicked things? Child! I am not expressing sorrow at the death of my son; I have been pained by his doings.

24-25. O noble souled one; You are my son; Manoram\^a is my sister; Child! I am not at all offended with you nor am I the least sorry for your obtaining the kingdom; Child! you are very fortunate; therefore you have obtained, by the grace of Bhagavat\^i, this kingdom without any enemies; now rule your subjects according to the prescribed rules of Dharma.''

26-28. Vy\^asa said :-- O king! The king Sudar\'sana heard L\^il\^avat\^i and bowed down at her feet. Then he went to the beautiful palace where Manoram\^a had previously gone and began to live there. Inviting the ministers and the astrologers, he asked them what was the auspicious day and the auspicious moment, that he can establish Durg\^a Dev\^i on a beautiful golden throne and he would worship Her.

29. ``O ministers! First I will install on the throne the Dev\^i, the Awarder of the four main objects of human pursuits (viz. virtue, wealth, enjoyment and final beatitude) and then I will govern my kingdom like the kings \'Sr\^i R\^ama Chandra and others.

30. All the people of this city of Ayodhy\^a ought also to worship this Auspicious \'Sakti, the Highest Energy, the Giver of all desires and Siddhis, and that is respected and adored by all.''

31. The ministers, on hearing his words, had a beautiful palace built by the engineers, artists and workmen and proclaimed in the city the king's proclamation.

32. Then the king Sudar\'sana had an image of the Dev\^i nicely built and got that installed with the help of the Pundits, versed in the Vedas, on an auspicious day and at an auspicious moment.

33. The intelligent king performed the worship and Homa ceremony, according to the prescribed rules, and thus finally settled the ceremony of invocation of the Deity into the new image and established it as an idol in the temple.

34. O Janamejaya! There the soundings of the various drums and other musical instruments, the chanting of the Veda mantrams by the Br\^ahmanas, and sweet music were heard; and various sorts of festivities and rejoicings were celebrated.

35. Vy\^asa said :-- Thus completing the installation ceremony of the Durg\^a Dev\^i by the Br\^ahmanas, versed in the Vedas, the king Sudar\'sana duly worshipped the image in various ways, etc.

36. Thus gaining his father's kingdom and worshipping the Dev\^i, he and the Dev\^i became celebrated throughout the kingdom.

37. The religious largehearted Sudar\'sana, on gaining his kingdom, brought all the other feudatory princes under his control by the sheer force of his religious character.

38. The subjects became happy and got honor in the reign of Sudar\'sana, as they got before in the reigns of Dil\^ip, Raghu and R\^amachandra.

39. The virtue of all the citizens under Varn\^a\'srama shone complete with all its four p\^adas; and there remained none in the world irreligious.

40. In villages after villages, the chief townsmen began to build temples, worship the Goddess there with all their jolliness. Thus everywhere in the Kosala kingdom spread the Dev\^i worship.

41. On the other hand, the king Sub\^ahu established the Idol in Benares, had temples built and worshipped there the Dev\^i.

42. The inhabitants of K\^a\'s\^i became then filled with devotion and intense love towards the Dev\^i and duly worshipped Her, as they used to do to \'Siva in the temple of Vi\'svan\^atha.

43. Thus the Durg\^a Dev\^i became very widely celebrated in this world. O king! Thus in different countries, the devotion began to increase towards the Goddess.

44. The Dev\^i Bhagavat\^i Bhav\^an\^i became in every way an object to be worshipped and adored by all people and everywhere in Bh\^aratavarsa.

45. The people began to recite slowly, meditate, and chant hymns as advocated by the Âgamas constantly and became deeply attached to the \'Sakti worship and began to be looked upon with the highest honour by others.

46. O king! From that time all the people used to worship, perform Homa ceremony and sacrifice duly in honour of the Dev\^i in every Navar\^atri (for the first nine days of the bright half in the months of Â\'svin and Chaitra).

Here ends the 25th Chapter on the installation of the Dev\^i in Ayodhy\^a and Benares in the Mah\^a Pur\^anam \'Sr\^i Mad Dev\^i Bh\^agavatam of 18,000 verses by Maharsi Veda Vy\^asa.