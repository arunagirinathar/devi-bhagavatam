\chapter{On shewing the departed ones}

S\^uta said :--  The chaste Draupadi was the common wife of all the five very beautiful sons of Kunti; and she bore five sons, one to every husband. Arjuna had one wife more; she was Subhadr\^a, the sister of \'Sr\^i Krisna. By the order of \'Sr\^i Krisna, Arjuna stole her away (took her by force). The great hero Abhimanyu was born of Subhadr\^a. This Abhimanyu and the five sons of Draupadi were killed in battle. Abhimanyu's wife Uttar\^a was the charming daughter of the king Vir\^at. She gave birth to one dead child, after all the boys, the descendants of the family were extinct. The above child died out of the arrows of A\'Svatth\^am\^a. The extraordinarily powerful \'Sr\^i Krisna Himself made alive again this his sister's dead grandson. As this son was born after the family had become extinct, he became known in the world by the name of Par\^iksit. When the sons were all destroyed, Dhritar\^astra became very sorry, and, tormented by the arrow-like words of Bh\^ima, remained in the kingdom of the P\^andavas. G\^andhar\^i, too, exceedingly distressed on the bereavement of the sons, remained there also. Yudhisthira, day and night, served Dhritar\^astra and G\^andh\^ari. The greatly religious Vidura always used to console, by the advice of Yudhisthira, his brother Dhritar\^astra, who possessed the eye of wisdom and he remained by his brother's side. Dharma`s son Yudhisthira used to serve his uncle Dhritar\^astra in such a way as he might forget the pain of the death of his sons. But Bh\^ima used to pierce his heart by

his arrow-like words that he pronounced so loud as to reach the ears of the old king Dhritar\^astra. Bh\^ima used to-say :-- ``In the battle field I killed all the sons, of the wicked blind king (Dhritar\^astra) and it was I that sucked well and drunk, full to the brim, the blood of the heart of Duh\'S\^asana. Now this blind king eats shamelessly like a crow and a dog, the mass of food (Pinda) given by me, and thus is bearing uselessly the burden of life. Daily Bh\^ima used to tell, thus, harsh words to him; whereas the religious Yudh\^isthira used to console him, saying ``Bh\^ima is a quite illiterate brute'' and so forth. The king Dhritar\^astra remained there with a grieved heart for eighteen years; the one day he proposed to the Dharma's son Yudh\^isthira about his intention to dwell in forest thus :-- ``To-day I wish to perform Tarpanas in the names my sons. True it is, that Bh\^ima performed the funeral obsequies of them all; but, having in view of the former enmity, he did not do anything for my sons. If you give me some money, I will, then, perform the funeral obsequies of my sons and then retire to the forest to perform tapasy\^a that I can go to Heaven.'' Vidura also asked Yudh\^isthira privately pay to Dhritar\^ashtra the sum that he wanted; Yudh\^isthira also intended to pay the required money. Then Yudh\^isthira, the lord of the world call his younger brothers and addressed them as follows :-- ``O highly fortunate ones! Our revered uncle is desirous to perform the funeral obsequies of his sons; so we will have to give him some money for the purpose.'' Hearing these words of his elder brother of indomitable valour, Pavana's son, the mighty armed Bh\^ima became very angry and spoke out follows :-- ``O highly lucky one! Is it that we will have to give wealth for the spiritual benefit of Duryodhana and others? What a great stupidity can there be than the fact that such a malevolent blind king is deriving so great happiness at your hands? O Ârya! It is by your bad counsel that we suffered endless troubles in the forest and the extremely good Draupadi was brought before the public in the hall by Duh\'S\^asana. O one of good vows! It is for your satisfaction alone that we, though we were very mighty, had to remain in the house of Matsya R\^aj Vir\^at as servants. Had you not been our elder brother and not been addicted to the gambling, would it have been possible I, who killed Jar\^asandha, would have been a cook to Vir\^at R\^aj! Never we had been put to so great a trouble! Never would the mighty armed Arjuna, the V\^asava's son, have acted the part of an actress (a dancer), dressing himself in a female garb, under the name Vrihannal\^a. Alas! What more painful could there be by assuming a human birth that the hands of Arjuna, that wielded always the G\^and\^iva bow, would have worn bracelets befitting a woman? I would have been happy then

had I, seeing the braid of hair on Arjuna's head and the collyrium in his eyes, cut off the head of Dhritar\^astra!

O Lord of the earth! Without asking you, I set fire on the house, named Jatugriha (a lac-house, as built by Duryodhana in order to burn up the P\^andavas) and therefore the vicious Virochana, who wanted to burn us, was himself burnt up. Again, O Lord of men! similarly, without asking you, I slew Kichaka; this is now the one thing I regret that I could not have killed in the same way the sons of Dhritar\^astra before the public hall. O king of kings! It was simply your stupidity that you liberated Duryodhana and other sons, the great enemies of the Gandarbhas, when they had been imprisoned by them. Again to-day you are willing to give wealth for the spiritual benefit of those Duryodhana and others! But, O Lord of the earth, I would never give wealth, even if you request me specially to do this.

Thus saying, Bh\^ima went away. Dharma's son Yudhisthira then consulted with the other three brothers and gave abundance of wealth to Dhritar\^astra. With this sum, the Ambik\^a's son Dhritar\^astra duly performed the Sr\^adh ceremony of his sons and gave away lots of things to the Br\^ahmanas. The king Dhritar\^astra, thus performing all the funeral obsequies, became ready to go early to the forest with G\^andh\^ar\^i, Kunti and Vidura. By the help of Sanjaya, the highly intelligent Dhritar\^astra became informed of the roads of the forest, and then went out of the house. S\^urasena's daughter Kunti, though stopped by her sons, followed them. Bh\^ima and other Kauravas went along with them weeping up to the banks of the Ganges and thence returned to Hastin\^apura.

The ascetics went to the auspicious \'Satay\^upa hermitage on the banks of the Ganges and building a hut practised tapasy\^a with their hearts concentrated. Thus six years elapsed when Yudhisthira, troubled by their bereavements, said to his younger brothers :-- `` I dreamt that our mother Kunti got very lean and thin. Now my mind wants bitterly to see mother, uncle, aunt, the high souled Vidura and the highly intelligent Sanjaya. If you approve, I want to go to there.'' Then the five brothers, P\^andu's sons, became desirous to see Kunti, and taking with them Draupad\^i, Subhadr\^a, Uttar\^a, and other persons went to the Satay\^upa hermitage and saw the persons there; but not seeing Vidura, Yudhisthira asked :-- ``Where is Vidura?'' Hearing this Dhritar\^astra said :-- ``Vidura has taken up Vair\^agyam (dispassion) and has gone alone to a solitary place and is meditating in his heart the eternal Brahm\^a.'' Next day while the king Yudhisthira was walking along the banks of the Ganges, he saw in the forest Vidura, engaged in his vow and become lean and thin by his tapasy\^a; he then exclaimed :-- ``I am the king Yudhisthira;

I am saluting you.'' The holy Vidura heard and remained motionless like a log of wood. Within an instant a wonderful halo came out of Vidura's face and entered the mouth of Yudhisthira, both of them being Dharma's parts. Vidura then died; Yudhisthira expressed great sorrow. When the Vidura's body was going to be set on fire, a celestial voice was heard :--``O king! He was very wise; so he ought not to be burnt; you can go away as you like.'' Hearing this, Yudhisthira bathed in the pure Ganges and returned to the A\'Srama and informed everything in detail to Dhritar\^astra. While the P\^andavas were staying in the hermitage with the other inhabitants of the city, Vedavy\^asa, N\^arada, and other high-souled Munis came there to Yudhisthira. Kunti then spoke to the auspicious Vy\^asa :-- ``O Krisna! I saw my son Karna, only just when he was born; my mind is being very much tormented for him; so, O great ascetic! Show him once to me. O highly fortunate One! You alone can do this; so O Lord! Satisfy my heart's desire.'' G\^andh\^ar\^i said :-- ``O Muni! I did not see while Duryodhan went to battle; so, O Muni! Show me Duryodhana with his younger brothers.'' Subhadr\^a said :-- ``O Omniscient one! I want very much to see the great hero Abhimanyu, dearer to me than my life even; O great ascetic! Show him once to me.'' (33-57.)

S\^uta said :-- Satyavat\^i's son Vy\^asa Deva, hearing their words, held Pr\^an\^ayama (deep breathing exercise) and meditated on the eternal Dev\^i, the force of Brahm\^a. When the evening time came, the Muni invited Yudhisthira and all others to the banks of the Ganges. He then bathed in the Ganges and began to chant hymns in praise of the Dev\^i Brahm\^amay\^i Prakriti, resting on the Purusa, the Dweller in the Mani Dv\^ipa, with attributes, at the same time transcending them, thus :-- ``O Dev\^i! When Brahm\^a was not, Visnu was not, Mahe\'Svara was not, nor when existing lndra, Varuna, Kuvera, Yama, and Agn\^i, Thou alone existed then; my salutation to Thee.

When there existed not water, V\^ayu, ether, earth and their Gunas, taste, smell, etc., when there were no senses, mind, Buddhi, Ahamk\^ara; when there existed no Sun, Moon nor anything, Thou alone existed then; so, O Dev\^i! I bow down again and again to Thee. O Mother! Thou holdest all these visible J\^iva lokas in the cosmic Hiranyagarbha; again Thou bringest this Hiranyagarbha, the sum-total of Linga Sar\^iras (the subtle bodies), with the Gunas Sattva, Rajas and Tamas to a state of equilibrium named S\^amy\^avasth\^a and remainest quite independent and apart for a Kalpa period. At that time even those that are possessed of the power of great discrimination and dispassion cannot fathom Thy nature. O Mother! These persons are praying to me to see their dead

ones; but I am quite incapable to do that. So kindly shew them their departed ones early.'' While Vy\^asa praised thus the Dev\^i, the Dev\^i Mah\^am\^ay\^a, the Lady of the Universe, of the nature of Universal Consciousness called all the departed ones from the Heavens and showed them to their relatives. Then Kunti, G\^andh\^ar\^i, Subhadr\^a, Uttar\^a, and the Pandavas became very glad to see their relatives come to them again. Vy\^asa, of indomitable valour, again remembering Mah\^am\^ay\^a, bade good bye to the departed ones; it seemed then, a great magic had occurred. The Pandavas and the Munis bade good bye to each other and went to their respective places. The king Yudhisthira talked on the way about Vy\^asa and ultimately came to Hastin\^a. (58-68.)

Thus ends the seventh chapter of the Second Skandha on shewing the departed ones in the Mah\^apur\^anam \'Sr\^i Mad Dev\^i Bh\^agavatam of 18,000 verses.