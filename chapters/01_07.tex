\chapter{On the praise of the Dev\^i}

1-26. S\^uta said :-- Seeing the two D\^anavas very powerful, Brahm\^a, the knower of all the \'S\^astras, thought of the means S\^ama, D\^ana, Bheda, Danda (conciliation, gifts, bribe, or sowing dissensions and war or punishment); which of these four he should apply. He thought thus :-- ``I do not know their strength and it is not advisable to enter into war without knowing their strength. Again if I offer praises to them puffed up with pride, it will be simply displaying my own weakness; and when they will come to know this, only one of them will be sufficient to kill me and this they will do certainly. To offer bribes is not also advisable; and how can I sow dissensions. Therefore it is best that I should rouse the four armed Jan\^ardana Visnu, who is very powerful, from his sleep on the thousand headed Ananta serpent. He will remove my difficulties''.

Thus thinking in his mind, Bhagav\^an Brahm\^a, the lotus-born remained in the tubular stalk of the lotus from Visnu's navel and thence took refuge mentally of Visnu, the remover of difficulties and began to chant auspicious hymns composed of various metres to Jagann\^ath N\^ar\^ayana, involved in deep Yoganidr\^a (meditative sleep). He said :-- ``O Refuge of the poor! O Hari! O Visnu! O V\^amana! O M\^adhava, Thou art the Lord of the universe and omnipresent. O Hrisike\'sa! Thou removest all the difficulties of Thy devotees; therefore leave your Yoganidr\^a and get up. O V\^asudeva! O Lord of the Universe! Thou residest within the hearts of all and knowest their desires.

O Thou, holder of the disc and club! Thou always destroyest the enemies of Thy devotees; O Omniscient One! Thou art the Lord of all the Lokas and all-powerful; no one can know what is Thy form; O Lord of the Devas! Thou art the destroyer of all pains and sufferings! So get up and protect me. O Protector of the Universe! nothing is concealed from Thy eyes! Every one becomes pure by hearing and chanting Thy name. Thou art Nir\^ak\^ara (without any form); yet Thou createst, preservest and destroyest the Universe. O Cause of the world! O Supporter of all! Thou art shining as king of kings over all and yet Thou dost not understand that the two D\^anavas, puffed up with pride have become ready to kill me. If thou dost neglect me, seeing me very much distressed and under your protection then Thy name as Preserver will become quite useless. Thus praised, when Visnu did not get up, Brahm\^a thought that "Bhagav\^an Visnu is now surely under the influence of sleep of the Primal Force \^ady\^a \'Sakti and is not therefore getting up; what am I to do now, thus distressed! These two D\^anavas, elated with pride are ready to kill me; now what am I to do and where shall I go? I don't find any body who can protect me any where.'' Thus thinking, Brahm\^a came to the conclusion and decided to chant hymns to Yoga Nidr\^a Herself with one pointed heart. Discussing in his mind, He thought that that \^ady\^a \'Sakti (the Prime Force) which kept Bhagav\^an Visnu, senseless and motionless would alone be able to save him. As a dead man cannot hear any sound so Hari, merged in deep sleep, knows not anything. When I have praised Him so much and when He has not awakened, then it is certain that sleep is not under Hari, but Hari is under sleep, and he who is under another becomes his slave; so this Yoga Nidr\^a is now exercising Her control over Hari. Again she, too, who brought Hari under control, that daughter of the Krs\^ira (milk) ocean is now under the control of Yoga Nidr\^a; so it seems that that Bhagavat\^i Mah\^am\^ay\^a has brought the whole Universe under Her control.

Whether it be Myself, or Visnu or \'Sambhu, or S\^avitr\^i or Ram\^a or Um\^a, all are under Her control; there is nothing to be doubted here? What to speak of other high souled persons! Now I will chant hymns to Yoga Nidr\^a, under whose influence Bhagav\^an Hari even is lying, under deep sleep, inert like an ordinary man. When the eternal V\^asudeva Jan\^ardana will be dispossessed by Her, He will no doubt fight with the D\^anavas. Thus deciding, Bhagav\^an Brahm\^a, seated on the tubular stalk of the lotus, began to chant hymns to Yoga Nidr\^a, residing on the body of Visnu, thus :--

27-32. Brahm\^a said :-- ``O Dev\^i! I have come to understand on the authority of all the words of the Vedas, that Thou art the only One Cause of this Universal Brahm\^anda. The more so when Thou hast brought the best Purusa Visnu, endowed with discrimination above all beings, under the control of sleep, then the above remark is self-evident.

O Thou, the Player in the minds of all beings! O Mother! I am extremely ignorant of the knowledge of Thy nature; when Bhagav\^an Hari is sleeping inert by Thy power, then who is there amongst Kotis and Kotis of wise men, who can understand completely the Pastime, Leel\^a, full of M\^ay\^a of Thine, who art beyond the Gunas. The S\^ankhya philosophers say (that the Purusa (the male aspect of \'Sakti)

is the pure, conscious being and that Thou art the Prakriti, without any consciousness, material inert, Creatrix of the universe; but, O Mother! art Thou really inert like that? Never like that; had it been like that, how is it that Thou hast made Bhagav\^an Hari, the receptable of the world quite unconscious like this? O Bhav\^ani! Thou, being beyond the Gunas art displaying like a dramatic performance these various dramatic plays by the conjunction with the three Gunas. It is Thy three qualities, Sattva, Rajas and Tamas that the Munis meditate every day in the morning mid-day, and evening, the three Sandhy\^as; but no one is aware of Thy ways of doings. O Dev\^i! Thou art of the nature of the judgment and understanding giving rise to knowledge of all the beings in the Universe; Thou art always the \'Sri (wealth and prosperity) giving pleasures to the Devas. O Mother! Thou art reigning in all as K\^irti (fame), mati (intellect), Dhriti (fortitude). K\^anti (beauty) \'Sraddh\^a (faith) and Rati (enjoyment). O Mother! Now I am put to great difficulties and therefore I have got eye witness of Thy nature; no need of further reasoning and discussing about it.

27-50. I have now known that verily, verily Thou art the only Mother of all the worlds as Thou hast brought Hari under the influence of sleep. O Dev\^i! Now when it is evident that all the worlds, etc., have come from Thee, then the Vedas have also come from Thee; what doubt is there? So the Vedas, too, do not know fully Thy nature; for the effect can never know its cause. So, it is very true that Thou art incomprehensible of the Vedas, O Mother! When I, Hari, Hara and the other Devas and my son N\^arada and other Munis have not able to realise Thy nature fully, then who else can be so intelligent in this world that will realise all Thy nature? So Thy glory is beyond the speech of all beings. O Dev\^i! If, in the place of sacrifice, the ritualists, the knowers of the Vedas, do not utter Thy name Sv\^ah\^a, then the Devas, participators of the offerings in Yajña, do not get their share, however hundreds of oblations be offered; so Thou art also the giver of sustenance allowances to the Devas. O Bhagavat\^i! In previous Kalpas, Thou hadst saved me terrified from the fear of the D\^anavas. O Dev\^i Varade! now, too, I am terrified at the sight of the terrible forms of Madhu and Kaitabha and take Thy refuge. O high-minded one! Now I thoroughly see that by Thee, by Thy power Yoganidr\^a the whole body of Bhagav\^an Visnu is senseless; but how is it that Thou dost not realise my sufferings. So, either dost Thou leave possession of this \^adi-Deva, or destroy Thyself these two Danavendras -- lost do either of the two as Thou likest. O Dev\^i! Those that do not know Thy extraordinary powers, those stupid ones meditate Hari, Hara, etc. But, O Mother! By Thy grace, I realise to-day, as eye-witness, that Visnu even is to-day lying unconscious in deep sleep, totally senseless of anything outside by Thy force. O Bhagavat\^i! Now, when Kamal\^a, the daughter of Sindhu is unable to rouse Her husband Hari, by her effort, or rather Thou hast made Her, too, perforce, sleep unconsciously, it seems she is without any effort and does not know anything of what is going on outside. O Dev\^i! Verily those are blessed who worship Thy lotus feet with their whole heart full of devotion and without any hope of getting rewards, abandoning the worship of other Devas and knowing Thee as the Creatrix of the whole world and the giver of all desires. Alas! now the intelligence, beauty, fame, and all good qualities have forsaken Hari and fled away to some unknown quarters. O Bhagavat\^i! Thou art really adorable in the three worlds for, by Thy power of Yoganidr\^a, Hari has been kept in prison, as it were, in this way. O Mother! Thou art the \'Sakti of all this universe and endowed with all prowess and energy; all other things are Thy creation. As a dramatic player, though one, plays in the theatre, assuming many forms, so Thou, too, being one, playest always in this charming theatre of world, created by Thy Gunas, in various forms.

O Mother! Thou, in the beginning of the Yugas, dost manifest first the Visnu form and givest him the pure Sattrik Sakti, free from any obscuration and thereby madest Him preserve the Universe; and now it is Thyself that hast kept Him thus unconscious; therefore, it is an undoubted fact that Thou art doing whatever Thou willest, O Bhagavat\^i! I am now in danger; if it be Thy desire not to kill me, then dost break the silence, look on me and show Thy mercy. O Bhav\^ani! If it be not Thy desire to kill me, then why hast Thou created these two D\^anavas, my death incarnate; or is it that Thou wantedest to put me to ridicule. I have come to know of Thy wonderful acts; Thou createst this whole Universe, and Thyself remaining aloof, playest always and in the time of Pralaya resolvest everything again into Thee. Therefore, O Bhav\^ani, what wonder is there, that Thou wouldst want to kill me in this way? But, O Mother! I won't feel any pain if Thou willingly killest me but this is to my great dishonour that being given power over these beings, I would then be made an object to be killed by the Daityas; this, indeed, is hard to me. So, O Thou Leel\^amay\^i like a sportive girl! get up! O Dev\^i! assumest the wonderful form Thyself and killest me or the two Daityas, as Thou willest; or rouse Hari who will then kill the Daityas. All these are in Thy hands.''

S\^uta said :-- Thus praised by Brahm\^a, the Nidr\^a Dev\^i (the goddess of sleep), of the nature of Tamo Gunas, quitted the body of Bhagav\^an Hari and stood by him. When thus left completely by the Dev\^i Yoga Nidr\^a, of unequalled brilliance and splendour, for the destruction of Madhu Kaitabha, Visnu began to move his body and at this Brahm\^a became very glad.

Thus ends the seventh chapter of the First Skandha on the praise of the Dev\^i in the Mah\^a Pur\^anam \'Srimad Dev\^i Bh\^agavatam of 18,000 verses, by Maharsi Vedavy\^as.