\chapter{On the incidents connected with Navar\^atri}

1. Janamejaya said :-- O Muni! How did R\^amchandra celebrate the Dev\^i's Pûj\^a, that leads to happiness? Who was He! And how was stolen away His S\^it\^a? How was He deprived of His kingdom? Please satisfy me by narrating all these incidents to me.

2. Vy\^asa said :-- O king! There lived, in days of yore, in the city of Ayodhy\^a, a prosperous king of the solar dynasty named Da\'saratha. He always worshipped the Devas and Br\^ahmanas.

3-5. He had four celebrated sons R\^ama, Laksmana, Bharata and Satrughna. These four sons were equally learned and beautiful and they always did actions agreeable to the king. Of these, R\^amachandra was the son of the Queen Kau\'salya, Bharata was the son of Kaikey\^i, and the good looking Laksmana and Satrughna were the twin sons of Sumitr\^a. While young, they learned the art of archery and began to play with bows and arrows in their hands.

6-7. Thus educated and purified, the four sons began to give delight more and more to the king; one day the Maharsi Vi\'sv\^amitra came to Ayodhy\^a and aked from the king Da\'saratha the help of his son R\^amachandra for the protection of his sacrificial ceremonies. The king could not cancel the Vi\'sv\^amitra's request and sent with him R\^ama, accompanied by Laksmana.

8-11. The lovely R\^ama and Laksmana accompanied the Muni on his way back. There lived a terrible looking R\^akhsas\^i, named T\^adak\^a, in a forest on their way, who used to give great troubles to the ascetics; and R\^ama killed her with only one arrow. Next he killed Sub\^ahu and shot arrows at another night-wanderer M\^ar\^icha and made him senseless, almost dead and threw him at a great distance and thus saved Vi\'sv\^amitra from all the obstacles troubling him in his sacrificial ceremonies. Thus fulfilling the great work, protecting the sacrificial ceremonies, R\^ama, Laksmana and the Muni Cowsick, the three, started for the kingdom of Mithil\^a. On his way, R\^ama Chandra rescued Ahaly\^a from the curse that she was suffering from.

12-13. At last the two brothers, accompanied by the Muni, reached the city Videhanagar. Just at this time the king Janaka of Ayodhy\^a made a vow to give in marriage S\^it\^a to anybody who will be able to break the bow of \'Siva; R\^ama broke that bow into two and married S\^it\^a, born of Laksm\^i's parts. The king Janaka gave in, marriage, to Laksmana his own-daughter Urmil\^a.

14. The good and auspicious Bharata and Satrughna married respectively M\^andavi and \'Srutak\^irti, the two daughters of Ku\'sadhvaja.

15. O king! Thus, in the great city of Mithil\^a, the four brothers performed their marriage ceremonies, according to the prescribed rules and rites.

16. The king Da\'saratha, then seeing R\^ama well qualified to take charge of the kingdom, proposed to install him on the throne of Ayodhy\^a.

17. The queen Kaikey\^i, seeing that various articles were being collected for the installation of R\^ama, asked for the two boons, promised before, from her husband Da\'saratha, who was completely under her control.

18. The first request was her own son, Bharata's becoming the king of Ayodhy\^a; and the second request was the banishing of R\^ama to the forest for fourteen years.

19. Thus R\^amachandra went accompanied by S\^it\^a and Laksmana to the Dandak\^a forest, frequented by the R\^aksasas.

20. The high souled king Da\'saratha felt very much due to bereavement of his son, remembered the curse given to him by Andhaka Muni and left his mortal coil.

21. Bharata, seeing that his father died solely on, account of his mother, refrained from becoming the king of Ayodhy\^a, the prosperous city and wanted the welfare of his brother R\^ama.

22. R\^amachandra went to the forest Pa\~nchavat\^i. One day the youngest sister of R\^avana, named Sûrpanakh\^a became very passionate and came to R\^ama; whereon R\^amachandra disfigured her by cutting off her nose and ears.

23. Seeing her nose thus cut away, the R\^aksasas Khara, Dûsana, and others fought very hard against the powerful R\^amachandra.

24. The truly powerful R\^ama killed Khara, Dûsana and all other powerful R\^aksasas, for the welfare of the Munis.

25. Then Sûrpanakh\^a went to Lank\^a and informed R\^avana of her nose having been cut and of the death of Khara, Dûsana and others.

26. The wicked and malignant R\^avana, hearing of their death, became filled with anger and, mounting on a chariot, quickly went to the forest of M\^ar\^icha.

27. R\^avana expressed his desire to take away S\^it\^a; so ordered that magician M\^ar\^icha to assume the form of a golden deer and go to R\^ama and entice him away.

28. The magician M\^ar\^icha assumed the form of a golden deer and reached the sight of J\^anak\^i. Then that variously spotted deer began to move about near the S\^it\^a Dev\^i.

29. Looking at the beautiful golden splendour of the body of that golden deer, S\^it\^a Dev\^i, prompted as it were by the great Fate, spoke to R\^amachandra like other independent women ``O Lord! Bring me the skin of the deer.''

30. R\^ama too, not judging at all, as if it was the work of Destiny, asked Laksamana to remain there and protect S\^it\^a, took hold of his bows and arrows and went after the deer.

31. Infinitely skilled in magic, the deer seeing Hari in the shape of R\^ama sometimes came and sometimes came not within his sight and travelled from one forest to another.

32. When R\^ama saw that He had come very far away from His place, He became angry and drew his bow and shot sharp arrows at that deer, the transformed M\^ar\^icha.

33. The deceitful conjuror R\^aksasa, being thus shot very violently and pained intensely, cried out ``O brother Laksmana! I am killed'' and breathed his last.

34-35. This loud awful cry reached J\^anak\^i's ears. She took that voice for R\^ama's voice and told to Laksmana in a grieved tone ``Laksmana, go quickly. I fear R\^ama is killed; hear the voice ‘O Laksmana! come quickly and deliver me' is calling you to go there.''

36. Laksmana then replied ``Mother! You are alone in this forest; therefore I cannot leave you thus even if R\^amchandra be killed.

37. O daughter of Janaka! R\^ama has ordered me to remain here. Now if I leave you and go elsewhere, then I will be charged with having disobeyed his order. Fearing that, I am unable to leave this place.

38. It seems to me, moreover, that some magician has carried R\^ama away from here; I am therefore unable to move a step from here and leave you alone.

39. Hold patience; let me consider; I find no such man as can kill R\^ama; I am unable to leave you by any means alone here and to go away, disobeying R\^ama's orders.''

40. Vy\^asa said :-- O king! Then the young wife of R\^ama, having handsome teeth, began to cry aloud, fearfully, as if made to do so by Destiny, and uttered the cruel words to the pure Laksmana.

41. ``O son of Sumitr\^a! I know why you are so much attached towards me? I know very well that you have been sent here by Bharata to accompany us simply to obtain me.

42. O vile Ksattriya, skilled in magic! I am not that sort of woman acting to my wanton will; never I will accept you of my will as my husband in case \'Sr\^i R\^amchandra be dead.

43. In case \'Sr\^i R\^ama does not return, I will certainly commit suicide; without him I would be very much grieved and afflicted with sorrows; and I would not be able to hold on my life.

44. O Saumitr\^i! Whether you remain here or do not remain, I won't request anything more to you; for I am quite unaware of your mind; but this much I like to say to you, where has your intimacy towards your religious elder brother now gone?''

45-46. Hearing thus the S\^it\^a Dev\^i's words, Laksmana became exceedingly sorry; and, being suffocated with heaving sighs on account of the internal pain told S\^it\^a ``O! One born from without any womb! Why are you uttering so cruel and malignant words; I clearly see when you are speaking such unworthy words, that some great evil is sure to befall on you very soon.''

47. O king! Thus saying, the spirited Laksmana left S\^it\^a and went out weeping very much, and, being very much afflicted with grief, traced the footsteps of his elder and went on in search of him.

48. When Laksmana thus departed, R\^avana entered into the hermitage in the guise of a deceitful beggar (Bhiksu wearing a red garb).

49. J\^anak\^i took that villain R\^avana to be a Yogi and respectfully gave him offerings of worship and forest fruits.

50-52. That villain asked S\^it\^a humbly, in a gentle tone, ``O beautiful! Your eyes are beautiful like Pal\^asa lotus leaves; therefore it seems that you are not an ordinary woman; how is it that you are here thus alone in a wild forest? O fair one! Who is your father? who is your brother and who is your husband? Being such a beautiful one, how is it that you are in this forest here like an ordinary woman, dumbfounded? O good looking one! You are worthy to live in a palace filled with nectar; why are you living, in this hovel, in this wild forest like an ordinary Muni's wife, when your beauty is shining in lustrous beams like a Deva girl?''

53-55. Vy\^asa said :-- The daughter of J\^anak\^i, hearing the words of R\^avana, the husband of Mandodar\^i, unfortunately took him to be a good Yogi and replied in the following way :-- ``Perhaps you have heard that a prosperous king Da\'saratha is reigning in the Ayodhy\^a city. He has four sons; the eldest of these, \'Sr\^i R\^am Chandra, is my husband. The king offered two boons to Kaikey\^i; due to which R\^am Chandra has been exiled in this forest and is with his brother Laksmana.

56. I am the daughter of the King Janaka; my name is S\^it\^a; R\^am Chandra has broken the bow of \'Siva and has married me.

57. Resting under his prowess of arms, I am resting here fearlessly in this wild forest; seeing a golden deer, he has gone out to kill that for me.

58. Laksmana, too, hearing his voice has gone just now. O Yogi! I am living here depending on the strength of these two brothers.

59. Thus I have told you all about our living in this forest; shortly they will come and worship you duly.

60-61. The man who has controlled his passions and has become a Yati is like Visnu incarnate; therefore I have worshipped you. O Yogi! Our Â\'sram is in the midst of this terrible forest, surrounded by R\^akhsasas. Therefore I am asking you how is it that you have been able to come here in this dress of Tridandi (a Sannyasi Yogi); please speak in the name of Truth before me.''

62. R\^avana said :-- ``O askance looking one! I am the king of Lank\^a, the husband of Mandodar\^i. O beautiful one! it is for you that I have put on this dress of Yati.

63. O beautiful! My two brothers Khara and Dûsana have been killed in this forest; and being urged by my sister I have come here.

64-65. Now leave your this man-husband, residing in the forest as a pauper, devoid of fortune and wealth; and worship me as a husband. O fair one! I am R\^avana, the king of kings; you now become my lord.

66. O daughter of Janaka! I am the lord of the Regents of the quarters; and yet I bow my head down to your lotus feet; better accept me and fulfil my desires today.

67-68. Formerly I asked of you from your father, the king Janaka; but he then said, that he had laid a pledge, ‘Whoever will break the \'Siva's bow will marry my daughter.' The Bhagv\^an Rudra is my Guru; hence I feared to break his bow, and therefore I was not present in your Svayamvara. But from that time my mind is always thinking of you and is in a state of bereavement for you.

69. O beautiful one! Hearing now that you are residing in this forest, I, impelled by my previous fascination for you, have now come hither; and you better now crown my labour with success.''

Thus ends the 28th Chapter on the incidents connected with the Navar\^atri and the description of R\^amayanam in \'Sr\^i Mad Dev\^i Bh\^agavatam of 18000 verses, by Maharsi Veda Vy\^asa in the 3rd Adhy\^aya.

Note: The story about the origin of S\^it\^a Dev\^i runs thus :-- R\^avana, the king of Ceylon (Lank\^a) practised very severe austerities and got extraordinary powers. He brought the three worlds under his subjection, levied taxes from all. The Devas and all the other inhabitants of the several worlds paid their taxes, as imposed by R\^avana. R\^avana sent messengers to the Risis and the Munis, the ascetics, dwelling in forests and asked them to pay their taxes. The Risis replied that they had no property. But R\^avana insisted. The Risis gave, then, blood, cutting their thighs, in a jar that was carried to Lank\^a. R\^avana kept that jar under the custody of his queen Mandodar\^i, and instructed her that the jar contained poison and that she should not eat that. Mandodar\^i, however, ate a portion of that, out of curiosity, and became pregnant and gave birth to a daughter. Fearing R\^avana, she floated the jar with the daughter, in the ocean, which, floating through oceans and rivers, came and touched the lands of the King Janaka. The peasants while tilling, found that and took the girl to the king, who reared her as his daughter. Thus S\^it\^a, born out of the blood of the Br\^ahmanas, took away subsequently the kingdom, life, and all of R\^avana.

Another version is this :-- As before, the messengers advised the Munis to give something; otherwise R\^avana would insist and put them to various troubles. So the Munis cut their thighs and gave blood as their tax, saying that that blood in the jar would cause ruin and desolation to the country where it will be kept. R\^avana, hearing this, ordered the jar to be carried to the kingdom of the king Janaka, thus causing ruin to him. The jar was brought and placed in the fields of Janaka.

Now it happened that there was a very severe drought; rains were absolutely wanting; and a dire famine was imminent. The Br\^ahmin Pundits informed the king that if the king and his wife ploughed themselves the fields, rains would fall. So the king with his wife did that, the king holding the plough and the queen holding the hand of the king. The fore end of the plough accidentally hit upon that jar, out of which came out S\^it\^a Dev\^i with two women Riddhi and Siddhi, waving chowries on her two sides. The two ladies disappeared and S\^it\^a Dev\^i looked like a girl. The king Janaka reared her, as if his daughter. S\^it\^a Dev\^i used to lift daily with her left hand the bow of \'Siva, kept in the king's house, and daily worshipped that, and thus cleansed the place. Seeing this, the king Janaka pledged the vow that, whoever would break the \'Siva's bow, would marry S\^it\^a.