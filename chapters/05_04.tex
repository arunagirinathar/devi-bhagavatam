\chapter{On the war counsels given by Indra}

1-17. Vy\^asa said :-- O King! The messenger of the D\^anavas having departed, Indra, the lord of the Devas, Yama, V\^ayu, Varuna, and Kuvera and other Devas, called an assembly and addressed thus :-- O Devas! the most powerful Mahisa, the son of Rambha, is now the king of the D\^anavas; he is particularly expert in hundreds of M\^ay\^as (magic) and has become haughty on the strength of his boon. O Devas! Mahisa has sent his messenger; he wants to take possession of the heaven; he came down to me and spoke thus :-- ``O Indra! Quit your this heaven and go any where you like, or be ready to pay your homage to the highsouled Mahis\^asura, the Lord of the D\^anavas. The D\^anava Chief never becomes angry with his opponent who becomes submissive like a servant; if you surrender and serve him, he will, out of mercy, grant an allowance to you. O Lord of the Devas! If this does not like you, then collect your forces and be ready for fight; no sooner I return, the Lord of the D\^anavas will come here at once ready to give battle to you.'' Thus saying, the messenger of that wicked D\^anava departed. Now what are we to do? O Devas! Think on that. O Devas! Even a weak enemy is not to be overlooked by a powerful opponent, especially when the enemy is powerful by his own powers and is ever energetic, never is he to be overlooked. It is always incumbent on us to make our efforts, as best as we can, both by our body and mind as far as lies in our power; the result, victory or defeat depends entirely on Fate. It is useless to make treaty with a deceitful and dishonest person; we therefore never should make treaty with this person; you are all honest; that D\^anava is dishonest; therefore ponder and ponder deeply and ponder again; do you that which is proper. It is not advisable to go out at once for fight when we are unaware of our enemy's strength; let us therefore send spies truthful, honest, motiveless, quick, to ascertain their strength, those who can easily enter amongst our enemies and yet who have no relation, nor any interest with them. The arrangements of their forces, their movements, their numbers, they will ascertain correctly who are their generals, what is their number and what is their strength, they will thoroughly examine and return here quickly. First, we will ascertain the strength of the forces of our opposite party and then we will decide at once whether we will start for battle or seek protection within forts. Wise persons always consider before they act; any act done rashly leads in all respects to many troubles, and anything done after mature prejudgments leads to happiness; so the wise do. The D\^anavas are all one in their heart and mind; therefore it is not advisable, in any way to apply the principle of Bheda (sowing principles of discord). Let our spies go there, ascertain their strength, return and inform us; we will then judge what principle is proper and apply to the expert D\^anavas. Any act done contrary to policy and expediency will undoubtedly produce effects contrary in every way just like a medicine which we have not tried already.

18-22. Vy\^asa said :-- O King! Thus counselling with the Devas; Indra sent expert spies to ascertain the true state of affairs. The spies, too, went into the abode of the Daityas, with no delay and made their searches thoroughly into every nook and corner and returned and told Indra all the strength of the D\^anava forces. Indra was very much startled to know, then, of their arrangements. He immediately bade all the Devas be ready for battle and called for his High priest Brihaspati, expert in giving advices and began to consult with him how to carry on the warfare with that indomitable enemy, the Lord of the Asuras. On Bhihaspati, the best and famous of the Angir\^a family, taking his excellent seat.

23-25. Indra thus said :-- ``O Guru of the Devas! O Learned! Please say what are we to do now in this critical juncture? You are omniscient; to-day you are our guide. The Demon Mah\^isa has become very powerful, very haughty; surrounded by D\^anavas he is now coming to fight with us. You are expert in mantras; find out the remedy for us. \'Sukr\^ach\^arya is the remover of all obstacles on their side; and that you are our safe guard is well known to us.''

26. Vy\^asa said :-- Hearing these words of Indra, Brihaspati, who is always ready to effect the Deva's purposes, thought intently on the subject, said very shortly thus :--

27-51. Brihaspati spoke :-- O Lord of the Devas! O Venerable One! Be peaceful; have patience; when a difficulty comes, one should not, all on a sudden, lose one's patience. O Chief of the Immortals! Victory or defeat is completely under the control of destiny: therefore intelligent ones should always be patient. O \'Satakratu! What will unavoidably be done must come to pass; knowing this as certain, one would always be an enthusiast and exert one's powers. Everything is guided by Fate. Knowing this, the Munis devote themselves at all times solely filled with energy in their meditation and Yoga practices for their final liberation. Therefore, to show one's energy, according to the rules of the daily practices, ought to be indispensably done; and one should not repel or feel pleasure on failure or success; for that is under Fate. Success sometimes comes without the exercise of one's own powers, as seen in cases of the lame and the blind; and that is not the reason why one should be very glad. The embodied beings are all under Daiva (Fate); therefore even if success be not attained, though one's own powers are exercised thoroughly, no one is to blame for that. O Lord of the Suras! What to say of forces, Mantras, or advices, what of chariots or weapons, nothing to lead to success; It is Daiva, and only Daiva that makes one successful. This whole universe is under Daiva; it is, therefore, that we see powerful persons suffering pains, and weak ones getting happiness; the intelligent ones sleeping without any food and fools enjoying merrily; distressed persons getting victory and powerful ones suffering defeats; what cares, then ought one to entertain in this. O Lord of the Suras! Whatever is inevitable to come to pass, be it success or failure, one will lead one's energies to that end; therefore one needs to consider beforehand whether one's energies will be successful or not. In times of distress, one sees distress too much and in times of pleasure, one seeks pleasure too much; one's self, therefore one should not surrender to one's enemies, pleasure and pain. Pain and suffering is not felt so much in patience as is felt when impatient; therefore one must practise patience when pain or pleasure comes. Indeed it is very difficult to bear oneself up in distress or happiness; therefore wise persons try not to let these feelings crop up at all from the very beginning. ``I am always full, undiminishable, I am beyond these Pr\^akritic qualities. Who is there to suffer? What is suffering?'' Thus one ought to think at that moment. I am beyond the twenty-four Tattvas; what pleasure or pain can, then, arise to me? Hunger and thirst are the Dharma of Pr\^ana; pain and insensibility is the Dharma of mind, age and death belong to this physical body. I am free from these six diseases; I am \'Siva. Grief and delusion are the qualities of this body what then do I care for them? ``I'' am not the qualities of the body nor ``I'' am the soul pertaining to that. I am beyond the seven transfigurations, changes, e. g., Mahat, etc., I am beyond this Prakriti, Nature, and beyond the sixteen changes wrought out by Prakriti; I am therefore eternally happy, I am beyond Prakriti and its transformation, then why am I to suffer pain always? O Lord of the Suras! Think on these and be without any passion. O \'Satakratu! This attachment is the root of all miseries and non-attachment is the source of all happiness; non-attachment therefore, is the chief means of the extirpation of all your troubles. Lord of \'Sachi! Nothing can be happier than contentment. In case you find it difficult to practise dispassion, apply, then, discrimination and think of Fate, that what comes inevitably to pass. O Lord of the Suras! Actions already done cannot die out without their effects being enjoyed. O Best of the Suras! Let all your intelligence be brought to action, let all the Devas lend their helping hands to you; what is inevitable must come to pass; what then can you care for your happiness or pain? O King! Happiness is felt for the expiation of good deeds and pain is felt for the expiation of bad deeds; therefore wise persons get thoroughly delighted when their punya ends. O King! Judge and hold a council to-day; then try your best. But what is unavoidable will come to pass, even if you try your best.

Here ends the Fourth Chapter of the Fifth Book on the counsels given by Indra in the M\^ah\^a Pur\^anam \'Sr\^i Mad Dev\^i Bh\^agavatam of 18,000 verses by Maharsi Veda Vy\^asa.