\chapter{On the slaying of Vid\^al\^aksa and Asilom\^a}

1-3. Vy\^asa said :-- O King! Hearing the two Demons slain by the Dev\^i, Mahis\^asura became very much amazed and sent the powerful Asilom\^a and Vid\^al\^aksa and the other D\^anavas to the battle to kill the Dev\^i. The D\^anavas, all very skilled in the art of warfare, marched on for battle, fully equipped with weapons and clad in armour, and we attended by a vast army. They arrived there and saw the Divine Mother with eighteen hands taking Her stand on a lion, with axes and shield in Her hands.

4-5. The calm-tempered Asilom\^a appeared before the Dev\^i ready to kill the Daityas, saluted Her and smilingly said :-- O Dev\^i! Why have You come here? and what for You are killing these faultless Daityas? O Beautiful One! Tell all these to me truly. We will make treaty with you.

6-17. Take gold, jewels, pearls and any other excellent things the you like and retire from the field as early as passible. Why do you like this warfare tending to increase misery; the wise persons say that it leads to the destruction of all happiness. Your body is very delicate; it cannot bear the stroke of flowers even; then why are you suffering the stroke of weapons on your bodies; I am very much puzzled to think these things. See! The cleverness is judged when peace is the result thereof; for it leads always to happiness. Then why are you liking to fight which will lead only to pain and suffering. Happiness is only to be had and pain is to be avoided; this is the rule. O Dev\^i! That happiness is again of two kinds :-- Permanent and transitory. The pleasure that comes out of the knowledge of Atmajñ\^an is permanent and that which is derived from enjoyments is transitory; these who know truly the Veda \'S\^astra, they avoid this transitory pleasure of enjoyments. If you follow the opinion of the Mim\^amsakas and do not believe in the existence of future births, even then you ought not to fight; when you have got this youthful age, you ought to enjoy the excellent pleasures in this world. O One of lean stomach! And if you doubt in the existence of the other worlds after death, even then you ought to desert from fighting and perform, in this life, such actions as will lead you to the attainment of Heavens. This fully developed womanhood is transient; knowing this do virtuous actions always; the wise ones always avoid tormenting others; thus one ought to perform things not contradictory to Dharma, Artha and Kama. Therefore, O Auspicious One! Do You also things virtuous always. O Mother! Why are you killing these Daityas without any cause? There is, again, the feeling of mercy; the lives again of all are dependent on Truth. Therefore the sages always preserve piety, mercy and Truth. O Beautiful One! Then what is the use in Your killing these Demons? Please say explicitly on this point.

18-27. The Dev\^i said :-- O Powerful one! Hear why I have come here and why I am killing the Daityas? I answer your question on the above points. O Demon! I, though merely a spectator, always go about all over the worlds, seeing the justices and injustices done by the several souls there. Never I possess any desire of enjoyment, nor have I greed for anything, nor have I enmity with any creature. Only to preserve the virtue and religion and to keep up the righteous, I roam over the worlds. This is My vow and I always adhere to it. To preserve the good and to put down the evil doers is My duty. Many Avataras are to take their incarnations, cycles after cycles, to preserve the Vedas; therefore I incarnate Myself in yugas after yugas. Now the wicked Mahisa is ready to destroy the Devas; seeing this, I have come here to kill him. I tell you verily that I will slay that vicious powerful Mahis\^asura, the enemy of the gods. Knowing this, you remain or depart, as you desire. Or you can go to Mahisa, that impious son of a she-buffalo, and say what is the use in sending other Asuras to the battle; he can come himself and fight. If your king likes to make a treaty, then let him avoid his enmity with the Devas and go down to the P\^at\^ala. Let him return to the Devas whatever he has taken perforce from them and go to the P\^at\^ala, where Prahl\^ada is residing.

28-29. Vy\^asa said :-- O King! Hearing thus the Dev\^i's words, Asilom\^a asked gladly, before the Dev\^i, the powerful Asura Vid\^al\^aksa :-- Well, Vid\^al\^aksa! You have heard just now all what the Dev\^i has said; now are we to observe treaty or declare war. What are we to do under the circumstances?

30-34. Vid\^al\^aksa said :-- Our king knows full well that his death will certainly take place in the battle; knowing this, he is not willing to make peace, out of his egoism and vanity. He is seeing before him daily the deaths of the D\^anavas and still he has sent us to battle. Who can overcome the destiny? The duty of a servant is a very difficult one; he will have to be always submissive and obedient, without caring the least for his own self-respect; just as the dancing dolls are completely under the hands of the actors and their movements vary according to the pulling of the wires employed in making them dance. How can we then go to our master and say such hard words as he would give away to the Devas all the gems and jewels and go down to P\^at\^ala with other D\^anavas. One considers it one's duty to speak pleasant words though untrue; true words cannot be beneficial; true and at the same time beneficial words are very rare in this world; at such critical cases, one ought to remain silent. Especially heroes ought never to excite their kings by useless words; this is the essence of politics. We should never go and advise our king with eagerness what is best or to ask advice from him about such things; the king would then certainly be very angry. Therefore we ought to do our duties to the king, even if our lives be at stake. To consider our lives as nothing and to fight for our king are what is best for us.

35-57. Vy\^asa said :-- O King! Thus thinking, the two heroes then wore their coats of armour, mounted on their chariots and, with bows and arrows in their hands, became ready for fight. First Vid\^al\^aksa shot seven arrows; the great warrior Asilom\^a stood aloof at a distance as a mere witness. The Divine Mother cut off those arrows to pieces with Her arrows, no sooner they reached Her, and then shot at Vid\^al\^aksa three arrows sharpened on stone. The demon Vid\^al\^aksa fell senselss by these arrows on the battle-field and after a short while died, as if ordained by Fate. Seeing Vid\^al\^aksa thus dead, Asilom\^a took up his bows and arrows and came up, for fight. The hero, then, raising his left hand, said briefly, thus :-- ``O Dev\^i! I know that death is inevitable to the D\^anavas; still I am ready to fight; for I am dependent and Mahisa is of very dull intellect; he cannot make any distinction between what is really good and what is merely pleasant. I will never speak before him unpleasant words, though beneficial. Rather I will sacrifice my life in the battle-field than advise him anything, be that auspicious or inauspicious. The D\^anavas are being killed no sooner they are shot at by your arrows; seeing this I consider Fate superior to all. Prowess does not lead to any success; Fie on one's prowess! Thus saying, the Demon began to shower arrows after arrows on the Dev\^i; the Dev\^i, too, cut them to pieces with Her own arrows before they came to Her; and, becoming angry, soon pierced him with arrows. The Devas witnessed this sight from above. The body of the

Demon was then covered with cuts and wounds; blood began to flow from them; the Demon consequently began to shine like the jovial Kimsuka tree. Asilom\^a then lifted aloft his heavy iron club and ran after Chandik\^a and hurt the lion on his head with anger. Not caring at all this severe stroke of the club inflicted by that powerful Demon, the lion tore asunder his arms with his claws. Then that dreadful Demon leapt with club in his hand and got up the shoulder of the lion and hit the Dev\^i very hard. O King! The Dev\^i, then, baffled the hit and cut off the Demon's head with Her sharp axe. The head being thus severed, the Demon was thrown on the ground with great force; seeing this, a general cry of distress arose among his soldiers. The Devas shouted aloud ``Victory to the Dev\^i'' and chanted hymns to Her. The drums of the Devas resounded and the Gandarbhas began to dance in great joy. Seeing the two Demons thus lying dead on the battlefield, the lion killed some of the remaining forces by his sheer strength and ate up others, and made the battlefield void of any persons. Some fled away in great distress to Mahis\^asura. The fugitives began to cry aloud, ``Save us, save u\'s' and said, ``O King! Asilom\^a and Vid\^al\^aksa are both slain; and those soldiers that remained were eaten up by the lion.'' Thus they told and plunged the King in an ocean of dire distress. Hearing their words, Mahisa became absent minded through pain and grief and began to think over the matter with great anxiety.

Here ends the Fifteenth Chapter of the Fifth Book on the slaying of Vid\^al\^aksa and Asilom\^a in \'Sr\^imad Dev\^i Bh\^agavatam, the Mah\^a Pur\^anam, of 18,000 verses by Maharsi Veda Vy\^asa.