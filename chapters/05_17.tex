\chapter{On Mandodar\^i's accounts}

1-2. Vy\^asa said :-- O King! Hearing thus, the Dev\^i asked the D\^anava, ``Who is that Mandodar\^i? Who is that king who was not first taken by her? And who is that king whom she married afterwards? And how did she repent afterwards? Describe all these in detail to me.''

3-26. Thus asked by the Dev\^i, Mahisa began to say :-- ``O Dev\^i! There is a place, named Simhala, noted in this earth and decorated with various trees and prosperous with wealth and grains. A virtuous king, named Chandrasena, used to reign there; he was calm, peaceful, truthful, heroic, charitable, steady, forbearing, well versed in polities, ethics and morals vast as a wide ocean, learned in \'S\^astras, knowing all forms of religions and much skilled in archery. He was mindful in governing his subjects and he used to punish according to the laws of Justice. The king had a beautiful well-qualified wife, very handsome and broad-hipped. She was very much devoted to her husband and always engaged in religious acts and of good conduct. This wife, endowed with all auspicious signs, gave birth to a beautiful daughter in her first delivery. The King Chandrasena, the father, was very pleased to have this beautiful daughter and gladly called her by the name of Mandodar\^i. This daughter began to grow daily like the phases of the Moon. When she grew ten years old, she became very handsome. The King now became anxious to have a suitable bridegroom and used to think of it everyday. The Br\^ahmins then told the king that there was a prince named Kambugr\^iva, the intelligent son of the powerful king Sudhanv\^a of Madra; this prince was endowed, with all kingly qualifications and versed in all knowledge and was therefore a fit match for your daughter. The king then asked his dear qualified wife that he would like to marry his daughter to Kambugr\^iva. The queen, hearing this, asked her daughter Mandodar\^i that her father was desiring to marry her to Kambugr\^iva, the son of the king of Madra. Hearing her mother's words, Mandodar\^i spoke thus :-- ``O Mother! I have got no desire to marry; I will not accept any husband; I will take the vow of leading a chaste virgin life and thus pass the rest of my life. O Mother! There is nothing more miserable in this ocean of world than dependence; I therefore prefer to lead incessantly a life of severe asceticism. The Pundits versed in the \'S\^astras say that taking up the vow of separateness and independence leads to salvation; I will thus be liberated; I have no need for a husband. At the time of marriage ceremony, one has to say before the consecrated Fire that one will remain always a dependent to one's husband in every way; besides in a father-in-law's house, one has to pass one's time as a slave, as it were, to one's mother-in-law and to husband's (younger) brothers; again one will have to think oneself as happy when one's husband is happy and as unhappy when one's husband is unhappy; this is the worst of all miseries. Again if the husband marries again another woman, then this misery of having a co-wife is extreme. O Mother! Jealousy arises then towards even one's own husband and therefore suffering is endless. Therefore what happiness can there be in this dream-like worlds; especially with women who are made dependent by Nature? O Mother! I heard that in days of yore the religious son of Utt\^anap\^ada, Uttama was younger than Dhruva; and yet he became King. And the King Utt\^anap\^ada banished his dear wife, solely devoted to her husband, without any cause, to the forest. Therefore women have to suffer such diverse pains while their husbands are living; and if by chance the husband dies, then women get interminable pains; the widowhood becomes the only source of grief and sorrow. Again if the husband be in foreign lands, women become subjected to the fire of Cupid, and then the house becomes an object of more agony. Thus whether the husband lives or dies, there is no happiness at any time. Thus, according to my opinion, I ought never to accept any husband.''

27-31. The Mother then told her husband all about what the daughter had said. Mandodar\^i would accept the vow of a life-long virgin; she had no desire to marry. She had brought forward many faults in a worldly life and thus would perform vows and Japams and pass her time alone.

She did not yearn after a husband. The King, hearing thus, came to know, that his daughter had no intention to marry and so began to pass his time without giving away his daughter in marriage. Thus the daughter lived in family protected by her father and mother; by that time signs of puberty were seen in the body of the daughter. Her comrades requested her repeatedly to select a bridegroom; but she spoke many words of wisdom and did not show any inclination for marriage.

32-44. Once, on an occasion, that beautiful faced woman went out with her female attendants on a pleasure trip to a garden, beautified with various trees. There the slender bodied one began to play and enjoy with her comrades in picking up various flowers and beautiful flowering creepers. Just at that time, the famous King of Kosala, the powerful V\^irasena came there accidentally. Alone he was on his chariot, attended by a few soldiers; his large army and retinue were coming slowly behind him at some distance. Her comrades, then, looking at that King from a distance, told Mandodar\^i, ``O friend! See! Somebody, strong and beautiful, like a second God of Love is coming towards us, mounted on a chariot. I think some King he will be and we are very lucky that he has come here.'' While thus talking, the King arrived there. The King, looking on that blue coloured woman with beautiful eyes became surprised and getting down from the chariot, asked the maidservant, ``O Gentle one! Who is this woman with large eyes! Who is her father? Tell me this without any delay.'' The attendant smiling, told him thus :-- O Beautiful-eyed One! Pray speak first who are you? What for have you come here? What do you want to do here? The female attendant thus asking him, the King replied :-- There is a very beautiful country named Kosala, in this earth; I am the King of that place; my name is V\^irasena. My fourfold army is coming at my will at my back. I have lost my way and have come here. Know me as the King of the country Kosala.

45-49. The female attendant said :-- ``O King! This lotus-eyed one is the daughter of the King Chandrasena; her name is Mandodar\^i. She has come here in this garden for sporting.'' Hearing thus the attendant's words, the King replied :-- ``O Sairandhri! You appear to be smart; therefore make the King's daughter understand my following words clearly! O Sweet-eyed one! I am the King descended from the Kakutstha line; O fair woman! Marry me according to the rules of Gandarbha marriage.''

Note :-- Gandharva marriage: one of the eight forms of marriage; this form of marriage proceeds entirely from love or the mutual inclination of a youth and maiden without ceremonies and without consulting relatives.

``O broad hipped One! I have no other wife; you are a beautiful woman, of a good family and of a marriageable age; I therefore like to marry you? Or your father may marry you to me according to rules and ceremonies; if so, I will no doubt be your husband as you desire.''

50-55. Mahisa said :-- O Dev\^i! The female attendant, expert in the science of love, hearing the King's words, spoke to the daughter smilingly and in sweet words. ``O Mandodar\^i! A very good-looking beautiful King of the solar dynasty has come here; he is very pretty, powerful, and of your age; O Beautiful! The King is entirely devoted to you and loves you very much. O Large-eyed One! Your time of marriage has come and yet you have not married; rather you are against it. Your father is, therefore, always very sorry and remorseful. See! How many a time your father sighed and told us, ‘O attendants! Always serve my daughter and awaken her to this.' But you are engaged in penances and austerities, in Hatha Dharma; therefore we cannot request you on this matter. The Munis have said :-- To serve the husband is the highest virtue of a woman. O Large-eyed! Women get Heaven if they serve their husband; therefore you better marry according to rules and ceremonies.''

56. Mandodar\^i said :-- I am not going to marry; better that I should perform an extraordinary tapasy\^a (asceticism); O Girls! You go and ask the King desist in his request; why is he shamelessly looking at me.

57-59. The female attendant then said, ``O Dev\^i! Passion is very hard to conquer; time is also surmountable with difficulty; so know my advice as the medicinal diet and keep my request. And if you do not keep it, surely danger will befall you.'' Hearing this, Mandodar\^i replied, ``O attendant! I know whatever is ordained by Fate will inevitably come to pass; for the present, I am not going to marry at all.''

60-61. Mahisa said :-- The female attendant, knowing this her obstinate view, told the King :-- ``O King! This woman likes not a good husband; you would better go wherever you like.'' The King heard and did not want to marry that woman any more; and, being sad and broken-hearted, went back with his army to Kosala.

Here ends the Seventeenth Chapter of the Fifth Skandha on Mandodar\^i's accounts in the Mah\^apur\^anam, \'Sr\^i Mad Dev\^i Bh\^agavatam, of 18,000 verses by Maharsi Veda Vy\^asa.