\chapter{On the creation and the Tattvas and their presiding deities}

1.O Brahm\^a said :-- N\^arada! Thus we three I, Visnu, and Mah\^adeva saw that highly effulgent Goddess: we also saw separately Her attendant goddesses, one after another, that form, as it were, a veil to her? Who were also preeminently grand.

2-3. Vy\^asa said :-- O king! N\^arada, the foremost of the Munis, hearing thus his father's words, was exceedingly pleased and asked :-- O Grandsire of all the Lokas! Now describe in detail that ancient and indestructible undecaying, unchangeable, eternal Purusa, that is Nirguna (free from Pr\^akritic qualities) that you have seen and realised.

4. Father! You have seen the \'Sakti (the Prime Energy) personified the Saguna energy, the Supreme Goddess, having hands and feet; but cannot understand of what kind is that Nirguna \'Sakti which cannot be seen and which is devoid of all Pr\^akritic qualities.

O Lotus-born! Be good enough to describe to me the real nature of that Prakriti and Purusa and thus satisfy me.

5-6. O Lord of Creation! I practised severe austerities in the Svetadv\^ipa (white island), so that I might realise and see the Nirguna Highest Self and the Nirguna \'Sakti, the Supreme Goddess; I saw there many other Mahatm\^as (high class spiritual persons) who attained siddhis (supernatural powers) practise Tapasy\^a with their passions and anger conquered. But I did not realise nor did I see anything about that Nirguna Highest Self. Father, I was not despaired; again and again I continued with my ascetic practices; but still I failed.

7. Father, you have been so successful as to see that beautiful \'Sakti with qualities; I have heard about Her from you, but how and of what sort, is that invisible attributeless energy as well as that Nirguna Purusa. Please narrate and explain all these and satisfy my desires that always reign in my breast.

8. Vy\^asa said :-- O King! Thus asked by N\^ar\^ada, the Lord of creation, the grandsire of the Lokas, smiled, and began to speak the truth in the following words :--

9. O best of Munis! The form of the Nirguna Purusa (the Supreme Spirit beyond the Pr\^akritic qualities) cannot exist or be visible; for everything that comes within the range of sight is transitory. How can, then, that Eternal Spirit have form and how can He become visible!

10. O N\^arada! The Nirguna Energy or Nirguna Purusa comes not easily within the range of knowledge; but both of them can be realised by the Munis in their meditation in their consciousness.

11. Prakriti and Purusa have no beginning nor end; they can be realised only through faith; those that have no faith can never realise them.

12. N\^arada! The universal consciousness, that is felt in all the beings, know that as the Highest Self; the Energy that is universal and is seen always in all the beings, know that as the Highest Self.

13. O blessed one! That Purusa and Prakriti pervade everywhere and exist in all the things; in this Universe nothing can exist without the presence of both of them.

14. Both of them are the highest intelligent self, nirguna (free from all material qualities), without any tinge of impurity, and undecaying. The one form that is a combination of these two is always to be meditated in the heart.

15. What is \'Sakti (energy) is the Highest Self; what is the Highest Self is the Highest \'Sakti. O N\^arada! Nobody can ascertain the subtle difference between these two.

16. O N\^arada! Merely the study of all the \'S\^astras and the Vedas with their Amgas without renunciation does not enable one to ascertain the difference between these two.

17. O Child! This whole universe, moving and non-moving, comes out of Ahamk\^ara (egoism). How can one ascertain the above difference even if he tries for one hundred kalpas, unless one frees oneself from Ahamk\^ara.

18. The J\^ivas are Saguna (with qualities), how can the Sagunas see the Nirguna One with their physical eyes? Therefore O Intelligent one! try to see the Saguna (Brahm\^a) only within your heart (until you free yourself from the material qualities and thus be fit to realise the Nirguna Brahm\^a).

19-20. O best of Munis! If the tongue (organ of taste) and eyes (organ of sight) be affected with over biliousness, the pungent taste and the yellow colour do not appear what it appeared before; so the hearts of J\^ivas, overpowered with material qualities, are quite unfit for realisation of the Nirguna Brahm\^an. O N\^arada! That heart again has come

out of Ahamk\^ara; how can then that heart be free from Ahamk\^ara?

21. Until one becomes able to cut asunder all connections with qualities, the seeing of that Nirguna Brahm\^a is impossible. No sooner one is totally free from Ahamk\^ara, than the Nirguna Brahm\^a is at once seen by him within his heart.

22-24. N\^arada said :-- O best of the Devas! Ahamk\^ara is three-fold, S\^atvik, R\^ajasik and T\^amasik; describe in detail the differences between these three sub-divisions as well the real nature of the Gunas. Also describe to me about that knowledge, knowing which will lead to my salvation. Also describe, in detail, the characteristics of the several Gunas, in due order.

25-26. Brahm\^a said :-- O Sinless one! The energy of Ahamk\^ara is of three kinds :-- Jn\^ana \'Sakti, Kriy\^a \'Sakti, and Artha or Dravya \'Sakti. The power by which knowledge is produced or obtained is the S\^attvic Ahamk\^ara; the power by which action or activity or motion is produced is the R\^ajasic Ahamk\^ara; and that by which the material things or objects of have senses are generated is called the T\^amasic Ahamk\^ara. O N\^arada! thus I described to you, in due order, the threefold Ahamk\^ara.

27-30. Now I describe to you their merits and workings in detail; hear. Out of the Dravya \'Sakti of the T\^amasic Ahamk\^ara come sound, touch, form, taste and smell. From these five qualities, the five Tanm\^atr\^as or the five subtle-elements (primary atoms) are produced.

Sound is the quality of Âk\^a'sa (ether); touch is the quality of V\^ayu (Air); the form is the quality of Agni (fire); the taste is the quality of Jala (water); and the smell is the quality of earth.

O N\^arada, these ten gross and subtle materials can, when combined, become endowed with power to work out results in the shape of earth, water, fire, etc., and when the Panch\^ikarana process is combined, the building of the whole cosmos takes place as a natural consequence of the T\^amasa Ahamk\^ara, endowed with the energy of generating material substances.

31-34. Now hear what are produced by the R\^ajasic energy. The five organs of hearing, touch, taste, sight, and smell (ears, skin, tongue, eyes and nose) called the five J\~n\^anendriyas (organs of senses); mouth, hands, feet, anus and the organs of generation called the five Karmendriyas (organs of action); and Pr\^ana, Ap\^ana, Vy\^ana, Sam\^ana, and Ud\^ana, the five V\^ayus. The creation out of these fifteen substances is called the R\^ajasic energy. N\^arada! All these organs of senses and actions endowed with the Kriy\^a \'Sakti, called the Karanas and the materials fashioned out of them are called the chidanuvritti or M\^ay\^a.

35-38. O N\^arada! From the \'S\^attvik Ahamk\^ara are produced the five presiding rulers of the five internal organs named Dik (quarters), V\^ayu, Sun, Varuna, and the twins Asvini Kum\^aras and the four presiding rulers of the four fold divisions of Antahkarana (Buddhis, manas, Ahamk\^ara and chitta) named Moon, Brahm\^a, Rudra, and Ksetraj\~na. Thus the above five organs of senses, the five organs of action, the five V\^ayus and mind, these sixteen substances are reckoned as the S\^attvic creation.

39-40. O Child! The Highest Self has two forms; one gross and the other subtle. The formless Self; the Consciousness incarnate, as it were, is the first form. The Seers consider this formless self to be the primary cause (the ultimatum) of all this phenomenal cosmos. (This is only for the best qualified J\~n\^anis, not for others).

The Second Form is the Gross Form for the meditation of the second class qualified persons; thus the sages say. This second form of the Supreme Goddess is conditioned by inherent M\^ay\^a (time, space and causation); this is also divided into gross and subtle, according as it is the outer or inner body of the second form (and the form suited for the meditation of the third class and the second class devotees).

41. My body is called Sûtr\^atm\^a; I will now tell you the gross body of Brahm\^an, the Highest Self.

O N\^arada! This my body and soul having the nature of a string or thread is called Hiranyagarbha; this is also the gross body of the Param\^atman; therefore the Param\^atman together with the Sûtr\^atm\^a, should also be worshipped. O N\^arada! I will now describe to you the outer gross body of Brahm\^an, the Highest Self; hear it attentively; if one hears it with faith and devotion, one is sure to get salvation.

42-43. I have mentioned to you before the five subtle elements, called the five Tanm\^atr\^as; these, now, when the Panch\^i Karana process is done, are converted into the five gross elements. Now hear what the Panch\^i Karana process means :--

44-46. Suppose you are to create the gross element of water. Divide into two equal parts the subtle element of water; divide also the other 4 elements into two equal parts respectively. Now set apart the first half of each of the five elements; divide the second half of each of the elements into four equal parts. Mix the first half of each of the elements with each of the fourth part of the other four elements; and you get one gross element. Similarly you get the other four gross elements. For example :-- You want to get the gross element of water :-- With the half of the subtle

element (\sfrac{1}{2}) of water mix the fourth part, of the halves of the other elements of ether, fire, air and earth; you get the gross element of water and so on.

The Panch\^ikarana process is clearly illustrated in the following table.

\begin{center}
\begin{tabular}{lccccc}
& \textbf{Ether} & \textbf{Air} & \textbf{Fire} & \textbf{Water} & \textbf{Earth}\\
Ether & \sfrac{1}{2} & \sfrac{1}{8} & \sfrac{1}{8} & \sfrac{1}{8} & \sfrac{1}{8} \\
Air & \sfrac{1}{8} & \sfrac{1}{2} & \sfrac{1}{8} & \sfrac{1}{8} & \sfrac{1}{8} \\
Fire & \sfrac{1}{8} & \sfrac{1}{8} & \sfrac{1}{2} & \sfrac{1}{8} & \sfrac{1}{8} \\
Water & \sfrac{1}{8} & \sfrac{1}{8} & \sfrac{1}{8} & \sfrac{1}{2} & \sfrac{1}{8} \\
Earth & \sfrac{1}{8} & \sfrac{1}{8} & \sfrac{1}{8} & \sfrac{1}{8} & \sfrac{1}{2} \\
Gross element & 1 & 1 & 1 & 1 & 1 \\
\end{tabular}
\end{center}

When the five gross elements are thus produced, consciousness then enters into these elements as their presiding deities; next comes the feeling of egoism (I ness) identifying itself with the body thus created out of the five elements. (I am this body and so forth).

47. This great ``I'', the great consciousness, creating and considering the Cosmos as its body is called the Bhagav\^an, Âdideva, N\^ar\^ayana or V\^aisvanara.

48. When, by the Panch\^ikarana process, the five gross elements, earth, ether, air, etc., are solidified and get their clear definite forms, one, two, three, four, five, qualities are seen to exist in ether, air, fire, water, and earth, respectively.

49-51. Thus ether has one quality only - that is sound: the air has got two qualities - sound and touch; the fire possesses three qualities - sound, touch, and form; the water has got four qualities - sound, touch, form and taste; the earth has got five qualities - sound, touch, form, taste and smell, and by the various combinations of these five gross elements, is produced this grand Cosmos, the great body of Brahm\^an.

52. Similarly the sum-total of J\^iv\^as is produced from the several parts of the whole Brahm\^anda; these J\^iv\^as are eighty four lakhs; so the sages say.

Thus ends the Seventh Chapter of the Third Skandha of \'Sr\^i Mad Dev\^i Bh\^agavatam, the Mah\^a Pur\^anam, of 18,000 verses, on the creation and the Tattvas and their presiding Deities.

Note :Of these J\^iv\^as, those who are the best qualified, the Uttam\^adhik\^aris, are known as the Brahm\^anas, J\^anaghana Tûr\^iyas, as denoted by Om Hr\^im; the middlings have their gross, subtle and causal bodies and are called as Brahm\^a Vai\'sv\^anara, Sûtra, Hiranyagarbhas; and the third class is known as Vi\'sva, Taijasa. and Pr\^aj\~nas and forms the body, as it were, of the Brahm\^an. There are others also, animals, etc., in the lowest class.

