\chapter{On the birth of Vy\^asa Deva}


1-10. Once on an occasion, the highly energetic Muni Par\^a\'Sara went out on pilgrimage and arrived on the banks of the Yamun\^a, and spoke to the religious fisherman who was taking his food then thus :-- ``O fisherman! Take me on your boat and carry me to the other side of the river.'' Hearing this, the fisherman spoke to the beautiful girl Matsyagandh\^a :-- ``O beautiful smiling one! This religious ascetic intends to cross the river; so take the boat and carry him to the other bank.'' Thus ordered by her father, the exceedingly beautiful Vasu girl Matsyagandh\^a began to steer the boat whereon sat the Muni. Thus while the boat was sliding on the waters of the Yamun\^a; the Muni Par\^a\'Sara saw the beautiful-eyed damsel Matsyagandh\^a and became as if under the command of the great destiny, greatly enamoured of her. He desired to enjoy Matsyagandh\^a, full of youth and beauty and with his right hand caught hold of her left hand; the blue coloured Matsya looking askance spoke out smilingly in the following words :-- O knower of Dharma! What are you going to do, pierced by the arrows of Cupid? What you desire now, is it worthy of your family or your study of the \'S\^astras or worthy of your Tapasy\^a; see, you are born in the line of Va\'Sistha and you are well known as of good character. O best of Br\^ahmins! You are quite aware that the attaining of a man-birth in this world is very rare; and over and above this the attainment of Br\^ahminhood is, as far as my knowledge goes, particularly difficult.

11-14. O Prince of Br\^ahmins! You are the foremost and best as far as your family, goodness, and learning in the Vedas and other \'S\^astras are concerned; you are well versed in Dharma; how is it, then, that you are going to do this act, not worthy of an \^arya, though you see me possessed of this bad smell of fish throughout my body. O one of unbaffled understanding! O best of twice-born! What auspicious sign do you see in my body that you are stricken with passion on my account that you have caught hold of my hand to enjoy me? Why have you gotten your own Dharma? Thus saying, Matsyagandh\^a thought within herself :-- ``Alas! This Br\^ahmin has certainly lost his brains in order to enjoy me; certainly he will be drowned just now in his attempt to enjoy me in this boat; his mind is so much agitated with the arrows of Cupid that no body, it seems, can act against his will.'' Thus thinking Matsyagandh\^a spoke again to the Muni :-- ``O highly fortunate one! Hold patience! let me first take you to the other side of the river; then you may do as you please.''

15-19. Hearing these reasonable words, the Muni let loose of her hand and took his seat on the boat and gradually got down on the other bank of the


river. But the Muni, becoming again extremely passionate caught hold of Matsyagandh\^a; when the young woman spoke to Par\^a\'Sara, in front of her, shuddering :-- ``O best of Munis! My body is emitting very bad smell; do you not feel this? You know very well that the sexual intercourse between male and female of similar types brings in happiness and comfort.'' Thus spoken to, Par\^a\'Sara made Matsyagandh\^a emit sweet scent like that of musk to a distance of one Yojana (8 miles) and her body exceedingly lovely and beautiful and, becoming extremely passionate, again caught hold of her right hand.

20-34. Then the auspicious Satyavat\^i addressed Par\^a\'Sara Muni, resolved to enjoy her, thus :-- ``O Muni! Behold! all are looking at us; my father too, is there on the bank of the Yamun\^a; so, O Muni! wait till night this beastly act before all is highly unsatisfactory to me. The wise persons declare it a great sin to commit sexual intercourse during day they have ordained night time as the best time of intercourse for men not the day time; the more so that many person's eyes are in this direction. So, O intelligent one! hold on your passion for a while; for the blame pronounced by the public is horrible.'' Hearing these reasonable words, the liberal minded Par\^a\'Sara created, by his influence of Tapasy\^a, a dense fog so that both the banks of the Yamun\^a became covered with darkness. Then Matsyagandh\^a gently spoke to the Muni :-- ``O best of Dv\^ijas! I am not as yet married; I am now a girl; you will go away after enjoying me; your semen virile is not fruitless; so Br\^ahman! What will be my fate? If I be pregnant today, what shall I say to my father? and what will be my future state? There is no doubt that, after enjoying me, you will go away; what will I do afterwards; kindly say.'' Hearing these words of Matsyagandh\^a, Par\^a\'Sara said :-- ``O beloved! after you have done my pleasant duty, you will remain a girl as you are now; yet, O timid one! ask from me any boon you like; I will grant it to you.'' Satyavat\^i then said :-- ``O best Br\^ahman, O giver of one's honour! grant me these things :-- That my father and mother do not know anything of this affair and that my virginity be again as ever the same. Also let an extraordinarily powerful energetic son be born to me like you; let this nice smell continue to remain always in my body and let my youth and beauty remain afresh and increase ever more. Hearing this, Par\^a\'Sara said :-- ``O beautiful one! a son, very pure and holy, will be born to you, from N\^ar\^ayana's part! his name will be famous in the three worlds. O beautiful one! never before my heart was agitated with such passion. I do not know why I have become so much passionate for you. I saw the unrivalled beauties of Apsar\^as but I never lost my patience; but seeing you, I have become attracted to you; it must be under the


direction of Providence; know it certain that there must be some mysterious cause in this. However Fate is unavoidable to all; otherwise you are full of so bad smell; why shall I be fascinated by your sight? O beautiful one! your son will be famed in the three worlds; will compose the Pur\^anas and will sub-divide the Vedas.

Thus saying, the Muni Par\^a\'Sara enjoyed Matsyagandh\^a, who became quite submissive; and after bathing in the Yamun\^a, quickly went away. On the other hand, the chaste Satyavat\^i, too, became pregnant and immediately gave birth on the island of Yamun\^a to a son beautiful, as if the Second K\^amadeva, the god of Love, K\^amadeva. No sooner that son, very fiery and highly potent, was born than he devoted his mind to tapasy\^a and spoke to his own mother Satyavat\^i thus :--``O Mother! now go wherever you like; I will also go to perform tapasy\^a. O highly fortunate one; No sooner you remember me, I will come to you. O Mother! where you will have any onerous duty, remember me and I will instantly come to you. Let all good be unto you; now I go. Avoid all cares and live happily. Thus saying, Vy\^asadeva went out. Matsyagandh\^a, too, went back to his father. Vy\^asa was named also Dvaip\^ayan (born in an island, a Dv\^ipa) in as much as Satyavat\^i gave birth to him in a Dv\^ipa island); and as he was born of Visnu's parts, he grew up no sooner he was born.

The Muni Dvaip\^ayana bathed in every T\^irtha and performed the highest asceticism. Thus Dvaip\^ayan Vy\^asa was born of Par\^a\'Sara in Satyavat\^i's womb. Seeing the advent of Kali Yuga, he adorned the tree of the Vedas with many \'S\^akh\^as (branches). It is because he expanded the Vedas many \'S\^akh\^as, that he is denominated also as VedaVy\^as; he composed eighteen Pur\^anas, Samhitas, the excellent Mah\^abh\^arat, subdivided the Vedas and made his disciples Sumantu, Jaimini, Paila, Vais\^amp\^ayan, Asita, Devala and his son \'Suka to study them.

S\^uta said :-- ``O Munis! Thus I have described to you the birth of the holy Vy\^asa, the son of Satyavat\^i and all the causes. O Munis! Do not allow any doubt enter your mind as regards his birth; for it is always advisable to take up only the good things as far as the lives of great persons and Munis are concerned. There must be some extraordinary mysterious cause owing to which Satyavat\^i was born of a fish, and she was first united to Par\^a\'Sara and then to S\^antanu. Otherwise how can one account for the fact of the Muni Par\^a\'Sara being so much agitated by passion and why he would behave like a mean low person in the committal of a a grossly blameable act? Now has been spoken the wonderful birth story of Vy\^asa Deva together with all incidents, and enveloped under the great mystery. If any man hears this holy narrative, he will be freed from sins and will never fall into difficulties and will always be happy.


Thus ends the Second Chapter of the Second Skandha on the birth of Vy\^asa Deva in the Mah\^apur\^anam \'Sr\^i Mad Dev\^i Bh\^agavatam of 18,000 verses.