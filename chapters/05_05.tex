\chapter{On the defeat of the D\^anava forces of Mahisa}

1-6. Vy\^asa said :-- The thousand eyed Indra, hearing this, again asked to Brihaspati that he would make preparations for war against Mahis\^asura. Without effort kingdoms are not attained; no - nor happiness, nor fame, nor anything; those who are weak, they extol effortlessness; but the powerful never praise that. Knowledge is the ornament of the ascetics and contentment is the ornament of the Br\^ahmanas; but those who desire lordship over powers, effort and prowess to destroy one's enemies are their excellent ornaments. O Muni! I will kill this Mahis\^asura by my heroism as I had, of old, destroyed Vritra, Namuchi and Bal\^asura. You are the Deva Guru; therefore you and my thunderbolt are my strength. The immortal Hari and Hara also will help me in this. O Guru! Preserver of my honour and prestige! Now recite the mantras calculated to remove all the obstacles towards my victory. I, too, am making preparations and raising up my own forces to wage up war against that D\^anava Mahisa.

7-13. Vy\^asa said :-- On hearing Indra's words, Brihaspati smiled and said ``O Lord of the Devas! I see you are bent on fight. I will neither stimulate you to fight nor shall I make you desist from the purpose. The issue is doubtful. There may be defeat or there may be victory. O Lord of \'Sach\^i! You are not to blame at all in this matter; what is written in the Book of Fate will come to pass, be it victory or defeat. I am not aware of the future in this respect. O Child! You know already what an amount of suffering I had to endure in times gone by when my wife had been stolen. O Destroyer of the enemies! My wife had been stolen by Moon who turned out my enemy; living in my stage of an householder I was put to all sort of miseries, deprived of all my happiness. O Lord of the Suras! I am renowned in all the worlds as a man of much wisdom and intelligence. Where then was my intelligence, when Moon carried away, perforce, my wife. O Lord of the Suras! To my mind, the success or failure depends entirely on destiny; yet intelligent ones should always resort to efforts and be energetic.

14-17. Vy\^asa said :-- O King! On hearing the words of Brihaspati, pregnant with truth, Indra went with him to Brahm\^a, took his refuge and saluting him said :-- O Grand Sire! The D\^anava is collecting a big army, and wants to conquer and take possession of the Heavens. All the other D\^anavas have enrolled themselves in the list of his army; they are eager to fight and they are all very powerful and skilled in arts of warfare. I am therefore very afraid and have come to you. You know everything; please help me in this matter.

18-20. Brahm\^a said :-- We all will go today to the Mount Kail\^a\'sa and take \'Sankara with us and go to Visnu. There all the Devas, assembled, will hold a council and consider the time and place, when it will be settled whether it is proper or not to fight. For one who dares to do any act without considering one's strength and without any judgment, certainly courts his own downfall.

21-35. Vy\^asa said :-- O King! Hearing this, Indra with the other Lok\^apalas and Devas, headed by Brahm\^a, went to Kail\^a\'s\^a. Then they came to \'Sankara and sang vedic hymns to him. Mahe\'svara became very much pleased and they taking Him went to Vaikuntha, the abode of Visnu. Indra saluted Visnu and sang hymns to him, and told him about his errand thus :-- ``Mahisa has become very haughty on account of the favour bestowed on him and therefore we are very afraid (and therefore ask your help to relieve us from this danger).'' Visnu, then, hearing the cause of fear, told them :-- ``We all will fight and kill that Demon.'' Vy\^asa said :--O king! Thus settling the question, Brahm\^a, Visnu, and Hari and Indra and the other Devas riding on their own V\^ahanas (means of conveyance) respectively dispersed. While Brahm\^a on his vehicle Swan, Visnu on his Garuda, \'Sankara on his Bull, Indra on his elephant Air\^avata, K\^artika on his peacock, and Yama, the god of death on his V\^ahana, the Buffalo, were on the point of going with the other Deva forces, the army of the D\^anava Mahisa met them on their way, all fully equipped with arms and weapons. A dreadful fight then ensued between the Devas and the D\^anavas.

Arrows, axes, Pr\^asas, Musalas (clubs), Para\'sus (pick axes), Gad\^as (clubs), Patti\'sas, \'S\^ulas (tridents), chakras (discus) \'Sakti (weapons), Tomaras, Mudgaras, Bhindip\^alas, L\^angalas, and various other deadly weapons appeared on the scenes with which they fought against one another. The Commander-in-Chief of Mahisa, the very powerful Chiksura, shot five sharp arrows at Indra. The ever-ready and light-handed Indra too, with his arrows cut off all of them and struck at his heart heavily with his Ardhachandra (half moon) arrow. The Commander-in-Chief, struck by this arrow fell senseless on the back of his elephant. Indra, then struck the trunk of the elephant with his Vajra (thunderbolt); the elephant then severely struck with the Vajra fled away into the D\^anava's forces. The Lord of the D\^anavas seeing this, got very angry and addressed the general Vid\^ala ``O Hero! You are very powerful; go then and kill first that haughty Indra; then kill Varuna and other Devas and come back to me.''

36-57. Vy\^asa said :-- The very powerful Asura Vid\^ala, on receiving the order came up at once to Indra, mounted on a very furious elephant. Seeing him coming, V\^asava shot at him angrily with very terrible and most powerful arrows that looked like deadly snakes. But the Demon, too, out off those arrows at once with his excellent arrows and quickly shot at V\^asava fifty arrows, sharpened on stones. Indra cut off all those and, being infuriated, shot again sharp deadly serpent-like arrows at him, and cutting off again all his enemie\'s arrows by arrows discharged from his bow, struck the elephant's trunk with his Gad\^a (club). The elephant, being thus struck on his head, cried aloud in a distressed tone and being afraid turned back, thus killing the D\^anava forces as he fled away. The general Vid\^ala, seeing the elephant fleeing away from the battle-field, mounted on a beautiful chariot and instantly appeared before the Devas to fight with them. Seeing the D\^anava coming again on a chariot, Indra shot at him sharp arrows after arrows like venomous snakes. The powerful D\^anava, too, infuriated hurled at him terrible arrows; then a sharp conflict ensued between V\^asava and the D\^anava. Finding the D\^anava powerful, V\^asava's senses were confounded with anger; he then took his son Jayanta before him and began to fight. Jayanta stretched his bow tight and shot at the breast of the D\^anava swelled with pride, five sharp arrows with his full strength. Thus shot at by the network of arrows, the D\^anava fell unconscious on the chariot; the charioteer then fled away with his chariot from the battle-field. Thus on the D\^anava Vid\^ala becoming unconscious and being taken away from the field, the Dunduvis (drums) of the Devas were resounded and great acclamations

of ``Victory to the Deva\'s' were heard. The Devas were very glad and sounded hymns before Indra; the Gandarbhas began to sing and the Apsar\^as began to dance. O king! Hearing the loud acclamations of victory to the Devas, Mahisa became very angry and ordered the D\^anava T\^amra, the destroyer of enemy's pride, to go to the battle-field. T\^amra appeared in the battle, and, coming face to face with many Deva warriors, hurled on them showers of arrows. Varuna appeared with his P\^a\'sa weapon and Yama, mounted on his buffalo, appeared with his Danda (staff). A terrible fight then ensued between the Devas and D\^anavas and the weapons, arrows, axes, Musalas, \'Saktis and Para\'sus glittered in the fields. Yama raising his Danda with his hands struck at T\^amra; but the powerful T\^amra, though severely struck, was not at all moved and remained firm in his place in the field. On the other hand T\^amra, violently drawing his bow, hurled a mass of sharp arrows at Indra and the other Devas. The Devas got angry and shot at the D\^anava multitudes of divine arrows sharpened on stone, and frequently called aloud ``Wait, wait.'' The D\^anava T\^amra thus shot at by the arrows of the Devas, fell unconscious in the battle-field; the D\^anava forces got afraid and a cry of universal consternation and distress arose.

Here ends the Fifth Chapter of the Fifth Skandha on the defeat of the D\^anava forces of Mahisa in the M\^ah\^apur\^anam \'Sr\^i Mad Dev\^i Bh\^agavatam of 18,000 verses.