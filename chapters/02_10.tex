\chapter{On the death of king Par\^iksit}

1-3. S\^uta said :-- ``O Risis! On that very day when the Br\^ahmin Ka\'Syapa went out of his house, Taksak, knowing the king Par\^iksit cursed, assumed an aged Br\^ahmin's form and went out of his abode.

The serpent Taksak met the Br\^ahmin Ka\'Syapa on the way. Seeing the Br\^ahmana, versed in the Mantras, Taksak asked him ``Where are you going so in haste, and what for are you taking this trouble?''

4-17. Thus questioned, Ka\'Syapa replied :-- I heard that the serpent Taksak will bite the king Par\^iksit; therefore I am going in haste to the king Par\^iksit to cure him of the serpent's poison. I know the mantra (mystic verse) that can destroy the effect of poison. If his life-period is not exhausted, I will certainly give him back his life. Taksak then. Said :-- ``O Br\^ahmana! I am that Taksak; I will bite him and take away his life. So you better desist. Will you be able to treat him whom I bite; certainly you will not.'' Ka\'Syapa said :-- ``O chief of snakes! When you will bite the king who has been cursed by the Br\^ahman, I will no doubt make him alive by the power of my mantra.'' Taksak said :-- ``O chief of Br\^ahmanas! If you have so thought that you will make the king alive after I bite him, then shew me your strength before hand. O sinless one! I will bite this Nyagrodha tree (the Indian fig-tree); just now make it alive.''

Ka\'Syapa said :-- ``Certainly I will make this tree alive, that will be burnt away by the venom of your teeth.'' S\^uta said :-- ``The snake Taksak then bit the tree, which was reduced to ashes; and asked Ka\'Syapa to bring back that tree to life.'' Seeing the tree reduced to ashes by the fire of venom of the snake, he collected all the ashes and said :-- ``O highly venomous serpent. See to-day the power of my mantra. Behold! While you are witnessing, I will enliven this tree. Thus the great mantra-knower Ka\'Syapa took water in his hand, and impregnating it with his mantra power, sprinkled the water on the ashes. Immediately, on the sprinkling of the mantra saturated water, the Nyagrodha tree got back its life as before. Taksak became greatly astonished to see the tree enlivened again and said to Ka\'Syapa :-- ``O chief of Br\^ahmans! What is your object in taking so much pains? Speak out what you want and I will fulfil your desires.'' Ka\'Syapa said :-- ``O chief of serpents! Knowing the king cursed, I am going to do good to him by my knowledge and to get in return abundant wealth.'' Hearing this, Taksak said :-- ``I will give you the amount of wealth that you desire; take that and go back to your house, and let my desire be also fulfilled.''

18-26. Ka\'Syapa, the knower of the highest state, heard Taksaka's words and pondered in his mind again and again. ``What is to be done now? If I take this wealth and go back to my house, my name and fame will not be known in this world, simply for my greed; but if the king be made alive again, my undying fame, abundant wealth, and greater

Punyam will accrue to me. Again fie to that wealth with which there is no fame; so one must try one's best to preserve one's fame. The king Raghu, in ancient days, gave away everything of his to the Br\^ahmanas for fame; the king Harischandra and Karna did not hesitate a bit to give away endless property. There is one point again to take into account, how can I trifle away the matter, seeing the king burnt up by the venomous fire?

If I can bring back the king's life, everyone will become happy. If the kingdom be without its king, the subjects will, no doubt, be ruined. So, following the king's death, sin will also incur on me due to the ruin of the subjects; and infamy will come on my head that I am a very greedy man.'' Thus meditating in his mind, the highly intelligent Ka\'Syapa began to meditate, and plunged himself in Dhy\^an; he thereby came to know that the king's life period was spent up. Thus knowing the king's death imminent, the virtuous Ka\'Syapa took the desired wealth from Taksak and returned home.

27-48. Thus making Ka\'Syapa to retire to his house on the seventh day Taksak went on to Hastin\^apur to bring death and destruction on to Par\^iksit. When he went close to the city, he heard that the king Par\^iksit was staying on the upper story of the palace; and the palace had been preserved by various gems, mantras, herbs and plant. Taksak became very anxious; and fearing, lest the curse of the Br\^ahmanas, will fall on his head, became very much agitated and thought. ``How shall I now enter the palace? How can I cheat this stupid hypocrite vicious king, cursed by the Br\^ahmana, who causes troubles to the Br\^ahmanas. Not a single man has taken birth in the Pandava family ever since that he coiled a dead serpent round the neck of an ascetic Br\^ahmin. The king has committed a very heinous crime and knowing the course of time to be in fallible, has placed sentries on all sides of the palace and has ascended to the top-most story of the building, thinking thereby to deceive Death and is staying in a peaceful mind. How can then he be smitten, in accordance with the Br\^ahmana's word? The king, of dull intellect, knows not that death cannot be prevented; for that reason he has placed guards and sentinels round the building, and himself has got up the house and is happily whiling away his time; but he is quite ignorant that when Fate who can never be violated, ordains the death, how can it be prevented though thousands of attempts are made to thwart it? This scion  of Pandu family knows that his death is at hand and yet wants to live and therefore is staying in his own place with a tranquil mind. The king ought now to make charities and other meritorious works; it is only by acts of Dharma that disease is destroyed and life is prolonged.

And if that be not the object then a dying man ought to take bath, to make charities and to await his time of death; he thereby attains heaven; otherwise hell is inevitable. The king committed great sin in the act of causing pains and trouble to the Br\^ahmin or other similar acts and therefore death is so close that the Br\^ahmin curse has fallen thus on his head. Is there no such Br\^ahmin who can make him understand this; or the Creator has ordained his death now as inevitable.'' Thus meditating, the chief serpent made other serpents following him assume then form of ascetic Br\^ahmanas and gave them roots and fruits to be taken to the king. The serpent Taksak himself entered within the fruits in the form of an insect. Then the ascetic serpents took the fruits and quickly went out of the place. They came to the palace where Par\^iksit was resting. Seeing them, the guards asked :-- ``What for have you come here?'' Hearing this, ``We are coming from the hermitage to prolong the life of the hero king, the son of Abhimanyu and the son of the P\^andava family, by chanting the mantras of the Atharvavedas, and we want to have an interview with the king; now you better go and inform the king that some Munis have come to see you. We will sprinkle water on him and give him some sweet fruits and then depart. We have never come across such gatekeepers in the family of Bharat as disallow the ascetic Muni visitors to go and see the king. We will ascend to the place where the Par\^iksit is staying and we will bless him, and wish him long life; we will communicate to him our orders and then depart to our own places.''

49-68. S\^ut\^a said :-- Hearing these words, the sentinels spoke as previously ordered by the king, as follows :-- ``O Br\^ahmanas! We think verily you won't be able to have an interview with the king to-day; you, all ascetics can come to-morrow to this palace. O Munis! Owing to the Br\^ahmana's curse, the king has built this place; then it follows, as a matter of course, that the Br\^ahmanas are not allowed to get up to the palace.'' Then the serpents, in the form of the Br\^ahmanas, spoke :-- ``O good sentinels! Then take these roots and fruits and offer them to the king and communicate to him our blessings.''

The sentinels went to the king, and informed him of the arrival of the ascetic Br\^ahmanas. The king replied :-- ``Bring here the roots and fruits offered by them and ask what for they have come. Give them my pran\^ams; to-day I cannot meet with them; let them come to-morrow morning.'' The sentinels went to the ascetics and got from them their roots and fruits and offered them with great respect to the king. When the serpents in the guise of the hypocrite Br\^ahmins went away, the king took those fruits and spoke to his ministers :--

``Take these fruits and let all my friends eat them. I will take only this one fruit given by the Br\^ahmanas and will eat it.'' Saying this, the Uttar\^a's son Par\^iksit gave away fruits to the friends and took one ripe fruit for himself, broke it and saw within it a very fine copper-coloured black eyed insect. At this the ministers were astonished; the King spoke to them :-- ``The sun has set; so there is no further chance of any fear from any poison to-day. I speak then to-day, fearing the Br\^ahman's curse, let this insect bite me.'' Thus saying the king took that insect and placed it on his neck. That Taksak in the form of an insect, when placed, during the sun-set, on the neck by the king, immediately assumed the form of the terrible K\^ala (Death), coiled round the king and beat him. The Ministers were greatly surprised and began to weep and cry with great pain and sorrow. Seeing that terrible serpent, the ministers, overwhelmed with terror, fled away on all sides. The guards cried out loudly. The terrible out-cry was raised on all sides. Then Uttar\^a's son, the king Par\^iksit, coiled by the serpent, saw that all his efforts were rendered fruitless, and remained silent and held fast to his patience. From the mouth of the serpent Taksak the terrible venomous flames came out burning all and immediately killed the king. Thus taking away the life of the king, Taksak went up in the celestial atmosphere; the people then saw that the serpent was ready as if to burn the world. The king fell down lifeless like a burnt tree; and all the persons cried out seeing the king dead.

Thus ends the tenth Chapter of the Second Skandha on the death of the king Par\^iksit in the Mah\^apur\^anam \'Sr\^i Mad Dev\^i Bh\^agavatam of 18,000 verses.