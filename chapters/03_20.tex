\chapter{On the Svayamvara hall and the king\'s conversation there}

1. Vy\^asa said :-- O Noble minded one! The king Yudh\^ajit replied to the address of the king of Keral, thus :--

2-3. O King! You are truthful and have restrained your passions. What you have told just now in this assembly of kings is all correct and approved by morality. O best of the kings! You are born of a high family; you better say how can this take place that when so many fit persons are present here to become the bridegroom, can an unworthy person take away the offer?

4. As a jackal never becomes fit to enjoy what are the dues of a lion, so this Sudar\'sana is also unfit to acquire this bride elect.

5. The Br\^ahmanas have the Vedas as their strength; the Ksattriya kings take bows and arrows to be their source of strength; this is ordained everywhere. Therefore O King! What wrong have I done in my statement. Kindly explain.

6-7. The power of the kings is the befitting money given to the parents of a bride; according to this, the strongest man is to acquire the bride, a jewel. The Ksattriyas that are weak can never acquire that. Make this the rule in this marriage. This earth is fit to be enjoyed by the heroes only and not by the cowards and intriguing persons. Otherwise quarrels are sure to ensue amongst the kings.

8. The dispute thus arose in that Svayamvara hall; and the king Sub\^ahu was called in.

9. The kings that could see the reality of things then addressed the king Sub\^ahu. ``O king! You are requested to establish a golden rule in this marriage ceremony.

10. What is your object in calling this Svayamvara. Better give it out after a mature consideration. Please be explicit whom have you intended to give over your daughter in this marriage?''

11-12. Sub\^ahu said :-- ``My daughter has mentally selected Sudar\'sana; I prevented her repeatedly from doing this; but she did not accept my word. What shall I do now? The mind of my daughter now is not at her will. Sudar\'sana, too, though uninvited, has come here singly and is residing calmly, without any signs of disturbance in his mind.''

13-14. Vy\^asa said :-- Then the chief kings all invited Sudar\'sana there; Sudar\'sana, too, came there quietly, and the princes, seeing his quiet nature, asked him, ``O one, engaged in practising vows! Who has invited you here? Why have you come here singly, in this assembly of kings?

15. You have no force, no ministers, no help, no wealth, and no army. O intelligent! Then explain why have you come here alone?

16. In this assembly of kings you see that the powerful monarchs are ready to fight with each other for the sake of this princess. What do you intend to do under those circumstances?

17. Your brother, too, is come here to have the princess; he has got his army and is also marked with his strength and valor. The powerful Yudh\^ajit has come here also to help him.

18. O observer of good vows! Seeing you without any army, we have narrated to you all the facts. Now think and do accordingly. If you please, remain here or go anywhere else.''

19-20. Sudar\'sana replied :-- ``True, I have no army, no help, no wealth, no forts, no friends or no kings to protect me. Hearing that Svayamvara will be held here, I have come here to witness it. But there is one particularity here; it is this :-- The Dev\^i Bhagavat\^i has ordered me in my dream to come here. Under Her command I have come here; there is no doubt in this.

21. I have got no other object in view; I have obeyed what the Bhagavat\^i Bhuvane\'svar\^i has ordered me to do. Today will happen, no doubt, what She has ordained.

22-23. O kings! I am seeing everywhere the Supreme Goddess Bhagavat\^i Bhav\^an\^i. Therefore there is no enemy of mine in this world; but he who will turn out an enemy of mine, will be duly punished by the Mah\^a Vidy\^a Mah\^a M\^ay\^a. I do not know what is enmity?

24. O kings! What is inevitable will surely come to pass. There will be nothing otherwise. I am always depending on Fate, Destiny. What is the use, then, in thinking what will be the result?

25. Everywhere in the Devas, in the spirits, in men, in all the beings, the Dev\^i's power is existent; it cannot be otherwise.

26. O kings! Whenever She wishes, She makes kings, masters of wealth or devoid of wealth. What is, then, the use of bothering my head in this?

27. When even the Gods Brahm\^a, Visnu, and Mahe\'sa without Her presence, become powerless enough to move their hands or feet, then why shall I be anxious for the result?

28. O kings! Whether I am unable or able or an ordinary person, you have nothing to consider; I have come here in this assembly hall under the command of the Supreme Bhagavat\^i.

29-30. What She has willed, She will do that. I am not to care for that. O high minded ones! You need not be afraid at all in this. I have told you all truth. Victory or defeat, I feel no shame in either of them. For I am always under the control of Bhagavat\^i; therefore if there be any shame here, it is all Hers.''

31-33. Vy\^asa said :-- Hearing thus his words, and seeing that his mind is firmly devoted to Bhagavat\^i, the kings saw each other and said thus :-- ``O Sudar\'sana! What you have said is quite true; it is never otherwise; still Yudh\^ajit, the king of Ujjain is intent on killing you. O intelligent! O sinless! we have all come to know that there is no trace of evil in you. We were all overcome with pity for you; hence we have informed you; now think and do the needful?''

34. Sudar\'sana said, ``You are all kind and large hearted; what you all have said is quite true. What shall I tell you, being a minor as yet!

35. O kings! No one can cause the death of another. All this world, moving and unmoving, is under the control of Fate.

36-38. No soul is independent; every one is under the effects of one's own Karma. The Pundits that have realised the Truth, say that Karma is of three kinds, Accumulated, Present, and Pr\^arabdha? This whole world is due to K\^ala (Time), Karma (action) and Svabh\^ava (Nature); unless the proper time comes, even the Devas cannot kill men. The men are killed on account of some cause, immediate; but the Great Time is the real Destroyer.

39. My father, though a destroyer of many others, was himself killed by a lion and my mother's father was slain by Yudh\^ajit in the battle.

40. The J\^ivas, though caring hard to preserve their lives, are killed by Fate in spite of all their cares; and they live thousand years though there is none to protect them.

41. O religious kings! I do not fear a bit from Yudh\^ajit. I consider Fate as the Supreme and I therefore remain always undisturbed, calm and quiet.

42. Daily and constantly I remember Bhagavat\^i, Who is the Mother of all this Universe. She will look after my welfare.

43. Behold! One will have certainly to bear the burden of one's past Karma, whether it be good or it be bad; one's own actions must bear their fruits. Then why shall he be sorry, who has come to know this?

44. The less intelligent deluded persons, on getting pain from their own actions, turn out enemies on very trifling matters.

45. I do not grieve nor do I fear on account of such enemies. I am staying here in this assembly of kings, cool-minded.

46. Under the order of Chandik\^a, I have come here to see this Svayamvara; whatever is inevitable will surely come to pass.

47. The words of the Bhagavat\^i are the best proof; I do not know any other. My mind is entirely given up to Her. There will be nothing otherwise than what She has ordained; whether it is good or whether it is bad.

48. O kings! Let Yudh\^ajit remain in peace. I have no enmity with him. He, who will deal inimically with me, will certainly reap his reward. There is not the least doubt in this.''

49. Vy\^asa said :-- O king! When Sudar\'sana addressed them thus, all the kings became very glad and they all remained there for the Svayamvara. Sudar\'sana, too, went to his camp and remained also calm and quiet.

50. Next day the king Sub\^ahu invited all the kings present in his city to their respective seats in the Svayamvara hall.

51. The princes and kings, decorated with best ornaments, came and took their seats on their respective platforms, covered with valuable carpets of best workmanship.

52. The kings then looked like the celestial Devas, wearing divine ornaments and apparels, blazing with the lustrous light of gems, and remained to see the Svayamvara affair.

53. Every one there had this foremost thought in his mind when will the princess, the bride elect, would come there; and who will be the man so fortunate as to be blessed with garlands offered by her (as a token of selection of the bridegroom)!

54. If, accidentally, she offers the garland to Sudar\'sana in this Svayamvara assembly, then will ensue, no doubt, desperate struggles amongst the kings.

55. While they were thus meditating, sounds of drums were loudly sounded.

56-58. Then Sub\^ahu, the king of Benares, went to her daughter and found that \'Sa\'sikal\^a had just taken her bath and put on her silken clothes, and adorned herself with various ornaments and sweet garlands. Thus, dressed in complete marriage dress, she began to shine like another Goddess Laksm\^i, the Goddess of wealth. The king, on seeing his daughter dressed in silken cloth, afflicted with anxious thoughts, just smiled and said, ``Child! Rise and take the beautiful garlands by your hands and go to the the Svayamvara hall and just look at the assembly of kings.

59. O lean bodied one! Whoever, well-qualified, beautiful, and of noble birth, amongst the kings is reigning in your mind, better select him.

60. O graceful! The kings from various quarters are adorning their respective seats; better go and see and select whomever you like.''

61. Vy\^asa said :-- When Sub\^ahu had spoken thus, \'Sa\'sikal\^a, who generally talked little, replied with sweet sonorous words, impregnated with religious truth.

62. ``Father! I won't go before the kings who are inspired by lust; women like me never go there; it is those that are dissolute that attend those places.

63. Father! I have heard from the religious texts that women should cast their glances on their husbands only and not on any other.

64. The woman that goes to many persons is mentally claimed by all; each of them contemplates strongly ``Let this woman be mine.'' Thus her chastity is destroyed.

65-66. Desirous of selecting her husband, when the woman holding in her hands, the garland for her would-be-husband, goes to the Svayamvara hall, then she turns out like an ordinary unchaste woman. As a prostitute going to a public shop looks on many persons and judges of their merits and demerits according to her own power of judgment, the maid that goes in the Svayamvara hall does exactly the same.

67. How can I behave myself in the hall of the assembly of kings like a prostitute, who does not attach her feelings firmly on a single individual but glances constantly at many lustful persons.

68. Though this system of Svayamvara is approved by the elderly persons, I am not going to follow that now. I will take the vow of a chaste woman and act up to that doctrine as perfectly.

69. I will never be able to act like an ordinary woman going in the Svayamvara hall, mentally determining many and finally selecting one.

70. Father! From the very beginning I have given myself up to Sudar\'sana in mind, word and deed. I have not the least inclination to leave him and select another in his stead.

71. O King! If you want to have my welfare, then give your daughter on an auspicious day and in an auspicious lagna to Sudar\'sana, according to the prescribed rites.''

Thus ends the 20th Chapter on the Svayamvara hall and the king\'s conversation there in \'Sr\^imad Dev\^i Bh\^agavatam of 18,000 verses by Mah\^arsi Veda Vy\^asa.