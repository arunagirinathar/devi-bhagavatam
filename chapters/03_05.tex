\chapter{On the chanting of hymns by Hara and Brahm\^a}

1. Brahm\^a said :-- O Nàrada! Thus speaking, Visnu stopped; Sankara, the Destroyer, then stepped in and, bowing down to the Dev\^i said :--

2. \'Siva said :-- O Dev\^i! If Hari be born by Thy power and the lotus-born Brahm\^a have come into existence from Thee, why, then, I who of Tamo Guna be not born of Thee! O Auspicious One! Thou art clever in creating all the Lokas! What wonder is there in My being created by Thee.

3. O Mother! Thou art the earth, water, air, \^ak\^asa and fire. Thou art, again, the organs of senses and the organs of perception; Thou art Buddhi, mind and Ahank\^ara (egoism).

4. Those who say that Hari, Hara, and Brahm\^a are respectively the Preserver, the Destroyer and the Creator of this whole Universe do know anything. All the three, above mentioned, are created by Thee; then they perform always their respective functions; their sole refuge being Thyself.

5. O Mother! If the Universe be created of the five elements, earth, air, ether, fire, and water, having the properties of touch, taste, etc., then how these five elements possessing attributes and of the nature of effects, can come into manifestation, without their being born from Thy Chit portion (Intelligence)?

6. O Auspicious Mother! It is Thou in the shape of Brahm\^a, Visnu and \'Siva, That art creating this Universe and it is Thou that hast assumed the form of this whole Universe, moving and non-moving. Thus Thou playest, as it wills Thee, under various forms, again and again. Thou dost cease from play (during pralaya) as it likes Thee.

7. O Mother! When Brahm\^a, Visnu and I become desirous to create the world, we execute our duties by taking the dust (earth, etc.) of Thy lotus feet.

8. O Mother! It it were not Thy mercy, then how Brahm\^a could have become endowed with Rajoguna, Visnu with Sattvaguna and I with Tamoguna?

9. O Mother! If there were no differences observed in Thy mind, then why hast Thou created in this world rich and poor, king and councillors, servants, etc., various classes of beings? Why hast Thou not created all alike happy or all alike miserable?

10. So Thou wilt have to show Thy mercy towards me. Thy three gunas are capable at all times to create, preserve and destroy the world; then Hari, Hara and Brahm\^a, whom Thou hast created as the cause of the three worlds, is simply Thy will.

11-12. O Bhav\^ani! If Thy Gunas had no power in the acts of creation, etc., then how can the fact that while we three Hari, Brahm\^a and I were coming in the aeroplane, we saw on our way new worlds created by Thee, become possible? Kindly dost Thou say on this. O World-Mother! It is Thou that desirest to create, preserve, and destroy this world by Thy part M\^ayik power. Thou art always enjoying with Purusa, Thy husband. O \'Siva! We cannot fathom Thy inscrutable ways.

13-15. O auspicious one! How can we understand Thy sport? O Mother! We are transformed into young women before Thee; let us serve Thy lotus feet. If we get our manhood, we will be deprived from serving Thy feet and thus of the greatest happiness. O Mother! O Sire! I do not like to leave Thy lotus feet and get my man-body again and reign in the three worlds. O Beautiful faced one! Now that I have got this youthful feminine form before Thee, there is not a trace of desire within me to get again my masculine form. What use is there in getting manhood, what happiness is there if I do not get sight of Thy lotus-feet!

16-18. O Mother! Let this unsullied fame of mine be spread over in the three worlds that I have got, in this young womanly form, the chance of serving Thy lotus feet that has got this effect that the idea of world goes away. Who is there that will leave Thy service and desire to enjoy the foeless kingdom in the world? Oh! even a moment appears a Yuga to him who has not got Thy lotus feet with him! O Mother! Those that

leave the worship of Thy lotus feet and become engaged in performing tapasy\^a are certainly deprived of the best thing by the Creator, though their minds be pure and holy. Their power from their Tapasy\^a may be acquired and they be entitled for Mukti; yet they get dire defeat from not having Thee.

19. O Unborn One! Austerities, control of passions, enlightenment or performance of sacrifices, as ordained in the Vedas, nothing can save, from this ocean of Sams\^ara. It is the devotional worship only of Thy lotus feet that can make one attain the Beatitude. O Dev\^i! If Thou be extremely merciful towards me, then initiate me in that wonderful holy mantra of Thine; I will repeat that omnipotent par-excellent nine-lettered mantra of the Chandik\^a Dev\^i and be happy.

20-26. O Mother! In my former birth I got the nine-lettered mantra but now I have forgotten it O Tarin\^i! O Saviour! Give me today that mantra and save me from this ocean of world. Brahm\^a said :-- When \'Siva of wonderful fire and energy, said this, the Dev\^i Ambik\^a clearly uttered the nine lettered mantra. Mah\^adeva accepted the mantra and became very glad. He fell down at the feet of the Dev\^i, and then and there began to repeat the nine-lettered mantra together with V\^ija (seed) that yields desires and liberation and can be easily pronounced. When I saw \'Sankara, the Auspicious One to all the Lokas, in that state, I fell down also at the feet of the Dev\^i and spoke to Mah\^a M\^ay\^a :-- O Mother! It is not that the Vedas are unable to ascertain Thy nature; for, in the performances of sacrifices and other minor actions, they do not mention Thy full Nature, the Ordainer of all but mention simply Indra and minor deities and Sv\^ah\^a Dev\^i, a portion of Thy essence as the presiding deities of the sacrificial offerings and oblations. So, O Dev\^i! It is Thou that hast been extolled in this Universe as the Universal Consciousness, all knowing and transcending all the Devas and all the Lokas.

Note :-- The nine lettered mantra is ``Om Hr\^im \'Sr\^im Chandik\^ayai namah.''

27. I have created this greatly wondrous Universe; I am the Lord of this Brahm\^anda. Who is there more powerful than me in these three worlds? When I am Brahm\^a, transcending all the Lokas, then I am blessed; there is no doubt in this. By reason of this vanity I am plunged in this widely extended ocean of Samsàra.

28-31. That now I have been able to get the dust of Thy lotus feet, has now made me really proud; and truly I am blessed today and by Thy grace this manifestation of pride on my part has become quite justified. Thou destroyest the fear of this Sams\^ara and givest Mukti. So, O Goddess! pray unto Thee that Thou dost cut asunder this iron chain of my delusion,

full of great troubles and make me devoted to Thee. O Auspicious One! I am born from the lotus discovered by Thee; now I am extremely anxious how I can get Mukti. I am Thy obedient servant; I am merged in the delusion of this ocean of world. Save me O \'Siva! from this Sams\^ara. Those who do not know Thy character, think that I am the Creator and Lord of this Universe; those, who do not worship Thee and worship Indra and other Devas and perform sacrifices to attain Heaven are certainly ignorant of Thy glory. O Prime M\^ay\^a! Thou art the Eternal Mah\^a M\^ay\^a! It is Thou that dost want to play this worldplay, and for that purpose hast created me as Brahm\^a. Then I created these four sorts of beings, engendered by heat and moisture (said of insects and worms), those that are oviparous, those that are sprung from germs or shoots, and those that are born from womb, viviparous and exhibit my pride ``That I am omniscient'' So forgive this sin of mine, this my pride.

32-37. O Mother ! Those ignorant persons blinded by passion, who take recourse to the eight-fold Yoga and Sam\^adhi and labour under it, do not know for certain, they would get Moksa, if they utter Thy name, even under a pretext. O Bhav\^an\^i! are they not deluded by error and blinded by passion for this world, who discriminate only the Tattvas (essences) and forget Thy name? For it is Thou that dost give Mukti from this world. O Thou Unborn! Can Hari, Hara, etc., and other ancient persons who have realised the highest Truth, forget, even for a second Thy holy character and Thy names \'Siva, Ambik\^a, \'Sakti, Isvar\^i and others? Canst Thou not create, by Thy glance merely, this fourfold creation? In fact, for mere recreation and will, it is Thou that hast made me as a Creator from the earliest times. Is it not that Thou didst save Hari in the ocean from the two Daityas Madhu and Kaitabha? Is it not again the fact that Thou destroyest Hara even who is the great destroyer, when Thou dissolvest the creation? Otherwise why is it that Hara becomes born from my eye-brows at the time of fresh creation? So Hari is not the Preserver of all. Hara is not the Destroyer of all. Had they been such, why would they be preserved and destroyed respectively by Thee? So Thou alone art the Creatrix and Preservatrix of all. O Bhav\^an\^i; no one has heard of or seen Thee taking birth; nobody knows whence Thou art born. Thou art, indeed, the One and only \'Sakti! Only the four Vedas can make one understand Thy Nature. O Mother! It is only by Thy help that I am able to create this creation; Hari, to preserve; and Hara, to destroy.

Without Thy aid, we are able to do nothing. There is nobody, in this world, born or that was born or that will be born, who does not become doubtful as we are. This Thine wondrously variegated Universe, full of Thy L\^il\^a, consisting in variety, is the common ground of dispute of the imperfect intellects; who are not deluded here! In this Sams\^ar\^a, full

of things, visible and invisible, there is another one who is more ancient than Thee; there is another Highest Person who is Thy substratum. If it be argued nicely, it will be seen that there is no other third Person that can be proved as far as evidences or proofs go to measure it. The wise persons, knowing all the laws, declare that there is the One God attributeless, inactive, without any object in view, without any up\^adhis or adjunct without any parts, who is the witness of Thy widely extended Leel\^a ``One alone exists; and that is Brahm\^an, and there is nothing else.'' This is the saying of the Vedas. Now I feel in my mind a doubt as to the discrepancy with this Veda saying. I cannot say that the Veda is false. So I ask Thee :-- Art Thou the Brahm\^an, the one and the secondless that is mentioned in the Vedas? or Is the other Person Brahm\^a? Kindly solve this doubt of mine. My mind is not completely free from doubts; this little mind is still discussing whether the Reality is dual or one; I cannot solve myself. So dost Thou say from Thy mouth and cut my doubts asunder. Whether Thou art male or female, describe in detail to me. So that, knowing the Highest \'Sakti, I be freed from this ocean Sams\^ara.

Thus ends the fifth chapter of the Third Skandha on the chanting of hymns by Hara and Brahm\^a in the Mah\^a Pur\^anam \'Sr\^imad Dev\^i Bh\^agavatam of 18,000 verses by Maharsi Veda Vy\^asa.