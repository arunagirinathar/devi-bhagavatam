\chapter{On the description of the greatness of the Dev\^i}

1-8. The king Suratha said :-- ``O Muni! This Vai\'sya is now become a friend of mine in this forest; he has been expelled from his home by his sons and wife and he has come lately here in this forest. He is now suffering very much from the bereavement of his family and has become very much troubled in his mind. He is not getting any peace whatsoever. I am also become like him and have become very distressed owing to my kingdom being robbed away. This thought, though really devoid of any substantial cause, is not leaving my heart now. Oh! My elephants and horses, now under my enemies, have become weak; My servants are suffering very much owing to my absence! My enemies will, within no time squander away forcibly all my hoarded riches. This thought is not giving me any happiness; nay, I cannot get any sleep owing to this care and anxiety. O Lord! I know that this world is false as a dream; yet my mind is so deluded that I cannot make me quiet. Who am I? What are those horses and elephants to me? They are not my brothers, sons, nor friends; yet I feel very much for them and am troubled with their troubles. O Muni! I know these all are delusions; still I am not able to make my mind free from them. This is very wonderful indeed! What is the cause of all this? O Lord! Nothing is veiled from your sight, you are fully able to solve all these doubts. Therefore, O Ocean of mercy! Kindly explain to me and this Vai\'sya the cause of all this delusion.''

9. Vy\^asa said :-- O King! When the King Suratha asked thus, the Muni in reply said to him the following words, full of wisdom, so that his delusion and sorrow might vanish.

10-25. The Muni said :-- ``O King! I am telling you the cause of bondage as well the cause of release of all the beings in this Universe. She is known as Mah\^a M\^ay\^a. She is the M\^ula Prakriti, the state of equilibrium of the three Gunas, S\^attva, R\^ajas and T\^amas. Even Brahm\^a, Visnu, Mahe\'svara, Indra, Varuna, V\^ayu, and the other Devas, Gandarbhas, Nagas, R\^aksasas, men, deer, animals, birds, trees and various kinds of creepers all are under M\^ay\^a; thus they are all bound; again they all get release when they are released by that M\^ay\^a. By Her is created all this world, moving and not moving, all the beings are caught in Her net and all are under the control of Her. You are a Ksattriya; so R\^ajoguna preponderates in you and your heart is thus rendered impure. She, by Her M\^ay\^a, deludes even the minds of those who are Jñ\^anins or wise; you are but an ordinary man compared to them. Even Brahm\^a, Visnu and Mahe\'sa, though possessed of vast wisdom, still roam, under the sway of M\^ay\^a, in the three worlds completely deluded by their attachments to the sensual objects. O King! In the Satya Yuga, in ancient times Visnu N\^ar\^ayana himself performed a very hard tapasy\^a in \'Svetadv\^ipa. He passed away full ten thousand years in meditation, with the object of attaining the unbroken everlasting Bliss and becoming steadfastly attached to Brahm\^a Vidy\^a. O King! Brahm\^a, too, became engaged in performing a tapasy\^a, meditating the Primordial Force, \^Ady\^a \'Sakti, in a very wonderful solitary place for the cessation of delusion. Once on a time V\^asudeva Hari wanted to go to another place; he got up and started to see other places. Brahm\^a, also, left his place and started for another destination. When they met each other in their way, each one asked the other, ``Who are you?'' The Praj\^apati answered :-- ``I am the Creator Brahm\^a.'' Hearing thus the Brahm\^a's words, Visnu said :-- ``O You Stupid! I am Achyuta Visnu; therefore I am the Creator of this world. You are inferior to Me as there is so much of R\^ajoguna in preponderance in you. Know Me as the eternal V\^asudeva, preponderating in S\^attva Guna. Do you not remember that I fought a dreadful battle for you and thus saved your life a short while ago, I slew the two D\^anavas Madhu and Kaitava when you were much distressed by them and took My refuge. How then do you boast now! O Fool! Quit your this vain boasting now. In this wide world, there is none superior to Me.''

26-31. The Risi said :-- Thus engaged in disputing with each other, their lips were quivering with anger and their eyes got red. When, Behold! there appeared suddenly between those two disputants, a nectar-like white phallic emblem (Lingam), wonderfully long and extensive. Then a voice, from without anybody, broke out in the Heavens and addressed Brahm\^a and Visnu who were quarrelling thus! Whoever amongst you will be able to go to the other end of this Lingam whether beyond its top or below its bottom, he is certainly the superior of you two; let one of you therefore go down to P\^at\^ala and let the other go up to the Heavens. Leave off your useless disputations and take my word as proof. It is always advisable to select an umpire to decide such a quarrel as this that has sprung up between you two.

32-39. The Risi said :-- O King! Hearing thus the divine word, both of them became ready and began energetically to measure the length of the wonderful Lingam that stood in front of them. Visnu went down to P\^at\^ala and Brahm\^a went up to \^Ak\^a\'sa to measure the Lingam and thus to ascertain their superiority. Going down some distance Visnu got tired and doing his best, when he could not find out the end of the Lingam, he returned and remained at the desired meeting place. On the other hand, Brahm\^a was ascending to the skies when he got one Ketak\^i flower dropping from the head of the Lingam. He became over glad and returned also to the desired meeting place. Brahm\^a became very much elated with vanity and when he returned, he at once showed that flower to Visnu and spoke thus the false words :-- ``O Visnu! This Ketak\^i flower has been obtained from the head of the Lingam. I have brought this to you simply that you would recognise it and be convinced in your heart.'' Hearing these words of Brahm\^a, Visnu saw the Ketak\^i flower and said :-- ``O Brahm\^a! Who is your witness in this matter? He whose words are true, who is equal to all, who is intelligent, pure, and always of good conduct, he can be the witness in such matters of dispute.''

40-44. Brahm\^a said :-- ``Who will come now as witness from that far off place? This Ketak\^i flower is the witness; this will give evidence.'' Thus saying, Brahm\^a requested Ketak\^i to give evidence; Ketak\^i soon replied thus to convince Visnu. O Visnu! I was on the head of Mah\^adeva; Brahm\^a has brought me from there down to this place; you ought not therefore to have any doubt on this point. My word is the evidence; Brahm\^a has gone to the other end of the Lingam. Some devotee of \'Siva put me on His head and Brahm\^a has got me down from there. Hearing thus the words of Ketak\^i, Visnu was very much astonished and said this :-- ``I cannot trust your word; if Mah\^a Deva comes and speaks this Himself, then I can trust and take it as a proof.''

45-53. The Risi said :-- O King! The eternal Mah\^a Deva, hearing the words of Visnu, spoke thus to Ketak\^i with great anger, ``O Liar! Do not utter such false words; You dropped down from My head and Brahm\^a while ascending up, picked you up on the way. Now as you have told a lie, I will never take you; you are henceforth forsaken by Me.'' Brahm\^a was then very much put to shame; he bowed down to Visnu; Mah\^a Deva, forsook the Ketak\^i flower from that date. O King! Such is the power of M\^ay\^a; when Brahm\^a, Visnu and other wise persons are so self-deluded by Her, what need to speak of other ordinary mortals! See! Visnu, the Lord of Laksm\^i, is self-deluded and is always deceiving the Daityas for the welfare of the Devas, without any fear whatsoever of the sin that he is thereby incurring. Though He is the Lord of all yet He has to take several incarnations in several wombs, forsaking the pleasures of the Heavens and fighting with the Daityas. O King! Visnu is omniscient and He is the Lord of this world; specially He is the only One, Supreme in the creation of the Gods. Now when M\^ay\^a exercises such a powerful influence on Visnu, what wonder is there that the other ordinary beings would be deluded by Her? O King! That Highest Prakriti draws away violently the hearts of the wise and drags them down into the ocean of world. That Omnipresent Bhagavat\^i is ever the cause of bondage of all when She casts Her net of delusion and She is again ever the cause of liberation when She imparts Her knowledge to them.

54. The King said :-- O Br\^ahman! What is the nature of Her? and what is the Supreme Force? What is the Cause of this creation? And where is Her highest place? Kindly narrate all these to me.

55-66. The Risi said :-- O King! She is beginningless; therefore She had no origin at any time; that Highest Dev\^i is Eternal and She is always the Cause of all Causes. (How then can any other be powerful like Her). O King! She resides in all the beings as the essential vital Force; deprived of that Force, every being is reduced to a dead carcass. She is pervading as the Universal Force of Consciousness in all the beings. The form of this \'Sakti (Force) is the form made up of consciousness itself, the Brahm\^a. (For the force of Fire is Fire itself; it is not seen in any other form). Her appearances and disappearances at times are simply for serving the purposes of the Gods. O King! Whenever the Devas and men worship Her, Ambik\^a makes Her appearance visible to destroy their pains and sufferings. She assumes various forms and possesses various powers. That Highest Î\'svar\^i comes down of Her free will to serve Her some purpose or other. She is not like the Devas, under the control of Daiva or Fate; She is not under the influence of Time (as both Fate and Time are created by Her). She puts always every being to action according to his capacity. Purusa is not the Doer; He is simply the Witness. This whole Universe is the object seen. That Dev\^i is the Mother of all this that is witnessed. She is the Manifested and She is the Unmanifested and She is the Effect also. She alone is the Actress and manifests thus the world and thus gives the colouring to the Purusa. When the Purusa is coloured thus, She destroys quickly these worlds. It is said that Brahm\^a, Visnu and Mahe\'sa are respectively the Creator, Preserver and Destroyer of the world; but this is merely a statement; really they are merely instruments in Her hands. Bhagavat\^i has created them in reality for Her Pastime and stationed them in their respective posts. She has bestowed to them Her part manifestations, i.e., Sarasvat\^i to Brahm\^a, Laksm\^i to Visnu, and Girij\^a to Mahe\'sa and has thus rendered them more powerful. They, the lords of the Devas, always meditate and worship Her as the Creatrix, Preservrix and Destructrix of this Universe. O King! I have thus described to you, as far as my intelligence and knowledge go, the holy greatness and the excellent glory of Her (in reality, I have not been able to come to the end of it).

``Aim Hr\^im Kl\^im Ch\^amund\^ayai bichche'' is the (9) nine lettered mantra.

Here ends the Thirty-third Chapter of the Fifth Book on the description of the greatness of the Dev\^i in \'Sr\^i Mad Dev\^i Bh\^agavatam, the Mah\^a Pur\^anam, of 18,000 verses by Maharsi Veda Vy\^asa.