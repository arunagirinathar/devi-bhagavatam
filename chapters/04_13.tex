\chapter{On cheating the Daityas}

1. The king said :-- What did the intelligent Brihaspat\^i do after he had assumed falsely the appearance of \'Sukr\^ach\^arya, and lived there as the spiritual guide of the Demons.

2. O Muni! Brihaspat\^i is the Guru of the Devas; he also devotes his time in studying the Vedas; and is the ocean of all knowledge; he is the son of the Maharsi Angir\^a and he is himself a Muni. Endorsed with all these good qualifications, how could he deceive the Demons.

3-4. In all the religious \'S\^astras, Truth is declared to be the essence of Dharma; and the Supreme Self is attained through Truth, so the wise sages say. How can we expect an ordinary householder to be true when such a man as Brihaspat\^i takes recourse to falsehood with the Demons.

5. If one acquires, as one's wealth, the whole Universe, still one does not require anything more than what is required in feeding one's belly; how is it that Brihaspat\^i could speak falsehood merely for the sake of his belly?

6. O Muni! The words sung by the ancient virtuous and respectable sages were true and had their corresponding objects denoted by those words; now they employed the term \'Sista meaning that there were virtuous, respectable persons as denoted by them. When Brihaspat\^i can even commit such condemnable deceitful acts and speak falsehood, we can expect no virtuous respectable persons in the world. Where then do you find the \'Sista persons, denoted by the word \'Sista, sung by the ancient sages? The word \'Sista is now meaningless!

7. The Devas are sprung from the S\^attvic qualities, men from R\^ajasic qualities and birds, etc. from the T\^amasic qualities.

8. When the Guru of the Immortals, the incarnate of S\^attvic qualities, can become a liar, how can one expect those who are R\^ajasic or T\^amasic to follow rigorously the truth?

9. Oh! This Trilok\^i is all pierced with falsehood! Where is the Religion! And what will be the ultimate goal of all these creatures!

10. When Bhagav\^an Har\^i, Brahm\^a, Indra and the best of the Devas when all can betake to pretext, fraud and trickery and show cleverness in them; what to speak of men!

11-12. O Giver of honour! When all the Devas, Va\'sistha, V\^amadeva, Vi\'svamitr\^a, Brihaspat\^i and other ascetic Munis get themselves overpowered by lust and anger, when their intelligence gets destroyed by covetousness and avarice, when they are addicted to vices and are expert in fraud, pre-text and trickery, then what fate, alas! can you expect of Dharma and what help is there of any religious persons!

13. Alas! lndra, Agni, Moon, and Brahm\^a when these get overpowered by the strong influence of lust, are in illicit love with other\'s wives, where is the goodness and virtuous behaviour in this Trilok\^i?

14. O Sinless One! To whom, then, can we look upon as our spiritual guide and our advice and law givers when all the Devas and Munis are corrupt with avarice?

15. Vy\^asa said :-- O king! Be he Indra, Brihaspat\^i, Brahm\^a, Visnu or Mahe\'sa, whoever is embodied or will put on bodies, he will have to be in touch with the previously mentioned Ahamk\^ara, and covetousness and other vices due to name and form.

16. O king! Brahm\^a, Visnu and Mahe\'sa are all attached to sensual objects; and what improper and sinful actions there can be that cannot be committed by persons devoted to sensual objects!

17. It is through cleverness and fraud that any one devoted to sensual objects can easily make oneself as cleverly free from M\^ay\^a; but when difficulty comes, then his trickery gets discovered and the respective qualities hidden in him are brought to bear their respective results. Know, then, the three qualities to be conjointly the cause of all these actions; as without any cause, no action gets visible.

18. These discrepancies in the case of Brahm\^a and others are caused by the three qualities; their bodies are all created from Pradh\^an Mahat and the other 25 Tattvas (essences).

19-20. O king! Brahm\^a and others are subject to death; then how can you doubt on other things? In advising others, everybody gives, as it were, good and virtuous advices; but the burden falls upon their own heads; they fall off from their advices and act according to their hidden natures; then they yield to lust, anger, envy, egoism and fascination.

21. No one who is embodied can get rid of passions, born of the 3 qualities. O king! Thus the Trilok\^i goes, is the saying of the Maharsis.

22-29. This Trilok\^i, auspicious, inauspicious, mixed, never gets any serious change; its nature remains always uniform. See Bhagav\^an Visnu sometimes practises severe asceticism; Indra, the lord of the Devas sometimes follows the practices of religious sacrifices. Again you find Visnu Bhagav\^an, full of youth, fond of the Leela, enjoying the company of Ram\^a in Vaikuntha; sometimes He is the ocean of mercy, is fighting dreadful battles with the Demons and being severely afflicted with their clusters of arrows; sometimes he gains victories, sometimes he gets defeat through the irony of Fate; thus he gets undoubtedly pleasures and pains. O king! some time N\^ar\^ayana draws all the worlds into his belly and takes his yogic sleep on the thousand headed serpent \'Se\'sa and again he gets himself awakened by Prakriti. O king! Brahm\^a, Visnu, Mahe\'sa, Indra, the Devas, and Munis all of them, live up to the limit of their ordained time and when the time of Pralaya, the Universal dissolution, ends, this whole Universe, moving and nonmoving, again comes into existence as before; there is no doubt in this. O king, at the expiry of the ordained time, Brahm\^a and all others will die , no doubt.

30-31. Again, in due course, Brahm\^a, Visnu, and Mahe\'sa and the other Devas come out and assume bodies and get all the passions, lust, etc., as ordained. O King! You need not be astonished; this Trilok\^i always goes on accompanied by lust, anger, etc.

32-34. Persons free from lust, anger and other passions are very rare in this world. He who is afraid of this world does not marry, and thus being free from the attachments to any worldly object, becomes free and roams fearless. The Moon stole away the wife of Brihaspat\^i, and Brihaspat\^i himself stole away the wife of his younger brother. Thus in this wheel of Sams\^ara, all the creatures are ever passioned with attachment, avarice, etc.

35. The householder can never expect to obtain freedom. Therefore those who want to be free, should carefully relinquish the idea of the stability of the world and worship the Eternal Mother Full and Sat, Chit and \^Anandam.

36. This world, moving and unmoving, O Mahe\'s\^an\^i, rolls in madness, overpowered by Her M\^ay\^a.

37. Intelligent persons worshipping Her, trample down the three qualities and become free. O king! No other Path exists for Freedom.

38-39. Until one gets the Grace from the Mahe\'s\^an\^i, one never gets happiness. True mercy is not found anywhere else but from Her. Then one should worship the All merciful, being of pure heart. For Her worship leads to freedom, even in this body-hood.

40. He who getting a human frame fails to worship Mahe\'s\^an\^i, gets down from the highest rung of the ladder. This is my opinion.

41-42. This Universe, composed of the three qualities, is encompassed with Ahamk\^ara and fastened to untruth; therefore freedom can never be expected without the worship of That Potent Goddess, O Muni! O king! Renounce every worldly object and serve the Goddess Bhuvane\'svar\^i; this is the highest duty of all.

43. The king said :-- What did, then, the Devaguru do in the disguise of \'Sukr\^ach\^arya? And when did the real \'Sukr\^ach\^arya come there? O respected Muni! Speak on these points.

44. Vy\^asa said :-- Please hear what the disguised Brihaspat\^i in the shape of \'Sukr\^ach\^arya did afterwards.

45. The demons were made to understand clearly by Brihaspat\^i; and then they took him for \'Sukr\^ach\^arya and placed implicit faith on him and began to think of him and him alone.

46. The Daityas, enchanted and deceived by the magic of Brihaspat\^i, took now his refuge for acquiring the knowledge from him, since they mistook him for \'Sukr\^ach\^arya. Who is there that is not enchanted by the idea of gaining something?

47. On the other hand, when the term of ten years was over, \'Sukr\^ach\^arya, the real Guru of the Daityas, ceased enjoying Jayant\^i and began to remember his disciples, the Daityas.

48. He now began to think that ``My disciples, the Daityas, are expecting every instant my return; and I would now go and see them, bewildered with fear.

49-51. They are my devotees and I ought to do such that they might not be afraid of the Devas.'' And then he exclaimed to Jayant\^i, ``O beautiful one! Let my sons take the shelter of the Gods; your term of ten years is today over; I now go therefore, to see my disciples; soon I will again come to you.''

52. ``Be it so'', replied Jayant\^i, the best of those who know religion, ``you can go where you like; I am not to destroy your Dharma.''

53-54. Hearing these words, \'Sukr\^ach\^arya went hurriedly to the Demons and saw the Devaguru Brihaspat\^i sitting before them in the guise of \'Sukr\^ach\^arya. He was explaining to them the Jaina doctrines, compiled by himself and finding fault with the act of envy, taking revenge and killing and cursing the sacrifices, etc.

55. He was telling them ``O Enemies of Gods! Truly, I am telling you words that will, no doubt, prove good to you. Non-killing is the highest virtue; even the enemies ought never to be killed.

56. It is the Br\^ahmanas, addicted to enjoyments and pleasures of the senses, who want to satisfy their tastes and pleasures that are found in the Veda's injunctions to kill animals; but there is no virtue higher than non-killing animals.''

57-58. O king! \'Sukr\^ach\^arya was perfectly astonished to hear Brihaspat\^i, the Guru of the Devas, speaking against the Vedas and began to think that Brihaspat\^i is certainly my enemy. My disciples have been duped by this cheat; there is no doubt in this.

59. Fie to Avarice! It is the seed of sin; very strong and the veritable gate to hell; Brihaspat\^i, even, the Guru of the Devas, is speaking lies, bound under the influence of this heinous avarice!

60. Oh! What wonder is this that the Guru of the Devas, who is the promulgator of all the religious \'S\^astras and whose word is accepted as the final decision, is now expounding the doctrines of atheists.

61. When Brihaspat\^i can become the expounder of atheistic doctrines, impelled by covetousness what to speak of those whose minds are not pure and whose intelligence is not sharp?

62. This Deva Guru, though a Br\^ahmin, is acting today like a rogue, wanting to take away all and is deceiving my disciples the Daityas, who have been confounded by his magic.

Here ends the Thirteenth Chapter in the Fourth Book of \'Sr\^i Mad Dev\^i Bh\^agavatam, the Mah\^apur\^anam of 18,000 verses on cheating the Daityas by Maharsi Veda Vy\^asa.