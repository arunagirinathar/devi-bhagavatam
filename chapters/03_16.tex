\chapter{On the glory of the Dev\^i}

1. Vy\^asa said :-- After gaining the victory in the battle, the king Yudh\^ajit returned to the city of Ayodhy\^a with his huge army and asked where are Sudar\'sana and Manoram\^a? He wanted to kill Sudar\'sana.

2. He repeatedly exclaimed ``Where are they gone?'' and sent his servants

on their search. Then, on an auspicious day, he installed his daughter's son on the throne.

3. Maharsi Va\'sistha was engaged as the priest; he and the other ministers began to chant the auspicious hymns of the Atharvaveda and with the jars, filled with water, and consecrated by these hymns, installed \'Satrûjit on the throne.

4. O best of the Kurus! Conch shells resounded; drums, bher\^is and tûriyas, resounded; and great festivals and rejoicings took place in the city.

5. The reciting of the Vedic mantrams by the Br\^ahmans, the chanting of hymns by the bards and the auspicious acclamation of Victory to the new king resounded the whole city of Ayodhy\^a with joy.

6. When the new king \'Satrûjit ascended on the throne, the subjects were filled with joy; everywhere hymns were sung; drums were resound. At this Ayodhy\^a looked as fresh as ever.

7-8. O King! Though there were thus great rejoicings and festivals, yet some good persons were found that remembered Sudar\'sana and gave vent to this feeling of sorrow, thus :-- ``Alas! Where has that prince gone? Where has that chaste queen Manoram\^a gone with her son? Oh! The enemies have killed his father for greed of kingdom.''

9. The saints with their views impartial, thus rendered uneasy, sorry, began to pass away their time there subject to \'Satrûjit.

10. After installing duly his daughter's son on the throne and after having made over the charge of the kingdom to the wise councillors, Yudh\^ajit turned towards his own city.

11-12. Afterwards Yudh\^ajit heard that Sudar\'sana was staying in the hermitage with the Munis. He started at once for Chitrakûta and went quickly to Durdar\'sa, the chief of the city of Sringavera, being accompanied by Bala, the chief of the N\^is\^adas.

13-15. Hearing that Yudh\^ajit was coming there with his army, Manoram\^a began to think that his son was a minor and became very sorrowful, distressed and terrified. Then she with tears in her eyes addressed the Muni thus :-- ``Yudh\^ajit is coming here; what shall I do and whither shall I go? He has slain my father, and has installed his daughter's son on the throne. Still he is not satisfied and he is now coming with his army here to kill my minor child.''

16-21. O Lord! In days of yore, I heard that the P\^andavas, when they went to the forest, lived in the holy hermitage of the Munis with Draupad\^i. One day the five brethren went a hunting, and the beautiful Draupad\^i stayed without any fear with other maid servants in the

hermitage where there was the chanting of the Vedas by Dhaumya, Attri, G\^alava, Paila, J\^av\^ali, Gautama, Bhrigu, Chyavana, Kanva of the Atrigotra, Jatu, Kratu, V\^itihotra, Sumantu, Yaj\~nadatt, Vatsala, R\^a\'s\^asana, Kahoda, Yavakr\^i, Yaj\~nakrit, Kratu and other holy high souled Risis like Bh\^aradv\^aja and others.

22-23. While the five great heroes Arjuna and others, the destroyers of their enemies, were roaming in the forests, Jayadratha, the king of Sindhu came with his army to the hermitage, hearing the reciting of the Vedic hymns.

24. Hearing thus, that king quickly descended from the chariot so that he might have a sight of these holy maharsis.

25-27. Accompanied by two attendants only, he approached to the Munis and, finding them engaged in the study of the Vedas, waited there with folded hands for an opportunity. O Lord! When the king Jayadratha on entering the hermitage took his seat, the wives of the Munis came there to see the king and began to enquire ``Who is this person?''

28. With the wives of the Munis came there also the beautiful Draupad\^i. Jayadratha looked upon Draupad\^i as if she were the second goddess Laksm\^i.

29-30. Looking at that lovely royal daughter who looked like the Deva girls Jayadratha asked the Maharsi Dhaumya, ``Who is this beautiful lotus eyed lady? Whose wife is she and who is her father? What is her name? Oh! From her beautiful appearance it seems that the goddess \'Sach\^i has come down on earth.''

31. This fair woman is shining like the celestial nymph Rambh\^a surrounded by the Raksasis or like the beautiful creeper Lavangalatika encircled by thorny trees.

32. O good ones! Tell truly whose beloved is she? O Br\^ahmins! It seems that she is the wife of some king, not the wife of a Muni.

33. Dhaumya said :-- ``O king of Sindhu! She is the daughter of P\^anch\^ala; her name is Draupad\^i; she is the wife of the P\^andavas; they are residing in this forest, having got rid of their fears.''

34. Jayadratha said ``Where have those powerful P\^andavas of great prowess gone now? Are they dwelling in this forest, free from fears?''

35. Dhaumya said :-- ``The five P\^andavas have gone out on hunting, ascended on a chariot. They will return at noon with their game.''

36-37. Hearing the Muni's words Jayadratha got up, and going near to Draupad\^i, bowed down to her and said :-- ``O Fair One! Is there every-

thing well with you? Where have your husbands gone? To-day it is eleven years that you are residing in the forest.''

38. Draupad\^i then said :-- ``O prince! Let all be well with you, wait here for a short while; the P\^andavas are coming quickly.''

39. While Draupad\^i thus spoke, that powerful king, being overpowered with greed and avarice, stole her away, disregarding all the Munis present there.

40-42. O Lord! The wise should never trust any body; if on any body he places his trust, he will surely come to grief. For example, see the case of the king Bali. Bali, the son of Virochana, and the grandson of Prahl\^ada, was prosperous, devoted to his religion, true to his promise, performer of sacrifices, generous, always giving protection to and liked by the saints and a great warrior. His mind never turned to any irreligious subject and he performed ninety nine Yaj\~nas with full Daksin\^as (remunerations).

43-44. But the Bhagav\^an Visnu, who is all full of S\^attvic purity and who is never affected with passions and changeless, who is always worshipped by the Yogis, He, in the form of a dwarf in his V\^amana incarnation as the son of Ka\'syapa Risi, to serve the Devas, stole away his whole seagirt earth and kingdom deceitfully on hypocritical pretext.

45. O Lord! I heard that the son of Virochana was a generous large hearted king. He truly resolved to give what was wanted; but Visnu behaved with him deceitfully to serve the cause of Indra.

46. When the pure, S\^attvik Visnu could assume this dwarf incarnation to bring about the hindrance to Bali's Yaj\~na, what wonder is that other ordinary mortals would practise things like that?

47. Therefore never trust on any body in any way. Lord! Where there are greed and avarice, reigning in one's heart, what fear can he have to perpetrate any evil deed?

48-49. O Muni! It is through avarice that men commit sinful deeds; they do not care what good or bad will happen to them in the next world. Thoroughly overpowered by greed, they take away in mind, word and deed other's things; and thus they become fallen.

50-51. Lo! Human beings always worship the Gods for wealth; but the Devas do not give them wealth instantly; they give them these things through others by making them carry on trade, make gifts, or shew their strength or by making them steal.

52. The Vai\'syas worship the Gods simply because they think they will be highly prosperous and therefore they sell many things as grains, cloth and the like.

53. O Controlled one! Is there not the desire to take away the other's property in this act of merchandise? Certainly there is. Besides the merchants, when they find that when people are in urgent need of buying articles from them, expect that the price of those articles might run higher.

54. O Muni! Thus every one is anxious to take away other's properties. How, then, can we trust them?

55. Those who are clouded by greed and delusion, their going to places of pilgrimages, their making charities, their reciting the Vedas, all are rendered useless. Though they go to the holy places, etc., still these things bear no fruits to them, as if they have not done these things at all.

56. Therefore O Enlightened one! You make Yudh\^ajit go back to his own place. Then I will be able to remain here, like S\^it\^a, with my son.

57-58. On Manoram\^a's thus speaking to the Muni, the fiery Maharsi went to Yudh\^ajit and said :-- ``O King! You better go back to your own place or anywhere else you like. The son of Manoram\^a is a minor; that queen is very much grieved; she cannot come to you now.''

59. Yudh\^ajit said ``O peaceful ones! Kindly cease showing this impudence and give me Manoram\^a. I will never go away leaving her. If you do not give her easily, I will take her away by force.''

60. The Risis said ``O King! If there be any strength in you, you can take away Manoram\^a by force; but the result will be similar to that when the King Visv\^amitra wanted to take away the heavenly cow by force from the hermitage of Va\'sistha.''

Thus ends the Sixteenth Chapter on the glory of the Dev\^i and the going of the King Yudh\^ajit to the hermitage of Bh\^aradv\^aja, to kill Sudar\'sana, in the the 3rd Adhy\^aya of \'Sr\^i Mad Dev\^i Bh\^agavatam by Maharsi Veda Vy\^asa.

