\chapter{On the killing of the sons of Devak\^i}

1. Vy\^asa said :-- O King! Thus, in due course, Devak\^i, the goddess incarnate, being united according to rules with V\^asudeva, became pregnant.

2-4. When full ten months were over, a good-looking and beautiful child (male) was first born to Devak\^i. Then the good-natured V\^asudeva remembered his promise, and also what is ordained by Heaven; and he spoke to Devak\^i, the part incarnate of Aditi ``O fair-looking! You know that I saved your life at your marriage by swearing on oath to deliver all your fresh-born children to Kamsa. Now has come the time to hand over your child to Kamsa.''

5. O good-haired woman! Now I will hand your this son to Kamsa. Know Kamsa is very cruel and wicked. I cannot say what step he will take, urged on by Fate, to kill your child. O Sweet One! We have no hands in the matter. The effects of Karma are exceedingly puzzling. Ordinary persons cannot know them.

6. All persons are subject to Time, the Destroyer, and enjoy the merits or demerits of their past deeds. The effects of past Karma are fashioned by the Creator; knowing this, allow me to take away your child.

7. Devak\^i said :-- O Lord! Certainly men have to suffer fully the effects of their past Karmas. But can that not be upset by residing in holy places, practising penances and asceticism or by making generous gifts?

8-9. The high minded Maharsis have fixed rules and penances for destroying the sins of past deeds; twelve year\'s vow to observe penances can purify one from sins, e.g., Br\^ahminicide, stealing gold, drinking or stealing the wife of one's preceptor and many others.

10-11. O Sinless one! Will not any one be freed from their sinful effects, if they observe practices and penances as are ordained by Manu or other Munis? If you don't accept penances to be sufficiently purificatory, do you mean to say, then, that what the Maharsis, seers, Y\^ajñavalkya and other promulgators of religious doctrines have uttered, they did so, as an act of dire falsehood and villainy?

12. O My husband! ``What is in the womb of Fate will surely come to pass,'' if this be taken as granted, then the whole Ayurveda (medicinal books) and Mantra v\^adas, the science and recitation of mantras or sacred formulae turn out utterly fruitless and false!

13-16. If all the actions are under the control of Fate, then no effect can come out of any effort; so all efforts are reduced to no effect. If what is ordained by Heaven is to come to pass then what is the use of taking recourse to any action and Agnistoma sacrifices, etc., that are declared to lead to Heavens. Judge! If Heaven or Fate you consider all-in-all, then the whole Vedas, the revelations from God's mouth turn out false; if the Vedas be false, then there is no reason why the whole Dharma will not be destroyed.

N.B. :-- Fate is here denounced.

17. Now when it is seen that effects do come out whenever any exertion is made, then you ought to think out carefully and find out some means to avert danger. Therefore do you judge and find out a good way of preserving the life of this new born baby.

The learned people say that to tell a lie is not a sin, if you can thereby save a life, and have an honest motive for the welfare of all.

Note :-- Here is a diplomatic statement!

18. V\^asudeva said :-- O blessed one! I now tell you what is truth and the matters connected with truth.

19-20. Effort, application and manifestation of energy are certainly the duties of man; but their effects are all under the Great Destiny or Fate.

The Pundits knowing the ancient lore say that there are three kinds of Karma mentioned in the Pur\^anas and \^Agamas :-- First, the Sanchita Karma (done in past births); the Pr\^arabdha Karma, the Karma already done; and the Vartam\^an Karma (Karma in hand).

21. The Karma, auspicious and inauspicious, done in many previous lives and preserved in seed forms, remains always inherent in a human soul. Urged on by this Karma, the J\^ivas quitting their previous bodies, enjoy Heaven or Hell as effects of these, their own acts.

22-23. According to their good or bad works, the J\^ivas acquire the higher happy body and enjoy various pleasures in the Heavens, or they take up very painful vicious bodies and suffer various pains in hell.

24-25. At the expiry of the above period in Heaven or Hell, when there comes the time of his assuming another body, the J\^iva becomes conscious of the subtle body (Linga Deha) and takes his birth again. When the Linga Deha comes into existence, the part of the Karma done in various previous births that are ripe and ready to yield their fruits, gets attached to the J\^iva by God (or Destiny).

26. Therefore the collective effect of Karma done in previous births always exists in a J\^iva's body. O Fair-eyed One! The effects of Pr\^arabdha Karma, ripened and ready to yield their fruits must have to be experienced by a J\^iva, whether happy or unhappy.

27. O beautiful young woman! Penances, performed according to rules, destroy the effect of Karmas that are in hand and are weak (i.e., not yet accumulated strongly as to remain in seed forms).

28. The Pr\^arabdha Karma, those acts out of all the previous acts done in previous births that are fully mature and ready to yield their fruits, cannot be averted; their effects must have to be experienced and then they can die away; they cannot be expiated by penances or any other remedial measures. Therefore you must hand over unconditionally your new born babe unto the hands of Kamsa.

29-30. O Goddess! I have never done any blameable action, nor have I told any lie. Therefore do you fulfill your truth and hand over your baby. O Devak\^i! Dharma is the only thing permanent and real in this fleeting world. Even the births and deaths of high souled persons are subject to the great Destiny. Therefore the J\^ivas ought not to be sorry when there is no help for it.

31. O dear one! What shall I say to you! Know this much that his life is spent in vain who is lost to Truth. O beautiful one! Whose this life is destroyed, what can he expect in the life to come!

32. Therefore, O Goddess! Give me your baby and I will hand it over to Kamsa. If we can observe this truth, we will meet with ample rewards afterwards; there is no doubt in it.

33. Where there are pains and pleasures for the J\^ivas, there it is highly incumbent on us to do good and meritorious deeds. If we can act according to Truth, we will certainly get good fruits.

34. Vy\^asa said :-- Thus addressed by V\^asudeva, the husband of Devak\^i, who was very much grieved and intelligent, gave over the newly born baby, her whole body trembling, to the hands of V\^asudeva.

35. The virtuous V\^asudeva took that baby and went out to the Kamsa's palace. On the way, the people, seeing him thus, were very much astonished and began to praise him.

36-37. The people said :-- ``O people! See how V\^asudeva is sensible to keep his words! He is taking his son to hand over to Kamsa. This truthful and highsouled man, free from malice, is going to give up his son to the hands of Kamsa who is the Death Personified. See his wonderful patience; this man's life is really high, noble and true.''

38. Vy\^asa said :-- O King! V\^asudeva, thus praised, reached at last the Kamsa's palace and handed over his newly-born son to Kamsa.

39-41. The King Kamsa, too, was very much astonished to see this wonderful patience of V\^asudeva. Then he held aloft the child and laughed and said :-- ``O son of \'Sûrasena, you have been blessed today by giving me your son just now. But the voice from Heaven said that your eighth son will be the cause of my death; this your first son is not my cause of death. Therefore I will not kill this baby; you can take your baby back to your home.

O High-minded One! Let me have your eighth son brought here, when he will be born; I hope you will positively do it.''

42. The cruel and wicked Kamsa returned the child and said :-- ``Let this child go back safely to his home.''

43-44. When the king Kamsa said thus, V\^asudeva, the son of \'Sûrasena gladly took his child back and came home. Then the King Kamsa told his ministers that the Heavenly voice told that the eighth son would he the cause of his death; and so there was no necessity to kill that child. There was no need to incur sin by killing the first child.

45. The ministers, hearing the king Kamsa's those words, began to praise him very much and exclaimed repeatedly ``Well done'' ``Well done.'' They went away to their respective homes, when ordered to do so by Kamsa.

46-49. Now N\^arada, the best of the Munis, arrived to Kamsa. The king Kamsa, the son of Ugrasena, stood up at once and offered him water to wash his mouth and with green grass and rice worshipped him devotedly and enquired of his welfare. He then asked the Muni about the cause of his untimely arrival there. The Maharsi N\^arada then smilingly and with sweet words repeatedly uttered ``Kamsa,'' ``Kamsa'' and then said, O blessed one! I went perchance to Sumeru Mountain. There Brahm\^a and other gods formed an assembly and were thus thinking out plans that Visnu, the Supreme God, would take His birth in the womb of Devak\^i, the wife of V\^asudeva to kill Kamsa.

50. Now I ask you, you are very expert as a politician; then why have you not killed the son of V\^asudeva? Kamsa said :-- ``I will kill the eighth son according to the Heavenly Voice.''

51. N\^arada said :-- O King! Now I understand that you do not understand anything of politics, leading to auspicious or inauspicious results; especially when you are quite ignorant of the M\^ay\^a of the Devas, then what shall I say to you!

52-53. The truth is this :-- The warriors, looking after their own welfare, never overlook the weakest of their foes. What have you understood when the Heavenly Voice uttered ``the eighth son.'' It means the children counted from the first and then finished upto eighth; it may mean first, second, third or upto eighth. Never forego your enemies; then why have you desisted in killing your enemy when you got that enemy in your possession. Nothing is shewn of you in this act save dire foolishness, and ignorance.

54. Thus saying, the Maharsi N\^arada vanished quickly. Kamsa, of little understanding brought back the son of V\^asudeva and killed him by dashing him against a stone and was relieved.

Note :-- This human body is a microcosm; the universe is the macrocosm. God resides in the centre and controls the two. In this human body also live the Devas and the D\^anavas. The left half of the body, the Îd\^a side, is the seat of the Devas. The right half, the Pingal\^a side, is the seat of the D\^anavas. In this body war is always going on between the Devas and D\^anavas. Sometimes the Devas get victory; sometimes the D\^anavas win. God is in the centre, the heart and controls the two.

Here ends the 21st Chapter of \'Sr\^i Mad Dev\^i Bh\^agavatam, the Mah\^a Pur\^anam of 18,000 verses composed by Veda Vy\^asa, on the killing of the sons of Devak\^i.