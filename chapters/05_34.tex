\chapter{On the methods of the worship of the Dev\^i}

1. The King said :-- O Bhagav\^an! Kindly narrate to me in detail now the methods how to serve and worship the Goddess and the Mantrams that are used on such occasions.

2-12. The Risi said :-- O King! I am now describing the method how to worship the Goddess. Hear. This leads to the fulfilment of all desires, to the liberation from one's bondage, to self-realisation and to the destruction of all miseries. The worshipper has to perform his bath; then putting on a white cloth, he will have to perform his Vaidik and T\^antrik Sandhy\^a; then he should, with his heart controlled, perform his \^Achamana ceremony and select a good auspicious site for his own Pooj\^a purposes. Next he should plaster the site with cow-dung and spread his sacred carpet (\^Asana) whereon he is to take his seat with a cheerful mind and sip water for \^Achamana three times. Then he is to collect the articles for worship according to his best capacity and place them duly in their respective positions. He is to perform Pr\^an\^ay\^ama (regulate his breath); and then follows the Bhuta-\'Suddhi, the purification of the old and the formation of the celestial body and J\^iva-\'Suddhi by which the S\^adhaka becomes the Devat\^a-maya; he then proceeds to M\^atrik\^a Ny\^asa (i.e., setting mentally in their several places in the six Chakras and then externally by physical action the letters of the alphabet which form the different parts of the body of the Devat\^a. He then places his hand on different parts of his body, uttering distinctly at the same time the appropriate M\^atrik\^a for that part).

Bhuta-\'Suddhi :-- Dissolve earth into water, water into fire, fire into air, air into ether; ether into Ahamk\^ara, Ahamk\^ara into Mahat and Mahat into Prakriti, the final Cause. This process is called Bhuta-\'Suddhi.

He is to mention then the time, date, tithi, and month of the year and make his Sankalpa; then he will have to assign to the different parts of his body the M\^atrik\^a Mantrams duly as well as his own Mantram; next he is to meditate in his own body the seat of the different Devat\^as and do the internal worship. He is to breathe life into the Deity outside to be worshipped as well as within himself to be meditated and worshipped; then he is to do the same with the articles for worship and purify them by sprinkling with water and Astra or Phat Mantram, thus removing all sources of obstacles that are likely to interfere with the act. Next, on an auspicious copper plate, he is to draw inside a six-angled (hexagonal) figure (two triangles crossing each other with their vertices one upward and the other below) with white sandal paste or with eight perfumed things and outside this figure, an octagonal figure of eight petals; outside this he is to draw the boundary lines that is called the Bh\^upura. On each of the eight petals he is to write each letter of the nine-lettered V\^ija (Seed) Mantram and the ninth letter in the central ovum. Next by the Mantram by which breath is infused or by the Vedic Mantram he will have to place the Yantra in the proper position and then worship the \^Adh\^ara \'Sakti (the vital Force) in the central ovum and the holy seat with the P\^itha Mantrams. He will have to invoke the Dev\^i, uttering the Seed Mantram over a golden plate or figure and carefully worship Her by offering seats and other articles duly as enunciated in the Y\^amala T\^antras, etc. Then he will have to perform the six-fold worship of the Ganas in the six angles and worship Indra, etc., and Vajra and others in the Bh\^upura (the boundary) and thus finish the Pooj\^a of the Yantra. (For the Pooj\^a see the Prapancha S\^ara.) Note :-- Bh\^upura is what is thought over outside front or in the beginning. Here the Gana Devat\^as are first thought over and worshipped. Worship outside, worship inside and See the Deity in and out, everywhere and be free is the motto of the worship. In the absence of the Yantra, one will have to make a metalled image of Bhagavat\^i and worship Her with the greatest caution with the Mantras as expounded by \'Siva in the Tantrams (of J\^amal\^a and others). Note :-- Yantra is that which restrains. This human body is the Yantra. And its imitation is placed outside in various shapes and figures. The Yantra is the mystical diagram used by the devotees for worship. Or one may use the Vaidik Mantrams in worshipping the Deity in accordance with the prescribed rules and with his mind controlled; then, merged in meditation, one is to mutter silently (perform the Japam of) the nine-lettered Mantram. (The Mantram is Kr\^im Daksine K\^alike Sv\^ah\^a). Japam (muttering or repeating silently the Mantram) is of two kinds :-- Nitya (daily) and Paura\'scharanik (repetition of the name of the deity accompanied with burnt offerings). In the Nitya Japam, Nitya Homas are performed and in the occasional Paura\'scharanik Japam, one tenth of this is offerred; Abhiseka, too, is one-tenth of this Homa; Tarpanam is one-tenth of Abhiseka and the feeding of the Br\^ahmanas is one-tenth of what is done in the Tarpanam. O King! Thus completing the Japam one is to read daily the Chand\^i (do the Chand\^ip\^atha) where the three glorious deeds of the Dev\^i are narrated; next he will have to allow the Deity invoked to depart to Her own place. The Navar\^atra Vrata (nine night vow) is next to be observed according to the proper rites and ceremonies. Hr\^im Mahisa Mardinyai Sv\^ah\^a is the Mantra.

13-31. In the bright fortnight of the month of \^Asvin or Chaitra, is to be observed the fasting of the Navar\^atra by those who desire for their own welfare. Homas are to be offered, many in number, and Mantrams are to be recited, the same as in one's own Mantram, good P\^ayasam with sugar, ghee, and honey mixed is to be offered in this ceremony. Goat meat, or holy leaves of the Bel tree, or red Karav\^ir flowers or til (sesamum seed) mixed with honey can be used instead in the Homa ceremony. The special days for the worship of the Dev\^i are the eighth, ninth, or fourteenth day (tithi) of the half month. The feeding of the Br\^ahmins must be done on each occasion. O King! Thus the poor become wealthy, the diseased get cured, and the persons that have no issue get obedient and well qualified sons. The King, expelled from his kingdom, gets back by the grace of Mah\^a M\^ay\^a, dominion over the whole earth and becomes able to destroy all those enemies of his, by whom he was before vanquished, when he worships the Dev\^i. The persons, desirous of learning, get undoubtedly the learning honourable and auspicious, provided he worships the Dev\^i with his senses restrained. Persons of all castes, Br\^ahmins, Ksatriyas, Vai\'syas or \'S\^udras can become masters of all pleasures and happiness provided they worship with devotion the Dev\^i, the Preserver of the World (the Jagaddh\^atr\^i). A man or woman whoever performs the Navar\^atra vow always full of devotion, gets all the desired fruits. Whoever celebrates the holy Navar\^atra ceremony in the bright fortnight of the month of \^A\'svin with his heart full of the thought of the Dev\^i, gets all his desired fruits. O King! Now I am describing the rites and ceremonies; here a square raised platform or altar is to be made according to the prescribed rules; a water-jar is then to be placed on it with the Vedic mantrams and due rites and ceremonies. One will have to make a beautiful Yantra according to the previously laid rules and the water-jar is to be placed on it; then spread the beautiful Yava grains all around the jar. An awning or pandal is to be erected over the altar and the place of worship, and the site is to be decorated with flowers. Lights and Dh\^upas, incense and perfumes are then to be used in the hall of the Chandik\^a Dev\^i. O King! The Dev\^i is to be worshipped thrice; morning, midday and evening; no miserliness is to be shown in spending wealth for this purpose. Light, dh\^up, good presents of rice and other edibles, flowers, and fruits of various kinds are to be offered in this worship of the Dev\^i; the chanting of the hymns of the Vedas, songs, and music with the various instruments are to be done and a grand festivity is to be made. Moreover, note this carefully that virgins are to be worshipped duly with sandal, ornaments, clothings, various edibles, sweet scented oil, and beautiful garlands. (This worship of the virgins is one of the essentials.) Thus completing the worship of the Dev\^i, Homa is to be done duly with Mantrams and other necessary articles on the eighth or the ninth tithi. Lastly the Br\^ahmins are to be fed duly; then the worshipper is to take his first meal after fasting (i.e., make p\^aranam) on the tenth day; then presents and various articles are to be offered to the Br\^ahmin, according to one's might and with devotion.

32-44. O King! Any man, or any chaste married woman or a chaste widow whoever performs thus the Navar\^atra Vrata gets in this world all the desired fruits and enjoys all sorts of enjoyments and gets unbounded happiness and after death goes to the highest place. And if, owing to some cause or other, he has to take his birth again in this world, he would be born in an excellent family and would become endowed with good conduct and qualifications and get the unflinching devotion towards the Ambik\^a Dev\^i. O King! I have thus described to you the rules of the Navar\^atra ceremony; this vow is the best of all; highest and greatest pleasures and happinesses are obtained in worshipping thus the auspicious Mah\^a M\^ay\^a. O King! Better worship Chandik\^a duly according to the prescribed rules; then you would be able, by Her grace, to conquer all your enemies and you will regain your excellent dominion, unshaken by any, and you will get again the highest pleasure and happiness when you will be reunited with your wife and sons in your own palace; there is no doubt in this. O Vai\'sya! You, too, better worship the same Mah\^a May\^a, the Goddess of the Universe, worshipping Whom leads to the fructification of all desires. You will then be able to regain all your worldly pleasures in your own home and be respected by your relatives and acquaintances and finally, after your death, you will go to the holy abode of the Dev\^i. There is no doubt in this. Those that do not worship the Dev\^i, go to Naraka or hell; moreover they suffer much from various diseases in this world. Those that do not worship the Dev\^i are always defeated by their enemies, are void of wife and sons, become stupid and suffer pains from their unsatisfied desires. And those that worship the Preservrix of this world with the Bel leaves, Karav\^ira flowers, \'Satapatra and Champaka flowers, that blessed man, devoted to the Dev\^i, gets filled with all sorts of enjoyments. O King! What more can I say than this, that those who have worshipped the Dev\^i Bhav\^an\^i with the Mantrams approved by the Nigama \'S\^astras, those very persons get honour in this world and are filled with all sorts of power and wealth. Verily, they stand foremost in the rank of best men, becoming the only repositories of all the best qualities in this world.

Here ends the Thirty-fourth Chapter of the Fifth Book on the methods of the worship of the Dev\^i in \'Sr\^i Mad Dev\^i Bh\^agavatam, the Mah\^a Pur\^anam of 18,000 verses by Maharsi Veda Vy\^asa.