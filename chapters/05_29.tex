\chapter{On the killing of Raktab\^ija}

1-21. Vy\^asa said :-- O King! Please hear attentively about the extraordinary boon that was given by Mah\^adeva, the God of gods, to the great warrior, Raktab\^ija. Whenever a drop of blood from the body of that great warrior will drop on the surface of the earth, immediately will arise innumerable D\^anavas, equal in form and power to him; thus the Deva Rudra granted the Demon the wonderful boon. Thus elated with the boon, he entered into the battlefield with great force in order to kill K\^alik\^a with Ambik\^a Dev\^i. Seeing the Vaisnav\^i \'Sakti, lotus-eyed, seated on the bird Garuda, the Demon struck Her with a violent weapon (named \'Sakti). She then baffled the weapon by Her club and hurled Sudar\'sana disc on the great Asura Raktab\^ija. Thus struck by the disc, blood began to ooze out from his body as the red stream of soft red sandstone comes out of a mountain-top. Wherever on the surface of the earth drops of blood fell from his body, then and there sprang out thousands and thousands of demons like him. Indr\^ani, the wife of Indra, became very angry and struck the terrible Raktab\^ija with his thunderbolt. Streams of blood then began to ooze out from his body. No sooner the drops of blood fell from the Demon's body, than were instantaneously born from the blood many powerful Asuras, of similar forms, having similar weapons and hard to be conquered in battle. Brahm\^an\^i then becoming enraged struck at him with the staff of Brahm\^a with greater force. M\^ahe\'svar\^i rent the D\^anava asunder by striking him with Her trident. N\^ara Simh\^i pierced the Asura with Her nails; V\^ar\^ah\^i struck at him with Her teeth. Then the D\^anava becoming angry shot at them all with sharpened arrows and pierced them all. Thus when the M\^atrik\^a Dev\^is were pierced by the club and other various weapons of that great Asura, they got very angry and pierced the D\^anavas in return with shots of arrows. Kaum\^ar\^i, too, struck at his breast with Her weapon, named \'Sakti. The D\^anavas then got angry and hurled on them multitude of arrows and began to pierce them. O King! The Chandik\^a Dev\^i, getting angry, cut off his weapons into pieces and shot violently at him other arrows. O King! Thus struck by severe blows, when blood began to flow in profuse quantities from his body, thousands and thousands of D\^anavas resembling Raktab\^ija sprang out instantly from it. So much so, that the heavens were all covered over with Raktab\^ijas that sprang up from the blood. They all covered all over their bodies with coats of armour, began to fight terribly with weapons in their hands. Then the Devas, seeing that the innumerable Raktav\^ijas were striking the Dev\^i, became very much frightened and were distressed with sorrow. They began to talk with each other with sorrowful countenances that thousands and thousands of huge bodied warriors were springing from the blood. These were all very powerful; so how could they be destroyed! In this battlefield there were now left only the M\^atrik\^as, K\^alik\^a, and Chandik\^a. It would be certainly extremely difficult for them to conquer all these D\^anavas. And if at that moment, \'Sumbha and Ni\'sumbha were to join them with his army, certainly a great catastrophe would occur.

22-28. Vy\^asa said :-- O King! When the Devas were thus extremely anxious, out of terror, Ambik\^a Dev\^i said to the lotus-eyed K\^al\^i :-- ``O Ch\^amund\^a! Open out your mouth quickly, and no sooner I strike Raktab\^ija with weapons, you would drink off the blood as fast as it runs out of his body. Instantly I will kill those D\^anavas sprung from the blood with sharpened arrows, clubs, swords and Musalas; and you would then be able to devour them all at your will, and, then, roam in this field as you like. O Large-eyed One! You would drink off all the jets of blood in such a way that not a drop of it escapes and falls on the ground. And then when they will all be devoured, no more D\^anavas would be able to spring. Thus they will surely be extirpated; otherwise they will never be destroyed. Let me begin to strike blows after blows on Raktab\^ija and you better drink off quickly all the blood, being intent on destroying the forces. O Chamunde! Thus, the D\^anavas being all exterminated, we will hand over to the Indra, the lord of the Devas, his Heavens without any enemy; and, thus, we can go peacefully and happily to our own places.''

29-47. Vy\^asa said :-- O King! The Ch\^amund\^a Dev\^i, of furious strength, hearing thus the Dev\^i's words began to drink the jets of blood coming out of the body of Raktab\^ija. The Dev\^i Ambik\^a began to cut the Demon's body into pieces and Ch\^amund\^a, of thin belly, went on devouring them. Then Raktab\^ija, becoming angry, struck Ch\^amund\^a with his club. But though She was thus hurt severely, She drank off the blood and then devoured all the limbs thereof. O King! Thus K\^alik\^a Dev\^i drank off the blood of all other powerful wicked D\^anava Raktab\^ijas that sprang out of the blood. Ambik\^a thus destroyed them. Thus, all the D\^anavas, created out of the blood were devoured; then, there was left, lastly, the real Raktab\^ija. Ambik\^a Dev\^i then cut him asunder into pieces by Her axe and thus killed him. Thus, when the dreadful Raktab\^ija was slain in the battle, the D\^anavas fled away trembling with fear. Without any weapons, covered all over their bodies with blood, and void of consciousness they uttered, dumb confounded ``Alas! Alas! What has happened, what has happened.'' Thus crying, they told their King \'Sumbha thus :-- ``O King of Kings! Ambik\^a Dev\^i has killed Raktab\^ija and Ch\^amund\^a has drunk off all their blood. The carrier (V\^ahana) of Dev\^i, the powerful ferocious Lion killed other powerful warriors and Kali devoured the remaining soldiers. O Lord of the D\^anavas! We have fled and come to you to give the news of the battle and to describe the wonderful doings of that Chandik\^a Dev\^i in the battlefield. O King! In our opinion, no one will be able to conquer that Lady, be he a Daitya, D\^anava, Gandarbha, Asura, Yaksa, Pannaga, Ch\^arana, R\^aksasa, or an Uraga. O King of Kings! The other Goddesses, Indr\^ani and others, have come to the battle, on their own carriers respectively and are fighting with various weapons. O Lord of the D\^anavas! The D\^anava forces are all slain by them with the excellent weapons in their hands. Even Raktab\^ija has been slain in no time. That Lion, of indomitable prowess, killed the R\^aksasas in the battle; The Dev\^i alone is hard to conquer; how much more would it be impossible to conquer Her, when She has been joined with other goddesses. So consult with the ministers and do what is reasonable. In our opinion it is better to make treaty with Her and quit your enmity. O King! Think over the fact that that Lady destroyed all the D\^anavas and at last drank off the blood of Raktab\^ija and at last killed him. What on earth can be more wonderful than this? O King! The Dev\^i Ambik\^a killed all the other Daityas and Ch\^amund\^a devoured their blood, flesh, and all. Considering all this, it is now better for us to serve the Dev\^i Ambik\^a or fly away to P\^at\^ala. No more fighting is desirable. She is not an ordinary woman; She is Mah\^a M\^ay\^a, there is not the least doubt in this. Only to serve the cause of the Gods, She has manifested Herself and is now destroying the R\^aksasa\'s race.''

48. Vy\^asa said :-- Hearing thus, \'Sumbha got confounded by K\^ala (Death), as his end was coming nigh, and said the following words, his lips quivering with anger.

49-54. You are struck with fear; so you all take the refuge of Chandik\^a or fly down to P\^at\^ala; but I will kill Her with all my exertion and effort. I conquered all the hosts of Devas and I have enjoyed their kingdom; shall I now, out of the fear of one Lady, fly and enter into the P\^at\^ala. All my attendants, Raktab\^ija and other heroes, are now slain in the battle and is it possible that I will now fly away out of the sake of preserving my life only. See! The death of all the beings is ordained by K\^ala and it is unavoidable. No sooner a being is born, he is liable to the fear of death. How can a man, then, out of fear of death, quit all his name and fame? O Ni\'sumbha! I will now go immediately to the battle, mounting on my chariot and will return after slaying Her in battle. And if I cannot kill Her, I will not then return any more. O Best of warriors! Better stand on my side with all your forces and kill that Lady in no time, with sharp arrows.

55-58. Ni\'sumbha said :-- Today I will go to the battle and slaying that K\^alik\^a, will shortly return here with Ambik\^a. O King! Do not think at all for that Lady; see my world-conquering strength and look at that weak woman; there is a vast difference. Cast aside your this great mental anxiety and trouble. Enjoy, O Brother, excellent things. I will bring that dignified Lady with all honours before you. O King! You ought not to go to the battle when I am alive. I will presently go to the fight and bring for you that Lady as a sign of our victory.

59-60. Vy\^asa said :-- O King! Thus saying, the younger brother, proud of his own strength, went hurriedly to the battlefield, mounting on his big chariot. He was protected all over his body by his coat of armour and he was well provided with various weapons and all other accoutrements of war. The bards began to sing hymns to him and various other propitious ceremonies were being performed.

Here ends the Twenty-ninth Chapter of the Fifth Book on the killing of Raktab\^ija in \'Sr\^i Mad Dev\^i Bh\^agavatam, the Mah\^a Pur\^anam, of 18,000 verses by Maharsi Veda Vy\^asa.

