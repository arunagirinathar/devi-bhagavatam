\chapter{On the Dev\^i Earth's going to the Heavens}

1. Vy\^asa said :-- O king! Hear in detail the complete life and the deeds in the incarnation of \'Sr\^i Krisna and also the various wonderful achievements by the Goddess of this universe.

2. Once on a time, the Earth was very much overburdened by the load of wicked kings and She was therefore very much afraid.

3. She then assumed the appearance of a cow and went to the Devaloka crying and very much dejected.

Indra, the lord of the Devas, asked her, O Vasundhare! What is the cause of your fear now? Who has troubled you? What afflictions are you merged in? Please tell me all these.

4. On hearing Indra's words, the Earth exclaimed :-- O Respect giver! When You have asked me, I am explaining to you the cause of all my afflictions and sorrows; at present I am overburdened with too much load.

5-8. Now is reigning in the earth J\^ar\^asandha, the king of Magadha, a very very vicious person. Thus the other \'Si\'sup\^al, the lord of the Chedis, the uncontrollable K\^a\'sir\^aj, Rukm\^i, the powerful Kamsa, the strong Naraka, the Sauvapati \'S\^alva, the wicked Ke\'s\^i, Dhenuka, and Batsaka all these are now in royal positions. O Lord of the Devas! These kings are all devoid of the least trace of virtue, quarrelsome against each other, infatuated with vanity, and addicted to vicious deeds. These have become kings as if they were personified Yamas, the Lords of Death, and are constantly troubling me. I am now unable to carry their loads; where shall I go now? This great thought is constantly ailing me.

9-11. O Vasava! What to tell! The Bhagav\^an in His Boar Incarnation is the cause of all these my afflictions; O Indra! These present troubles I am fallen into only through Him; for when the cruel Daitya Hir\^any\^aksa; the son of Ka'syapa stole me away and drowned me in the great ocean, then it was Visnu in his Boar incarnation that killed him and rescued me from the ocean and then kept me in this my stable position.

12. Had he not then lifted me up, I would have rested safe in the depths of Ras\^atala; O Lord of the Devas! Now I am quite unable to bear the load of these vicious persons.

13. O Surendra! The vicious twenty eight Kali is coming quickly in front. Thinking of His influence, it seems to me that I will be very troubled then and will have to go down to Ras\^atala.

14. Therefore, O Lord of the Devas! I am bowing down before Your feet, kindly relieve me of my burden and save me from these endless troubles.

15. Indra said :-- O Earth! I cannot do anything for you. You better go and take refuge of Brahm\^a. I am also going to Him. He will remove all your troubles.

16. Hearing Indra's words the Earth hurriedly went to the realm of Brahm\^a and Indra and all the other Devas followed Her; and all reached the Brahmaloka.

17-18. O King! The Grand Father Brahm\^a saw the Earth coming to him and through the power of meditation, found out the cause of Her coming and said :-- O Auspicious One! why are You crying? What troubles You have now? What wicked person has given You troubles.

19. The Earth said :-- O Lord of the Earth! The vicious Kali is coming before; under Its influence the subjects will be horribly vicious; therefore I am very much afraid of this Kali.

20. In the beginning of this Kali Yuga, the ancient enemies, the Asuras have now incarnated on this earth as kings. They will be extremely wicked, quarreling against each other, and will be clever in stealing other\'s things. There is no doubt in these.

21. O Grand Father! Now kill these vicious kings and relieve my burden. O Lord! I am very much troubled by the armies of these kings.

22. Brahm\^a said :-- O Goddess! I, too, am unable like Indra to remove your load. Let us go to that Visnu, the Holder of the disc.

23. That Jan\^ardana will remove your burden. I thought of this well before and settled what to do.

24-25. Vy\^asa said :-- Thus saying, the four faced Brahm\^a, the Author of the Vedas, rode on His Hamsa Vehicle taking in front the Goddess Earth; and the Devas went to Visnu and began to praise Visnu Jan\^ardana, the Deva of the Devas, with the words of the Vedas with full devotion.

26. Brahm\^a said :-- Thou hast thousand heads, thousand faces, thousand feet. Thou art the Purusa of the Vedas, Thou art the Deva of the Devas, and Thou art Eternal.

27. O Omnipresent! Thou art the Past, Present, and Future! O Lord of Laksm\^i! Thou hast awarded immortality unto us.

28. Thou art the Creator of universe, the Preserver and the Destroyer; Thou art the One and the Only goal and thou art the God. Everybody knows that all these glories belong to Thee.

29. Vy\^asa said :-- O king! When Brahm\^a praised thus, Visnu whose sign was Garuda, was highly pleased and appeared before Brahm\^a and the other Devas.

30. The Bhagav\^an asked them about their welfare and enquired in detail into the cause of their arrival there.

31. Brahm\^a them bowed down to Him and, remembering the cause of the sorrows of the Goddess Earth, said :-- O Lord! Thou art now to relieve the burden of the Earth.

32. O Thou, Ocean of mercy! When the end of the Dv\^apara yuga will come Thou art to incarnate in the world and kill the wicked kings and thus to relieve the burden of the Earth.

33-34. Visnu said :-- I am not independent in these matters; why I? Brahm\^a Mahe\'sa, Indra, Agni, Yama, Visvakarm\^a, the Sun and Varuna and other Devas, nobody is independent. This whole universe, moving and unmoving is existing under the control of the Yoga M\^ay\^a; and from Brahm\^a up to the clot of grass, all are strung into the thread of Her Three qualities.

35. O One of good vows! Whatever that Yoga M\^ay\^a, the Supreme Goddess, Who is All will, Whose mouth is inward, Who does good at all times, what She wills She does that at any time. You should all know that we are entirely under Her control.

36-41. You better think that were I independent, what for would I have stayed in the great ocean, incarnating in the Fish and Tortoise Bodies! O Br\^ahmin! What name or pleasure is there in enjoyment in the body of lower animals! What holy merit or any other reward may I expect from being born in the wombs of lower animals? What is the reason that I assume the body of a Boar? or of a Man-Lion? or of a Dwarf? Why was I born as the son of Jamadagni. Especially why did I, being born of that highsouled Br\^ahman Jamadagni, and being the best of the Br\^ahmins, do the most atrocious act like that of a heartless brute and fill up the lakes with their blood. Alas! I killed the Ksatriyas mercilessly; to say nothing more than that I killed the sons that were then in the wombs. Were I independent, what for I would have done these horrible and cruel deeds! O Lord of the Devas! See again. In my R\^ama incarnation I roamed on foot, helpless and without any provision, in the fearful Dandaka forest unfrequented by anybody, wearing clotted hair, bark, rags, like a man who feels no shame, and behaved like a hunter and killed many animals.

42-44. Being under the delusion of M\^ay\^a, I could not make out the real nature of the golden deer; consequently leaving Janak\^i in the thatched cottage, I went out pursuing the deer. Though repeatedly warned by me not to leave the place, Laksmana was moved by the qualities of Prakriti, forsook her and went out on my search.

45. Then the hypocrite R\^avana, the king of the R\^aksasas, under the garb of a beggar; stole away by force the daughter of Janaka, who had become very lean on account of sorrows.

46. I was very much distressed owing to the separation from my dear wife and roamed about weeping sorely in forest and formed friendship with Sugr\^iva, under the influence of the circumstances.

47. It was an act of gross injustice on my part to kill B\^al\^i, the king of the monkeys. I freed him from his curse; afterwards, aided by the monkeys, I had to go to Lank\^a.

48. When my younger Laksmana and myself were both enchained under the chain of the serpents, N\^agap\^a\'sa, and were senseless, the monkeys all were astonished.

49. Then Garuda came and freed us the two brothers, from those N\^agap\^a\'sas! I considered then what adverse inauspicious circumstances Fate sometimes ordains on our lot.

50. I lost my kingdom, lived in the forest, my father died, Janak\^i was stolen and I had to suffer extreme troubles in very deadly battles; I could not know what worse fate still awaited for us?

51. O Suras! What more calamity can you expect to befall any person than that I was from the very first deprived of my kingdoms and wealth, and had to go to the forest with the princess S\^it\^a dwelling in and taking shelter in a dense forest!

52. At the time of my going to the forest my father did not give a single penny; penniless and helpless I had to get out of Ayodhy\^a on foot.

53. I was compelled to leave my Ksattriya Dharma and take up the avocation of a hunter and thus to spend fourteen years in forest.

54. After that, under the benign influence of Fate, I was able to kill that Asura R\^avana and got the victory in the battle and was able to bring back dear S\^it\^a to Ayodhy\^a.

55. There I succeeded in becoming the ruler of the kingdom Ko\'sala with its subjects and got the full kingdom and enjoyed for a few years the pleasures of the world.

56-57. The stealing away of S\^it\^a took place at the first outset; next I got my kingdom; then the subjects began to circulate the bad name regarding Janak\^i; and I being afraid of that, deported her into exile in the forest. At that time I had to suffer again extreme pain and agony due to the separation from my wife. Then the daughter of the Goddess Earth penetrated into the Earth and got down to the P\^at\^ala.

58. O Devas! When I had to depend on Fate and to suffer so many troubles incessantly, where else can you dare to say that an independent man exists.

59. Afterwards under the influence of Time, I had to go to Heaven with my brothers. Let all this point to what it may, the intelligent learned people can say what an amount of mishaps takes place to one who is dependent!

60. O One born from the Lotus! You hear my word; I am in every way dependent; why I? Rudra, You and all those Suras are fully dependent.

Here ends the Eighteenth Chapter of the Fourth Book of the Mah\^a Pur\^anam \'Sr\^i Mad Dev\^i Bh\^agavatam of 18,000 verses by Maharsi Veda Vy\^asa on the Dev\^i Earth's going to the Heavens.