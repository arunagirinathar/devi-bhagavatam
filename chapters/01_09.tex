\chapter{On the killing of Madhu Kaitabha}

1-49. S\^uta said:--O Munis! When the excellent T\^amasi \'Sakti, the Goddess presiding over sleep came out from the eyes, mouth, nose, heart, etc., of the body of the World-Guru Visnu and rested in the celestial space, then the powerful Lord Visnu began to yawn and got up. He saw the Praj\^apati Brahm\^a, terrified and spoke to him in words, deep like rumbling in the clouds :-- ``O Bhagav\^an Padmayone! What makes you come here, and leave your tapasy\^a? Why do you look so terrified and anxious?'' Hearing this, Brahm\^a said: ``O Deva! The two very powerful and very terrible Daityas, Madhu Kaitabha sprung from the wax of Thy ears approached to kill me; terrified thus, I have come to Thee. So, O Lord of the Universe! O V\^asudeva! now I am quite out of senses and terrified; save me.'' Visnu, then, said :-- ``Now go and rest in peace, without any fear; let those two stupids, whose life has been well nigh exhausted, come to me for battle; I will certainly kill them.''

S\^uta said :-- While Bhagav\^an Visnu, the Lord of all the Devas, was thus saying, those two very powerful D\^anavas, elated with pride, came up there, in their search for Brahm\^a. O Munis! The two proud Daityas stood there in the midst of the Pralaya water without anything to stand on and with calm attitude. They then spoke to Brahm\^a as follows :-- O You have fled and come here? You cannot escape. Go on and fight. I will kill You before this one.

Then I will kill him also who sleeps on this bed of serpent. Either give us battle or acknowledge yourself as our servant. Hearing their words, Jan\^ardan Visnu addressed them as follows:-- ``O two D\^anavas, mad for war! Come and fight with me as you like; I will surely curb your pride. O two powerful ones! If you trust me, come and fight''. Hearing this, the two D\^anavas in the midst of that mass of water rest-

ing without any support, came up there to fight, with their eyes rolling with anger. The Daitya, named Madhu, became very angry, came up quickly to fight while Kaitabha remained where he stood. Then the hand-to-hand fight ensued between the  two mad athletes; Bhagav\^an Hari and Madhu became tired; Kaitabha came up and began to fight. After that Madhu and Kaitabha joined and, blind with rage, began to fight again and again hand to hand with the very powerful Visnu. And Bhagav\^an Brahm\^a and the Prime Force \^ady\^a \'Sakti witnessed this from the celestial Heavens. So the fight lasted for a long, long, time; yet the two D\^anavas did not feel a bit tired: rather Bhagav\^an Visnu became tired. Thus five thousand years passed away; Hari then began to ponder over their mode of death. He thought :-- ``I fought for five thousands of years but the two formidable D\^anavas have not been tired anything, rather I am tired; it is a matter of great surprise. Where has gone my prowess? and what for the two D\^anavas were not tired; What is the cause? This is a matter, now, seriously to be thought over''. Seeing Bhagav\^an Hari thus sunk in cares, the two haughty D\^anavas spoke to him with great glee and with a voice like that of the rumbling of cloud :-- ``O Visnu ! If you feel tired, if you have no sufficient strength to fight with us, then raise your folded palms on your head and acknowledge that you are our servant; or if you can, go on fighting. O intelligent one!

We will take away your life first; and then slay this four-faced Brahm\^a''. Hearing the words of the two D\^anavas, resting there without any support in the vast ocean, the high-minded Visnu spoke to them in sweet consoling words :-- ``See, O heroes! no one ever fights with one who is tired, afraid, who is weaponless, who is fallen and who is a child; this is the Dharma of the heroes. Both of you fought with me for five thousands of years. But I am single handed; you are two, and both equally powerful and both of you took rest at intervals. I will therefore take rest for a while, and then fight with certainty. Both of you are very powerful and very much elated in vanity. Therefore rest a while. After taking rest for a while I will fight with you according to the just rules of warfare.'' Hearing these words of Hari, the two chief D\^anavas trusted and remained far aloof, ready to fight again.

Now the four-armed V\^asudeva, seeing them at a sufficient distance, began to ponder in his mind thus :-- ``How their death can be effected? Thinking for a time, he came to know that they have been granted, by the Supreme \'Sakti Dev\^i, death at their will (Ichch\^amrityu); and therefore they are not getting tired. I fought so long with them uselessly; my labour went in vain. How then can I now fight, with this certain knowledge. Again if I do not fight, how these two D\^anavas, elated with their boon and

giving troubles to all, be destroyed? When the boon is granted by the Dev\^i their death is also well nigh impracticable. Who wants his own death, even placed in very great distressed circumstances. Attacked by terrible diseases, reduced to indigent poverty no one wants to die; so how can it be expected that these two haughty D\^anavas would want their death themselves? Therefore it is advisable me to take refuge of that \^ady\^a \'Sakti, the giver of the fruits of all desires. No desires can ever be fructified unless She is thoroughly pleased. Thus thinking, Bhagav\^an Visnu saw the beautiful Dev\^i Yoga Nidra \'Siva, shining in the air. Then the supreme Yogi, Bhagav\^an Visnu, of immeasurable spirit began to praise with folded palms that great Bhuvane\'svar\^i Mah\^a K\^ali, the giver of boons for the destruction of the D\^anavas. ``O Dev\^i! I bow down to Thee O Mah\^am\^ay\^a, the Creatrix and Destructrix! Thou beginningless and deathless! O auspicious Chandike! The Bestower of enjoyment and liberation I do not know Thy Saguna or Nirguna forms; how then can I know Thy glorious deeds, innumerable as they are. To-day Thy undescribable prowess has been experienced by me, I being made senseless and unconscious by Thy power of sleep. Being tried again and again by Brahm\^a with great care to bring me back to my consciousness, I could not become conscious, so much my senses were contracted down. O Mother! By Thy power, Thou didst keep me unconscious and Thou again hast freed me from sleep, and I also fought so many times. O giver of one's honour! Now I am tired; but Thou hast granted boon to the two D\^anavas and therefore they are not getting tired. These two D\^anavas, puffed up with pride, were ready to kill Brahm\^a; and therefore I challenged them to fight with me and they too are fighting fiercely with me in this vast ocean. But Thou hast granted them the wonderful boon that they will die whenever they will; and therefore I have now come to Thy refuge, as Thou protectest those that come under Thy shelter. Therefore, O Mother, the remover of the Deva\'s troubles! These two D\^anavas are become exceedingly elated by Thy boon and I, too, am tired. Therefore dost Thou help me now. See! those two sinners are ready to kill me; without Thy grace, what can I do now? and where to go?''

50-59. Thus praised by the eternal V\^asudeva Jagann\^atha Hari, with humility and pranams, the Dev\^i Mah\^a K\^ali, resting in the air, said smiling :-- ``O Deva deva Hari! Fight again; O Visnu! These two heroes, when deluded by My M\^ay\^a, would be slain by you; I will delude them certainly, by My side long glance; O N\^ar\^ayana! then slay quickly the two D\^anavas, when conjured by My m\^ay\^a''.

S\^uta said :-- Hearing these loving words of Bhagavat\^i, Bhagav\^an Visnu went to the scene of battle in the middle of that ocean, when the

two powerful D\^anavas of serene tempers and eager to light, became very glad on seeing Visnu in the battle and said :-- ``O four-armed one! we see your desire is very lofty indeed; well  stand! Stand! now be ready for battle, knowing that victory or defeat is surely dependent on Destiny. You should think now thus :-- Though it is generally true that the more powerful one wins victory; but it also happens sometimes that the weak gets the victory by queer turn of Fate; so the high souled persons should not be glad at their victories, nor should express their sorrows at their defeat; so don’t be glad, thinking, that you on many former occasions fought with many D\^anavas who were your enemies, and got the victory; nor be sorry that now you are defeated by the two D\^anavas''. Thus saying, the powerful Madhu Kaitabha came up to fight. Seeing this, Bhagav\^an Visnu struck them immediately by fist with great violence; the two D\^anavas, elated with their strength, struck Hari in return with their fists. Thus fighting went on vigorously.

60-87. Now seeing the two D\^anavas of great powers, fighting on incessantly, N\^ar\^ayana Hari cast a glance expressive of great distress, towards the face of the Dev\^i Mah\^ak\^ali. Seeing Visnu thus distressed, the Dev\^i laughed loudly and began to look constantly with eyes somewhat reddish and shot towards the two Asuras side-long glances, of love and amorous feelings which were like arrows from the Cupid. The two vicious Daityas became fascinated by the side-long glances of the Dev\^i and took great pleasures in them; being extremely agitated by these amorous darts, looked with one steady gaze towards the Dev\^i, of spotless lustre. Bhagav\^an Hari, too, saw the wonderful enchanting pastime of the Dev\^i. Then Hari, perfectly expert in adopting means to secure ends, began to speak smiling and in voice like that of the rambling cloud, knowing the two D\^anavas enchanted by Mah\^am\^ay\^a, thus :--

O two heroes! I am very glad at the mode of your fighting. So ask from me boons. I will grant that to you. I saw many D\^anavas before, fighting; but never I saw them expert like you, nor I heard like this. I am therefore, very much satisfied by your such unrivalled powers. Therefore, O greatly powerful pair of D\^anavas! I wish to grant both of you any boon that you want. Seeing the Dev\^i Mah\^am\^ay\^a, the gladdener, of the Universe, the two D\^anavas felt themselves amorous; and therefore they became proud on hearing Visnu's those words and told Visnu, with their lotus-like eyes wide open, thus :--

O Hari! what do you like to give us? We are not beggars; we do not want anything from you. O Lord of the Devas! Rather we will give you whatever you desire; we are donors; not receivers. So O V\^asudeva! Hrisi Kesa! We are glad to see your- wonderful fight; so ask from

us any boon that you desire. Hearing their words, Bhagav\^an Jan\^ardan said :-- `` If you both are so much pleased with me, then I want this that both of you be killed by me.'' Hearing these words of Visnu, Madhu Kaitabha became very much wondered and thinking ``we are now cheated'' remained for some time merged in sorrow. Then reflecting that there is water everywhere and solid earth nowhere, they said :--

``O Jan\^ardana Hari! We know that you are truthful; therefore now we want this desired boon from you that you wanted to grant us before now grant us this desired boon of ours. O Madhus\^udana! We will be slain by you; but kill us, O M\^adhava! on a solid earth, free from any water; and thus keep your word.

\'Sri Bhagav\^an Hari laughed and remembering His Sudarsan disc said :-- ``O two highly fortunate ones! Verily, I will kill both of you on the vast solid spot without any trace of water. Thus saying, the Devadeva Hari expanded His own thighs and showed to those D\^anavas the vast solid earth on the surface of water and said :--

``O two D\^anavas! See, here is no water. Place your two heads here; thus I will keep my word and you would keep your word.'' Hearing this, Madhu Kaitabha thought over in their minds and expanded their bodies to ten thousand Yojanas. Bhagav\^an Visnu Hari also extended his thighs to twice that amount. Seeing this, they were greatly, suprised and laid their heads on the thighs of Visnu. Visnu of wonderful prowess, then cut off quickly with His Sudarsan disc the two very big heads over His thighs. Thus the two D\^anavas Madhu Kaitabha passed away; and the marrow (meda) of them filled the ocean. O Munis! For this reason, this earth is named Medin\^i and the earth is unfit for eatable purpose.

Thus I have described to you all that you asked. The sum and substance is this that the wise persons should serve Mah\^am\^ay\^a with all thei hearts. The Supreme \'Sakti is worshipped by all the Devas. Verily verily, I say unto you that this is decided, in all the Vedas and other \'Sastras that there is nothing higher than this \^ady\^a \'Sakti. Therefore this Supreme \'Sakti should be worshipped anyhow; either in Her Saguna form or in Her Nirguna state.

Thus ends the ninth Chapter of the first Skandha on the killing of Madhu Kaitabha in the Mah\^apurana \'Srimad Dev\^i Bh\^agavatam of 18,000 verses by Maharsi Veda Vy\^as.