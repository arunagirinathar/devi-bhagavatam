\chapter{On the description of the marriage of \'S\^uka}

1-4. \'Sr\^i \'S\^uka said :-- O king! This great doubt arises in my mind how a man can be free from desires and the rewards of their actions, when he lives in the midst of this Sams\^ara, that is all full of M\^ay\^a? When even by the acquiring of wisdom of the \'S\^astras and the capability to judge which is real and which is unreal, the delusion of the mind is not dispelled until one resorts to the practice of Yoga, how then can freedom from desires and liberation come to a householder? The darkness of a room is not destroyed by the mere mention of lamp, light; so the wisdom acquired by reading the \'S\^astras can never dispel the darkness of delusion that reigns in the inside of a man. O lion of kings! If one wants Moksa, one ought not to commit any act of revenge or injury or killing any being; how can this be possible to a householder?

5-17. Your desires to acquire wealth, to enjoy royal pleasures and to get victory in battle have not yet subsided; how then can you be a Jivan mukta? O king! You consider yet a thief, thief and a saint, saint; you consider a man as your relative or other than that; these ideas have not vanished from you; how then can you be called Videha? O king! You feel the pungent, bitter, astringent, sour tastes and the like; you feel good and bad rasas respectively; you become glad when success comes to you and you feel sorrow when you happen to fail; and you experience the three states, waking, dreaming, and deep sleep as an ordinary man does, how then can you be called to attain the Tur\^iya (fourth) state? May I ask :-- Whether you cherish this idea that all these infantries, cavalries, chariots, and that all these elephants are mine; I am the lord of all the wealth and things? Or whether you do not cherish this idea? O king! I think you eat sweet and good things, and, at times, feel pleasure

and, at other times, feel pain! So, O king! How can you look on the garland of flowers and the snake as one and the same? O king! He who is a Muktapurusa considers a lump of earth, a piece of stone, and gold as of one and the same value; he considers everything to be the same \^atm\^an and does good to all the beings. Whatever that may be, I do not find any pleasure at present with houses, wife, etc., or with anything, in fact. What my heart’s desire is that I roam alone always without any desires in my heart. Therefore I like not to have any companion; to be free from any attachment and to be peaceful, and calm; I do not wish to accept anything from anybody; I will forego all pleasures and pains from cold, warmth, etc., and I will sustain my life on roots, fruits, and leaves, obtained without any effort and will roam, as I like, like a deer. When I have not got the least attachment to the household life and when I am beyond all the attributes, what necessity have I then of house, wealth or a suitable wife? And when you think of various things with loving heart, and yet say that you are a Jivan mukta, that is nothing but a mere vanity of yours! O king! When you think and become anxious about your enemies, about your wealth or sometimes about your army, how then can you be said to be free from cares? What more can be said than the fact that many Munis, eating moderately and controlling their senses, and leading an anchorite's life, and knowing the unreality of the world, fall victims to the M\^ay\^a!

18-27. Then what need there is to talk of you? O king! know that the hereditary title ``Videha'' to your line of kings indicates downright insincerity; nothing can be other than this as the name ``Vidy\^a Dhara'' (holder of knowledge) is applied to an illiterate man; as the name ``Div\^akara'' (sun) is given to a born-blind man, as the name ``Laksm\^idhara'' (holder of wealth) is given to even a poor man, as these names are quite useless to me. I have heard that the kings of your family who were your predecessors were called ``Videha'' in name only and not in deed. O king! In your family there reigned a king named ``Nimi.'' Once on a time that royal sage invited his Guru Va\'Sistha to perform a sacrifice, when Va\'sistha said :-- ``I am already invited by Indra, the lord of the Devas, to perform his sacrifice; so O king! let me first finish his work; I will then take up your work. Better go on collecting the sacrificial materials till my that work is complete.'' Thus saying, Va\'sistha went away to perform Indra's sacrifice; on the other hand, the royal sage Nimi selected another priest and made him his Guru and began his sacrifice. Hearing all this, Maharsi Va\'Sistha became angry and cursed him thus :-- ``O forsaker of your Guru! For the crime of forsaking your Guru, let thy body be destroyed today!'' At this, the royal sage, too, cursed Va\'sistha in his turn ``Let your body fall off also.'' Then the bodies of both the

persons fell. But, O king! this curiousity came to my mind, how the royal sage, whose body fell before, cursed his own Guru afterwards.

28-35. Janaka said :-- O Lord of Br\^ahmins! what you have said is, in my opinion, all quite true; nothing is false. Still hear. Know what my most worshipful Guru Deva has spoken to me is, in fact, true (and nothing else). You are now intending to quit the company of your father and go to the forest; well and good! but even then you will undoubtedly have the company of deer, etc.; see, also, that when the five elements, earth, water, air, etc., are present, encompassing everywhere, how, then, can you expect to be free from all companions? So, O Muni! when you will have to think always of your food, how, then, can you be said to be free from all cares? Again, even if you go to the forest, you will have to think there also for your staff, deer skin, etc.; so you can take my case, too, of thinking of my kingdom, whether I think or not, as your thinking of staff, deer skin, etc., your heart is tainted with Vikalpa Jñ\^an (knowledge of doubt, duality, etc.); and therefore you have come here from a far-off country. But my heart is free from any such doubt and I am remaining quite cheerful here. O best of Br\^ahmins! I have got no doubt whatsoever on any point, and therefore I take my food and go to sleep with great pleasure. ``I am not bound up by this world'' this idea gives me constant happiness of the highest degree. But you consider that you are bound and therefore you feel always constant pain. So leave off your idea that you are bound, and be happy. ``This body is mine'' this knowledge leads to my bondage; and ``This body is not mine'' this knowledge leads to freedom so know this verily that all this wealth, kingdom, etc., are not mine.

36-45. S\^uta said :-- Hearing these words of the royal sage, \'S\^uka Deva became exceedingly glad and pronounced ``Sadhu'' ``S\^adhu'' (true saint, indeed a true saint, well said) and went away without any delay to the pleasant \^a\'Srama of Vy\^asa. Vy\^asa, too, seeing his son come back, became very glad and embraced him and took the smell of his head and asked about his welfare again and again. Then \'S\^uka Deva, well conversant with the \'S\^astras and ever ready in studying the Vedas, sat by the side of his father, with an enlightened mind, in his lovely \^a\'Srama and thinking of the state of the highsouled Janaka in his kingdom, began to feel the highest peace. Though \'S\^uka adopted the path of Yoga, yet he married the daughter of a Muni, named Pivar\^i, very beautiful, fortunate, enhancing the glory of her father's family. Then were born first the four sons named Krishna, Gauraprabha, Bh\^uri, and Deva\'sruta out of the sperm of \'S\^uka and the ovum of Pivari; and next a daughter was born named K\^irti of them. Vy\^asa's son \'S\^uka, endowed with the fire of asceticism gave the daughter K\^irti in marriage in due time with the high-souled

An\^uha, the son of Vibhr\^aja. As time passed on, a son was born of the womb of K\^irti and the sperm of An\^uha, a son who became the powerful king Brahmadatta, the knower of Brahm\^a and endowed with wealth and prosperity. Some time elapsed when An\^uha, the son-in-law of \'S\^uka Deva, getting from N\^arada the M\^ay\^av\^ija and highest knowledge of Yoga handed over his kingdom to his son and went to the hermitage of Vadarik\^a and became liberated.

The Devars\^i N\^arada gave him the mantra, the v\^ija of M\^ay\^a; and by the influence of that mantra and by the grace of the Dev\^i, the knowledge of the Supreme Brahm\^a, arose in him without any obstacle and gave him liberation.

46-51. On the other hand \'S\^uka Deva, always averse to any company, left his father and went to the beautiful mountain Kail\^a\'sa. He began to meditate on the unmoving Brahm\^a and thus remained there. After some time the highly energetic \'S\^uka Deva attained Siddhi (supernatural powers) Anim\^a, Laghim\^a, etc., rose up high in the air from the top of the mountain and began to roam there, and then he appeared like a second Sun. When \'S\^uka arose from the peak, it severed into two and various ominous signs became visible. When \'S\^uka Deva, appearing like a second Sun by the dazzling brilliancy of his body, suddenly vanished away like air and became diluted in the Param\^atman, entering into everything and became invisible, then the Devarsis began to chant hymns to him. On the other hand, Vy\^asa Deva became very much distressed with the separation from his son and cried out frequently ``Oh, my son! Alas! my son Where are you gone?'' and went to the summit of the mountain where \'S\^uka did go and wept bitterly. Then \'S\^uka Deva, who was then residing as the Param\^atman, the Internal controller of all the beings and with all the beings, knowing Vy\^asa Deva as very much fatigued, distressed, and crying, spoke out as an echo from the mountains and trees thus :-- ``O Father! There is no difference between you and me, considered in the light of \^atman; then why are you weeping for me?''

52-59. Even today the above echo is clearly heard (almost daily). Seeing Vy\^asa Deva grieved very much for the separation from his son and always crying ``Oh! my son! Oh! my son!'' Bhagav\^an Mahe\'svara came there and consoled him saying ``O Vy\^asa Deva! your son is the foremost of the Yogis; he has attained the highest state, so very rare to the ordinary persons that are not self controlled. So do not be sorry any more. O Sinless One! when you have realised the Brahm\^a-tattva, then you ought not to express any sorrow for your \'S\^uka who is now stationed in that Br\^ahman. Your fame is now unrivalled, only on account

of your having got a son like him.'' Vy\^asa Deva said :-- ``O Lord of the Dev\^as! O Lord of the world! What am I to do now? My grief does not quit my heart anyhow or other. My eyes are as yet satisfied in seeing my son; they like still to see the son.'' Hearing these sorrowful words of Vy\^asa, Bhag\^avan Mah\^adeva said :-- ``O Muni S\^ardula! I grant this boon to you that you will see the form of your son abiding in shadow, very beautiful, by the side of you. O Destroyer of enemies! Now abandon your grief by seeing that shadow form of your son.'' When Bhagav\^an Mahe\'svara said so, Vy\^asa began to see the bright shade form of his son. Granting thus the boon, Bhagav\^an Mah\^adeva vanished then and there. When He vanished away, Vy\^asa became very much distressed with sorrow for the bereavement of his son and returned with heavy heart to his own hermitage.

Thus ends the nineteenth chapter of the first Skandha on the description of the marriage of \'S\^uka in the Mah\^apur\^ana \'sr\^i Mad Dev\^i Bh\^agavatam of 18,000 verses.