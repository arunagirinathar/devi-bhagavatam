\chapter{On the description of the origin and the form of the Dev\^i}

1-4. Vy\^asa said :-- Soon the Devas reached Vaikuntha, protected by Visnu; they at once began to look at the exquisite indescribable beauty of the place. At intervals they saw nice lovely divine houses, shining and appearing very splendid; pools and lakes were seen in front of them beautified with Kalh\^ara lotus flowers. They began to see, at other places, rivers flowing; swans, cranes, Chakrav\^akas and other aquatic birds were swimming there easily and warbling lovely sounds. At other places again, beautiful gardens came to their sight adorned exquisitely by Champaka, A\'soka Mand\^ara, Bakula, \^Amr\^ataka, Tilaka, Kuruvaka and Mallik\^a and various other flower trees, the cuckoos were seen there cooing melodiously, bees humming gently and peacocks dancing beautifully.

5-6. In the centre was situated the golden palace of Hari, towering to heavens, the rooms and quadrangles were all charming; at places, they were bedecked with gems and jewels and adorned with various paintings. There was the Divine Seat in the centre, composed wholly of gems and jewels; and Visnu was occupying this place. There were Visnu's P\^arisadas or attendants, Sunanda, Nandana, and others; they were so much devoted to their master that their hearts never become attached to any other thing; so they were devotedly singing His praises and chanting His hymns with undivided attention.

7-10. There were dancing the Apsar\^as (celestial nymphs) and the Devas, Gandarbhas, and Kinnaras were singing in melodious tunes. Those who love the chanting of the Vedas, such calm-tempered Munis were reciting the Vedic S\^uktas and thus highly extolled Him. The two lovely gate-keepers Jaya and Vijaya were waiting at the entrance gate with golden sticks in their hands; the Devas coming nigh the city of Visnu caught sight of them and said :-- ``Any of you may go and inform Visnu that Brahm\^a, Rudra, and the whole host of gods are waiting at His door to see Him.''

11. Vy\^asa said :-- O king! Hearing their words, Vijaya went away at once to Visnu; and, saluting Him, informed Him of the arrival of the Devas.

12-13. Vijaya said :-- O Lord! Thou destroyest the enemies of the gods; hence Thou art the most worshipped of them. O Lord of Ram\^a! The whole hosts of gods have come and are waiting at Thy door, O Bibhu!

Brahm\^a, Rudra, Indra, Varuna, Fire and Yama and other gods, anxious to see Thee, are all praising Thee by proper hymns.

14-32. Vy\^asa said :-- Hearing Vijaya's words, Visnu, the Lord of Ram\^a became very anxious and soon went out of his room to see the Devas. Hari came up to them and seeing the Devas waiting at the doors very morose and tired, cheered them up by casting a favourable glance full affection and love. The gods bowed down and praised hymns to Jagann\^atha the Deva of the Devas, the enemy of the Daityas and revealed in the Vedas. O Deva of the Devas! Thou art the Creator, Preserver and the Destroyer of the worlds; Thou art the ocean of mercy and the sole refuge of this Universe; O Lord! We have come to Thee as our Great Refuge; therefore dost Thou save us from the present difficulty. Thus praised by the gods, Visnu said :-- O Immortals! Take your respective seats and speak how are you all? Why have you all in a body come here? Why are you so much depressed and worn out with cares? Why do you look so melancholy? Say soon for what purpose you with Brahm\^a and Rudra have come here. The Devas said :-- ``O Lord! The Asura Mahisa is very cruel and wicked; always addicted to vicious acts; now that most sinful D\^anava has become very much puffed up with pride and is tormenting us always. What more shall we say than this, he is appropriating to himself the share of the Yajñas performed by the Br\^ahmins; we are therefore, terror-stricken and are wandering in mountains and fastnesses. O Destroyer of Madhu! He has become unconquerable due to his being granted the boon; considering, therefore, the gravity of our situation we have taken refuge unto Thee. O Krisna! Thou art acquainted with all the tricks and M\^ay\^a of the Daityas; therefore Thou art capable to kill them. Therefore Thou alone art able to deliver us from the present difficulty; be pleased, therefore; to Dev\^ise means for that purpose. The Creator Brahm\^a has granted him this boon that the demon could not killed by any man; therefore we are asking you where can we get a female who will be able to kill that hypocrite in battle. Mahisa has turned out very wicked on the strength of that boon; say, therefore, who amongst Um\^a, Laksm\^i, \'Sach\^i, or Vidy\^a or any other woman will be able to kill him. Therefore, O Gracious One to faithful worshippers and attendants! Thou art the Preserver of this world; now Dev\^ise specially the cause of his death and carry out the purpose of the gods.'' Vy\^asa said :-- O king! Visnu on hearing their words, spoke smiling ``We fought before; but this Asura could not at that time be killed. Hence if some beautiful female Deity be now created out of the collective energy and form of the \'Saktis of each of the Devas, then that Lady would be able easily to destroy that Demon by sheer force. The Lady Deity then sprung from the collective energy of ours, would at once be able to destroy that Mahisa, elated on his getting the power, though he is skilled in hundreds of M\^ay\^as (magics). Therefore ask ye now all, with your wives respectively, boons from that portion which resides in you all in the form of Fiery Energy, that the collected energy thus manifested may assume the form of a Lady. We will then offer unto Her, all the Divine weapons, the trident, etc., that belong to us. That Deity, then, full of energy and with all the weapons in Her hands would kill that wicked Demon, vicious and swelled with vanity.''

33-46. Vy\^asa said :-- On Visnu, the Lord of the Devas, saying thus, came out spontaneously, at once, of the face of Brahm\^a, the brilliant fiery energy, very difficult to conceive. That energy looked red like gems and pearls, hot, at the same time, a little cool, having a beautiful form, and encircled by a halo of light. O King! The high-souled Hari and Hara, of mighty valor, were astonished to see this Fire, emitted from Brahm\^a. Next came out of the body of \'Sankara, His fiery spirit, quite in abundance and very wonderful to behold; it was silvery white, terrible, unbearable, and incapable of being seen even with difficulty. It extended like a mountain and looked horrible as if the incarnation of the Tamo Guna like another Tamo Guna (\'Siva is the incarnation of Tamo Guna that destroys everything). It was very surprising to the Devas and very fearful to the Daityas. Next a dazzling light of blue colour emanated from the body of Visnu. The light that came out of the body of Indra was hardly bearable, of a beautiful variegated colour, and comprised in itself the three qualities. Thus masses of lights came out respectively from Kuvera, Yama, Fire and Varuna. The other Devas, too, gave their shares of fiery lights, very lustrous and splendid. Then these all united into a great Mass of Fire and Light. Like another Himalayan mountain shone full their lustrous Divine light; Visnu and the other Devas were all extremely surprised to see this. While the Devas were thus looking steadfastly on that Fire, an exquisitely handsome Lady was born out of it, causing excitement and wonder to all. This Lady was Mah\^a Laksm\^i; composed of the three qualities of the three colours, beautiful, and fascinating to the universe. Her face was white, eyes were black, her lips were red and the palms of her hands were copper-red. She was adorned with divine ornaments. The Goddess was now manifest with eighteen hands, though She had a thousand hands (in Her unmanifested state). Now She became manifest out of the mass of fire, for the destruction of the Asuras.

47-52. Janamejaya said :-- O Best of the Munis! O Krisna! You are highly fortunate and you are all-knowing. Kindly describe, in detail the birth of Her body. O Deva! Please say whether the energies of all the gods united into one or remained separate? Whether Her body and Her limbs were all luminous. Was it that Her face, nose, eyes, etc., and all other parts of Her body were created out of the different fires respectively or whether was it that those limbs were fashioned when the different fires blended into one huge mass? Describe, in detail, the origin of the body and the several limbs thereof; also inform me the limbs that were produced out of the corresponding Deva's fiery part; as well tell me the several ornaments and several weapons given by the several Devas respectively. I am very desirous to hear all these from your lotus-like mouth. O Brahm\^an! Hearing from your lotus-like mouth the life and doings of Mah\^a Laksm\^i, the sweet juice as they are, I am as yet not satiated (and am desirous to hear more).

53. S\^uta said :-- Veda Vy\^asa, the son of Satyavat\^i, hearing his words addressed him in the following sweet words :--

54. ``O Best of Kuras! Very fortunate you are. I will describe in detail, to the best of my understanding, the origin of Her body.

55. Even Brahm\^a, Visnu, Mahe\'sa and Indra are never competent enough to describe Her form properly.

56. As I already told you that She sprung at the instant the word was spoken, how then can I ascertain the form or likeness of the Dev\^i.

57. She is constant, She is always existent; though She is one, yet She assumes different forms for the fulfilment of the Deva's ends, whenever their positions become serious.

58-59. Though the actor is one, yet for the entertainment of the spectators, he assumes different forms in the stage, so the Nirgun\^a Dev\^i, though formless, assumes in Her pastime, many different forms of S\^attvic, R\^ajasic or T\^amasic qualities, to fulfill the Deva's purposes.

60. There are various names given to Her, according as the works done by Her vary immensely in their natures, just as the meanings of one root vary, some being principal and some secondary, according to the meanings and objects they convey.

61. O King! I will now describe to you as far as my knowledge goes, the Excellent Form that came out of that mass of Celestial Light.

62. Her grand beautiful white lotus-like face was created out of the fiery energy of \'Sankara.

63. Her glossy black beautiful hairs of the head, overhanging to the knees, were formed out of the light of Yama; these all came to a fine pointed end.

64. Her three eyes came out of the energy of Fire; the pupils of those eyes were of a black colour; the middle parts were of a white colour and the ends were red.

65. The two eyebrows of the Dev\^i were black and came out of the spirit of Sandhy\^a (twilights); they were nicely curved and were looking spirited, like the bow of the Cupid and they were shedding, as it were, cooling rays.

66. From the light of V\^ayu (air), Her two ears were created; they were not very long, nor very short, beautiful like the swinging seat (rocking chair) of the God of Love.

67. Her nose was fashioned out of the fire of Kuvera, the Lord of wealth; it looked like the til flower, glassy and exquisitely charming.

68. O King! Her pointed rows of glossy and brilliant teeth, looking like gems, came out of the energy of Daksa; they looked like the Kunda flowers.

69. Her lower lip was deep red and it came out of the fire of Aruna (the charioteer of the Sun); Her beautiful upper lip came out of the energy of K\^artika.

70. Her eighteen hands came out of the Tejas of Visnu and Her red fingers came out of the Tejas of the Vasus.

71. Her breasts came out of the energy of Soma and Her middle (navel) with three folds was created out of the spirit of Indra.

72. Her thighs and legs were from Varuna and Her spacious loins came out from Earth.

73-74. O King! Thus from the various Tejas, contributed by the Devas, that Heavenly Lady came out. Her body and the several parts thereof were beautiful; Her form was incomparably graceful and the voice was exquisitely sonorous and lovely. The Devas, oppressed by Mahis\^asura, became overpowered with joy seeing this well decorated Dev\^i, having beautiful eyes and teeth, and charming in all respects.''

75. Visnu then addressed all the Devas to give all their auspicious ornaments and weapons, He said :-- ``O Devas! Better give, all you the various arms and weapons, endowed with strength, created out of your own weapons and give them all today to the Dev\^i.''

Here ends the Eighth Chapter of the Fifth Skandha on the description of the origin and the form of the Dev\^i in \'Sr\^imad Dev\^i Bh\^agavatam, the Mah\^a Pur\^anam, of 18,000 verses by Maharsi Veda Vy\^asa.