\chapter{On the Birth of the several Avat\^aras of Visnu and their deeds}

1-2. Janamejaya spoke :-- O best of Munis! How did Visnu, of wonderful deeds, get his incarnation owing to the curse, cast on Him by Bhrigu? what were His different incarnations in different Manvantaras respectively? O Thou, well versed in religion! O Br\^ahmana! Kindly narrate those sin-destroying deeds of Hari in His several incarnations, that are the source of happiness, peace and welfare to all humanity.

3. Vy\^asa said :-- O king! Hear, I am narrating to you the incarnations of \'Sr\^i Bhagav\^an Hari which He had in the several Manvantaras and in the several Yugas respectively.

4. I will tell you now, in brief, what forms He took and what deeds He did in the various incarnations.

5. In the Ch\^aksusa Manvantara, the Bhagav\^an Hari took the incarnation of Dharma; and the two sons of Dharma, Nara N\^ar\^ayana, were widely celebrated in this world.

6. Then, in the present Va\^ivasvata Manvantara, under the reign of Va\^ivasvata Manu in the second Yuga, Bhagav\^an Hari incarnated as Datt\^atreya, in the shape of the son of Atr\^i Risi.

7. Anasûy\^a the wife of Atr\^i, was desirous to have, as her sons, the three Devas Brahm\^a, Visnu and Rudra; and in fulfilment of her desires, the Devas took their births in her womb.

8. Anasûy\^a, was foremost amongst the chaste and virtuous women and on her praying, Brahm\^a, Visnu and Rudra the Trinity at once agreed to become her sons.

9. Brahm\^a was born as Soma, Hari was born as Datt\^atreya and Rudra was born as Durv\^as\^a.

10. In the fourth Yuga, the Bhagav\^an assumed the beautiful double form in one, the upper part resembling a lion and the lower part a human being to accomplish the noble purpose of the Devas.

11. It was to kill Hiranyaka\'sipu that the Bhagav\^an Hari assumed this appearance, wonderful even to the Devas.

12. In the Tret\^a Yuga, the superior and the best of all the Yugas, the Bhagav\^an incarnated as V\^amana ( the Dwarf), the son of Maharsi Ka\'syapa, to curb the power of Bali.

13. The Dwarf Hari took away by pretext, the kingdom of Bali, while he was performing a sacrifice and sent him down into the P\^at\^ala (the lower regions).

14. Afterwards, in the nineteenth Yuga, known as the Tret\^a Yuga, \'Sr\^i Bhagav\^an Hari incarnated as Para\'sur\^ama, very powerful and the son of Jamadagn\^i Risi.

15. He was very beautiful and graceful in his body, truthful and the conqueror of his senses. He extirpated the Ksattriya race and gave the whole world over to the high minded Risi Ka\'syapa.

16. O king! He is the Para\'sur\^ama, the sin-destroyer, the incarnation of Hari, and the doer of wonderful deeds.

17-20. After that the Bhagav\^an Hari incarnated as R\^ama, the son of Da\'saratha. Next in the twenty-eighth Dv\^apara Yuga, He incarnated as the very powerful Arjuna and \'Sr\^i Krisna, the Am\'sas of Nara N\^ar\^ayana. To remove the load of the earth, these two were born; and they fought deadly battles in the battlefield of Kuruksettra. O king! Thus the several incarnations of Hari arose, according to the requirements of Prakriti. O King! These three worlds are under the control of Prakriti.

21. Whatever the Prakriti wishes at any time, She can fashion the world in that way. And She does this incessantly in accordance with the Word Divine, the Highest \'Sakti, to please the Purusa, without any cessation.

22-23. In days of yore, the most ancient Bhagav\^an, the Highest, above all the qualities of M\^ay\^a, formless, all pervading, difficult to be conceived, without any decay, self-supporting, without any want, created these worlds, moving and unmoving and He manifested Himself as the Trinity, Brahm\^a, Visnu, Mahe\'sa in the shape of the three qualities S\^attva, R\^ajas and T\^amas, and which is called the Highest Prakriti.

24. This all auspicious Prakriti shines differently according to the differences in time and circumstances. This threefold Prakriti, the Great Enchantress of the world is creating, preserving the worlds and is destroying them at the end of the Kalpas.

25. O King! Whenever there takes place the union with this Prakriti, Brahm\^a creates, Visnu preserves, and the all-auspicious God \'Sankara destroys the worlds.

26. It was She That gave birth to K\^akutstha, the best of the kings; and to conquer the D\^anavas, She placed him at a certain place.

27. O king! Thus all men controlled by the Great Law in this world, enjoy sometimes the pleasures, enjoy sometimes pains and thus exist in the world.

Here ends the Sixteenth Chapter in the Fourth Book of \'Sr\^i Mad Dev\^i Bh\^agavatam, the Mah\^apur\^anam of 18,000 verses, by Maharsi Veda Vy\^asa, on the Birth of the several Avat\^aras of Visnu and their deeds.