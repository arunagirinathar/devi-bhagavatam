\chapter{On the questions asked by Janamejaya}

1. Janamejaya said :-- O Muni! You told before that the heavenly prostitutes sent by Indra in the hermitage of Nara N\^ar\^ayana became lustful and desired to live with N\^ar\^ayana only, whose heart was calm and quiet.

2. At that moment when N\^ar\^ayana was about to curse them, his brother Nara desisted him from taking that step.

3-4. Now I ask you what did that triumphant N\^ar\^ayana Muni do, in the critical juncture, when he was repeatedly asked by those prostitutes, sent by lndra, to satisfy their lust?

5. O Grand Sire! I am very eager to know the deeds of N\^ar\^ayana, leading to one's freedom. Kindly describe in detail and fulfill my wishes.

6. Vy\^asa said :-- Hear, O king! I am describing to you in detail, what that high souled son of Dharma did.

7. When N\^ar\^ayana Hari was ready to curse them, the Risi Nara, seeing this, consoled him and desisted him.

8. Then the great sage, the ascetic son of Dharma, N\^ar\^ayana, leaving aside his anger, began to address them in sweet words with countenance smiling.

9-10. O Fair women! We have determined to practise asceticism in this life; it does not therefore behove us to accept any wife; therefore shew your kindness unto us and go back to your Heaven. You would better think that those who know what is religion, they never desire to break the vow of another.

11. O beautiful eyed ones! In the sexual pleasure, it is the delightful feeling of passionate joy that is requisite; and we are wanting in those feelings; then how can we effect that union?

12. No action can come out of no cause; this is all clear. The poets say that the sexual power and pleasure, is the feeling, the mental attitude that corresponds; and that is the only thing, that lasts. And we have no desire for that.

13. However my limbs are all very graceful, I am very fortunate and blessed in this world, otherwise how can I be the object of your sincere love towards me.

14. You all are very fortunate; therefore do now show this mercy unto me ``do not break my vow.'' I pray now that, in a subsequent birth, I may become your husband.

15-16. O large eyed fair women! In the twenty-eighth Dv\^apara Yuga, I will certainly incarnate on earth to effect the purpose of the Devas; then you all also would respectively incarnate as the daughters of kings and would also become my wives.

17. N\^ar\^ayana thus consented to marry them in some other next birth; and consoling them, made them go back to their Heavens. They also abandoned their mental disquietude and, on reaching back to Heavens, they explained everything to Indra.

18-19. Indra heard (from these heavenly women) what the two two Risis did and saw before him Urva\'s\^i and other women created by N\^ar\^ayana from his thighs, etc., and began to extol the merits of the high souled N\^ar\^ayana.

20. Indra said :-- O! How wonderful is the patience of the Muni? What is the wonderful influence of his Tapas! Oh! He has created, by the sheer force of his Tapas, Urva\'s\^i and these fair women, unrivalled for their beauties, from his thighs.

21. The Lord of the Devas thus extolled his merits and became freed from his anxieties. The virtuous N\^ar\^ayana, too, devoted himself to the practice of his Tapasy\^a.

23. O king! Thus I have described to you, in detail, all the wonderful accounts regarding Nara N\^ar\^ayana.

O Superior in the descendants of Bharata! These two Nara and N\^ar\^ayana afterwards incarnated themselves, due to Bhrigu's curse as the two great heroes Arjuna and Krisna, to relieve the burden of the earth.

24. The king said :-- O respect giving Muni! Now describe in detail the life of the Avatar Krisna and dispel my mental doubt.

25-26. O best of the Munis! Why were V\^asudeva and Devak\^i, who were chosen by the very powerful Hari and Ananta as their parents, doomed to so many miseries and afflictions. Why had these parents to remain for good many years in the prison of Kamsa, who pleased directly by their Tapasy\^a that Bhagav\^an Jan\^ardana.

27. Why did Krisna taking his birth at Mathur\^a, go to Gokula? Also what was his object to go to Dv\^ark\^a, situated in the ocean, when he killed the enemy Kamsa?

28. Also why did his father, mother and relatives, leave their old holy places of residences and go abroad to live in a wretched old country?

29-31. Why was the Yadu race destroyed by the curse from a Br\^ahmana! How did \'Sr\^i Krisna V\^asudeva leave finally His body after He had relieved the burden of the earth and was about to enter into His Heaven? The evildoers of the earth were slain by Krisna and Arjuna, of unequalled prowess;  but how was it, that those who plundered the wives of \'Sr\^i Hari, were not at all punished by Him?

32-33. The great personages Bh\^isma, Drona, Karna, the king V\^alh\^ika, Vir\^ata, Vikarna, Dhristadyumna, the king Somadatta were destroyed for relieving the burden of the earth; and the plunderers were acquitted! Kindly explain the cause of this.

34. How was it that those chaste and virtuous wives of \'Sr\^i Krisna go into troubles at the latter end of their lives? There has arisen a doubt in my mind on this point.

35. Why did the virtuous V\^asudeva leave his mortal coil owing to the death of his sons and why did he die an unusual death?

36. O best of Munis! The P\^andavas were devoted to Krisna and they were religious; they had to suffer so many troubles!

37. Why was Draupad\^i so very unfortunate and she had to suffer so much miseries, and pains, who was born of Laksm\^i from amidst the sacrificial place and from the altar.

38-39. Why did Duh\'s\^asan drag Her by Her hairs while She was in Her menstruation period, in the hall of audience and why was it that Sindhu R\^aj Jayadratha, the king of Sindhu, gave Her exceedingly mental troubles?

40. Why was it that Her five sons residing in Her house were killed by A\'svathth\^am\^a? What was the cause that the son of Subhadr\^a had to die in the battlefield?

41. Why did the king Kamsa kill the six sons of Devak\^i; and why was it that \'Sr\^i Hari who was capable of averting the Fate did not at all prevent that?

42. What a wonder is this that in the matters of Br\^ahmana's curse toward the J\^adavas, their being killed in the Prabh\^asa, the total extermination the Jadu race and the plundering of His wives, why did He allow Fate to do these great momentous things?

43. If He was the all-powerful God and He Himself N\^ar\^ayana, that why did He incessantly act like a slave towards Ugrasena.

N.B. -- Ugrasena was the king of Mathur\^a and father of Kamsa. He was deposed by his son; but Krisna after having slain Kamsa restored him to the throne.

44-45. All these bring doubt in our minds regarding N\^ar\^ayana Muni that His deeds are always like those of ordinary persons; why did his pleasures and pains resemble those of ordinary human beings? Were he God, why his actions were not Godly? (i.e., superhuman)

46. Therefore dost Thou describe in detail all the Divine Leelas (playful sports) done by Hari of superhuman powers in this world.

47. O Best of Munis! When one's longevity expires, one dies; then I cannot understand what glory was manifested by Hari in killing the Daityas? For Fate Killed them; not Hari.

48. Was not the doing of Hari like a thief when he stole away the Lady Rukmin\^i and fled quickly to his own place.

49. What did it mean when he fled to Dv\^ark\^a city, and quitted his own highly prosperous town Mathur\^a simply out of the fear of Jar\^asandha?

50. Did not anybody at that time recognise that he was \'Sr\^i Bhagav\^an Hari? O Respected One! Were he Bhagav\^an, why did He hide himself in Vraja? Please explain the cause to me.

51. O Muni! These and many other doubts always exist in my mind; you are the best of the Dvijas and blessed; I pray, dost thou remove these doubts.

52-53. O best of Munis! Another doubt exists and is not dispelled and that is secret. Was not the taking of the five husbands by P\^anch\^al\^i for herself shameful and despised by the society? The good manners and doings are always considered by the learned as the proofs of virtue. Why did those P\^andavas, then, capable in every respect, do this thing like brutes?

54. And what did Bh\^isma do living like a Deva in this world? May I ask, was his act of producing two sons by a widow and thus preserving his line of ancestors worthy of his name?

55. The religious sanction advocated by the Munis ``Procreate sons in any way whatsoever'' is simply shameful. Fie to this religious sanction.

Here ends the 17th Chapter in the 4th Book of \'Sr\^i mad Dev\^i Bhagavatam of 18000 verses by Maharsi Veda Vy\^asa on the questions asked by Janamejaya.

