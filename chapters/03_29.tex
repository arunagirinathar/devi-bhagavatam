\chapter{On the stealing of S\^it\^a and the sorrows of R\^ama}

1-2. Vy\^asa said :-- Hearing these vicious words, J\^anak\^i became very much confounded with fear and began to tremble; somehow collecting herself she began to say :-- ``O descendant of the family of Pulastya! Why are you, prompted by lust, uttering these sinful words? I am born of the family of Janaka; therefore I cannot act wantonly according to my own inclination.

3. O ten faced one! Better you go to Lank\^a quickly; else R\^amchandra will take away your life; you will no doubt incur death for my sake.''

4-5. Thus saying, S\^it\^a Dev\^i went towards the Sacred fire called G\^arhapatya, placed in the house, with words ``go away'' ``go away'' in her mouth. He, whose wickedness has caused all the Lokas cry out ``save'' ``save'', the same R\^avana, of perverted intellect, then assumed his real form, went towards the hut and caught hold of S\^it\^a Dev\^i who was crying, bewildered with fear.

6. S\^it\^a cried ``R\^ama'' ``R\^ama'' ``Laksmana'', and the sinful R\^avana caught hold of her and quickly mounting her on the chariot, fast got away.

7-9. On the way Jat\^ayu, the son of Aruna, met R\^avana; and a terrible fight then ensued between the two, when the evil minded R\^avana, the king of the Demons, killed Jat\^ayu. R\^avana carried S\^it\^a to Lank\^a. Then S\^it\^a cried like a forlorn deer and R\^avana kept her in the A\'soka forest (Jaffna),

surrounded and guarded by the R\^aksasis. The king of Lank\^a tempted S\^it\^a with comforting words, and the kingdoms, etc., but she never swerved from her own pure and stainless chastity.

10-12. On the other side, R\^amachandra after killing the deer and taking it was coming back calmly, when he saw Laksmana going to him and said ``O Laksmana! What a great blunder you have committed! Hearing the voice of that villain conjurer, how is it that you have left my dear S\^it\^a alone and come here!'' Laksmana said :-- ``O Lord! Being pierced sharply by S\^it\^a Dev\^i's words (coming like sharpened arrows) and being driven away by her, under the guidance of the Inevitable Destiny as it were, I have come here; there is no doubt in this.''

13. They, then, both hurriedly went to their hut, made of leaves; and there not finding S\^it\^a, they were very much afflicted with sorrows and went in quest of J\^anak\^i.

14. R\^ama and Laksmana in their search for S\^it\^a, came at last to the spot where Jat\^ayu, the king of birds, was lying on the surface of the earth, with his life ultimately on the point of parting away from his body.

15-16. Jat\^ayu said :-- R\^avana, the king of Lank\^a, carried away today stealthily S\^it\^a Dev\^i; I resisted that villain who then fought with me on that account and threw me down on this spot by weapons. Thus saying, the king of birds died; whereupon R\^amchandra performed the burning of his dead body as well his funeral ceremonies. Then both of them went out of that place.

17. Then the Lord R\^amachandra killed Kabandha and freed him from his curse; and, through his advice, he made friendship with Sugr\^iva, the king of the monkeys, and was thus bound under a tie.

18. Next R\^ama killed the hero Bal\^i as a duty and gave the excellent kingdom of Kiskindhy\^a to his new friend Sugr\^iva according to his promise.

19. Then, he began to ceaselessly think of the stealing away of S\^it\^a by R\^avana and passed away the four months of the rainy season there with his brother Laksmana.

20. R\^ama, being very much shaken on account of the bereavement of S\^it\^a, began to address Laksmana thus :-- ``O Saumitre! The desires of the daughter of the king of Kekaya are now fulfilled.

21. J\^anak\^i will no more be obtained; without J\^anak\^i I will not go back to Ayodhy\^a; without J\^anak\^i I won't be able to live any longer.

22. Kingdom lost, dwelling in forests happened, father left his body, at last the dear wife is lost; the cruel hands of Destiny are tormenting me now thus; what more it will inflict, how can I say now?

23. O Brother Laksmana! What is to happen is very hard to be known beforehand by men; I cannot say, what is written on my fate after this, painful or otherwise.

24. See! Both of us, the descendants of Manu, though born in a royal family, are exiled in forests due to our past deeds.

25. O Laksmana! It is by Fate, too, that you, abandoning the pleasures of the royal surroundings, have come out with me; and you, too, are now suffering heaps of dire troubles with me.

26. No one in our family suffered so much as we are suffering; why we talk of our family! No human being was ever born or will ever take his birth that suffered or will suffer like me so many troubles, will be like me incapacitated and a penniless pauper.

27. O Saumitre! I am drowned in the ocean of pains and troubles; What am I to do now? I have no means to cross this ocean; I am quite helpless, no doubt.

28. No money, nor armies, O hero! you are my one and only one companion; O brother! On whom shall I be angry when I am suffering on account of my own deeds?

29. Alas! The kingdom that could have been compared in prosperity to the Indra Sabha, was almost obtained by me when, in an instant, I lost it and am now in exile in forest. Laksman! Who can ascertain what is in the womb of Destiny?

30. Oh! That soft bodied S\^it\^a, with her child like nature came out with us in this forest; but the inexorable Fate has now drowned her, that perfectly beautiful woman, into an ocean of sorrows, difficult to be crossed?

31. That fair daughter of Janaka is extremely devoted to me; she is pure and holy. How will she be able to suffer troubles in the house of the king of Lank\^a!

32. O Laksmana! S\^it\^a Dev\^i will never come under the control of R\^avana; how can that excellent chaste woman act like an ordinary public woman?

33. O Laksmana! Rest assured that in case R\^avana exercises, out of his lordly position, any violence on S\^it\^a, she will rather put an end to her life than come under his control.

34. O Laksmana! And when J\^anak\^i sacrifices her life, I will assuredly do the same; for, of what use, then is this body to me when that fair S\^it\^a has gone away with her life?''

35. While the lotus eyed R\^amchandra was thus weeping and expressing his regrets and sorrows, the religious Laksmana consoled him with the following sweet, truthful, words :--

36. ``O Hero of the heroes! Kindly cast aside this weakness and have patience; I will soon kill that villain demon R\^avana and get you back your S\^it\^a Dev\^i.

37. The wise steady persons remain on account of their fortitude, unshaken in their hearts whether in joy or in sorrow; whereas men, of little intellect, indulge in sorrows when they are happy.

38. Coming in union and going out in disunion, both are under the hands of Destiny; What, then, there is the need for expressing sorrows for this body, which is not soul.

39. As we have been banished from our kingdom into this forest, as there has happened this bereavement of S\^it\^a, so, in proper time, we will again get back S\^it\^a Dev\^i.

40. O Darling of J\^anak\^i! There must come a time when sorrows will be converted into happiness and vice versa; there will be nothing otherwise. So avoid this sorrow now and have firmness.

41. There are multitudes of monkeys, who are our helping hands; they will go to all the four quarters and bring back to us the news of the daughter of Janaka; there is no doubt in this.

42. O Lord! Knowing the way to Lank\^a, we will go there and kill by our prowess the villainous R\^avana and bring back S\^it\^a Dev\^i.

43. Or we will call Bharata with Satrughna and with all the armies we all united will kill our enemy; why, then, are you thus expressing sorrows in vain.

44. O Lord! our ancestor Raghu, the hero of heroes, the monarch; won his victories over the ten quarters; and you belong to that family and are now plunged in grief!

45. Alone, I can defeat all the Devas and the Demons; and if I get help, is there any doubt, then, in my killing, that R\^avana, the disgrace of the family of R\^aksasas.

46. O Powerful One! We may call to, our aid the king of Janaka and root out that wicked source of enemy to the Devas.

47-48. O Descendant of Raghu! Like the rim of a wheel, happiness and pain come alternately; it is not that happiness, or pain comes and remains for ever. He whose mind is very much overwhelmed. with pain or happiness, is the man who is always plunged in an ocean of misery; and he can never expect to become happy.

49. See! In days of yore, Indra once got addicted into vicious habits. The Devas united put in place of Indra, the king Nahusa.

50. Then Indra, terrified, relinquished his post and passed very many years into an unknown and unnoticed state within the lotus.

51. Again, when time changed, he got his own post back; and the king Nahusa fell down on this earth and became transformed into a boa constrictor (a big serpent), through the curse of a Risi.

52. The king Nahusa wanted the wife of Indra and insulted a Br\^ahmin; therefore, he was, under the curse of Maharsi Agasti, transformed into a snake on the earth.

53. Therefore, O R\^aghava! One ought not to plunge in grief, when a danger comes; rather one should be quite energetic in times of danger and remain firm; thus, the sages do.

54. O Lord of the world! You are high minded, omniscient and omnipotent; why are you now overwhelmed with grief, like an ordinary mortal.''

55. Vy\^asa said :-- Oh king! Thus consoled by Laksmana, R\^ama discarded all his heavy sorrows and began to remain with his heart firm and at rest.

Thus ends the 29th chapter on the stealing of S\^it\^a and the sorrows of R\^ama in the 3rd Skandha of \'Sr\^i Mad Dev\^i Bh\^agavatam of 18,000 verses by Maharsi Veda Vy\^asa.