\chapter{On the characteristics of the Gunas}

1-2. N\^arada said :-- Father! You have described to me the characteristic of the three qualities; though I have drunk the sweet juice from your lotus like mouth, still I am not quite satisfied. Kindly describe to me, in detail, in due order, how I can recognise clearly the three qualities so that I can get the highest peace of mind.

3. Vy\^asa said :-- O King! The Creator of the world, Brahm\^a, originated from the Rajo Guna, asked by his high minded son N\^arada, began to speak in the following terms.

4. O N\^arada! I myself do not possess fully the complete knowledge of the three qualities; but, as far as I know, I am telling that to you.

5. The pure Sattva quality is not found alone to exist anywhere; it manifests itself always, in mixed condition, in combination with the other qualities.

6-9. As a beautiful woman, well decorated with ornaments and endowed with amorous gestures, gives delight, on the one hand, to her husband, father, mother and friends; and, on the other hand, becomes a source of pain and delusion to her rival wives, so the Sattva quality, personified as a beautiful woman, engenders the S\^attvic happiness of the mind to some individual, at one time, and at another time becomes a source of pain to the same individual (or at one and the same time becomes a source of happiness to one and a source of pain to another.) Thus the Rajas or the Tamas quality, personified respectively as a beautiful woman becomes a source of pain or delusion to an individual at one time, and at another time, a source of happiness to the same man. So it is easily seen that one quality cannot remain single; it remains in union with the other qualities.

Note :-- It is very possible that a man, possessing the S\^attvic quality at any time, can be said not to possess only the S\^attvic quality but also the Rajas and the Tamas to a certain degree. At any subsequent time the Rajas might get preponderance, and that man may be in circumstances requiring money or so forth; but, due to his Sattva quality before hand he did not collect money and therefore he feels pain afterwards. So with the Rajas. Or it may be thus :-- Suppose an earning member is S\^attvic. He earns just sufficient to meet his wants. But his family members require more money, for they are R\^ajasic. Therefore the earning member is happy for his S\^attvic quality; but the other members are unhappy for his S\^attvic quality. A man is, as it were, wedded to the three wives, Sattva, Rajas, and Tamas.

10. O N\^arada! When the three qualities remain each in their own real natures, then the effects produced by them also remain always the same; no changes are perceived owing to the difference of time or person. But when they get combined, then each of them produces effects sometimes counter to their natures.

11-13. A young beautiful woman, shy, modest and of sweet qualities, well versed in her religious learning, and full of good behaviour, skilled in love practices and full of sweet sentiments becomes a source of loving delight

to her beloved and also a source of pain to her rival wives so each of the three qualities assume no doubt, different aspects according to differences in time and in the nature of the person.

O N\^arada! As one woman gives pain and delusion to her rival wives and gives pleasure to her husband and friends, so the Sattva quality, when perverted, gives pain and delusion to the persons.

14-19. As the police sepoys and constables are, on the one hand, delight to the saints, troubled by thieves, and, on the other hand, sources of pain and confusion to the thieves and robbers; again as the heavy shower of rain in a pitch dark night, in the rainy season, when the sky over clouded, and when there are flashes of lightning and thunder, is on the one hand, a source of highest delight to a farmer, who has all seeds and necessary things and implements, and, on the other hand is a source of pain to the unfortunate householder, whose house is not yet completely thatched with grass or who has not been able to collect his beams and grass for necessary roofing, and a source of utter bewildering confusion to the young woman, whose husband is abroad expected back at that time, so the three Gunas produce contrary results when perverted by contact with the remaining Gunas, instead what they would have produced, had they not been perverted so.

20-25. O Child! Again I speak to you of the characteristics of the the Gunas. The Sattva guna is pure, clear, illumining, light (not heavy) white. When the senses, eyes, etc., and the limbs are felt very light (without any heaviness) and the heart and brain clear, when there is dispassion towards the R\^ajasic and the T\^amasic enjoyments, know then that the Sattva quality has grown in preponderance in a body. When there is a tendency to yawn, when there is rigidity and suppression of the functions of faculties and when one feels drowsiness, consider that the R\^ajasic quality has gone to excess. Again, when one seeks after quarrels and goes to another village, one is always restless and ready to fight, when one feels heaviness in body, as if wrapped by a very heavy darkness, when one's limbs and senses are heavy and obscure, when one's mind is vacant, and when one does not like to go to sleep, know that the Tamas has increased too much, N\^arada!

26. N\^arada said :-- O Father! You have described the different characteristics of the three Gunas; but I cannot understand how they act all in conjunction?

27. As those who are enemies to one another do not work united, so these Gunas, of opposite characteristics, are enemies, as it were, to one another; how can, then, they act in unison? Kindly explain this to me.

28-30. Brahm\^a said :-- O N\^arada! The three Gunas may be likened to a lamp. As a lamp manifests a certain object, so these three qualities united do manifest or reveal a certain thing. See the wick, oil, and flame are all of different characteristics; though the oil goes against fire, still it unites with the fire. The oil, wick and fire though running against each other, all these united, serve the one common purpose of illumining, revealing a certain object.

31. So, O N\^arada! All the three qualities, though of contrary natures, go to prove the same thing.

N\^arada said :-- O Son of Satyavat\^i! The lotus born Brahm\^a thus described the three qualities, as born of Prakriti; and they are the causes of this Universe. What I heard of you about the nature of Prakriti, I have now described before you.

32. Vy\^asa said :-- O King! What you asked me, I asked before the same to N\^arada and he described thus (as I told you above) to me about the characteristics and the effects of the three Gunas in regular order and in detail.

33. O King! Wherever in the \'S\^astras whatever is said, the essence of all that is this -- that the Highest Energy, the Supreme Force, the Great Goddess who is pervading the Universe, is always with qualities and without qualities, according to the differences in the manifestation. This Supreme Force is to be worshipped with the highest devotion.

34. The Brahm\^an, the Purusa (the Supporter, the Ultimate Substratum) the Highest Energy considered as the Male Principle though It is Undecaying, Supreme and Full, is still without any desires or emotions. It is not able to accomplish any action (without the help of its inherent force); this Mah\^am\^ay\^a, the Supreme Force is doing all the functions, real and unreal, of the universe.

35-37. Brahm\^a, Visnu, Rudra, the Sun, Moon, Indra, the twin Asvins, the Vasus, Visvakarm\^a, Kuvera, Varuna, Fire, Air, Pûs\^a, the Sad\^anan, and Ganesa all are united with \'Sakti and can do their respective functions; else they are unable to move themselves. Therefore O king! Know that Supreme Goddess Mah\^am\^ay\^a as the cause of this Universe.

38. O Lord of men! You worship this Goddess, perform sacrifices in honour of Her and worship Her with the highest devotion.

39. O king! That Mah\^am\^ay\^a is Mah\^a Laksm\^i, She is Mah\^a K\^al\^i, She is Mah\^a Sarasvat\^i; She is the Goddess of all the bhûtas and She is the Cause of all causes.

40. That all peaceful, easily worshipped and the ocean of mercy, when

worshipped, fulfills all the desires of Her devotees; what to say, the mere utterance of Her name is sufficient for the granting of the desires.

41. In days of yore Brahm\^a, Visnu, Mahe\'svara and all the Devas and many other self controlled ascetics worshipped Her to attain liberation.

42. O king! What shall I speak now about Her more than this :-- If one takes Her name even with indistinctness, She grants the desired purposes, even if they are quite unattainable.

43. In the midst of forest, on the sight of tigers and other ferocious animals, if one becoming afraid, cries aloud Her seed mantra (twice) ``Ai, Ai'' without the Vindu (incorrectly) instead of ``Aim, Aim'' She grants immediately his desires.

44-45. O best of kings! There is an example of Satyavrata on this point. That the mere utterance of the name of Bhagavat\^i gives unforeseen results, has been witnessed by us and other high minded Munis. Also in the assembly of the Br\^amanas I have heard fully many sages quoting in detail many instances on the above point.

46-47. O king! There was a Brahm\^an, named Satyavrata, quite illiterate, a thorough block-head. Once he heard the letter ``Ai, Ai'' being uttered by a pig; and in course of a talk he himself uttered incidentally that letter and thereby became the one of the best Pundits.

N. B. – ``Aim'' is the seed mantra of Sarasvat\^i, the Goddess of learning.

48. The Goddess Dev\^i, the Ocean of mercy, hearing the letter ``Ai'' being pronounced by that Brahmin, became very glad and made him the best of the poets.

Here ends the Ninth Chapter of the 3rd Skandha on the characteristics of the Gunas in \'Sr\^imad Dev\^i Bh\^agavatam, the Mah\^a Pur\^anam 18,000 verses by Maharsi Veda Vy\^asa.