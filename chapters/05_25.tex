\chapter{On the killing of Dh\^umralochana}

1-7. Vy\^asa said :-- O Janamejaya! When Dh\^umralochana ceased speaking, the Dev\^i K\^alik\^a made a wild laughter and began to speak sweetly thus :-- O Stupid! Skilled in flattery, you know only how to use jugglery of words like an actor; do you think that your ends will be served if you speak only sweet words; this can never be. O Stupid! Fight now; there is no need of useless words. You are strong and have been sent by that wicked Demon with a great army. This Dev\^i, out of wrath, will kill you, \'Sumbha, and Ni\'sumbha and other commanders by Her arrows and will then return to Her abode. Where is that stupid \'Sumbha? And where is this Dev\^i, the Great Enchantress of the Universe!

Their marriage in this world is entirely out of question and can never take place. O Stupid! What do you think that a lioness becoming very passionate, would make an ordinary jackal her husband? or would a she-elephant prefer an ass? or would a heavenly Cow like a bison? Go to \'Sumbha and Ni\'sumbha and tell truly to them :-- ``Fight or go instantly to P\^at\^ala.''

8-10. Vy\^asa said :-- O Fortunate One! The Demon Dh\^umralochana, hearing thus the K\^alik\^a's words, became very angry and spoke with reddened eyes :-- ``O Ugly One! I will slay Thee and this lion infatuated with pride in battle and take this Fair One to the king. O K\^al\^i! I have not been able to do this, simply it would break our amorous love sentiments. O Quarrelsome One! Otherwise I would have undoubtedly slain Thee just now with my sharpened arrows tipped with irons.''

11. Hearing thus, K\^alik\^a said :-- O Fool! Why do you boast vainly? this is not the religion of a hero with bows and arrows in their hands. Shoot your arrows with all your might; I will send you to the realm of Death.

12-31. Vy\^asa said :-- O King! Hearing the Dev\^i's words, Dh\^umralochana caught hold of his very strong bow and began to shoot arrows after arrows at K\^alik\^a. Indra and the other Devas came out to see the fight on their best cars in the celestial space and shouted ``Victory to the Dev\^i'' and thus eulogised Her. Then a deadly fight ensued between them with arrows, axes, clubs, \'Saktis, and Musalas and various other weapons. K\^alik\^a cut off at the very outset all the asses that carried the chariot by Her arrows and next broke his chariot and began to laugh repeatedly. O Bh\^arata! Then Dh\^umralochana becoming angry mounted on another chariot and began to shoot deadly arrows at K\^alik\^a. K\^alik\^a Dev\^i, too, out off those arrows into pieces before they reached Her and shot arrows after arrows on the D\^anava in quick succession. Thousands of his soldiers near to him were killed; the asses and the charioteer were killed and the chariot was broken. She cut off his arrows by Her swift serpent-like arrows and blew Her conchshell. The Devas seeing this became very glad. Dh\^umralochana, seeing himself displaced from his chariot, took up with anger his very strong Parigha weapon and came near to the chariot of the Dev\^i. Then the D\^anava, terrific like death, began to abuse the Dev\^i and said :-- ``O Ugly tawny-eyed K\^al\^i! I will kill Thee just now.'' Thus saying, he suddenly went near to Her and when he was about to throw his Parigha weapon on Her, the Ambik\^a Dev\^i burnt him to ashes simply by Her loud shout (of defiance). Seeing Dh\^umralochana burnt to ashes, his soldiers became panic-stricken, and fled away immediately, crying aloud ``O Father! O Father!'' The Devas saw this and and gladly showered from high heaps of flowers on the Dev\^i. O King! The battle ground then assumed a dreadful appearance; at some places the slain D\^anavas; at others, the horses; at other places elephants and at some other places the asses lay scattered on the field. The herons, crows, vultures, the Pi\'s\^achas of the class Batabaraphas and jackals and other carnivorous animals, began to dance wildly and clamour hideously at the sight of the dead bodies, lying on the field. The Ambik\^a Dev\^i then quitting the field, went to a distant place and blew Her conchshell so furiously and terribly that \'Sumbha heard that terrific noise, while he was sitting in his own residence. At the next moment, he saw that the D\^anava forces had retreated, and they were coming there crying. Some of them were besmeared with blood all over the bodies; some had got their feet, some their arms, cut asunder, some were devoid of eyes, some had got their backs broken; some had their waists broken; some got their necks broken and some were going on bedsteads. Seeing them thus, \'Sumbha and Ni\'sumbha asked them :-- ``Where is Dh\^umralochana? Why have you all retreated? And why have you not brought that Lady? Where are the other forces? Who has blown this horrible conch-shell? O Fools! Inform me quickly and truly all these things.''

32-33. The soldiers said :-- ``O King! Dh\^umralochana has been slain by K\^alik\^a; She has destroyed all the soldiers and has done extraordinary deeds.'' O King! Know the blowing of the conchshell that has caused terror in the hearts of the D\^anavas and has enhanced the joy of the Devas and is being resounded in the celestial space, is that done by the Ambik\^a Dev\^i. (Note: In the M\^arkandeya Pur\^ana, Ambik\^a killed Dh\^umra.)

34-45. O Lord! When the Dev\^i broke the chariot of Dh\^umralochana by the multitude of Her arrows and killed the horses and at last slew Dh\^umralochana himself, when all the forces were slain by Her who appeared like a lion and when the rest of the army retreated, the Devas seeing all these were very much gladdened and showered flowers from the celestial sky. O King! We have come to a perfect conclusion that we will not get the victory; now consult with your expert ministers and do what is needful. O King! The Supreme Goddess of the Universe is waiting there alone to fight with you without any help of any other forces; this is a great wonder to us. O King! Intoxicated with Her power, that Girl, fearless, is reigning there taking Her stand on the lion. All these seem wonderful to us. O King! Consult with your councillors and out of the four policies peace, fight, retreat or remaining neutral, accept what is best. O Tormentor of the foes! True! There are no forces with the Dev\^i, but the whole host of the Devas will take up Her cause in crisis, there is no doubt. In due time, Hari and Hara both will come and assist Her; now the guardians of the several quarters, the Lokap\^alas are waiting by Her side in the celestial space. O Tormentor of the Gods! Know that the Gandarbhas, Kinnaras, and human beings all will come timely and help Her. O King! We guess all these. But that Lady does not want the assistance of anyone nor does She expect that any other body would do the work for Her. You must know this certainly, that She alone can destroy this whole Universe. What to speak of the D\^anavas only! O Highly Fortunate One! Knowing all these, do as you like. It is the duty of the servants to speak beneficial and at the same time true words with moderation.

46-51. Vy\^asa said :-- O King! \'Sumbha, the tormentor of others, hearing their words asked his younger brother in private :-- ``O Brother! This K\^alik\^a has slain today Dh\^umralochana with his forces; the few retreated and came over to me. Now the Ambik\^a Dev\^i, puffed up with pride is blowing Her conchshell. Brother! The ways of Time are knowable even to the wise. The grass becomes a thunderbolt and the thunderbolt becomes like a grass and powerless. Know thus the course of Destiny. O Fortunate One! Now I ask you, what are we to do now? Are we to entertain yet the desire of enjoying Ambik\^a, or are we to fly away from here or are we to fight on? Say quickly. Though younger, in times of difficulty, I consider you as my elder.''

52-54. Hearing thus the \'Sumbha's words, Ni\'sumbha said :-- ``O Sinless One! Flight or taking refuge in a fort is not reasonable. To fight with this Lady is the best course. I will take the best generals and soldiers with me and will slay that Lady and quickly return. And if Fate be strong and prove it otherwise, then, after my death, think out again and again and do what is best.''

55-60. Hearing thus the younger brother's words, \'Sumbha said, ``You better wait; let Chanda and Munda go to the battle, surrounded with forces. To kill a hare it is not necessary to send an elephant. This is a very trifling matter; the two great warriors Chanda and Munda will be freely able to slay Her.'' Thus saying his younger brother, the King \'Sumbha addressed Chanda Munda, who were waiting before him, thus :-- O Chanda! O Munda! Take your forces and go quickly to kill that shameless Lady, puffed up with pride. O Pair of Warriors! Kill that tawny-eyed K\^alik\^a in the battle and bring that Ambik\^a Dev\^i here quickly. Do this Great Service. And if that haughty Ambik\^a be unwilling to come here, though taken as a captive, then kill that Durg\^a, the ornament of the battle, too, by sharp arrows.

Here ends the Twenty-fifth Chapter of the Fifth Book on the killing of Dh\^umralochana in \'Sr\^i Mad Dev\^i Bh\^agavatam, the Mah\^a Pur\^anam, of 18,000 verses by Maharsi Veda Vy\^asa.