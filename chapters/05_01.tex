\chapter{On the superiority of Rudra over Visnu}

1-5. The Risis said :-- ``The great legendary story, the life of \'Sr\^i Krisna, supremely divine, destructive of all sins, has been narrated by you, O S\^uta! But, O Blessed One! You, though highly intelligent, have dwelt on it not at a great length; hence many doubts are cropping up in our minds. A very difficult tapasy\^a was performed by V\^asudeva the part incarnate of Visnu, who had to go to forest to worship \'Siva. Next, it has been known that the Dev\^i P\^arvat\^i, the part incarnate of the Great Mother, the Mother of the universe, the Supreme, and Perfect offered boons to \'Sr\^i Krisna. How did it then come to pass that \'Sr\^i Krisna, being himself the God, had to worship P\^arvat\^i and Mah\^adeva? Is it that \'Sr\^i Krisna was inferior to Mah\^adeva and P\^arvat\^i? This is our doubt.''

6-7. S\^uta spoke :-- ``Hear then, the reasons, O noble Risis! that I heard from Vy\^asa; I will now sing before you those meritorious deeds \'Sr\^i Krisn\^a.'' The son of Par\^iksit, the intelligent Janamejaya had also the same doubts that you now have, when he heard the story before from Vy\^asa; and he asked the same questions that you now ask.''

8-11. Janamejaya said :-- ``O son of Bhagavat\^i! I have heard from you much about the Supreme Goddess, the Highest Cause; still the doubts are not leaving me. O Fortunate One! Krisna the Deva of the Devas, the Visnu incarnate, worshipped Sambh\^u and had to perform dire penances; this is my great wonder! He is the soul of all the J\^ivas, the One Ruler and Lord of this world and He is able to confer all the Siddhis; how is it, then, that the Lord Hari had to perform very difficult asceticism like an ordinary mortal. He who is able to create this universe, moving and non-moving, He who is able to preserve and destroy it, why did He practise such a terrible penance.''

12-54. Vy\^asa said :-- ``True it has been said by you that V\^asudeva the Jan\^ardana, is the destroyer of the Daityas and He is able to create and preserve the Devas and do all other acts for them. But the Great Lord assumed a human body; therefore he had to perform his duties like a man and observe the Varna and \^A\'srama Dharmas pertaining to human beings. Respecting the elderly persons, worshipping the spiritual teachers, doing service to the Brahm\^anas, adoring and propitiating the Devas, feeling sorrow at times of sorrow, feeling pleasure at times of happiness, feeling dejection or expressing censure or scandal, or having sexual intercourse with women, in other words, to feel lust, anger, greediness and other passions when their proper time arises. All these are natural to all human beings; how can, then, \'Sr\^i Krisna though intrinsically of pure qualities, become Nirguna (devoid of human qualities) when he assumed a human body which is Saguna, i.e., with qualities. O Ruler of men! The extinction of the Y\^adava race by the curse of G\^andh\^ar\^i, the daughter of Subala, and the curse of a Br\^ahmin, Krisna's leaving his human coil, the stealing away of his wives, the robbing of their wealth on the way by the dacoits of the \^Av\^ira tribe, Arjuna's becoming powerless to hurl any weapons on those dacoits, Krisna's not knowing anything about the stealing away of Pradyumna and Aniruddha from his Dv\^ark\^a palace, these all correspond verily to exertions and failings appropriate to human bodies. Again the Risi N\^ar\^ayana is the part incarnate of Visnu, and V\^asudeva is the part incarnate of the Risi N\^ar\^ayana; hence what wonder is there, if V\^asudeva be seen to adore and propitiate \'Siva? \'Siva is the God of gods; and He is the Lord of all the causal bodies that exist; in the state of Susupti (deep sleep). In this respect, \'Siva is the creator of Visnu and Visnu worships Him in this light. R\^ama, Krisna and others are all part incarnations of Visnu; so there is no wonder if they worship \'Siva. The letter A is Bhagv\^an Brahm\^a ; the letter ``U'' is Bhagv\^an Hari; the letter ``M'' is Bhagv\^an Rudra and the half letter m is Mahe\'svar\^i, the Supreme Mother of the universe. The sages, therefore, consider Visnu superior to Brahm\^a; they again consider Rudra superior to Visnu and M\^ahe\'svar\^i (Tur\^iya State) again superior to Rudra. The speciality of the half letter is that it can never be uttered; it is the symbol of the Eternal Dev\^i. In all the \'S\^astras, therefore, the superiority of the Dev\^i is established. Visnu is superior to Brahm\^a; Rudra is superior to Visnu. Therefore no doubt can arise in Krisna's worshipping \'Siva. It is through the will of \'Siva that a second Rudra originated from the forehead of Brahm\^a to offer boons to him (i. e., to Brahm\^a). This second Rudra is venerable and entitled to all worship; what to speak of the First Rudra? O King! It is through the proximity of the Dev\^i that the importance and superiority of \'Siva is thus established. Thus the incarnations of Hari arise in yugas after yugas through the intervention of the Yoga

M\^ay\^a; so there is no need to discuss on this point. Why to Achyuta alone, to Brahm\^a and \'Siva also She gives troubles for getting involved into incarnations, She the Yoga M\^ay\^a who is indirectly, with the twinklings of Her eyes, creating, preserving and destroying this universe. It is the Yoga M\^ay\^a that caused Krisna to be transferred from his lying-in chamber to the village Vraja and then protected him in the house of the cowherd Nanda; afterwards took him to Mathur\^a for the destruction of Kamsa, whence he was led again out of Jar\^asandha's fear to the city Dv\^ark\^a. It is She that created from Her Ownself the eight N\^aik\^as (the leading mistresses) and also sixteen thousand and fifty women for the pleasure and enjoyment of Krisna Bhagw\^an, the incarnation of Ananta (Visnu Bhagav\^an); thus Krisna Bhagav\^an was made completely subservient to them just like a perfect slave. When a young woman, though she is alone, can bind a man down by the network of M\^ay\^a, like strong iron chain, what wonder is there that the sixteen thousand and fifty women would make Krisna play in their hands like a \'Suka bird and make him an instrument to serve any purpose that they liked. \'Sr\^i Krisna got himself so much under the control of Satyabh\^am\^a that He went gladly under her commands to Indra's heavens to get the P\^arij\^ata flowers. There he had to fight with Indra and subsequently stole away the P\^arij\^ata tree and gave it to Satyabh\^am\^a as a very valuable ornament to be kept in her room. Behold! The same Krisna, by His own prowess, defeated \'Si\'sup\^ala and others for the preservation of religion and then stole away Rukmin\^i, the daughter of Bh\^ima and afterwards married her as his legal wife; where is the rule, then observed that it is a sin to take away another's wife? Thus all embodied beings get themselves subdued by Ahamk\^ara and do acts, good or bad, confounded and deluded by the network of Moha that always drags one down below. From the M\^ul\^a Prakriti are born Brahm\^a, Visnu, and Hara and from the T\^amasic Ahamk\^ara of Prakriti is created this whole cosmos, moving and non-moving. The lotus-born Brahm\^a becomes free when he is free from Ahamk\^ara; otherwise He becomes engaged in this world affairs. When freed from this Ahamk\^ara, all the J\^ivas become free; and their houses, wealth, wives, sons and brothers are quite powerless to tie them down; but when bound by Ahamk\^ara, the J\^ivas come under their control. O king! This Ahamk\^ara is the cause bondage to all the beings; ``I am the doer, this work is done by my power; or this I will do myself'' thinking thus, the embodied beings fall themselves under this bondage. An earthen pot cannot be made without earth; no effects can be visible without a cause; consequently Visnu is preserving this universe, because of this Ahamk\^ara (imposed on him by Prakriti). The human beings are always drowned in their cares and anxieties simply because they are bound by this Ahamk\^ara; when they become free from this Ahamk\^ara, their cares and anxieties at once vanish. Moha (delusion) comes out of Ahamk\^ara; world and the enjoyments thereof come out of Moha; otherwise how can it be accounted for, that Hari and others, the mine of all good and auspiciousness, take their several incarnations in various wombs? Neither Moha nor this world comes to those that are bereft of Ahamk\^ara. Men are of three kinds, S\^attvic, R\^ajasic, and T\^amasic; O king! Brahm\^a, Visnu and \'Siva are sprung respectively from the R\^ajasic, S\^attvic, and T\^amasic Ahamk\^aras. In these three, the three Ahamk\^aras are always to be found, so the Munis, that have realised the Real Essence, declare. They are all bound by this Ahamk\^ara; there is no doubt in this. The Pundits of dull intellect, and deluded by M\^ay\^a declare that Visnu takes various incarnations out of his own free will; for when it is seen that men of even inferior intellects do not entertain any desire to enter into the wombs, painful and terrible; how will Visnu, then, the Holder of the discus, like to come into this womb! The slayer of Madhu, the Vaisnavas say, entered all at once into the wombs of Kau\'saly\^a and Devak\^i, full of faeces and other dirty things, of His own free will. But you must think out what happiness can Madhus\^udana, quitting his Vaikuntha Heavens, attain in this womb, full of so many troubles, and where arise, like poisons, hundreds of cares and thoughts to torment an individual! Especially when it is seen that human beings perform asceticism, sacrifice Yajñas and do various charities, that they would avoid thus entering in wombs, which is very painful and terrible. How can Bhagav\^an Visnu be called independent? If so, He would never have yielded to enter into various wombs. Therefore, O king! Know this that this whole universe is under the control of Yoga M\^ay\^a; the Devas, men, birds, what more everything from Brahm\^a down to a blade of grass are all under the control of Yoga M\^ay\^a. Brahm\^a, Visnu and Hara all are bound by the rope of Her M\^ay\^a. So they roam easily by Her M\^ay\^a from womb to womb like a spider.

Here ends the First Chapter of the Fifth Book on the superiority of R\^udra over Visnu in the Mah\^a Pur\^anam of \'Sr\^imad Dev\^i Bh\^agavatam by Maharsi Veda Vy\^asa, consisting of eighteen thousand verses.