\chapter{On the worship by the gods to the Dev\^i}

1-22. Vy\^asa said :-- On hearing Visnu's words, the Devas became very glad and presented immediately their own weapons, ornaments and clothings. The Ksiroda (Milk) Ocean presented to Her gladly, the well fitted necklace, clear as crystal, and a pair of divine cloths, of a red colour, never becoming old and very fine. Vi\'svakarm\^a was very much gratified in his heart and presented a divine jewel to be worn in Her diadem or crest blazing like hundreds of suns; white earrings; bracelets for Her wrist, bracelets for Her upper arm, and other bracelets decked with various gems and jewels and anklets brilliant like gems, of a clear Sun-like lustre, decked with jewels, and tinkling nicely. The architect of the gods, the ocean of intellect, Vi\'svakarm\^a gave Her as offerings beautiful ornaments also for the neck, all very beautiful, as well as for the fingers decked with gems and jewels, all shining splendidly. Varuna gave for Her head garland of lotuses, never fading away, of such a sweet fragrance as bees constantly hover round them and the Vaijayant\^i garland for Her breast. The mountain Him\^aly\^a gladly offered Her various gems and a beautiful lion, of a golden colour for Her conveyance. Then that beautiful Lady, having all the auspicious signs, wishing welfare to all, and decorated with the divine ornaments began to look grand and splendid, mounted on Her conveyance, the Lion. Visnu then created another thousand spoked discus (Chakram) from His own Chakra, capable to take off the head of any Asura, and offered it to Her. \'Sankara created another excellent Tri\'s\^ula from his own Trident, terrible and demon-killing, and offered it to the Dev\^i. Varuna created another bright conch from his own conch and offered it gladly to the Dev\^i. Fire offered Her a weapon named \'Sataghni which kills violently the demons, as if that is another god of death. Maruta (wind), the chief of the gods, offered Her a wonderful bow and arrow case filled with arrows. The bow can be drawn with great difficulty and emits a very harsh sound. Indra created another dreadful thunderbolt from his own thunderbolt and gave it at once to the Dev\^i; as well the beautiful sonorous bell that used to hang from the elephant

Air\^avata. Yama, the God of Death, created another beautiful staff from his own sceptre which takes away when time comes, the life of all beings. Brahm\^a gladly gave Her a divine Kamandalu, filled with the Ganges water; and Varuna offered Her a weapon called P\^a\'sa. O King! Time gave Her an axe and a shield and Vi\'svakarm\^a gave Her a sharp Para\'su. Kuvera, the Lord of wealth, gave her a golden drinking cup, filled with wine; and V\^aruna offered Her a divine beautiful lotus. Vi\'svakarm\^a became very glad and gave Her the Kaumodak\^i gad\^a, capable to kill the enemy of the gods and whence hundreds of bells are hanging, an impenetrable armour and various other weapons. The Sun gave to the Divine Mother his own rays. The Devas, seeing Her adorned with ornaments and weapons, began to praise and chant hymns to that most Auspicious Goddess, the Great Enchantress of the three worlds.

23-29. The Deva said :-- ``Salutation to \'Siva, Salutation to the Most Auspicious; Thou art peace and nourishment; we salute again and again to Thee. Salutation to Thee, the Bhagavat\^i Dev\^i; Thou art the Goddess Rudr\^an\^i (the terrible), we always salute again and again to Thee. Thou art the K\^alar\^atri (the night of destruction at the end of the world); Thou art the Indr\^an\^i. Thou art the Mother, we salute again and again to Thee; Thou art the success, Thou art the intelligence, Thou art the growth, Thou art the Vaisnav\^i; salutation again and again to Thee. Thou art within the earth; yet the earth does know know Thee. Thou art again the inmost of the earth and controllest the things within this earth; we offer our salutations to that Supreme Cause, the Highest Goddess. Thou art within this M\^ay\^a (the unborn) yet the M\^ay\^a does not know Thee. Thou residest again within the innermost of the M\^ay\^a and directest that Unborn One, the M\^ay\^a, we salute again and again to that Supreme Cause, the Great Directress, the \'Siv\^a (the most auspicious). O Mother! Do what is good to us; we are oppressed by our enemy, dost then protect us; by Thy own power dost Thou overpower and kill that Mahisa. That demon is vulnerable by woman only, he is deceitful, cunning, dreadful, and swollen with pride on his having got the blessing; he assumes many forms and torments the Devas. O One, devoted to the Bhaktas! Thou art the only refuge of all the gods; O Thou art the supreme goddess, we are very much harassed and oppressed by the D\^anava; therefore dost Thou now protect us; we bow down to Thee.''

30. Vy\^asa said :-- When the Devas had praised thus, the Highest Goddess, the Giver of all happiness, then smilingly said in the following auspicious terms :--

31. ``O Devas! Today in the battle ground I will overpower that wicked Mahisa, of cruel disposition and take away his life.''

32-40. Vy\^asa said :-- Speaking thus in a melodious voice, the Supreme One smiled and again said :-- ``This world is all full of error and delusion. Really, it is very wonderful that Brahm\^a, Visnu, Indra and other gods are all shuddering out of fear from Mahisa D\^anava. The power of Destiny is exceedingly great and terrible; its influence cannot be overcome even by the best of the Devas. O king! The Time is the Lord of happiness and pain; Time is, therefore, the God. The wonder is this that even those who can create, preserve and destroy this world, they are being overpowered and tormented by Mahisa. The Dev\^i, thinking thus, smiled; then laughed and laughed very hoarsely; it seemed that a roar of laughter then arose. And the D\^anavas were struck with terror at that very dreadful sound. The earth trembled at that extraordinary sound; the mountains began to move and the vast oceans that remained calm began to be agitated with billows. The uproar filled all the quarters and the mountain Meru trembled. Then the D\^anavas, hearing the tumultuous uproar, were all filled with tremendous fear. The Devas became very glad and said thus :-- ``O Dev\^i! Let victory be Yours; save us.'' The intoxicated Mahisa, too, hearing those words, became very angry. Mahisa, struck with terror at those words, asked the Daityas ``O Messengers! Go and ascertain how has originated this sound.

41-48. Who has made this harsh sound? Bring that devil who has made this hoarse noise, be he a Deva, D\^anava, or anyone else unto me and I will kill that roaring villain, who, it seems, has been puffed with egoism and vanity. The Devas are not making this noise, for they are vanquished and terror-stricken; The Asuras are not doing so, for they are my subjects; then, who is the stupid fellow that has done so? Surely he is of very little understanding; his days are numbered; and I will carry him to the home of Death. Go you, ascertain the cause of sound and come back to me; then I will go there and destroy that wretch who made this noise to no purpose.'' Vy\^asa said :-- No sooner the messengers heard these words of Mahisa, than they at once went to the Dev\^i and saw that Her body and the several parts thereof were all very beautiful; She had eighteen hands, She was decorated completely with various ornaments all over Her body, all the auspicious signs were being seen in Her body and that She was holding excellent divine weapons. That auspicious Goddess beautiful, was holding in Her hands, the cup and drinking wine again and again. Beholding Her this form, they were afraid and fled at once to the Mahisa and informed him the cause of that sound.

49-54. The Daityas said :-- ``O Lord! We have seen one grown up woman; whose whereabouts we are quite ignorant. The Dev\^i is decorated with jewels and ornaments all over Her body; She is not human nor Asur\^i but Her form is extraordinary and beautiful. That noble Lady is mounted on a lion, holding weapons on all Her eighteen hands and is roaring loudly; She is drinking wine; so it seems that She is puffed up with liquor. It is quite certain that She has no husband. The Devas are gladly chanting praises from the celestial space that Let Victory be to Her side and that She save the Devas, O Lord! We don't know at all who is that handsome woman? or whose wife is she; why has she come there? and what is Her motive? Sentiments of love, heroism, laughter, terror and wonder are all fully shining in Her; therefore we are very much overpowered by the halo emitted from Her; and we could not even see Her well.

Note :-- Rasas means sentiments. The rasas are usually eight. Sring\^ara, H\^asya, Karun\^a, Raudra, V\^ira, Bhay\^anak\^ah, Bibhats\^adbh\^u tasangau, Chetyastau, Natyan, Ras\^ah smrit\^ah but sometimes S\^antarasah, is added thus making the total number nine; sometimes a tenth, V\^atsalyarasa is also added.

55. O King! In compliance with your order, we have come back to you no sooner we had seen the Lady, without even addressing Her in any way. Now order us what we are to do.''

56-58. Mahisa said :-- ``O Best of ministers! O Hero! Under my command, go there with all the forces and use the means, conciliation, etc., and bring that woman, having a beautiful face (like the Moon), to me. If that Lady do not come even when the three policies, S\^ama (conciliation), D\^ana (making gifts), and Bheda (sowing dissensions in an enemy's party and thus winning him over to one's side, one of the four Up\^ayas or means of success against an enemy) are adopted by you, then apply the last resort Danda, (or war) in such a way that Her life be not destroyed and bring that beautiful woman to me. I will gladly make Her, of black curling hairs, my queen-consort. In case that deer-eyed one comes gladly, then do my desires without causing any unpleasant feeling; (a cessation of sentiment). I am enchanted on hearing about Her beauties and wealth.''

59-67. Vy\^asa said :-- The prime minister, on hearing the words of Mahisa, took with him elephants, horses, and chariots and hurriedly went to the desired place. On coming near to the Dev\^i, the minister began to address Her in sweet words from a sufficient distance in a very humble and courteous way. O Sweet speaking! Who art Thou? What has caused Thee to come here? O Highly fortunate! My master has asked through me these questions. My master cannot be killed by all the Devas and men; he has conquered all the Lokas (worlds). O Beautiful-eyed! On account of getting his boon from Brahm\^a, the Lord of the Daityas has become very powerful and consequently being very proud, assumes different forms at will. He, our King-Emperor Mahisa, the lord of the earth, hearing about Thy beauty and dress, has expressed a desire to see Thee. O Beautiful one! Whether he will appear before Thee in a human form? He will do whatever Thou likest. O Deer-eyed One! Be pleased now to go to that intelligent King. In case Thou dost not go, we will bring the King, Thy devotee, to Thee. O Lord of the Devas! Our King has heard of Thy beauty and grandeur and has become very much submissive to Thee. We will therefore do exactly what Thou desirest. Therefore, Thou having thighs thick and round like those of a young of an elephant! Be pleased to express what Thou likest and we will do quickly as Thou desirest.

Here ends the Ninth Chapter of the Fifth Book on the worship offered by the gods to the Dev\^i and the weapons offered by them in the Mah\^a Pur\^anam, \'Sr\^i Mad Dev\^i Bhag\^avatam, of 18,000 verses by Maharsi Veda Vy\^asa.

