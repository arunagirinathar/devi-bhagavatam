\chapter{On the birth of the P\^andavas}

1-12. S\^uta said :—Thus \'Santanu married Satyavat\^i; two sons were born to her and they died in course of time. Out of Vy\^asa Deva's semen, Dhritar\^astra was born. Ambik\^a Dev\^i, the mother of Dhritar\^astra closed her eyes on seeing Veda Vy\^asa; hence Dhritar\^astra was born blind. Seeing Dhritar\^astra blind Satyavat\^i asked Vy\^asa to go to Amb\^alik\^a (P\^andu's mother); the princess Amb\^alik\^a, mother of P\^andu turned pale at the sight of Vy\^asa; hence her son became of a pale colour out of Vy\^asa's wrath. Hence the name of the son was P\^andu. Next the maid servant, expert in the science of amorous pleasures, satisfied Vy\^asa; hence her son Vidura was born of Dharma's part and became truthful and holy. Though P\^andu was younger, the ministers installed him on the throne. Dhritar\^astra could not become king, as he was blind. By the permission of Bh\^isma the powerful P\^andu obtained the sovereignty; and the intelligent Vidura became his minister. Dhritar\^astra had two wives G\^andh\^ari and Sauvali; this Sauvali was Vaishy\^a; she was engaged in the household affairs. The king P\^andu had two wives, too; the first was Kunti, the daughter of S\^urasena; and the other was M\^adri, the daughter of the Madra king. G\^andh\^ari gave birth to one hundred beautiful sons; Vai\'Sy\^a Sauvali gave birth to one beautiful son named Yuyutsu. While Kunti was a virgin, she gave birth, through the medium of the Sun, the lovely Karna; next he became the wife of P\^andu. Hearing this, the Risis said :-- ``O Muni S\^uta! What are you saying ? First Kunti brought forth a child and afterwards she was married to P\^andu; this

is wonderful, indeed! How was Karna, born of Kunti, unmarried? and how came Kunti to be married afterwards? describe all these in detail.''

13-35. S\^uta then said :-- ``O Dvija! While S\^urasena's daughter Kunti was a virgin girl, the king Kuntibhoja asked for Kunti that she might become her girl and S\^urasena gave her to the king Kuntibhoja who brought up this beautifully smiling girl. He put to her the service of Agni of Agnihotra. Once, on an occassion, Durv\^as\^a Muni, engaged in the vow, lasting for four months, came there; Kunti served him during that period; the Muni became greatly pleased and gave her a very auspicious, mantra, by virtue of which any Deva, when called upon by that mantra will come to Kunti and satisfy her desires. When the Muni went away, Kunti, remaining in her house, wanted to test the accuracy of the mantra and asked within herself  ``What Devat\^a to call upon.'' Seeing the God S\^urya had arisen in the sky, Kunti uttered the Mantra and invoked him. The Sun, then, assuming an excellent human form, came down from the Heavens and appeared before Kunti in the same room. Seeing the Deva Sun, Kunti became greatly surprised and began to shudder and instantly became endowed with the inherent natural quality of passion (had menstruation). The beautiful-eyed Kunti, with folded palm; spoke to S\^urya Deva standing before :-- ``I am highly pleased to-day seeing Thy form; now go back to Thy sphere.''

S\^urya Deva said :-- ``O Kunti! What for you called me, by virtue of the Mantra? Calling me, why do you not worship me, standing before you? O beautiful blue one! Seeing you, I have become passionate; so come to me. By means of the mantra, you have made me your subservient so take me for intercourse.'' Hearing this, Kunti said :-- ``O Witness of all! O knower of Dharma! You know that I am a virgin girl. O Suvrata! I bow down to you; I am a family daughter; so do not speak ill to me.'' S\^urya then said :-- ``If I go away in vain, I will be an object of great shame, and, no doubt, will be laughed amongst the gods; So, O Kunti! If you do not satisfy me, I will immediately curse you and the Br\^ahmin who has given you this mantra. O Beautiful one! If you satisfy me, your virginity will remain; no body will come to know and there will be born a son to you, exactly like me.'' Thus saying S\^urya Deva enjoyed the bashful Kunti, with her mind attracted towards him; He granted her the desired boons and went away. The beautiful Kunti became pregnant and began to remain in a house, under great secrecy. Only the dear nurse knew that; her mother or any other person was quite unaware of the fact. In time, a very beautiful son like the second Sun and K\^artikeya, decked with a lovely Kavacha coat of mail and two ear-rings, was born there. Then the nurse caught hold

of the hand of the bashful Kunti and said :-- ``O Charming one! What care can you possibly have as            long as I am living.'' Kunti then, placed the son in a box and said :-- ``O son! What shall I do? Being afraid of shame, I am leaving you, though you are dear to me as my life itself! I am exceedingly fortunate that I am casting aside this all auspicious son. May the attributeless Bh\^agavat\^i Ambik\^a, the World Mother and the Lady of all, endowed with attributes, protect Thee! May K\^aty\^ayani, the giver of all desires, feed you with Her milk! Alas! I am quitting you, born of S\^urya's semen in this solitary forest like a vitiated wanton woman. I do not know, when shall I see your lotus like beautiful face, dearest to me like my self. Alas! I never worshipped in my former birth \'Siv\^an\^i, the mother of the three worlds; I never meditated Her lotus like feet, the Giver of all happiness; hence I am so very unfortunate. O Dear son! I must perform great tapasy\^a to expiate for this terrible sin, that I knowingly commit in relinquishing you in the forest.''

36-48. S\^uta said :-- Thus saying to the son within the casket, Kunti gave over that to the hands of her nurse, terrified, lest some one might see her. Kunti then bathed and remained with a fearful heart in her father's house. A carpenter (charioteer?) named Adhiratha got accidentally that casket floating in the Ganges. The carpenter's wife R\^adh\^a, prayed for the son and nourished him under her care. Thus nourished in the carpenter's house, the famous Kunti's son Karna became a very powerful warrior. The king P\^andu then married Kunti in a Svayamvara, a marriage in which the girl chooses her husband from among a number of suitors, assembled together. And the all auspicious daughter of the king of Madra became also the second wife of P\^andu. Once, on an occasion, the powerful P\^andu, while hunting in the forest killed a Muni, in the form of a deer, engaged in the act of co-habitation, thinking it to be a deer. The dying Muni became inflated with wrath, cursed P\^andu :-- ``If you co-habit, certainly you will die.'' Thus cursed by the Muni, P\^andu became very sorrowful and abandoned his kingdom and began to live in the forest. O Munis! His two wives Kunt\^i and M\^adr\^i, followed their husband as chaste women do, to serve him in the forest. Dwelling in the hermitage of the Munis, P\^andu listened to the Dharma \'S\^astras and practised severe penance. Once while he was listening to the religious discourses of the Munis, he heard unmistakeably the Munis telling that the man who is sonless can never go to the Heavens; so he must get a son somehow or other. The Pundits declare that the sons born of the father's semen, the sons born of their daughters, the Ksettraja, the Goloka, the Kunda, the Sahoda, the K\^an\^ina, the Kr\^ita, one obtained in the forest, or one offered by another father, unable to

nourish his son, all are entitled to inherit the wealth of the father; but the sons, enumerated successively are more and more inferior.

N.B.: Ksettraja - of a son, the off spring of the wife by a kinsman appointed to procreate issue to the husband.

        Goloka - Bastard child of a widow.

        Kunda - a child born in adultery.

        Sahoda - the son of a woman pregnant at the time of marriage.

        K\^an\^ina - the son born of a young and unmarried woman.

        Kr\^ita - purchased

49-52. Hearing this, P\^andu spoke to the lotus-eyed Kunti to procreate sons for him soon by a great ascetic Muni :-- ``By my order, you will not incur any sin in doing this. I heard that in ancient times the high souled kin Saud\^asa got son from Va\'Sistha.'' Kunti, then spoke to the king :-- ``O Lord! I know one Siddha mantra; it was given to me before by the Muni Durv\^as\^a. Whichever Devat\^a I will invoke by that Mantra, he will instantly come to my side, controlled by that Mantra.''

53-71. At the request of the husband, Kunt\^i invoked Dharma, the best of the Devas; and after being impregnated by him, gave birth to Yudhisthira. Then she got through Pavana Deva, the son Vrikodara; and through Indra the Lord of the Devas, Arjuna. Thus, in every year, Kunti gave birth to one son and so in three years she gave birth to three very powerful and mighty sons. At this Màdri spoke to her husband :-- ``O king, the best of the Kurus! What shall I do now? Kindly suggest to me the means of procreating sons; O Lord, remove my pain.'' P\^andu asked Kunti for this; Kunti, moved with pity, gave her the mantra, so that she might get one son. Then the beautiful M\^adr\^i, invoked the twin A\'Svin under the advice of her husband and got a pair of twins Nakula and Sahadeva through them. O Munis! Thus five P\^andavas were born successively in every following year to the wives of P\^andu by the seed of the Devas. Once on a time P\^andu, whose end was drawing nigh became very passionate at the sight of M\^adr\^i in that solitary hermitage. He, though forbidden repeatedly by M\^adri, warmly embraced her, as if dictated by the great destroyer, and fell to the ground. As the creeper falls down when the tree is felled, so M\^adr\^i dropped on the ground and began to cry violently. Having heard the wailings of M\^adr\^i, Kunti and the five sons of P\^andu came there weeping and crying; a tumult then ensued and the great Munis also appeared on the scene. Then those Munis, practising great vows, knew that Pàndu was dead and performed duly, on the banks of the Ganges, the ceremony of burning the dead.

At that time M\^adri gave over to Kunti the charge of her two sons and followed the Sat\^i practice along with her husband to go to Satyaloka.

The Munis, then, performed Tarpana ceremonies in honour of P\^andu and M\^adri and took Kunti and the five sons to Hastin\^apur. Knowing that Kunti has come, Bh\^isma, Vidura and the relatives of Dhritar\^astra within the city, all came to Kunti. They all asked Kunti :-- ``O beautiful one! Whose are these five sons?'' Kunti, then, remembered the curse on P\^andu and sorrowfully expressed :-- `` These are the Deva's sons born in Kuru family.'' In order to convince the people assembled there, Kunti invoked the Devas who came in the celestial space above and said :-- ``Yes, these are the sons born of our seeds.'' Bh\^isma, then, paid respect to the words of the Devas and honoured duly the boys. Bh\^isma then took the five sons and P\^andu's wife to Hastin\^a and gladly nourished them. O Munis! The sons of Prith\^a were thus born and nourished by Bh\^isma.

Thus ends the sixth chapter of the second Adhy\^aya on the birth of the P\^andavas in the Mah\^apur\^anam \'Sr\^i Mad Dev\^i Bh\^agavatam.