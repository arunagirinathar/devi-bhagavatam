\chapter[On the narration of the Navar\^atra ceremony]{On the narration of the Navar\^atra ceremony by N\^arada and the performance of that by R\^ama Chandra}

1-2. Vy\^asa said :-- O king! R\^ama and Laksmana, discussing thus, remained silent; when the Risi N\^arada appeared there from the sky above, singing the Rath\^antara S\^ama Veda hymns in tune and musical gamut with his renowned lute.

3-4. R\^amachandra, of indomitable prowess, on beholding him rose up from his seat and gave him quickly an excellent seat and offerings of water for washing his feet. Then he worshipped the Muni and stood with folded hands. When the Muni ordered him, he took his seat close by N\^arada.

5-8. On R\^amachandra taking his seat there with Laksmana with a grievous heart, N\^arada asked him in a sweet tone ``O Descendant of Raghu! Why are you being afflicted with sorrows like an ordinary mortal? I know that the evil minded R\^avana has stolen S\^it\^a Dev\^i. I heard while in the heavens that R\^avana, the descendant of Pulastya, stole away J\^anak\^i, out of fascination, could not know that would be the cause of his death. O Descendant in the family of K\^akutstha! It is for the killing of R\^avana that your birth has taken place; and for that purpose J\^anak\^i has been stolen now.

Note :-- The real J\^anak\^i was not stolen; Her shadow form was stolen.

9-12. O R\^aghava! The Dev\^i J\^anak\^i, in her previous birth, was the daughter of a Muni and practised asceticism. While engaged in her austerities, in her holy hermitage, R\^avana came and looking at her, prayed that beautiful woman to become his wife. Hearing this, she gave a good reproach to R\^avana, when he perforce caught hold of her hairs. That ascetic woman got very angry, and, considering her body polluted by the devil's contact, resolved to put an end to her life and cursed R\^avana, thus :-- ``O Villain! I will be born on the surface of the earth, not from any womb but simply for your destruction and ruin.'' Thus saying, she parted with her life.

13. O Tormentor of the foes! R\^avana, the king of the R\^aksasas, mistook a garland for the extremely poisonous serpent and has stolen away S\^it\^a Dev\^i, the part incarnation of Laksm\^i, in order to root out his race.

14. O K\^akutstha! When the Devas prayed for the destruction of that wicked insolent R\^avana, difficult to be subdued, you are born on this earth, in the family of Aja, as a part incarnate of Hari, beyond birth, old age and death.

15. O mighty-armed! Have patience; S\^it\^a Dev\^i is meditating you, day and night.

16-17. Indra himself, the king of the Devas, sends the nectar and the Heavenly Cow's Milk in a pot to Her daily; and She subsists on that, alone.

O Lord! On drinking the Heavenly Cow's Milk, the lotus eyed S\^it\^a Dev\^i is living without any hunger or thirst! I use to see Her daily.

18. O Descendant of Raghu! I am now telling how that R\^avana can be killed. Perform, in this very month of Â\'svin, the vow with devotion.

19. Fasting for nine nights, the worship of the Bhagavat\^i, and repeating the Mantram silently and performing the Homa ceremony, observing all the rules, will certainly fulfill one's all the desires.

20. O the best in the race of Raghu! You should offer the sacrifice before the Goddess of a sacred and unblameable animal, perform Japam and Homa ceremony equivalent to one-tenth of Japam. If you do all this, you will certainly be able to release S\^it\^a.

21. In days of yore, Visnu, \'Siva and Brahm\^a and the Devas in the Heavens all performed this worship of the Goddess.

22. Therefore, O R\^aghava! Every person desiring happiness, specially those that have fallen under great difficulties, ought to do this auspicious ceremony, without the least hesitation.

23-24. O K\^akutstha! Vi\'svamitra, Bhrigu, Va\'sistha and Ka\'syapa all of them did this worship before. When some stole away the wife of Brihaspat\^i, the Guru of the Devas, he, too, by the force of this worship, got his wife back. Therefore O king! dost thou also celebrate the Pûj\^a for the destruction of R\^avana.

25-26. O high minded one! This vow was practised before by Indra for the destruction of Vritra, by \'Siva for killing the demon Tripur\^a, by N\^ar\^ayana for the killing of the demons Madhu and Kaitava; so you should also firmly resolve to perform duly this vow with your whole heart.''

27. R\^ama replied :-- ``O Ocean of Knowledge! Who is that Dev\^i? What is Her influence; whence has She sprung? What is Her Name? And how is that vow to be duly observed? Kindly describe all these to me in detail.''

28. N\^arada answered :-- ``Listen, O R\^aghava! That Goddess is Eternal and Ever Constant Primordial Force. If you worship Her, all your difficulties will be removed and all your desires will be fulfilled.

29. She is the source of Brahm\^a, Visnu and others and of all these living beings. Without Her force, no body would be able even to move their limbs.

30. That Supreme Auspicious Goddess is the preserving energy of Visnu, is the creative power of Brahm\^a, and is the destroying force of \'Siva.

31. Whatever there exists in this infinite Universe, whether Temporal or Eternal, She is the Underlying Force of all; how, then, can She have an origin!

32-33. Her origin is not Brahm\^a, Visnu, Mahe\'sa, Sun, Indra, or the other Devas, not this Earth nor this Upholder of the Earth; She is devoid of any qualities, the Giver of Salvation of all, the Full Prakriti. In the time of the final dissolution of this Universe, She lives with the Supreme Purusa.

34. She is also Saguna, full of qualities, and is the Creatrix of Brahm\^a, Visnu and Mahe\'sa, and has empowered them, in every way, to create the three Lokas.

35. She is the Supreme Knowledge, existing before the Vedas, and the Originator of the Vedas. The individual souls, knowing Her Nature, become able to free themselves from the bondages of the world.

36. She is known by endless names. The Brahm\^a, and the other Devas might choose to call Her according to their actions and qualities. I am unable to describe those names.

37. O descendant in the race of Raghu! Her endless names are formed by the various combinations of the various vowels and consonants from the letter ‘A' to the letter ‘Ksa.'''

38. R\^ama said :-- ``O best of the Munis! Describe briefly all the rules and regulations as to how that vow and worship are to be performed. With my heart, full of devotion and faith, I will worship the Goddess today.''

39. N\^arada said :-- ``O R\^aghava! On a level plot of ground, prepare an altar. Place the Goddess there and fast for nine days.

40. O king! I will be your priest and I will, with great energy, carry out this yaj\~na to fullfil the work of the Gods.''

41-42. Vy\^asa said :-- Then the powerful Bhagav\^an Hari, hearing all from the Muni, believed them to be true; and, on the approach of the month of Â\'svin, prepared the altar on the top of a hill and placed the Auspicious Goddess, the World Mother and, observing all the rules, performed the vow and worshipped the Goddess.

43. Fasting for nine days, R\^ama celebrated the vow and duly offered sacrifices, performed the worship and Homa ceremonies.

44-46. When, on the grand night of the Eighth lunar day, the two brothers completed the vow as told by N\^arada, the Supreme Bhagavat\^i was pleased with the worship and appeared before them, mounted on a lion, and remaining there on the mountain top, addressed R\^ama and Laksmana, in a sweet grave tone, like the rumbling of a rain cloud, thus :-- ``R\^ama, I am satisfied with your worship; ask from me what you desire.

47. R\^ama! You are sent by the gods for the destruction of R\^avana and are born as a part incarnate of N\^ar\^ayana, in the pure and stainless family of Manu.

48. It is You that, in ancient times, incarnated as a fish for serving the purpose of the Devas and preserved the Vedas by killing the terrible R\^aksasas for the welfare of the Universe.

49. It is You that incarnated as a tortoise and held aloft the Mandara mountain, churned the ocean and nourished the Devas.

50-51. O R\^ama! It is You that incarnated, in days of yore, as a boar and held aloft on your teeth this earth. It is You that assumed the form of a Man-Lion and preserved Prahl\^ada, by tearing asunder the body of Hiranya Ka\'sipu, by Your sharp nails.

52. O Descent of Raghu! It is You that assumed, in ancient times, the form of a dwarf and served the purpose of the Devas, by deceitfully cheating Bal\^i, the younger of Indra.

53. O son of Kau\'salya! You incarnated as the son of Jamadagni in the Br\^ahmin family, extirpated the line of Ksattr\^iya kings and gave over this whole earth to Bhagav\^an Ka\'syapa Risi.

54. So You are now born as the son of Da\'saratha, in the stainless race of K\^akutstha, at the request of the Devas, harassed by R\^avana.

55-56. These powerful monkeys, born as Deva incarnates, all endowed with great power by Me, will help you. Your younger Laksmana is the incarnate of \'Sesa serpent; this indomitable man will kill undoubtedly Indragit, the son of R\^avana.

57. You will kill R\^avana; then you would worship Me, with great devotion, in the vernal season and then enjoy your kingdom according to your liking.

58. O best of the Raghus! For full eleven thousand years you will reign on this earth; and after that reenter your heavenly abode.''

59. Vy\^asa said :-- O king! Thus saying, the Dev\^i disappeared. R\^ama Chandra became very glad and, completing that most auspicious ceremony, performed the Bejoy\^a Pûj\^a on the tenth day and gave lots of presents to N\^arada and made him go towards the ocean.

60-61. O king! Thus stimulated by the Supreme Energy, the Highest Goddess brought front to front, R\^amachandra, the husband of Kamal\^a, went to the shores of the ocean, accompanied by Laksmana and the monkeys. Then he erected the bridge across the ocean and killed R\^avana, the enemy of the gods. His unparalleled fame spread everywhere throughout the three Lokas.

62. He who hears with devotion this excellent account of the Dev\^i, will get the greatest happiness in this world, and, in the end, will get the final beatitude. There is no doubt in this.

63. O king! There are extant many other Pur\^anas, but none is equal to this \'Sr\^i Mad Dev\^i Bh\^agavatam. Know, this is my firm belief.

Here ends the thirtieth chapter on the narration of the Navar\^atra ceremony by N\^arada and the performance of that by R\^ama Chandra in the 3rd Adhy\^aya in \'Sr\^i Mad Dev\^i Bh\^agavatam of 18,000 verses by Maharsi Veda Vy\^asa. Here ends the Third Book.

The Third Skandha completed.