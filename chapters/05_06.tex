\chapter{On the Deva D\^anava fight}

1-8. Vy\^asa said :-- O King! On the Daitya T\^amra becoming unconscious Mahisa became very angry and, raising his Gad\^a (club), came up before the Devas and said :-- ``Devas! O Ye powerless like crows; wait; with one stroke of Gad\^a, I will kill you.'' Thus saying, the powerful Mahisa swelled with pride, seeing Indra before him mounted on his elephant instantly struck him on his arms. Indra, again lost no time, and struck violently with his thunderbolt and cut the D\^anava's Gad\^a into pieces, and came up very close, wanting to strike at him. Mahisa, too, becoming very angry took up his lustrous sword and came to Indra to attack him with this weapon. A fight then occurred between the two, terrible to all the Lokas and wonderful to the Munis, where various weapons were showered from both the sides. The Demon Mahisa spread then his \'S\^amvar\^i M\^ay\^a, destructive to all the worlds and fascinating to the Munis.

Hundreds and hundreds of powerful buffalo-like appearances resembling Mahisa became, then, visible on the battle-field; they all began to kill the Deva forces with weapons in their hands.

9-14. Seeing this magic of the D\^anava, Indra became thunderstruck and very much confounded with terror. Varuna, Kuvera, the Lord of wealth, Yama, Fire, Moon, Sun, and other Devas all fled with terror. Indra then, being surrounded by the network of magic, began to call Brahm\^a, Visnu and Mahe\'sa in his mind. At the instant when they were called in mind, Brahm\^a, Visnu and Mahe\'sa riding on respective conveyances Swan, Garuda, and Bull, came up there with best weapons in their hands for Indra's protection. Visnu seeing the play of that fascinating magic hurled his bright discus, Sudar\'san; and caused the magic to vanish at once. Seeing the three, the Creator, the Preserver, and the Destroyer, the D\^anava Mahisa came up there with his Parigha (a club tipped with iron) weapon, desirous to fight with them.

15-16. Then the general Chiksura, Ugr\^asya, Ugrav\^irya, Asilom\^a, Trinetra, V\^askala, Andhaka and other warriors came up to fight.

17-23. Those Proud D\^anavas, clad in armour and mounted on chariots with bows in their hands besieged the Devas, like a tiger attacking an heifer. Then those D\^anavas swelled with pride began to shower on arrows after arrows; the Devas, too, began to do the same, desiring to extirpate them.The General Andhaka, coming up to Hari, drew his bow with great force up to his ear and shot at him five sharp arrows tipped with poison. V\^asudeva, the Destroyer of the enemies, cut off those arrows no sooner they came up before him; and He shot at the D\^anavas five arrows. Then Hari and the D\^anava struck each other with various weapons and arrows, swords, discus, Musala, clubs, \'Sakti, and Para\'su. Here, on the other hand, the fight lasted for fifty days between Mahe\'sa versus Andhaka; and it was a very close conflict, causing horripilation. Thus severe fights ensued between V\^askala and Indra, Mahisa and Rudra, Trinetra and Yama, Mah\^a Hanu and Kuvera, Asilom\^a and Varuna.

24-39. The D\^anava Mahisa struck Garuda, the conveyance of Hari, with his club; Garuda, being very much distressed with the blow, sat down, gasping. Visnu then comforted the powerful Garuda, the son of Vinat\^a and made him calm and quiet. Wanting to kill Andhaka, Jan\^ardana became infuriated, and, drawing his bow made of horn, call \'S\^arnga, shot at him arrows after arrows. The D\^anava cut off all those arrows to pieces with his own mass of arrows. Then, becoming very angry, he shot fifty sharp arrows at Hari. V\^asudeva quickly made all those arrows useless

and hurled Sudar\'sana Chakra with thousand spokes on the D\^anava with great violence. Andhaka thwarted this with his own discus and shouted aloud with such a great force that all the Devas became confused and confounded. Visnu's Chakra being baffled, the Devas became distressed with grief and the D\^anavas got elated. Seeing the Devas thus grieved, Visnu held aloft his Kaumodak\^i Gad\^a (club) and came hurriedly before the D\^anava. Hari struck then with his Gad\^a on the D\^anava's head whereon he fell senseless on the ground. The hot-tempered Mahisa, seeing Andhaka senseless, bellowed aloud and, terrifying Hari, came up there. Seeing him there, V\^asudeva made such a thundering noise with his bow string that the Devas became highly glad. Then the Bhagav\^an shot showers of arrows on Mahisa; and Mahisa, too, cut those arrows while they were seen in the air. O king! Then a very close fight ensued between the two, Ke\'sava struck on the head of the D\^anava with his club. Thus struck, he fell in a swoon on the ground and a general cry of distress arose amongst the D\^anavas. In a moment the D\^anava got up again, free from trouble; he then struck again on Visnu's head with his Parigha (a club mounted with iron, a mace). Struck by that mace, Jan\^ardan lay senseless; Garuda, seeing him thus unconscious, immediately took him away from the battle field.

40-55. When Visnu thus fled, Indra and the Devas were much distressed with fear and began to cry aloud. Hearing the Devas cry, \'Sankara became wrathful and, quickly coming before Mahisa, struck him with his trident (\'S\^ula). The wicked Mahisa made his weapon ineffectual and bellowed aloud and struck on the breast of \'Sankara with his \'Sakti (a kind of missile). Thus wounded in his breast \'Sankara did not feel any pain; rather, with his eyes red with anger, He struck him again with Tris\^ula. Seeing \'Sankara engaged with Mahisa, Hari becoming conscious came again on the battle-field. Seeing the two powerful Deva-chiefs, Hari and Hara, in the battle-field Mahisa became very much angry: he then assumed a buffalo body and wagging his tail to and fro came in front of them with a desire to fight. That terrible Mahisa of a huge body shook his horns and bellowed so deep like a thunder cloud that even the Devas got frightened. He began to hurl the huge mountain peaks with his two horns. The two powerful Devas Hari and Hara, began to shoot at the D\^anava deadly arrows after arrows. Seeing these two gods shower arrows upon him, Mahisa began to hurl mountains on them by his tail. Visnu cut off those mountains into hundred pieces by his arrow; and struck at him instantly with his Chakra. Struck thus by Chakra, the Lord of the D\^anavas fainted, but he instantly rose up with a human body. The mountain-like terrible D\^anava with a club in his hand frightened the Devas and uttered grave sounds like those of rumbling rain clouds. Hearing that, the Bhagav\^an Visnu sounded a more terrible sound with his Pañchajanya \'Sankha (conchshell). Hearing the sound of that conchshell, the D\^anavas were struck with terror and the ascetic Risis and Devas became exalted with joy.

Here ends the Sixth Chapter of the Fifth Skandha on the Deva D\^anava fight in \'Sr\^i Mad Dev\^i Bh\^agavatam, the Mah\^a Pur\^anam, of 18,000 versus by Maharsi Veda Vy\^asa.