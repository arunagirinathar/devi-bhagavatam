\chapter{On the prowess of Kau\'sik\^i}

1-7. Vy\^asa said :-- O King! When the tormented Devas praised thus, the Dev\^i created from Her body another supremely beautiful form. This created form, the Ambik\^a Dev\^i, became known in all the worlds as Kau\'sik\^i, as She came out of the physical sheath of the Dev\^i Parvat\^i. When Kau\'sik\^i was created out of the body of Parvat\^i, the Parvat\^i's body became transformed and turned out into a black colour and became known as K\^aIik\^a. Her terrible black appearance, when beheld, increases the terror even of the Daityas. O King! This Dev\^i is now become known in this world as K\^alar\^atri, the night of destruction, at the end of the world, identified with Durg\^a, the Fulfiller of all the desires. The Ambik\^a Dev\^i, then, began to look splendid, decked with various ornaments; Her beautiful form began to look very lovely. The Dev\^i Ambik\^a then smiled a little and said, ``Better be fearless; I will slay just now your enemies. It is My incumbent duty to carry out your purposes; I will therefore slay in battle Ni\'sumbha and others for the sake of your happiness.''

8-30. Thus saying, the Dev\^i Bhagavat\^i, elated with pride, mounted on lion and, taking K\^alik\^a with Her, entered into the city of \'Sumbha, the enemy of the gods. Ambik\^a went to a garden adjoining the city accompanied by K\^alik\^a, and began to sing in such a sweet melodious tune that enchants even the God of Love, who fascinates the whole world. What more can be said than the fact that, hearing that sweet melodious song, the birds and beasts became enchanted; the Devas then began to feel much pleasure from the Sky. In the meanwhile Chanda, Munda the two dreadful Asuras, and attendants of \'Sumbha, came out accidentally there on their sportive excursions and saw the beautiful Ambik\^a Dev\^i singing and K\^alik\^a Dev\^i sitting before Her. O best of Kings! No sooner Chanda, Munda saw the extraordinary beauty of the Goddess Bhagavat\^i, than they went at once to \'Sumbha. On approaching towards the lord of the Daityas sitting in his room, they bowed down and told thus in a sweet voice :-- "O King! Here has come from the

Him\^alay\^as a woman accidentally, mounted on a lion; Her limbs are shining with all good signs so much so that even the God of love would be enchanted by Her sight. Nowhere, in the Devalokas, the Gandarbha Lokas or in this earth can be found such a beautiful lady; we never saw nor heard about such a lady before. O King! That lady is singing so beautifully and pleasingly to all that even the deer are standing motionless by Her side enchanted, as it were, by Her melodious voice. O King! That Lady is fit for you; therefore determine first whose daughter is this lady, what for she has come there and then marry Her. Know this as certain that such a beautiful lady is not to be found anywhere in this world. Therefore do you bring Her to your house and marry Her. O Lord of men! You have acquired all the gems and jewels of the Devas; why not, then, accept this Gem in the form of a lady? O King! You have taken by force the exquisitely beautiful Air\^avata elephant of Indra, the P\^arij\^ata Tree, the seven faced horse Uchchai\'srav\^a, and many other jewels. You have acquired by your might the Prince of Jewels, the celestial car of the Creator Brahm\^a, ensigned by the emblematic Swan. You have dispossessed Kuvera of his treasure of the value of a Padma (one thousand billion) and Varuna, the God of oceans, of his white umbrella. O King! When Varuna was defeated, your brother Ni\'sumbha took perforce his P\^a\'sa weapon. O King! The Great Ocean gave you, out of terror, various jewels and honoured you by presenting a garland of lotuses which never fade away. What more can be said than the fact that you have conquered the Death and took away His force and that you have easily conquered Yama, the God of Death and have taken from Him His horrible staff. O King! You have brought that Heavenly cow which came out when the ocean was churned; that cow is still with you; what more to say than that Menak\^a and other Apsar\^as are under your control. Thus you have got by your strength all the jewels. Why, then, are you not taking this exquisitely beautiful lady, the Prince of Jewels, amongst women. O King! All the jewels in your house, will serve their real purpose, no doubt, then and then only when they will shine with this queen of jewels, this Lady. O Lord of the Daityas! There cannot be seen in all the Trilokas such a Beautiful Lady as this that I have now described before you. Therefore bring this Beautiful Lady quickly and accept Her as your wife.

31-35. Vy\^asa said :-- O King! Hearing thus the sweet words of Chanda and Munda, \'Sumbha spoke gladly to Sugr\^iva who was close by :-- ``Go, Sugr\^iva, do my messenger's work; you are well skilled in these things. Speak so that the Beautiful Lady of thin waist may come over to me. Those who are well versed in the science of amorous love declare

that only two methods are to be adopted by the clever persons towards the female sex :-- (1) conciliation and gentle words and (2) gifts and presents. For if the policy of division or sowing dissensions be applied, then hypocrisy is shewn and that means the improper manifestation of love sentiment; whereas if chastisement be applied then the love sentiment becomes interrupted. Therefore, the wise have condemned these as corrupt means. O Messenger! Where is that woman who does not come round excited with passion when good and sweet words are spoken to her in accordance with the \'Sama and D\^ana methods?''

36-37. Vy\^asa said :-- Sugr\^iva, hearing the nice skilled words of \'Sumbha went hurriedly to the spot where existed the Mother of the Universe. He saw the Fair Lady mounted on a lion, saluted Her and spoke gently and sweetly as follows :--

38-49. The messenger said :-- ``O Beautiful One! \'Sumbha, the enemy of the Gods and the King of all, is beautiful in all respects, the ruler of the three Lokas, a great hero and conqueror of all. Hearing your beauty and loveliness, that high-souled monarch is so much attached to you and has become so very passionate that he has sent me to you to express his views. O One of delicate limbs! Please hear what that Lord of the Daityas has spoken to Thee, after duly saluting Thee, words full of love and affection towards Thee :-- O Beloved! I have defeated all the Devas and have thus become the Lord of the three worlds; specially I partake of all the offerings made in sacrificial acts and ceremonies, without moving away from my house. I have taken away all the gems, jewels and wealth that belonged to the Devas; consequently the abode of the Gods has become now worthless, on account of all its jewels being carried sway. O Fair One! I am now enjoying all the jewels that exist in the Trilokas; so much so that all the Devas, Asuras, and human beings are passing away their times, subservient to Me. But no sooner Thy qualifications reached my ears Thou hast penetrated into my heart and has made me completely subservient to Thee; O Fair One! What am I to do now? Whatever Thou commandest, I am ready to do that; verily I am now Thy servant; so Thou oughtest to save me from the darts of passion. O One having swan-like eyes! I am verily made your captive. Specially I am extremely agitated by the arrows of Cupid; therefore dost Thou serve me when Thou wilt be made the Lord of the three worlds and thus enjoy the incomparably excellent things. O Beloved! I will remain ever Thy obedient servant up to the last moment of death. O Excellent One! I cannot ever be killed by the Devas, Asuras and human beings. O Fair faced One! Thou wilt be always prosperous and fortunate. Thou wilt be able to sport anywhere Thou likest. O Dev\^i! Please ponder over the above words of the Lord of the Daityas in Thy heart and speak out Thy views gladly and with the same sweetness in reply; O Brisk One! I will go immediately to \'Sumbha and inform him about Thy mind.''

50. Vy\^asa said :-- O King! The Dev\^i, ready to serve the cause of the Gods, heard the messenger's gentle words and replied smiling and sweetly.

51-66. \'Sr\^i Dev\^i spoke :-- I know fully well \'Sumbha and Ni\'sumbha; the King \'Sumbha is very powerful, the conqueror of all the Devas, and the destroyer of enemies. He is the repository of all good qualities, the enjoyer of all pleasures, very valorous, charitable and is beautiful, in fact a second Cupid. He is adorned with thirty-two auspicious signs; particularly he is a hero and cannot be killed by the Devas or human beings. O Messenger! Knowing this I have come here to have a look of that great warrior \'Sumbha. The jewel comes in contact with gold to increase its lustre; so I have come here from afar to see my husband. On seeing all the Devas, Gandharbhas, R\^aksasas and the eminent beautiful persons on the earth I have come to know that they are all terror stricken and almost unconscious and shudder at the name of \'Sumbha. So, on hearing about his abilities, I have now come here to see him. O Messenger! O Fortunate One! Better now go back to the great hero \'Sumbha and speak to him in private the following sweet words of Mine :-- ``That you are foremost amongst the powerful; beautiful of the beautifuls, skilled in all the branches of learning, well qualified, charitable, clever, born of a high noble family, energetic, and conqueror of the Devas; especially, by the sheer force of your arms, you are so much exalted and you now enjoy all the gems and jewels. Therefore, O King! Knowing your qualifications, I have come truly of my own accord to your city with the desire of getting for Me a husband. O High-souled One! I am fit for your consort. O Lord of the Daityas! There is a slight hitch in My marriage. It is this: In my early days while I was playing with My comrades, I promised before them privately partly out of childishness and partly out of vanity for bodily strength that I will certainly marry that hero who is powerful like Me and who will defeat Me in battle, thus testing his powers and weaknesses. My comrades laughed at my words and spoke with wonder, ‘Why has this girl made such an extraordinarily difficult promise?' Therefore, O Monarch! Better marry Me and fulfil My desires after knowing My strength and defeating Me in a battle. O Beautiful One in all respects! Better come yourself or your younger Ni\'sumbha and perform the marriage ceremony after defeating Me in the battlefield.''

Here ends the Twenty-third Chapter in the Fifth Book on the prowess of Kau\'sik\^i in \'Sr\^i Mad Dev\^i Bh\^agavatam, the Mah\^apur\^anam of 18,000 verses by Maharsi Veda Vy\^asa.