\chapter{On \'Siva’s granting boons}

1-37. The Risis said :-- ``O S\^uta! You told before, that Vy\^asa Deva, unrivalled spirit, composed all the Pur\^anas and taught them to his own \'Suka Deva; but how did Vy\^asadeva, who was incessantly engaged in tapasy\^a, procreate \'Suka? Describe to us in detail what you heard direct from Krisna Dvaip\^ayana Vy\^asa''. S\^uta said :-- ``O Risis! Hear how \'Suka Deva, the best of the Munis and the foremost of the Yogis, was born of Vy\^asa Deva, the Satyavati's son.

On the very beautiful summit of Mount Meru, Vy\^asa, the son of Satyavati, firmly determined, practised very severe austerities for the attainment of a son. Having heard from N\^arada, he, the great ascetic, repeated the one syllabled mantra of V\^ak and worshipped the Highest Mah\^amay\^a with the object of getting a son. He asked, Let a son be born to me as pure and as spirited and powerful as fire, air, earth, and \^ak\^a\'Sa. He thought over in his mind that the man possessed of \'Sakti is worshipped in this world and the man devoid of \'Sakti is censured here, and thus came to the conclusion that \'Sakti is therefore worshipped every where; and, therefore,worshipped Bhagav\^an Mahe\'svara coupled with the auspicious \^ady\^a \'Sakti and spent away one hundred years without any food. He began his tapasy\^a on that mountain summit which was ornamented with the garden of Karnik\^ar, where all the Devas play, and where live the Munis highly ascetic, the \^adityas, Vasus, Rudras, Marut, the two A\'sv\^ins, and the other mindful Risis, the knowers of Brahm\^a and where the Kinnaras always resound the air with their songs of music, etc.; such a place Vy\^asa Deva preferred for his tapasy\^a.''

The whole universe was pervaded with the spirit of asceticism of the intelligent Par\^asara's son Vy\^asa Deva; and the hairs of his head were clotted and looked tawny, of the colour of flames. Seeing the fire of his asceticism, Indra, the lord of \'Sachi became exceedingly terrified. Bhagav\^an Rudra, seeing Indra thus afraid, fatigued and morose, asked him :-- ``O Indra, why do you look so fear-stricken to-day? O Lord of the Devas! What is the cause of your grief? Never show your jealousy and anger to the ascetics; for the mindful ascetics always practise severe asceticism with a noble object and worship Me, knowing Me to be possessed of the all powerful \'Sakti; they never want ill of any body''. When Bhagav\^an Rudra said this, Indra asked him :-- ``What is his object?'' At this \'Sankara said :-- For the attainment of a son, Pras\^ara's son is practising so severe austerities; now one-hundred years is being completed; I will go to him, and give him to-day the auspicious boon of a son. Thus speaking to Indra, Bhagav\^an Rudra, the Guru of the world, went to Vy\^asa Deva and, with merciful eyes, said :-- ``O sinless V\^asavi's son! Get up; I grant to you the boon, that you will get a son very fiery, luminous and spirited like the five elements fire, air, earth, water and \^ak\^a\'sa, the supreme Jñ\^an\^i, the store of all auspicious qualities, of great renown, beloved to all, ornamented with all Sattvik qualities, truthful and valorous.

Hearing these sweet words of Bhagav\^an \'S\^ulap\^ani Maharsi Krisna Dvaip\^ayana bowed down to Him and went back to his own hermitage. Tired with the labour of penance for many years, he wanted to kindle fire by rubbing two fuels (Aran\^i) with each other. While doing this the high souled man suddenly began to think strongly in his mind about procreating a son. He thought :-- ``Will it be that my son will be born as this fire is produced by the friction of the two churning sticks? I have not got the wife, which the Pundits designate a ``Putr\^arani'', the youthful wife endowed with beauty, born of a noble family, the chaste one I have not got with me. But the wife, though chaste and fit to beget a son, is undoubtedly a chain to both the legs so how can I get such a one for my wife? This is known to all that a chaste wife, though clever in doing all household duties, beautiful and giving happiness to one's desires, is yet always a sort of bondage. What more than this, that the ever Bhagav\^an Mahe\'svara is always under the bondage of woman. How, then, knowing and hearing all these I can accept this difficult householder's life? While he was thinking thus, the extraordinarily beautiful Apsar\^a Ghrit\^achi fell to his sight close to him in the celestial air.

Though Vy\^asa Deva was a Brahm\^ach\^ari (holding in control the secret power of generation) of a very high order, yet seeing suddenly the agile Apsar\^a (a celestial nymph) coming close to him and looking askance at him, he became soon smitten with the arrows of cupid and feeling himself distressed, began to think what shall I do in this critical moment.

Unbearable amorous feelings now have come over me; now if I take this celestial nymph, knowing that Dharma is everywhere looking, and woman has come to take away my precious fire of spirit acquired by my tapasy\^a, then I will be laughed at by the high souled ascetic Munis who will think that I have lost my senses altogether. Alas! Why I who have practised for one hundred years the most terrible ascetism, have become so powerless by the mere sight of this Apsar\^a! The Pundits declare the household life as the source of getting son, one's heart’s desire and the source of all happiness; so much so that it leads all the virtuous souls to the pleasures of Heaven, and ordains Moksa (liberation) to those who are Jñ\^anins; and if I get such unrivalled happiness from this householder's life, I can have this Deva Kany\^a (the celestial nymph) though blameable. But again that happiness will not occur to me through her; there is no doubt in this. So how can I take her. I heard from N\^arada how, in ancient days, a king name Pururav\^a fell under the clutches of Urva\'s\^i and ultimately felt great pain, being defeated by her.

Thus ends the tenth chapter on \'Siva's granting boon in the Mahapur\^ana \'Sr\^imad Dev\^i Bh\^agavatam of 18,000 verses by Maharsi Veda Vy\^as.