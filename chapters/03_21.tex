\chapter{On the king of Benares fulfilling the advice of his daughter}

1. Vy\^asa said :-- On hearing the sound words of his daughter \'Sa\'sikal\^a, pregnant with reason, Sub\^ahu the king of Benares, became very anxious and began to think what he would now, so short a space of time in this momentous occasion, thus :--

2-3. ``The powerful kings, all, have come here on intention that they would fight and therefore they are all attended with their armies and followers respectively; and they are now sitting on their respective daises in the Svayamvara hall. If I go now and tell them that my daughter

\'Sa\'sikal\^a is not willing to come of her own accord in the hall, the evil minded kings will certainly kill me out of their wrath.

4. I have not so much strength, whether in my army or in forts, as to be able to decline these kings and drive them away from my kingdom.

5. Sudar\'sana, too is, alone, helpless, wealthless, and a mere boy. What shall I do now? Alas! I am now plunged in deep sorrow.''

6. Thinking thus, with head bowed down by humility, the king went to the kings, and said thus :--

7. ``O Kings! The girl, though requested repeatedly by me and her mother, is not willing to come to this hall. What can I do now?

8-9. I am your servant and, bowing my head at the feet of you all, pray to you, to accept my worship and return to your own cities respectively. I am ready to give a sufficient quantity of gems and jewels, clothes, elephants, chariots. Kindly accept these and go back to your own homes.

10. My daughter is as yet a girl; if I chastise her, she may commit suicide; and I will be exceedingly sorry; therefore I am very much distressed with this thought.

11. You all are fortunate, energetic, and of a merciful disposition; what will it serve you to accept the daughter of mine, who is disobedient and unfortunate?

12. I am your obedient servant; shew your mercy on me and it is your duty to consider my daughter as your own daughter.''

13. Vy\^asa said :-- Hearing Sub\^ahu's words, the kings did not utter a single word; but Yudh\^ajit, with his eyes reddened out of wrath, began to address the king of Benares in an angry tone :--

14. ``O King! You are a veteran fool; what do you say now after committing a most blameable act? Had you any doubt as to your proceedings, why have you, out of sheer delusion, called this meeting hall of Svayamvara, without thinking the matter before-hand.

15. You have invited the kings and princes in this marriage ceremony Svayamvara; and they all have assembled here; how can they now go back to their homes?

16. Are you going now to insult these? and will you give your daughter in marriage to Sudar\'sana? Nothing can be more ignoble than this?

17. The welfare-seeking person ought to judge before-hand and then to act. But you have started your work without any previous judgment and decision. You will have to reap its fruit; there is no doubt this.

18. Why are you now thinking of giving your daughter to this helpless, wealthless Sudar\'sana in the presence of kings that are powerful and that command a great militia.

19. O thou most sinful! Today I will certainly kill you; next I will kill Sudar\'sana and then give your daughter to my daughter's son; know that this is my firm resolve.

20. Who is there, when I am standing, in this assembly that can aspire to carry away the bridegroom elect by force or theft? Nothing to speak of Sudar\'sana who is powerless, wealthless and a mere boy!

21. I spared his life before in the hermitage of Bh\^aradv\^aja at the Muni's request; but today I will not spare the boy under any circumstances whatsoever.

22. Therefore, please go and consult with your wife and daughter and give your dear beautiful daughter to my daughter's son.

23. Be engaged in a marriage tie with me by giving your exquisitely beautiful daughter to my daughter's son. You can very well judge that it is always proper and advisable that a great man shall come under the protection of another great man.

24. What happiness can you expect from this Sudar\'sana, who is helpless and banished from his kingdom, that you are going to give him your dear and auspicious daughter!

25. Family, wealth, army, appearance, kingdoms, forts and true friends and other helping persons; these a man should consider when he is going to give away his daughter in marriage to anybody; else there is no surety of happiness. Think over the royal custom and the never failing Dharma and do what is proper. Never it is advisable to do any act, abandoning the path of Dharma and morals.

26. You are my intimate friend; therefore I am telling you these good words. O king! Better bring your daughter, surrounded by her attendant maids, in this hall of Svayamvara.

27. Let this daughter select any man other than Sudar\'sana; I have got no cause of quarrel; and the marriage will then be celebrated according to your will.

28-29. O best of kings! The other kings are all of high descent; and they have armies and are all in positions befitting your connection! If the daughter chooses any one amongst them, then no quarrels would arise. But if the daughter chooses Sudar\'sana, then certainly I will carry her by force. Therefore, O king! act in such a way that no quarrels occur in future.''

30-31. Vy\^asa said :-- Thus addressed by Yudh\^ajit, the king of Benares was very sorrowful, and, after a heavy sigh, went to his palace and told with a grieved heart to his wife, thus :-- ``O fair eyed one! Now I am completely under your control; you better explain to \'Sa\'sikal\^a that a dreadful quarrel is now to occur; what am I to do now?''

32-33. Vy\^asa said :-- Hearing her husband's words, the queen went to her daughter and spoke thus :-- ``O child! Quarrels have now ensued amongst the kings for your sake; your father has become very sorrowful; therefore, O fair one! Choose any other man your husband than Sudar\'sana.

34-35. O Child! If you do not judge and rashly choose Sudar\'sana, then the powerful king Yudh\^ajit, possessing a large army, will no doubt kill you, me and Sudar\'sana. It might be, if quarrels ensue, you might be married to another husband; therefore better think now and act.

36. O dear eyed! It is now your incumbent duty to choose another king for your husband, if you want your and my welfare and happiness. Leave Sudar\'sana.''

37. The mother thus advised her daughter; the king, too, afterwards explained and tried to convince her. The girl spoke fearlessly.

38. ``O king! What you have said is all true; but you know my firm resolve already. I won't ever select any other king than Sudar\'sana.

39-40. O king! If you are afraid and be in agony, then do this thing: better give me in marriage to Sudar\'sana and then drive us away from your city. He will put me in his chariot and go away out of your city. After that what is inevitable will surely come to pass. There cannot be anything otherwise.

41. O king! You need not fear anything about what is kept in the womb of future by Destiny. What is inevitable will happen; there is no doubt in this.''

42. The king said :-- ``O child! The intelligent persons never show too much rashness and insolence. The learned people, versed in the Vedas, say it is never advisable to quarrel with many persons.

43. How can I give my daughter in marriage to one and then banish them both? The kings have turned out enemies. There is no heinous crime, that they cannot commit now.

44. O child! If it be your opinion, I can pledge something as a pawn for your marriage, as the king Janaka pledged in days of yore for her daughter Sit\^a.

45-47. I will also put forward a pawn very difficult to be carried out, as Janaka originally made an offer of the hands of Sit\^a to whomsoever, who would break the \'Siva's strong bow. Thus the quarrels amongst the kings might

be diminished; for he who will be able to fulfill the promise, will be able to accept you. Then, be he Sudar\'sana or any other king, whoever will be strong to fulfill the promise will take you as his wife.

48. Thus the quarrels will cease and I will also be able to perform your marriage ceremony in peace and happiness.''

49. The daughter said :-- ``Father! On hearing from you, I am merged in an ocean of doubt, for it seems to me what you are saying is the act of a fool; already, I have chosen in my mind Sudar\'sana for my husband; now it cannot be otherwise.

50. O king! The mind is the source of virtue and vice. When I have mentally selected, how can I now forego him and choose another?

51-52. O king! If you keep any pledge, then I will be subject to any and everybody; if one, two, or more fulfill the same pledge, I will be then subject to any or all of them. Father! in that case quarrels may arise. What shall I do then? I cannot give my vote on this doubtful point.

53. O king! You need not fear anything. Better give me in marriage to Sudar\'sana according to the prescribed rules; then, in that case, the Goddess Chandik\^a will certainly protect us.

54. O king! Taking Whose Name destroys a whole host of sins, take Her Name and think the Almighty and perform carefully our marriage ceremony.

55. Better go to the king's assembly today, and, with folded hands, tell them come tomorrow to the hall of Svayamvara.

56-57. Thus bidding goodbye to the kings, perform in the right spirit, according to the prescribed rites, our marriage ceremony. Next, after giving fit dowries and other articles after the marriage, better tell the prince Sudar\'sana to depart. The son of Dhruvasandhi will take me away with him.

58. If, at this, the kings get angry and be ready to quarrel with you, then in that case, the Goddess Bhagavat\^i will no doubt help us.

59. Sudar\'sana then will fight against those kings; and if he loses his life perchance in the battle, then I will also follow him and die.

60. O king! Let all good come unto you! Better give me in marriage to Sudar\'sana and remain here with your army. I will go alone with him, the object of my love.''

61. Vy\^asa said :-- Hearing these words from her daughter, the king Sub\^ahu trusted her, and firmly resolved to act according to that, and to celebrate the marriage of \'Sa\'sikal\^a.

Thus ends the 21st chapter on the king of Benares fulfilling the advice of his daughter in \'Sr\^imad Dev\^i Bh\^agavatam of 18,000 verses by Maharsi Veda Vy\^asa.